\hd{The below is a condensed version of the DH game, possibly for use in a shorter version or possibly just as a sanity check.}
%\gamehop{game:abort-dhe}
	In this game, we add another flag to the game and abort the game when it is set.
	Namely, flag $\bad_\dhe$ is set if the adversary ever queries a random oracle
	\[
		\RO_x(\psk, \dhe, \Thash(\cid[s]))
	\]
	for any $(x,s) \in \{\mathlist{(\chtk,3), (\shtk,4), (\sfin, 5) (\cats,5), (\sats,6), (\ems,7), (\cfin, 8), (\rms,8)}\}$ such that
	\begin{itemize}
		\item there is a query $\RevLongTermKey(u,v,\pskid)$ with $\pskeys[(u,v,\pskid)] = \psk$, i.e. $\psk$ is corrupted,
		\item there are honest sessions $\pi_u^i$ and $\pi_v^j$ that are contributively partnered in stage $s$ with $\cid[s] = (\CH, \CKS, \CPSK, \SH, \SKS, \SPSK,\dotsc)$, and
		\item $\dhe = g^{xy}$ such that $\CKS = g^x$ and $\SKS = g^y$.
	\end{itemize}

We bound the probability that $\bad_{\dhe}$ is set by a reduction $\advB_4$ to the Strong Diffie--Hellman problem in group $\G$.

In this problem, the adversary $\advB_4$ gets as input a strong DH challenge $(A = g^a, B = g^b)$ as well as access to an oracle $\stDH_a$ for the Decisional Diffie--Hellman problem with the first argument fixed. 
Given inputs $C:=g^c$ and $W$ for any $c \in \Z_p$, $\stDH_a(C, W)$ returns $\true$ if and only if $W = g^{ac} = C^a$.
Its goal is to submit $Z$ to its $\Finalize$ oracle such that $Z = g^{ab}$.

To do this, $\advB_4$ simulates \thisGame for adversary $\advA^*$.
At a high level, it embeds its strong DH challenges into the key shares of every initiator session and every partnered responder session.
When $\advA^*$ triggers the $\bad_{\dhe}$ flag in the simulated game, $\advB_4$ will learn the Diffie--Hellman secret $\dhe$ associated with two of these embedded key shares.
Then it can extract the strong DH challenge secret using some basic algebra.
However, sessions with embedded key shares must be handled carefully, because $\advB_4$ does not know their Diffie--Hellman secrets and therefore cannot generate the keys as it should in \thisGame.
In these situations, it will instead sample random keys and $\Finished$ messages.
We then use several look-up tables to appropriately program random oracle queries and maintain consistency with these random keys.
(However, programming is only necessary after the pre-shared key is corrupted thanks to \refGameText{game:mac-forgery-prep}.)

Lookup table $\Theta$ tracks internal session state that is unique to the reduction.
Table $D$ is maintained by the random oracles and contains the Diffie--Hellman secret $\dhe$ from each random oracle query.
Sessions will use $D$ to check if their $\dhe$ value has already been guessed by the adversary.
Table $Q$ is maintained by the sessions and contains their randomly sampled keys and all of the information necessary for the random oracles to program later queries.
Table $H$ is the table used to implement all of our random oracles via lazy sampling.
We also rely tables $S$, $F$, and $R$ from earlier games (respectively \refGameText{game:log-keys} and \refGameText{game:mac-forgery-prep}).


We will describe the behavior of $\advB_4$ in more detail.
It honestly executes the $\Initialize$, $\RevSessionKey$, $\Finalize$, and $\Test$ queries as in \thisGame.
We address $\Send$, $\RO_x$, and $\RevLongTermKey$ queries individually.

For $\Send$ queries, $\advB_4$ performs all the bookkeeping steps as in $\thisGame$.
However, it changes the implementation of both initiator and responder sessions significantly.
Whenever a $\Send$ query creates a new initiator session $\pi_u^i$, we generate the $\ClientKeyShare$ by randomizing the strong Diffie--Hellman challenge $A$:
After sampling a nonce $r_C\sample \bits^{\nl}$, the new session picks an integer  $\tau_u^i \sample \Z_p$ uniformly at random and sets $\CKS \assign A \cdot g^{\tau_u^i}$.
Then, it logs $\tau_u^i$ under index $(r_C, \CKS)$ in the new look-up table $\Theta$.

New responder sessions also generate their key shares in this way, with two differences: first, they randomize $B$ instead of $A$.
Second, responder sessions only use randomized key shares if they have an honest partner in stage $2$ ($\eems$).
They can detect the existence of such a partner using table $S$.
Since \refGameText{game:copy-keys}, responder sessions already do this to copy keys $\ets$ and $\eems$.

Sessions without embedded key shares will behave in the reduction exactly as they do in \thisGame.
For sessions with embedded key shares, however, we cannot compute the Diffie--Hellman secret $\dhe$, and we cannot complete any step of the game which requires this value.
These steps include key derivation and computing the $\Finished$ messages for sessions whose $\psk$ has been corrupted, and logging $\RO$ query inputs for sessions with uncorrupted $\psk$s (cf.\ \refGameText{game:mac-forgery-prep}).
We therefore explain how to handle these in the reduction.

When a session with corrupted $\psk$ wants to derive a key $k \in \{\shtk, \chtk, \cats, \sats, \ems, \rms\}$, it would either copy that key from an honest partner or make a query to $\RO_k$ containing $\dhe$ as an input.
In the reduction, sessions which can copy their keys still do so.
Sessions which would query $\RO_k$ first check to see if such a query has already occurred.

Concretely, $\advB_4$ computes the context $d$ that would be input to $\RO_k$ as the third argument.
If a key is being derived from an embedded key share, we know that the initiator's key share must have been embedded. 
The responder's key share may be embedded or it may be adversarially controlled. 
We therefore search prior $\RO$ queries for the Diffie--Hellman secret using only the initiator session's randomizer $\tau_u^i$, found using table $\Theta$, and the responder session's key share $Y$.
$\advB_4$ loops through all the entries in table $D_k[\psk, d]$.
(We will discuss the origin of these entries below.)
For each entry with value $Z$, $\advB_4$ queries $\stDH_a(Y, Z \cdot Y^{-\tau_u^i})$.

Although $\advB_4$ may query $\stDH_a$ several times in response to a single $\Send$ query, it remains efficient because it only checks $\RO$ queries whose context is $d$.
Due to the lack of nonce/group element and hash collisions established in games \refGameText{game:hello-coll} and \refGameText{game:hash-coll}, the context $d$ is unique to session $\pi_u^i$ and its partner. 
Therefore each entry in $D_{\chtk}[\psk, d]$ will be checked at most twice over the course of the entire reduction.

If one of the $\stDH_a$ queries returns true, the adversary has computed the correct Diffie--Hellman secret $Z$.
If only one of the key shares corresponding to $\dhe$ is embedded with the challenge, the adversary may control the other and compute the secret that way. 
In this case, the session sets $\dhe$ to $Z$ and derives its keys as in \thisGame. 
If instead both key shares are embedded, then the $\bad_{\dhe}$ flag would be set because the adversary has computed a Diffie--Hellman secret corresponding to two partnered sessions.
In this case, $\advB_4$ looks up the responder's randomizer $\tau_v^j$ in table $\Theta$.
Then it wins the strong Diffie--Hellman game by querying $\Finalize$ on input
\[
Z \cdot Y^{-\tau_u^i} \cdot A^{-\tau_v^j} = Y^a \cdot A^{-\tau_v^j} = (g^a)^{b + \tau_v^j} \cdot (g^a)^{-\tau_v^j} = g^{ab}
\]

If the $\stDH_a$ oracle never returns true, then the session creates a log entry in table $Q_k$ to help program future $\RO$ queries.
If $k$ is the randomly sampled key for stage $s$, then $\advB_4$ logs 
\[
(\psk, d = \ROthash(\sid[s]) \mapsto \left(  \tau_u^i, \tau_v^j, (\sid[s]), k \right).
\]
(If the responder key share is not embedded, let $\tau_v^j = \bot$ here.)

When $\psk$ is not corrupted, we do not need to worry about consistency with any prior $\RO$ queries, and the $\bad_{\dhe}$ flag cannot be triggered.
For these sessions, $\advB_4$ samples keys at random and creates a log in $Q_k$.
Since we have no pre-shared key (due to \refGameText{game:mac-forgery-prep}), the log entry has the form 
\[
((u,v,\pskid), d = \ROthash(\sid[s]) \mapsto \left(  \tau_u^i, \tau_v^j, (\sid[s]), k \right).
\]

The reduction handles $\Finished$ messages in much the same way as keys, but there are is an important difference.
Because $\Finished$ messages are not keys, they do not have associated session IDs and sessions must know $\dhe$ to find or log their values in table $F$.
In our reduction, initiator sessions which compute $\cfin$ and responder sessions which compute $\sfin$ do this in the same way they derive keys.
Wherever a $\sid$ is called for, we use the transcript that enters the $\Finished$ message computation: $\CH \concat \ldots \EE$ for $\sfin$ and $\CH \concat \ldots \SF$ for $\cfin$.
To verify $\Finished$ messages, sessions check for the existence of an honest partner in the next stage and extract the $\Finished$ message from that partner's session ID.
An initiator receiving $\EE$ and $\SF$ checks for a partnered responder session in stage $4$ ($\shtk$). If this partner exists, it will also have a stage $5$ session ID containing the $\EncryptedExtensions$ and $\ServerFinished$ message it sent.
The initiator checks $\EE$ and $\SF$ against its partner's session $ID$. If they match, it accepts stage $5$ and copies its partner's $\sid$ and keys. 
If they do not match, it terminates.
This is an accurate simulation of $\thisGame$ because the session cannot accept unless its partner exists thanks to game \refGameText{game:honest-mac-forgery}.
Responder sessions analogously check the stage-$8$ session ID of their stage-$7$ partner to verify the $\ClientFinished$ message.

Next, we discuss the implementation of the random oracles.
If $\RO_k$ receives a query that was already answered it answers consistently.
However, if there is a new query of the form $(\psk, Z, d)$, we append $Z$ to the list $D_k[\psk, d]$.
Then, if $Q_k[\psk, d] \neq \bot$, we know an honest session with pre-shared key $\psk$ and context hash $d$ randomly sampled the key $k$ without knowing the correct Diffie--Hellman secret.
We therefore use the $\stDH_a$ oracle to check whether $Z$ is that secret.
Let $(\tau_u^i, \tau_v^j, \mathit{ctxt}, \mathit{key})$ be the entry of $Q_k[\psk, d]$, where $\tau_u^i$ and $\tau_v^j$ denote the randomness used by the client and the server to randomize the stDH challenge, respectively, $\mathit{ctxt} = \CH \concat \dotsb$ denotes the transcript such that $d = \RO_\Thash(\mathit{ctxt})$ and $\mathit{key}$ denotes the key chosen by the session.
Using this information, it fetches $\SKS = Y$ and queries $\stDH_a(Y, Z \cdot Y^{-\tau_u^i})$.
If this is answered positively, we know that the right DH value $Z$ was queried.
Then if $\tau_u^j = \bot$, i.e. the log in $Q_k$ was set by a client without a honestly partnered server, we need to program the random oracle to be consistent.
That is, $H_k[\psk, Z, d] \assign \mathit{key}$.
Otherwise, we know that the $Q_k$ entry was set by an honestly partnered session, and thus $Z$ is a randomized solution to our stDH challenge.
Thus, $\advB_4$ wins the strong Diffie--Hellman game with the query $\Finalize(Z \cdot Y^{-\tau_u^i} \cdot A^{-\tau_v^j})$.
Finally, $\RO_k$ outputs $H_k[\psk, Z, d]$.

Lastly, we need to handle corruptions by the $\RevLongTermKey$ oracle.

Since $\refGameText{game:mac-forgery-prep}$, the $\RevLongTermKey$ oracle upon input $(u, v, \pskid)$ samples a fresh $\psk$.
That game also used lists $R_x$ to program all the random oracles $\RO_{x}$ for consistency with all sessions using $(u,v, \pskid)$.

Adversary $\advB_4$ still does this, but in our reduction, embedded sessions do not update the lists $R_x$. 
Instead, they create log entries in $Q_x$ with indices of the form $(u,v,\pskid), d$. 
We cannot use these entries to program past $\RO_x$ queries, but this is not necessary since any past $\RO_x$ query containing $\psk$ would set the $\bad_{\psk}$ flag and cause the game to abort.
We also cannot program future $\RO$ queries because we still do not know $\dhe$. 
Instead, we just update each matching entry in $Q_x$ so that its index is $(\psk, d)$ instead of $((u, v, \pskid), d)$. 
Future $\RO_x$ queries containing $\psk$ will then handle strong DH checking and programming for us.

By the considerations above, we have that if $\bad_\dhe$ is set the $\advB_4$ wins the strong DH challenge.
The identical-until-bad-lemma gives us that
\begin{align} \nonumber
\Pr[ \prevGameMath[\advA^*] \outputs 1 ] &\leq \Pr[ \thisGameMath[\advA^*] \outputs 1 ] + \Pr[ \bad_\dhe ] \\
&\leq \Pr[ \thisGameMath[\advA^*] \outputs 1 ] + \genAdv{\stDH}{\G}{\advB_4}.
\end{align}



