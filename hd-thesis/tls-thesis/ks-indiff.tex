\section{Key-Schedule Indifferentiability}\label{sec:ks-indiff}

In this section we will argue that the key schedule of TLS~1.3 PSK modes, where the underlying cryptographic hash function is modeled as a random oracle (i.e., the left-hand side of Figure~\ref{fig:tls-handshake} with the underlying hash function modeled as a random oracle), is \emph{indifferentiable}~\cite{TCC:MauRenHol04} from a key schedule that uses \emph{independent} random oracles for each step of the key derivation (i.e., the right-hand side of Figure~\ref{fig:tls-handshake} with all $\TLSKDF_x$ functions modeled as independent random oracles).
%Previous analyses of TLS~1.3 \cite{ACNS:DavGun21,JC:DieJag21} simply assumed that one can view the individual steps of the key derivation as 
%
We stress that this step not only makes our main security proof in Section~\ref{sec:ke-proof} significantly simpler and cleaner, but also it puts the entire protocol security analysis on a firmer theoretical ground than previous works.
For some background on the indifferentiability framework, see
	Section~\ref{app:indiff-background}.

In their proof of tight security, Diemert and Jager~\cite{JC:DieJag21} previously modeled the TLS~1.3 key schedule as four independent random oracles.
Davis and Günther~\cite{ACNS:DavGun21} concurrently modeled the functions $\HKDFExtr$ and $\HKDFExpnd$ used by the key schedule as two independent random oracles. 
Neither work provided formal justification for their modeling. 
Most importantly, both neglected potential dependencies between the use of the hash function in multiple contexts in the key schedule and elsewhere in the protocol. 
In particular, no construction of $\HKDFExtr$ and $\HKDFExpnd$ as independent ROs from one hash function could be indifferentiable, because $\HKDFExtr$ and $\HKDFExpnd$ both call $\HMAC$ directly on their inputs, with $\HKDFExpnd$ only adding a counter byte. 
Hence, the two functions are inextricably correlated by definition. 
We do not claim that the analyses of \cite{JC:DieJag21,ACNS:DavGun21} are incorrect or invalid, but merely point out that their modeling of independent random oracles is currently not justified and might not be formally reachable if one only wants to treat the hash function itself as a random oracle.
%
This is undesirable because the gap between an instantiated protocol and its abstraction in the random oracle model can camouflage serious attacks, as Bellare et al.~\cite{EC:BelDavGun20} found for the NIST PQC KEMs. 
Their attacks exploited dependencies between functions that were also modeled as independent random oracles but instantiated with a single hash function.

In contrast, in this section we will show that our modeling of the TLS~1.3 key schedule is indifferentiable from the key schedule when the underlying cryptographic hash function is modeled as a random oracle. 
To this end, we will require that inputs to the hash function do not appear in multiple contexts. 
For instance, a protocol transcript might collide with a Diffie--Hellman group element or an internal key (i.e., both might be represented by exactly the same bit string, but in different contexts). 
For most parameter settings, we can rule out such collisions by exploiting serendipitous formatting, but for one choice of parameters (the PSK-only handshake using \SHA{384} as hash function), an adversary could conceivably force this type of collision to occur; see 
	Appendix~\ref{app:domsep}
for a detailed discussion.
While this does not lead to any known attack on the handshake, it precludes our indifferentiability approach for that case.

\paragraph{Insights for the design of cryptographic protocols.}
One interesting insight for protocol designers that results from our attempt of closing this gap with a careful indifferentiability-based analysis is that proper domain separation might enable a cleaner and simpler analysis, whereas a lack of domain separation leads to uncertainty in the security analysis. 
No domain separation means stronger assumptions in the best case, and an insecure protocol in the worst case, due to the potential for overlooked attack vectors in the hash functions. 
A simple prefix can avoid this with hardly any performance loss.

\paragraph{Indifferentiability of the TLS~1.3 key schedule.}
Via the indifferentiability framework, we replace the complex key schedule of TLS~1.3 with $12$ independent random oracles: one for each first-class key and $\MAC$ tag, and one more for computing transcript hashes. 
In short, we relate the security of TLS~1.3 as described in the left-hand side of Figure~\ref{fig:tls-handshake} to that of the simplified protocol on the right side of Figure~\ref{fig:tls-handshake} with the key derivation and $\MAC$ functions $\TLSKDF_x$ and modeled as independent random oracles.
We prove the following theorem, which formally justifies our abstraction of the key exchange protocol by reducing its security to that of the original key exchange game.

\begin{theorem}
	\label{thm:full-ks-indiff}
	Let $\ROhash \colon \bits^{*} \to \bits^{\hashlen}$ be a random oracle.
	%
	Let $\KE$ be the TLS~1.3 PSK-only or PSK-(EC)DHE handshake protocol described on the left hand side of Figure~\ref{fig:tls-handshake} with $\abstractHash := \ROhash$
	and $\abstractMAC$, $\abstractExtract$, and $\abstractExpand$ defined from~$\abstractHash$ as in Section~\ref{sec:tls13-psk-protocol}. 
%\fg{That definition is currently missing in Sec.2, \TODO{check}.}
	%
	Let $\KE'$ be the corresponding (PSK-only or PSK-(EC)DHE) handshake protocol on the right hand side of Figure~\ref{fig:tls-handshake}, with $\abstractHash := \ROthash$ and  $\TLSKDF_x := \RO_{x}$, where $\ROthash$, $\RObinder$, \dots, $\ROrms$ are random oracles with the appropriate signatures
		(cf.~\iffull Section~\else Appendix~\fi\ref{sec:many-ros}
	for the signature details).
%
%For any adversary $\advA$ attacking the $\KESEC$ security of $\KE$, 
%Llet $t_\advA$ denote the runtime of $\advA$, and let $\qRO$ and $\qSend$ capture the number of queries $\advA$ makes to the $\RO$ and $\Send$ oracles respectively. 
	Then, %for $t \approx t'$,
%there exists an adversary $\advB$ against the security of $\KE'$ such that 
	\begin{align*}
		\Adv^{\KESEC}_{\KE}(t, \qNewSecret, \qSend, \qRevSessionKey, \qRevLongTermKey, \qTest, \qRO)
		\leq
			\Adv^{\KESEC}_{\KE'}(t, \qNewSecret, \qSend, \qRevSessionKey, \qRevLongTermKey, \qTest, \qRO) \\
			+ \frac{2(12\qSend+\qRO)^2}{2^{\hashlen}}
			+ \frac{2\qRO^2}{2^{\hashlen}}
			+ \frac{8(\qRO+36\qSend)^2}{2^{\hashlen}}.
 	\end{align*}
%	Adversary $\advB$ has runtime approximately equal to $t_{\advA}$ and makes the same number of queries to each of its oracles in the $\KESEC$ game.
\end{theorem}

We establish this result via three modular steps in the indifferentiability framework introduced by Maurer, Renner, and Holenstein~\cite{TCC:MauRenHol04}.
More specifically we will leverage a recent generalization proposed by Bellare, Davis, and Günther (BDG)~\cite{EC:BelDavGun20}, which in particular formalizes indifferentiability for constructions of \emph{multiple} random oracles.

%We can also make a similar claim for read-only indifferentiability. The proof needs only two small adjustments: first, we require that the $\KE-SEC$ game restricts all of its queries to be within the working domain $\workDom$, and second, we have $\advB$ initialize the simulator's state before running $\advA$. 
% Since the simulated game of $\advD$ restricts all its queries to be within $\workDom$, the restriction of the $\PrivO$ oracle does not affect the honesty of the simulation, and the same bound can be shown. 

\subsection{Indifferentiability for the TLS~1.3 Key Schedule in Three Steps}
We move from the left of Figure~\ref{fig:tls-handshake} to the right via three steps.
Each step introduces a new variant of the TLS~1.3 protocol with a different set of random oracles by changing how we implement $\abstractHash$, $\abstractMAC$, $\abstractExpand$, $\abstractExtract$, and eventually the whole key schedule.
Then we view the prior implementations of these functions as constructions of new, independent random oracles.
We prove security for each intermediate protocol in two parts: first, we bound the indifferentiability advantage against that step's construction; then we apply the indifferentiability composition theorem based on~\cite{TCC:MauRenHol04}
	(cf.\ \iffull Section~\else Appendix~\fi\ref{app:indiff-background}, Theorem~\ref{thm:indiff-comp})
to bound the multi-stage key exchange ($\KESEC$) security of the new protocol.

We give a brief description of each step; all details and formal theorem statements and proofs can be found in 
	\iffull Sections~\else Appendices~\fi \ref{sec:domsep}, \ref{sec:hmac}, and \ref{sec:many-ros}, respectively.

\begin{description}
	\setlength{\itemsep}{0.5em}
	
	\item[From one random oracle to two.]
	TLS~1.3 calls its hash function~$\abstractHash$, which we initially model as random oracle~$\ROhash$, for two purposes:
	to hash protocol transcripts, and as a component of $\abstractMAC$, $\abstractExtract$, and $\abstractExpand$ which are implemented using $\HMAC[\abstractHash]$.
	Our eventual key exchange proof needs to make full use of the random oracle model for the latter category of hashes, but we require only collision resistance for transcript hashes. 
	
	Our first intermediate handshake variant, $\KE_1$, replaces $\abstractHash$ with two new functions: $\Thash$ for hashing transcripts, and $\Chash$ for use within $\abstractMAC$, $\abstractExtract$, or $\abstractExpand$.
	While $\KE$ uses the same random oracle $\ROhash$ to implement $\Thash$ and $\Chash$, the $\KE_1$ protocol instead uses two independent random oracles $\ROthash$ and $\ROhmac$.
	To accomplish this without loss in $\KESEC$ security, we exploit some possibly unintentional domain separation in how inputs to these functions are formatted in TLS~1.3 to define a so-called \emph{cloning functor}, following BDG~\cite{EC:BelDavGun20}.
	%
	Effectively, we partition the domain~$\bits^*$ of~$\ROhash$ into two sets $\Dom_{\Thash}$ and $\Dom_{\Chash}$ such that $\Dom_{\Thash}$ contains all valid transcripts and $\Dom_{\Chash}$ contains all possible inputs to $\abstractHash$ from $\HMAC$. 
	We then leverage Theorem~1 of~\cite{EC:BelDavGun20} that guarantees composition for any scheme that only queries $\ROchash$ within the set $\Dom_{\Chash}$ and $\ROthash$ within the set $\Dom_{\Thash}$.
	
	We defer details on the exact domain separation to 
		Appendix~\ref{app:domsep},
	but highlight that the PSK-only handshake with hash function \SHA{384} \emph{fails} to achieve this domain separation
	and consequently this proof step cannot be applied and leaves a gap for that configuration of TLS~1.3. 
	
	
	\item[From SHA to HMAC.]

	Our second variant protocol, $\KE_2$, rewrites the $\abstractMAC$ function. Instead of computing $\HMAC[\ROchash]$, $\abstractMAC$ now directly queries a new random oracle $\ROhmac \colon\allowbreak \bits^{\hashlen} \times \bits^* \to \bits^{\hashlen}$.
	Since $\ROchash$ was only called by $\abstractMAC$, we drop it from the protocol, but we do continue to use $\ROthash$,
	i.e., $\KE_2$ uses two random oracles: $\ROthash$ and $\ROhmac$.
	The security of this replacement follows directly from Theorem~4.3 of Dodis et al.~\cite{C:DRST12}, which proves the indifferentiability of $\HMAC$ with fixed-length keys.%
	\footnote{This requires PSKs to be elements of $\bits^{\hashlen}$, which is true of resumption keys but possibly not for out-of-band PSKs.}

	\item[From two random oracles to 12.]
	Finally, we apply a ``big'' indifferentiability step which yields $12$ independent random oracles and moves us to the right-hand side of Figure~\ref{fig:tls-handshake}.
	The $12$ ROs include
	the transcript-hash oracle~$\ROthash$ and 11 oracles that handle each key(-like) output in TLS~1.3's key derivation, named
		$\RObinder$,
		$\ROets$,
		$\ROeems$,
		$\ROchtk$,
		$\ROcfin$,
		$\ROshtk$,
		$\ROsfin$,
		$\ROcats$,
		$\ROsats$,
		$\ROems$, and
		$\ROrms$.
	(The signatures for these oracles are given in 
		Appendix~\ref{sec:many-ros}.)
	For this step, we view $\TLSKDF$ as a construction of $11$ random oracles from a single underlying oracle ($\ROhmac$).
	We then give our a simulator in pseudocode and prove the indifferentiability of $\TLSKDF$ with respect to this simulator.
	Our simulator uses look-up tables to efficiently identify intermediate values in the key schedule and consistently program the final keys and $\MAC$ tags.
\end{description}

Combining these three steps yields the result in Theorem~\ref{thm:full-ks-indiff}.
In the remainder of the paper, we can therefore now work with the right-hand side of Figure~\ref{fig:tls-handshake}, modeling $\abstractHash$ and the $\TLSKDF$ functions as $12$ independent random oracles.

\iffull
	\def\StepOneTitle{Step 1: Domain-separating the Transcript Hash}
\iffull
	\subsubsection{\StepOneTitle}
\else
	\subsection{\StepOneTitle}
\fi
\label{sec:domsep}
In the original TLS~1.3 PSK/PSK-(EC)DHE handshake, the hash function~$\abstractHash$ is used in two different ways.
It is used directly to compute digests of a \emph{transcript} and it is used as a \emph{component} of $\abstractMAC$, $\abstractExtract$, and $\abstractExpand$.
We will argue now that these two uses are entirely distinct, and we can accordingly write two functions~$\Thash$ and $\Chash$ in place of the two uses of~$\abstractHash$,
and, following BDG~\cite{EC:BelDavGun20}, go from modeling~$\abstractHash$ as one random oracle to modeling $\Thash$ and $\Chash$ as two independent random oracles.

We will refer to our two new random oracles as $\ROthash$ (modeling the \emph{transcript hash} function~$\Thash$) and $\ROchash$ (modeling the \emph{component hash} $\Chash$).
Because TLS~1.3 fully specifies the inputs to each hash function call,
we can show that in PSK-(EC)DHE mode and in PSK-only mode when $\hashlen=256$, TLS~1.3 will never call the same string as an input to both $\Thash$ and $\Chash$. 
This is due to some fortunate coincidences of formatting in the standard, which we describe in full in Appendix~\ref{app:domsep}. 
We can therefore define two disjoint sets $\Dom_{\Thash}$ and $\Dom_{\Chash}$ such that $\Dom_{\Thash} \cup \Dom_{\Thash} = \bits^*$ split up $\hash$'s domain. 

If we define the domain of $\ROthash$ to be $\Dom_{\Thash}$ and the domain of $\ROchash$ to be $\Dom_{\Chash}$, we could prove indifferentiability using a construction called the \emph{identity (cloning) functor}~$\construct{I}$ from~\cite{EC:BelDavGun20}.
The identity functor constructs two or more random oracles $\RO_1, \RO_2,\ldots$ from $\ROhash$ by forwarding all $\RO_i$ queries to $\ROhash$ unchanged.
However, the definitions of sets $\Dom_{\Thash}$ and $\Dom_{\Chash}$ are somewhat complex, especially in PSK-only mode.
We would instead prefer to define both $\ROthash$ and $\ROchash$ with domains $\bits^*$.
This would greatly simplify our later use of $\ROchash$ as a component of $\HMAC$.
Unfortunately, when the domains of $\ROthash$ and $\ROchash$ overlap, the identity functor is \emph{not} indifferentiable.
We can however still provide the desired result by turning to the read-only indifferentiability framework of Bellare, Davis, and Günther~\cite{EC:BelDavGun20}.

Read-only indifferentiability (a.k.a. $\rdindiff$) is similar to standard indifferentiability~\cite{TCC:MauRenHol04}.
One notable change (and the one we will leverage here) is that it is parameterized by a set~$\workDom$ called the ``working domain.''
The security game places a restriction on the $\PrivO$ oracle so that it only responds to queries within~$\workDom$.
Read-only indifferentiability supports a broader composition thoerem than Theorem~\ref{thm:indiff-comp}, which covers security games which call their random oracles only within the working domain.
BDG prove~\cite[Theorem~1]{EC:BelDavGun20}, which states that when $\workDom$ consists of disjoint sets like $\Dom_{\Thash}$ and $\Dom_{\Chash}$, the identity functor is read-only indifferentiable even when the full domains of $\ROthash$ and $\ROchash$ are not disjoint.
Furthermore, the read-only indifferentiability advantage is upper-bounded by $0$, and BDG give a simulator that runs in linear time on the length of its inputs and makes at most one query per execution. 
When we apply the read-only indifferentiability composition theorem, the adversary's runtime and query bounds will not increase.

We formalize this with a lemma:
\begin{lemma}%		Let $\KE$ be the TLS~1.3 key exchange protocol described in Figure~\ref{fig:tls13-psk}. Let $\hh:\bits^{*} \to \KEkeyspace$ be the random oracle used by the protocol. Also, let $\KE'$ be the key exchange protocol of Figure \TODO{?} which uses 2 random oracles $\Thash, \Chash: \bits^* \to bits^{\hashlen}$. Let $\Dom_{\Thash}$ and $\Dom_{\Chash}$ be two disjoint sets such that $\Dom_{\Thash}\cup \Dom_{\Chash} = \bits^*$. For any adversary $\advA$ attacking the $\KESEC$ security of $\KE$, let $t_\advA$ denote the runtime of $\advA$, and let $q_{\RO}$ and $q_{\Send}$ capture the number of queries $\advA$ makes to the $\RO$ and $\Send$ oracles respectively. Then there exists an adversary $\advB$ against the security of $\KE'$ such that
	%	\[ \genAdv{\KESEC}{\KE}{\advA} \leq \genAdv{\KESEC}{\KE'}{\advB}. \]
	%	Adversary $\advB$ has runtime approximately equal to $t_{\advA} + q_{\RO}$, and it makes the same number of queries to each of its oracles in the $\KESEC$ game. 
	
	\label{thm:ks-indiff-hop-1-comp}
	Let $\KE$ be the TLS~1.3 key exchange protocol of Theorem~\ref{thm:full-ks-indiff}.
	Let $\ROthash, \ROchash\colon\allowbreak \bits^* \to \bits^{\hashlen}$ be two random oracles, and
	let  $\KE_1$ be the protocol on the left-hand side of Figure~\ref{fig:tls-handshake}, where
	\begin{itemize}
		\item $\abstractHash := \ROthash$
		\item $\abstractMAC := \HMAC[\ROchash]$
	\end{itemize}
	and $\abstractExpand$ and $\abstractExtract$ are as in $\KE$ (using the new definition of $\abstractMAC$).
	Let $\Dom_{\Thash}$ and $\Dom_{\Chash}$ be two disjoint sets such that $\KE_1.\KERun$ only queries $\ROthash$, resp. $\ROchash$ in $\Dom_{\Thash}$, resp. $\Dom_{\Chash}$, and $\Dom_{\Thash}\cup \Dom_{\Chash} = \bits^*$.
	Furthermore, let $\Dom_{\Thash}$ have an efficient membership function.	
	
	Let $\advA$ be an adversary against the $\KESEC$ security of~$\KE$, running in time~$t_\advA$ and making $\qRO$ and $\qSend$ queries to its random oracle resp.\ $\Send$ oracle.
	Then there exists an adversary $\advB$ against the security of $\KE'$, such that 
	\[
	\genAdv{\KESEC}{\KE}{\advA}
	\leq \genAdv{\KESEC}{\KE_1}{\advB}.
	\]
	
	Adversary~$\advB$'s runtime is $\bigO(t_{\advA} + \qRO)$, and it makes the same number of queries to each of its oracles as~$\advA$ in the $\KESEC$ game. 
\end{lemma}
\begin{proof}
	The function space of $\KE$ is $\SSp = \AllFuncs(\bits^*, \bits^{\hashlen})$, and the function space of $\KE_1$ is $\ESp = \AllFuncs( \{\Thash, \Chash\} \times \bits^*, \bits^{\hashlen})$. 
	We can construct $\ESp$ from $\SSp$ via a construction called the ``identity functor'' defined by BDG~\cite{EC:BelDavGun20}.
	This construction is parameterized by a set $\workDom:= (\{\Thash \}\times \Dom_{\Thash}) \cup (\{\Chash \}\times \Dom_{\Chash})$.
	To answer any query $(i, s)$, the identity functor simply forwards $s$ to its own oracle, regardless of whether $i$ is $\Thash$ or $\Chash$.
	Because $\workDom$ is the union of two disjoint sets with efficient membership functions, the simulator $\Sim$ defined by BDG's Theorem $1$ has the property that for any distinguisher $\advD$,
	\[\genAdv{\rdindiff}{\construct{I}_\workDom, \workDom, \Sim}{\advD} = 0.\]
	$\Sim$ works by using the membership function of $\Dom_{\Thash}$ to check which of the two oracles is being simulated; then it forwards the query to the appropriate oracle.
	
	For this (or any) simulator, the composition theorem for read-only indifferentiability grants the existence of adversary $\advB$ and a distingisher $\advD$ such that
	
	\[
	\genAdv{\KESEC}{\KE}{\advA} \leq \genAdv{\KESEC}{\KE_1}{\advB} + \genAdv{\rdindiff}{\construct{I}_\workDom, \workDom, \Sim}{\advD}
	\leq \genAdv{\KESEC}{\KE_1}{\advB}.
	\]
	This composition theorem crucially rests on the fact that $\KE_1.\KERun$ queries $\ROthash$ and $\ROchash$ only within $\workDom$. The lemma follows.
	
	We require that $\Dom_{\Thash}$ and $\Dom_{\Chash}$ are disjoint sets. We define specific choices of $\Dom_{\Thash}$ and $\Dom_{\Chash}$ based on the low-level formatting of TLS~1.3 in Appendix~\ref{app:domsep}, and there we give detailed arguments that the sets are disjoint for 3 of 4 standardized settings of the PSK/PSK-(EC)DHE handshake. 
	
	In the fourth setting, PSK-only mode with hash function~$\SHA{384}$, there are no disjoint choices for $\Dom_{\Thash}$ and $\Dom_{\Chash}$ with efficient membership functions.
	This is due to a lack of careful domain separation of the hash function calls in TLS~1.3. 
	We therefore cannot apply this indifferentiability step for the PSK-only/$\SHA{384}$ handshake protocol.
	Any security proof of this handshake must either rely on stronger, possibly falsifiable abstractions in the random oracle model, or use a model $\SHA{384}$ as a single random oracle, with no guarantees of independence.
	We avoid the latter approach in order to maintain a modular and readable proof. 
	
	The second inequality follows from our choice of simulator and Theorem~1 of~\cite{EC:BelDavGun20}, which makes at most one query to its random oracle per execution.
	Their simulator, as mentioned above, must efficiently determine for every query $s$ whether to query $\ROthash$ or $\ROchash$. 
	This induces the requirement that $\Dom_{\Thash} \cup \Dom_{\Chash} = \bits^*$, so every possible query can be routed appropriately, and the requirement that $\Dom_{\Thash}$ has an efficient membership function so that the simulator is itself efficient.
	$\Dom_{\Thash}$ and $\Dom_{\Chash}$ satisfy these requirements thanks to the rules given in Appendix~\ref{app:domsep}.
\end{proof}

%\begin{lemma}\label{th-ks-indiff}
%	Let $\KEkeyspace$ be a set of strings, and let $\SSp$ be the space of all functions with domain $[1] \times \bits^*$ and range $\KEkeyspace$. Additionally, let $\Dom$ be a subset of $\bits^*$. Let $\ESp$ be the space of all functions with domain $(\{1\}\times \Dom) \cup (\{2\} \times (\bits \setminus \Dom))$ and range $\KEkeyspace$.
%	Given an oracle $\hh \in \SSp$, we define the following construction $\construct{F}: \SSp \to \ESp$
%	\[ \construct{F}[\hh](i,\Y) := \hh(\Y).\]
%	 For any adversary $\advD$ making $\qPriv$ queries to the $\PrivO$ oracle and $\qPub$ queries to the $\PubO$ oracle, there exists a simulator $\Sim$ with runtime \TODO{runtime}  \TODO{bound}such that
%	\[\genAdv{\indiff}{\construct{F}, \SSp, \ESp, \Sim}{\advD}\leq \frac{6q_{\PubO}^2}{|\KEkeyspace|} + \frac{8(q_{\text{max length}} \cdot q_{\PrivO})^2}{\KEkeyspace}.\]
%\end{lemma}
%\begin{proof}
%	The construct $\construct{F}$ is the identity functor defined by ~\cite{EC:BDG20}. Let the ``working domain''  $W$ be the set $\ESp.\Domain$. In that paper, we defined a working domain as one that separates domains when contains no two points $(i_1, \Y)$ and $(i_2, \Y)$ such that $i_1 \neq i_2$. For our choice of $W$, if $i_1 \neq i_2$, then for any two points $(i_1, \Y)$ and $(i_2, \Y')$, exactly one of $\Y$ and $\Y'$ may be an element of $\Dom$. Therefore $W$ separates domains, and the construct $\construct{F}$ is read-only indifferentiable over the working domain $W$ by Theorem 4.2 of~\cite{EC:BDG20}. We also note from that paper that read-only indifferentiability implies standard indifferentiability when the working domain equals $\ESp.\Domain$, as it does here. 
%\end{proof}

\def\StepTwoTitle{Step 2: Applying the Indifferentiability of HMAC}
\iffull
	\subsubsection{\StepTwoTitle}
\else
	 \subsection{\StepTwoTitle}
\fi

\label{sec:hmac}
Our next key exchange protocol, $\KE_2$, replaces the construction~$\HMAC[\Chash]$ with a single random oracle~$\ROhmac$ in the implementation of $\abstractMAC$ and by extension $\abstractExtract$ and $\abstractExpand$.
We rely on the proof of $\HMAC$'s indifferentiability by Dodis et al.~\cite[Theorem~3]{C:DRST12}.
As a prerequisite for this theorem, we need to restrict HMAC to keys of a fixed length less than the block length of the hash function ($512$~bits for $\SHA{256}$ and $1024$~bits for $\SHA{384}$).
This is consistent with $\HMAC$'s usage in TLS~1.3, where the keys are almost always of length~$\hashlen \in \{256, 384\}$.
The only exception is when pre-shared keys of another length are negotiated out-of-band; we exclude this case.
\begin{lemma}
	\label{thm:ks-indiff-hop-2-comp}
	Let $\ROthash, \ROchash \colon \bits^* \to \bits^{\hashlen}$ and $\ROhmac \colon \bits^{\hashlen} \times \bits^* \to \bits^{\hashlen}$ be random oracles.
	Let $\KE_1$ be the TLS~1.3 key exchange protocol described in Theorem~\ref{thm:ks-indiff-hop-1-comp} using random oracles~$\ROthash$ and~$\ROchash$.
	Let $\KE_2$ be the key exchange protocol given on the left-hand side of  Figure~\ref{fig:tls-handshake}, where
	\begin{itemize}
		\item $\abstractHash := \ROthash$
		\item $\abstractMAC := \ROhmac$
	\end{itemize}
	and $\abstractExtract$ and $\abstractExpand$ are defined as Section~\ref{sec:tls13-psk-protocol}.
	Let $\advA$ be an adversary against the $\KESEC$ security of~$\KE_1$, running in time~$t_\advA$ and making $\qRO$ and $\qSend$ queries to its random oracle resp.\ $\Send$ oracle.
	Then there exists an adversary $\advB$ against the security of $\KE_2$ such that 
	\[
	\genAdv{\KESEC}{\KE_1}{\advA}
	\leq \genAdv{\KESEC}{\KE_2}{\advB} + \frac{2(12\qSend+\qRO)^2}{2^{\hashlen}}.
	\]
	Adversary~$\advB$ has runtime $\bigO(t_{\advA} + \qRO)$ and makes the same number of queries to each of its oracles as~$\advA$ in the $\KESEC$ game.
\end{lemma}
\begin{proof}
	$\KE_1$ uses function space $\ESp$, defined in the proof of Lemma~\ref{thm:ks-indiff-hop-1-comp}, and $\KE_2$ uses function space $\ESp_2 = \AllFuncs((\{\Thash\} \times \bits^*) \cup (\{\HMAC\} \times \bits^{\hashlen} \times \bits^*), \bits^{\hashlen})$.
	The construction $\construct{C}$ of $\ESp_2$ from $\ESp$ simply forwards all queries to $\ROthash$. It answers $\ROhmac$ queries with $\HMAC [\ROchash ]$.
	
	For any simulator~$\Sim$, Theorem $5$ grants the existence of a distinguisher $\advD$ and an adversary $\advB$ such that 
	\[ \genAdv{\KESEC}{\KE_1}{\advA} \leq \genAdv{\KESEC}{\KE_2}{\advB} + \genAdv{\indiff}{\construct{C},\Sim}{\advD}. \]
	The distinguisher $\advD$ makes up to $12$ queries to $\PrivO$ for each $\Send$ query made by $\advA$, and makes one $\PubO$ query for each $\RO$ query of $\advA$. 
	
	We consider the simulator~$\Sim_2$ defined by Dodis et al. for~\cite[Theorem~4.3]{EPRINT:DRST13} (the full version of~\cite[Theorem~3]{C:DRST12}). 
	This simulator relies on the requirement that $\HMAC$ keys are a fixed length, and shorter than the block length of the underlying hash function.
	$\HMAC$ pads its keys with zero bits up to the block length, so each hash function call made by $\HMAC$ contains a segment containing the byte \texttt{0x36} for the first of the two calls and \texttt{0x5c} for the second.
	$\Sim_2$ uses this segment to identify whether a particular query is intended to simulate the first or second hash function call.
	It answers the ``first'' calls with random strings and logs these responses.
	Then it programs the ``second'' calls by using its stored intermediate values to find which $\ROhmac$ query should be simulated.
	We augment the simulator to forward all queries to $\ROthash$; this does not change its runtime or effectiveness.
	This simulator works perfectly unless there is a collision among the $2\qPriv+\qPub$ intermediate values, which Dodis et al. bound with a birthday bound.
	That theorem states that for a distinguisher $\advD$ making $12\qSend$ queries to $\PrivO$ and $\qRO$ queries to $\PubO$, 
	\[ \genAdv{\indiff}{\construct{C}, \Sim}{\advD} \leq \frac{2(12\qSend+\qRO)^2}{2^{\hashlen}}. \]
	The lemma follows.
\end{proof}

\def\StepThreeTitle{Step 3: Applying Indifferentiability to the TLS Key Schedule}
\iffull
	\subsubsection{\StepThreeTitle}
\else
	 \subsection{\StepThreeTitle}
\fi

\label{sec:many-ros}
In the last step, we move to the right-hand side of Figure~\ref{fig:tls-handshake} and introduce $11$ new independent random oracles to model the key schedule.
We start by rephrasing the TLS key schedule and message authentication codes as eleven functions $\TLSKDF_{\binder}$, \dots, $\TLSKDF_{\rms}$ as in Section~\ref{sec:tls13-psk-protocol}.
This abstraction does not change any of the operations performed by the key schedule; the $\TLSKDF$ functions simply rename the key derivation steps already performed by $\KE_2$. 
In our last key exchange protocol $\KE'$, we model each $\TLSKDF$ function as a independent random oracle. 
We name these oracles after the keys or values they derive:
{\allowdisplaybreaks
\begin{align*}
1.~ &\RO_{\binder}[\ROhmac]	&&\colon \bits^{\hashlen}  \times \bits^{\hashlen}\to \bits^{\hashlen} \\
2.~ &\RO_{\ets}[\ROhmac]	&&\colon \bits^{\hashlen} \times \bits^{\hashlen} \to \bits^{\hashlen} \\
3.~ &\RO_{\eems}[\ROhmac]	&&\colon \bits^{\hashlen} \times \bits^{\hashlen}  \to \bits^{\hashlen}\\
4.~ &\RO_{\chtk}[\ROhmac]	&&\colon \bits^{\hashlen}\times \G \times \bits^{\hashlen} \to \bits^{\hashlen + \ivlen} \\
5.~ &\RO_{\cfin}[\ROhmac]		&&\colon \bits^{\hashlen} \times \G \times \bits^{\hashlen} \times \bits^{\hashlen} \to \bits^{\hashlen} \\
6.~ &\RO_{\shtk}[\ROhmac]	&&\colon \bits^{\hashlen} \times \G \times \bits^{\hashlen} \to \bits^{\hashlen + \ivlen} \\
7.~ &\RO_{\sfin}[\ROhmac]		&&\colon \bits^{\hashlen} \times \G \times \bits^{\hashlen} \times \bits^{\hashlen} \to \bits^{\hashlen} \\
8.~ &\RO_{\cats}[\ROhmac]	&&\colon \bits^{\hashlen}\times \G \times \bits^{\hashlen} \to \bits^{\hashlen} \\
9.~ &\RO_{\sats}[\ROhmac] &&\colon \bits^{\hashlen} \times \G \times \bits^{\hashlen} \to \bits^{\hashlen} \\
10.~ &\RO_{\ems}[\ROhmac]	&&\colon \bits^{\hashlen} \times \G \times \bits^{\hashlen} \to \bits^{\hashlen} \\
11.~ &\RO_{\rms}[\ROhmac]	&&\colon \bits^{\hashlen} \times \G \times \bits^{\hashlen} \to \bits^{\hashlen}
\end{align*}
}
The $12^\text{th}$ random oracle is $\ROthash$, used to hash transcripts as in $\KE_1$ and $\KE_2$.


Now we can state Lemma~\ref{th-ks-indiff}.
\begin{lemma}\label{th-ks-indiff}
	Let $\KE_2$ be the key exchange protocol of Lemma~\ref{thm:ks-indiff-hop-2-comp}, and let $\KE'$ be the key exchange protocol of Theorem~\ref{thm:full-ks-indiff}. 
	
	For any adversary $\advA$ against the $\KESEC$ security of $\KE_2$, with runtime $t$ and making $\qRO$ random oracle queries and $\qSend$ queries to $\Send$, there exists adversary $\advB$ against the $\KESEC$ security of $\KE'$ such that
	\[
	\genAdv{\KESEC}{\KE_1}{\advA}
	\leq \genAdv{\KESEC}{\KE_2}{\advB} + \frac{2\qPub^2}{2^{\hashlen}} + \frac{8(\qPub+6\qPriv)^2}{2^{\hashlen}}.
	\]
	Adversary $\advB$ runs in time at most $t + \qRO t_{\G}$, where $t_{\G}$ is the time to perform one group operation in the Diffie--Hellman group $\G$. 
	It makes no more queries to each of the oracles in the $\KESEC$ game than does $\advA$.
\end{lemma}

\begin{proof}
	We view $\TLSKDF$ as defined in Section~\ref{sec:tls13-psk-protocol} as a construction of the function space $\ESp'$ of $\KE'$:
	the arity-$12$ function space whose first subspace is $\AllFuncs(\bits^*\cab \bits^\hashlen)$ and whose remaining 11 subspaces are the spaces of all functions with the domains and ranges specified in the above list. 
	This $\TLSKDF$ construction takes an oracle from $\ESp_2$, the function space of $\KS_2$. 
	
	As in the prior two steps, we consider a particular simulator~$\Sim$ (cf.\ Figure~\ref{fig-sim-ks-indiff}) and rely on Theorem $5$ for the existence of a distinguisher $\advD$ and an adversary $\advB$ such that 
	\[ \genAdv{\KESEC}{\KE_2}{\advA} \leq \genAdv{\KESEC}{\KE'}{\advB} + \genAdv{\indiff}{\construct{\TLSKDF},\Sim}{\advD}. \]
	The distinguisher $\advD$ will make no more than $12$ queries to $\PrivO$ for each $\Send$ query made by $\advA$ and one query to $\PubO$ per $\RO$ query.
	
	Via a sequence of code-based games, we will show that the indifferentiability advantage of any distinguisher $\advD$ making $\qPriv$ queries to the $\PrivO$ oracle and $\qPub$ queries to the $\PubO$ oracle is
	\[\genAdv{\indiff}{\construct{\TLSKDF}, \SSp, \ESp, \Sim}{\advD}\leq \frac{2\qPub^2}{2^{\hashlen}} + \frac{8(\qPub+6\qPriv)^2}{2^{\hashlen}}.\]
	We give fully specified pseudocode for each of our games. 
	
	First, we explain the high-level strategy of our simulator.
	Our simulator takes two inputs: an index $\simIndex \in \{\Thash, \HMAC \}$ and a string $\simInString \in \bits^*$. 
	When $\simIndex = \Thash$, the simulator simulates $\ROthash(\simInString)$ easily; it simply forwards the query to its own random oracle $\ROthash$.
	When $\simIndex = \HMAC$, the simulator will parse $\simInString$ into a key $\hmackey \in \bits^{\hashlen}$ and a context string $\hmactext \in \bits^*$ and simulate $\ROhmac(\hmackey,\hmactext)$. 
	This simulation should be compatible with a view of the random oracles $\RO_x$ as computing $\TLSKDF_x[\ROhmac]$. 
	
	Initially, $\Sim$ randomly samples the response $\simOut$ to any simulated $\ROhmac$ query from $\bits^{\hashlen}$.
	Repeated queries are cached in a table $\cachetable$.
	Next, $\Sim$ checks whether the query could be part of an attempt to compute $\TLSKDF_x[\Sim]$ for some $x$.
	If so, it may have to program its response for consistency with $\RO_x$, or it may store its response in a lookup table $\rlookuptable$ to enable future programming.
	
	The only values that need programming are the first-class keys and MAC values.
	These are all outputs of $\abstractExpand[\ROhmac]$. 
	$\Sim$ can tell if a particular $\ROhmac$ query is made by $\abstractExpand$ by checking its formatting.
	The inputs $\hmactext$ of all $\abstractExpand$'s queries in the key schedule start with $3$ bytes of fixed values and a label $\lbl$ between $8$ and $18$ bytes long that starts with the string \texttt{``tls13''}. They end with a $1$~byte counter that TLS~1.3 fixes to \texttt{0x01}.
	$\Sim$ pattern-matches this label to determine which key is being derived. 
	It has a subroutine $\lblT$ to translate the few labels which are used in the last derivation step for multiple keys.	
	
	Whenever $\simulator$ detects the label of an intermediate key derivation query like the $\abstractExpand$ calls used to compute $\es$, $\hs$, or $\ms$, it stores the response to this query in table $\rlookuptable$ under the name of the key in question.
	If $\advD$ computes $\TLSKDF$ honestly, these tables will allow the simulator to backtrack through the execution to identify all of the inputs to $\TLSKDF$. 
	Inputs to $\ROhmac$ queries made by $\HKDFExtr$ do not contain labels, so some tables contain multiple intermediate values.
	Even without labels, each intermediate value should only appear in one key derivation except in the unlikely event of a collision in $\ROhmac$.  
	
	\begin{figure}[tp]
		\begin{minipage}[t]{0.45\textwidth}
			\NewExperiment[$\Sim(\simIndex,\simInString)$]
			
			\begin{oracle}{$\Sim[\RO](\simIndex,\simInString)$}
				\item if $\cachetable[\simInString] \neq \bot$
				\item \quad then return $\cachetable[\simInString]$
				\item if $\simIndex=\Thash$ then return $\ROthash(K \| \Y)$
				\item[] \comment{If not, this query should simulate $\ROhmac$}
				\item $\hmackey$, $\hmactext \gets \simInString$
				\item[] \comment{Randomly sample a response}
				\item $\simOut \getsr \bits^{\hashlen}$
				
				%computing ES from PSK
				\item if $\hmactext = 0$ 
				\item \quad $\rlookuptable_{\psk}[\simOut] \gets \hmackey$
				
				%computing MS from DHS
				\item else if $\hmackey = 0$ 
				\item \quad $\rlookuptable_{\dhs}[\simOut] \gets \hmactext$ 
				
				% computing MAC tag from BFK, CFK, or SFK
				\item else if $\rlookuptable_{\bfk/\cfk/\sfk}[\hmackey]\neq \bot$
				
				% computing binder from BFK
				\item \quad $\es \gets \rlookuptable_{\es}[\rlookuptable_{\bk/\chts/\shts}[\hmackey]]$
				\item \quad $\psk \gets \rlookuptable_{\psk}[\es]$
				\item \quad if $\psk \neq \bot$ 
				\item \quad \quad $\simOut \gets \RObinder(\psk, \hmactext)$
				
				% computing finished msg from CFK or SFK
				\item \quad \quad $\hts \gets \rlookuptable_{\bk/\chts/\shts}[\hmackey]$
				\item \quad $(\kdflbl', \hs, \hash_2) \gets \rlookuptable_{\hs/\hashcontext}[\hts]$
				\item \quad $(\des, \dhe) \gets \rlookuptable_{\des/\dhe}[\hs]$
				\item \quad $\psk \gets \rlookuptable_{\psk}[\rlookuptable_{\es/\hs}[ \des]]$
				\item \quad if $\psk \neq \bot$ 
				\item \quad \quad $y \gets \RO_{\kdflbl'[1]}(\psk, \dhe,\hash_2, \hmactext)[	\lblT(\kdflbl)]$
				
				%computing HS from DES + DHE
				\item else $\rlookuptable_{\des/\dhe}[\simOut] \gets(\hmackey, \hmactext)$
				
				\item if $(\hmactext[0\ldots2] \neq \hashlen)$
				\item[]\quad$\vee\hmactext[2] < 8) \vee(\hmactext[2] > 18)$
				\item[] \quad $\vee(\hmactext[3\ldots9] \neq \mathsf{"tls13 "})$
				\item[]\quad$\vee (\hmactext[|\hmactext|-1] \neq  1)$ 
				\item[] \quad \comment{This query does not match $\HKDFExpnd$ formatting.}
				
				\item \qquad $\cachetable[\simInString] \gets \simOut$
				\item \qquad return $\simOut$
				
				\item[] \comment{Parse the $\abstractExpand$ formatting to find the label.} 
				\item $\kdflbllen \gets \hmactext[2]$
				\item $\kdflbl \gets \hmactext[3\ldots (3+\kdflbllen)]$
				\item $\hashcontext \gets \hmactext[(3+\kdflbllen)\ldots|\hmactext|]$
				\item[] \ldots \comment{continued in next column}
			\end{oracle}
			
			
		\end{minipage}
		\begin{minipage}[t]{0.54\textwidth}
			\ExptSepSpace
			\begin{oracle}{$\Sim[\RO](\simIndex,\simInString)$\comment{continued}}
				%query computes BK from ES
				\item if $\kdflbl = \kdflbl_{\binder}$ and $\hashcontext = \hash(\emptymessage)$
				\item \quad  $\rlookuptable_{\es}[\simOut] \gets \hmackey$
				
				% query computes dES or dHS from ES or HS
				\item else if $\kdflbl = \kdflbl_{\des/\dhs}$ and $\hashcontext = \hash(\emptymessage)$
				\item \quad $\rlookuptable_{\es/\hs}[\simOut] \gets \hmackey$
				
				% Computing CHTS or SHTS from HS
				\item  else if $\kdflbl \in \{\kdflbl_{\chts}, \kdflbl_{\shts}\}$
				\item \quad $\rlookuptable_{\hs/\hashcontext}[\simOut] \gets (\lblT(\kdflbl), \hmackey, \hashcontext)$
				
				% Computing ETS or EEMS from ES
				\item else if $\exists k \in \{\ets, \eems\}$ with  $\kdflbl  = \kdflbl_k$ and $\rlookuptable_{\psk}[\hmackey] \neq \bot$
				\item \quad $\simOut \gets \RO_{k}(\rlookuptable_{\psk}[\hmackey],\hashcontext)$
				
				% computing CATS, SATS, EMS or RMS from MS
				\item else if $\exists k \in \{\cats, \sats, \ems, \rms\}$ with  $\kdflbl  = \kdflbl_k$
				\item \quad $(\des, \dhe) \gets \rlookuptable_{\des/\dhe}[\rlookuptable_{\es/\hs} [ \rlookuptable_{\dhs}[\hmackey]]]$
				\item \quad $\psk \gets \rlookuptable_{\psk}[\rlookuptable_{\es/\hs}[ \des]]$
				\item \quad if $\psk\neq \bot$
				\item \quad \quad  $\simOut \gets \RO_{k}(\psk, \dhe, \hashcontext)$ 
				
				% Computing BFK, CFK, or SFK from BK, CHTS, or SHTS
				\item else if $\kdflbl = \kdflbl_{\fk}$ and $\hashcontext = \emptymessage$ 
				%\item[] \comment{This query must simulate either the binder key (case 1), or the client or server finished key (case 2)}
				\item \quad $\rlookuptable_{\bk/\chts/\shts}[\simOut] \gets \hmackey$
				
				% computing tkchs or tkshs (key or iv) from CHTS or SHTS
				\item else if $\kdflbl \in \{\texttt{"tls13 key"}, \texttt{ "tls13 iv"} \}$
				\item \quad and $\hashcontext = \hash(\emptymessage)$
				\item \quad $(\kdflbl', \hs, \hash_2) \gets \rlookuptable_{\hs/\hashcontext}[\hmackey]$
				\item \quad $(\des, \dhe) \gets \rlookuptable_{\des/\dhe}[\hs]$
				\item \quad $\psk \gets \rlookuptable_{\psk}[\rlookuptable_{\es/\hs}[ \des]]$
				\item \quad if $\psk \neq \bot$ 
				\item \quad \quad $y \gets \RO_{\kdflbl'[0]}(\psk, \dhe, \hash_2)[	\lblT(\kdflbl)]$
				\item []
				
				\item $\cachetable[\simInString] \gets \simOut$
				\item return $\simOut$
			\end{oracle}
			
			\ExptSepSpace	
			\begin{oracle}{Label translator $\lblT(\kdflbl)$}
				\item if $\kdflbl  = \kdflbl_{\chts}$
				\item \quad return $\chtk, \ClientFinished$
				\item if $\kdflbl  = \kdflbl_{\shts}$
				\item \quad return $\shtk, \ServerFinished$
				\item if $\kdflbl =\texttt{"tls13 key"}$
				\item \quad return $0$
				\item if $\kdflbl =\texttt{"tls13 iv"}$ 
				\item \quad return $1$
				\item return $\bot$
			\end{oracle}
		\end{minipage}
		\label{fig-sim-ks-indiff}
		\caption{Simulator $\Sim$ used in the proof of Lemma~\ref{th-ks-indiff}.}
	\end{figure}
	
	The first game in our sequence is $\Gm_0$~\, which is the ``ideal world'' setting of the indifferentiability game.
	Here, $\PrivO$ queries are answered using a random function $\RO$ drawn from $\ESp$, and $\PubO$ queries are answered with $\Sim[\RO]$.
	
	In $\Gm_1$ (cf.\ Figure~\ref{fig:gm1-ks-indiff}), we set a bad flag $\bad_C$ and abort whenever $\Sim$ samples a random answer $\simOut$ that collides with the input or output of any previous simulator query. 
	We track these inputs and outputs in a list $L$. 
	For each new query, there are at most $2\qPub$ points to collide with.
	Since $\simOut$ is sampled uniformly from $\bits^{\hashlen}$, the probability of such a collision over all queries is at most $\frac{2\qPub^2}{2^{\hashlen}}$ by a birthday and union bound). Then 
	
	\[ |\Pr[\Gm_1] - \Pr[\Gm_0]| \leq \frac{2\qPub^2}{2^{\hashlen}}. \] 
	
	In $\Gm_2$ (Figure\ref{fig:gm23-ks-indiff}), the $\Finalize$ oracle computes $\TLSKDF[\ROhmac]$ on the input to every query to the $\PrivO$ oracle, using $\PubO$ as its hash function. It discards the results of this computation, so this change can affect the outcome of the game only if one of the additional $\PubO$ queries sets the $\bad_C$ flag. The $\TLSKDF$ function queries its oracle at most~$6$ times per execution, so there are no more than $6\qPriv$ new queries. There are now a total of $\qPub + 6\qPriv$ queries to $\PubO$, so the probability that $\bad_C$ is set increases by another birthday bound.
	\[ |\Pr[\Gm_2] - \Pr[\Gm_1]| \leq \frac{2(\qPub+6\qPriv)^2}{2^{\hashlen}}. \] 
	
	
	The next step is the most subtle. 
	In $\Gm_3$ (Figure~\ref{fig:gm23-ks-indiff}), we move the new computations of $\TLSKDF$ from the $\Finalize$ oracle into $\PrivO$.
	When $\PrivO$ is called with index $\simIndex$ and input $X$, it still returns $\RO_{\simIndex}(X)$. 
	First, however, it computes $\TLSKDF_{\simIndex}[\PubO](X)$. 
	It discards the result of this computation, so the behavior of the $\PrivO$ oracle does not change in the adversary's view.
	
	However, queries to $\PrivO$ now run the simulator $\Sim$.
	They can update its state and set the global $\bad_C$ flag.
	This has two consequences.
	First, the changed order of $\PubO$ queries may cause $\bad_C$ to be set in $\Gm_3$ when it was not set in $\Gm_2$, or vice versa. 
	Second, queries to $\PrivO$ in $\Gm_3$ can add entries to the reverse lookup table $\rlookuptable$.
	These new entries can be used to satisfy the conditions the simulator uses to check if a full execution of $\TLSKDF$ has been completed.
	Then the simulator in $\Gm_3$ may program responses that were not programmed in $\Gm_2$. 
	
	We claim that despite the changed order of the queries, $\Gm_3$ and $\Gm_2$ behave identically in the adversary's view except when one of them would set the $\bad_C$ flag, assuming that the same random coins are used in both games.
	Let $E$ denote the event that $\bad_C$ is set either when $\advA$ plays $\Gm_2$ or when $\advA$ plays $\Gm_3$.
	Differences between the two games about when this flag is set are obviously irrelevant unless event $E$ occurs. 
	
	The argument that $\PubO$ responses are identical in both games except when event $E$ occurs is more subtle.
	Assume event $E$ does not occur.
	There must be a first adversarial query to $\PubO$ that gives different responses in $\Gm_3$ and $\Gm_2$, all oracles behave identically in both games.
	We name this query $Q$.
	Both games sample the same random responses, so query $Q$ has its response programmed by the simulator in at least one of the two games.
	
	The simulator decides whether to program based on the entries of reverse lookup table $\rlookuptable$, so we consider the differences in this table between our two games.
	Let $\rlookuptable_2$ be the table in $\Gm_2$ at the time when Query $Q$ is made, and let $\rlookuptable_3$ be the table at the same point in $\Gm_4$.
	Entries in the reverse lookup table are indexed by randomly sampled values $\simOut$, so they cannot be overwritten by later queries unless event $E$ occurs.
	Furthermore, until query $Q$ is made, every $\PubO$ query in $\Gm_2$ that updates $\rlookuptable$ gives the identical response in $\Gm_3$, so every entry in $\rlookuptable_2$ is also an entry in $\rlookuptable_3$.
	Therefore any query which is programmed in $\Gm_2$, up to and including query $Q$, will be programmed to the same response in $\Gm_3$.
	The contrapositive statement says that any response which is randomly sampled in $\Gm_3$ will be also be randomly sampled in $\Gm_2$.
	
	It follows that query $Q$ must have a randomly sampled response in $\Gm_2$ but be programmed in $\Gm_3$.
	There must exist a sequence of entries in $\rlookuptable_3$ that correspond to a full execution of $\TLSKDF[\PubO]$ on some input. 
	We name the queries that created these entries $Q_1, \ldots, Q_i$. 
	In each execution, our simulator either stores an entry in $\rlookuptable$, or it programs the response $\simOut$, never both.
	Therefore queries $Q_1, \ldots Q_i$ have randomly sampled responses.
	By the definition of $\TLSKDF$, the output of each query $Q_j$ is contained in the input of the next query $Q_{j+1}$.
	The output of $Q_i$ is contained in the input of $Q$, so we identify query $Q$ with $Q_{i+1}$.
	
	In $\Gm_2$, one of the entries in the sequence is not present in $\rlookuptable_2$. 
	Therefore one of the queries $Q_1, \ldots, Q_i$ is not made before query $Q$ in $\Gm_2$.
	This query, $Q_j$ must have been one of the $\Finalize$ queries of $\Gm_2$ that were moved earlier in $\Gm_3$.
	It will therefore be made in $\Finalize$, after all of the other queries, including $Q_{j+1}$.
	The randomly sampled output of $Q_j$ will collide with the input of earlier query $Q_{j+1}$, setting $\bad_C$ and causing event $E$ to occur.
	
	The difference in advantage in $\Gm_3$ and $\Gm_2$ is therefore bounded by the probability of event $E$.
	Both games make $\qPub+6\qPriv$ queries to $\PubO$, each of which sets $\bad_C$ is set with probability at most $\frac{2(\qPub+6\qPriv)}{2^{\hashlen}}$. 
	By a union bound,
	\[|\Pr[\Gm_3] - \Pr[\Gm_2]| \leq \frac{4(\qPub+6\qPriv)^2}{2^{\hashlen}}.\]
	
	%%% OLD DRAFT
	%In case $1$, one of the if-conditions in lines $13$, $31$, $36$, or $45$ of our original simulator (cf.\ Figure~\ref{fig-sim-ks-indiff}) must have been satisfied in $\Gm_4$. 
	%In $\Gm_2$, none of these four if-conditions are satisfied when Query $1$ is made.
	%For any of the four conditions to be satisfied in $\Gm_4$, there must exist a sequence of table entries in $\rlookuptable$ such that the contents of each entry are the index of the next entry. 
	%The index of the first entry is $\hmackey_1$, and the contents of the last entry are $\psk_1 \neq \bot$. 
	%In $\Gm_2$, at least one of the entries in this sequence is not present in $\rlookuptable$.
	%That means that  
	%The last query in the sequence must have output either $\hmackey_1$ or $\hmactext_1$. 
	%We break the logic down by each of the four possible conditions which could be satisfied in $\Gm_4$, starting with line $31$.
	%If $\rlookuptable_{\psk}[\hmackey^*] = \bot$ in $\Gm_2$ and $\rlookuptable_{\psk}[\hmackey^*]\neq \bot$ in $\Gm_4$, then some query 
	%$\hmackey^*$ must have been a randomly sampled response to an earlier $\PubO$ query.
	%However, this earlier query took place only in $\Gm_4$, meaning it must have been made by the $\PrivO$ oracle and not the adversary.
	%In $\Gm_2$, the same query would be made in the $\Finalize$ oracle at the end of the game.
	%This means that a query would randomly sample the response $\hmackey*$ \emph{after} the earlier query $\PubO(\hmackey^*, \hmactext^*)$ used the same value as an input. 
	%The later query will set $\bad_C$, and our claim holds in this case.
	%
	%Each of the other three conditions also requires that an entry in $\rlookuptable_{\psk}$ is set to $\bot$ in $\Gm_2$ but not $\Gm_4$. 
	%In these three cases, however, the entry in $\rlookuptable$ may have index $\bot$ in $\Gm_2$ because an entry in another reverse lookup table was set to $\bot$ in $\Gm_2$ but not $\Gm_4$.
	%Regardless of which table contains this entry, the value must have been sampled randomly by an earlier $\PubO$ query, one which takes place in $\PrivO$ in $\Gm_4$ and in $\Finalize$ in $\Gm_2$.
	%Let this be Query $1$.
	%Let $Z$ be the output of this Query $1$ (the index of the entry).
	%
	%There are two situations:
	%either $Z$ contains one or both values $\hmackey*$ and $\hmactext*$, or $Z$ itself is the contents of a $\rlookuptable$ entry which is identical in $\Gm_2$ and $\Gm_4$.
	%In the former case, $Z$ will collide with $\hmackey^*$ or $\hmactext^*$ in the $\Finalize$ oracle of $\Gm_2$, and the $\bad_C$ flag will be set.
	%In the latter case, we give the index of the entry containing $Z$ the name $Z'$
	%Unlike $Z$, $Z'$ is defined in both $\Gm_2$ and $\Gm_4$.
	%$Z'$ is added to $\rlookuptable$ in both $\Gm_2$ and $\Gm_4$ before $\Finalize$, so it is an output of some adversarial query Query $2$.
	%Note that Query $2$ must have occurred before Query $1$; otherwise the entry with index $Z$ would not exist.
	%The input of Query $2$ is $Z$.
	%Then when Query $1$ occurs in the $\Finalize$ oracle of $\Gm_2$ and samples response $Z$, it will collide with the logged input of Query $2$ and set the $\bad_C$ flag.
	%
	%Therefore any time $\PrivO$ would give different responses in $\Gm_4$ than $\Gm_2$, $\Gm_2$ would have set $\bad_C$. It follows that 
	%
	%\[|\Pr[Gm_4] - \Pr[Gm_2]| \leq Pr[\Gm_2\text{ sets }\bad_C] \leq \frac{2(\qPub+6\qPriv)^2}{2^{\hashlen}}.\]
	
	%%%% OLDEST DRAFT
	
	%If this condition is true in $\Gm_2$ and false in $\Gm_4$, then $\rlookuptable_{\psk}[\hmackey^*]$
	%
	%Therefore there is some entry $\rlookuptable_{\psk}[\simOut^*]$ which contains $\bot$ in $\Gm_2$ but does not contain $\bot$ in $\Gm_4$. 
	%This entry must have been created by one of the $\PubO$ queries we moved from $\Finalize$ to $\PrivO$; it was not created by an adversarial $\PubO$ query or it would have created the entry in both $\Gm_4$ and $\Gm_2$.
	%$\simOut^*$ was a random response sampled by this query; since the $\PrivO$ oracle discards the output of $\TLSKDF$, it was not returned to the adversary. it also appears as the contents of an entry in either table $\rlookuptable_{\es}$ or table $\rlookuptable_{\es/\hs}$. 
	%This means $\simOut^*$ was also an input to some $\PubO$ query
	%
	%
	%
	%We claim that if this events. Let $\PubO(K, \Y)$ be the first query which gives a different response in $\Gm_4$ than in $\Gm_3$. This means that in $\Gm_3$, the response to $\PubO(K, \Y)$ was uniformly distributed, but in $\Gm_4$, the query triggered an if-condition at one of lines $29$, $35$, $41$, $46$, or $53$. 
	%
	%Although the exact if-conditions differ, they all require that for some index $j$ and some tuple $t$, $T_j[t] \neq \bot$ and $K \in t$. For $K$ to be stored in a table $T$, some earlier query $\PubO(K', \Y')$ must have randomly sampled $K$ as its response. In $\Gm_3$, where $\PubO(K, \Y)$ is not programmed, the query $\PubO(K', \Y')$ did not occur. The adversary's view is the same in both games until $\PubO(K, \Y)$ occurs, so the order of its queries do not change. Then the query $\PubO(K', \Y')$ was not made by the adversary; instead, it must be one of the queries we moved from $\Finalize$ to $\PrivO$ between games $\Gm_3$ and $\Gm_4$. What happens in $\Gm_3$, when $\PubO(K', \Y')$ is queried in $\Finalize$? There are two possibilities: either it is answered randomly as in $\Gm_4$, or it is programmed. In the former case, the query returns $K$, which is in $L$ thanks to $\PubO(K, \Y)$. This causes a collision and $\bad_C$ is set. In the latter case, $\PubO(K', \Y')$ was itself programmed in $\Gm_3$; its response is therefore different from $K$. We claim that a collision must also have occured in this case.
	%
	%Since $\PubO(K', \Y')$ occurs in $\Finalize$ in $\Gm_3$, it arises during an honest computation of $\TLSKDF$. Since it is not programmed in $\Gm_4$, it is not the last query in this computation; its output represents an intermediate value. However, that it is programmed means that this query is also the final query in a different, possibly adversarial computation of $\TLSKDF$. The two computations have distinct inputs (otherwise the queries would be cached), so in order for them to contain identical queries, there must be a collision in the outputs of earlier queries. \hd{this logic is very unsound.} This earlier collision would set $\bad_C$. 
	%
	%The changed state of the simulator also affects whether the $\bad_C$ flag will be set in $\Gm_4$. Collisions among outputs are no more or less likely depending on the order of the queries. However, in $\Gm_3$ we also set the $\bad_C$ flag when an output from $\PubO$ collides with a prior input by the adversary. This captures the possibility that the adversary guesses an intermediate state of the KDF. Because the $\PrivO$ oracle also queries $\PubO$, the $\bad_C$ flag now may not be set in the event of such a guess. This decreases $\Pr[\Gm_4]$ by an amount no greater than $\Pr[\Gm_3 \text{sets} \bad_C]$.
	%
	%We know that $\Pr[\Gm_4 \text{ sets } \bad_C] \leq \Pr[\Gm_3 \text{ sets } \bad_C] \leq\frac{2\qPub^2}{2^{\hashlen}} + \frac{2(6 \qPriv)^2}{2^{\hashlen}}$, so we have
	%\[|\Pr[Gm_4] - \Pr[Gm_3]| \leq Pr[\Gm_3\text{ sets }\bad_C] \leq \frac{2(\qPub+6\qPriv)^2}{2^{\hashlen}}.\]
	Pseudocode for the last three games is given in Figure~\ref{fig:gm456-ks-indiff}.
	Now we adjust $\PrivO$ in $\Gm_4$ to return the result of $\construct{C}[\PubO]$ instead of querying $\RO$. 
	Unless $\bad_C$ is set, $\TLSKDF[\PubO](\roIndex, \X) = \RO_{\roIndex}( \X)$. 
	The function $\TLSKDF$ makes sequential queries to $\PubO$ that are properly formatted, so our $\Sim$ will program the last query in the sequence for consistency with the appropriate $\RO$. 
	This programming occurs every time $\TLSKDF[\PubO]$ is called, unless the last query is a repeated query.
	In that case, it will be answered using table $\cachetable$ instead of $\RO$.
	However, if the queries in the sequence occur out of order, they will always cause $\bad_C$ to be set because the output of a later query will match the input to an earlier query.
	Then the adversary wins in $\Gm_4$ with the same likelihood as $\Gm_3$, unless $\bad_C$ is set. 
	If $\bad_C$ is set, both games have a win probability of $0$ thanks to the check in the $\Finalize$ oracle, so
	\[\Pr[\Gm_4] = \Pr[\Gm_3].\]
	
	Starting with $\Gm_5$, we stop returning $0$ in $\Finalize$ when $\bad_C$ is set. This increases the win probability by at most $\Pr[\Gm_4 \text{ sets }\bad_C] \leq \frac{2(\qPub+6\qPriv)^2}{2^{\hashlen}}$, by the same birthday and union bounds over the $\qPub + 6\qPriv$ queries to $\PubO$. 
	\[|\Pr[\Gm_5] - \Pr[\Gm_4]| \leq \frac{2(\qPub+6\qPriv)^2}{2^{\hashlen}}.\]
	
	From $\Gm_4$ onward, all queries to $\ROhmac$ are made by $\simulator$. In $\Gm_6$, therefore, we can inline the lazily sampled $\ROhmac$ oracle as part of the simulator. 
	Repeated queries to $\simulator$ are cached, so the random oracle does not need to maintain its own lookup table. 
	Now all responses from $\PubO$ are randomly sampled from $\bits^{\hashlen}$, regardless of the contents of table $\rlookuptable$. 
	The table and the conditional statements used to maintain it are now redundant bookkeeping, as is the unused $\bad_C$ flag after $\Gm_5$. 
	We eliminate all of this code from $\Gm_6$ without detection by the adversary.
	Then 
	\[ \Pr[\Gm_6] = \Pr[\Gm_5]. \]
	
	The remaining code of $\simulator$ just implements random oracles $\ROhmac$ and $\ROthash$.
	Consequently $\Gm_6$ is identical to the ideal indifferentiability game for the $\TLSKDF$ construction. 
	Collecting bounds proves the theorem.
	
	
	%%%%%%%%%%%%%%%%%%%%%%%%%%%%%%%%%%%%%%%%%%%%%%%%%%%%%
	
	%In $\Gm_1$, we introduce a private random oracle $\hh'$ with domain $\bits^*$ and range $\KEkeyspace$. The simulator computes the output $y$ by calling $\hh'((\HKDFExtr, \HKDFExpnd)||\Y)$ instead of randomly sampling $y$. Using a random oracle means that repeated queries will receive identical responses; however this was already guaranteed since $\PubO$ caches its responses in table $M$. Then $\Pr[\Gm_0] = \Pr[\Gm_1].$
	%
	%Next we exclude collisions among the simulator's responses. Starting with game $\Gm_2$, the simulator stores all its queries and responses in a list $L$. If it generates a response $y$ that collides with a prior response or with part of a prior query, we set a flag called $\bad_C$. In $\Gm_3$, the Finalize oracle returns $0$ if the $\bad_C$ flag has been set. By caching queries as well as responses, we are also excluding the possibility that the adversary can compute $\TLSKDF$ by guessing an intermediate value. Accurate guesses will cause the $\bad_C$ flag to be set. 
	%We have that $\Pr[\Gm_2]=\Pr[\Gm_1]$ because the list $L$ is fully internal to the game,
	%and we have $\Pr[\Gm_3] \replace{\leq}{\geq} \Pr[\Gm_2]$ because the Finalize check strictly decreases the probability that the game returns $1$ and no other oracle's behavior is changed.
	%
	%In game $\Gm_4$, we execute the construct $\construct{C}$ in the $\PrivO$ oracle using the simulator. The calls to $\simulator$ from $\PrivO$ use a separate, independent state from those in $\PubO$. Therefore repeated queries from $\PrivO$ and $\PubO$ do not cache their responses, and prior queries from $\PrivO$ do not influence programming in $\PubO$. Simulator calls made by $\PubO$, update both states, so programming in $\PrivO$ is dependent on prior queries from $\PubO$. The responses of the $\PrivO$ oracle do not change. For this reason, the adversary's view of $\PrivO$ and $\PubO$ does not change between $\Gm_3$ and $\Gm_4$. However, the simulator calls in $\PrivO$ use the same list $L$ to track collisions as those in $\PubO$, so the additional queries increase the chance that the $\bad_C$ flag will be set. Therefore $\Pr[\Gm_4]\leq \Pr[\Gm_3]$. 
	%
	%In game $\Gm_5$, $\Sim$ sets a new flag if it encounters an $\HKDFExpnd$ that should be programmed based on tables $T'$ but not based on tables $T$. In $\Gm_6$, we return $0$ in $\Finalize$ whenever this flag has been set. (This occurs only when the adversary has guessed an intermediate state of some $\TLSKDF$ without obtaining it from $\Sim$.) Also in $\Gm_6$, we do program when the flag is set, meaning that $\Sim$ effectively programs using tables $T'$. 
	%
	%In $\Gm_7$, $\Sim$ only uses tables $T'$ and list $M'$. This changes nothing: $\Sim$ was already programming based on tables $T'$ and any value cached in $M'$ would have been identically derived by $\Sim$ because $\Sim$ and $\Sim_2$ use the same private RO to sample random coins. (This could only be untrue if $\Sim_2$ sampled a random output for $y$ because it did not need to be programmed at the time, but later $\Sim$ wished to program $y$. Since $\Sim_2$ is only called in evaluating $\TLSKDF$, however, the last query in the chain is always programmed. Therefore $\Sim$ would need to program a query that was intermediate in a different $\TLSKDF$ execution. This would imply a collision occurred and the $\bad_C$ flag would have been set.) $\Sim$ does still maintain table $T$ in order to set the bad flag from $\Gm_5$.
	%
	%In $\Gm_8$, we stop using a private random oracle and return to random sampling. Since all values are cached between $\Sim$ and $\Sim_2$ by means of $M'$, it does not matter that repeated queries become inconsistent. 
	%
	%In $\Gm_9$, the Priv oracle returns the output of $\Sim_2$, not of the RO. We claim that the output is always identical unless one of our bad flags is set. (This is true since the last query in any $\TLSKDF$ execution is always programmed except in the event of a collision.) From this point forward, the random oracle $\RO$ is not accessible to the adversary. Additionally, no query to $\RO$ is repeated by $\Sim$ or $\Sim_2$ except when a collision flag would be set. Therefore in $\Gm_{10}$ we replace $\RO$ with random sampling. 
	%
	%In $\Gm_{11}$, we stop returning $0$ when the $\bad$ flags have been set. We bound the probability of this event using a birthday bound for collisions and a union bound for guesses. At this point, the checks for programming the queries and setting the bad flags are redundant, so we remove them in $\Gm_{12}$ along with the unused $T$ and $T'$ tables. The remaining code in both $\Sim$ and $\Sim_2$ is just that of a random oracle. Therefore we are now in the real world.
	
\end{proof}

We have now established that in order to give a (tight) security proof for TLS~1.3 PSK-only and PSK-(EC)DHE, it suffices to prove (tight) security of the protocol on the right-hand side of Figure~\ref{fig:tls-handshake}. 
%\tj{With or without encryption? Swap Sections \ref{sec:ks-indiff} and \ref{sec:modularizing}?}

%%% Local Variables:
%%% mode: latex
%%% TeX-master: "main"
%%% End:


\begin{figure}[tp]
	\begin{minipage}[t]{0.48\textwidth}
		\NewExperiment[Game $\Gm_0$]
		
		\begin{algorithm}{$\Initialize()$}
			\item $b \gets 0$
			\item $\RO \getsr \ESp$
			\item $\state \getsr \emptystring$
		\end{algorithm}
			\ExptSepSpace
			\begin{oracle}{$\Sim(\simIndex,\simInString,\state)$}
				\item if $\simIndex=\Thash$ then return $\ROthash, (\simInString)$
				\item $\rlookuptable, \cachetable \gets \state$
				\item if $\cachetable[\simInString] \neq \bot$
				\item \quad then return $\cachetable[\simInString]$
				\item $\hmackey,\hmactext \gets \simInString$
				\item $\simOut \gets \simulator[\RO](\hmackey, \hmactext,\rlookuptable)$
				\item $\cachetable[\simInString] \gets \simOut$
				\item return $y$
			\end{oracle}
		\ExptSepSpace
		\begin{oracle}{$\Sim[\RO](\hmackey, \hmactext, \rlookuptable)$}
		\item[] \comment{Randomly sample a response}
		\item $\simOut \getsr \bits^{\hashlen}$
		
		%computing ES from PSK
		\item if $\hmactext = 0$ 
		\item \quad $\rlookuptable_{\psk}[\simOut] \gets \hmackey$
		
		%computing MS from DHS
		\item else if $\hmackey = 0$ 
		\item \quad $\rlookuptable_{\dhs}[\simOut] \gets \hmactext$ 
		
		% computing MAC tag from BFK, CFK, or SFK
		\item else if $\rlookuptable_{\bfk/\cfk/\sfk}[\hmackey]\neq \bot$
		
		% computing binder from BFK
			\item \quad $\es \gets \rlookuptable_{\es}[\rlookuptable_{\bk/\chts/\shts}[\hmackey]]$
		\item \quad $\psk \gets \rlookuptable_{\psk}[\es]$
		\item \quad if $\psk \neq \bot$ 
		\item \quad \quad $\simOut \gets \RObinder(\psk, \hmactext)$
		
		% computing finished msg from CFK or SFK
		\item \quad \quad $\hts \gets \rlookuptable_{\bk/\chts/\shts}[\hmackey]$
		\item \quad $(\kdflbl', \hs, \hash_2) \gets \rlookuptable_{\hs/\hashcontext}[\hts]$
		\item \quad $(\des, \dhe) \gets \rlookuptable_{\des/\dhe}[\hs]$
		\item \quad $\psk \gets \rlookuptable_{\psk}[\rlookuptable_{\es/\hs}[ \des]]$
		\item \quad if $\psk \neq \bot$ 
		\item \quad \quad $y \gets \RO_{\kdflbl'[1]}(\psk, \dhe,  \hash_2, \hmactext)[	\lblT(\kdflbl)]$
		
		%computing HS from DES + DHE
		\item else $\rlookuptable_{\des/\dhe}[\simOut] \gets(\hmackey, \hmactext)$
		
		\item if $(\hmactext[0\ldots2] \neq \hashlen)$
		\item[]\quad$\vee\hmactext[2] < 8) \vee(\hmactext[2] > 18)$
		\item[] \quad $\vee(\hmactext[3\ldots9] \neq \mathsf{"tls13 "})$
		\item[]\quad$\vee (\hmactext[|\hmactext|-1] \neq  1)$ 
		\item[] \quad \comment{This query does not match $\HKDFExpnd$ formatting.}
		
		\item \qquad return $\simOut$
		
		\item[] \comment{Parse the $\abstractExpand$ formatting to find the label.} 
		\item $\kdflbllen \gets \hmactext[2]$
		\item $\kdflbl \gets \hmactext[3\ldots (3+\kdflbllen)]$
		\item $\hashcontext \gets \hmactext[(3+\kdflbllen)\ldots|\hmactext|]$
		\item[] \ldots \comment{continued in next column}
		\end{oracle}
		
			
		\end{minipage}
		\begin{minipage}[t]{0.49\textwidth}
			\ExptSepSpace
			\begin{oracle}{$\simulator[\RO](\hmackey, \hmactext, \rlookuptable)$\comment{...continued}}
			%query computes BK from ES
			\item if $\kdflbl = \kdflbl_{\binder}$ and $\hashcontext = \hash(\emptymessage)$
			\item \quad  $\rlookuptable_{\es}[\simOut] \gets \hmackey$
			
			% query computes dES or dHS from ES or HS
			\item else if $\kdflbl = \kdflbl_{\des/\dhs}$ and $\hashcontext = \hash(\emptymessage)$
			\item \quad $\rlookuptable_{\es/\hs}[\simOut] \gets \hmackey$
			
			% Computing CHTS or SHTS from HS
			\item  else if $\kdflbl \in \{\kdflbl_{\chts}, \kdflbl_{\shts}\}$
			\item \quad $\rlookuptable_{\hs/\hashcontext}[\simOut] \gets (\lblT(\kdflbl), \hmackey, \hashcontext)$
			
			% Computing ETS or EEMS from ES
			\item else if $\exists k \in \{\ets, \eems\}$ with  $\kdflbl  = \kdflbl_k$ and $\rlookuptable_{\psk}[\hmackey] \neq \bot$
			\item \quad $\simOut \gets \RO_{k}(\rlookuptable_{\psk}[\hmackey],\hashcontext)$
			
			% computing CATS, SATS, EMS or RMS from MS
			\item else if $\exists k \in \{\cats, \sats, \ems, \rms\}$ with  $\kdflbl  = \kdflbl_k$
			\item \quad $(\des, \dhe) \gets \rlookuptable_{\des/\dhe}[\rlookuptable_{\es/\hs}[ \rlookuptable_{\dhs}[\hmackey])]]$
			\item \quad $\psk \gets \rlookuptable_{\psk}[\rlookuptable_{\es/\hs}[ \des]]$
			\item \quad if $\psk\neq \bot$
			\item \quad \quad  $\simOut \gets \RO_{k}(\psk, \dhe, \hashcontext)$ 
			
			% Computing BFK, CFK, or SFK from BK, CHTS, or SHTS
			\item else if $\kdflbl = \kdflbl_{\fk}$ and $\hashcontext = \emptymessage$ 
			%\item[] \comment{This query must simulate either the binder key (case 1), or the client or server finished key (case 2)}
			\item \quad $\rlookuptable_{\bk/\chts/\shts}[\simOut] \gets \hmackey$
			
			% computing tkchs or tkshs (key or iv) from CHTS or SHTS
			\item else if $\kdflbl \in \{\texttt{"tls13 key"}, \texttt{ "tls13 iv"} \}$
			\item \quad and $\hashcontext = \hash(\emptymessage)$
			\item \quad $(\kdflbl', \hs, \hash_2) \gets \rlookuptable_{\hs/\hashcontext}[\hmackey]$
			\item \quad $(\des, \dhe) \gets \rlookuptable_{\des/\dhe}[\hs]$
			\item \quad $\psk \gets \rlookuptable_{\psk}[\rlookuptable_{\es/\hs}[ \des]]$
			\item \quad if $\psk \neq \bot$ 
			\item \quad \quad $y \gets \RO_{\kdflbl'[0]}(\psk, \dhe, \hash_2)[	\lblT(\kdflbl)]$
			\item []
			
			\item return $\simOut$
		\end{oracle}
			\ExptSepSpace
	\begin{algorithm}{$\PubO(\simIndex,\simInString)$}
		\item $(z,\state) \gets \Sim(\simIndex,\simInString,\state)$
		\item return $z$
	\end{algorithm}
	\ExptSepSpace
	\begin{algorithm}{$\PrivO(\roIndex,\X)$}
		\item return $\RO_{\roIndex}(\X)$
	\end{algorithm}	
			\ExptSepSpace
	\begin{algorithm}{$\Finalize(b')$}
		\item return $b'$
	\end{algorithm}
		\end{minipage}
		\label{fig:gm0-ks-indiff}
		\caption{Indiff game instantiated with simulator $\simulator$, also Game $\Gm_0$ in the proof of Lemma~\ref{th-ks-indiff}.}
	\end{figure}

\begin{figure}[tp]
	\begin{minipage}[t]{0.46\textwidth}
		\NewExperiment[Games $\Gm_1$]
	
			\begin{oracle}{$\Sim(\simIndex,\simInString,\state)$}
			\item if $\simIndex=\Thash$ then return $\ROthash(\simInString)$
			\item $\rlookuptable, \cachetable, \gamechange{L} \gets \state$
			\item if $\cachetable[\simInString] \neq \bot$
			\item \quad then return $\cachetable[\simInString]$
			\item $\hmackey,\hmactext \gets \simInString$
			\item $\simOut \gets \simulator[\RO](\hmackey, \hmactext,\rlookuptable, \gamechange{L})$
			\item $\cachetable[\simInString] \gets \simOut$
			\item \gamechange{$L \gets L \cup \{\simOut, \simInString\}$}
			\item return $y$
		\end{oracle}
	\end{minipage}
	\begin{minipage}[t]{0.49\textwidth}
			\ExptSepSpace
		\begin{oracle}{$\Sim[\RO](\hmackey, \hmactext, \rlookuptable,\gamechange{L})$}
			\item $\simOut \getsr \bits^{\hashlen}$
			\item \gamechange{if $\simOut \in L$ or $\exists t \in L$ such that $\simOut \in t$}
			\item \quad \gamechange{$\bad_C \gets \true$}
			\item[] \ldots
		\end{oracle}
	\ExptSepSpace
	\begin{algorithm}{$\Finalize(b')$}
		\item \gamechange{if $\bad_C$ then return $0$}
		\item return $b'$
	\end{algorithm}
	\end{minipage}
	\label{fig:gm1-ks-indiff}
	\caption{ Game $\Gm_1$ in the proof of Lemma~\ref{th-ks-indiff}. }
\end{figure}

\begin{figure}[tp]
	\begin{minipage}[t]{0.48\textwidth}
		\NewExperiment[Game $\Gm_2$]
		
			\begin{algorithm}{$\PrivO(\roIndex,\X)$}
				\item $Q \gets Q \bigcup \{(\roIndex,\X)\}$
				\item return $\RO_{\roIndex}(\X)$
			\end{algorithm}	
			\ExptSepSpace
			\begin{algorithm}{$\Finalize(b')$}
				\item \gamechange{for $(\roIndex,\X) \in Q$ do }
				\item \quad \gamechange{$z \gets\TLSKDF_{\roIndex}[\PubO](\X)$}
				\item if $\bad_C$ then return $0$
				\item return $b'$
			\end{algorithm}
	
	\end{minipage}
\vline
\hspace{.02\textwidth}
	\begin{minipage}[t]{0.49\textwidth}
		\NewExperiment[Game $\Gm_3$]
	
		\ExptSepSpace
	\begin{algorithm}{$\PrivO(\roIndex,\X)$}
		\item \gamechange{$z \gets\TLSKDF_{\roIndex}[\PubO](\X)$}
		\item return $\RO_{\roIndex}(\X)$
	\end{algorithm}	
	\ExptSepSpace
\begin{algorithm}{$\Finalize(b')$}
	\item if $\bad_C$ then return $0$
	\item return $b'$
\end{algorithm}
\end{minipage}
\label{fig:gm23-ks-indiff}
\caption{ Games $\Gm_2$ and $\Gm_3$ in the proof of Lemma~\ref{th-ks-indiff}.}
\end{figure}

\begin{figure}[tp]
	\begin{minipage}[t]{0.48\textwidth}
		\NewExperiment[Games \fbox{$\Gm_4$}, $\Gm_5$]
		
		\ExptSepSpace
		\begin{algorithm}{$\PrivO(\roIndex,\X)$}
			\item $z \gets\TLSKDF_{\roIndex}[\PubO](\X)$
			\item return \gamechange{$z$}
		\end{algorithm}	
		\ExptSepSpace
		\begin{algorithm}{$\Finalize(b')$}
			\item \fbox{if $\bad_C$ then return $0$}
			\item return $b'$
		\end{algorithm}
	\end{minipage}
	\vline
	\hspace{.02\textwidth}
	\begin{minipage}[t]{0.8\textwidth}
		\NewExperiment[Game $\Gm_6$]

		\begin{oracle}{$\simulator[\RO](\simIndex, \simInString,\rlookuptable)$}
			\item $\simOut \getsr \bits^{\hashlen}$
			\item return $\simOut$
		\end{oracle}
		
	\end{minipage}
	
	\label{fig:gm456-ks-indiff}
	\caption{ Games $\Gm_4$, $\Gm_5$, and $\Gm_6$ in the proof of Lemma~\ref{th-ks-indiff}.}
\end{figure}

\fi

%%% Local Variables:
%%% mode: latex
%%% TeX-master: "main"
%%% End:
