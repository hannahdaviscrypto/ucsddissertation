\section{The TLS~1.3 Pre-shared Key Handshake Protocol}
\label{sec:tls13-psk-protocol}


\paragraph{Overview.}
We consider the pre-shared key mode of TLS~1.3, used in a setting where both client and server already share a common secret, a so-called \emph{pre-shared key} (PSK).
A PSK is a cryptographic key which may either be manually configured, negotiated out-of-band, or (and most commonly) be obtained from a prior and possibly not PSK-based TLS session to enable fast \emph{session resumption}.
The TLS~1.3 PSK handshake comes in two flavors:
PSK-only, where security is established from the pre-shared key alone,
and PSK-(EC)DHE, which includes an (finite-field or \underline{e}lliptic-\underline{c}urve) Diffie--Hellman key exchange for added forward secrecy.
Both PSK handshakes essentially consist of two phases (cf.\ Figure~\ref{fig:tls-handshake}). 

\begin{enumerate}
	\item The client sends a random nonce and a list of offered pre-shared keys to the server, where each key is identified by a (unique) identifier~$\pskid$.%
	\footnote{%
	In this work, we do not consider negotiation of pre-shared keys in situations where client and server share multiple keys, but focus on the case where client and server share only one PSK and the client therefore offers only a single $\pskid$.
	However, we expect that our results extend to the general case as well.
	}
	The server then selects one $\pskid$ from the list, and responds with another random nonce and the selected $\pskid$.
	In PSK-(EC)DHE mode, client and server additionally perform a Diffie--Hellman key exchange, sending group elements along with the nonces and PSK identifiers. 
	%
	In both modes, the client also sends a so-called binder value,
	which applies a \textit{message authentication code} (MAC) to the client's nonce and $\pskid$ (and the Diffie--Hellman share in PSK-(EC)DHE mode)
	and binds the PSK handshake to the (potential) prior handshake in which the used pre-shared key was established (see~\cite{SP:CHSv16,CCS:Krawczyk16} for analysis rationale behind the binder value).

	\item Then client and server derive \emph{unauthenticated} cryptographic keys from the PSK and the established Diffie--Hellman key (the latter only in (EC)DHE mode, of course).
	This includes, for instance, the \emph{client} and \emph{server handshake traffic keys} ($\chtk$ and $\shtk$) used to encrypt the subsequent handshake messages, as well as \emph{finished keys} ($\cfk$ and $\sfk$) used to compute and exchange \emph{finished messages}.
	The finished messages are MAC tags over all previous messages, ensuring that client and server have received all previous messages exactly as they were sent.

	After verifying the finished messages, client and server ``accept'' \emph{authenticated} cryptographic keys, including the \emph{client} and \emph{server application traffic secret} ($\cats$ and $\sats$), the \emph{exporter master secret} ($\ems$), and the \emph{resumption master secret} ($\rms$) for future session resumptions.
\end{enumerate}




\begin{figure}[tp]
	\centering
	
	\scalebox{0.7}{%
\framebox{%
\begin{minipage}[t]{10.5cm}
\begin{tikzpicture}

	% Set the X coordinates of the client, server, and arrows
	\edef\ClientX{0}
	\edef\ArrowLeft{0}

		\edef\ArrowRight{9}
		\edef\ServerX{9}
	% Set the starting Y coordinate
	\edef\Y{0}

	% Draw header boxes
	\node [rectangle,draw,inner sep=5pt,right] at (\ClientX,\Y) {\textbf{Client}};
	\node [rectangle,draw,inner sep=5pt,left] at (\ServerX,\Y) {\textbf{Server}};

	\NextLine[1.5]
	
	\ClientAction{$\clientNonce \sample \bits^{\nl}$}
	\NextLine
	\ClientAction{$\clientExponent \sample \Z_p$, $\clientKeyShare \assign g^\clientExponent$}
	\NextLine
	\ClientAction{\TLSmsg{$\ClientHello$}: $\clientNonce$}
	\NextLine
	\ClientAction{\PSKECDHEonly{+~$\ClientKeyShare$: $\clientKeyShare$}}
	\NextLine
	
	\SharedAction{$\es \assign \abstractExtract(0, \psk)$}
	\NextLine
	\SharedAction{$\des \assign \abstractExpand(\es, \labelDerived \concat \Hash(\emptymessage))$}
	\NextLine
	\SharedAction{$\bk \assign \abstractExpand(\es, \labelExtBinder \concat \abstractHash(\emptymessage))$}
	\NextLine
	\SharedAction{$\bfk \assign \abstractExpand(\bk, \labelFinished)$}
	\NextLine
	
	\ClientAction{$\binder \assign \abstractMAC(\bfk, \abstractHash(\CH^{-}))$}
	\NextLine
	\ClientAction{\TLSmsg{+~$\ClientPreSharedKey$}: $\pskid, \binder$}
	
	%%%%%%%%%%%%%%%%%%%%%%%%%%%%%%%%%%%%%%%%%%%%%%%%%%%%%%%%%%%%%%%%%%%%%%%%%%%%%%%%%%%%%%%%%%%%%%%%%%%%%%%
	\ClientToServer{}{}
	\NextLine
	%%%%%%%%%%%%%%%%%%%%%%%%%%%%%%%%%%%%%%%%%%%%%%%%%%%%%%%%%%%%%%%%%%%%%%%%%%%%%%%%%%%%%%%%%%%%%%%%%%%%%%%
	
	\ServerAction{\textbf{abort} if $\binder \neq \abstractMAC(\bfk, \abstractHash(\CH^{-}))$}
	\NextLine

	\AcceptStage{1}{$\ets \assign \abstractExpand(\es, \labelETS \concat \abstractHash(\CH))$}
	\NextLine[1.5]
	\AcceptStage{2}{$\eems \assign \abstractExpand(\es, \labelEEMS \concat \abstractHash(\CH))$}
	\NextLine[1.5]
	
	\ServerAction{$\serverNonce \sample \bits^{\nl}$}
	\NextLine
	\ServerAction{$\serverExponent \sample \Z_p$, $\serverKeyShare \assign g^\serverExponent$}
	\NextLine
	\ServerAction{\TLSmsg{$\ServerHello$}: $\serverNonce$}
	\NextLine
	\ServerAction{\PSKECDHEonly{+~$\ServerKeyShare$: $\serverKeyShare$}}
	\NextLine
	\ServerAction{\TLSmsg{+~$\ServerPreSharedKey$}: $\pskid$}
	
	%%%%%%%%%%%%%%%%%%%%%%%%%%%%%%%%%%%%%%%%%%%%%%%%%%%%%%%%%%%%%%%%%%%%%%%%%%%%%%%%%%%%%%%%%%%%%%%%%%%%%%%
	\ServerToClient{}{}
	\NextLine
	%%%%%%%%%%%%%%%%%%%%%%%%%%%%%%%%%%%%%%%%%%%%%%%%%%%%%%%%%%%%%%%%%%%%%%%%%%%%%%%%%%%%%%%%%%%%%%%%%%%%%%%
	
	\ClientAction{\PSKECDHEonly{$\dhe \assign Y^x$}}
	\ServerAction{\PSKECDHEonly{$\dhe \assign X^y$}}
	\SharedAction{\PSKonly{$\dhe \assign 0$}}
	\NextLine
	
	\SharedAction{$\hs \assign \abstractExtract(\des, \dhe)$}
	\NextLine
	\SharedAction{$\chts \assign \abstractExpand(\hs, \labelClientHTS \concat \abstractHash(\CH \concat \SH))$}
	\NextLine
	\SharedAction{$\shts \assign \abstractExpand(\hs, \labelServerHTS \concat \abstractHash(\CH \concat \SH))$}
	\NextLine
	\SharedAction{$\dhs \assign \abstractExpand(\hs, \labelDerived \concat \abstractHash(\emptymessage))$}
	\NextLine
	\AcceptStage{3}{$\chtk \assign \DeriveTrafficKeys(\chts)$}
	\NextLine[1.5]
	\AcceptStage{4}{$\shtk \assign \DeriveTrafficKeys(\shts)$}
	\NextLine[1.5]
	
	\ServerAction{\TLSmsg{$\{ \EncryptedExtensions \}$}}
	\NextLine
	
	\SharedAction{$\sfk \assign \abstractExpand(\shts, \labelFinished)$}
	\NextLine
	
	\ServerAction{$\sfin \assign \abstractMAC(\sfk, \abstractHash(\CH \concat \dotsb \concat \EE ))$}
	\NextLine
	\ServerAction{\TLSmsg{$\{ \ServerFinished \}$}: $\sfin$}
	
	%%%%%%%%%%%%%%%%%%%%%%%%%%%%%%%%%%%%%%%%%%%%%%%%%%%%%%%%%%%%%%%%%%%%%%%%%%%%%%%%%%%%%%%%%%%%%%%%%%%%%%%
	\ServerToClient{}{}
	\NextLine
	%%%%%%%%%%%%%%%%%%%%%%%%%%%%%%%%%%%%%%%%%%%%%%%%%%%%%%%%%%%%%%%%%%%%%%%%%%%%%%%%%%%%%%%%%%%%%%%%%%%%%%%
	
	
	\ClientAction{\textbf{abort} if $\sfin \neq \HMAC(\sfk, \abstractHash(\CH \concat \dotsb \concat \EE))$}
	\NextLine
	\SharedAction{$\ms \assign \abstractExtract(\dhs, 0)$}
	\NextLine
	\AcceptStage{5}{$\cats \assign \abstractExpand(\ms, \labelClientATS \concat \abstractHash(\CH \concat \dotsb \concat \SF))$}
	\NextLine
	\AcceptStage{6}{$\sats \assign \abstractExpand(\ms, \labelServerATS \concat \abstractHash(\CH \concat \dotsb \concat \SF))$}
	\NextLine
	\AcceptStage{7}{$\ems \assign \abstractExpand(\ms, \labelEMS \concat \abstractHash(\CH \concat \dotsb \concat \SF))$}
	\NextLine[1.5]
	\SharedAction{$\cfk \assign \abstractExpand(\chts, \labelFinished)$}
	\NextLine
	
	\ClientAction{$\cfin \assign \abstractMAC(\cfk, \abstractHash(\CH \concat \dotsb \concat \SF))$}
	\NextLine
	\ClientAction{\TLSmsg{$\{ \ClientFinished \}$}: $\cfin$}
	
	%%%%%%%%%%%%%%%%%%%%%%%%%%%%%%%%%%%%%%%%%%%%%%%%%%%%%%%%%%%%%%%%%%%%%%%%%%%%%%%%%%%%%%%%%%%%%%%%%%%%%%%
	\ClientToServer{}{}
	\NextLine
	%%%%%%%%%%%%%%%%%%%%%%%%%%%%%%%%%%%%%%%%%%%%%%%%%%%%%%%%%%%%%%%%%%%%%%%%%%%%%%%%%%%%%%%%%%%%%%%%%%%%%%%
	
	
	\ServerAction{\textbf{abort} if $\cfin \neq \abstractMAC(\cfk, \abstractHash(\CH \concat \dotsb \concat \SF))$}
	\NextLine
	\AcceptStage{8}{$\rms \assign \abstractExpand(\ms, \labelRMS \concat \abstractHash(\CH \concat \dotsb \concat \CF))$}
\end{tikzpicture}
\end{minipage}
}
%
\framebox{%
\begin{minipage}[t]{11.5cm}
\begin{tikzpicture}

	% Set the X coordinates of the client, server, and arrows
	\edef\ClientX{0}
	\edef\ArrowLeft{0}
		\edef\ArrowRight{10}
		\edef\ServerX{10}

	% Set the starting Y coordinate
	\edef\Y{0}

	% Draw header boxes
	\node [rectangle,draw,inner sep=5pt,right] at (\ClientX,\Y) {\textbf{Client}};
	\node [rectangle,draw,inner sep=5pt,left] at (\ServerX,\Y) {\textbf{Server}};

	\NextLine[1.5]
	
	\ClientAction{$\clientNonce \sample \bits^{\nl}$}
	\NextLine
	\ClientAction{$\clientExponent \sample \Z_p$, $\clientKeyShare \assign g^\clientExponent$}
	\NextLine
	\ClientAction{\TLSmsg{$\ClientHello$}: $\clientNonce$}
	\NextLine
	\ClientAction{\PSKECDHEonly{+~$\ClientKeyShare$: $\clientKeyShare$}}
	\NextLine
	
% 	\SharedAction{\old{$\es \assign \HKDFExtr(0, \psk)$}}
% 	\NextLine
% 	\SharedAction{\old{$\des \assign \HKDFExpnd(\es, \labelDerived \concat \Thash(\emptymessage))$}}
% 	\NextLine
% 	\SharedAction{\old{$\bk \assign \HKDFExpnd(\es, \labelExtBinder \concat \Thash(\emptymessage))$}}
% 	\NextLine
% 	\SharedAction{\old{\TODO{Missing before!} $\bfk \assign \HKDFExpnd(\bk, \labelFinished)$}}
% 	\NextLine
	\NextLine[4]
	
	\ClientAction{$\binder \assign \TLSKDF_{\binder}(\psk, \abstractHash(\CH^{-}))$}
	\NextLine
	\ClientAction{\TLSmsg{+~$\ClientPreSharedKey$}: $\pskid, \binder$}
	
	%%%%%%%%%%%%%%%%%%%%%%%%%%%%%%%%%%%%%%%%%%%%%%%%%%%%%%%%%%%%%%%%%%%%%%%%%%%%%%%%%%%%%%%%%%%%%%%%%%%%%%%
	\ClientToServer{}{}
	\NextLine
	%%%%%%%%%%%%%%%%%%%%%%%%%%%%%%%%%%%%%%%%%%%%%%%%%%%%%%%%%%%%%%%%%%%%%%%%%%%%%%%%%%%%%%%%%%%%%%%%%%%%%%%
	
	\ServerAction{\textbf{abort} if $\binder \neq \TLSKDF_{\binder}(\psk, \abstractHash(\CH^{-}))$}
	\NextLine

	\AcceptStage{1}{$\ets \assign \TLSKDF_{\ets}(\psk, \abstractHash(\CH))$}
	\NextLine[1.5]
	\AcceptStage{2}{$\eems \assign \TLSKDF_{\eems}(\psk, \abstractHash(\CH))$}
	\NextLine[1.5]
	
	\ServerAction{$\serverNonce \sample \bits^{\nl}$}
	\NextLine
	\ServerAction{$\serverExponent \sample \Z_p$, $\serverKeyShare \assign g^\serverExponent$}
	\NextLine
	\ServerAction{\TLSmsg{$\ServerHello$}: $\serverNonce$}
	\NextLine
	\ServerAction{\PSKECDHEonly{+~$\ServerKeyShare$: $\serverKeyShare$}}
	\NextLine
	\ServerAction{\TLSmsg{+~$\ServerPreSharedKey$}: $\pskid$}
	
	%%%%%%%%%%%%%%%%%%%%%%%%%%%%%%%%%%%%%%%%%%%%%%%%%%%%%%%%%%%%%%%%%%%%%%%%%%%%%%%%%%%%%%%%%%%%%%%%%%%%%%%
	\ServerToClient{}{}
	\NextLine
	%%%%%%%%%%%%%%%%%%%%%%%%%%%%%%%%%%%%%%%%%%%%%%%%%%%%%%%%%%%%%%%%%%%%%%%%%%%%%%%%%%%%%%%%%%%%%%%%%%%%%%%
	
	\ClientAction{\PSKECDHEonly{$\dhe \assign Y^x$}}
	\ServerAction{\PSKECDHEonly{$\dhe \assign X^y$}}
	\SharedAction{\PSKonly{$\dhe \assign 0$}}
	\NextLine
	
% 	\SharedAction{\old{$\hs \assign \HKDFExtr(\des, \dhe)$}}
% 	\NextLine
% 	\SharedAction{\old{$\chts \assign \HKDFExpnd(\hs, \labelClientHTS \concat \Thash(\CH \concat \SH))$}}
% 	\NextLine
% 	\SharedAction{\old{$\shts \assign \HKDFExpnd(\hs, \labelServerHTS \concat \Thash(\CH \concat \SH))$}}
% 	\NextLine
% 	\SharedAction{\old{$\dhs \assign \HKDFExpnd(\hs, \labelDerived \concat \Thash(\emptymessage))$}}
% 	\NextLine
	\NextLine[4]
	
	\AcceptStage{3}{$\chtk \assign \TLSKDF_{\chtk}(\psk, \dhe, \abstractHash(\CH \concat \SH))$}
	\NextLine[1.5]
	\AcceptStage{4}{$\shtk \assign \TLSKDF_{\shtk}(\psk, \dhe, \abstractHash(\CH \concat \SH))$}
	\NextLine[1.5]
	
	\ServerAction{\TLSmsg{$\{ \EncryptedExtensions \}$}}
	\NextLine
	
% 	\SharedAction{\old{$\sfk \assign \HKDFExpnd(\shts, \labelFinished)$}}
% 	\NextLine
	\NextLine[1]
	
	\ServerAction{$\sfin \assign \TLSKDF_{\sfin}(\psk, \dhe, \abstractHash(\CH \concat \SH), \abstractHash(\CH \concat \dotsb \concat \EE ))$}
	\NextLine
	\ServerAction{\TLSmsg{$\{ \ServerFinished \}$}: $\sfin$}
	
	%%%%%%%%%%%%%%%%%%%%%%%%%%%%%%%%%%%%%%%%%%%%%%%%%%%%%%%%%%%%%%%%%%%%%%%%%%%%%%%%%%%%%%%%%%%%%%%%%%%%%%%
	\ServerToClient{}{}
	\NextLine
	%%%%%%%%%%%%%%%%%%%%%%%%%%%%%%%%%%%%%%%%%%%%%%%%%%%%%%%%%%%%%%%%%%%%%%%%%%%%%%%%%%%%%%%%%%%%%%%%%%%%%%%
	
	
	\ClientAction{\textbf{abort} if \scriptsize$\sfin \neq \TLSKDF_{\sfin}(\psk, \dhe, \abstractHash(\CH \concat \SH), \abstractHash(\CH \concat \dotsb \concat \EE ))$}
	\NextLine
	
% 	\SharedAction{\old{$\ms \assign \HKDFExtr(\dhs, 0)$}}
% 	\NextLine
	\NextLine[1]
	
	\AcceptStage{5}{$\cats \assign \TLSKDF_{\cats}(\psk, \dhe, \abstractHash(\CH \concat \dotsb \concat \SF))$}
	\NextLine
	\AcceptStage{6}{$\sats \assign \TLSKDF_{\sats}(\psk, \dhe, \abstractHash(\CH \concat \dotsb \concat \SF))$}
	\NextLine
	\AcceptStage{7}{$\ems \assign \TLSKDF_{\ems}(\psk, \dhe, \abstractHash(\CH \concat \dotsb \concat \SF))$}
	\NextLine[1.5]
	
% 	\SharedAction{\old{$\cfk \assign \HKDFExpnd(\chts, \labelFinished)$}}
% 	\NextLine
	\NextLine[1]
	
	\ClientAction{$\cfin \assign \TLSKDF_{\cfin}(\psk, \dhe, \abstractHash(\CH \concat \SH), \abstractHash(\CH \concat \dotsb \concat \SF))$}
	\NextLine
	\ClientAction{\TLSmsg{$\{ \ClientFinished \}$}: $\cfin$}
	
	%%%%%%%%%%%%%%%%%%%%%%%%%%%%%%%%%%%%%%%%%%%%%%%%%%%%%%%%%%%%%%%%%%%%%%%%%%%%%%%%%%%%%%%%%%%%%%%%%%%%%%%
	\ClientToServer{}{}
	\NextLine
	%%%%%%%%%%%%%%%%%%%%%%%%%%%%%%%%%%%%%%%%%%%%%%%%%%%%%%%%%%%%%%%%%%%%%%%%%%%%%%%%%%%%%%%%%%%%%%%%%%%%%%%
	
	
	\ServerAction{\textbf{abort} if \scriptsize$\cfin \neq \TLSKDF_{\cfin}(\psk, \dhe, \abstractHash(\CH \concat \SH), \abstractHash(\CH \concat \dotsb \concat \SF ))$}
	\NextLine
	\AcceptStage{8}{$\rms \assign \TLSKDF_{\rms}(\psk, \dhe, \abstractHash(\CH \concat \dotsb \concat \CF))$}
\end{tikzpicture}
\end{minipage}
}
}

	
	%
	% Legend
	%
	\begin{minipage}{0.95\textwidth}
	\vspace{0.25cm}
	\scriptsize
	\textbf{Legend} \\
	\begin{tabular}{ll}
		\TLSmsg{$\mathtt{MSG}$}:~$Y$	& message $\mathtt{MSG}$ sent, containing $Y$ \\
		\TLSmsg{+~$\mathtt{MSG}$}	& extension sent within previous message \\
		\TLSmsg{$\{\mathtt{MSG}\}$}	& $\mathtt{MSG}$ sent AEAD-encrypted with~$\chtk$/$\shtk$ \\
		\PSKECDHEonly{\dots}		& present only in PSK-(EC)DHE \\
		\PSKonly{\dots}			& present only in PSK \\
	\end{tabular}
	\begin{tabular}{ll}
		$\CH^{-}$			& partial $\ClientHello$ up to (incl.) $\pskid$ \\
		$\shortLabelCmd{x}$		& label value, distinct for distinct~$x$ \\
		&\\
		&\\
		&\\
	\end{tabular}
	\begin{tabular}{l}
		$\DeriveTrafficKeys(\mathit{HTS}) := \abstractExpand(\mathit{HTS}, \labelk \concat \Thash(\emptymessage), \hashlen) \concat \abstractExpand(\mathit{HTS}, \labeliv \concat \Thash(\emptymessage), \ivlen)$\\
		\quad(traffic key computation, deriving a $\hashlen$-bit key and a $\ivlen$-bit IV)
	\end{tabular}
	\end{minipage}
	
	\caption{%
		TLS~1.3 PSK and PSK-(EC)DHE handshake modes with (optional) 0-RTT keys (stages~1 and~2),
		with detailed key schedule (left) and our representation of the key schedule through functions~$\TLSKDF_{x}$ (right), explained in the text.
		Centered computations are executed by both client and server with their respective messages received, and possibly at different points in time.
		Dotted lines indicate the derivation of session (stage) keys together with their stage number.
		The labels $\shortLabelCmd{x}$ are distinct for distinct index~$x$, see
				Table~\ref{tab:labels}
		for their definition.
	}
	\label{fig:tls-handshake}
\end{figure}

	\begin{table}[htp]
	\centering
	\setlength{\tabcolsep}{4pt}
	\renewcommand{\arraystretch}{1.1}
	\begin{tabular}{@{}ll@{\qquad}ll@{}}
		\toprule
		\textbf{Value} & \textbf{Label}                                                  & \textbf{Value}                   & \textbf{Label}                        \\ \midrule
		$\des$         & $\labelDerived = \labelDerivedText$                             & $\chtk$ &    $\labelk = \labelkText$ \& $\labeliv= \labelivText$                         \\
		$\bk$          & $\labelExtBinder = \labelExtBinderText$ / $\labelResBinderText$ & $\shtk$                          & $\labelk = \labelkText$ \& $\labeliv= \labelivText$                                    \\
		$\bfk$         & $\labelFinished = \labelFinishedText$                           & $\sfk$                           & $\labelFinished = \labelFinishedText$ \\
		$\ets$         & $\labelETS = \labelETSText$                                     & $\cats$                          & $\labelClientATS = \labelClientATSText$         \\
		$\eems$        & $\labelEEMS = \labelEEMSText$                                   & $\sats$                          & $\labelServerATS = \labelServerATSText$         \\
		$\chts$        & $\labelClientHTS = \labelClientHTSText$                                   & $\ems$                           & $\labelEMS = \labelEMSText$           \\
		$\shts$        & $\labelServerHTS = \labelServerHTSText$                                   & $\cfk$                           & $\labelFinished = \labelFinishedText$ \\
		$\dhs$         & $\labelDerived = \labelDerivedText$                             & $\rms$                           & $\labelRMS = \labelRMSText$           \\ \bottomrule
	\end{tabular}
	\medskip
	\caption{%
		Definitions of the short labels used in \autoref{fig:tls-handshake}.
		We simplify the labeling of $\abstractExpand$ in our presentation.
		In the specification each $\abstractExpand$ is not only labeled by $\labelsym \concat H$ for some label $\labelsym$ and some hash $H$,
		but it is prefixed by the output length of the respecitive $\abstractExpand$ call and the constant label ``$\labelfont{tls13 }$''.
		As the output length for all of the above calls is equal (namely, the output length $\hashlen$ of $\abstractHash$), we leave this constant prefix out to reduce complexity.
		}
	\label{tab:labels}
\end{table}



\paragraph{Detailed specification.}
For our proofs we will need fully-specified descriptions for each of the TLS~1.3 PSK and PSK-(EC)DHE handshake protocols. 
Pseudocode for these protocols can be found in Figure~\ref{fig:tls-handshake}, where we let $(\G, p, g)$ be a cyclic group of prime order $p$ such that $\G = \langle g \rangle$.

The two descriptions on the left and right in Figure~\ref{fig:tls-handshake} show the same protocol, but they use different abstractions to highlight how we capture the complex way TLS~1.3 calls its hash function.
This one hash function is used in some places to condense transcripts, in others to help derive session keys, and in still others as part of a message authentication code. 
We call this function $\Hash$, and let its output length be $\hashlen$ bits so that we have $\abstractHash \colon \bits^* \to \bits^{\hashlen}$. 
Depending on the choice of ciphersuite, TLS~1.3 instantiates $\Hash$ with either $\SHA{256}$ or $\SHA{384}$ \cite{NIST:FIPS-180-4}. In our security analysis, we will model $\Hash$ as a random oracle.

On the left-hand side of Figure~\ref{fig:tls-handshake}, we distinguish four named subroutines of TLS~1.3 which use $\Hash$ for different purposes:
\begin{itemize}
%	\item Since we now only directly call a hash function on transcripts, we refer to this as the ``transcript hash''.

	\item A message authentication code $\abstractMAC  \colon \bits^{\hashlen} \times \bits^* \to \bits^{\hashlen}$, which  calls $\Hash$ via the $\HMAC$ function $\abstractMAC(K, M) := \HMAC[\Hash](K, M)$ where
\[
\HMAC[\Hash](K, M) := \Hash( (K \concat 0^{\blocklen - \hashlen}) \oplus \opad) \concat \Hash( (K \concat 0^{\blocklen - \hashlen} \oplus \ipad) \concat M )) 
\]
Here $\opad$ and $\ipad$ are $\blocklen$-bit strings,
where each byte of $\opad$ and $\ipad$ is set to the hexadecimal value \texttt{0x5c}, resp.\ \texttt{0x36}.
We have $\blocklen=512$ when $\SHA{256}$ is used and $\blocklen=512$ for $\SHA{384}$. When modeling $\SHA{256}$ resp. $\SHA{384}$ as a random oracle, we keep the corresponding value of $\blocklen$.
%HMAC assumes that $\Hash$ has a Merkle--Damg\r{a}rd structure with block length $\blocklen$;
%where $\SHA{256}$ has $\blocklen=512$ bits and $\SHA{384}$ has $\blocklen = 1024$ bits. \TODO{How do we defie HMAC if H is a RO?}
%Then functions $\abstractMAC$, $\abstractExtract$, and $\abstractExpand$, and $\abstractHash$ are defined as follows:


% 	\item $\replace{\abstractExtract}{\abstractExtract} \colon \bits^{\hashlen} \times \bits^* \to \bits^{\hashlen}$, a subroutine for \emph{extracting} key material in the key schedule. 
% 	\item $\replace{\abstractExpand}{\abstractExpand}  \colon \bits^{\hashlen} \times \bits^* \to \bits^{\hashlen}$, a subroutine for \emph{expanding} key material in the key schedule.
	\item $\abstractExtract, \abstractExpand \colon \bits^{\hashlen} \times \bits^* \to \bits^{\hashlen}$, two subroutines for \emph{extracting} and \emph{expanding} key material in the key schedule, following the HKDF key derivation paradigm of Krawczyk~\cite{C:Krawczyk10,rfc5869}. These functions are defined
\begin{itemize}
	\item $\abstractExtract(K, M) := \HKDFExtr(K,M) = \abstractMAC(K,M)$.
	
	\item $\abstractExpand(K, M) := \HKDFExpnd(K, M) = \abstractMAC(K, M \concat \texttt{0x01})$.%
	\footnote{$\HKDFExpnd$~\cite{rfc5869} is defined for any output length (given as third parameter).
	In TLS~1.3, $\abstractExpand$ always derives at most $\hashlen$ bits, which can be trimmed from a $\hashlen$-bit output; we hence in most places omit the output length parameter.}
\end{itemize} 
\end{itemize}
%We can establish the independence of these subroutines more easily by viewing them as named parameters within the TLS protocol.
Despite the new naming conventions, this abstraction closely mimics the TLS~1.3 standard:
$\abstractMAC$, $\abstractExtract$, and $\abstractExpand$ can be read as more generic ways of referring to the $\HMAC$, $\HKDFExtr$, and $\HKDFExpnd$ algorithms~\cite{rfc2104,rfc5869}. 


% \TODO{Define $\abstractMAC$ in terms of $\abstractHash$. (Cf. Theorem~\ref{thm:full-ks-indiff})}

% \begin{itemize}
% 	\item 
	

	
% 	\item $\abstractHash(M) := \Hash(M)$.
% \end{itemize}

The right-hand side of Figure~\ref{fig:tls-handshake} separates the key derivation functions for each first-class key as well as the binder and finished MAC values derived.
This way of modeling TLS~1.3 makes it easier to establish key independence for the many keys computed in the key schedule, as we will see in Section~\ref{sec:ks-indiff}.
We introduce $11$ functions $\TLSKDF_{\binder}$, $\TLSKDF_{\ets}$, $\TLSKDF_{\eems}$, $\TLSKDF_{\chtk}$, $\TLSKDF_{\cfin}$, $\TLSKDF_{\shtk}$, $\TLSKDF_{\sfin}$, $\TLSKDF_{\cats}$, $\TLSKDF_{\sats}$, $\TLSKDF_{\ems}$, and $\TLSKDF_{\rms}$ (indexed by the value they derive) and use them to abstract away many intermediate computations.
Note that we are not changing the protocol, though:
we define each $\TLSKDF$ function to capture the same steps it replaces.

Take as an example $\TLSKDF_{\sfin}$, the function used to derive the MAC in the $\ServerFinished$ message.
In the prior abstraction, a session would first use the key schedule to derive a finished key~$\sfk$ from the hashed transcript and the secrets~$\psk$ and~$\dhe$. 
It would then call $\abstractMAC$, keyed with $\sfk$, to generate the $\ServerFinished$ message authentication code on the hashed transcript and encrypted extensions.
Accordingly, we define $\TLSKDF_{\sfin} \colon \bits^{\hashlen} \times \G \times \bits^{\hashlen}\times \bits^{\hashlen} \to \bits^{\hashlen}$ as in Figure~\ref{fig:TLSKDF-sfin}.
In the protocol, $\TLSKDF_{\sfin}$ takes inputs the pre-shared key~$\psk$ and Diffie--Hellman secret~$\dhe$ and hash digests $\digest_1 = \Thash(\CH \concat \SH)$ and $\digest_2 = \Thash(\CH \concat \dotsb \concat \EE)$, and it outputs a $\MAC$ tag for the $\ServerFinished$ message.
%
The remaining key derivation functions are defined the same way; we give their signatures 
 
		below for completeness.
	{
\allowdisplaybreaks
\begin{align*}
1.~ &\TLSKDF_{\binder}[\ROhmac]	&&\colon \bits^{\hashlen} \times \bits^{\hashlen} \to \bits^{\hashlen} \\
2.~ &\TLSKDF_{\ets}[\ROhmac]	&&\colon \bits^{\hashlen} \times \bits^{\hashlen} \to \bits^{\hashlen} \\
3.~ &\TLSKDF_{\eems}[\ROhmac]	&&\colon \bits^{\hashlen} \times \bits^{\hashlen}  \to \bits^{\hashlen}\\
4.~ &\TLSKDF_{\chtk}[\ROhmac]	&&\colon \bits^{\hashlen}\times \G \times \bits^{\hashlen} \to \bits^{\hashlen + \ivlen} \\
5.~ &\TLSKDF_{\cfin}[\ROhmac]		&&\colon \bits^{\hashlen} \times \G \times \bits^{\hashlen} \times \bits^{\hashlen} \to \bits^{\hashlen} \\
6.~ &\TLSKDF_{\shtk}[\ROhmac]	&&\colon \bits^{\hashlen} \times \G \times \bits^{\hashlen} \to \bits^{\hashlen + \ivlen} \\
7.~ &\TLSKDF_{\sfin}[\ROhmac]		&&\colon \bits^{\hashlen} \times \G \times \bits^{\hashlen} \times \bits^{\hashlen} \to \bits^{\hashlen} \\
8.~ &\TLSKDF_{\cats}[\ROhmac]	&&\colon \bits^{\hashlen}\times \G \times \bits^{\hashlen} \to \bits^{\hashlen} \\
9.~ &\TLSKDF_{\sats}[\ROhmac] &&\colon \bits^{\hashlen} \times \G \times \bits^{\hashlen} \to \bits^{\hashlen} \\
10.~ &\TLSKDF_{\ems}[\ROhmac]	&&\colon \bits^{\hashlen} \times \G \times \bits^{\hashlen} \to \bits^{\hashlen} \\
11.~ &\TLSKDF_{\rms}[\ROhmac]	&&\colon \bits^{\hashlen} \times \G \times \bits^{\hashlen} \to \bits^{\hashlen}
\end{align*}
}


\begin{figure}[t]
	\centering
	\begin{minipage}{5.5cm}
	\begin{algorithm}{$\TLSKDF_{\sfin}(\psk, \dhe, \digest_1, \digest_2)$}
		\item $\es \gets \abstractExtract (0, \psk)$
		\item $\des \gets \abstractExpand(\es, \labelDerived \concat \Thash(\emptymessage))$
		\item $\hs \gets \abstractExtract(\des, \dhe)$
% 		\item $\shts \gets \abstractExpand(\hs, \labelServerHTS \concat \digest_1)$
% 		\item $\sfk \gets \abstractExpand(\shts, \labelFinished)$
% 		\item $\sfin \gets \abstractMAC(\sfk, \digest_2)$
% 		\item return $\sfin$
	\end{algorithm}
	\end{minipage}
	%
	\begin{minipage}{5.5cm}
	\begin{code}[start=4]
% 		\item $\es \gets \abstractExtract (0, \psk)$
% 		\item $\des \gets \abstractExpand(\es, \labelDerived \concat \Thash(\emptymessage))$
% 		\item $\hs \gets \abstractExtract(\des, \dhe)$
		\item $\shts \gets \abstractExpand(\hs, \labelServerHTS \concat \digest_1)$
		\item $\sfk \gets \abstractExpand(\shts, \labelFinished)$
		\item $\sfin \gets \abstractMAC(\sfk, \digest_2)$
		\item return $\sfin$
	\end{code}
	\end{minipage}
	
	\caption{Definition of $\TLSKDF_{\sfin}$, deriving the $\ServerFinished$ MAC.}
	\label{fig:TLSKDF-sfin}
\end{figure}


% 	\end{minipage}
% \caption{Pseudocode for function $\TLSKDF_{\sfin}$ to derive the $\ServerFinished$ message in the TLS~1.3 PSK and PSK-(EC)DHE (right) handshakes. (In the PSK-only handshake, $\dhe$ will always be the string $0^{\hashlen}$).}
% \label{fig:tlskdf-example}
% \end{figure}

Note that the definition of the 11 functions induces a lot of redundancy as we derive every value independently and therefore compute intermediate values (e.g., $\es$, $\des$, and $\hs$) multiple times over the execution of the handshake.
However, this is only conceptual.
Since the computations of these intermediate values are deterministic, the intermediate values will be the same for the same inputs and could be cached.




% \begin{figure}
% 	\centering
% 	
% 	\input{old-fig-tls-handshake}
% 	
% 	\caption{The TLS~1.3 PSK handshake without 0-RTT. Every TLS handshake message is denoted as ``$\msgfont{MSG} \colon C$'', where $C$ denotes the message's content. Similarly, an extension is denoted by ``$\msgfont{+ MSG} \colon C$''. Further, we denote by ``$\{ \msgfont{MSG} \} \colon C$'' messages containing $C$ and being AEAD-encrypted under the handshake traffic keys $\chtk$/$\shtk$. Messages and computations that are written in ``$[\dotsc]$'' only appear in the PSK-(EC)DHE handshake. Centered computations are executed by both client and server with their respective messages received, and possibly at different points in time. The labels are defined in \autoref{fig:labels}.
% 	%
% 	\fg{I'd suggest to introduce some color coding (akin to \cite[Fig.~1]{JC:DFGS21}) for messages, TLSKDF/RO-derived keys, etc.;
% 	this could match the to-be-pictured key schedule. (I can take care of both, if we want it.)}}
% 	\TODO{@fg: adapt, add stage numbers, ...}
% 	\TODO{@fg: left side: current figure, right side: figure with 11 RO-KDF + 1 RO $\Thash$, similar to \cite[Fig.~1]{JC:DFGS21}}
% 	\label{fig:tls13-psk}
% \end{figure}

%%% Local Variables:
%%% mode: latex
%%% TeX-master: "main"
%%% End:
