The pre-shared key (PSK) handshake modes of TLS~1.3 allow for the performant, low-latency resumption of previous connections and are widely used on the Web and by resource-constrained devices, e.g., in the Internet of Things.
Taking advantage of these performance benefits with optimal and theoretically-sound parameters requires tight security proofs.
We give the first tight security proofs for the TLS~1.3 PSK handshake modes.

Our main technical contribution is to address a gap in prior tight security proofs of TLS~1.3 which modeled either the entire key schedule or components thereof as independent random oracles to enable tight proof techniques.
These approaches ignore existing interdependencies in TLS~1.3's key schedule, arising from the fact that the same cryptographic hash function is used in several components of the key schedule and the handshake more generally.
We overcome this gap by proposing a new abstraction for the key schedule and carefully arguing its soundness via the indifferentiability framework.
Interestingly, we observe that for one specific configuration, PSK-only mode with hash function SHA-384, it seems difficult to argue indifferentiability due to a lack of domain separation between the various hash function usages.
We view this as an interesting insight for the design of protocols, such as future TLS versions.

For all other configurations however, our proofs significantly tighten the security of the TLS~1.3 PSK modes, confirming standardized parameters (for which prior bounds provided subpar or even void guarantees) and enabling a theoretically-sound deployment.


%We consider a transform, called Derive-then-Derandomize, that hardens a given
signature scheme against randomness failure and implementation error.
We prove that it works. We then give a general lemma showing indifferentiability of $\ShrinkMD$, a class of constructions that apply a shrinking output transform to an MD-style hash function.
Armed with these tools, we give new
proofs for the widely standardized
and used $\EdDSA$ signature scheme, improving prior work in two ways:
(1) we give proofs for the case that the hash function is an MD-style one,
reflecting the use of SHA512 in the NIST standard, and (2)
we improve the tightness of the reduction so that one has guarantees for group
sizes in actual use. 


% Notes:
% \begin{itemize}
% \item TLS~1.3 widely used for session resumption, also for deployments without PKI or public key techniques at all
% \item Are there any security proofs for TLS~1.3 PSK modes yet?
% \item First tight security proof
% \item Main technical contribution is to address a gap in prior tight security proofs of TLS~1.3, where key schedule or components thereof are modeled as independent random oracles. We explain why this must be done carefully, due to potential interdependencies, and we give a careful analysis based on indifferentiability.
% \item This enables a theoretically-sound deployment of TLS~1.3 PSK modes with optimal cryptographic parameters, independent of the number of users, sessions, or other parameters.
% \end{itemize}

