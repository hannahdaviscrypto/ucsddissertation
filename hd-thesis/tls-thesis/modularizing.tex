\section{Modularizing Handshake Encryption}
\label{sec:modularizing}
%\TODO{possibly rename $KE_1$ and $KE_2$ in section 5 to match naming in corollary section}

Next will argue that using ``internal'' keys to encrypt handshake messages on the TLS~1.3 record-layer does not impact the security of other keys established by the handshake. 
%This will then enable us to use an as-simple-as-possible key exchange security model.
%In particular, we avoid considering such record-layer encryption within the model, as in ACCE-style security models~\cite{C:JKSS12}, for instance, which reduces the complexity of the model and proofs significantly. 
%

	Theorem~\ref{thm:maul}
		below %
formulates our argument in a general way, applicable to any multi-stage key exchange protocol, so that future analyses of similar protocols might take advantage of this modularity as well.

Intuitively, we argue as follows. Let $\KE_{2}$ be a protocol that provides multiple different stages with different external keys (i.e., none of the keys is used in the protocol, e.g., to encrypt messages), and let $\KE_{1}$ be the same protocol, except that some keys are ``internal'' and used, e.g., to encrypt certain protocol messages.
We argue that either using ``internal'' keys in $\KE_{1}$ does not harm the security of \emph{other} keys of $\KE_{1}$, or $\KE_{2}$ cannot be secure in the first place.
This will establish that we can prove security of a variant TLS~1.3 \emph{without} handshake encryption (in an accordingly simpler model),
and then lift this result to the actual TLS~1.3 protocol \emph{with} handshake encryption and the handshake traffic keys treated as ``internal'' keys.


\begin{theorem}
	\label{thm:TLS-transform}
	Let $\KE_1$ be the TLS~1.3 PSK-only resp.\ PSK-(EC)DHE mode \emph{with} handshake encryption (i.e., with internal stages $\KE_1.\INT = \{3,4\}$) as specified on the right-hand side in Figure~\ref{fig:tls-handshake}.
	Let $\KE_2$ be the same mode \emph{without} handshake encryption (i.e., $\KE_1.\INT = \emptyset$ and AEAD-encryption/decryption of messages is omitted).
	Let $\MaulSend$ and $\MaulRecv$ be the AEAD encryption resp.\ decryption algorithms deployed in TLS~1.3 and $\MaulKeys = \KE_1.\INT = \{3,4\}$.
Then we have	
%	For any adversary~$\advA$ against the multi-stage key exchange security of~$\KE_2$, running in time $t$ and making $\qSend$ queries to its $\Send$ oracle,
%	there exists an adversary~$\advB$ that runs in time $\approx t + \qSend \cdot t_{\mathrm{AEAD}}$ (where $t_{\mathrm{AEAD}}$ is the maximum time required to execute AEAD encryption or decryption of TLS~1.3 messages) and makes at most $\qSend$ queries to $\RevSessionKey$ in addition to queries made by $\advA$, and the same number of queries as~$\advA$ to all other oracles in the $\KESEC$ game,
%	such that
	\shortlongeqn[,]{
		\Adv^{\KESEC}_{\KE_1}(t, \qNewSecret, \qSend, \qRevSessionKey, \qRevLongTermKey, \qTest, \qRO)
		\leq
		\Adv^{\KESEC}_{\KE_2}(t+ t_{\mathrm{AEAD}} \cdot \qSend, \qNewSecret, \qSend, \qRevSessionKey + \qSend, \qRevLongTermKey, \qTest, \qRO)
% 		\genAdv{\KESEC}{\KE_2}{\advA} \leq \genAdv{\KESEC}{\KE_1}{\advB}.
	}
	where $t_{\mathrm{AEAD}}$ is the maximum time required to execute AEAD encryption or decryption of TLS~1.3 messages.
\end{theorem}




% we establish the following formal result saying that to analyze security of the TLS~1.3 PSK-only or PSK-(EC)DHE mode \emph{with} handshake encryption (captured as~$\KE_2$), it suffices to analyze the security of the same protocol \emph{without} handshake encryption (denoted $\KE_1$)---in an accordingly simpler model only treating ``external'' keys---, on which we can apply the AEAD encryption/decryption transformation.


For TLS~1.3 this means that we will not consider any security guarantees provided by the additional encryption of handshake messages.
We consider this as reasonable for PSK-mode ciphersuites, because the main purposes of handshake message encryption in TLS~1.3 is to hide the identities of communicating parties, e.g., in digital certificates, cf.\ \cite{PoPETS:ABFNO19}. 
In PSK mode there are no such identities. The $\pskid$ might be viewed as a string that could identify communicating parties, but it is sent unencrypted in the $\ClientHello$ message, anyway; the encryption of subsequent handshake messages would not contribute to its protection. 
%Therefore we consider it reasonable to essentially ignore the additional privacy-protecting properties of handshake message encryption in our model and proofs.


% For the case of the TLS~1.3 PSK modes,
% we will treat as $\KE_2$ the TLS~1.3 PSK-only resp.\ PSK-(EC)DHE mode as specified on the left-hand side of Figure~\ref{fig:tls-handshake} (i.e., \emph{with} handshake encryption using the internal keys from stages~3 and~4 ($\chtk$ and $\shtk$),
% and as $\KE_1$ the corresponding PSK handshake \emph{without} handshake encryption.


%\TODO{update Theorem 3 in Section 5 to take place after Section 4}

\iffull
	\iffull
	\subsection{Handshake Encryption as a Modular Transformation}
\else
	\section{Handshake Encryption as a Modular Transformation}
	\label{app:modular-transform}
\fi

Formally, let $\KE_2 = (\KEKGen, \KEActivate, \KERun)$ be a key exchange protocol with no internal keys. 
We define another key exchange protocol~$\KE_1$ which is parameterized by two functions $\MaulSend$ and $\MaulRecv$ and a list $\MaulKeys \subseteq \{1, \ldots, \STAGES\}$, where $\STAGES$ is the number of stages of $\KE_{2}$. $\KE_1$ inherits its key generation and activation algorithms from $\KE_2$. 
In its $\KE_1.\KERun$ algorithm, described in Figure~\ref{fig:encrypted-ke}, it essentially applies $\MaulRecv$ to a message before calling $\KE_2.\KERun$, and then $\MaulSend$ to the returned message, to transform the protocol messages as they pass over a wire.
This transformation may be, for instance, the encryption and decryption of messages of $\KE_{2}$ using an internal key.

\begin{figure}[p]
	\centering
	\begin{minipage}[t]{9cm}
		\begin{algorithm}{$\KE_1.\KERun(u,\pi^i_u, \pskey, m)$}
			\item $\keys \gets (\pi^i_u.\skey[\stage]$ for $\stage \in \MaulKeys)$
			\item $\acc \gets (\pi^i_u.\taccepted[\stage] \neq \infty$ for $\stage$ in $[1\ldots\STAGES])$
			\item $\tilde{m} \gets \MaulRecv(\keys, \pi^i_u.\role, \acc, m)$
			\item $(\pi^i_u, \tilde{m}') \gets \KE_2.\KERun(u,\pi^i_u, \pskey, \tilde{m})$
			\item $\keys \gets (\pi^i_u.\skey[\stage]$ for $\stage \in \MaulKeys)$
			\item $\acc \gets (\pi^i_u.\taccepted[\stage] \neq \infty$ for $\stage$ in $[1\ldots\STAGES])$
			\item $m' \gets \MaulSend(\keys, \pi^i_u.\role, \acc, \tilde{m}')$
			\item return $(\pi^i_u, m')$
		\end{algorithm}
	\end{minipage}
	\caption{Key exchange $\KE_1$ built by transforming protocol messages of $\KE_2$.}
	\label{fig:encrypted-ke}
\end{figure}

In addition to the messages, both algorithms take as input the list of stages that have been accepted by the current session, its role (initiator or responder) in the protocol, and a list of the keys from all stages in $\MaulKeys$. 
In the security game for $\KE_1$, the stages in $\MaulKeys$ will produce internal keys; all other keys remain external.

Although $\MaulSend$ and $\MaulRecv$ change the messages as they pass over the wire, the way that the messages are processed after receipt by $\KE_2.\KERun$ must not change.
In particular, $\KE_2.\KERun$, internally run within~$\KE_1.\KERun$, still expects messages of the same format and content;
also, $\KE_1$ defines its session and contributive identifiers, as well as all other session-specific information in the same way as $\KE_2$.


\paragraph{Correctness.}

Not all choices of $\MaulSend$ and $\MaulRecv$ are ``good choices''. 
For example, if mauling overwrites critical pieces of the protocol messages, then no honest session would ever accept a key. 
The resulting key exchange $\KE_2$ would be vacuously ``secure'' because it would be unusable. 


For our perspective to be meaningful, we therefore need a correctness property that guarantees that two honest parties executing $\KE_1$ with no adversarial interference will accept at all stages. 
%
Informally, we wish that if two sessions honestly executing $\KE_2$ will accept keys for stage $s$ with probability $p$, then two sessions honestly executing $\KE_1$ will accept keys for stage $s$ with probability close to $p$. 
This property only needs to hold when the protocol messages are relayed honestly, with no changes or delivery failures beyond those caused by the application of $\MaulSend$ and $\MaulRecv$.

We do not give a formal definition or proof of correctness for TLS~1.3, but we note that in TLS~1.3, the transformation algorithms are AEAD encryption and decryption.
Since decryption failures cannot occur in the standardized AEAD algorithms if messages are honestly relayed (due to their perfect correctness),
received messages will always match their corresponding sent message, and correctness of $\MaulSend$ and $\MaulRecv$ follows.


\paragraph{Security.}

We wish $\KE_1$ to be secure if $\KE_2$ is secure.
This should be independent of $\MaulSend$ and $\MaulRecv$, i.e., should hold even if $\MaulSend$ leaks its keys and fully overwrites all protocol messages.
The following theorem established this result, using that the keys used for the transformation are internal and $\MaulSend$ and $\MaulRecv$ have no access to other privileged information.
Therefore, their behavior can be mimicked by a reduction to the security of~$\KE_2$ as long as $\KE_2$ has ``public session matching'' for the stages in~$\MaulKeys$ of $\KE_1$, i.e., session partnering (or matching) for those stages is decidable from the publicly exchanged messages.%
\footnote{The property of ``public session matching'' has already already come up when considering the composition of (regular or multi-stage) key exchange protocols with subsequent symmetric-key protocols~\cite{CCS:BFWW11,CCS:DFGS15,EPRINT:DFGS15,thesis:Guenther18}.}

\begin{theorem}
	\label{thm:maul}
	Let $\KE_2$ be a key exchange protocol with $\STAGES$ stages, $\KE_2.\INT$ being empty, and public session matching.
	Let $\MaulSend$ and $\MaulRecv$ be algorithms as above and $\MaulKeys \subseteq \{1, \ldots, \STAGES\}$.
	Define key exchange $\KE_1$ such that $\KE_1.\KERun$ is described in Figure~\ref{fig:encrypted-ke}, $\KE_1.\INT = \MaulKeys$, and all other attributes of $\KE_1$ are identical to those of $\KE_2$.
	
	Let $\advA$ be an adversary with running time $t$ against the multi-stage key exchange security of $\KE_1$, making $\qSend$ queries to the $\Send$ oracle.
	Then there exists an adversary $\advB$ with running time $\approx t + \qSend m$, where $m$ is the maximum running time of $\MaulSend$ and $\MaulRecv$, such that
	\[
	\genAdv{\KESEC}{\KE_1}{\advA} \leq \genAdv{\KESEC}{\KE_2}{\advB}.
	\]
	$\advB$ makes at most $\qSend$ queries to $\RevSessionKey$ in addition to queries made by $\advA$
	and the same number of queries as~$\advA$ to all other oracles in the $\KESEC$ game.
\end{theorem}

\begin{proof}
	Adversary $\advB$ runs adversary $\advA$ and relays all of its queries to the appropriate oracles in its own $\KESEC$ game, except for $\Send$ queries.
	It maintains the time~$\time$ of the $\KESEC$ game itself, incrementing it once per query.
	For each session $\pi_u^i$, it maintains a list $\keys_u^i$ that is initially empty and a list $\acc_u^i$ in which $\acc_u^i[\stage]$ is initially $\false$ for each $\stage \in \MaulKeys$. 
	
	When $\advA$ makes a query $\Send(u, i, m)$, $\advB$ first checks for each $\stage \in \MaulKeys$ with $\acc_u^i[\stage] = \false$ whether $\pi_u^i.\accepted[\stage] \neq \infty$. 
	For each $\stage$ which satisfies this condition, $\advB$ checks whether $\pi_u^i.\tested[\stage]$ or $\pi_u^i.\revealed[\stage]$ is true and if $\pi_u^i$ has a partnered session (matching $\sid[\stage]$) which has been tested or revealed.
	(The latter check for partnering is possible because $\KE_1$ has public session matching.)
	% 	\hd{In order to do this, session IDs must be public. This is true of TLS~1.3, but we may need to make that assumption explicit.}
	If any of these conditions is true, then $\advB$ knows $\pi_u^i.\skey[\stage]$. 
	Otherwise, it makes an extra query $\RevSessionKey(u, i, \stage)$ and adds the response to $\keys_u^i$. 
	Then it marks $\acc_u^i[\stage] \gets \true$ and computes $\tilde{m} \gets \MaulRecv(\keys_u^i, \pi^i_u.\role, \acc_u^i, m)$. 
	It queries its own $\Send$ oracle on the tuple $(u,i, \tilde{m})$ and captures the response $\tilde{m'}$. 
	Then it returns $m' \gets \MaulSend(\keys_u^i, \pi^i_u.\role, \acc_u^i, \tilde{m}')$ to $\advA$.
	
	$\advB$ perfectly simulates $\KE_1$ for $\advA$, so we wish that if $\advA$ wins its simulated game, $\advB$ should also win its game.
	%
	$\advA$ can win the $\KESEC$ game in one of three ways: it can violate the $\Sound$ predicate, it can violate the $\ExplicitAuth$ predicate, or it can satisfy the $\Fresh$ predicate and guess the secret bit $b$. 
	All of the variables tracked by the $\ExplicitAuth$ and $\Sound$ predicates are maintained by the $\KESEC$ game for $\KE_1$, not by $\advB$.
	Therefore $\advA$ wins the simulated game by violating $\Sound$ or $\ExplicitAuth$ only if $\Sound$ or $\ExplicitAuth$ is violated in the $\KESEC$ game for $\KE_2$. 
	In this case, $\advB$ also wins. 
	
	If $\advA$ wins by guessing the secret bit $b$, the story is more complicated. 
	The bit $b$ is chosen by the $\KESEC$ game, so if $\advA$ guesses correctly, then so will $\advB$. 
	However, a correct guess only matters if the queries do not violate the $\Fresh$ predicate.
	Even if $\advA$ did not violate the $\Fresh$ predicate, $\advB$ makes up to $\qSend$ additional $\RevSessionKey$ queries.
	Each of these could cause $\Fresh$ to be set to false. 
	We claim that none of these queries violate the $\Fresh$ predicate.
	
	The $\Fresh$ predicate requires that no session be both tested and revealed.
	$\advB$ only reveals keys that have not already been tested, so the only  worry is that $\advA$ will test this key later. 
	However, all keys that $\advB$ reveals are in $\MaulKeys$, which is a subset of $\KE_1.\INT$, meaning they are internal keys.
	These keys cannot be tested if any session which has accepted it has moved on with the protocol.
	Since $\advB$ only reveals a key when a session has both accepted that key and received the next protocol message, it will have moved on and $\advA$ can not make any later $\Test$ queries on a key that $\advB$ has revealed.
	
	The next condition of $\Fresh$ is that a tested session's partner cannot be tested or revealed. 
	$\advB$ ensures that such a $\Test$ query does not occur before the $\RevSessionKey$ query. 
	Again, the $\Test$ query cannot happen after the $\RevSessionKey$ query because the session whose key was revealed has moved on with the protocol.
	Since all the revealed keys are internal in the simulated game, $\advA$ cannot test them after this point. 
	
	The remaining three conditions of the $\Fresh$ predicate establish different levels of forward secrecy. 
	They check for the existence of a contributive partner.
	We want to exclude the situation that a contributive partner exists in $\advA$'s simulated game, but not in $\advB$'s game.
	However, contributive identifiers are defined identically in $\KE_1$ and $\KE_2$. 
	Therefore if two sessions $\pi_u^i$ and $\pi_v^j$ have matching contributive identifiers in the simulated game for $\KE_2$, they will also have matching identfiers in the game for $\KE_1$.
	
	It is therefore not possible for $\advA$ to win its simulated $\KESEC$ game unless $\advB$ also wins its $\KESEC$ game, and the theorem follows.
\end{proof}
\else
\fi

 
%%% Local Variables:
%%% mode: latex
%%% TeX-master: "main"
%%% End:
