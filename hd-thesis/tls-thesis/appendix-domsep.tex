\section{A Careful Discussion of Domain Separation}
\label{app:domsep}

In our indifferentiability treatment of the TLS~1.3 key schedule (cf.\ Section~\ref{sec:ks-indiff}), we change what we capture as random oracles in the key exchange model.
We start with one random oracle, $\ROhash$, used wherever the hash function $\abstractHash$ would be called in the protocol.
We change this to classify queries to $\ROhash$ into two types:
\begin{description}
	\item[Type 1 queries:]
	\emph{component hashes} (via function~$\Chash$) used within $\abstractExtract$, $\abstractExpand$, and $\abstractMAC$ to compute $\HKDFExtr$, $\HKDFExpnd$, resp.\ $\HMAC$.
	
	\medskip
	
	\item[Type 2 queries:]
	\emph{transcript hashes} (via function~$\Thash$) computing hash values of protocol transcripts (or empty strings).
\end{description}
We wish to model $\Chash$ and $\Thash$ now as \emph{two} independent random oracles: $\ROchash$ resp.\ $\ROthash$.

To change the model, we can just change the pseudocode of the protocol to replace $\ROhash$ with whichever of $\ROchash$ and $\ROthash$ seems more appropriate. 
However, we must define an explicit construction that performs this substitution in a systematic way in order to give a formal proof of security.
This construction needs a Boolean condition to determine which of $\ROchash$ and $\ROthash$ should be queried, and this condition cannot be dependent on the higher-level context of the protocol's usage. 
Instead, we must define two disjoint sets $\Dom_{\Chash}$ and $\Dom_{\Thash}$ such that honest executions of TLS~1.3 only query $\ROhash$ on inputs in $\Dom_{\Chash}$ when computing $\HKDFExtr$, $\HKDFExpnd$, or $\HMAC$, and it otherwise only queries $\ROhash$ on inputs in $\Dom_{\Thash}$.


This separation must hold even when an honest session is responding to adversarially-chosen messages.
We do make some assumptions about the way that honest sessions process incoming messages. 
We assume that a server receiving a first $\ClientHello$ message from a client will not respond or execute the protocol unless the message contains correct encodings of all of the mandatory parameters for TLS~1.3. 
If the client fails to specify a valid group and key share in PSK-(EC)DHE mode, or version number, mode, and pre-shared key in any mode, the server should abort.
Of course, the $\ClientHello$ message may also contain invalid encodings of these values or even arbitrary data; we do not exclude this possibility.
Note that our conditions apply only to random-oracle queries made by honest executions of the protocol. 
An adversary may of course call $\ROhash$ on any input it chooses in either $\Dom_{\Chash}$ or $\Dom_{\Thash}$.

The TLS~1.3 handshake protocol does not provide any intentional domain separation between Type~$1$ and Type~$2$ queries.
We therefore turn to the formatting of queries to $\ROhash$ in the hopes of finding some unintentional separation.
We identify seven subtypes of query: five subtypes of Type $1$ and two subtypes of Type $2$.
Queries of each subtype have some unique formatting: a fixed length, a byte with a particular value, an encoded label. 
These attributes are heavily dependent on the specific configuration of the TLS~1.3 protocol; we therefore analyze four separate cases: two modes of operation (PSK-(EC)DHE and PSK-only mode) and two ciphersuites defining $\ROhash$ as $\SHA{256}$ and $\SHA{384}$ respectively.
Throughout, we will assume that any pre-shared-keys are the same length as the output length of $\ROhash$, i.e., $\hashlen$~bits.
This is true of resumption keys, but may not be true in general for pre-shared keys negotiated out-of-band. 
As TLS~1.3 fields length are given in (full) \emph{bytes}, we will be talking about \emph{byte lengths} if not otherwise stated in the following and use the shorthand $\hashlenBytes := \hashlen/8$ for the output length of $\ROhash$ in \emph{bytes}.
We also assume that if a Diffie--Hellman group is used, it is one of the standardized elliptic curve or finite field groups. 

All Type $1$ queries to $\ROhash$ are intermediate steps in the computation of $\HMAC$, $\HKDFExtr$, and $\HKDFExpnd$.
They consequently share some formatting which we discuss here before addressing each subtype individually.
$\HKDFExtr$ and $\HMAC$ are two names for the same function.
Given a key $K$ and input $s$, $\HKDFExpnd(K,s)$ pads $s$ with a single trailing counter byte with value \texttt{0x01}, then returns $\HMAC(K, s\|\|\texttt{0x01})$.
Therefore all Type $1$ queries to $\ROhash$ arise in the computation of $\HMAC$. 
$\HMAC[\ROhash](K,s)$ takes a key $K$ of length $\hashlenBytes$ bytes.
It then pads this key with zeroes up to the block length $\blocklenBytes$ of its hash function.
The block lengths of $\SHA{256}$ and $\SHA{384}$ are $64$ and $128$ bytes respectively. 
We call the padded key $K'$. 
Then $\HMAC[\ROhash]$ makes two queries to $\ROhash$:
\begin{enumerate}
	\item $d \gets \ROhash(K' \oplus \ipad || s)$
	\item $\ROhash(K' \oplus \opad || d)$
\end{enumerate}
$\ipad$ and $\opad$ are strings of $\blocklenBytes$ bytes. Each byte in $\ipad$ is fixed to \texttt{0x36}, and each byte in $\opad$ is fixed to \texttt{0x5c}.
The padded key $K'$ is $\blocklenBytes$ long, longer than $K$, so every Type $1$ query has a segment of length $\blocklenBytes-\hashlenBytes$ bytes in which each byte equals one of \texttt{0x36} and \texttt{0x5c}. 

Now we can present the seven subtypes of queries made by TLS~1.3. The first five types are Type~$1$ queries, and the last two (Empty and Transcript) are Type~$2$ queries.

The seven subtypes of queries are: 
\begin{enumerate}
	\item \textbf{Outer $\HMAC$ queries.}
	These queries are the second query made in the computation of $\HMAC$.
	Its key has length $\hashlenBytes$, and the digest $d$ also has length $\hashlenBytes$. 
	In between these is a segment containing $\blocklenBytes-\hashlenBytes$ bytes \texttt{0x5c}. 
	We will often refer to this segment as the ``fixed region''.  
	When the hash function is $\SHA{256}$, resp. $\SHA{384}$, the fixed region is $32$, resp $80$ bytes long. 
	The total query is $96$, resp. $176$ bytes long.
	
	\item \textbf{Inner $\HMAC$ queries.} We divide the first $\ROhash$ query made by $\HMAC$ into several subtypes; this type includes only those where the input to $\HMAC$ is an arbitrary string of length $\hashlenBytes$.
	This subtype is formatted identically to an outer $\HMAC$ query, except that the bytes of the fixed region are fixed to the value \texttt{0x36} instead of \texttt{0x5c}.  
	TLS~1.3 makes inner $\HMAC$ queries while computing $\Finished$ and $\binder$ messages (where the input is a hashed transcript), the early and master secrets, and in PSK-only mode, also the handshake secret.
	
	\item \textbf{Diffie--Hellman $\HMAC$ query.} In PSK-(EC)DHE mode, TLS~1.3 computes the handshake secret by calling $\HMAC$ on an encoded Diffie--Hellman key share. $\HMAC$'s first query is a Diffie--Hellman $\HMAC$ query. The formatting is the same as an inner $\HMAC$ hash except that the segment following the fixed region has a different length. The byte lengths ($|\G|/8$) of the encodings for each standardized Diffie--Hellman group can be found in Table~\ref{tbl:tls-groups}.

	\item \textbf{\tlsfunction{Derive-Secret} hashes.} The \tlsfunction{Derive-Secret} function is a component of the TLS key schedule~\cite[Section 7.1]{rfc8446}. Its inputs are a key of length $\hashlenBytes$, a label string of $2$ to $12$-bytes in length, and an input \tlsfield{Messages} string. 
	
	\tlsfunction{\tlsfunction{Derive-Secret}} queries $\ROhash$ three times: once to hash the \tlsfield{Messages} string, and twice as part of $\HKDFExpnd$. 
	The first of these three queries is a transcript query, and the third is an Outer $\HMAC$ query.
	The second query we call a \tlsfunction{Derive-Secret} query. 
	The \tlsfunction{Derive-Secret} query has the same formatting as Inner $\HMAC$ queries and Diffie--Hellman queries, but the segment following the fixed region contains a strictly formatted \tlsfield{HkdfLabel} struct~\cite[Section 7.1]{rfc8446}.
	
	This struct begins with a two-byte field encoding the integer value $\hashlenBytes$.
	This is followed by a variable-length vector with a $1$-byte length field containing the string \texttt{''tls13 ''} followed by a label string with length between $2$ and $12$ bytes. 
	Lastly comes a vector of length $\hashlenBytes$, prefixed with a $1$-byte field encoding its length. 
	The last byte in the input contains the \texttt{0x01}. 
	This byte is the counter mandated by the definition of $\HKDFExpnd$; however since $\HKDFExpnd$ is never called on inputs longer than $\hashlenBytes$, the counter never reaches a value higher than $1$.
	
	The total length of a the label struct, including the counter byte, is at least $\hashlenBytes+13$ bytes and at most $\hashlenBytes+23$ bytes. 
	
	\item \textbf{$\Finished$ hash.} The \tlsfunction{HKDF-Expand-Label} function is a subroutine of the \tlsfunction{Derive-Secret} function, but also called during the computation of $\Finished$ messages and the $\binder$ value~\cite[Section 4.4.4]{rfc8446}. 
	\tlsfunction{HKDF-Expand-Label} makes two calls to $\ROhash$. The second is an Outer $\HMAC$ hash; we call the first a $\Finished$ hash. 
	A $\Finished$ hash is identical to a \tlsfunction{Derive-Secret} hash, except that the label string is fixed to \texttt{finished} and the final vector has length $0$. 
	The counter byte is still present. 
	In total, the label struct occupies $19$ bytes.

	\item \textbf{Empty hashes.} Occasionally in the key schedule, TLS~1.3 calls $\ROhash$ on the empty string. 
	
	\item \textbf{Transcript hashes.} The last use of $\ROhash$ is to condense partial transcripts. 
	Each transcript includes at least a partial $\ClientHello$ message.
	We assume calling $\ROhash$. on a transcript which includes at least a partial $\ClientHello$.
	The minimum length of a partial $\ClientHello$ message in PSK-only mode is $69$~bytes.
	This includes the following fields~\cite[Section~4.1.2]{rfc8446}:
	\begin{itemize}
		\item 2 bytes \tlsfield{legacy\_version} fixed to \texttt{0x0303}
		\item 32 bytes \tlsfield{random}
		\item 1 byte \tlsfield{legacy\_session\_id} (for an empty vector with $1$-byte length field)
		\item 4 bytes \tlsfield{ciphersuites} (must include a $2$-byte length field and the value, e.g., \texttt{0x1301})
		\item 2 bytes \tlsfield{legacy\_compression\_methods} (must include a $1$-byte length field and the value \texttt{0x00})
		\item 2 bytes encoded length of \tlsfield{extensions} field
		\item 7 bytes \tlsfield{supported\_versions extension} extension~\cite[Section~4.2.1]{rfc8446} (must start with \texttt{0x002b} and include \texttt{0x0304})
		\item 6 bytes \tlsfield{psk\_key\_exchange\_modes} extension~\cite[Section~4.2.9]{rfc8446} (must start with \texttt{0x002d} and include \texttt{0x00})
		\item 9 bytes \tlsfield{pre\_shared\_key} extension~\cite[Section~4.2.11]{rfc8446} (partial: excluding the binder list; must come last, must start with \texttt{0x0029})
	\end{itemize}
	The first $43$ bytes (through the \tlsfield{extensions}' length encoding), must appear in the order displayed, although the \tlsfield{legacy\_session\_id}, \tlsfield{ciphersuites}, and \tlsfield{legacy\_compression\_methods} fields can be longer than the lengths given above. 
	We will occasionally refer to this segment as the ``fixed preface'' of a $\ClientHello$ because it must appear at the beginning of every well-formed $\ClientHello$ message. 
	The extensions can be reordered arbitrarily (except for the \tlsfield{pre\_shared\_key} extension) and additional extensions and ciphersuites can be added or repeated, up to a maximum length of $2^{16}-1$~bytes of ciphersuites and $2^{16}-2$ bytes for extensions. The overall maximum length of a $\ClientHello$ is then $2^{32} + 289$~bytes.
	A full $\ClientHello$ in PSK-only mode, including the binder list, adds at least another $3+\hashlenBytes$~bytes for a \tlsfield{binder} vector with a $3$~bytes of encoded length. The $\ClientHello$ message thus contains a minimum of $72+\hashlenBytes$~bytes and a maximum of $2^{32}+292+\hashlenBytes$ bytes.
		
	In PSK-(EC)DHE mode, two additional extensions are also mandatory: the \tlsfield{key\_share} and \tlsfield{supported\_groups} extensions~\cite[Section 9.2]{rfc8446}, so the minimum $\ClientHello$ length increases by at least $17$+$|\G|/8$ bytes, cf.\ Table~\ref{tbl:tls-groups}.
	This increase occurs for both truncated and full $\ClientHello$ messages.
	In this mode, a truncated $\ClientHello$ message is at least $86+|\G|/8$ bytes long, and a full $\ClientHello$ is at least $89+|\G|/8$ bytes long.
	
\end{enumerate}

\begin{figure}[tp]
	\centering
	\begin{tabular}{p{2.5cm}p{4cm}p{3.5cm}}
		\toprule
		Group name & \tlsfield{NamedGroup} enum value & Encoding length $|\G|/8$ \\ \midrule
		\groupname{secp256r1}~\cite{FIPS:186-4} & \texttt{0x0017} & $32$ \\ 
		\groupname{secp384r1}~\cite{FIPS:186-4} & \texttt{0x0018} & $48$ \\ 
		\groupname{secp521r1}~\cite{FIPS:186-4} & \texttt{0x0019} & $66$ \\ 
		\groupname{x25519}~\cite{rfc7748} & \texttt{0x001d} & $32$ \\ 
		\groupname{x448}~\cite{rfc7748} & \texttt{0x001E} & $56$ \\ 
		\groupname{ffdhe2048}~\cite{rfc7919} & \texttt{0x0100} & $128$ \\ 
		\groupname{ffdhe3072}~\cite{rfc7919} & \texttt{0x0101} & $192$ \\ 
		\groupname{ffdhe4096}~\cite{rfc7919} & \texttt{0x0102} & $256$ \\ 
		\groupname{ffdhe6144}~\cite{rfc7919} & \texttt{0x0103} & $384$ \\ 
		\groupname{ffdhe8192}~\cite{rfc7919} & \texttt{0x0104} & $512$ \\ \bottomrule
	\end{tabular}
	\caption{Table displaying the standardized groups for use with TLS~1.3, their encodings in the \tlsfield{NamedGroup} enum, and the length of an encoded group element in bytes.}
	\label{tbl:tls-groups}
\end{figure}
		
 
\subsection{PSK-only mode with \SHA{256}}

The block length of this hash function is $64$~bytes, and the output length is $32$~bytes.
In Table~\ref{tbl:domsep-psk-256}, we give the minimum and maximum input lengths for each of the six call types. (Diffie--Hellman $\HMAC$ calls do not occur in this mode.)

\begin{table}[tp]
	\centering
	\begin{tabular}{ccc}
		\toprule
		Type & Minimum length (bytes)~~ & Maximum length (bytes) \\
		\midrule
		Outer $\HMAC$ & $96$ & $96$ \\
		Inner $\HMAC$ & $96$ & $96$ \\
		\tlsfunction{Derive-Secret} & $109$ & $119$ \\
		$\Finished$ & $83$ & $83$ \\
		Empty & $0$ & $0$ \\
		Transcript & $69$ & $2^{32}+324$ \\
		\bottomrule
	\end{tabular}
	\medskip
	
	\caption{Table showing input lengths for hash function calls made by TLS~1.3 in PSK-only mode with \SHA{256}.}
	\label{tbl:domsep-psk-256}
\end{table}

In Table~\ref{tbl:domsep-psk-256} we note the minimum and maximum input lengths of each type of message.
For those types with overlapping length ranges, we must show they have separate domains by other means.
Outer and Inner $\HMAC$ hashes have identical lengths; however each of them has a $32$-byte fixed region. 
In outer $\HMAC$ hashes, the fixed region contains $\opad$; in inner $\HMAC$ hashes, it contains $\ipad$. 
These are distinct values, so no string can be both an outer and an inner $\HMAC$ hash. 

Transcript hashes are not domain-separated by length from any hash except the empty hashes.
We therefore turn to formatting to separate these from other types.
In the following, we visually lay out each byte of potentially overlapping inputs.

For a string to be both a transcript and an $\HMAC$ hash (outer or inner), it must be $96$~bytes (cf.~Table~\ref{tbl:domsep-psk-256}) long.
We diagram and compare a transcript hash containing a partial $\ClientHello$%
\footnote{A full $\ClientHello$ contains at least $72+\hashlenBytes\geq 104$ bytes, which is too long to be an $\HMAC$ hash.}
and an $\HMAC$ hash (outer or inner) in Figure~\ref{fig:domsep-PSKonly-256:partial-vs-HMAC}.
	
% 	\begin{tabular}{|p{4.3cm}|p{4.4cm}|p{0.9cm}|}
% 		\hline
% 		Fixed preface: $43$~bytes & Extension data: $44$~bytes & End of \tlsfield{pre\_shared\_key} extension: $9$~bytes \\
% 		\hline
% 	\end{tabular}
% 
% Next, we diagram an $\HMAC$ hash (outer or inner)
% 	
% 	\begin{tabular}{|p{3.2cm}|p{3.2cm}|p{3.2cm}|}
% 		\hline
% 		Key: $32$~bytes & Fixed region ($\ipad$ or $\opad$):  $32$~bytes & Arbitrary string: $32$~bytes \\
% 		\hline
% 	\end{tabular}

\begin{figure}[h]
	\centering
	\begin{tabular}{|p{4.3cm}|p{4.4cm}|p{0.9cm}|}
		\hline
		Fixed preface: $43$\;B & Extension data: $44$\;B & End \tlsfield{PSK}: $9$\;B \\
		\hline
	\end{tabular}
	
	\noindent
	\begin{tabular}{|p{3.2cm}|p{3.2cm}|p{3.2cm}|}
		\hline
		Key: $32$\;B & Fixed $\ipad$/$\opad$: $32$\;B & Arbitrary string: $32$~B \\
		\hline
	\end{tabular}
	
	\caption{%
		Domain separation in PSK-only mode with \SHA{256}:
		Transcript hash containing a partial $\ClientHello$ (top)
		vs.\
		(outer or inner) $\HMAC$ hash (bottom).
		``End \tlsfield{PSK}'' is the end of the \tlsfield{pre\_shared\_key} extension.
	}
	\label{fig:domsep-PSKonly-256:partial-vs-HMAC}
\end{figure}

We can see that the fixed preface of the transcript hash overlaps the fixed region of the $\HMAC$ hash that is fixed to either $\ipad$ or $\opad$. 
Consequently, the \tlsfield{legacy\_session\_id} vector must begin within the fixed region (at byte~$35$). 
This is a variable-length vector preceded by a $1$-byte length field, and its maximum length is $32$ bytes~\cite[Section 4.1.2]{rfc8446}. 
Therefore the maximum value of the length field is \texttt{0x20} and it cannot contain either byte \texttt{0x36} or \texttt{0x5c}. 
Any string containing a valid partial $\ClientHello$ therefore cannot also be a correctly formatted $\HMAC$ hash.
	
The same argument applies to $\Finished$ and \tlsfunction{Derive-Secret} hashes, both of which contain the same fixed region in the same location as inner $\HMAC$ hashes.
%	\item \textbf{Transcript vs.\ $\Finished$.}
%	Collisions may occur only on $83$-byte transcript inputs. 
%	A transcript hash of $83$~bytes contains:\\
%	\begin{tabular}{|p{4.3cm}|p{3.1cm}|p{0.9cm}|}
%		\hline
%		Fixed preface ($43$~bytes) & Arbitrary extensions ($31$~bytes) & End of PSK extension ($9$~bytes) \\
%		\hline
%	\end{tabular}
%
%	A $\Finished$ hash \new{contains}:\\
%	\begin{tabular}{|p{3.2cm}|p{3.2cm}|p{1.9cm}|}
%		\hline
%		Key ($32$~bytes) & Fixed to $\ipad$ ($32$~bytes) & \texttt{0x00200e} ||\texttt{``tls13 finished''} ||\texttt{0x0001} ($19$~bytes) \\
%		\hline
%	\end{tabular}
%	As for $\HMAC$ hashes, the fixed preface will differ from $\ipad$ in at least one byte between bytes $33$ and $43$ because this section contains the ciphersuite encodings.
%	
%	\item \textbf{Transcript vs.\ \tlsfunction{Derive-Secret}.}
%	Collisions may occur on transcripts of any length between $109$ and $119$~bytes.
%
%	We diagram a $109$~byte transcript, but the same principle applies to any input.
%	 
%	\begin{tabular}{|p{4.3cm}|p{5.7cm}|p{0.9cm}|}
%		\hline
%		Fixed preface ($43$~bytes) & Arbitrary extensions ($57$~bytes) & End of PSK extension ($9$~bytes) \\
%		\hline
%	\end{tabular}
%
%	A $109$-byte \tlsfunction{Derive-Secret} hash:
%	 
%	\begin{tabular}{|p{3.2cm}|p{3.2cm}|p{1.1cm}|p{3.4cm}|}
%		\hline
%		Key ($32$~bytes) & Fixed to $\ipad$ or $\opad$ ($32$~bytes) & \small{ \texttt{0x0020} || \texttt{0x08} || \texttt{``tls13 iv''} || \texttt{0x20} } & Transcript hash vector ($33$~bytes) || \texttt{0x01} \\
%		\hline
%	\end{tabular}
%	
%	As for $\HMAC$ hashes, the fixed preface will differ from $\ipad$ in at least one byte between bytes $33$ and $43$ because this section contains the ciphersuite encodings.
%\end{itemize}

For this mode, we define the set $\Dom_{\Thash}$ to include of the empty string and all strings of length greater than or equal to $69$~bytes for which the $35^{\text{th}}$ byte is not equal to $\ipad$ or $\opad$.
We let $\Dom_{\Chash}$ contain all other elements of $\bits^*$.

\subsection{Pre-shared key with Diffie--Hellmann mode with \SHA{256}}

Again, we present the minimum and maximum lengths of each hash type; see Table~\ref{tbl:domsep-psk-dhe-256}.
We now include Diffie--Hellman $\HMAC$ hashes, and transcript hashes include additional mandatory extensions for PSK-(EC)DHE mode.
\begin{table}[tp]
	\centering
	\begin{tabular}{ccc}
		\toprule
		Type & Minimum length (bytes)~~ & Maximum length (bytes) \\ \midrule
		Outer $\HMAC$ & $96$ & $96$ \\ 
		Inner $\HMAC$ & $96$ & $96$ \\ 
		Diffie--Hellman $\HMAC$ & $64+|\G|/8$ & $64+|\G|/8$ \\
		\tlsfunction{Derive-Secret} & $109$ & $119$ \\ 
		$\Finished$ & $83$ & $83$\\ 
		Empty & $0$ & $0$ \\ 
		Transcript & $86+|\G|/8$ & $2^{32}+324$ \\ 
		\bottomrule
	\end{tabular}
	\medskip
	
	\caption{Table showing input lengths for hash function calls made by TLS~1.3 in PSK-(EC)DHE mode with \SHA{256}. For transcript hashes, the encoding lengths $|\G|/8$ can be found in Table~\ref{tbl:tls-groups}.}
	\label{tbl:domsep-psk-dhe-256}
\end{table}

In this mode, Diffie--Hellman $\HMAC$ hashes may collide with Inner $\HMAC$ or \tlsfunction{Derive-Secret} hashes for certain choices of $\G$.
This is not a failure of domain separation because these inputs to these three types will all belong to $\Dom_{\Chash}$.
Transcript hashes now only have length overlaps with Diffie--Hellman $\HMAC$ and \tlsfunction{Derive-Secret} hashes.
In both cases, however, the same argument about the $35^{\text{th}}$ byte containing the length of \tlsfield{legacy\_session\_id} applies, and no string can be two different types. 

For this mode, the set $\Dom_{\Thash}$ consists of the empty string and all strings of length greater than or equal to $86+|\G|$~bytes for which the $35^{\text{th}}$ byte is not equal to $\ipad$ or $\opad$. $\Dom_{\Chash}$ contains all other elements of $\bits^*$. 

\subsection{Pre-shared key with Diffie--Hellmann mode with \SHA{384}}

Table~\ref{tbl:domsep-psk-dhe-384} shows the minimum and maximum lengths of each hash type for this configuration.
The hash function \SHA{384} has $48$-byte output and $128$-byte block length, so the fixed region in $\HMAC$, $\Finished$, and \tlsfunction{Derive-Secret} hashes will be $80$ bytes long.

Because $48$~byte $\HMAC$ keys are longer than the $43$~byte fixed preface of a $\ClientHello$, we cannot rely on the distinction between \tlsfield{legacy\_session\_id} and the fixed region for domain separation. Instead, we consider whether a minimum-length $\ClientHello$ can accommodate the mandatory extensions for this mode.

\begin{table}[tp]
	\centering
	\begin{tabular}{ccc}
		\toprule
		Type & Minimum length (bytes)~~ & Maximum length (bytes) \\ \midrule 

		Outer $\HMAC$ & $176$ & $176$ \\
		Inner $\HMAC$ & $176$ & $176$ \\ 
		Diffie--Hellman $\HMAC$ & $128+|\G|/8$ & $128+|\G|/8$ \\  
		\tlsfunction{Derive-Secret} & $189$ & $199$ \\
		$\Finished$ & $147$ & $147$\\ 
		Empty & $0$ & $0$ \\
		Transcript & $86+|\G|/8$ & $2^{32}+324$ \\ \hline 
	\end{tabular}
	\medskip
	
	\caption{Table showing input lengths for hash function calls made by TLS~1.3 in PSK-(EC)DHE mode with \SHA{384}.}
	\label{tbl:domsep-psk-dhe-384}
\end{table} 

We worry only about possible collisions between transcript hashes and the other types: $\Finished$, $\HMAC$, and \tlsfunction{Derive-Secret}.
We diagram a transcript hash of $176$~bytes together with an outer $\HMAC$ hash as a demonstration of the domain-separation argument in Figure~\ref{fig:domsep-PSKDHE-384:transcript-vs-HMAC}, but the same argument applies to all. 

% Transcript hash of $176$ bytes: 
% 
% \begin{tabular}{|p{3.21cm}|p{9.3cm}|p{0.7cm}|}
% 	\hline
% 	Fixed preface: $43$~bytes & Extension data: $124$~bytes & End of \tlsfield{pre\_shared\_key} extension: $9$~bytes \\
% 	\hline
% \end{tabular}
% 
% Outer $\HMAC$ hash:
% 
% \begin{tabular}{|p{3.6cm}|p{6cm}|p{3.6cm}|}
% 	\hline
% 	Key: $48$~bytes) & Fixed region ($\opad$): $80$~bytes & Arbitrary string: $48$~bytes \\
% 	\hline
% \end{tabular}

\begin{figure}[h]
	\centering
	
	\scalebox{0.9}{%
	\begin{tabular}{|p{3.21cm}|p{9.3cm}|p{0.7cm}|}
		\hline
		Fixed preface: $43$\;B & Extension data: $124$\;B & End \tlsfield{PSK}: $9$\;B \\
		\hline
	\end{tabular}
	}

	\noindent
	\scalebox{0.9}{%
	\begin{tabular}{|p{3.6cm}|p{6cm}|p{3.6cm}|}
		\hline
		Key: $48$\;B & Fixed region ($\opad$): $80$\;B & Arbitrary string: $48$\;B \\
		\hline
	\end{tabular}
	}
	
	\caption{%
		Domain separation in PSK-(EC)DHE mode with \SHA{384}:
		Transcript hash of $176$~bytes (top)
		vs.\
		outer $\HMAC$ hash (bottom).
		``End \tlsfield{PSK}'' is the end of the \tlsfield{pre\_shared\_key} extension.
	}
	\label{fig:domsep-PSKDHE-384:transcript-vs-HMAC}
\end{figure}

There are no obvious conflicts here: the fixed preface of a $\ClientHello$ message is covered by the key section of the $\HMAC$ hash, and the \tlsfield{pre\_shared\_key} extension is covered by the arbitrary string at the end.
However, notice that of the $124$ bytes available for extension data in the $\ClientHello$, $80$~of them must be fixed to $\opad$ to allow a collision. 
Even including the $5$~bytes immediately after the fixed preface and $9$~bytes reservedf or the \tlsfield{pre\_shared\_key} extension, this leaves only $58$~bytes. 
In PSK-(EC)DHE mode, five extensions are mandatory even for truncated $\ClientHello$ messages. 
They are \tlsfield{supported\_versions}~\cite[Section 4.2.1]{rfc8446} (minimum $7$~bytes), \tlsfield{supported\_groups}~\cite[Section 4.2.7]{rfc8446} (minimum $7$~bytes), \tlsfield{key\_share}~\cite[Section 4.2.8]{rfc8446} (minimum $16+|\G|/8$~bytes), \tlsfield{psk\_key\_exchange\_modes}~\cite[Section 4.2.9]{rfc8446} (minimum $6$~bytes), and \tlsfield{pre\_shared\_key}~\cite[Section 4.2.11]{rfc8446} (minimum $13$~bytes). 
Even for the smallest choice of $\G$, at least $71$ bytes are required to contain these extensions. 
At least one of the extensions must overlap with the fixed field, and will differ from $\opad$ in at least one byte.

Any valid transcript hash will need at least $92+|\G|/8$ bytes outside the fixed region: $43$~bytes for the preface and $49+|\G|/8$ for the mandatory extensions. An outer $\HMAC$ hash has only $124$ unfixed bytes and cannot meet this threshold. 
This is true also for inner $\HMAC$ hashes ($96$ unfixed bytes), and Diffie--Hellman $\HMAC$ hashes, which have $48+|\G|/8$ unfixed bytes. 
It is true for $\Finished$ hashes, which have $48$ unfixed bytes. 
And it is true for \tlsfunction{Derive-Secret} hashes, which have at most $119$ unfixed bytes.

Let us be even more clear about why this overlap means no collision is possible. 
We cannot fit all of the extensions in the $48+|\G|$ bytes after the fixed region. 
Therefore one of the extensions must start either in the fixed region, or before the fixed region. 
None of these extensions can start in the fixed region because they all begin with an extension type different from $\ipad$ or $\opad$. 
Therefore one of them must start before the fixed region and continue into the fixed region. 
We call this the ``first extension''.
The \tlsfield{pre\_shared\_key} extension must be the last extension, so it cannot be the first extension.
Therefore  the first extension is one of \tlsfield{key\_share}, \tlsfield{supported\_groups}, and \tlsfield{psk\_key\_exchange\_modes}, and \tlsfield{supported\_versions}.

All extensions start with a $4$~byte encoding of their type and length. 
This means that the first extension may contain only one arbitrary byte of data before $80$ bytes of $\ipad$ or $\opad$. All four possible extensions consist of variable-length vectors. TLS encodes all variable-length vectors with a $1$ or $2$~byte prefix encoding their length. Consequently, the entries of the vector fall in or after the fixed region. 

Each of the vector entries in the four possible first extensions begins with an element from an enum: either the \tlsfield{NamedGroup}, \tlsfield{ProtocolVersion}, or \tlsfield{PskKeyExchangeMode} enums. 
Luckily, none of these enums contain the bytes \texttt{0x36} or \texttt{0x5c}. To demonstrate this, we present the \tlsfield{NamedGroup} values in Table~\ref{tbl:tls-groups}~\cite[Section 4.2.7]{rfc8446}. The \tlsfield{ProtocolVersion} encoding for TLS~1.3 is \texttt{0x0304} \cite[Section 4.2.1]{rfc8446}, and the elements of the \tlsfield{PskKeyExchangeMode} enums are \texttt{0x00}, \texttt{0x01}, and \texttt{0xff}~\cite[Section 4.2.9]{rfc8446}. 
Of course, a $\ClientHello$ message can contain badly formed extensions. 
We assume, however, that each of the mandatory extensions must contain one correctly formatted vector entry. 
Without these entries, communication partners will not be able to select the correct version, group, or mode to execute the protocol; we assume that in this case they would abort. 
Because the fixed region contains no valid enum elements, this correctly formatted vector entry must begin after the fixed region. 
Therefore the first extension uses at most $1$ byte of the fixed region to encode meaningful data (a possible second byte of the vector length encoding). 
The mandatory extensions must occupy no more than $5$~bytes before the fixed region, $1$~byte in the fixed region, and either $71$~bytes after the fixed region (for the longest possible \tlsfunction{Derive-Secret} hash) or $|\G|/8$~bytes after (for an inner $\HMAC$ hash). But summing their minimum lengths gives $49+|\G|/8$ bytes. Even for the smallest possible $|\G|/8 = 32$, the extensions just do not fit in the given space.
It is therefore impossible to construct a valid $\ClientHello$ message, truncated or otherwise, that collides with a possible $\HMAC$, \tlsfunction{Derive-Secret}, or \tlsfunction{Finished} hash. 

Consequently we can set $\Dom_{\Thash}$ to contain the empty string and all strings of at least $86$ bytes for which at least one of bytes $48$ through $128$ does not equal either $\ipad$ or $\opad$. As usual, we set $\Dom_{\Chash}$ to be all other elements of $\bits^*$. 


\subsection{PSK-only mode with \SHA{384}}

In this mode/hash function combination, the transcript hash can collide with outer $\HMAC$ hashes.
There are other collisions as well, but one is sufficient to demonstrate the lack of domain separation.
We illustrate this via a $176$-byte transcript hash (containing a truncated $\ClientHello$) and an outer $\HMAC$ hash, shown in Figure~\ref{fig:domsep-PSKonly-384:transcript-vs-HMAC}.

% A $176$-byte transcript hash (containing a truncated $\ClientHello$):
% 
% \begin{tabular}{|p{3.2cm}|p{6.42cm}|p{1.81cm}|p{1.69cm}|}
% 	\hline
% 	Fixed preface ($43$~bytes) & \tlsfield{supported\_version} ($87$~bytes) & \tlsfield{cookie} ($24$~bytes)&Mandatory extensions ($22$~bytes) \\
% 	\hline
% \end{tabular}
% 
% An outer $\HMAC$ hash: 
% 
% \begin{tabular}{|p{3.6cm}|p{6.0cm}|p{3.6cm}|}
% 	\hline
% 	Key: $48$~bytes & Fixed region ($\opad$): $80$~bytes & Arbitrary string: $48$~bytes \\
% 	\hline
% \end{tabular}


\begin{figure}[h]
	\centering
	
	\scalebox{0.68}{%
% 	\begin{tabular}{|p{3.21cm}|p{9.3cm}|p{0.7cm}|}
% 		\hline
% 		Fixed preface: $43$\;B & Extension data: $124$\;B & End \tlsfield{PSK}: $9$\;B \\
% 		\hline
% 	\end{tabular}

	\begin{tabular}{|p{4.3cm}|p{8.7cm}|p{2.4cm}|p{1.755cm}|} %% last column should be 2.2cm, but this tabular has one column more, so adjust for spacing here
	\hline
		Fixed preface: $43$\;B & \tlsfield{supported\_version}: $87$\;B & \tlsfield{cookie}: $24$\;B & Mandatory extensions: $22$\;B \\
		\hline
	\end{tabular}
	}

	\noindent
	\scalebox{0.68}{%
% 	\begin{tabular}{|p{3.6cm}|p{6cm}|p{3.6cm}|}
% 		\hline
% 		Key: $48$\;B & Fixed region ($\opad$): $80$\;B & Arbitrary string: $48$\;B \\
% 		\hline
% 	\end{tabular}
	\begin{tabular}{|p{4.8cm}|p{8.0cm}|p{4.8cm}|}
		\hline
		Key: $48$~bytes & Fixed region ($\opad$): $80$~bytes & Arbitrary string: $48$~bytes \\
		\hline
	\end{tabular}
	}
	
	\caption{%
		Failing domain separation in PSK-only mode with \SHA{384}:
		Transcript hash of $176$~bytes, containing a truncated $\ClientHello$, (top)
		vs.\
		outer $\HMAC$ hash (bottom).
		``End \tlsfield{PSK}'' is the end of the \tlsfield{pre\_shared\_key} extension.
	}
	\label{fig:domsep-PSKonly-384:transcript-vs-HMAC}
\end{figure}



We construct the following message, which is both a truncated $\ClientHello$ (and therefore a transcript hash) and an outer $\HMAC$ hash. 
We let the first extension be the \tlsfield{supported\_versions} extension. 
This extension will contain $41$ protocol versions. 
The first $40$ versions will be two bytes of $\opad$: \texttt{0x5c5c}; the last will be the real version number \texttt{0x0304}. 
These extra version numbers match the $\HMAC$ key padding, and the real version number lies in the last $48$~bytes, which are unrestricted by the formatting of an $\HMAC$ hash. 

We place the remaining mandatory extensions at the end of the content section. 
In PSK-only mode, these are only  \tlsfield{psk\_key\_exchange\_modes}, and (the truncated) \tlsfield{pre\_shared\_key}, and they take up $22$~ bytes. 
This leaves $24$ bytes unaccounted for between the end of the \tlsfield{supported\_versions} extension and the start of \tlsfield{psk\_key\_exchange\_modes}. 
We can fill these with a \tlsfield{cookie} ~\cite[Section 4.2.2]{rfc8446} extension with arbitrary content. 
(We can also fill these bytes without including additional extensions.)

This type of collision is unavoidable, so there are no disjoint sets $\Dom_{\Thash}$ and $\Dom_{\Chash}$ that capture the way TLS~1.3 calls \SHA{384} in pre-shared key only mode. Consequently the indifferentiability step of Section~\ref{sec:domsep} does not apply to this mode. 

\iffull
\subsection{Repairing domain separation for TLS~1.3-like protocols}\label{apx-domsep-fix}
The above analysis demonstrates that complete domain separation is nontrivial to achieve for a protocol like TLS~1.3 which uses a hash function for multiple purposes and at multiple levels of abstraction.
We would like to present our suggestions for how this could be achieved most simply and efficiently in future iterations of TLS and other schemes. 
As discussed by Bellare et al.~\cite{EC:BelDavGun20}, the most well-known method of domain separation is the inclusion of distinct labels into each hash function call; this is precisely the method adopted by TLS~1.3 to distinguish calls to its \tlsfunction{Derive-Secret} function.
Ideally, a future scheme could specify a unique label string for each purpose: not only the various derived secrets, but also each time the transcript is hashed and each internal call made by $\HMAC$, $\HKDFExtr$, and $\HKDFExpnd$.

Unfortunately, this ideal method is not compatible with the existing specifications of $\HMAC$ and $\HKDF$. 
Both of these functions make ``Outer $\HMAC$ queries'' as discussed above; these calls have a fixed input length of $\blocklenBytes+ \hashlenBytes$ bytes and this input does not include a label.
A protocol could avoid this roadblock by using a custom implementation of $\HMAC$ or $\HKDF$ whose underlying hash function prepends an $\HMAC$-specific label to its input.
This approach would be both standard-compliant and efficient, but we don't recommend it because existing cryptographic libraries already have trustworthy $\HMAC$ and $\HKDF$ functionality and encouraging custom implementations for every new protocol increases the probability of accidental errors in these new implementations.
Instead, we suggest making no adjustments to the internal execution of $\HMAC$ or $\HKDF$ and instead altering direct hash function calls (the other six subtypes we discuss) to avoid collisions.

In practice, this means that under our recommendation, all hash function calls which are not outer $\HMAC$ queries should obey two simple rules: first, they should end with a unique label and second, that their input must not be $\blocklenBytes+\hashlenBytes$ long.
To conform with the first rule, TLS 1.3 would need to make the following changes.
\begin{enumerate}
	\item Add distinct labels to the end of each transcript before hashing; for clarity we suggest using the names of the last message in the transcript; i.e. ``$\PartialClientHello$'', ``$\ClientHello$'', ``$\ServerHello$'', etc. If $\HKDF$ is used, we would also recommend that these labels should not end with the byte \texttt{0x01}.
	\item Add distinct labels to the end of the input each time $\HMAC$ is called; this would include for Inner $\HMAC$ queries, Diffie--Hellman $\HMAC$ queries, $\Finished$ queries, and $\tlsfunction{Derive-Secret}$ queries. Note that the labels should be postpended to the $\HMAC$ payload and not the key. The labels used by $\tlsfunction{Derive-Secret}$ could then be omitted, although this is not necessary.
	\item Ensure that none of the labels used is a suffix of another; this can introduce collisions even if the labels are distinct.
\end{enumerate}
We encourage using suffixes for domain separation, although prefixes are more commonly-used, because they are easier to use in conjunction with $\HMAC$ and $\HKDF$.
Although we are not applying labels to outer $\HMAC$ queries, we would still like to use them to domain separate inner $\HMAC$ queries (and the other subtypes).
The inputs to these queries begin with the $\HMAC$ key, which undergoes an XOR operation with $\ipad$ before it is hashed.
So prefixed labels would need to remain unique and prefix-free after this XOR operation; this introduces some confusion that we prefer to avoid.
Additionally, the second step of our indifferentiability proof relies crucially on the fact that $\HMAC$ uses fixed-length keys shorter than $\blocklenBytes$; prefixed labels would therefore need to share a fixed length shorter than $\blocklenBytes-\hashlenBytes$ bytes. 
With suffixes, we still need to contend with the counter byte that $\HKDFExpnd$ appends to its input, but in TLS~1.3 where this byte is always \texttt{0x01}, this presents less of a restriction.

To conform with the second rule, TLS~1.3 would need to enforce that it never hashes a string of $\blocklenBytes+\hashlenBytes$ except as an Outer $\HMAC$ query. 
The easiest and least error-prone way to do this would be to pad every non-empty hash function call and input to $\HMAC$ and $\HKDF$ with exactly $\blocklenBytes + \hashlenBytes$ bytes (before the suffixed labels); all calls would strictly longer than $\blocklenBytes+\hashlenBytes$.
This method adds two additional compression function calls to each hash function execution.
There are some ways to lessen this requirement without impacting the effectiveness of the length-based domain separation.
Calls which already have input longer than $\blocklenBytes+\hashlenBytes$ bytes can omit the padding; so can calls which have strictly shorter input.
It would also be possible to use only as much padding is needed to make input at least $\blocklenBytes+\hashlenBytes+1$ bytes long.
However, non-uniform padding should be done carefully so that, for example, two previously distinct $\ClientHello$ messages do not collide after being padded.
\fi