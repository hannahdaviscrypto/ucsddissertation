\documentclass[11pt]{ucsddissertation}
% mathptmx is a Times Roman look-alike (don't use the times package)
% It isn't clear if Times is required. The OGS manual lists several
% "standard fonts" but never says they need to be used.
\newcommand\hmmax{0}
\newcommand\bmmax{2}
\usepackage{mathptmx}
\usepackage[NoDate]{currvita}
\usepackage{array}
\usepackage{tabularx}
\usepackage{booktabs}
\usepackage{ragged2e}
\usepackage{microtype}
\usepackage{multirow}
\usepackage{setspace}
\usepackage{xspace}
\usepackage{stmaryrd}
\usepackage[usenames,dvipsnames,table,x11names]{xcolor}
%\usepackage{color}
\PassOptionsToPackage{hyphens}{url}

%%% DOMSEP PACKAGES%%%%%%%%%%%%%%%%
\usepackage{cryptocode}
\usepackage{breakcites}
\usepackage{latexsym}
\usepackage{amscd,amsmath}
\usepackage{amsfonts}
\usepackage{float}
\usepackage{longtable}
%
%
\renewcommand{\topfraction}{0.9}
\renewcommand{\bottomfraction}{0.9}
\renewcommand{\textfraction}{0.1}
%
\usepackage{enumitem}
\setlist[itemize]{noitemsep, topsep=0pt}
\setlist[enumerate,itemize,description]{labelindent=1em,leftmargin=2em}
%
%%added by Yi
%\usepackage{mathtools}
%
%%added by Felix
\usepackage{crypto-environments}
\usepackage[normalem]{ulem}
\usepackage{lmodern}
%
%%added by Hannah
\usepackage{dsfont}
\usepackage{scrextend}

%%%%%%%%%%% TIGHT-AKE PACKAGES %%%%%%%%%%%%%

\newif\iffull
\fulltrue  % comment out for llncs

\newif\ifanon
\anonfalse

%\usepackage[letterpaper,hmargin=1in,vmargin=1in]{geometry}

\usepackage[T1]{fontenc}
\usepackage[utf8]{inputenc}

\usepackage{xifthen}
\usepackage{soul}	\let\strikethrough\st\let\st\undefined % \st already used...
\usepackage{amssymb}
\usepackage{multicol}
\usepackage{graphicx}
\usepackage[english]{babel}	
% \usepackage{authblk}
\iffull\else
% unset llncs.cls definition of proof environment
\let\proof\relax
\let\endproof\relax
\fi
\usepackage{amsthm}


\usepackage{sigma-thesis/collect}
%\usepackage{eso-pic}
\usepackage{sigma-thesis/crypto-environments}

\usepackage{tikz}
\usetikzlibrary{calc,matrix,arrows.meta,automata,shapes.geometric,positioning}
\usepackage{pgfplots}


\pgfplotsset{compat=1.15}

%%%%%%%%%%%TIGHT-TLS PACKAGES %%%%%%%%%%%%%%
\usepackage[breaklinks=true,pdfborder={0 0 0},pdftex, pdftitle={},
       pdfauthor={},
       pdfpagelabels=true, linktocpage=true, backref=page,urlcolor=blue,
       colorlinks,citecolor=blue,linkcolor=blue,
       bookmarksdepth=3,bookmarksopen=true]{hyperref}

%%%%%%%%%%EDDSA PACKAGES %%%%%%%%%%%

\usepackage{listings}
\AtBeginDocument{%
	\settowidth\cvlabelwidth{\cvlabelfont 0000--0000}%
}

%%%%%%%%%%%VDAF PACKAGES%%%%%%%%%%%
\usepackage{subfig}
\usepackage{wrapfig}
  \usepackage[yyyymmdd]{datetime}
\renewcommand{\dateseparator}{/}
\usepackage{changepage}
\usepackage{varwidth}
% OGS recommends increasing the margins slightly.
\increasemargins{.1in}


%\usepackage{showframe} % This package was just to see page margins
\usepackage[english]{babel}
\usepackage{blindtext}
\overfullrule5pt
% ---

% Required information
\title{Securing the Standards: Bringing Cryptographic Security Proofs Closer To the Real World}
\author{Hannah Elizabeth Davis}
\degree{Doctor of Philosophy}{Computer Science}
% Each member of the committee should be listed as Professor Foo Bar.
% If Professor is not the correct title for one, then titles should be
% omitted entirely.
\chair{Professor Mihir Bellare}
% \cochair{Professor Gamma Delta} % Optional
% Your committee members (other than the chairs) must be in alphabetical order
\committee{Professor Farinaz Koushanfar}
\committee{Professor Daniele Micciancio}
\committee{Professor Stefan Savage}
\committee{Professor Deian Stefan}
\degreeyear{2023}

%%%%%%%%%% HEADER FILES %%%%%%%%%%%%%
\makeatletter
\AtBeginDocument{%
	\def\doi#1{\url{https://doi.org/#1}}}
\makeatother

\renewcommand*{\backref}[1]{(Cited on page~#1.)}
\DeclareMathAlphabet{\mathsc}{OT1}{cmr}{m}{sc}
\DeclareMathAlphabet{\mathslbf}{OT1}{cmr}{bx}{sl}

\setlength{\fboxsep}{1pt}

% ====================================================================
%
\newcommand{\gamesfontsize}{\small}
\newcommand{\gameName}[1]{\underline{$\main$ #1}\\[2pt]}


\newcommand{\twoColsNoDivide}[4]{
\begin{center}
        \framebox{
        \begin{tabular}{c@{\hspace*{.4em}}c@{\hspace*{.4em}}c}
        \begin{minipage}[t]{#1\textwidth}\setstretch{1.0}\gamesfontsize #3 \end{minipage}
        &
        \begin{minipage}[t]{#2\textwidth}\setstretch{1.0}\gamesfontsize #4 \end{minipage}
        \end{tabular}
        }
\end{center}
}

\newcommand{\oneCol}[2]{
\begin{center}
        \framebox{
        \begin{tabular}{c}
        \begin{minipage}[t]{#1\textwidth}\setstretch{1.0}\gamesfontsize #2 \end{minipage}
        \end{tabular}
        }
\end{center}
}


\newcommand{\twoCols}[4]{
\begin{center}
        \framebox{
        \begin{tabular}{c@{\hspace*{.4em}}|c@{\hspace*{.4em}}}
        \begin{minipage}[t]{#1\textwidth}\setstretch{1.0}\gamesfontsize #3 \end{minipage}
        &
        \begin{minipage}[t]{#2\textwidth}\setstretch{1.0}\gamesfontsize #4 \end{minipage}
        \end{tabular}
        }
\end{center}
}
%
%% Theorem environments
%
\newtheorem{thm}{Theorem}
\newtheorem{lem}{Lemma}


\newenvironment{theorem}{\begin{thm}}{\end{thm}}
\newenvironment{lemma}{\begin{lem}}{\end{lem}}

\newcommand{\secref}[1]{Section~\ref{#1}}
\newcommand{\apref}[1]{Appendix~\ref{#1}}
\newcommand{\thref}[1]{Theorem~\ref{#1}}

\newcommand{\figref}[1]{Figure~\ref{#1}}
\newcommand{\tabref}[1]{Table~\ref{#1}}

%%%fancy proof env %%%
\def\qedsym{\hspace{1pt}\rule[-1pt]{3pt}{9pt}}

\newlength{\saveparindent}
\setlength{\saveparindent}{\parindent}
\newlength{\saveparskip}
\setlength{\saveparskip}{\parskip}


\def\qsym{\vrule width0.6ex height1em depth0ex}
\newcount\proofqeded
\newcount\proofended
\def\qed{{\hspace{1pt}\rule[-1pt]{3pt}{9pt}}
\end{rm}\addtolength{\parskip}{-0pt}
\setlength{\parindent}{\saveparindent}
\global\advance\proofqeded by 1 }
\def\qedenv{
\end{rm}\addtolength{\parskip}{-0pt}
\setlength{\parindent}{\saveparindent}
\global\advance\proofqeded by 1 }
\renewenvironment{proof}%
 {\proofstart}%
 {\ifnum\proofqeded=\proofended~\qed\fi \global\advance\proofended by 1
  \medskip}
\newenvironment{proofenv}%
 {\proofenvstart}%
 {\ifnum\proofqeded=\proofended\qedenv\fi \global\advance\proofended by 1
  \medskip}
\makeatletter
\def\proofstart{\@ifnextchar[{\@oprf}{\@nprf}}
\def\proofenvstart{\@ifnextchar[{\@osprf}{\@nsprf}}
\def\@oprf[#1]{\begin{rm}\protect\vspace{6pt}\noindent{\bf Proof of #1:\ }%
\addtolength{\parskip}{5pt}\setlength{\parindent}{0pt}}
\def\@osprf[#1]{\begin{rm}\protect\vspace{6pt}\noindent
\addtolength{\parskip}{5pt}\setlength{\parindent}{0pt}}
\def\@nprf{\begin{rm}\protect\vspace{6pt}\noindent{\bf Proof:\ }%
\addtolength{\parskip}{5pt}\setlength{\parindent}{0pt}}
\def\@nsprf{\begin{rm}\protect\vspace{6pt}\noindent%
\addtolength{\parskip}{5pt}\setlength{\parindent}{0pt}}


% ========================================================================

% Lists

\newcounter{ctr}
\newenvironment{bignewenum}{%
\begin{list}{{\bf (\arabic{ctr})}\hfill}{\usecounter{ctr} \labelwidth=17pt%
\labelsep=7pt \leftmargin=24pt \topsep=4pt%
\setlength{\listparindent}{\saveparindent}%
\setlength{\parsep}{\saveparskip}%
\setlength{\itemsep}{4pt} }}{\end{list}}


\newenvironment{newmath}{\begin{displaymath}%
\setlength{\abovedisplayskip}{4pt}%
\setlength{\belowdisplayskip}{4pt}%
\setlength{\abovedisplayshortskip}{5pt}%
\setlength{\belowdisplayshortskip}{5pt} }{\end{displaymath}}

% =========================================================================


%\makeatletter
%\def\appearsin#1{\gdef\@appearsin{#1}}
%\def\maketitle{\par
% \begingroup
% \def\thefootnote{\arabic{footnote}}
% \def\@makefnmark{\hbox
% to 0pt{$^{\@thefnmark}$\hss}}
% \if@twocolumn
% \twocolumn[\@maketitle]
% \else \newpage
% \global\@topnum\z@ \@maketitle \fi\thispagestyle{plain}\@thanks
% \endgroup
% \setcounter{footnote}{0}
% \let\maketitle\relax
% \let\@maketitle\relax
% \gdef\@thanks{}\gdef\@author{}\gdef\@title{}\gdef\@appearsin{}
%          \let\thanks\relax}
%\def\@maketitle{\newpage
% \noindent \@appearsin
% \vskip 0.5in \begin{center}
% {\LARGE \@title \par} \vskip 1.5em {\large \lineskip .5em
%\begin{tabular}[t]{c}\@author
% \end{tabular}\par}
% \vskip 1em {\normalsize \@date} \end{center}
% \par
% \vskip 1.5em}
%\def\abstract{\if@twocolumn
%\section*{Abstract}
%\else \small
%\begin{center}
%{\bf Abstract\vspace{-.5em}\vspace{0pt}}
%\end{center}
%\quotation
%\fi}
%\def\endabstract{\if@twocolumn\else\endquotation\fi}
%\mark{{}{}}
%


% =========================================================================

% General
%\newcommand{\hindent}{\quad}

\newcommand{\mysubsubsection}[1]{\subsubsection{#1}}
\newcommand{\headingb}[1]{{\vspace{5pt}\noindent\textbf{#1.}}}
\newcommand{\headingg}[1]{{\sc{#1}}}
\newcommand{\heading}[1]{{\vspace{5pt}\noindent\sc{#1}}}

\newcommand{\tuple}[1]{\langle{#1}\rangle}
\def\from{\mbox{from}\ }
\def\From{\mbox{From}\ }
\def\bits{\{0,1\}}
\def\cross{\times}
\newcommand{\cab}{,\allowbreak}
\newcommand{\xor}{\oplus}
\newcommand{\Colon}{{:\;\;}}
\def\emptystring{\varepsilon}

\newcommand{\OL}[0]{\mathsf{OL}}

\newcommand{\Dom}{\mathsf{Dom}}
\newcommand{\DomR}{\mathsf{DomR}}
\newcommand{\Rng}{\mathsf{Rng}}
\newcommand{\RngR}{\mathsf{RngR}}
\newcommand{\calA}{{\cal A}}
\newcommand{\calB}{{\cal B}}
\newcommand{\calC}{{\cal C}}
\newcommand{\calF}{{\cal F}}
\newcommand{\N}{{{\mathds N}}}
\newcommand{\Z}{{{\mathds Z}}}
\newcommand{\R}{{{\mathds R}}}
\def\union{\cup}
\def\intersection{\cap}
\newcommand{\set}[2]{\{#1 \,:\, #2\}}
\newcommand{\getsr}{{{\,\leftarrow{\hspace*{-3pt}\raisebox{.75pt}{$\scriptscriptstyle\$$}}\,}}}
\newcommand{\OR}{{\textstyle\bigvee}}
\newcommand{\Var}{{\mbox{\bf Var}}}
\newcommand{\E}{\mathbf{E}}
\def\e{{\epsilon}}
\newcommand{\concat}{{\,\|\,}}
% ========================================================================

% ===================================================================
\newcommand{\comment}[1]{\hspace{8pt}{\small /$\!\!$/\ #1}}
\newcommand{\Comment}[1]{\hspace{5pt}{/$\!\!$/\ #1}}
\newcommand{\ccomment}[1]{\hspace{1pt}{/$\!\!$/\ #1}}

\newcommand{\authnote}[2]{ \begin{center}\fbox{\begin{minipage}{5.7in}
\textbf{#1 says:} #2\end{minipage}}\end{center}}

\newcommand{\nGame}[2]{\mathbf{G}^{\mathrm{#1}}_{#2}}
\newcommand{\pk}{pk}
\newcommand{\dk}{dk}
\newcommand{\FSOutSet}[0]{\mathrm{Out}}

\newcommand{\schemefont}[1]{{\mathsf{#1}}}
\newcommand{\algfont}[1]{{\mathrm{#1}}}
\newcommand{\Adv}{\mathbf{Adv}}
\newcommand{\genAdv}[3]{\mathbf{Adv}^{\mathrm{#1}}_{#2}(#3)}
\newcommand{\prfAdv}[1]{\genAdv{prf}{#1}}

\newcommand{\advA}{\mathcal{A}}
\newcommand{\advB}{\mathcal{B}}
\newcommand{\advC}{\mathcal{C}}
\newcommand{\advD}{\mathcal{D}}
\newcommand{\INDCCA}{\mbox{IND-CCA}\xspace}

\newcommand{\atk}{\mathrm{atk}}
\newcommand{\cpa}{\mathrm{cpa}}
\newcommand{\cca}{\mathrm{cca}}
\newcommand{\ind}{\mathrm{ind}}

\addtolength{\belowcaptionskip}{-2mm}
\addtolength{\abovecaptionskip}{-3mm}
\addtolength{\topsep}{-1mm}
\newcommand{\schalg}[2]{\mathsf{#1.#2}}


\newcommand{\procfont}[1]{\mathsc{#1}}
\newcommand{\tablefont}[1]{\mathsf{#1}}
\newcommand{\Initialize}{\Init}
\newcommand{\Finalize}{\Fin}
\newcommand{\Init}{\procfont{init}}
\newcommand{\Fin}{\procfont{fin}}

\newcommand{\RO}{\procfont{RO}}
\newcommand{\ROSim}{\procfont{ROSim}}
\newcommand{\FUNC}{\procfont{FUNC}}
\newcommand{\DecO}{\procfont{Dec}}

\newcommand{\win}{\mathsf{win}}

\newcommand{\Fimage}[0]{\mathrm{Im}}
\newcommand{\simulator}[0]{\mathsf{Sim}}
\newcommand{\Gensimulator}[1]{\mathsf{S}_{#1}}


\newcommand{\group}{\mathds{G}}
\newcommand{\wins}{\textrm{ wins}}
\newcommand{\true}{\mathsf{true}}
\newcommand{\false}{\mathsf{false}}
\newcommand{\AND}{\;\wedge\;}
\newcommand{\bad}{\mathsf{bad}}
\newcommand{\chal}{\mathsf{chal}}
\newcommand{\Good}{\mathsf{Good}}
\newcommand{\Gm}{\textnormal{G}}
\newcommand{\fn}{\footnotesize}

\newcommand{\vecx}{\mathbf{x}}
\newcommand{\pfvec}[0]{\mathbf{p}}
\newcommand{\vecY}{\mathslbf{Y}}
\newcommand{\vecV}{\mathslbf{V}}
\newcommand{\vecU}{\mathslbf{U}}
\newcommand{\vecW}{\mathslbf{W}}
\newcommand{\vecxx}[0]{\mathbf{x}}
\newcommand{\Oracle}{\mathsc{O}}
\newcommand{\pkeScheme}{\schemefont{PKE}}
\newcommand{\pkeEnc}{\schalg{\pkeScheme}{E}}
\newcommand{\pkeDecR}{\schalg{\pkeScheme}{DecR}}
\newcommand{\pkeML}{\schalg{\pkeScheme}{ml}}
\newcommand{\prf}{\mathsf{G}}
\newcommand{\prfKl}{\prf.\mathsf{kl}}
%
\newcommand{\pkeToKem}[0]{\mathbf{T}}
\newcommand{\QpkeToKem}[0]{\mathbf{T}}
\newcommand{\pkeCiph}[0]{C}
\newcommand{\kemCiph}[0]{C^*}
%
\newcommand{\kemScheme}{\schemefont{KE}}
\newcommand{\kemKg}{\schalg{\kemScheme}{K}}
\newcommand{\kemEnc}{\schalg{\kemScheme}{E}}
\newcommand{\kemDec}{\schalg{\kemScheme}{D}}
\newcommand{\kemRl}{\schalg{\kemScheme}{rl}}
\newcommand{\kemKl}{\schalg{\kemScheme}{kl}}
\newcommand{\kemRoSp}{\schalg{\kemScheme}{FS}}

\newcommand{\FkemScheme}{\overline{\mathsf{KE}}}
\newcommand{\FkemRoSp}{\schalg{\FkemScheme}{FS}}
\newcommand{\FkemEnc}{\schalg{\FkemScheme}{E}}
%
\newcommand{\GkemScheme}[1]{\schemefont{KE}_{#1}}
\newcommand{\GkemEnc}[1]{\schalg{\GkemScheme{#1}}{E}}
%
\newcommand{\DFunc}[0]{\mathsf{D}}
\newcommand{\KLenFunc}[0]{\mathsf{k}^*}
%

\newcommand{\advQINDIFF}[2]{\genAdv{QINDIFF}{#1}{#2}}
\newcommand{\advTINDIFF}[2]{\genAdv{TINDIFF}{#1}{#2}}
\newcommand{\advINDIFF}[2]{\genAdv{indiff}{#1}{#2}}
\newcommand{\advRINDIFF}[2]{\genAdv{reset\text{-}indiff}{#1}{#2}}

\newcommand{\Func}[1][]{\mathcal{F}_{#1}}
\newcommand{\FuncHon}[1][]{\mathcal{F}_{#1}.hon}
\newcommand{\FuncAdv}[1][]{\mathcal{F}_{#1}.adv}

\newcommand{\lazysample}{\mathsf{ls}}

\newcommand{\secPropPKE}[0]{\mathrm{S}_{\mathrm{pke}}}


%
% Game hops
%
\newcommand{\gamehopchange}[2][0.85]{\hspace{0pt}\smash{\fbox{\rule[-1.75pt]{0pt}{#1\baselineskip}\smash{#2}}}} % frame around changes from one game to another, use optional parameter to adapt box height (as factor of \baselineskip), leading \hspace{0pt} prevents weird jump if nothing before \gamehopchange{..}

\newcommand{\supportQuT}[1]{\mathbf{sup}(#1)}
\newcommand{\algOutput}[0]{\mathrm{OUT}}
%
% FREE / DOMSEP predicates
%
\newcommand{\aPrefix}{P}
\newcommand{\prefix}{\preceq}
\newcommand{\listRO}{\mathcal{L}}

\newcommand{\gameFREE}[4]{\mGame{#3,FREE}{#4,#1,#2}{\advA}}
\newcommand{\gameFIXED}[4]{\mGame{#3,FIXED}{#4,#1,#2}{\advA}}
\newcommand{\gameDOMSEP}[5]{\mGame{#2,#3,DOMSEP}{#4,#5,#1}{\advA}}

\newcommand{\ngameCORR}[1]{\nGame{CORR}{#1}}
\newcommand{\ngamePRF}[1]{\nGame{prf}{#1}}

\newcommand{\ngameINDCCA}[1]{\nGame{ind-cca}{#1}}
\newcommand{\ngameINDCPA}[1]{\nGame{INDCPA}{#1}}
\newcommand{\ngameOWCPA}[1]{\nGame{OWCPA}{#1}}
\newcommand{\ngameOWPCA}[1]{\nGame{OWPCA}{#1}}
\newcommand{\ngameDOM}[1]{\nGame{wdom}{#1}}
\newcommand{\ngameQINDIFF}[1]{\nGame{QINDIFF}{#1}}
\newcommand{\ngameTINDIFF}[2]{\nGame{TINDIFF-{#1}}{#2}}

\newcommand{\DRTransform}[0]{\mathbf{DR}}
\newcommand{\RTransform}[0]{\mathbf{R}}
\newcommand{\InsTransform}[0]{\mathbf{Inst}}

\newcommand{\Aset}[2]{\{#1\,:\,#2\}}
\newcommand{\domain}[0]{\mathcal{D}}
\newcommand{\rangeSet}[0]{\mathcal{R}}
\newcommand{\domainFunc}[0]{\mathcal{F}}
\newcommand{\usedDomain}[0]{\mathcal{U}}
\newcommand{\SchemeQuery}{\mathsf{sq}}
\newcommand{\restrict}[2]{#1|_{#2}}

\newcommand{\pfFunctor}[1]{\construct{F}_{\mathrm{pf}(#1)}}
\newcommand{\ldFunctor}[0]{\construct{F}_{\mathrm{ld}}}
\newcommand{\idFunctor}[0]{\construct{F}_{\mathrm{id}}}
\newcommand{\splFunctor}[0]{\construct{F}_{\mathrm{spl}}}

\newcommand{\queryRO}[0]{\mathbf{Q}}
\newcommand{\GenqueryRO}[1]{\mathbf{#1}}
\newcommand{\FixedprefixqueryRO}[1]{\QuT_{\mathrm{pf}(#1)}}
\newcommand{\FixedprefixanswerRO}[1]{\AnT_{\mathrm{pf}(#1)}}
\newcommand{\FixedprefixqueryInv}[1]{\QuTInv_{\mathrm{pf}(#1)}}
\newcommand{\FixedprefixanswerInv}[1]{\AnTInv_{\mathrm{pf}(#1)}}

\newcommand{\VarprefixqueryRO}{\GenqueryRO{QPv}}
\newcommand{\LengthqueryRO}{\QuT_{\mathrm{ld}}}
\newcommand{\LengthanswerRO}{\AnT_{\mathrm{ld}}}
\newcommand{\IdqueryRO}{\QuT_{\mathrm{id}}}
\newcommand{\IdanswerRO}{\AnT_{\mathrm{id}}}
\newcommand{\IdqueryInv}{\QuTInv_{\mathrm{id}}}
\newcommand{\IdanswerInv}{\AnTInv_{\mathrm{id}}}
\newcommand{\SplittingqueryRO}{\QuT_{\mathrm{spl}}}
\newcommand{\SplittingqueryInv}{\QuTInv_{\mathrm{spl}}}
\newcommand{\SplittinganswerRO}{\AnT_{\mathrm{spl}}}
\newcommand{\SplittinganswerInv}{\AnTInv_{\mathrm{spl}}}
\newcommand{\answerRO}{\GenqueryRO{AT}}
\newcommand{\answerInv}{\answerRO^{-1}}
\newcommand{\NIntrusivequeryRO}{\GenqueryRO{QNI}}
\newcommand{\IntrusivequeryRO}{\GenqueryRO{QI}}
\newcommand{\AddgqueryRO}[1]{\GenqueryRO{QAGx}[#1]}
\newcommand{\IDqueryRO}{\GenqueryRO{QId}}

\newcommand{\Outputsplitting}{\mathbf{OS}}


\newcommand{\FComb}[1]{[[#1]]}
\newcommand{\aFunc}[1]{\mathit{#1}}
\newcommand{\aFuncDom}[0]{\mathsf{Dom}}
\newcommand{\aFuncRng}[0]{\mathsf{Rng}}

\newcommand{\bFunc}[1]{\mathrm{#1}}
\newcommand{\bFuncDom}[0]{\mathsf{Dom}}
\newcommand{\bFuncRng}[0]{\mathsf{Rng}}

\newcommand{\encode}[1]{\langle #1\rangle}
\newcommand{\Encode}{\texttt{Encode}}


%\newcommand{\AllFuncs}[2]{\llbracket #1\rightarrow#2\rrbracket}
\newcommand{\AllFuncs}[2]{\mathrm{FUNC}(#1,#2)}
\newcommand{\AllSOLFuncs}[2]{\mathrm{SOL}(#1,#2)}
\newcommand{\AllXOLFuncs}[1]{\mathrm{XOL}(#1)}
\newcommand{\FuncSp}[1]{\mathsf{#1}}
\newcommand{\aRO}[0]{\aFunc{F}}
\newcommand{\roSp}[0]{\mathsf{FS}}
\newcommand{\GenroSp}[1]{\mathsf{#1}}
\newcommand{\GenroSpCardinality}[1]{\GenroSp{#1}.\mathsf{n}}
\newcommand{\GenroSpFuncs}[1]{\GenroSp{#1}.\mathsf{F}}
\newcommand{\GenroSpDom}[1]{\mathrm{Dom}(\GenroSp{#1})}
\newcommand{\GenroSpDomP}[1]{\mathrm{Dom}_{*}(\GenroSp{#1})}
\newcommand{\GenroSpRng}[1]{\mathrm{Rng}(\GenroSp{#1})}
\newcommand{\roSpCardinality}[0]{\roSp.\mathsf{n}}
\newcommand{\roSpFuncs}[0]{\roSp.\mathsf{Fncs}}
\newcommand{\roSpDom}[0]{\mathrm{Dom}(\roSp)}
\newcommand{\roSpRng}[0]{\mathrm{Rng}(\roSp)}
\newcommand{\GGenroSpCardinality}[1]{{#1}.\mathsf{n}}
\newcommand{\GGenroSpFuncs}[1]{{#1}.\mathsf{Fncs}}
\newcommand{\GGenroSpDom}[1]{\mathrm{Dom}({#1})}
\newcommand{\GGenroSpDomP}[1]{\mathrm{Dom}_{*}({#1})}
\newcommand{\GGenroSpRng}[1]{\mathrm{Rng}({#1})}

\newcommand{\appendRO}[2]{{#1}\!\!+\!\!{#2}}
\newcommand{\chopRO}[1]{\mathrm{Chop}(#1)}
\newcommand{\addRO}[1]{\mathrm{Add}(#1)}
\newcommand{\expandRO}[1]{\mathrm{Expand}(#1)}
\newcommand{\compressRO}[1]{\mathrm{Compress}(#1)}

%\newcommand{\SHA}[1]{\mathsf{SHA#1}}
\newcommand{\SHAA}[2]{\mathsf{SHA#1}\mbox{-}\mathsf{#2}}
\newcommand{\SHAKE}[1]{\mathsf{SHAKE#1}}
\newcommand{\cSHAKE}{\mathsf{cSHAKE}}

\newcommand{\ulheading}[1]{\medskip\noindent\textsf{\underline{\smash{#1}}}}
\newcommand{\uldheading}[2]{\medskip\noindent\textsf{\underline{\smash{#1:}} #2}}

% \newcommand{\pqcname}[1]{\texttt{#1}}
% \newcommand{\pqcheading}[1]{\medskip\noindent\underline{\smash{\pqcname{#1}}}}
\newcommand{\pqcnameRoundOne}[1]{\texttt{\color{darkgray}#1}}
\newcommand{\pqcnameRoundTwo}[1]{\texttt{\fontseries{b}\selectfont#1}}

\newcommand{\functionOut}{e}
\newcommand{\functionIn}{s}
\newcommand{\functionInSet}{SS}
\newcommand{\functionOutSet}{ES}

\newcommand{\construct}[1]{\mathbf{#1}}
\newcommand{\constructDom}[1]{\construct{#1}.\mathsf{Dom}}
\newcommand{\constructEv}[1]{\construct{#1}.\mathsf{Ev}}
\newcommand{\constructRng}[1]{\construct{#1}.\mathsf{Rng}}

\newcommand{\commoncoins}{st}
\newcommand{\ccell}{st\ell}
\newcommand{\SimgenCC}{\simulator.\algfont{Setup}}
\newcommand{\Simeval}{\simulator.\algfont{Ev}}
\newcommand{\GenSimgenCC}[1]{\Gensimulator{#1}.\algfont{Setup}}
\newcommand{\GenSimeval}[1]{\Gensimulator{#1}.\algfont{Ev}}
\newcommand{\ccsample}{\mathsf{sample}}

\newcommand{\UHF}{H} % universal hash function

\newcommand{\functionality}{\mathfrak{f}}
\newcommand{\functionalityClass}{\mathfrak{F}}
\newcommand{\priv}{\procfont{priv}}
\newcommand{\pub}{\procfont{pub}}
\newcommand{\FnO}{\procfont{FnO}}
\newcommand{\functionalityPriv}{\functionality_{\priv}}
\newcommand{\functionalityPub}{\functionality_{\pub}}


\newcommand{\ngameCCINDIFF}[1]{\nGame{rd\mbox{-}indiff}{#1}}
\newcommand{\AdvCCINDIFF}[2]{\genAdv{rd\mbox{-}indiff}{#1}{#2}}
\newcommand{\indccaAdv}[2]{\genAdv{ind\mbox{-}cca}{#1}{#2}}
\newcommand{\wdomAdv}[2]{\genAdv{wdom}{#1}{#2}}


\newcommand{\ngameTI}[1]{\nGame{ti}{#1}}
\newcommand{\AdvTI}[2]{\genAdv{ti}{#1}{#2}}

\newcommand{\ngameEXEC}[1]{\nGame{exec}{#1}}

\newcommand{\ngameMSReal}[1]{\nGame{ms\mbox{-}1}{#1}}
\newcommand{\ngameMSIdeal}[1]{\nGame{ms\mbox{-}0}{#1}}
\newcommand{\ngameMS}[1]{\nGame{ms}{#1}}

\newcommand{\env}{\algfont{FG}}

\newcommand{\execoracle}{OR}
\newcommand{\execfinalize}{\mathsf{Finalize}}

\newcommand{\workDom}[0]{{\cal W}}
\newcommand{\WDInv}[0]{\mathrm{In}}

\newcommand{\QuT}[0]{\mathsf{QT}}
\newcommand{\AnT}[0]{\mathsf{AT}}
\newcommand{\QuTInv}[0]{\mathsf{QTI}}
\newcommand{\AnTInv}[0]{\mathsf{ATI}}

\newcommand{\Codewords}{\mathcal{C}}

\newcommand{\lncsorfull}[1]{{#1}}
\DeclareMathVersion{normal1}
%
%% LNCS format modifiers
%\makeatletter
%\newcommand{\lncsonly}{\relax\iffull\expandafter\@gobble\else\expandafter\@firstofone\fi}
%\newcommand{\lncselse}{\relax\iffull\expandafter\@secondoftwo\else\expandafter\@firstoftwo\fi}
\newcommand{\fullonly}{\relax\iffull\expandafter\@firstofone\else\expandafter\@gobble\fi}
\newcommand{\fullelse}{\relax\iffull\expandafter\@firstoftwo\else\expandafter\@secondoftwo\fi}
%\newcommand{\nfullonly}{\relax\iffull\expandafter\@gobble\else\expandafter\@firstofone\fi}
%\makeatother
%\newcommand{\lncsbreak}{\lncsonly{\allowbreak}}
%\newcommand{\lncsforcebreak}{\lncsonly{\linebreak}}
%\newcommand{\lncsnewline}{\lncsonly{\newline}}
%\newcommand{\lncsdot}{\lncsonly{.}}  % lncs-only ending dot for inline headings (e.g., \subsubsection)

%
% Formatting modifiers
%
\newcommand{\shortlongeqn}[2][]{\[ #2 #1 \]}
\newcommand{\lightparagraph}[1]{\paragraph{#1}}


%
% Fonts
%

\DeclareMathAlphabet{\mathsc}{OT1}{cmr}{m}{sc}


%
% Colors
%
\definecolor{darkblue}{rgb}{0,0,0.5}
\definecolor{darkgreen}{rgb}{0,0.5,0}
\definecolor{darkred}{rgb}{0.5,0,0}

\definecolor{tudred}{RGB}{230,0,26}
\definecolor{tudgreen}{RGB}{153,192,0}
\definecolor{tudorange}{RGB}{245,163,0}
\definecolor{tudblue}{RGB}{0,104,157}
\definecolor{tudpurple}{RGB}{149,17,105}
\definecolor{tudbrown}{RGB}{169,73,19}

%
% Basics
%
\newcommand{\minus}{\text{\normalfont-}}

\newcommand{\sample}{\xleftarrow{\smash{\raisebox{-1.75pt}{$\scriptscriptstyle\$$}}}}
\newcommand{\tor}{\xrightarrow{\smash{\raisebox{-1.75pt}{$\scriptscriptstyle\$$}}}}

\newcommand{\conc}{\|}

%\newcommand{\comment}[1]{\textcolor{gray}{\scriptsize/\!\!/\,#1}}

%
% Word macros
%

% no word-break in n-RTT
\newcommand{\ZRTT}{\mbox{0-RTT}\xspace}
\newcommand{\OneRTT}{\mbox{1-RTT}\xspace}

\newcommand{\SIGMA}{\mbox{SIGMA}\xspace}
\newcommand{\SIGMAI}{\mbox{SIGMA-I}\xspace}

% math-mode macros (in theorems)
\newcommand{\mSIGMAI}{\mathrm{SIGMA{\minus}I}}
\newcommand{\mTLS}{\mathrm{TLS\,1.3}}


%
% Redefine standard commands
%

% \paragraph, \subparagraph: automatically append a dot after paragraph titles
\let\originalparagraph\paragraph
\renewcommand{\paragraph}[1]{%
  \ifthenelse{\endswith{#1}{.}}{%
    \originalparagraph{#1}%
  }{%
    \originalparagraph[#1]{#1.}%
  }%
}

\let\originalsubparagraph\subparagraph
\renewcommand{\subparagraph}[1]{%
  \ifthenelse{\endswith{#1}{.}}{%
    \originalsubparagraph{#1}%
  }{%
    \originalsubparagraph[#1]{#1.}%
  }%
}


%
% Definitions, Theorems, etc.
% (already present in LNCS)
%
%
%\iffull
%\theoremstyle{plain}
%\newtheorem{theorem}{Theorem}[section]
%\newtheorem{lemma}[theorem]{Lemma}
%\newtheorem{proposition}[theorem]{Proposition}
%\newtheorem{corollary}[theorem]{Corollary}
%
 \theoremstyle{definition}
\newtheorem{definition}{Definition}
%\newtheorem{construction}[theorem]{Construction}
%
%\theoremstyle{remark}
%\newtheorem{remark}[theorem]{Remark}
%\newtheorem{example}[theorem]{Example}
%\fi



\newcommand{\NN}{\mathds{N}}
\newcommand{\ZZ}{\mathds{Z}}

%\newcommand{\group}{\mathbb{G}}

\newcommand{\gamefont}[1]{\mathrm{#1}}
%\newcommand{\procfont}[1]{\textsc{#1}}
\newcommand{\setfont}[1]{\mathcal{#1}}

%\newcommand{\Adv}{\mathsf{Adv}}
% \newcommand{\genAdv}[3]{\Adv^{\mathsf{#1}}_{#2}(#3)}
%\newcommand{\genAdv}[3]{\Adv^{\mathsf{#1}}_{#2,#3}}
\newcommand{\genAdvA}[2]{\genAdv{#1}{#2}{\advA}}
%\newcommand{\bad}[1][]{\mathsf{bad}_{#1}}
%\newcommand{\Gm}{\gamefont{G}}
%\newcommand{\advA}{\mathcal{A}}
%\newcommand{\advB}{\mathcal{B}}


% ----- Colors -----
\newcommand{\gray}[1]{{\color{gray}#1}}
\newcommand{\grey}{\gray}

% ----- Highlights, etc. -----


%\definecolor{highlight-gray}{gray}{0.90}

%\newcommand{\gamechange}[1][highlight-gray]{\setlength{\fboxsep}{0pt}\colorbox{#1}}

%
% Macro styles
%

\newcommand{\algostyle}[1]{\mathsf{#1}} % algorithm
\newcommand{\predstyle}[1]{\mathsf{#1}} % predicate
\newcommand{\notionstyle}[1]{\mathsf{#1}} % security notion
\newcommand{\oraclestyle}[1]{\mathsc{#1}} % oracle

\newcommand{\protvarstyle}[1]{#1} % protocol variable
\newcommand{\protvalstyle}[1]{\mathsf{#1}} % protocol value
\newcommand{\gamevarstyle}[1]{\mathsf{#1}} % game variable

\newcommand{\querynum}[1]{q_{\oraclestyle{#1}}} % number of queries to oracle #1 (latter may be abbreviated)


%
% Components
%

\newcommand{\New}{\oraclestyle{New}} % multi-user new key query
\newcommand{\qNew}{\querynum{Nw}}
\newcommand{\Corrupt}{\oraclestyle{Corrupt}} % multi-user corrupt query
\newcommand{\qCorrupt}{\querynum{C}}

% PRFs
\newcommand{\PRF}{\algostyle{PRF}}
\newcommand{\muPRFSEC}{\notionstyle{mu{\minus}PRF}}
\newcommand{\PRFfn}{\oraclestyle{Fn}} % PRF function oracle
\newcommand{\qPRFfn}{\querynum{Fn}}
\newcommand{\qPRFfnU}{\querynum{Fn/U}}
%\newcommand{\FUNC}{\mathsf{FUNC}} % function space

% input labels / constants to PRF etc.
\newcommand{\inputlabel}[1][]{\mathtt{L}_{#1}}
\newcommand{\constant}[1][]{\mathtt{C}_{#1}}

% Hash
\newcommand{\Hash}{\algostyle{H}} % the hash function
\newcommand{\COLL}{\notionstyle{CR}} % collision resistance

% Random oracle
%\newcommand{\RO}{\algostyle{RO}}
\newcommand{\qRO}{\querynum{RO}} % number of RO queries

%Encryption
\newcommand{\ENCScheme}{\algostyle{E}}
\newcommand{\ENCEnc}{\algostyle{Enc}}
\newcommand{\ENCDec}{\algostyle{Dec}}

% Signatures
\newcommand{\SIGScheme}{\algostyle{S}}
\newcommand{\SIGKGen}{\algostyle{KGen}}
\newcommand{\SIGSign}{\algostyle{Sign}}
\newcommand{\SIGVerify}{\algostyle{Vrfy}}
\newcommand{\OSign}{\oraclestyle{Sign}}
\newcommand{\qSign}{\querynum{Sg}}
\newcommand{\qSignU}{\querynum{Sg/U}}

% MACs
\newcommand{\MACScheme}{\algostyle{M}}
\newcommand{\MACKGen}{\algostyle{KGen}}
\newcommand{\MACTag}{\algostyle{Tag}}
\newcommand{\MACVerify}{\algostyle{Vrfy}}
\newcommand{\OTag}{\oraclestyle{Tag}}
\newcommand{\qTag}{\querynum{Tg}}
\newcommand{\qTagU}{\querynum{Tg/U}}
\newcommand{\OVerify}{\oraclestyle{Vrfy}}
\newcommand{\qVerify}{\querynum{V\!f}}
\newcommand{\qVerifyU}{\querynum{V\!f/U}}
\newcommand{\EUFCMA}{\notionstyle{EUF{\minus}CMA}}
\newcommand{\muEUFCMA}{\notionstyle{mu{\minus}EUF{\minus}CMA}}

% HKDF
\newcommand{\HKDF}{\algostyle{HKDF}}
\newcommand{\HMAC}{\algostyle{HMAC}}
\newcommand{\Extract}{\algostyle{Extract}} % HKDF Extract function
\newcommand{\Expand}{\algostyle{Expand}} % HKDF Expand function
\newcommand{\sExtract}{\algostyle{Ext}} % short HKDF Extract function
\newcommand{\sExpand}{\algostyle{Exp}} % short HKDF Expand function

% Strong Diffie--Hellman
\newcommand{\strongDH}{\notionstyle{stDH}} % strongDH assumption
\newcommand{\DDH}{\algostyle{DDH}} % DDH oracle
\newcommand{\qstDH}{\querynum{sDH}} % number of DDH/stDH oracle queries

%
% Key Exchange
%

\newcommand{\KE}{\algostyle{KE}}
\newcommand{\KEKGen}{\algostyle{KGen}}
\newcommand{\KEActivate}{\algostyle{Activate}}
\newcommand{\KERun}{\algostyle{Run}}
\newcommand{\KEkeyspace}{\KE.\algostyle{KS}}
\newcommand{\TLS}{\algostyle{TLS}}

% protocol variables
\def\PROTOCOLVARIABLES{pk,sk,id,peerid,peerpk,role,st,sid,status,skey,state,X,Y,W,N,Recv,E,x,y,r,kl,mk}
\foreach \protvar in \PROTOCOLVARIABLES {%
	\expandafter\xdef\csname\protvar\endcsname{\noexpand\protvarstyle{\protvar}}
}
% some more, manually
\newcommand{\nonce}{\protvarstyle{n}}
\newcommand{\ol}{\protvarstyle{ol}}
\newcommand{\nl}{\protvarstyle{nl}}

% protocol values
\def\PROTOCOLVALUES{running,accepted,rejected,initiator,responder}
\foreach \protval in \PROTOCOLVALUES {%
	\expandafter\xdef\csname\protval\endcsname{\noexpand\protvalstyle{\protval}}
}

\newcommand{\labelis}{\inputlabel[is]}
\newcommand{\labelim}{\inputlabel[im]}
\newcommand{\labelrs}{\inputlabel[rs]}
\newcommand{\labelrm}{\inputlabel[rm]}
\newcommand{\ks}{k_s}
\newcommand{\kt}{k_t}
\newcommand{\ke}{k_e}
\newcommand{\Zz}{Z}
\newcommand{\ciph}{c}
\newcommand{\sidhash}{d}

% StrongDH
\newcommand{\stDH}{\mathsf{stDH}} % strongDH oracle

%
% SIGMA
%
\newcommand{\RunInitI}{\algostyle{RunInit1}}
\newcommand{\RunInit}{\algostyle{RunInit2}}
\newcommand{\RunRespI}{\algostyle{RunResp1}}
\newcommand{\RunRespII}{\algostyle{RunResp2}}


%
% Key Exchange Model
%

\newcommand{\KESEC}{\notionstyle{KE{\minus}SEC}}

% Oracles
%\newcommand{\Initialize}{\oraclestyle{Initialize}}
%\newcommand{\Finalize}{\oraclestyle{Finalize}}
\newcommand{\Send}{\oraclestyle{Send}}
\newcommand{\NewUser}{\oraclestyle{NewUser}}
\newcommand{\RevSessionKey}{\oraclestyle{RevSessionKey}}
\newcommand{\RevLongTermKey}{\oraclestyle{RevLongTermKey}}
\newcommand{\Test}{\oraclestyle{Test}}

% number of queries
\newcommand{\qNewUser}{\querynum{N}}
\newcommand{\qSend}{\querynum{S}}
\newcommand{\qRevSessionKey}{\querynum{RS}}
\newcommand{\qRevLongTermKey}{\querynum{RL}}
\newcommand{\qTest}{\querynum{T}}
\newcommand{\qRevState}{\querynum{RSt}}

% Predicates
\newcommand{\Sound}{\predstyle{Sound}}
\newcommand{\ExplicitAuth}{\predstyle{ExplicitAuth}}
\newcommand{\Fresh}{\predstyle{Fresh}}

% game variables
\def\GAMEVARIABLES{revealed,tested,revltk,time,users,S,Q,Sent}
\foreach \gamevar in \GAMEVARIABLES {%
	\expandafter\xdef\csname\gamevar\endcsname{\noexpand\gamevarstyle{\gamevar}}
}
% some more, manually
\newcommand{\taccepted}{\gamevarstyle{t_{acc}}}


%
% Game hop proofs, environments, counters
%

% initiate a proof, giving it a (reference) name
\newcommand{\startproof}[1]{%
	\xdef\currentproof{#1}
	\firstproofngame
}
\newcommand{\currentproof}{}

\newcounter{proofngame} % counter for games in proofs without multiple cases

\renewcommand*{\theHproofngame}{\currentproof.\theproofngame} % make proofngame counter references unique per proof (for hyperref)

\newcommand{\curgame}{}  % label of current game
\newcommand{\prevgame}{} % label of previous game

\newcommand{\firstproofngame}[1][0]{\setcounter{proofngame}{-1+#1}} % reset counter to be 0 next time
\newcommand{\proofngame}[1][]{%  %%% optional argument is used as \label
	\refstepcounter{proofngame}%
	\ifthenelse{\isempty{#1}}{}{%
		\label{game:\currentproof:#1}%
		\edef\prevgame{\curgame}%
		\edef\curgame{game:\currentproof:#1}%
	}%
	\proofparagraph{Game \theproofngame}
}
\newcommand{\proofngames}[2][]{%  %%% (second) argument is number of game steps, optional argument is used as \label for the last gamd
	\stepcounter{proofngame}
	\xdef\oldproofngame{\theproofngame}
	\addtocounter{proofngame}{#2} % add number of game steps
	\addtocounter{proofngame}{-2} % -2 for step before and \ref*step* counter
	\refstepcounter{proofngame}%
	\ifthenelse{\isempty{#1}}{}{%
		\label{game:\currentproof:#1}%
		\edef\prevgame{\curgame}%
		\edef\curgame{game:\currentproof:#1}%
	}%
	\proofparagraph{Games \oldproofngame--\theproofngame}
}

% spacing between proof paragraphs
\newcommand{\proofsep}{\iffull\bigskip\else\medskip\fi}

% proof paragraph style
\makeatletter
\newcommand\gobblepars{% removes \par after command, used to make \proofparagraph a run-in heading
    \@ifnextchar\par%
        {\expandafter\gobblepars\@gobble}%
        {}}
\newcommand{\proofparagraph}[1]{%
% 	\iffullversion\bigskip\else\smallskip\fi
% 	\noindent\textit{#1.}\hspace{1em}\@afterindentfalse\@afterheading\gobblepars} %% this seems to not indent the first paragraph following the paragraph start
	
	\proofsep
	\noindent\textbf{#1.}\hspace{1em}\gobblepars}
\makeatother

% macros for the current, previous, or labeled games
\makeatletter
\newcommand{\curGm}{\@ifstar{\@curGmNL}{\@curGmL}}
	\newcommand{\@curGmNL}{\Gm_{\ref*{\curgame}}} % without hyperlink
	\newcommand{\@curGmL}{\Gm_{\ref{\curgame}}} % with hyperlink
\newcommand{\prevGm}{\@ifstar{\@prevGmNL}{\@prevGmL}}
	\newcommand{\@prevGmNL}{\Gm_{\ref*{\prevgame}}} % without hyperlink
	\newcommand{\@prevGmL}{\Gm_{\ref{\prevgame}}} % with hyperlink
\newcommand{\lblGm}{\@ifstar{\@lblGmNL}{\@lblGmL}}
	\newcommand{\@lblGmNL}[1]{\Gm_{\ref*{game:\currentproof:#1}}} % without hyperlink
	\newcommand{\@lblGmL}[1]{\Gm_{\ref{game:\currentproof:#1}}} % with hyperlink
\makeatother

% macros for current and new \advB reduction
\newcommand{\curadvB}{\advB_{\arabic{advB-\currentproof}}} % current \advB_x (does not increment counter)
\newcommand{\newadvB}{\stepcounter{advB-\currentproof}\curbdv} % new \advB_x (increments counter)



%
% Protocol Figures
%

% For any new figure, you to set X coordinates for client, server, and arrows; and initial Y coordinate, e.g.:
% 	\edef\ClientX{0}
% 	\edef\ArrowLeft{0}
% 	\edef\ArrowRight{9}
% 	\edef\ServerX{9}
% 	\edef\Y{0}

% ClientAction and ServerAction
% Print a command executed by the client/server
% 1st argument: the text
\newcommand{\ClientAction}[2][]{
	\node[right,#1] at (\ClientX, \Y) {#2};
}
\newcommand{\ServerAction}[2][]{
	\node[left,#1] at (\ServerX, \Y) {#2};
}
\newcommand{\SharedAction}[2][]{
	\node[#1] at ($1/2*(\ClientX, \Y)+1/2*(\ServerX, \Y)$) {#2};
}

% ClientToServer and ServerToClient
% Draws a message flow from client-to-server or server-to-client, with text above and below
% 1st argument (optional): line type, default ->
% 2nd argument: text above
% 3rd argument: text below
% Example: \ClientToServer{$Y$}{}
% Example: \ClientToServer[<->,double]{$Y$}{over an encrypted channel}
\newcommand{\ClientToServer}[3][->]{
	\NextLine[0.5]
	\draw[>=latex, very thick, #1] (\ArrowLeft+1,\Y) -- node[above=-0.1] {#2} node[below] {#3} (\ArrowRight-1,\Y) ;
	\NextLine[0.25]
}
\newcommand{\ServerToClient}[3][->]{
	\NextLine[0.5]
	\draw[>=latex, very thick, #1] (\ArrowRight-1,\Y) -- node[above=-0.1] {#2} node[below] {#3} (\ArrowLeft+1,\Y) ;
	\NextLine[0.25]
}

% Spacing factor for NextLines
\iffull
\def\NextLineSpacing{0.55}
\else
\def\NextLineSpacing{0.45}
\fi

% NextLine
% 1st argument (optional): amount of spacing to increment by, default 1.0
% Example: \NextLine
% Example: \NextLine[1.5]
\newcommand{\NextLine}[1][1.0]{
	\pgfmathparse{\Y-\NextLineSpacing*#1}
	\edef\Y{\pgfmathresult}
}

%
% stage separator line
%
\newcommand{\StageSeparator}[1]{
	\draw[very thick,dotted,StageSeparatorColor] (\ArrowLeft,\Y-0.5*\NextLineSpacing) -- (\ArrowRight+0.15,\Y-0.5*\NextLineSpacing) node[right,anchor=west,font=\footnotesize] {stage~#1} ;
}

%
% accept key and stage
%
\newcommand{\AcceptStage}[2]{
	\SharedAction{{\color{StageSeparatorColor}\textbf{accept} #2}}
	\StageSeparator{#1}
}



%%%%%%%%%%%%%%%%%%%%%%%%%%%%%%%%%%%%%%%%%%%%%%%%%%%%%%%%%%%%%%%%%%%%%%%%%%%%%%%%%%%%%%%%%%
% TLS
%%%%%%%%%%%%%%%%%%%%%%%%%%%%%%%%%%%%%%%%%%%%%%%%%%%%%%%%%%%%%%%%%%%%%%%%%%%%%%%%%%%%%%%%%%


%
% TLS messages
%

% long versions
\newcommand{\CHELO}{\mathtt{ClientHello}}
\newcommand{\SHELO}{\mathtt{ServerHello}}
\newcommand{\CKEYS}{\mathtt{ClientKeyShare}}
\newcommand{\SKEYS}{\mathtt{ServerKeyShare}}
%\newcommand{\CPSK}{\mathtt{ClientPreSharedKey}}
%\newcommand{\SPSK}{\mathtt{ServerPreSharedKey}}
\newcommand{\ENCEX}{\mathtt{EncryptedExtensions}}
\newcommand{\CERTR}{\mathtt{CertificateRequest}}
\newcommand{\SCERT}{\mathtt{ServerCertificate}}
\newcommand{\SCERTV}{\mathtt{ServerCertificateVerify}}
\newcommand{\CCERT}{\mathtt{ClientCertificate}}
\newcommand{\CCERTV}{\mathtt{ClientCertificateVerify}}
\newcommand{\SFIN}{\mathtt{ServerFinished}}
\newcommand{\CFIN}{\mathtt{ClientFinished}}

% common names
\newcommand{\HELO}{\mathtt{Hello}}
\newcommand{\KEYS}{\mathtt{KeyShare}}
\newcommand{\PSKS}{\mathtt{PreSharedKey}}
\newcommand{\CERT}{\mathtt{Certificate}}
\newcommand{\CERTV}{\mathtt{CertificateVerify}}
\newcommand{\FIN}{\mathtt{Finished}}

% middle versions
\newcommand{\mCERTR}{\mathtt{CertRequest}}
\newcommand{\mSCERT}{\mathtt{ServerCert}}
\newcommand{\mSCERTV}{\mathtt{ServerCertVfy}}
\newcommand{\mCCERT}{\mathtt{ClientCert}}
\newcommand{\mCCERTV}{\mathtt{ClientCertVfy}}
\newcommand{\mSFIN}{\mathtt{ServerFin}}
\newcommand{\mCFIN}{\mathtt{ClientFin}}

% short versions
\newcommand{\sCHELO}{\mathtt{CH}}
\newcommand{\sSHELO}{\mathtt{SH}}
\newcommand{\sCKEYS}{\mathtt{CKS}}
\newcommand{\sSKEYS}{\mathtt{SKS}}
\newcommand{\sCPSK}{\mathtt{CPSK}}
\newcommand{\sSPSK}{\mathtt{SPSK}}
\newcommand{\sENCEX}{\mathtt{EE}}
\newcommand{\sCERTR}{\mathtt{CR}}
\newcommand{\sSCERT}{\mathtt{SCRT}}
\newcommand{\sSCERTV}{\mathtt{SCV}}
\newcommand{\sCCERT}{\mathtt{CCRT}}
\newcommand{\sCCERTV}{\mathtt{CCV}}
\newcommand{\sSFIN}{\mathtt{SF}}
\newcommand{\sCFIN}{\mathtt{CF}}


%
% TLS keys and computed values
%

\newcommand{\DHE}{\mathrm{DHE}} % Diffie--Hellman shared value
\newcommand{\PSK}{\mathrm{PSK}} % pre-shared key

\newcommand{\ES}{\mathrm{ES}}   % early secret
\newcommand{\dES}{\mathrm{dES}} % "derived" early secret
\newcommand{\HS}{\mathrm{HS}}   % handshake secret
\newcommand{\dHS}{\mathrm{dHS}} % "derived" handshake secret
\newcommand{\MS}{\mathrm{MS}}   % master secret

\newcommand{\BK}{\mathrm{BK}}     % binder key
\newcommand{\ETS}{\mathrm{ETS}}   % early traffic secret
\newcommand{\EEMS}{\mathrm{EEMS}} % early exporter master secret

\newcommand{\HTS}{\mathrm{HTS}}   % handshake traffic secret
\newcommand{\CHTS}{\mathrm{CHTS}} % client handshake traffic secret
\newcommand{\SHTS}{\mathrm{SHTS}} % server handshake traffic secret
\newcommand{\CFK}{\mathrm{fk}_C}  % client finished key
\newcommand{\SFK}{\mathrm{fk}_S}  % server finished key

\newcommand{\CATS}{\mathrm{CATS}} % client application traffic secret
\newcommand{\SATS}{\mathrm{SATS}} % server application traffic secret
\newcommand{\EMS}{\mathrm{EMS}}   % exporter master secret
\newcommand{\RMS}{\mathrm{RMS}}   % resumption master secret
\newcommand{\ATS}{\mathrm{ATS}}   % generalized application traffic secret (combining client+server)

\newcommand{\tkead}{\mathrm{tk}_{\text{eapp}}}    % early application data traffic key
\newcommand{\tkchs}{\mathrm{tk}_{\text{chs}}}     % client handshake traffic key
\newcommand{\tkshs}{\mathrm{tk}_{\text{shs}}}     % server handshake traffic key
\newcommand{\tkcapp}{\mathrm{tk}_{\text{capp}}}   % client application traffic key
\newcommand{\tksapp}{\mathrm{tk}_{\text{sapp}}}   % server application traffic key


%
% TLS figure commands
%
\newcommand{\TLSmsg}[1]{{\color{TLSmsgcolor}#1}}
\newcommand{\PSKECDHEonly}[2][0]{{\color{PSKECDHEonlycolor}[#2]$^{\dagger}$}}
\newcommand{\PSKonly}[2][0]{{\color{PSKonlycolor}[#2]$^{\diamond}$}}

%
% TLS colors
%
\definecolor{TLSmsgcolor}{named}{OliveGreen}
\definecolor{StageSeparatorColor}{named}{MidnightBlue}
\definecolor{PSKonlycolor}{named}{BurntOrange}
\definecolor{PSKECDHEonlycolor}{named}{BrickRed}

\definecolor{inputsecretcolor}{named}{tudorange}
\definecolor{internalcolor}{named}{tudred}
\definecolor{stagekeycolor}{named}{tudblue}
\definecolor{hkdfcolor}{named}{tudgreen}

%
% GGM 
% 

\newcommand{\OP}{\oraclestyle{OP}}
\newcommand{\Bijections}{\mathrm{Bijections}}
\newcommand{\sgn}{\text{ sgn }}
\newcommand{\GL}{GL}
\newcommand{\generator}{g}
\newcommand{\one}{\mathds{1}}
\newcommand{\VE}{\oraclestyle{VE}}
\newcommand{\TV}{TV}
\newcommand{\TI}{TI}

\DeclareMathVersion{normal2}
%
%% LNCS format modifiers
%\makeatletter
%\newcommand{\lncsonly}{\relax\iffull\expandafter\@gobble\else\expandafter\@firstofone\fi}
%\newcommand{\lncselse}{\relax\iffull\expandafter\@secondoftwo\else\expandafter\@firstoftwo\fi}
\newcommand{\fullonly}{\relax\iffull\expandafter\@firstofone\else\expandafter\@gobble\fi}
\newcommand{\fullelse}{\relax\iffull\expandafter\@firstoftwo\else\expandafter\@secondoftwo\fi}
%\newcommand{\nfullonly}{\relax\iffull\expandafter\@gobble\else\expandafter\@firstofone\fi}
%\makeatother
%\newcommand{\lncsbreak}{\lncsonly{\allowbreak}}
%\newcommand{\lncsforcebreak}{\lncsonly{\linebreak}}
%\newcommand{\lncsnewline}{\lncsonly{\newline}}
%\newcommand{\lncsdot}{\lncsonly{.}}  % lncs-only ending dot for inline headings (e.g., \subsubsection)

%
% Formatting modifiers
%
\newcommand{\shortlongeqn}[2][]{\[ #2 #1 \]}
\newcommand{\lightparagraph}[1]{\paragraph{#1}}


%
% Fonts
%

\DeclareMathAlphabet{\mathsc}{OT1}{cmr}{m}{sc}


%
% Colors
%
\definecolor{darkblue}{rgb}{0,0,0.5}
\definecolor{darkgreen}{rgb}{0,0.5,0}
\definecolor{darkred}{rgb}{0.5,0,0}

\definecolor{tudred}{RGB}{230,0,26}
\definecolor{tudgreen}{RGB}{153,192,0}
\definecolor{tudorange}{RGB}{245,163,0}
\definecolor{tudblue}{RGB}{0,104,157}
\definecolor{tudpurple}{RGB}{149,17,105}
\definecolor{tudbrown}{RGB}{169,73,19}

%
% Basics
%
\newcommand{\minus}{\text{\normalfont-}}

\newcommand{\sample}{\xleftarrow{\smash{\raisebox{-1.75pt}{$\scriptscriptstyle\$$}}}}
\newcommand{\tor}{\xrightarrow{\smash{\raisebox{-1.75pt}{$\scriptscriptstyle\$$}}}}

\newcommand{\conc}{\|}

%\newcommand{\comment}[1]{\textcolor{gray}{\scriptsize/\!\!/\,#1}}

%
% Word macros
%

% no word-break in n-RTT
\newcommand{\ZRTT}{\mbox{0-RTT}\xspace}
\newcommand{\OneRTT}{\mbox{1-RTT}\xspace}

\newcommand{\SIGMA}{\mbox{SIGMA}\xspace}
\newcommand{\SIGMAI}{\mbox{SIGMA-I}\xspace}

% math-mode macros (in theorems)
\newcommand{\mSIGMAI}{\mathrm{SIGMA{\minus}I}}
\newcommand{\mTLS}{\mathrm{TLS\,1.3}}


%
% Redefine standard commands
%

% \paragraph, \subparagraph: automatically append a dot after paragraph titles
\let\originalparagraph\paragraph
\renewcommand{\paragraph}[1]{%
  \ifthenelse{\endswith{#1}{.}}{%
    \originalparagraph{#1}%
  }{%
    \originalparagraph[#1]{#1.}%
  }%
}

\let\originalsubparagraph\subparagraph
\renewcommand{\subparagraph}[1]{%
  \ifthenelse{\endswith{#1}{.}}{%
    \originalsubparagraph{#1}%
  }{%
    \originalsubparagraph[#1]{#1.}%
  }%
}


%
% Definitions, Theorems, etc.
% (already present in LNCS)
%
%
%\iffull
%\theoremstyle{plain}
%\newtheorem{theorem}{Theorem}[section]
%\newtheorem{lemma}[theorem]{Lemma}
%\newtheorem{proposition}[theorem]{Proposition}
%\newtheorem{corollary}[theorem]{Corollary}
%
 \theoremstyle{definition}
\newtheorem{definition}{Definition}
%\newtheorem{construction}[theorem]{Construction}
%
%\theoremstyle{remark}
%\newtheorem{remark}[theorem]{Remark}
%\newtheorem{example}[theorem]{Example}
%\fi



\newcommand{\NN}{\mathds{N}}
\newcommand{\ZZ}{\mathds{Z}}

%\newcommand{\group}{\mathbb{G}}

\newcommand{\gamefont}[1]{\mathrm{#1}}
%\newcommand{\procfont}[1]{\textsc{#1}}
\newcommand{\setfont}[1]{\mathcal{#1}}

%\newcommand{\Adv}{\mathsf{Adv}}
% \newcommand{\genAdv}[3]{\Adv^{\mathsf{#1}}_{#2}(#3)}
%\newcommand{\genAdv}[3]{\Adv^{\mathsf{#1}}_{#2,#3}}
\newcommand{\genAdvA}[2]{\genAdv{#1}{#2}{\advA}}
%\newcommand{\bad}[1][]{\mathsf{bad}_{#1}}
%\newcommand{\Gm}{\gamefont{G}}
%\newcommand{\advA}{\mathcal{A}}
%\newcommand{\advB}{\mathcal{B}}


% ----- Colors -----
\newcommand{\gray}[1]{{\color{gray}#1}}
\newcommand{\grey}{\gray}

% ----- Highlights, etc. -----


%\definecolor{highlight-gray}{gray}{0.90}

%\newcommand{\gamechange}[1][highlight-gray]{\setlength{\fboxsep}{0pt}\colorbox{#1}}

%
% Macro styles
%

\newcommand{\algostyle}[1]{\mathsf{#1}} % algorithm
\newcommand{\predstyle}[1]{\mathsf{#1}} % predicate
\newcommand{\notionstyle}[1]{\mathsf{#1}} % security notion
\newcommand{\oraclestyle}[1]{\mathsc{#1}} % oracle

\newcommand{\protvarstyle}[1]{#1} % protocol variable
\newcommand{\protvalstyle}[1]{\mathsf{#1}} % protocol value
\newcommand{\gamevarstyle}[1]{\mathsf{#1}} % game variable

\newcommand{\querynum}[1]{q_{\oraclestyle{#1}}} % number of queries to oracle #1 (latter may be abbreviated)


%
% Components
%

\newcommand{\New}{\oraclestyle{New}} % multi-user new key query
\newcommand{\qNew}{\querynum{Nw}}
\newcommand{\Corrupt}{\oraclestyle{Corrupt}} % multi-user corrupt query
\newcommand{\qCorrupt}{\querynum{C}}

% PRFs
\newcommand{\PRF}{\algostyle{PRF}}
\newcommand{\muPRFSEC}{\notionstyle{mu{\minus}PRF}}
\newcommand{\PRFfn}{\oraclestyle{Fn}} % PRF function oracle
\newcommand{\qPRFfn}{\querynum{Fn}}
\newcommand{\qPRFfnU}{\querynum{Fn/U}}
%\newcommand{\FUNC}{\mathsf{FUNC}} % function space

% input labels / constants to PRF etc.
\newcommand{\inputlabel}[1][]{\mathtt{L}_{#1}}
\newcommand{\constant}[1][]{\mathtt{C}_{#1}}

% Hash
\newcommand{\Hash}{\algostyle{H}} % the hash function
\newcommand{\COLL}{\notionstyle{CR}} % collision resistance

% Random oracle
%\newcommand{\RO}{\algostyle{RO}}
\newcommand{\qRO}{\querynum{RO}} % number of RO queries

%Encryption
\newcommand{\ENCScheme}{\algostyle{E}}
\newcommand{\ENCEnc}{\algostyle{Enc}}
\newcommand{\ENCDec}{\algostyle{Dec}}

% Signatures
\newcommand{\SIGScheme}{\algostyle{S}}
\newcommand{\SIGKGen}{\algostyle{KGen}}
\newcommand{\SIGSign}{\algostyle{Sign}}
\newcommand{\SIGVerify}{\algostyle{Vrfy}}
\newcommand{\OSign}{\oraclestyle{Sign}}
\newcommand{\qSign}{\querynum{Sg}}
\newcommand{\qSignU}{\querynum{Sg/U}}

% MACs
\newcommand{\MACScheme}{\algostyle{M}}
\newcommand{\MACKGen}{\algostyle{KGen}}
\newcommand{\MACTag}{\algostyle{Tag}}
\newcommand{\MACVerify}{\algostyle{Vrfy}}
\newcommand{\OTag}{\oraclestyle{Tag}}
\newcommand{\qTag}{\querynum{Tg}}
\newcommand{\qTagU}{\querynum{Tg/U}}
\newcommand{\OVerify}{\oraclestyle{Vrfy}}
\newcommand{\qVerify}{\querynum{V\!f}}
\newcommand{\qVerifyU}{\querynum{V\!f/U}}
\newcommand{\EUFCMA}{\notionstyle{EUF{\minus}CMA}}
\newcommand{\muEUFCMA}{\notionstyle{mu{\minus}EUF{\minus}CMA}}

% HKDF
\newcommand{\HKDF}{\algostyle{HKDF}}
\newcommand{\HMAC}{\algostyle{HMAC}}
\newcommand{\Extract}{\algostyle{Extract}} % HKDF Extract function
\newcommand{\Expand}{\algostyle{Expand}} % HKDF Expand function
\newcommand{\sExtract}{\algostyle{Ext}} % short HKDF Extract function
\newcommand{\sExpand}{\algostyle{Exp}} % short HKDF Expand function

% Strong Diffie--Hellman
\newcommand{\strongDH}{\notionstyle{stDH}} % strongDH assumption
\newcommand{\DDH}{\algostyle{DDH}} % DDH oracle
\newcommand{\qstDH}{\querynum{sDH}} % number of DDH/stDH oracle queries

%
% Key Exchange
%

\newcommand{\KE}{\algostyle{KE}}
\newcommand{\KEKGen}{\algostyle{KGen}}
\newcommand{\KEActivate}{\algostyle{Activate}}
\newcommand{\KERun}{\algostyle{Run}}
\newcommand{\KEkeyspace}{\KE.\algostyle{KS}}
\newcommand{\TLS}{\algostyle{TLS}}

% protocol variables
\def\PROTOCOLVARIABLES{pk,sk,id,peerid,peerpk,role,st,sid,status,skey,state,X,Y,W,N,Recv,E,x,y,r,kl,mk}
\foreach \protvar in \PROTOCOLVARIABLES {%
	\expandafter\xdef\csname\protvar\endcsname{\noexpand\protvarstyle{\protvar}}
}
% some more, manually
\newcommand{\nonce}{\protvarstyle{n}}
\newcommand{\ol}{\protvarstyle{ol}}
\newcommand{\nl}{\protvarstyle{nl}}

% protocol values
\def\PROTOCOLVALUES{running,accepted,rejected,initiator,responder}
\foreach \protval in \PROTOCOLVALUES {%
	\expandafter\xdef\csname\protval\endcsname{\noexpand\protvalstyle{\protval}}
}

\newcommand{\labelis}{\inputlabel[is]}
\newcommand{\labelim}{\inputlabel[im]}
\newcommand{\labelrs}{\inputlabel[rs]}
\newcommand{\labelrm}{\inputlabel[rm]}
\newcommand{\ks}{k_s}
\newcommand{\kt}{k_t}
\newcommand{\ke}{k_e}
\newcommand{\Zz}{Z}
\newcommand{\ciph}{c}
\newcommand{\sidhash}{d}

% StrongDH
\newcommand{\stDH}{\mathsf{stDH}} % strongDH oracle

%
% SIGMA
%
\newcommand{\RunInitI}{\algostyle{RunInit1}}
\newcommand{\RunInit}{\algostyle{RunInit2}}
\newcommand{\RunRespI}{\algostyle{RunResp1}}
\newcommand{\RunRespII}{\algostyle{RunResp2}}


%
% Key Exchange Model
%

\newcommand{\KESEC}{\notionstyle{KE{\minus}SEC}}

% Oracles
%\newcommand{\Initialize}{\oraclestyle{Initialize}}
%\newcommand{\Finalize}{\oraclestyle{Finalize}}
\newcommand{\Send}{\oraclestyle{Send}}
\newcommand{\NewUser}{\oraclestyle{NewUser}}
\newcommand{\RevSessionKey}{\oraclestyle{RevSessionKey}}
\newcommand{\RevLongTermKey}{\oraclestyle{RevLongTermKey}}
\newcommand{\Test}{\oraclestyle{Test}}

% number of queries
\newcommand{\qNewUser}{\querynum{N}}
\newcommand{\qSend}{\querynum{S}}
\newcommand{\qRevSessionKey}{\querynum{RS}}
\newcommand{\qRevLongTermKey}{\querynum{RL}}
\newcommand{\qTest}{\querynum{T}}
\newcommand{\qRevState}{\querynum{RSt}}

% Predicates
\newcommand{\Sound}{\predstyle{Sound}}
\newcommand{\ExplicitAuth}{\predstyle{ExplicitAuth}}
\newcommand{\Fresh}{\predstyle{Fresh}}

% game variables
\def\GAMEVARIABLES{revealed,tested,revltk,time,users,S,Q,Sent}
\foreach \gamevar in \GAMEVARIABLES {%
	\expandafter\xdef\csname\gamevar\endcsname{\noexpand\gamevarstyle{\gamevar}}
}
% some more, manually
\newcommand{\taccepted}{\gamevarstyle{t_{acc}}}


%
% Game hop proofs, environments, counters
%

% initiate a proof, giving it a (reference) name
\newcommand{\startproof}[1]{%
	\xdef\currentproof{#1}
	\firstproofngame
}
\newcommand{\currentproof}{}

\newcounter{proofngame} % counter for games in proofs without multiple cases

\renewcommand*{\theHproofngame}{\currentproof.\theproofngame} % make proofngame counter references unique per proof (for hyperref)

\newcommand{\curgame}{}  % label of current game
\newcommand{\prevgame}{} % label of previous game

\newcommand{\firstproofngame}[1][0]{\setcounter{proofngame}{-1+#1}} % reset counter to be 0 next time
\newcommand{\proofngame}[1][]{%  %%% optional argument is used as \label
	\refstepcounter{proofngame}%
	\ifthenelse{\isempty{#1}}{}{%
		\label{game:\currentproof:#1}%
		\edef\prevgame{\curgame}%
		\edef\curgame{game:\currentproof:#1}%
	}%
	\proofparagraph{Game \theproofngame}
}
\newcommand{\proofngames}[2][]{%  %%% (second) argument is number of game steps, optional argument is used as \label for the last gamd
	\stepcounter{proofngame}
	\xdef\oldproofngame{\theproofngame}
	\addtocounter{proofngame}{#2} % add number of game steps
	\addtocounter{proofngame}{-2} % -2 for step before and \ref*step* counter
	\refstepcounter{proofngame}%
	\ifthenelse{\isempty{#1}}{}{%
		\label{game:\currentproof:#1}%
		\edef\prevgame{\curgame}%
		\edef\curgame{game:\currentproof:#1}%
	}%
	\proofparagraph{Games \oldproofngame--\theproofngame}
}

% spacing between proof paragraphs
\newcommand{\proofsep}{\iffull\bigskip\else\medskip\fi}

% proof paragraph style
\makeatletter
\newcommand\gobblepars{% removes \par after command, used to make \proofparagraph a run-in heading
    \@ifnextchar\par%
        {\expandafter\gobblepars\@gobble}%
        {}}
\newcommand{\proofparagraph}[1]{%
% 	\iffullversion\bigskip\else\smallskip\fi
% 	\noindent\textit{#1.}\hspace{1em}\@afterindentfalse\@afterheading\gobblepars} %% this seems to not indent the first paragraph following the paragraph start
	
	\proofsep
	\noindent\textbf{#1.}\hspace{1em}\gobblepars}
\makeatother

% macros for the current, previous, or labeled games
\makeatletter
\newcommand{\curGm}{\@ifstar{\@curGmNL}{\@curGmL}}
	\newcommand{\@curGmNL}{\Gm_{\ref*{\curgame}}} % without hyperlink
	\newcommand{\@curGmL}{\Gm_{\ref{\curgame}}} % with hyperlink
\newcommand{\prevGm}{\@ifstar{\@prevGmNL}{\@prevGmL}}
	\newcommand{\@prevGmNL}{\Gm_{\ref*{\prevgame}}} % without hyperlink
	\newcommand{\@prevGmL}{\Gm_{\ref{\prevgame}}} % with hyperlink
\newcommand{\lblGm}{\@ifstar{\@lblGmNL}{\@lblGmL}}
	\newcommand{\@lblGmNL}[1]{\Gm_{\ref*{game:\currentproof:#1}}} % without hyperlink
	\newcommand{\@lblGmL}[1]{\Gm_{\ref{game:\currentproof:#1}}} % with hyperlink
\makeatother

% macros for current and new \advB reduction
\newcommand{\curadvB}{\advB_{\arabic{advB-\currentproof}}} % current \advB_x (does not increment counter)
\newcommand{\newadvB}{\stepcounter{advB-\currentproof}\curbdv} % new \advB_x (increments counter)



%
% Protocol Figures
%

% For any new figure, you to set X coordinates for client, server, and arrows; and initial Y coordinate, e.g.:
% 	\edef\ClientX{0}
% 	\edef\ArrowLeft{0}
% 	\edef\ArrowRight{9}
% 	\edef\ServerX{9}
% 	\edef\Y{0}

% ClientAction and ServerAction
% Print a command executed by the client/server
% 1st argument: the text
\newcommand{\ClientAction}[2][]{
	\node[right,#1] at (\ClientX, \Y) {#2};
}
\newcommand{\ServerAction}[2][]{
	\node[left,#1] at (\ServerX, \Y) {#2};
}
\newcommand{\SharedAction}[2][]{
	\node[#1] at ($1/2*(\ClientX, \Y)+1/2*(\ServerX, \Y)$) {#2};
}

% ClientToServer and ServerToClient
% Draws a message flow from client-to-server or server-to-client, with text above and below
% 1st argument (optional): line type, default ->
% 2nd argument: text above
% 3rd argument: text below
% Example: \ClientToServer{$Y$}{}
% Example: \ClientToServer[<->,double]{$Y$}{over an encrypted channel}
\newcommand{\ClientToServer}[3][->]{
	\NextLine[0.5]
	\draw[>=latex, very thick, #1] (\ArrowLeft+1,\Y) -- node[above=-0.1] {#2} node[below] {#3} (\ArrowRight-1,\Y) ;
	\NextLine[0.25]
}
\newcommand{\ServerToClient}[3][->]{
	\NextLine[0.5]
	\draw[>=latex, very thick, #1] (\ArrowRight-1,\Y) -- node[above=-0.1] {#2} node[below] {#3} (\ArrowLeft+1,\Y) ;
	\NextLine[0.25]
}

% Spacing factor for NextLines
\iffull
\def\NextLineSpacing{0.55}
\else
\def\NextLineSpacing{0.45}
\fi

% NextLine
% 1st argument (optional): amount of spacing to increment by, default 1.0
% Example: \NextLine
% Example: \NextLine[1.5]
\newcommand{\NextLine}[1][1.0]{
	\pgfmathparse{\Y-\NextLineSpacing*#1}
	\edef\Y{\pgfmathresult}
}

%
% stage separator line
%
\newcommand{\StageSeparator}[1]{
	\draw[very thick,dotted,StageSeparatorColor] (\ArrowLeft,\Y-0.5*\NextLineSpacing) -- (\ArrowRight+0.15,\Y-0.5*\NextLineSpacing) node[right,anchor=west,font=\footnotesize] {stage~#1} ;
}

%
% accept key and stage
%
\newcommand{\AcceptStage}[2]{
	\SharedAction{{\color{StageSeparatorColor}\textbf{accept} #2}}
	\StageSeparator{#1}
}



%%%%%%%%%%%%%%%%%%%%%%%%%%%%%%%%%%%%%%%%%%%%%%%%%%%%%%%%%%%%%%%%%%%%%%%%%%%%%%%%%%%%%%%%%%
% TLS
%%%%%%%%%%%%%%%%%%%%%%%%%%%%%%%%%%%%%%%%%%%%%%%%%%%%%%%%%%%%%%%%%%%%%%%%%%%%%%%%%%%%%%%%%%


%
% TLS messages
%

% long versions
\newcommand{\CHELO}{\mathtt{ClientHello}}
\newcommand{\SHELO}{\mathtt{ServerHello}}
\newcommand{\CKEYS}{\mathtt{ClientKeyShare}}
\newcommand{\SKEYS}{\mathtt{ServerKeyShare}}
%\newcommand{\CPSK}{\mathtt{ClientPreSharedKey}}
%\newcommand{\SPSK}{\mathtt{ServerPreSharedKey}}
\newcommand{\ENCEX}{\mathtt{EncryptedExtensions}}
\newcommand{\CERTR}{\mathtt{CertificateRequest}}
\newcommand{\SCERT}{\mathtt{ServerCertificate}}
\newcommand{\SCERTV}{\mathtt{ServerCertificateVerify}}
\newcommand{\CCERT}{\mathtt{ClientCertificate}}
\newcommand{\CCERTV}{\mathtt{ClientCertificateVerify}}
\newcommand{\SFIN}{\mathtt{ServerFinished}}
\newcommand{\CFIN}{\mathtt{ClientFinished}}

% common names
\newcommand{\HELO}{\mathtt{Hello}}
\newcommand{\KEYS}{\mathtt{KeyShare}}
\newcommand{\PSKS}{\mathtt{PreSharedKey}}
\newcommand{\CERT}{\mathtt{Certificate}}
\newcommand{\CERTV}{\mathtt{CertificateVerify}}
\newcommand{\FIN}{\mathtt{Finished}}

% middle versions
\newcommand{\mCERTR}{\mathtt{CertRequest}}
\newcommand{\mSCERT}{\mathtt{ServerCert}}
\newcommand{\mSCERTV}{\mathtt{ServerCertVfy}}
\newcommand{\mCCERT}{\mathtt{ClientCert}}
\newcommand{\mCCERTV}{\mathtt{ClientCertVfy}}
\newcommand{\mSFIN}{\mathtt{ServerFin}}
\newcommand{\mCFIN}{\mathtt{ClientFin}}

% short versions
\newcommand{\sCHELO}{\mathtt{CH}}
\newcommand{\sSHELO}{\mathtt{SH}}
\newcommand{\sCKEYS}{\mathtt{CKS}}
\newcommand{\sSKEYS}{\mathtt{SKS}}
\newcommand{\sCPSK}{\mathtt{CPSK}}
\newcommand{\sSPSK}{\mathtt{SPSK}}
\newcommand{\sENCEX}{\mathtt{EE}}
\newcommand{\sCERTR}{\mathtt{CR}}
\newcommand{\sSCERT}{\mathtt{SCRT}}
\newcommand{\sSCERTV}{\mathtt{SCV}}
\newcommand{\sCCERT}{\mathtt{CCRT}}
\newcommand{\sCCERTV}{\mathtt{CCV}}
\newcommand{\sSFIN}{\mathtt{SF}}
\newcommand{\sCFIN}{\mathtt{CF}}


%
% TLS keys and computed values
%

\newcommand{\DHE}{\mathrm{DHE}} % Diffie--Hellman shared value
\newcommand{\PSK}{\mathrm{PSK}} % pre-shared key

\newcommand{\ES}{\mathrm{ES}}   % early secret
\newcommand{\dES}{\mathrm{dES}} % "derived" early secret
\newcommand{\HS}{\mathrm{HS}}   % handshake secret
\newcommand{\dHS}{\mathrm{dHS}} % "derived" handshake secret
\newcommand{\MS}{\mathrm{MS}}   % master secret

\newcommand{\BK}{\mathrm{BK}}     % binder key
\newcommand{\ETS}{\mathrm{ETS}}   % early traffic secret
\newcommand{\EEMS}{\mathrm{EEMS}} % early exporter master secret

\newcommand{\HTS}{\mathrm{HTS}}   % handshake traffic secret
\newcommand{\CHTS}{\mathrm{CHTS}} % client handshake traffic secret
\newcommand{\SHTS}{\mathrm{SHTS}} % server handshake traffic secret
\newcommand{\CFK}{\mathrm{fk}_C}  % client finished key
\newcommand{\SFK}{\mathrm{fk}_S}  % server finished key

\newcommand{\CATS}{\mathrm{CATS}} % client application traffic secret
\newcommand{\SATS}{\mathrm{SATS}} % server application traffic secret
\newcommand{\EMS}{\mathrm{EMS}}   % exporter master secret
\newcommand{\RMS}{\mathrm{RMS}}   % resumption master secret
\newcommand{\ATS}{\mathrm{ATS}}   % generalized application traffic secret (combining client+server)

\newcommand{\tkead}{\mathrm{tk}_{\text{eapp}}}    % early application data traffic key
\newcommand{\tkchs}{\mathrm{tk}_{\text{chs}}}     % client handshake traffic key
\newcommand{\tkshs}{\mathrm{tk}_{\text{shs}}}     % server handshake traffic key
\newcommand{\tkcapp}{\mathrm{tk}_{\text{capp}}}   % client application traffic key
\newcommand{\tksapp}{\mathrm{tk}_{\text{sapp}}}   % server application traffic key


%
% TLS figure commands
%
\newcommand{\TLSmsg}[1]{{\color{TLSmsgcolor}#1}}
\newcommand{\PSKECDHEonly}[2][0]{{\color{PSKECDHEonlycolor}[#2]$^{\dagger}$}}
\newcommand{\PSKonly}[2][0]{{\color{PSKonlycolor}[#2]$^{\diamond}$}}

%
% TLS colors
%
\definecolor{TLSmsgcolor}{named}{OliveGreen}
\definecolor{StageSeparatorColor}{named}{MidnightBlue}
\definecolor{PSKonlycolor}{named}{BurntOrange}
\definecolor{PSKECDHEonlycolor}{named}{BrickRed}

\definecolor{inputsecretcolor}{named}{tudorange}
\definecolor{internalcolor}{named}{tudred}
\definecolor{stagekeycolor}{named}{tudblue}
\definecolor{hkdfcolor}{named}{tudgreen}

%
% GGM 
% 

\newcommand{\OP}{\oraclestyle{OP}}
\newcommand{\Bijections}{\mathrm{Bijections}}
\newcommand{\sgn}{\text{ sgn }}
\newcommand{\GL}{GL}
\newcommand{\generator}{g}
\newcommand{\one}{\mathds{1}}
\newcommand{\VE}{\oraclestyle{VE}}
\newcommand{\TV}{TV}
\newcommand{\TI}{TI}

\DeclareMathVersion{normal2}
\renewcommand*{\backref}[1]{(Cited on page~#1.)}
\DeclareMathAlphabet{\mathsc}{OT1}{cmr}{m}{sc}
\DeclareMathAlphabet{\mathslbf}{OT1}{cmr}{bx}{sl}

\setlength{\fboxsep}{1pt}

% ====================================================================
%
\newcommand{\gamesfontsize}{\small}
\newcommand{\gameName}[1]{\underline{$\main$ #1}\\[2pt]}


\newcommand{\twoColsNoDivide}[4]{
\begin{center}
        \framebox{
        \begin{tabular}{c@{\hspace*{.4em}}c@{\hspace*{.4em}}c}
        \begin{minipage}[t]{#1\textwidth}\setstretch{1.0}\gamesfontsize #3 \end{minipage}
        &
        \begin{minipage}[t]{#2\textwidth}\setstretch{1.0}\gamesfontsize #4 \end{minipage}
        \end{tabular}
        }
\end{center}
}

\newcommand{\oneCol}[2]{
\begin{center}
        \framebox{
        \begin{tabular}{c}
        \begin{minipage}[t]{#1\textwidth}\setstretch{1.0}\gamesfontsize #2 \end{minipage}
        \end{tabular}
        }
\end{center}
}


\newcommand{\twoCols}[4]{
\begin{center}
        \framebox{
        \begin{tabular}{c@{\hspace*{.4em}}|c@{\hspace*{.4em}}}
        \begin{minipage}[t]{#1\textwidth}\setstretch{1.0}\gamesfontsize #3 \end{minipage}
        &
        \begin{minipage}[t]{#2\textwidth}\setstretch{1.0}\gamesfontsize #4 \end{minipage}
        \end{tabular}
        }
\end{center}
}
%
%% Theorem environments
%
\newtheorem{thm}{Theorem}
\newtheorem{lem}{Lemma}


\newenvironment{theorem}{\begin{thm}}{\end{thm}}
\newenvironment{lemma}{\begin{lem}}{\end{lem}}

\newcommand{\secref}[1]{Section~\ref{#1}}
\newcommand{\apref}[1]{Appendix~\ref{#1}}
\newcommand{\thref}[1]{Theorem~\ref{#1}}

\newcommand{\figref}[1]{Figure~\ref{#1}}
\newcommand{\tabref}[1]{Table~\ref{#1}}

%%%fancy proof env %%%
\def\qedsym{\hspace{1pt}\rule[-1pt]{3pt}{9pt}}

\newlength{\saveparindent}
\setlength{\saveparindent}{\parindent}
\newlength{\saveparskip}
\setlength{\saveparskip}{\parskip}


\def\qsym{\vrule width0.6ex height1em depth0ex}
\newcount\proofqeded
\newcount\proofended
\def\qed{{\hspace{1pt}\rule[-1pt]{3pt}{9pt}}
\end{rm}\addtolength{\parskip}{-0pt}
\setlength{\parindent}{\saveparindent}
\global\advance\proofqeded by 1 }
\def\qedenv{
\end{rm}\addtolength{\parskip}{-0pt}
\setlength{\parindent}{\saveparindent}
\global\advance\proofqeded by 1 }
\renewenvironment{proof}%
 {\proofstart}%
 {\ifnum\proofqeded=\proofended~\qed\fi \global\advance\proofended by 1
  \medskip}
\newenvironment{proofenv}%
 {\proofenvstart}%
 {\ifnum\proofqeded=\proofended\qedenv\fi \global\advance\proofended by 1
  \medskip}
\makeatletter
\def\proofstart{\@ifnextchar[{\@oprf}{\@nprf}}
\def\proofenvstart{\@ifnextchar[{\@osprf}{\@nsprf}}
\def\@oprf[#1]{\begin{rm}\protect\vspace{6pt}\noindent{\bf Proof of #1:\ }%
\addtolength{\parskip}{5pt}\setlength{\parindent}{0pt}}
\def\@osprf[#1]{\begin{rm}\protect\vspace{6pt}\noindent
\addtolength{\parskip}{5pt}\setlength{\parindent}{0pt}}
\def\@nprf{\begin{rm}\protect\vspace{6pt}\noindent{\bf Proof:\ }%
\addtolength{\parskip}{5pt}\setlength{\parindent}{0pt}}
\def\@nsprf{\begin{rm}\protect\vspace{6pt}\noindent%
\addtolength{\parskip}{5pt}\setlength{\parindent}{0pt}}


% ========================================================================

% Lists

\newcounter{ctr}
\newenvironment{bignewenum}{%
\begin{list}{{\bf (\arabic{ctr})}\hfill}{\usecounter{ctr} \labelwidth=17pt%
\labelsep=7pt \leftmargin=24pt \topsep=4pt%
\setlength{\listparindent}{\saveparindent}%
\setlength{\parsep}{\saveparskip}%
\setlength{\itemsep}{4pt} }}{\end{list}}


\newenvironment{newmath}{\begin{displaymath}%
\setlength{\abovedisplayskip}{4pt}%
\setlength{\belowdisplayskip}{4pt}%
\setlength{\abovedisplayshortskip}{5pt}%
\setlength{\belowdisplayshortskip}{5pt} }{\end{displaymath}}

% =========================================================================


%\makeatletter
%\def\appearsin#1{\gdef\@appearsin{#1}}
%\def\maketitle{\par
% \begingroup
% \def\thefootnote{\arabic{footnote}}
% \def\@makefnmark{\hbox
% to 0pt{$^{\@thefnmark}$\hss}}
% \if@twocolumn
% \twocolumn[\@maketitle]
% \else \newpage
% \global\@topnum\z@ \@maketitle \fi\thispagestyle{plain}\@thanks
% \endgroup
% \setcounter{footnote}{0}
% \let\maketitle\relax
% \let\@maketitle\relax
% \gdef\@thanks{}\gdef\@author{}\gdef\@title{}\gdef\@appearsin{}
%          \let\thanks\relax}
%\def\@maketitle{\newpage
% \noindent \@appearsin
% \vskip 0.5in \begin{center}
% {\LARGE \@title \par} \vskip 1.5em {\large \lineskip .5em
%\begin{tabular}[t]{c}\@author
% \end{tabular}\par}
% \vskip 1em {\normalsize \@date} \end{center}
% \par
% \vskip 1.5em}
%\def\abstract{\if@twocolumn
%\section*{Abstract}
%\else \small
%\begin{center}
%{\bf Abstract\vspace{-.5em}\vspace{0pt}}
%\end{center}
%\quotation
%\fi}
%\def\endabstract{\if@twocolumn\else\endquotation\fi}
%\mark{{}{}}
%


% =========================================================================

% General
%\newcommand{\hindent}{\quad}

\newcommand{\mysubsubsection}[1]{\subsubsection{#1}}
\newcommand{\headingb}[1]{{\vspace{5pt}\noindent\textbf{#1.}}}
\newcommand{\headingg}[1]{{\sc{#1}}}
\newcommand{\heading}[1]{{\vspace{5pt}\noindent\sc{#1}}}

\newcommand{\tuple}[1]{\langle{#1}\rangle}
\def\from{\mbox{from}\ }
\def\From{\mbox{From}\ }
\def\bits{\{0,1\}}
\def\cross{\times}
\newcommand{\cab}{,\allowbreak}
\newcommand{\xor}{\oplus}
\newcommand{\Colon}{{:\;\;}}
\def\emptystring{\varepsilon}

\newcommand{\OL}[0]{\mathsf{OL}}

\newcommand{\Dom}{\mathsf{Dom}}
\newcommand{\DomR}{\mathsf{DomR}}
\newcommand{\Rng}{\mathsf{Rng}}
\newcommand{\RngR}{\mathsf{RngR}}
\newcommand{\calA}{{\cal A}}
\newcommand{\calB}{{\cal B}}
\newcommand{\calC}{{\cal C}}
\newcommand{\calF}{{\cal F}}
\newcommand{\N}{{{\mathds N}}}
\newcommand{\Z}{{{\mathds Z}}}
\newcommand{\R}{{{\mathds R}}}
\def\union{\cup}
\def\intersection{\cap}
\newcommand{\set}[2]{\{#1 \,:\, #2\}}
\newcommand{\getsr}{{{\,\leftarrow{\hspace*{-3pt}\raisebox{.75pt}{$\scriptscriptstyle\$$}}\,}}}
\newcommand{\OR}{{\textstyle\bigvee}}
\newcommand{\Var}{{\mbox{\bf Var}}}
\newcommand{\E}{\mathbf{E}}
\def\e{{\epsilon}}
\newcommand{\concat}{{\,\|\,}}
% ========================================================================

% ===================================================================
\newcommand{\comment}[1]{\hspace{8pt}{\small /$\!\!$/\ #1}}
\newcommand{\Comment}[1]{\hspace{5pt}{/$\!\!$/\ #1}}
\newcommand{\ccomment}[1]{\hspace{1pt}{/$\!\!$/\ #1}}

\newcommand{\authnote}[2]{ \begin{center}\fbox{\begin{minipage}{5.7in}
\textbf{#1 says:} #2\end{minipage}}\end{center}}

\newcommand{\nGame}[2]{\mathbf{G}^{\mathrm{#1}}_{#2}}
\newcommand{\pk}{pk}
\newcommand{\dk}{dk}
\newcommand{\FSOutSet}[0]{\mathrm{Out}}

\newcommand{\schemefont}[1]{{\mathsf{#1}}}
\newcommand{\algfont}[1]{{\mathrm{#1}}}
\newcommand{\Adv}{\mathbf{Adv}}
\newcommand{\genAdv}[3]{\mathbf{Adv}^{\mathrm{#1}}_{#2}(#3)}
\newcommand{\prfAdv}[1]{\genAdv{prf}{#1}}

\newcommand{\advA}{\mathcal{A}}
\newcommand{\advB}{\mathcal{B}}
\newcommand{\advC}{\mathcal{C}}
\newcommand{\advD}{\mathcal{D}}
\newcommand{\INDCCA}{\mbox{IND-CCA}\xspace}

\newcommand{\atk}{\mathrm{atk}}
\newcommand{\cpa}{\mathrm{cpa}}
\newcommand{\cca}{\mathrm{cca}}
\newcommand{\ind}{\mathrm{ind}}

\addtolength{\belowcaptionskip}{-2mm}
\addtolength{\abovecaptionskip}{-3mm}
\addtolength{\topsep}{-1mm}
\newcommand{\schalg}[2]{\mathsf{#1.#2}}


\newcommand{\procfont}[1]{\mathsc{#1}}
\newcommand{\tablefont}[1]{\mathsf{#1}}
\newcommand{\Initialize}{\Init}
\newcommand{\Finalize}{\Fin}
\newcommand{\Init}{\procfont{init}}
\newcommand{\Fin}{\procfont{fin}}

\newcommand{\RO}{\procfont{RO}}
\newcommand{\ROSim}{\procfont{ROSim}}
\newcommand{\FUNC}{\procfont{FUNC}}
\newcommand{\DecO}{\procfont{Dec}}

\newcommand{\win}{\mathsf{win}}

\newcommand{\Fimage}[0]{\mathrm{Im}}
\newcommand{\simulator}[0]{\mathsf{Sim}}
\newcommand{\Gensimulator}[1]{\mathsf{S}_{#1}}


\newcommand{\group}{\mathds{G}}
\newcommand{\wins}{\textrm{ wins}}
\newcommand{\true}{\mathsf{true}}
\newcommand{\false}{\mathsf{false}}
\newcommand{\AND}{\;\wedge\;}
\newcommand{\bad}{\mathsf{bad}}
\newcommand{\chal}{\mathsf{chal}}
\newcommand{\Good}{\mathsf{Good}}
\newcommand{\Gm}{\textnormal{G}}
\newcommand{\fn}{\footnotesize}

\newcommand{\vecx}{\mathbf{x}}
\newcommand{\pfvec}[0]{\mathbf{p}}
\newcommand{\vecY}{\mathslbf{Y}}
\newcommand{\vecV}{\mathslbf{V}}
\newcommand{\vecU}{\mathslbf{U}}
\newcommand{\vecW}{\mathslbf{W}}
\newcommand{\vecxx}[0]{\mathbf{x}}
\newcommand{\Oracle}{\mathsc{O}}
\newcommand{\pkeScheme}{\schemefont{PKE}}
\newcommand{\pkeEnc}{\schalg{\pkeScheme}{E}}
\newcommand{\pkeDecR}{\schalg{\pkeScheme}{DecR}}
\newcommand{\pkeML}{\schalg{\pkeScheme}{ml}}
\newcommand{\prf}{\mathsf{G}}
\newcommand{\prfKl}{\prf.\mathsf{kl}}
%
\newcommand{\pkeToKem}[0]{\mathbf{T}}
\newcommand{\QpkeToKem}[0]{\mathbf{T}}
\newcommand{\pkeCiph}[0]{C}
\newcommand{\kemCiph}[0]{C^*}
%
\newcommand{\kemScheme}{\schemefont{KE}}
\newcommand{\kemKg}{\schalg{\kemScheme}{K}}
\newcommand{\kemEnc}{\schalg{\kemScheme}{E}}
\newcommand{\kemDec}{\schalg{\kemScheme}{D}}
\newcommand{\kemRl}{\schalg{\kemScheme}{rl}}
\newcommand{\kemKl}{\schalg{\kemScheme}{kl}}
\newcommand{\kemRoSp}{\schalg{\kemScheme}{FS}}

\newcommand{\FkemScheme}{\overline{\mathsf{KE}}}
\newcommand{\FkemRoSp}{\schalg{\FkemScheme}{FS}}
\newcommand{\FkemEnc}{\schalg{\FkemScheme}{E}}
%
\newcommand{\GkemScheme}[1]{\schemefont{KE}_{#1}}
\newcommand{\GkemEnc}[1]{\schalg{\GkemScheme{#1}}{E}}
%
\newcommand{\DFunc}[0]{\mathsf{D}}
\newcommand{\KLenFunc}[0]{\mathsf{k}^*}
%

\newcommand{\advQINDIFF}[2]{\genAdv{QINDIFF}{#1}{#2}}
\newcommand{\advTINDIFF}[2]{\genAdv{TINDIFF}{#1}{#2}}
\newcommand{\advINDIFF}[2]{\genAdv{indiff}{#1}{#2}}
\newcommand{\advRINDIFF}[2]{\genAdv{reset\text{-}indiff}{#1}{#2}}

\newcommand{\Func}[1][]{\mathcal{F}_{#1}}
\newcommand{\FuncHon}[1][]{\mathcal{F}_{#1}.hon}
\newcommand{\FuncAdv}[1][]{\mathcal{F}_{#1}.adv}

\newcommand{\lazysample}{\mathsf{ls}}

\newcommand{\secPropPKE}[0]{\mathrm{S}_{\mathrm{pke}}}


%
% Game hops
%
\newcommand{\gamehopchange}[2][0.85]{\hspace{0pt}\smash{\fbox{\rule[-1.75pt]{0pt}{#1\baselineskip}\smash{#2}}}} % frame around changes from one game to another, use optional parameter to adapt box height (as factor of \baselineskip), leading \hspace{0pt} prevents weird jump if nothing before \gamehopchange{..}

\newcommand{\supportQuT}[1]{\mathbf{sup}(#1)}
\newcommand{\algOutput}[0]{\mathrm{OUT}}
%
% FREE / DOMSEP predicates
%
\newcommand{\aPrefix}{P}
\newcommand{\prefix}{\preceq}
\newcommand{\listRO}{\mathcal{L}}

\newcommand{\gameFREE}[4]{\mGame{#3,FREE}{#4,#1,#2}{\advA}}
\newcommand{\gameFIXED}[4]{\mGame{#3,FIXED}{#4,#1,#2}{\advA}}
\newcommand{\gameDOMSEP}[5]{\mGame{#2,#3,DOMSEP}{#4,#5,#1}{\advA}}

\newcommand{\ngameCORR}[1]{\nGame{CORR}{#1}}
\newcommand{\ngamePRF}[1]{\nGame{prf}{#1}}

\newcommand{\ngameINDCCA}[1]{\nGame{ind-cca}{#1}}
\newcommand{\ngameINDCPA}[1]{\nGame{INDCPA}{#1}}
\newcommand{\ngameOWCPA}[1]{\nGame{OWCPA}{#1}}
\newcommand{\ngameOWPCA}[1]{\nGame{OWPCA}{#1}}
\newcommand{\ngameDOM}[1]{\nGame{wdom}{#1}}
\newcommand{\ngameQINDIFF}[1]{\nGame{QINDIFF}{#1}}
\newcommand{\ngameTINDIFF}[2]{\nGame{TINDIFF-{#1}}{#2}}

\newcommand{\DRTransform}[0]{\mathbf{DR}}
\newcommand{\RTransform}[0]{\mathbf{R}}
\newcommand{\InsTransform}[0]{\mathbf{Inst}}

\newcommand{\Aset}[2]{\{#1\,:\,#2\}}
\newcommand{\domain}[0]{\mathcal{D}}
\newcommand{\rangeSet}[0]{\mathcal{R}}
\newcommand{\domainFunc}[0]{\mathcal{F}}
\newcommand{\usedDomain}[0]{\mathcal{U}}
\newcommand{\SchemeQuery}{\mathsf{sq}}
\newcommand{\restrict}[2]{#1|_{#2}}

\newcommand{\pfFunctor}[1]{\construct{F}_{\mathrm{pf}(#1)}}
\newcommand{\ldFunctor}[0]{\construct{F}_{\mathrm{ld}}}
\newcommand{\idFunctor}[0]{\construct{F}_{\mathrm{id}}}
\newcommand{\splFunctor}[0]{\construct{F}_{\mathrm{spl}}}

\newcommand{\queryRO}[0]{\mathbf{Q}}
\newcommand{\GenqueryRO}[1]{\mathbf{#1}}
\newcommand{\FixedprefixqueryRO}[1]{\QuT_{\mathrm{pf}(#1)}}
\newcommand{\FixedprefixanswerRO}[1]{\AnT_{\mathrm{pf}(#1)}}
\newcommand{\FixedprefixqueryInv}[1]{\QuTInv_{\mathrm{pf}(#1)}}
\newcommand{\FixedprefixanswerInv}[1]{\AnTInv_{\mathrm{pf}(#1)}}

\newcommand{\VarprefixqueryRO}{\GenqueryRO{QPv}}
\newcommand{\LengthqueryRO}{\QuT_{\mathrm{ld}}}
\newcommand{\LengthanswerRO}{\AnT_{\mathrm{ld}}}
\newcommand{\IdqueryRO}{\QuT_{\mathrm{id}}}
\newcommand{\IdanswerRO}{\AnT_{\mathrm{id}}}
\newcommand{\IdqueryInv}{\QuTInv_{\mathrm{id}}}
\newcommand{\IdanswerInv}{\AnTInv_{\mathrm{id}}}
\newcommand{\SplittingqueryRO}{\QuT_{\mathrm{spl}}}
\newcommand{\SplittingqueryInv}{\QuTInv_{\mathrm{spl}}}
\newcommand{\SplittinganswerRO}{\AnT_{\mathrm{spl}}}
\newcommand{\SplittinganswerInv}{\AnTInv_{\mathrm{spl}}}
\newcommand{\answerRO}{\GenqueryRO{AT}}
\newcommand{\answerInv}{\answerRO^{-1}}
\newcommand{\NIntrusivequeryRO}{\GenqueryRO{QNI}}
\newcommand{\IntrusivequeryRO}{\GenqueryRO{QI}}
\newcommand{\AddgqueryRO}[1]{\GenqueryRO{QAGx}[#1]}
\newcommand{\IDqueryRO}{\GenqueryRO{QId}}

\newcommand{\Outputsplitting}{\mathbf{OS}}


\newcommand{\FComb}[1]{[[#1]]}
\newcommand{\aFunc}[1]{\mathit{#1}}
\newcommand{\aFuncDom}[0]{\mathsf{Dom}}
\newcommand{\aFuncRng}[0]{\mathsf{Rng}}

\newcommand{\bFunc}[1]{\mathrm{#1}}
\newcommand{\bFuncDom}[0]{\mathsf{Dom}}
\newcommand{\bFuncRng}[0]{\mathsf{Rng}}

\newcommand{\encode}[1]{\langle #1\rangle}
\newcommand{\Encode}{\texttt{Encode}}


%\newcommand{\AllFuncs}[2]{\llbracket #1\rightarrow#2\rrbracket}
\newcommand{\AllFuncs}[2]{\mathrm{FUNC}(#1,#2)}
\newcommand{\AllSOLFuncs}[2]{\mathrm{SOL}(#1,#2)}
\newcommand{\AllXOLFuncs}[1]{\mathrm{XOL}(#1)}
\newcommand{\FuncSp}[1]{\mathsf{#1}}
\newcommand{\aRO}[0]{\aFunc{F}}
\newcommand{\roSp}[0]{\mathsf{FS}}
\newcommand{\GenroSp}[1]{\mathsf{#1}}
\newcommand{\GenroSpCardinality}[1]{\GenroSp{#1}.\mathsf{n}}
\newcommand{\GenroSpFuncs}[1]{\GenroSp{#1}.\mathsf{F}}
\newcommand{\GenroSpDom}[1]{\mathrm{Dom}(\GenroSp{#1})}
\newcommand{\GenroSpDomP}[1]{\mathrm{Dom}_{*}(\GenroSp{#1})}
\newcommand{\GenroSpRng}[1]{\mathrm{Rng}(\GenroSp{#1})}
\newcommand{\roSpCardinality}[0]{\roSp.\mathsf{n}}
\newcommand{\roSpFuncs}[0]{\roSp.\mathsf{Fncs}}
\newcommand{\roSpDom}[0]{\mathrm{Dom}(\roSp)}
\newcommand{\roSpRng}[0]{\mathrm{Rng}(\roSp)}
\newcommand{\GGenroSpCardinality}[1]{{#1}.\mathsf{n}}
\newcommand{\GGenroSpFuncs}[1]{{#1}.\mathsf{Fncs}}
\newcommand{\GGenroSpDom}[1]{\mathrm{Dom}({#1})}
\newcommand{\GGenroSpDomP}[1]{\mathrm{Dom}_{*}({#1})}
\newcommand{\GGenroSpRng}[1]{\mathrm{Rng}({#1})}

\newcommand{\appendRO}[2]{{#1}\!\!+\!\!{#2}}
\newcommand{\chopRO}[1]{\mathrm{Chop}(#1)}
\newcommand{\addRO}[1]{\mathrm{Add}(#1)}
\newcommand{\expandRO}[1]{\mathrm{Expand}(#1)}
\newcommand{\compressRO}[1]{\mathrm{Compress}(#1)}

%\newcommand{\SHA}[1]{\mathsf{SHA#1}}
\newcommand{\SHAA}[2]{\mathsf{SHA#1}\mbox{-}\mathsf{#2}}
\newcommand{\SHAKE}[1]{\mathsf{SHAKE#1}}
\newcommand{\cSHAKE}{\mathsf{cSHAKE}}

\newcommand{\ulheading}[1]{\medskip\noindent\textsf{\underline{\smash{#1}}}}
\newcommand{\uldheading}[2]{\medskip\noindent\textsf{\underline{\smash{#1:}} #2}}

% \newcommand{\pqcname}[1]{\texttt{#1}}
% \newcommand{\pqcheading}[1]{\medskip\noindent\underline{\smash{\pqcname{#1}}}}
\newcommand{\pqcnameRoundOne}[1]{\texttt{\color{darkgray}#1}}
\newcommand{\pqcnameRoundTwo}[1]{\texttt{\fontseries{b}\selectfont#1}}

\newcommand{\functionOut}{e}
\newcommand{\functionIn}{s}
\newcommand{\functionInSet}{SS}
\newcommand{\functionOutSet}{ES}

\newcommand{\construct}[1]{\mathbf{#1}}
\newcommand{\constructDom}[1]{\construct{#1}.\mathsf{Dom}}
\newcommand{\constructEv}[1]{\construct{#1}.\mathsf{Ev}}
\newcommand{\constructRng}[1]{\construct{#1}.\mathsf{Rng}}

\newcommand{\commoncoins}{st}
\newcommand{\ccell}{st\ell}
\newcommand{\SimgenCC}{\simulator.\algfont{Setup}}
\newcommand{\Simeval}{\simulator.\algfont{Ev}}
\newcommand{\GenSimgenCC}[1]{\Gensimulator{#1}.\algfont{Setup}}
\newcommand{\GenSimeval}[1]{\Gensimulator{#1}.\algfont{Ev}}
\newcommand{\ccsample}{\mathsf{sample}}

\newcommand{\UHF}{H} % universal hash function

\newcommand{\functionality}{\mathfrak{f}}
\newcommand{\functionalityClass}{\mathfrak{F}}
\newcommand{\priv}{\procfont{priv}}
\newcommand{\pub}{\procfont{pub}}
\newcommand{\FnO}{\procfont{FnO}}
\newcommand{\functionalityPriv}{\functionality_{\priv}}
\newcommand{\functionalityPub}{\functionality_{\pub}}


\newcommand{\ngameCCINDIFF}[1]{\nGame{rd\mbox{-}indiff}{#1}}
\newcommand{\AdvCCINDIFF}[2]{\genAdv{rd\mbox{-}indiff}{#1}{#2}}
\newcommand{\indccaAdv}[2]{\genAdv{ind\mbox{-}cca}{#1}{#2}}
\newcommand{\wdomAdv}[2]{\genAdv{wdom}{#1}{#2}}


\newcommand{\ngameTI}[1]{\nGame{ti}{#1}}
\newcommand{\AdvTI}[2]{\genAdv{ti}{#1}{#2}}

\newcommand{\ngameEXEC}[1]{\nGame{exec}{#1}}

\newcommand{\ngameMSReal}[1]{\nGame{ms\mbox{-}1}{#1}}
\newcommand{\ngameMSIdeal}[1]{\nGame{ms\mbox{-}0}{#1}}
\newcommand{\ngameMS}[1]{\nGame{ms}{#1}}

\newcommand{\env}{\algfont{FG}}

\newcommand{\execoracle}{OR}
\newcommand{\execfinalize}{\mathsf{Finalize}}

\newcommand{\workDom}[0]{{\cal W}}
\newcommand{\WDInv}[0]{\mathrm{In}}

\newcommand{\QuT}[0]{\mathsf{QT}}
\newcommand{\AnT}[0]{\mathsf{AT}}
\newcommand{\QuTInv}[0]{\mathsf{QTI}}
\newcommand{\AnTInv}[0]{\mathsf{ATI}}

\newcommand{\Codewords}{\mathcal{C}}

\newcommand{\lncsorfull}[1]{{#1}}
\DeclareMathVersion{normal1}
\renewcommand*{\backref}[1]{(Cited on page~#1.)}
\DeclareMathAlphabet{\mathsc}{OT1}{cmr}{m}{sc}
\DeclareMathAlphabet{\mathslbf}{OT1}{cmr}{bx}{sl}

\setlength{\fboxsep}{1pt}

% ====================================================================
%
\newcommand{\gamesfontsize}{\small}
\newcommand{\gameName}[1]{\underline{$\main$ #1}\\[2pt]}


\newcommand{\twoColsNoDivide}[4]{
\begin{center}
        \framebox{
        \begin{tabular}{c@{\hspace*{.4em}}c@{\hspace*{.4em}}c}
        \begin{minipage}[t]{#1\textwidth}\setstretch{1.0}\gamesfontsize #3 \end{minipage}
        &
        \begin{minipage}[t]{#2\textwidth}\setstretch{1.0}\gamesfontsize #4 \end{minipage}
        \end{tabular}
        }
\end{center}
}

\newcommand{\oneCol}[2]{
\begin{center}
        \framebox{
        \begin{tabular}{c}
        \begin{minipage}[t]{#1\textwidth}\setstretch{1.0}\gamesfontsize #2 \end{minipage}
        \end{tabular}
        }
\end{center}
}


\newcommand{\twoCols}[4]{
\begin{center}
        \framebox{
        \begin{tabular}{c@{\hspace*{.4em}}|c@{\hspace*{.4em}}}
        \begin{minipage}[t]{#1\textwidth}\setstretch{1.0}\gamesfontsize #3 \end{minipage}
        &
        \begin{minipage}[t]{#2\textwidth}\setstretch{1.0}\gamesfontsize #4 \end{minipage}
        \end{tabular}
        }
\end{center}
}
%
%% Theorem environments
%
\newtheorem{thm}{Theorem}
\newtheorem{lem}{Lemma}


\newenvironment{theorem}{\begin{thm}}{\end{thm}}
\newenvironment{lemma}{\begin{lem}}{\end{lem}}

\newcommand{\secref}[1]{Section~\ref{#1}}
\newcommand{\apref}[1]{Appendix~\ref{#1}}
\newcommand{\thref}[1]{Theorem~\ref{#1}}

\newcommand{\figref}[1]{Figure~\ref{#1}}
\newcommand{\tabref}[1]{Table~\ref{#1}}

%%%fancy proof env %%%
\def\qedsym{\hspace{1pt}\rule[-1pt]{3pt}{9pt}}

\newlength{\saveparindent}
\setlength{\saveparindent}{\parindent}
\newlength{\saveparskip}
\setlength{\saveparskip}{\parskip}


\def\qsym{\vrule width0.6ex height1em depth0ex}
\newcount\proofqeded
\newcount\proofended
\def\qed{{\hspace{1pt}\rule[-1pt]{3pt}{9pt}}
\end{rm}\addtolength{\parskip}{-0pt}
\setlength{\parindent}{\saveparindent}
\global\advance\proofqeded by 1 }
\def\qedenv{
\end{rm}\addtolength{\parskip}{-0pt}
\setlength{\parindent}{\saveparindent}
\global\advance\proofqeded by 1 }
\renewenvironment{proof}%
 {\proofstart}%
 {\ifnum\proofqeded=\proofended~\qed\fi \global\advance\proofended by 1
  \medskip}
\newenvironment{proofenv}%
 {\proofenvstart}%
 {\ifnum\proofqeded=\proofended\qedenv\fi \global\advance\proofended by 1
  \medskip}
\makeatletter
\def\proofstart{\@ifnextchar[{\@oprf}{\@nprf}}
\def\proofenvstart{\@ifnextchar[{\@osprf}{\@nsprf}}
\def\@oprf[#1]{\begin{rm}\protect\vspace{6pt}\noindent{\bf Proof of #1:\ }%
\addtolength{\parskip}{5pt}\setlength{\parindent}{0pt}}
\def\@osprf[#1]{\begin{rm}\protect\vspace{6pt}\noindent
\addtolength{\parskip}{5pt}\setlength{\parindent}{0pt}}
\def\@nprf{\begin{rm}\protect\vspace{6pt}\noindent{\bf Proof:\ }%
\addtolength{\parskip}{5pt}\setlength{\parindent}{0pt}}
\def\@nsprf{\begin{rm}\protect\vspace{6pt}\noindent%
\addtolength{\parskip}{5pt}\setlength{\parindent}{0pt}}


% ========================================================================

% Lists

\newcounter{ctr}
\newenvironment{bignewenum}{%
\begin{list}{{\bf (\arabic{ctr})}\hfill}{\usecounter{ctr} \labelwidth=17pt%
\labelsep=7pt \leftmargin=24pt \topsep=4pt%
\setlength{\listparindent}{\saveparindent}%
\setlength{\parsep}{\saveparskip}%
\setlength{\itemsep}{4pt} }}{\end{list}}


\newenvironment{newmath}{\begin{displaymath}%
\setlength{\abovedisplayskip}{4pt}%
\setlength{\belowdisplayskip}{4pt}%
\setlength{\abovedisplayshortskip}{5pt}%
\setlength{\belowdisplayshortskip}{5pt} }{\end{displaymath}}

% =========================================================================


%\makeatletter
%\def\appearsin#1{\gdef\@appearsin{#1}}
%\def\maketitle{\par
% \begingroup
% \def\thefootnote{\arabic{footnote}}
% \def\@makefnmark{\hbox
% to 0pt{$^{\@thefnmark}$\hss}}
% \if@twocolumn
% \twocolumn[\@maketitle]
% \else \newpage
% \global\@topnum\z@ \@maketitle \fi\thispagestyle{plain}\@thanks
% \endgroup
% \setcounter{footnote}{0}
% \let\maketitle\relax
% \let\@maketitle\relax
% \gdef\@thanks{}\gdef\@author{}\gdef\@title{}\gdef\@appearsin{}
%          \let\thanks\relax}
%\def\@maketitle{\newpage
% \noindent \@appearsin
% \vskip 0.5in \begin{center}
% {\LARGE \@title \par} \vskip 1.5em {\large \lineskip .5em
%\begin{tabular}[t]{c}\@author
% \end{tabular}\par}
% \vskip 1em {\normalsize \@date} \end{center}
% \par
% \vskip 1.5em}
%\def\abstract{\if@twocolumn
%\section*{Abstract}
%\else \small
%\begin{center}
%{\bf Abstract\vspace{-.5em}\vspace{0pt}}
%\end{center}
%\quotation
%\fi}
%\def\endabstract{\if@twocolumn\else\endquotation\fi}
%\mark{{}{}}
%


% =========================================================================

% General
%\newcommand{\hindent}{\quad}

\newcommand{\mysubsubsection}[1]{\subsubsection{#1}}
\newcommand{\headingb}[1]{{\vspace{5pt}\noindent\textbf{#1.}}}
\newcommand{\headingg}[1]{{\sc{#1}}}
\newcommand{\heading}[1]{{\vspace{5pt}\noindent\sc{#1}}}

\newcommand{\tuple}[1]{\langle{#1}\rangle}
\def\from{\mbox{from}\ }
\def\From{\mbox{From}\ }
\def\bits{\{0,1\}}
\def\cross{\times}
\newcommand{\cab}{,\allowbreak}
\newcommand{\xor}{\oplus}
\newcommand{\Colon}{{:\;\;}}
\def\emptystring{\varepsilon}

\newcommand{\OL}[0]{\mathsf{OL}}

\newcommand{\Dom}{\mathsf{Dom}}
\newcommand{\DomR}{\mathsf{DomR}}
\newcommand{\Rng}{\mathsf{Rng}}
\newcommand{\RngR}{\mathsf{RngR}}
\newcommand{\calA}{{\cal A}}
\newcommand{\calB}{{\cal B}}
\newcommand{\calC}{{\cal C}}
\newcommand{\calF}{{\cal F}}
\newcommand{\N}{{{\mathds N}}}
\newcommand{\Z}{{{\mathds Z}}}
\newcommand{\R}{{{\mathds R}}}
\def\union{\cup}
\def\intersection{\cap}
\newcommand{\set}[2]{\{#1 \,:\, #2\}}
\newcommand{\getsr}{{{\,\leftarrow{\hspace*{-3pt}\raisebox{.75pt}{$\scriptscriptstyle\$$}}\,}}}
\newcommand{\OR}{{\textstyle\bigvee}}
\newcommand{\Var}{{\mbox{\bf Var}}}
\newcommand{\E}{\mathbf{E}}
\def\e{{\epsilon}}
\newcommand{\concat}{{\,\|\,}}
% ========================================================================

% ===================================================================
\newcommand{\comment}[1]{\hspace{8pt}{\small /$\!\!$/\ #1}}
\newcommand{\Comment}[1]{\hspace{5pt}{/$\!\!$/\ #1}}
\newcommand{\ccomment}[1]{\hspace{1pt}{/$\!\!$/\ #1}}

\newcommand{\authnote}[2]{ \begin{center}\fbox{\begin{minipage}{5.7in}
\textbf{#1 says:} #2\end{minipage}}\end{center}}

\newcommand{\nGame}[2]{\mathbf{G}^{\mathrm{#1}}_{#2}}
\newcommand{\pk}{pk}
\newcommand{\dk}{dk}
\newcommand{\FSOutSet}[0]{\mathrm{Out}}

\newcommand{\schemefont}[1]{{\mathsf{#1}}}
\newcommand{\algfont}[1]{{\mathrm{#1}}}
\newcommand{\Adv}{\mathbf{Adv}}
\newcommand{\genAdv}[3]{\mathbf{Adv}^{\mathrm{#1}}_{#2}(#3)}
\newcommand{\prfAdv}[1]{\genAdv{prf}{#1}}

\newcommand{\advA}{\mathcal{A}}
\newcommand{\advB}{\mathcal{B}}
\newcommand{\advC}{\mathcal{C}}
\newcommand{\advD}{\mathcal{D}}
\newcommand{\INDCCA}{\mbox{IND-CCA}\xspace}

\newcommand{\atk}{\mathrm{atk}}
\newcommand{\cpa}{\mathrm{cpa}}
\newcommand{\cca}{\mathrm{cca}}
\newcommand{\ind}{\mathrm{ind}}

\addtolength{\belowcaptionskip}{-2mm}
\addtolength{\abovecaptionskip}{-3mm}
\addtolength{\topsep}{-1mm}
\newcommand{\schalg}[2]{\mathsf{#1.#2}}


\newcommand{\procfont}[1]{\mathsc{#1}}
\newcommand{\tablefont}[1]{\mathsf{#1}}
\newcommand{\Initialize}{\Init}
\newcommand{\Finalize}{\Fin}
\newcommand{\Init}{\procfont{init}}
\newcommand{\Fin}{\procfont{fin}}

\newcommand{\RO}{\procfont{RO}}
\newcommand{\ROSim}{\procfont{ROSim}}
\newcommand{\FUNC}{\procfont{FUNC}}
\newcommand{\DecO}{\procfont{Dec}}

\newcommand{\win}{\mathsf{win}}

\newcommand{\Fimage}[0]{\mathrm{Im}}
\newcommand{\simulator}[0]{\mathsf{Sim}}
\newcommand{\Gensimulator}[1]{\mathsf{S}_{#1}}


\newcommand{\group}{\mathds{G}}
\newcommand{\wins}{\textrm{ wins}}
\newcommand{\true}{\mathsf{true}}
\newcommand{\false}{\mathsf{false}}
\newcommand{\AND}{\;\wedge\;}
\newcommand{\bad}{\mathsf{bad}}
\newcommand{\chal}{\mathsf{chal}}
\newcommand{\Good}{\mathsf{Good}}
\newcommand{\Gm}{\textnormal{G}}
\newcommand{\fn}{\footnotesize}

\newcommand{\vecx}{\mathbf{x}}
\newcommand{\pfvec}[0]{\mathbf{p}}
\newcommand{\vecY}{\mathslbf{Y}}
\newcommand{\vecV}{\mathslbf{V}}
\newcommand{\vecU}{\mathslbf{U}}
\newcommand{\vecW}{\mathslbf{W}}
\newcommand{\vecxx}[0]{\mathbf{x}}
\newcommand{\Oracle}{\mathsc{O}}
\newcommand{\pkeScheme}{\schemefont{PKE}}
\newcommand{\pkeEnc}{\schalg{\pkeScheme}{E}}
\newcommand{\pkeDecR}{\schalg{\pkeScheme}{DecR}}
\newcommand{\pkeML}{\schalg{\pkeScheme}{ml}}
\newcommand{\prf}{\mathsf{G}}
\newcommand{\prfKl}{\prf.\mathsf{kl}}
%
\newcommand{\pkeToKem}[0]{\mathbf{T}}
\newcommand{\QpkeToKem}[0]{\mathbf{T}}
\newcommand{\pkeCiph}[0]{C}
\newcommand{\kemCiph}[0]{C^*}
%
\newcommand{\kemScheme}{\schemefont{KE}}
\newcommand{\kemKg}{\schalg{\kemScheme}{K}}
\newcommand{\kemEnc}{\schalg{\kemScheme}{E}}
\newcommand{\kemDec}{\schalg{\kemScheme}{D}}
\newcommand{\kemRl}{\schalg{\kemScheme}{rl}}
\newcommand{\kemKl}{\schalg{\kemScheme}{kl}}
\newcommand{\kemRoSp}{\schalg{\kemScheme}{FS}}

\newcommand{\FkemScheme}{\overline{\mathsf{KE}}}
\newcommand{\FkemRoSp}{\schalg{\FkemScheme}{FS}}
\newcommand{\FkemEnc}{\schalg{\FkemScheme}{E}}
%
\newcommand{\GkemScheme}[1]{\schemefont{KE}_{#1}}
\newcommand{\GkemEnc}[1]{\schalg{\GkemScheme{#1}}{E}}
%
\newcommand{\DFunc}[0]{\mathsf{D}}
\newcommand{\KLenFunc}[0]{\mathsf{k}^*}
%

\newcommand{\advQINDIFF}[2]{\genAdv{QINDIFF}{#1}{#2}}
\newcommand{\advTINDIFF}[2]{\genAdv{TINDIFF}{#1}{#2}}
\newcommand{\advINDIFF}[2]{\genAdv{indiff}{#1}{#2}}
\newcommand{\advRINDIFF}[2]{\genAdv{reset\text{-}indiff}{#1}{#2}}

\newcommand{\Func}[1][]{\mathcal{F}_{#1}}
\newcommand{\FuncHon}[1][]{\mathcal{F}_{#1}.hon}
\newcommand{\FuncAdv}[1][]{\mathcal{F}_{#1}.adv}

\newcommand{\lazysample}{\mathsf{ls}}

\newcommand{\secPropPKE}[0]{\mathrm{S}_{\mathrm{pke}}}


%
% Game hops
%
\newcommand{\gamehopchange}[2][0.85]{\hspace{0pt}\smash{\fbox{\rule[-1.75pt]{0pt}{#1\baselineskip}\smash{#2}}}} % frame around changes from one game to another, use optional parameter to adapt box height (as factor of \baselineskip), leading \hspace{0pt} prevents weird jump if nothing before \gamehopchange{..}

\newcommand{\supportQuT}[1]{\mathbf{sup}(#1)}
\newcommand{\algOutput}[0]{\mathrm{OUT}}
%
% FREE / DOMSEP predicates
%
\newcommand{\aPrefix}{P}
\newcommand{\prefix}{\preceq}
\newcommand{\listRO}{\mathcal{L}}

\newcommand{\gameFREE}[4]{\mGame{#3,FREE}{#4,#1,#2}{\advA}}
\newcommand{\gameFIXED}[4]{\mGame{#3,FIXED}{#4,#1,#2}{\advA}}
\newcommand{\gameDOMSEP}[5]{\mGame{#2,#3,DOMSEP}{#4,#5,#1}{\advA}}

\newcommand{\ngameCORR}[1]{\nGame{CORR}{#1}}
\newcommand{\ngamePRF}[1]{\nGame{prf}{#1}}

\newcommand{\ngameINDCCA}[1]{\nGame{ind-cca}{#1}}
\newcommand{\ngameINDCPA}[1]{\nGame{INDCPA}{#1}}
\newcommand{\ngameOWCPA}[1]{\nGame{OWCPA}{#1}}
\newcommand{\ngameOWPCA}[1]{\nGame{OWPCA}{#1}}
\newcommand{\ngameDOM}[1]{\nGame{wdom}{#1}}
\newcommand{\ngameQINDIFF}[1]{\nGame{QINDIFF}{#1}}
\newcommand{\ngameTINDIFF}[2]{\nGame{TINDIFF-{#1}}{#2}}

\newcommand{\DRTransform}[0]{\mathbf{DR}}
\newcommand{\RTransform}[0]{\mathbf{R}}
\newcommand{\InsTransform}[0]{\mathbf{Inst}}

\newcommand{\Aset}[2]{\{#1\,:\,#2\}}
\newcommand{\domain}[0]{\mathcal{D}}
\newcommand{\rangeSet}[0]{\mathcal{R}}
\newcommand{\domainFunc}[0]{\mathcal{F}}
\newcommand{\usedDomain}[0]{\mathcal{U}}
\newcommand{\SchemeQuery}{\mathsf{sq}}
\newcommand{\restrict}[2]{#1|_{#2}}

\newcommand{\pfFunctor}[1]{\construct{F}_{\mathrm{pf}(#1)}}
\newcommand{\ldFunctor}[0]{\construct{F}_{\mathrm{ld}}}
\newcommand{\idFunctor}[0]{\construct{F}_{\mathrm{id}}}
\newcommand{\splFunctor}[0]{\construct{F}_{\mathrm{spl}}}

\newcommand{\queryRO}[0]{\mathbf{Q}}
\newcommand{\GenqueryRO}[1]{\mathbf{#1}}
\newcommand{\FixedprefixqueryRO}[1]{\QuT_{\mathrm{pf}(#1)}}
\newcommand{\FixedprefixanswerRO}[1]{\AnT_{\mathrm{pf}(#1)}}
\newcommand{\FixedprefixqueryInv}[1]{\QuTInv_{\mathrm{pf}(#1)}}
\newcommand{\FixedprefixanswerInv}[1]{\AnTInv_{\mathrm{pf}(#1)}}

\newcommand{\VarprefixqueryRO}{\GenqueryRO{QPv}}
\newcommand{\LengthqueryRO}{\QuT_{\mathrm{ld}}}
\newcommand{\LengthanswerRO}{\AnT_{\mathrm{ld}}}
\newcommand{\IdqueryRO}{\QuT_{\mathrm{id}}}
\newcommand{\IdanswerRO}{\AnT_{\mathrm{id}}}
\newcommand{\IdqueryInv}{\QuTInv_{\mathrm{id}}}
\newcommand{\IdanswerInv}{\AnTInv_{\mathrm{id}}}
\newcommand{\SplittingqueryRO}{\QuT_{\mathrm{spl}}}
\newcommand{\SplittingqueryInv}{\QuTInv_{\mathrm{spl}}}
\newcommand{\SplittinganswerRO}{\AnT_{\mathrm{spl}}}
\newcommand{\SplittinganswerInv}{\AnTInv_{\mathrm{spl}}}
\newcommand{\answerRO}{\GenqueryRO{AT}}
\newcommand{\answerInv}{\answerRO^{-1}}
\newcommand{\NIntrusivequeryRO}{\GenqueryRO{QNI}}
\newcommand{\IntrusivequeryRO}{\GenqueryRO{QI}}
\newcommand{\AddgqueryRO}[1]{\GenqueryRO{QAGx}[#1]}
\newcommand{\IDqueryRO}{\GenqueryRO{QId}}

\newcommand{\Outputsplitting}{\mathbf{OS}}


\newcommand{\FComb}[1]{[[#1]]}
\newcommand{\aFunc}[1]{\mathit{#1}}
\newcommand{\aFuncDom}[0]{\mathsf{Dom}}
\newcommand{\aFuncRng}[0]{\mathsf{Rng}}

\newcommand{\bFunc}[1]{\mathrm{#1}}
\newcommand{\bFuncDom}[0]{\mathsf{Dom}}
\newcommand{\bFuncRng}[0]{\mathsf{Rng}}

\newcommand{\encode}[1]{\langle #1\rangle}
\newcommand{\Encode}{\texttt{Encode}}


%\newcommand{\AllFuncs}[2]{\llbracket #1\rightarrow#2\rrbracket}
\newcommand{\AllFuncs}[2]{\mathrm{FUNC}(#1,#2)}
\newcommand{\AllSOLFuncs}[2]{\mathrm{SOL}(#1,#2)}
\newcommand{\AllXOLFuncs}[1]{\mathrm{XOL}(#1)}
\newcommand{\FuncSp}[1]{\mathsf{#1}}
\newcommand{\aRO}[0]{\aFunc{F}}
\newcommand{\roSp}[0]{\mathsf{FS}}
\newcommand{\GenroSp}[1]{\mathsf{#1}}
\newcommand{\GenroSpCardinality}[1]{\GenroSp{#1}.\mathsf{n}}
\newcommand{\GenroSpFuncs}[1]{\GenroSp{#1}.\mathsf{F}}
\newcommand{\GenroSpDom}[1]{\mathrm{Dom}(\GenroSp{#1})}
\newcommand{\GenroSpDomP}[1]{\mathrm{Dom}_{*}(\GenroSp{#1})}
\newcommand{\GenroSpRng}[1]{\mathrm{Rng}(\GenroSp{#1})}
\newcommand{\roSpCardinality}[0]{\roSp.\mathsf{n}}
\newcommand{\roSpFuncs}[0]{\roSp.\mathsf{Fncs}}
\newcommand{\roSpDom}[0]{\mathrm{Dom}(\roSp)}
\newcommand{\roSpRng}[0]{\mathrm{Rng}(\roSp)}
\newcommand{\GGenroSpCardinality}[1]{{#1}.\mathsf{n}}
\newcommand{\GGenroSpFuncs}[1]{{#1}.\mathsf{Fncs}}
\newcommand{\GGenroSpDom}[1]{\mathrm{Dom}({#1})}
\newcommand{\GGenroSpDomP}[1]{\mathrm{Dom}_{*}({#1})}
\newcommand{\GGenroSpRng}[1]{\mathrm{Rng}({#1})}

\newcommand{\appendRO}[2]{{#1}\!\!+\!\!{#2}}
\newcommand{\chopRO}[1]{\mathrm{Chop}(#1)}
\newcommand{\addRO}[1]{\mathrm{Add}(#1)}
\newcommand{\expandRO}[1]{\mathrm{Expand}(#1)}
\newcommand{\compressRO}[1]{\mathrm{Compress}(#1)}

%\newcommand{\SHA}[1]{\mathsf{SHA#1}}
\newcommand{\SHAA}[2]{\mathsf{SHA#1}\mbox{-}\mathsf{#2}}
\newcommand{\SHAKE}[1]{\mathsf{SHAKE#1}}
\newcommand{\cSHAKE}{\mathsf{cSHAKE}}

\newcommand{\ulheading}[1]{\medskip\noindent\textsf{\underline{\smash{#1}}}}
\newcommand{\uldheading}[2]{\medskip\noindent\textsf{\underline{\smash{#1:}} #2}}

% \newcommand{\pqcname}[1]{\texttt{#1}}
% \newcommand{\pqcheading}[1]{\medskip\noindent\underline{\smash{\pqcname{#1}}}}
\newcommand{\pqcnameRoundOne}[1]{\texttt{\color{darkgray}#1}}
\newcommand{\pqcnameRoundTwo}[1]{\texttt{\fontseries{b}\selectfont#1}}

\newcommand{\functionOut}{e}
\newcommand{\functionIn}{s}
\newcommand{\functionInSet}{SS}
\newcommand{\functionOutSet}{ES}

\newcommand{\construct}[1]{\mathbf{#1}}
\newcommand{\constructDom}[1]{\construct{#1}.\mathsf{Dom}}
\newcommand{\constructEv}[1]{\construct{#1}.\mathsf{Ev}}
\newcommand{\constructRng}[1]{\construct{#1}.\mathsf{Rng}}

\newcommand{\commoncoins}{st}
\newcommand{\ccell}{st\ell}
\newcommand{\SimgenCC}{\simulator.\algfont{Setup}}
\newcommand{\Simeval}{\simulator.\algfont{Ev}}
\newcommand{\GenSimgenCC}[1]{\Gensimulator{#1}.\algfont{Setup}}
\newcommand{\GenSimeval}[1]{\Gensimulator{#1}.\algfont{Ev}}
\newcommand{\ccsample}{\mathsf{sample}}

\newcommand{\UHF}{H} % universal hash function

\newcommand{\functionality}{\mathfrak{f}}
\newcommand{\functionalityClass}{\mathfrak{F}}
\newcommand{\priv}{\procfont{priv}}
\newcommand{\pub}{\procfont{pub}}
\newcommand{\FnO}{\procfont{FnO}}
\newcommand{\functionalityPriv}{\functionality_{\priv}}
\newcommand{\functionalityPub}{\functionality_{\pub}}


\newcommand{\ngameCCINDIFF}[1]{\nGame{rd\mbox{-}indiff}{#1}}
\newcommand{\AdvCCINDIFF}[2]{\genAdv{rd\mbox{-}indiff}{#1}{#2}}
\newcommand{\indccaAdv}[2]{\genAdv{ind\mbox{-}cca}{#1}{#2}}
\newcommand{\wdomAdv}[2]{\genAdv{wdom}{#1}{#2}}


\newcommand{\ngameTI}[1]{\nGame{ti}{#1}}
\newcommand{\AdvTI}[2]{\genAdv{ti}{#1}{#2}}

\newcommand{\ngameEXEC}[1]{\nGame{exec}{#1}}

\newcommand{\ngameMSReal}[1]{\nGame{ms\mbox{-}1}{#1}}
\newcommand{\ngameMSIdeal}[1]{\nGame{ms\mbox{-}0}{#1}}
\newcommand{\ngameMS}[1]{\nGame{ms}{#1}}

\newcommand{\env}{\algfont{FG}}

\newcommand{\execoracle}{OR}
\newcommand{\execfinalize}{\mathsf{Finalize}}

\newcommand{\workDom}[0]{{\cal W}}
\newcommand{\WDInv}[0]{\mathrm{In}}

\newcommand{\QuT}[0]{\mathsf{QT}}
\newcommand{\AnT}[0]{\mathsf{AT}}
\newcommand{\QuTInv}[0]{\mathsf{QTI}}
\newcommand{\AnTInv}[0]{\mathsf{ATI}}

\newcommand{\Codewords}{\mathcal{C}}

\newcommand{\lncsorfull}[1]{{#1}}
\DeclareMathVersion{normal1}
% Start the document
\begin{document}
% Begin with frontmatter and so forth
\frontmatter
\maketitle
\makecopyright
\makesignature
% Optional
\begin{dedication}
\setsinglespacing
\raggedright % It would be better to use \RaggedRight from ragged2e
\parindent0pt\parskip\baselineskip
In recognition of reading this manual before beginning to format the
doctoral dissertation or master's thesis; for following the
instructions written herein; for consulting with OGS Academic Affairs
Advisers; and for not relying on other completed manuscripts, this
manual is dedicated to all graduate students about to complete the
doctoral dissertation or master's thesis.

In recognition that this is my one chance to use whichever
justification, spacing, writing style, text size, and/or textfont that
I want to while still keeping my headings and margins consistent.
\end{dedication}
% Optional
% \begin{epigraph}
% \vskip0pt plus.5fil
% \setsinglespacing
% {\flushright
% True ease in writing comes from art, not chance,\\
% As those move easiest who have learn'd to dance.\\
% 'T is not enough to no harshness gives offence,---\\
% The sound must seem an echo to the sense.

% \vskip\baselineskip
% \textit{Alexander Pope}\par}
% \vfil
% \begin{center}
% You write with ease to show your breeding,\\
% But easy writing's curst hard reading.

% \vskip\baselineskip
% \textit{Richard Brinsley Sheridan}
% \end{center}
% \vfil
% \noindent Writing, at its best, is a lonely life. Organizations for
% writers palliate the writer's loneliness, but I doubt if they improve
% his writing. He grows in public stature as he sheds his loneliness and
% often his work deteriorates. For he does his work alone and if he is a
% good enough writer he must face eternity, or the lack of it, each day.

% \vskip\baselineskip
% \hskip0pt plus1fil\textit{Ernest Hemingway}\hskip0pt plus4fil\null

% \vfil
% \end{epigraph}

% Next comes the table of contents, list of figures, list of tables,
% etc. If you have code listings, you can use \listoflistings (or
% \lstlistoflistings) to have it be produced here as well. Same with
% \listofalgorithms.
\tableofcontents
\listoffigures
\listoftables

% Preface
% \begin{preface}
% Almost nothing is said in the manual about the preface. There is no
% indication about how it is to be typeset. Given that, one is forced to
% simply typeset it and hope it is accepted. It is, however, optional
% and may be omitted.
% \end{preface}

% Your fancy acks here. Keep in mind you need to ack each paper you
% use. See the examples here. In addition, each chapter ack needs to
% be repeated at the end of the relevant chapter.
\begin{acknowledgements}
%I would like to acknowledge Professor Eta Theta for his support as the
%chair of my committee. Through multiple drafts and many long nights,
%his guidance has proved to be invaluable.
%
%I would also like to acknowledge the ``Smith Clan'' of lab~28, without
%whom my research would have no doubt taken fives times as long. It is
%their support that helped me in an immeasureable way.
%
%Chapter 2, in full, is a reprint of the material as it appears in
%Numerical Grid Generational in Computational Fluid Mechanics~2009.
%Smith, Laura; Smith, Jane~D., Pineridge Press,~2009. The dissertation
%author was the primary investigator and author of this paper.
%
%Chapter 3, in part, has been submitted for publication of the material
%as it may appear in Education Mechanics,~2009, Smith, Laura; Smith,
%Jane~D., Trailor Press,~2009. The dissertation author was the primary
%investigator and author of this paper.
%
%Chapter 5, in part is currently being prepared for submission for
%publication of the material. Smith, Laura; Smith, Jane~D\@. The
%dissertation author was the primary investigator and author of this
%material.
Placeholder for acknowledgments.
\end{acknowledgements}

% Stupid vita goes next
\begin{vita}
\noindent
\begin{cv}{}
\begin{cvlist}{}
\item[2014] Bachelor of Mathematics, University of Minnesota, Twin Cities
\item[2014] Bachelor of Computer Science, University of Minnesota, Twin Cities 
\item[2019--2021] Teaching Assistant, Department of Computer Science\\University of California, San Diego
\item[2020] Master of Computer Science, University of California, San Diego
\item[2018--2023] Research Assistant, University of California, San
Diego
\item[2023] Doctor of Philosophy, University of California, San Diego
\end{cvlist}
\end{cv}

% This puts in the PUBLICATIONS header. Note that it appears inside
% the vita environment. It is optional.
\publications
\noindent ``Power Dissipation in Fractal AC Circuits'' Chen J., Rogers J., Anderson L., Andrews U.,
Brzoska A., Coffey A., Davis H., Fisher L., Hansalik M., Loew S., Teplyaev A. Journal of
Physics A: Mathematical and Theoretical vol. 50, num. 32, 2017.

\noindent ``Separate Your Domains: NIST PQC KEMs, Oracle Cloning and Read-Only Indifferentiability'' Bellare M., Davis
H., G{\"{u}}nther F. Proceedings of 39th Annual International Conference on the Theory and Applications of Cryptographic
Techniques (EUROCRYPT), 2020.

\noindent ``Tighter Proofs for the SIGMA and TLS 1.3 Key Exchange Protocols'' Davis H., G{\"{u}}nther F.
Proceedings of International Conference on Applied Cryptography and Network Security (ACNS), 2021.

\noindent ``On the Concrete Security of {TLS} 1.3 {PSK} Mode'' Davis H., Diemert D., G{\"{u}}nther F., Jager T.
Proceedings of 41st Annual International
Conference on the Theory and Applications of Cryptographic Techniques (EUROCRYPT), 2022.

\noindent ``Hardening Signature Schemes via Derive-then-Derandomize: Stronger
Security Proofs for EdDSA'' Bellare M., Davis H., Di Z. Proceedings of 26th {IACR} International Conference
on Practice and Theory of Public-Key Cryptography (PKC), 2023.

\noindent ``Verifiable Distributed Aggregation Functions'' Davis H., Patton C., Rosulek M., Schoppmann P.
Proceedings of Privacy Enhancing Technology Symposium (PETS), 2023.

% This puts in the FIELDS OF STUDY. Also inside vita and also
% optional.
% \fieldsofstudy
% \noindent Major Field: Engineering (Specialization or Focused Studies)
% \vskip\baselineskip
% Studies in Applied Mathematics\par
% Professors Alpha Beta and Gamma Delta
% \vskip\baselineskip
% Studies in Mechanices\par
% Professors Epsilon Zeta and Eta Theta
% \vskip\baselineskip
% Studies in Electromagnetism\par
% Professors Iota Kappa and Lambda Mu
\end{vita}

% Put your maximum 350 word abstract here.
\begin{dissertationabstract}
	
	Cryptographic standards published by organizations like NIST, ISO, and the
	IETF provide guidance for developers choosing and implementing cryptographic
	algorithms for their applications. In recent years, formal proofs of security
	have become an important part of validation for standardized algorithms;
	however, these proofs rely on abstractions which sometimes differ
	significantly from the schemes and protocols used in practice.

	In this work, I will begin with a study of the ongoing NIST standardization
	process of post-quantum key-encapsulation mechanisms and highlight
	vulnerabilities in several (former) candidate algorithms which arise from a
	systematic mismatch between abstract primitives used in cryptographic models
	 and their actual instantiation in implementations. I will then present a
	 library of secure instantiation techniques and a way to extend schemes'
	 existing proofs to their instantiations. Next, I will address the Transport
	Layer Security (TLS 1.3) Handshake Protocol and demonstrate by a concrete
	evaluation that prior work fails to prove practical security levels for many
	of the standardized parameter sets. I will then show tighter proofs that do
	justify these parameter sets and which additionally give the first fully
	justified abstraction of the TLS 1.3 key schedule in the random oracle model,
	and I will explain how certain parts of the TLS 1.3 design hinder the
	application of useful abstractions.
	
	I will also explain how inaccurate portrayals of hash functions in the random
	oracle model impact the security analysis of the standardized EdDSA signature
	scheme and present an improved proof of security with better tightness and modularity.
	I conclude by introducing my work on the proposed standard for privacy-preserving
	measurement, including a new security model for Verifiable Distributed
	Aggregation Functions. Within this model, I discuss results for Prio3, an optimized
	version of the massively scalable, widely used Prio construction for private
	data collection, and Doplar, a new construction for private histogram generation.
\end{dissertationabstract}

% This is where the main body of your dissertation goes!
\mainmatter

% Optional Introduction
\begin{dissertationintroduction}
\section{Introduction}\label{sec:intro}

In designing schemes, and proving them secure, theoreticians implicitly assume certain things, such as on-demand fresh randomness and correct implementation. In practice, these assumptions can fail. Weaknesses in system random-number generators are common and have catastrophic consequences. (An example relevant to this paper is the well-known key-recovery attack on $\Schnorr$ signatures when signing reuses randomness. Another striking example are Ps and Qs attacks~\cite{USENIX:HDWH12,EPRINT:LHABKW12}.) Meanwhile, implementation errors can be exploited, as shown by Bleichenbacher's attack on RSA signatures~\cite{Bl-rump-C06}. 

In light of this, practitioners may try to ``harden'' theoretical schemes before standardization and usage. A prominent and highly successful instance is $\EdDSA$, a hardening of the $\Schnorr$ signature scheme proposed by Bernstein, Duif, Lange, Schwabe, and Yang (BDLSY)~\cite{bernstein2012high}. It incorporates explicit, simple key-derivation, makes signing deterministic, adds protection against sidechannel attacks via ``clamping,'' and for simplicity confines itself to a single hash function, namely $\SHAfive$. The scheme is widely standardized~\cite{NIST:EdDSA,RFC:EDDSA} and used~\cite{Ed-uses}.

There is however a subtle danger here, namely that the hardening attempt introduces new vulnerabilities. In other words, hardening needs to be done right; if not, it may even ``soften'' the scheme! Thus it is crucial that the hardened scheme be vetted via a proof of security. This is of particular importance for $\EdDSA$ given its widespread deployment. In that regard, Brendel, Cremers, Jackson and Zhao (BCJZ)~\cite{SP:BCJZ21} showed that $\EdDSA$ is secure if the Discrete-Log (DL) problem is hard and the hash function is modeled as a random oracle. This is significant as a first step but has at least two important limitations: (1) Due to the extension attack, a random oracle is not an appropriate model for the $\SHAfive$ hash function $\EdDSA$ actually uses, and (2) the reduction is so loose that there is no security guarantee for group sizes in use today. 

Extrapolating $\EdDSA$, the first part of this paper defines a general hardening transform on signature schemes called Derive-then-Derandomize ($\DRTransform$), and proves its soundness. Next we prove the indifferentiability of a general class of constructions, that we call shrink-MD; %  they apply a ``shrinking'' output transform to the result of an MD-style hash function. 
it includes the well-studied chop-MD construction~\cite{C:CDMP05} and also the modulo-a-prime construction arising in $\EdDSA$. 
% Furthermore our proof closes a gap in earlier analyses of chop-MD~\cite{hfrobook}. 
Armed with these results, the second part of the paper returns to give new proofs for $\EdDSA$ that in particular fill the above gaps.
We begin with some background.

% As part of a broader treatment that delivers more general results, our work will fill these gaps. 


 
\heading{Respecting hash structure in proofs.} Recall that the MD-transform~\cite{C:Merkle89a,C:Damgaard89b} defines a hash function $\HH = \construct{MD}[\compF] \Colon\bits^*\to\bits^{2k}$ by iterating an underlying compression function $\compF\Colon\bits^{b+2k}\to\bits^{2k}$. (See Section~\ref{sec-prelims} for details.) $\SHAtwo$ and $\SHAfive$ are obtained in this way, with $(b, k)$ being $(512, 128)$ and $(1024, 256)$, respectively. This structure gives rise to attacks, of which the most well known is the extension attack. The latter allows an attacker given $t \gets \MD[\compF](e_2\|M)$, where $e_2$ is a secret unknown to the attacker and $M\in\bits^*$ is public, to compute compute $t'=\MD[\compF](e_2\|M')$, for some $M'\in\bits^{*}$ of its choice. This has been exploited to violate the UF-security of the so-called prefix message authentication code $\pfMAC_{e_2}(M) = \HH(e_2\|M)$ when $\HH$ is an MD-hash function; $\HMAC$~\cite{C:BelCanKra96} was designed to overcome this.

A proof of security of a scheme (such as $\EdDSA$) that uses a hash function $\HH$ will often model $\HH$ as a random oracle~\cite{CCS:BelRog93}, in what we'll call the $(\HH,\HH)$-model: scheme algorithms, and the adversary, both have oracle access to the same random $\HH$.  However the presence of the above-discussed structure in ``real'' hash functions led Dodis, Ristenpart and Shrimpton (DRS) \cite{EC:DodRisShr09} to argue that the ``right'' model in which to prove security of a scheme that uses $\HH = \construct{MD}[\compF]$ is to model the compression function $\compF$ ---rather than the hash function $\HH=\construct{MD}[\compF]$--- as a random oracle. We'll call this the $(\construct{MD}[\compF],\compF)$-model: the adversary has oracle access to a random $\compF$, with scheme algorithms having access to $\construct{MD}[\compF]$. There is now widespread agreement with the DRS thesis that proofs of security of MD-hash-using schemes should use the $(\construct{MD}[\compF],\compF)$ model.

Giving from-scratch proofs in the $(\construct{MD}[\compF],\compF)$ model is, however, difficult. Maurer, Renner and Holenstein (MRH)~\cite{TCC:MauRenHol04} show that if a construction $\construct{F}$ is indifferentiable (abbreviated indiff) and a scheme is secure in the $(\HH,\HH)$ model, then it remains secure in the $(\construct{F}[\compF],\compF)$ model. (This requires the game defining security of the scheme to be single-stage~\cite{EC:RisShaShr11}, which is true for the relevant ones here.)
Unfortunately, $\construct{F}=\MD$ is provably \textit{not} indiff~\cite{C:CDMP05}, due exactly to the extension attack. So the MRH result does not help with $\MD$. This led to a search for indiff variants. DRS~\cite{EC:DodRisShr09} and YMO~\cite{yoneyama2009leaky} (independently) offer public-indiff and show that it suffices to prove security, in the $(\construct{MD}[\compF],\compF)$ model, of schemes that use $\construct{MD}$ in some restricted way. However, $\EdDSA$ does not obey these restrictions. Thus, other means are needed.




%\begin{figure}[t]
%\twoCols{0.35}{0.55}
%{
%  \begin{algorithm-initial}{$\DS_{\Schnorr}.\Kg$}
%  \item $\s \getsr \Z_{\Prime}$ ; $\curvepoint{A} \gets \s\cdot \generator$
%  \item Return $(\curvepoint{A},\s)$
%  \end{algorithm-initial}
%  \ExptSepSpace
%\begin{algorithm-subsequent}{$\DS_{\Schnorr}.\Sign[\HH](\s, \curvepoint{A}, \msg)$}
%\item $r \getsr \Z_{\Prime}$ ; $\curvepoint{R} \gets \littler\cdot \generator$
%\item $c \gets \HH(\curvepoint{R}\|\curvepoint{A}\| \msg)$
%\item $\z \gets (\s c + \littler) \bmod \Prime$
%\item Return $(\curvepoint{R}, \z)$
%  \end{algorithm-subsequent}
%  \ExptSepSpace
%  \begin{algorithm-subsequent}{$\DS_{\Schnorr}.\Vf[\HH](\curvepoint{A}, \msg, \sigma)$}
%  \item $(\curvepoint{R}, \z) \gets \sigma$
%  ;  $c \gets \HH(\curvepoint{R}\|\curvepoint{A}\| \msg)$
%  \item Return ($\z \cdot\generator = \curvepoint{R}+ c\cdot\curvepoint{A} $)
%  \end{algorithm-subsequent}  \vspace{2pt}
% }
% {
%	\begin{algorithm-subsequent}{$\DS_{\EdDSA}.\Kg[\HH]$}
%		\item $\sk \getsr \bits^{k}$ 
%		;  $\e \gets \HH(\sk)$ 
%		% \item $\sk\gets  \HH(\emptystring)$ 
%		; $\e_1 \gets \e[1..k]$ 
%		\item $\s \gets \Clamp(\e_1)$ ; $\curvepoint{A} \gets \s\cdot \generator$
%		\item Return $(\curvepoint{A}, \sk)$
%	\end{algorithm-subsequent}\ExptSepSpace
%	\begin{algorithm-subsequent}{$\DS_{\EdDSA}.\Sign[\HH](\sk,\curvepoint{A}, \msg)$}
%		\item $\e \gets \HH(\sk)$ 
%		; $\e_1 \gets \e[1..k]$ ; $\e_2 \gets \e[k\! +\! 1..2k]$ 
%		\item $\s \gets \Clamp(\e_1)$ %; $\curvepoint{A} \gets \s\cdot \generator$
%		\item $\littler \gets \HH(\e_2\|\msg) \bmod \Prime$ 
%		; $\curvepoint{R} \gets \littler \cdot \generator$
%		\item $c \gets \HH(\curvepoint{R}\|\curvepoint{A}\|\msg) \bmod \Prime$
%		; $\z \gets (\s c + \littler) \bmod \Prime$
%		\item Return $(\curvepoint{R},\z)$
%	\end{algorithm-subsequent}\ExptSepSpace
%	\begin{algorithm-subsequent}{$\DS_{\EdDSA}.\Vf[\HH](\curvepoint{A}, \msg, \sigma)$}
%		\item $(\curvepoint{R}, \z) \gets \sigma$
%		; $c \gets \HH(\curvepoint{R}\|\curvepoint{A}\|\msg) \bmod \Prime$
%		\item Return ($\z\cdot \generator = \curvepoint{R}+ c\cdot\curvepoint{A} $)
%	\end{algorithm-subsequent}  \vspace{2pt}
%	}
%\vspace{-8pt}
%\caption{On the left is the $\Schnorr$ signature scheme. On the right is $\EdDSA$. Here $\generator$ is a generator of an additively-written group $\group$ that has prime order $\Prime$. $\HH$ is a hash function. $\Clamp$ is the $\EdDSA$ clamping function.}
%\label{fig-eddsa-intro}
%\hrulefill
%\vspace{-10pt}
%\end{figure}

 \heading{The $\EdDSA$ scheme.} The Edwards curve Digital Signature Algorithm ($\EdDSA$) is a Schnorr-based signature scheme introduced by Bernstein, Duif, Lange, Schwabe and Yang~\cite{bernstein2012high}. $\Edtwo$, which uses the Curve25519 Edwards curve and $\SHAfive$ as the hash function, is its most popular instance. The scheme is standardized by NIST~\cite{NIST:EdDSA} and the IETF~\cite{RFC:EDDSA}. It is used in TLS 1.3, OpenSSH, OpenSSL, Tor, GnuPGP, Signal and WhatsApp. It is also the preferred signature scheme of the Corda, Tezos, Stellar and Libra blockchain systems. Overall, IANIX~\cite{Ed-uses} reports over 200 uses of $\Edtwo$. Proving security of this scheme is accordingly of high importance.
 
  Figure~\ref{fig-eddsa} shows $\EdDSA$ on the right, and, on the left, the classic $\Schnorr$ scheme~\cite{JC:Schnorr91} on which $\EdDSA$ is based. % (Note there are several $\Schnorr$ variants, that differ in details. See Section~\ref{sec-schemes} for a discussion of how the one used here relates to others.)
  The schemes are over a cyclic, additively-written group $\group$ of prime order $\Prime$ with generator $\generatorEDSA$. The public verification key is $\curvepoint{A}$. The $\Schnorr$ hash function has range $\Z_{\Prime} = \{0,\ldots,\Prime-1\}$, while, for $\EdDSA$, function $\HH_1$ has range $\bits^{2k}$ where $k$, the bit-length of $\Prime$, is $256$ for $\Edtwo$. Functions $\HH_2,\HH_3$ have range $\Z_{\Prime}$.

$\EdDSA$ differs from $\Schnorr$ in significant ways. While the $\Schnorr$ secret key $\s$ is in $\Z_{\Prime}$, the $\EdDSA$ secret key $\sk$ is a $k$-bit string. This is hashed and the $2k$-bit result is split into $k$-bit halves $e_1\|e_2$. A Schnorr secret-key $\s$ is derived by applying to $e_1$ a clamping function $\Clamp$ that zeroes out the three least significant bits of $e_1$. (Note: This means $\s$ is \textit{not} uniformly distributed over $\Z_{\Prime}$.) Clamping increases resistance to side-channel attacks~\cite{bernstein2012high}. Signing is made deterministic by a standard de-randomization technique~\cite{C:Goldreich86a,SAC:MNPV98,AC:BelPoeSte16,EC:BelTac16}, namely obtaining the Schnorr randomness $\littler$ by hashing the message $\msg$ with a secret-key dependent string $e_2$. We note that all of $\HH_1,\HH_2,\HH_3$ are instantiated via the same hash function, namely $\SHAfive$.

 % function $\HH$ is used in three ways for three purposes: to derive keys (lines~1,12), to de-randomize signing (line~14) and, as in $\Schnorr$, for the Fiat-Shamir hashing (line~15)~\cite{C:FiaSha86}. 


% We write the scheme as using three hash functions $\HH_1,\HH_2,\HH_3$, but ---this is important for our proposed work--- in $\Edtwo$ they are all $\SHAfive$. 

\heading{Prior work and our questions.} Recall that the security goal for a signature scheme is UF (UnForgeability under Chosen-Message Attack)~\cite{GolMicRiv88}. $\Schnorr$ is well studied, and proven UF under $\DLP$ (Discrete Log in $\group$) when $\HH$ is a random oracle~\cite{JC:PoiSte00,EC:AABN02}. The provable security of $\EdDSA$, however, received surprisingly little attention until the work of Brendel, Cremers, Jackson and Zhao (BCJZ)~\cite{SP:BCJZ21}. They take the path also used for $\Schnorr$ and other identification-based signature schemes~\cite{JC:PoiSte00,EC:AABN02}, seeing $\EdDSA$ as the result of the Fiat-Shamir transform on an underlying identification scheme $\EdID$ that they define, proving security of the latter under $\DLP$, and concluding UF of $\EdDSA$ under $\DLP$ when $\HH$ is a random oracle. This is an important step forward, but the BCJZ proof~\cite{SP:BCJZ21} remains in the $(\HH,\HH)$ model. We ask and address the following two questions.


\medskip
\textbf{1. Can we prove security in the $(\MD[\compF],\compF)$ model?} The NIST standard~\cite{NIST:EdDSA} mandates that $\Edtwo$ uses $\SHAfive$, which is an MD-hash function. Accordingly, as explained above, the BCJZ proof~\cite{SP:BCJZ21}, being in the $(\HH,\HH)$ model, does not guarantee security; to do the latter, we need a proof in the $(\MD[\compF],\compF)$ model.

The gap is more than cosmetic. As we saw above with the example of the prefix MAC, a scheme could be secure in the $(\HH,\HH)$ model, yet totally insecure in the more realistic $(\MD[\compF],\compF)$ model, and thus also in practice. And $\EdDSA$ skirts close to the edge: line~14 is using the prefix-MAC that the extension attack breaks, and overlaps in inputs across the three uses of $\HH$ could lead to failures. Intuitively what prevents attacks is that the MAC outputs are taken modulo $\Prime$, and inputs to $\HH$ in two of the three uses involve secrets. Thus, we'd expect that the scheme is indeed secure in the $(\MD[\compF],\compF)$ model. 

Proving this, however, is another matter. We already know that $\construct{MD}$ is not indiff. It is public indiff~\cite{EC:DodRisShr09,yoneyama2009leaky}, but this will not suffice for $\EdDSA$ because $\HH_1,\HH_2$ are being called on secrets. We ask, first, can $\EdDSA$ be proved secure in the $(\MD[\compF],\compF)$ model, and second, can this be done in some modular way, rather than from scratch?

\medskip
\textbf{2. Can we improve reduction tightness?} The reduction of BCJZ~\cite{SP:BCJZ21} is so loose that, in the 256-bit curve over which $\Edtwo$ is implemented, it guarantees little security. Let's elaborate. Given an adversary $\advAUF$ violating the UF-security of $\EdDSA$ with probability $\epsUF$, the reduction builds an adversary $\advADL$ breaking $\DLP$ with probability \smash{$\epsDL = \epsUF^2/q_h$} where $q_h$ is the number of $\HH$-queries of $\advAUF$ and the two adversaries have about the same running time $t$. (The square arises from the use of rewinding, analyzed via the Reset Lemma of~\cite{C:BelPal02}.) In an order $\Prime$ elliptic curve group, \smash{$\epsDL \approx t^2/p$} so we get \smash{$\epsUF = t\cdot \sqrt{q_h/p}$}. $\Edtwo$ has \smash{$p\approx 2^{256}$}. Say $t=q_h=2^{70}$, which (as shown by BitCoin mining capability) is not far from attacker reach. Then $\epsDL = 2^{-116}$ is small but \smash{$\epsUF = 2^{70}\cdot 2^{-(256-70)/2} = 2^{-23}$} is in comparison quite high. 

Now, one might say that one would not expect better because the same reduction loss is present for $\Schnorr$. The classical reductions for $\Schnorr$ \cite{JC:PoiSte00,EC:AABN02} did indeed display the above loss, but that has changed: recent advances for $\Schnorr$ include a tighter reduction from $\DLP$~\cite{C:RotSeg21}, an almost-tight reduction from the MBDL problem~\cite{INDOCRYPT:BelDai20} and a tight reduction from $\DLP$ in the Algebraic Group Model~\cite{EC:FucPloSeu20}. We'd like to put $\EdDSA$ on par with the state of the art for $\Schnorr$. We ask, first, is this possible, and second, is there a modular way to do it that leverages, rather than repeats, the (many, complex) just-cited proofs for $\Schnorr$? 

% \authnote{Reviwer}{Goal 1 is very clear.  Goal 2 not as much.  Setting $q_h = 2^{80}$ seems extreme.  }{red}

%\medskip
%\textbf{3. Weak multi-user security.} UF-security, the target in all the above, pertains to the classical setting where there is just one user (key) under attack. There is now broad acceptance that the more realistic setting is the multi-user one~\cite{EC:BelBolMic00,galbraith2002public,menezes2004security}, where there are $u\geq 1$ users, with independent keys. (This in particular is the setting in TLS.) Denoting security here by MUF, if $\epsMUF,\epsUF$ are the MUF and UF advantages (success probabilities), respectively, a standard hybrid argument shows that $\epsMUF \leq u\cdot\epsUF$~\cite{galbraith2002public}. But with just $u = 300$ users (TLS has way more), and $\epsUF=2^{-8}$ as above, this fails to even guarantee $\epsMUF<1$. 

%Now, for $\Schnorr$, one can do better than the hybrid argument; it is known that $\epsMUF\approx \epsUF$~\cite{EPRINT:Bernstein15,C:KilMasPan16}. Can we show the same for $\EdDSA$?


\heading{Contributions for $\EdDSA$.} We simultaneously simplify and strengthen the security proofs for $\EdDSA$ as follows.  
\smallskip

\textbf{1. Reduction from $\Schnorr$.} Rather than, as in prior work, give a reduction from $\DLP$ or some other algebraic problem, we give a simple, direct reduction from $\Schnorr$ itself. That is, we show that if the $\Schnorr$ signature scheme is UF-secure, then so is $\EdDSA$. Furthermore, the reduction is \textit{tight} up to a constant factor. This allows us to leverage prior work~\cite{C:RotSeg21,INDOCRYPT:BelDai20,EC:FucPloSeu20} to obtain tight proofs for $\EdDSA$ under various algebraic assumptions and justify security for group sizes in actual use. But there are two further dividends. First, $\Schnorr$~\cite{JC:Schnorr91} is over 30 years old and has withstood the tests of time and cryptanalysis, so our proof that $\EdDSA$ is just as secure as $\Schnorr$ allows the former to inherit, and benefit from, this confidence. Second, our result formalizes and proves what was the intuition and belief in the first place~\cite{bernstein2012high}, namely that, despite the algorithmic differences, $\EdDSA$ is a sound hardening of $\Schnorr$.
\smallskip

\textbf{2. Accurate modeling of the hash function.} As noted above, BCJZ~\cite{SP:BCJZ21} assume the hash function $\HH$ is a random oracle, but
% , as Coron, Dodis, Malinaud and Puniya (CDMP)~\cite{C:CDMP05} explained, 
this, due to the extension attack, is not an accurate model for the MD-hash function $\SHAfive$ used by $\EdDSA$. We fill this gap by instead proving security in the $(\MD[\compF],\compF)$ model, where $\HH=\MD[\hh]$ is derived via the MD-transform~\cite{C:Merkle89a,C:Damgaard89b} and the compression function $\hh$ is a random oracle.
% We explain why this is \textit{not} obtained directly by combining the BCJZ result with indifferentiability or public indifferentiability. 
  

\begin{sloppypar}
\heading{Approach and broader contributions.} The above-mentioned results on $\EdDSA$ are obtained as a consequence of more general ones.
\end{sloppypar}
\smallskip
\textbf{3. The $\DRTransform$ transform and its soundness.}
We extend the hardening technique used in $\EdDSA$ to define a general transform that we call Derive-then-Derandomize ($\DRTransform$). It takes an \textit{arbitrary} signature scheme $\DS$, and with the aid of a PRG $\HH_1$ and a PRF $\HH_2$, constructs a hardened signature scheme $\fDS$. We provide (Theorem~\ref{th-dd}) a strong and general validation of $\DRTransform$, showing that $\fDS$ is UF-secure assuming $\DS$ is UF-secure. Moreover \textit{the reduction is tight} and the proof is simple. This shows that the $\EdDSA$ hardening method is generically sound.

\smallskip
\textbf{4. Indifferentiability of Shrink-MD.} It is well-known that $\MD$ is not indifferentiable~\cite{TCC:MauRenHol04} from a random oracle, but that
the $\ChopMD$~\cite{C:CDMP05}, which truncates the output of an an $\MD$ hash by some number of bits, is indifferentiable.
Unfortunately, we identified gaps in two prominent proofs of indifferentiability of $\ChopMD$~\cite{C:CDMP05,hfrobook}.
$\EdDSA$ uses a similar construction that reduces the $\MD$ hash output modulo a prime $\Prime$ sufficiently smaller than the size of the range of $\MD$, due to which we refer to this construction as $\ModMD$.
The $\ModMD$ construction has not been proven indifferentiable.
We simultaneously give new proofs of indifferentiability for $\ChopMD$ and $\ModMD$ as part of a more general class of constructions that we call $\ShrinkMD$ functors.
These are constructions of the form $\Out(\MD)$ where $\Out$ is some output-processing function, and we prove indifferentiability under certain ``shrinking'' conditions on $\Out$.
  

\smallskip
\textbf{5. Application to $\EdDSA$.} $\EdDSA$ is obtained as the result $\fDS$ of the $\DRTransform$ transform applied to the $\DS=\Schnorr$ signature scheme, and with the PRG and PRF defined via $\MD$, specifically $\HH_1(\sk) = \MD[\hh](\sk)$ and $\HH_2(e_2,M) = \MD[\hh](e_2\|M)\bmod\Prime$ where $\Prime$ is the prime order of the underlying group. Additionally, the hash function used in $\Schnorr$ is also $\HH_3(X) = \MD[\hh](X)\bmod\Prime$. Due to Theorem~\ref{th-dd} validating $\DRTransform$, we are left to show the PRG security of $\HH_1$, the PRF security of $\HH_2$ and the UF-security of $\Schnorr$, all with $\hh$ modeled as a random oracle. We do the first directly. We obtain the second as a consequence of the indifferentiability of $\ModMD$. (In principle it follows from the PRF security of AMAC~\cite{EC:BelBerTes16}, but we found it difficult to extract precise bounds via this route.) For the third, we again exploit indifferentiability of $\ModMD$, together with a technique from BCJZ~\cite{SP:BCJZ21} to handle clamping, to reduce to the UF security of regular $\Schnorr$, where the hash function is modeled as a random oracle. Putting all this carefully together yields our above-mentioned results for $\EdDSA$. We note that one delicate and important point is that the idealized compression function $\hh$ is \textit{the same} across $\HH_1,\HH_2$ and $\HH_3$, meaning these are not independent. This is handled through the building blocks in Theorem~\ref{th-dd} being functors~\cite{EC:BelDavGun20} rather than functions.




%\heading{Our answers, in brief.} We give affirmative answers to both questions above. The first element of our approach is that our reduction for $\EdDSA$, rather than being from $\DLP$ or some other algebraic problem, is directly from $\Schnorr$ itself, and is \textit{tight} up to a constant factor. This in one step answers the second question (tightness) discussed above because we immediately inherit, for $\EdDSA$, the guarantees of the known tight(er) proofs of $\Schnorr$~\cite{C:RotSeg21,INDOCRYPT:BelDai20,EC:FucPloSeu20}. Our proof is, moreover, in the $(\MD[\compF],\compF)$-model. To obtain it in a modular way, we introduce filtered indifferentiability (f-indiff), show that f-indiff of $\MD$ suffices to prove security of $\EdDSA$, and separately establish f-indiff of $\MD$ (despite its lack of indiff) using a combination of new techniques and results from the indifferentiability literature~\cite{EC:DodRisShr09,C:CDMP05}. This answers the first question above. We now look at all this in more detail.
%
%\heading{Reduction from $\Schnorr$.} Let's write $\probP\reducesTo\DS$ to mean that we prove UF security of signature scheme $\DS$ with a reduction from (i.e., assuming hardness of) problem $\probP$ in the ROM. (The notation says nothing about tightness, which will be discussed separately.) Thus, BCJZ~\cite{SP:BCJZ21} show that $\DLP\reducesTo\EdDSA$. We show instead that $\Schnorr\reducesTo\EdDSA$. That is, if $\Schnorr$ is secure, so is $\EdDSA$. Furthermore, while the BCJZ reduction is loose, ours is \textit{tight} up to a small constant factor. 
%
%The immediate dividend is that any (known, or even future) proof $\probP\reducesTo\Schnorr$ automatically yields, via our result, a proof $\probP\reducesTo\EdDSA$, with only a constant factor loss in tightness compared to the original proof. In particular we get a tighter $\DLP\reducesTo\EdDSA$ proof via~\cite{C:RotSeg21}, an almost tight $\MBDLP\reducesTo\EdDSA$ proof via~\cite{INDOCRYPT:BelDai20} and an up-to-constant tight $\DLP\reducesTo\EdDSA$ Algebraic Group Model proof via~\cite{EC:FucPloSeu20}. This answers the second question (tightness) above.
%
%%\authnote{Reviwer}{Not the biggest fan of this $X \reducesTo Y$ notation, which conceals both the model (e.g., ROM) and the tightness of the reduction, in a paper that focuses on both the model and the tightness of the reduction.   I'd really like this groundbreaking paper to be more precise/specific documenting the implications in that "Reduction from Schnorr" section.  }{red}
%
%But there are two further dividends. First, $\Schnorr$~\cite{JC:Schnorr91} is over 30 years old and has withstood the tests of time and cryptanalysis. Our proof that $\EdDSA$ is just as secure as $\Schnorr$ allows the former to inherit, and benefit from, this confidence. Second, our result formalizes and proves what was the intuition and belief for $\EdDSA$ in the first place, namely that, despite the algorithmic differences, it is ``the same'' as $\Schnorr$ in security.
%
%Our proof that $\Schnorr\reducesTo\EdDSA$ would be novel and interesting already even in the basic $(\HH,\HH)$ model. However, we actually prove this in the $(\MD[\compF],\compF)$ model, so that we also answer the first question above. We now turn to this.
%
%


%\heading{Outline of proof.} Our main result (Theorem~\ref{th-eddsa-md}) is a tight (up to a constant factor) $\Schnorr\reducesTo\EdDSA$ reduction in the $(\MD[\compF],\compF)$ model. (The constant is 16 when $k=256$.)  That is, we show UF-security of $\EdDSA$, assuming only UF-security of $\Schnorr$, even when the hash function used is an MD-style one like $\SHAfive$. The statement of this result does not involve f-indiff or filters; these arise only in the proof. The latter has a few steps that we now outline. A fuller explanation is in Section~\ref{sec-schemes} and a picture is in Figure~\ref{pic-indiff}.

% \authnote{Reviewer}{The repeated claim of being tight "up to a constant factor" for Theorem 2 has me curious what that constant is.  Why not be explicit?}{red}

%Let $\DS$ denote the target $\EdDSA$ scheme, whose security we consider in the UF game. The first step is to cast $\DS$ as an alternative scheme $\fDS$ (shown on the right in Figure~\ref{fig-eddsa}) whose security we consider in a filtered unforgeability (fUF) game that we define via Figure~\ref{fig:fUF}. The idea is that the role of the secret signing key is now played by the filter seed. The filter here is our $\FltEDDSA$ one discussed above, and Lemma~\ref{lm-fUF-eq-UF} says that the schemes have equivalent security. We now need to show fUF security of $\fDS$. 
%
%The next step, Theorem~\ref{th-use-findiff}, is an f-indiff composition theorem, in the vein of the indiff composition theorem of~\cite{TCC:MauRenHol04}. This reduces the task to two sub-tasks. The first is to show fUF security of $\fDS$ when the oracle called by the $\FltEDDSA$ filter is a random oracle rather than $\MD[\compF]$. That is, the scheme should be secure in an ``ideal,'' even if still filtered, setting. The second sub-task is to show that $\MD$ is f-indiff relative to the $\FltEDDSA$ filter.
%
%The first sub-task is handled by Theorem~\ref{th-ideal-eddsa}, and this is where $\Schnorr$ enters, the reduction being from the latter. As an abstraction boundary, we use a version of $\Schnorr$ in which the signing key is drawn, not uniformly, but via a key-generation algorithm that performs the $\EdDSA$ clamping. A technique of BCJZ~\cite{SP:BCJZ21} separately allows $\Schnorr$ itself to reduce to clamped $\Schnorr$ with the above constant factor loss in advantage. 
% (The constant is 16 when $k=256$.)

%The final and most technical step is Theorem~\ref{th-md-indiff}, showing that $\MD$ is f-indiff relative to the $\FltEDDSA$ filter. This first exploits the presence of the seed to reduce the task to showing a weak form of public indiff for $\MD$. It concludes by exploiting the public indifferentiability of $\MD$ as shown in DRS~\cite{EC:DodRisShr09}. An alternative proof would first generalize and extend the indiff of truncated $\MD$ shown in~\cite{C:CDMP05} to $\MD$ taken modulo a prime whose bit-length is half that of the $\MD$ output, and then reduce to this.

%\authnote{Reviewer}{Figure~\ref{fig-ge1} appears on page 17, but is really needed to understand the discussion here.  Also this page makes lots of other references (e.g., Lemma~\ref{lm-fUF-eq-UF}) that assume the reader is already familiar with the later main body material.  Can you make this more self-contained with pointers forward?}{red}

 
\heading{Discussion and related work.} Both BCJZ~\cite{SP:BCJZ21} and CGN~\cite{10.1007/978-3-030-64357-7_4} note that there are a few versions of $\EdDSA$ out there, the differences being in their verification algorithms. What Figure~\ref{fig-eddsa} shows is the most basic version of the scheme, but we will be able to cover the variants too, in a modular way, by reducing from $\Schnorr$ with the same verification algorithm.

BBT~\cite{EC:BelBerTes16} define the function $\AMAC[\compF]$ to take a key $e_2$ and message $M$, and return $\MD[\compF](e_2\|M)\bmod\Prime$. This is the $\HH_2$ in $\EdDSA$. We could exploit their results to conclude PRF security of $\HH_2$, but it requires putting together many different pieces from their work, and it is easier and more direct to establish PRF security of $\HH_2$ by using our lemma on the indifferentiability of $\ModMD$.

In the Generic Group Model (GGM)~\cite{EC:Shoup97}, it is possible to prove UF-security of $\Schnorr$ under standard (rather than random oracle) model assumptions on the hash functions~\cite{neven2009hash,C:CLMQ21}. But use of the GGM means the result applies to a limited class of adversaries. Our results, following the classical proofs for identification-based signatures~\cite{JC:PoiSte00,C:OhtOka98,EC:AABN02,C:KilMasPan16}, instead use the standard model for the group, while modeling the hash function (in our case, the compression function) as a random oracle. 

In an earlier version of this paper, our proofs had relied on a variant of indifferentiability that we had introduced. At the suggestion of a Crypto 2022 reviewer, this has been dropped in favor of a direct proof based on PRG and PRF assumptions on $\HH_1,\HH_2$. We thank the (anonymous) reviewer for this suggestion.

Theorem~\ref{th-dd} is in the standard model if the PRG, PRF and starting signature scheme $\DS$ are standard-model, hence can be viewed as a standard-model justification of the hardening template underlying $\EdDSA$. However, when we want to justify $\EdDSA$ itself, we need to consider the specific, $\MD$-based instantiations of the PRG, PRF and $\Schnorr$ hash function, and for these we use the model where the compression function is ideal.

Several works study de-randomization of signing by deriving the coins via a PRF applied to the message, considering different ways to key the PRF~\cite{C:Goldreich86a,SAC:MNPV98,AC:BelPoeSte16,EC:BelTac16}. We use their techniques in the proof of Theorem~\ref{th-dd}.

One might ask how to view the UF-security of $\Schnorr$ signatures as an assumption. What is relevant is not its form (it is interactive) but that (1) it can be seen as a hub from where one can bridge to other assumptions that imply it, such as DL (non-tightly)~\cite{JC:PoiSte00,EC:AABN02} or MBDL (tightly)~\cite{INDOCRYPT:BelDai20}, and (2) it is validated by decades of cryptanalysis.

Our results have been stated for UF but extend to SUF (Strong unforgeability), meaning our proofs also show SUF-security of $\EdDSA$ in the $(\MD[\compF],\compF)$ model assuming SUF security of $\Schnorr$, with a tight (up to the usual constant factor) reduction.

$\EdDSA$ could be used with other hash functions such as $\SHAKE{256}$. The extension attack does not apply to the latter, so the proof of BCJZ~\cite{SP:BCJZ21} applies, but gives a loose reduction from DL; our results still add something, namely a tight reduction from $\Schnorr$ and thus improved tightness in several ways as discussed above.

% Given the way our work uses code-based games, it could benefit from being cast in the state-separation~\cite{AC:BDFKK18} or constructive cryptography~\cite{ICS:MauRen11,TCC:MauRen16} frameworks. We leave these as directions for future work.










\end{dissertationintroduction}

\chapter{Separate Your Domains}\label{chap:domsep}
\mathversion{normal1}
\section{Introduction}\label{sec:intro}

In designing schemes, and proving them secure, theoreticians implicitly assume certain things, such as on-demand fresh randomness and correct implementation. In practice, these assumptions can fail. Weaknesses in system random-number generators are common and have catastrophic consequences. (An example relevant to this paper is the well-known key-recovery attack on $\Schnorr$ signatures when signing reuses randomness. Another striking example are Ps and Qs attacks~\cite{USENIX:HDWH12,EPRINT:LHABKW12}.) Meanwhile, implementation errors can be exploited, as shown by Bleichenbacher's attack on RSA signatures~\cite{Bl-rump-C06}. 

In light of this, practitioners may try to ``harden'' theoretical schemes before standardization and usage. A prominent and highly successful instance is $\EdDSA$, a hardening of the $\Schnorr$ signature scheme proposed by Bernstein, Duif, Lange, Schwabe, and Yang (BDLSY)~\cite{bernstein2012high}. It incorporates explicit, simple key-derivation, makes signing deterministic, adds protection against sidechannel attacks via ``clamping,'' and for simplicity confines itself to a single hash function, namely $\SHAfive$. The scheme is widely standardized~\cite{NIST:EdDSA,RFC:EDDSA} and used~\cite{Ed-uses}.

There is however a subtle danger here, namely that the hardening attempt introduces new vulnerabilities. In other words, hardening needs to be done right; if not, it may even ``soften'' the scheme! Thus it is crucial that the hardened scheme be vetted via a proof of security. This is of particular importance for $\EdDSA$ given its widespread deployment. In that regard, Brendel, Cremers, Jackson and Zhao (BCJZ)~\cite{SP:BCJZ21} showed that $\EdDSA$ is secure if the Discrete-Log (DL) problem is hard and the hash function is modeled as a random oracle. This is significant as a first step but has at least two important limitations: (1) Due to the extension attack, a random oracle is not an appropriate model for the $\SHAfive$ hash function $\EdDSA$ actually uses, and (2) the reduction is so loose that there is no security guarantee for group sizes in use today. 

Extrapolating $\EdDSA$, the first part of this paper defines a general hardening transform on signature schemes called Derive-then-Derandomize ($\DRTransform$), and proves its soundness. Next we prove the indifferentiability of a general class of constructions, that we call shrink-MD; %  they apply a ``shrinking'' output transform to the result of an MD-style hash function. 
it includes the well-studied chop-MD construction~\cite{C:CDMP05} and also the modulo-a-prime construction arising in $\EdDSA$. 
% Furthermore our proof closes a gap in earlier analyses of chop-MD~\cite{hfrobook}. 
Armed with these results, the second part of the paper returns to give new proofs for $\EdDSA$ that in particular fill the above gaps.
We begin with some background.

% As part of a broader treatment that delivers more general results, our work will fill these gaps. 


 
\heading{Respecting hash structure in proofs.} Recall that the MD-transform~\cite{C:Merkle89a,C:Damgaard89b} defines a hash function $\HH = \construct{MD}[\compF] \Colon\bits^*\to\bits^{2k}$ by iterating an underlying compression function $\compF\Colon\bits^{b+2k}\to\bits^{2k}$. (See Section~\ref{sec-prelims} for details.) $\SHAtwo$ and $\SHAfive$ are obtained in this way, with $(b, k)$ being $(512, 128)$ and $(1024, 256)$, respectively. This structure gives rise to attacks, of which the most well known is the extension attack. The latter allows an attacker given $t \gets \MD[\compF](e_2\|M)$, where $e_2$ is a secret unknown to the attacker and $M\in\bits^*$ is public, to compute compute $t'=\MD[\compF](e_2\|M')$, for some $M'\in\bits^{*}$ of its choice. This has been exploited to violate the UF-security of the so-called prefix message authentication code $\pfMAC_{e_2}(M) = \HH(e_2\|M)$ when $\HH$ is an MD-hash function; $\HMAC$~\cite{C:BelCanKra96} was designed to overcome this.

A proof of security of a scheme (such as $\EdDSA$) that uses a hash function $\HH$ will often model $\HH$ as a random oracle~\cite{CCS:BelRog93}, in what we'll call the $(\HH,\HH)$-model: scheme algorithms, and the adversary, both have oracle access to the same random $\HH$.  However the presence of the above-discussed structure in ``real'' hash functions led Dodis, Ristenpart and Shrimpton (DRS) \cite{EC:DodRisShr09} to argue that the ``right'' model in which to prove security of a scheme that uses $\HH = \construct{MD}[\compF]$ is to model the compression function $\compF$ ---rather than the hash function $\HH=\construct{MD}[\compF]$--- as a random oracle. We'll call this the $(\construct{MD}[\compF],\compF)$-model: the adversary has oracle access to a random $\compF$, with scheme algorithms having access to $\construct{MD}[\compF]$. There is now widespread agreement with the DRS thesis that proofs of security of MD-hash-using schemes should use the $(\construct{MD}[\compF],\compF)$ model.

Giving from-scratch proofs in the $(\construct{MD}[\compF],\compF)$ model is, however, difficult. Maurer, Renner and Holenstein (MRH)~\cite{TCC:MauRenHol04} show that if a construction $\construct{F}$ is indifferentiable (abbreviated indiff) and a scheme is secure in the $(\HH,\HH)$ model, then it remains secure in the $(\construct{F}[\compF],\compF)$ model. (This requires the game defining security of the scheme to be single-stage~\cite{EC:RisShaShr11}, which is true for the relevant ones here.)
Unfortunately, $\construct{F}=\MD$ is provably \textit{not} indiff~\cite{C:CDMP05}, due exactly to the extension attack. So the MRH result does not help with $\MD$. This led to a search for indiff variants. DRS~\cite{EC:DodRisShr09} and YMO~\cite{yoneyama2009leaky} (independently) offer public-indiff and show that it suffices to prove security, in the $(\construct{MD}[\compF],\compF)$ model, of schemes that use $\construct{MD}$ in some restricted way. However, $\EdDSA$ does not obey these restrictions. Thus, other means are needed.




%\begin{figure}[t]
%\twoCols{0.35}{0.55}
%{
%  \begin{algorithm-initial}{$\DS_{\Schnorr}.\Kg$}
%  \item $\s \getsr \Z_{\Prime}$ ; $\curvepoint{A} \gets \s\cdot \generator$
%  \item Return $(\curvepoint{A},\s)$
%  \end{algorithm-initial}
%  \ExptSepSpace
%\begin{algorithm-subsequent}{$\DS_{\Schnorr}.\Sign[\HH](\s, \curvepoint{A}, \msg)$}
%\item $r \getsr \Z_{\Prime}$ ; $\curvepoint{R} \gets \littler\cdot \generator$
%\item $c \gets \HH(\curvepoint{R}\|\curvepoint{A}\| \msg)$
%\item $\z \gets (\s c + \littler) \bmod \Prime$
%\item Return $(\curvepoint{R}, \z)$
%  \end{algorithm-subsequent}
%  \ExptSepSpace
%  \begin{algorithm-subsequent}{$\DS_{\Schnorr}.\Vf[\HH](\curvepoint{A}, \msg, \sigma)$}
%  \item $(\curvepoint{R}, \z) \gets \sigma$
%  ;  $c \gets \HH(\curvepoint{R}\|\curvepoint{A}\| \msg)$
%  \item Return ($\z \cdot\generator = \curvepoint{R}+ c\cdot\curvepoint{A} $)
%  \end{algorithm-subsequent}  \vspace{2pt}
% }
% {
%	\begin{algorithm-subsequent}{$\DS_{\EdDSA}.\Kg[\HH]$}
%		\item $\sk \getsr \bits^{k}$ 
%		;  $\e \gets \HH(\sk)$ 
%		% \item $\sk\gets  \HH(\emptystring)$ 
%		; $\e_1 \gets \e[1..k]$ 
%		\item $\s \gets \Clamp(\e_1)$ ; $\curvepoint{A} \gets \s\cdot \generator$
%		\item Return $(\curvepoint{A}, \sk)$
%	\end{algorithm-subsequent}\ExptSepSpace
%	\begin{algorithm-subsequent}{$\DS_{\EdDSA}.\Sign[\HH](\sk,\curvepoint{A}, \msg)$}
%		\item $\e \gets \HH(\sk)$ 
%		; $\e_1 \gets \e[1..k]$ ; $\e_2 \gets \e[k\! +\! 1..2k]$ 
%		\item $\s \gets \Clamp(\e_1)$ %; $\curvepoint{A} \gets \s\cdot \generator$
%		\item $\littler \gets \HH(\e_2\|\msg) \bmod \Prime$ 
%		; $\curvepoint{R} \gets \littler \cdot \generator$
%		\item $c \gets \HH(\curvepoint{R}\|\curvepoint{A}\|\msg) \bmod \Prime$
%		; $\z \gets (\s c + \littler) \bmod \Prime$
%		\item Return $(\curvepoint{R},\z)$
%	\end{algorithm-subsequent}\ExptSepSpace
%	\begin{algorithm-subsequent}{$\DS_{\EdDSA}.\Vf[\HH](\curvepoint{A}, \msg, \sigma)$}
%		\item $(\curvepoint{R}, \z) \gets \sigma$
%		; $c \gets \HH(\curvepoint{R}\|\curvepoint{A}\|\msg) \bmod \Prime$
%		\item Return ($\z\cdot \generator = \curvepoint{R}+ c\cdot\curvepoint{A} $)
%	\end{algorithm-subsequent}  \vspace{2pt}
%	}
%\vspace{-8pt}
%\caption{On the left is the $\Schnorr$ signature scheme. On the right is $\EdDSA$. Here $\generator$ is a generator of an additively-written group $\group$ that has prime order $\Prime$. $\HH$ is a hash function. $\Clamp$ is the $\EdDSA$ clamping function.}
%\label{fig-eddsa-intro}
%\hrulefill
%\vspace{-10pt}
%\end{figure}

 \heading{The $\EdDSA$ scheme.} The Edwards curve Digital Signature Algorithm ($\EdDSA$) is a Schnorr-based signature scheme introduced by Bernstein, Duif, Lange, Schwabe and Yang~\cite{bernstein2012high}. $\Edtwo$, which uses the Curve25519 Edwards curve and $\SHAfive$ as the hash function, is its most popular instance. The scheme is standardized by NIST~\cite{NIST:EdDSA} and the IETF~\cite{RFC:EDDSA}. It is used in TLS 1.3, OpenSSH, OpenSSL, Tor, GnuPGP, Signal and WhatsApp. It is also the preferred signature scheme of the Corda, Tezos, Stellar and Libra blockchain systems. Overall, IANIX~\cite{Ed-uses} reports over 200 uses of $\Edtwo$. Proving security of this scheme is accordingly of high importance.
 
  Figure~\ref{fig-eddsa} shows $\EdDSA$ on the right, and, on the left, the classic $\Schnorr$ scheme~\cite{JC:Schnorr91} on which $\EdDSA$ is based. % (Note there are several $\Schnorr$ variants, that differ in details. See Section~\ref{sec-schemes} for a discussion of how the one used here relates to others.)
  The schemes are over a cyclic, additively-written group $\group$ of prime order $\Prime$ with generator $\generatorEDSA$. The public verification key is $\curvepoint{A}$. The $\Schnorr$ hash function has range $\Z_{\Prime} = \{0,\ldots,\Prime-1\}$, while, for $\EdDSA$, function $\HH_1$ has range $\bits^{2k}$ where $k$, the bit-length of $\Prime$, is $256$ for $\Edtwo$. Functions $\HH_2,\HH_3$ have range $\Z_{\Prime}$.

$\EdDSA$ differs from $\Schnorr$ in significant ways. While the $\Schnorr$ secret key $\s$ is in $\Z_{\Prime}$, the $\EdDSA$ secret key $\sk$ is a $k$-bit string. This is hashed and the $2k$-bit result is split into $k$-bit halves $e_1\|e_2$. A Schnorr secret-key $\s$ is derived by applying to $e_1$ a clamping function $\Clamp$ that zeroes out the three least significant bits of $e_1$. (Note: This means $\s$ is \textit{not} uniformly distributed over $\Z_{\Prime}$.) Clamping increases resistance to side-channel attacks~\cite{bernstein2012high}. Signing is made deterministic by a standard de-randomization technique~\cite{C:Goldreich86a,SAC:MNPV98,AC:BelPoeSte16,EC:BelTac16}, namely obtaining the Schnorr randomness $\littler$ by hashing the message $\msg$ with a secret-key dependent string $e_2$. We note that all of $\HH_1,\HH_2,\HH_3$ are instantiated via the same hash function, namely $\SHAfive$.

 % function $\HH$ is used in three ways for three purposes: to derive keys (lines~1,12), to de-randomize signing (line~14) and, as in $\Schnorr$, for the Fiat-Shamir hashing (line~15)~\cite{C:FiaSha86}. 


% We write the scheme as using three hash functions $\HH_1,\HH_2,\HH_3$, but ---this is important for our proposed work--- in $\Edtwo$ they are all $\SHAfive$. 

\heading{Prior work and our questions.} Recall that the security goal for a signature scheme is UF (UnForgeability under Chosen-Message Attack)~\cite{GolMicRiv88}. $\Schnorr$ is well studied, and proven UF under $\DLP$ (Discrete Log in $\group$) when $\HH$ is a random oracle~\cite{JC:PoiSte00,EC:AABN02}. The provable security of $\EdDSA$, however, received surprisingly little attention until the work of Brendel, Cremers, Jackson and Zhao (BCJZ)~\cite{SP:BCJZ21}. They take the path also used for $\Schnorr$ and other identification-based signature schemes~\cite{JC:PoiSte00,EC:AABN02}, seeing $\EdDSA$ as the result of the Fiat-Shamir transform on an underlying identification scheme $\EdID$ that they define, proving security of the latter under $\DLP$, and concluding UF of $\EdDSA$ under $\DLP$ when $\HH$ is a random oracle. This is an important step forward, but the BCJZ proof~\cite{SP:BCJZ21} remains in the $(\HH,\HH)$ model. We ask and address the following two questions.


\medskip
\textbf{1. Can we prove security in the $(\MD[\compF],\compF)$ model?} The NIST standard~\cite{NIST:EdDSA} mandates that $\Edtwo$ uses $\SHAfive$, which is an MD-hash function. Accordingly, as explained above, the BCJZ proof~\cite{SP:BCJZ21}, being in the $(\HH,\HH)$ model, does not guarantee security; to do the latter, we need a proof in the $(\MD[\compF],\compF)$ model.

The gap is more than cosmetic. As we saw above with the example of the prefix MAC, a scheme could be secure in the $(\HH,\HH)$ model, yet totally insecure in the more realistic $(\MD[\compF],\compF)$ model, and thus also in practice. And $\EdDSA$ skirts close to the edge: line~14 is using the prefix-MAC that the extension attack breaks, and overlaps in inputs across the three uses of $\HH$ could lead to failures. Intuitively what prevents attacks is that the MAC outputs are taken modulo $\Prime$, and inputs to $\HH$ in two of the three uses involve secrets. Thus, we'd expect that the scheme is indeed secure in the $(\MD[\compF],\compF)$ model. 

Proving this, however, is another matter. We already know that $\construct{MD}$ is not indiff. It is public indiff~\cite{EC:DodRisShr09,yoneyama2009leaky}, but this will not suffice for $\EdDSA$ because $\HH_1,\HH_2$ are being called on secrets. We ask, first, can $\EdDSA$ be proved secure in the $(\MD[\compF],\compF)$ model, and second, can this be done in some modular way, rather than from scratch?

\medskip
\textbf{2. Can we improve reduction tightness?} The reduction of BCJZ~\cite{SP:BCJZ21} is so loose that, in the 256-bit curve over which $\Edtwo$ is implemented, it guarantees little security. Let's elaborate. Given an adversary $\advAUF$ violating the UF-security of $\EdDSA$ with probability $\epsUF$, the reduction builds an adversary $\advADL$ breaking $\DLP$ with probability \smash{$\epsDL = \epsUF^2/q_h$} where $q_h$ is the number of $\HH$-queries of $\advAUF$ and the two adversaries have about the same running time $t$. (The square arises from the use of rewinding, analyzed via the Reset Lemma of~\cite{C:BelPal02}.) In an order $\Prime$ elliptic curve group, \smash{$\epsDL \approx t^2/p$} so we get \smash{$\epsUF = t\cdot \sqrt{q_h/p}$}. $\Edtwo$ has \smash{$p\approx 2^{256}$}. Say $t=q_h=2^{70}$, which (as shown by BitCoin mining capability) is not far from attacker reach. Then $\epsDL = 2^{-116}$ is small but \smash{$\epsUF = 2^{70}\cdot 2^{-(256-70)/2} = 2^{-23}$} is in comparison quite high. 

Now, one might say that one would not expect better because the same reduction loss is present for $\Schnorr$. The classical reductions for $\Schnorr$ \cite{JC:PoiSte00,EC:AABN02} did indeed display the above loss, but that has changed: recent advances for $\Schnorr$ include a tighter reduction from $\DLP$~\cite{C:RotSeg21}, an almost-tight reduction from the MBDL problem~\cite{INDOCRYPT:BelDai20} and a tight reduction from $\DLP$ in the Algebraic Group Model~\cite{EC:FucPloSeu20}. We'd like to put $\EdDSA$ on par with the state of the art for $\Schnorr$. We ask, first, is this possible, and second, is there a modular way to do it that leverages, rather than repeats, the (many, complex) just-cited proofs for $\Schnorr$? 

% \authnote{Reviwer}{Goal 1 is very clear.  Goal 2 not as much.  Setting $q_h = 2^{80}$ seems extreme.  }{red}

%\medskip
%\textbf{3. Weak multi-user security.} UF-security, the target in all the above, pertains to the classical setting where there is just one user (key) under attack. There is now broad acceptance that the more realistic setting is the multi-user one~\cite{EC:BelBolMic00,galbraith2002public,menezes2004security}, where there are $u\geq 1$ users, with independent keys. (This in particular is the setting in TLS.) Denoting security here by MUF, if $\epsMUF,\epsUF$ are the MUF and UF advantages (success probabilities), respectively, a standard hybrid argument shows that $\epsMUF \leq u\cdot\epsUF$~\cite{galbraith2002public}. But with just $u = 300$ users (TLS has way more), and $\epsUF=2^{-8}$ as above, this fails to even guarantee $\epsMUF<1$. 

%Now, for $\Schnorr$, one can do better than the hybrid argument; it is known that $\epsMUF\approx \epsUF$~\cite{EPRINT:Bernstein15,C:KilMasPan16}. Can we show the same for $\EdDSA$?


\heading{Contributions for $\EdDSA$.} We simultaneously simplify and strengthen the security proofs for $\EdDSA$ as follows.  
\smallskip

\textbf{1. Reduction from $\Schnorr$.} Rather than, as in prior work, give a reduction from $\DLP$ or some other algebraic problem, we give a simple, direct reduction from $\Schnorr$ itself. That is, we show that if the $\Schnorr$ signature scheme is UF-secure, then so is $\EdDSA$. Furthermore, the reduction is \textit{tight} up to a constant factor. This allows us to leverage prior work~\cite{C:RotSeg21,INDOCRYPT:BelDai20,EC:FucPloSeu20} to obtain tight proofs for $\EdDSA$ under various algebraic assumptions and justify security for group sizes in actual use. But there are two further dividends. First, $\Schnorr$~\cite{JC:Schnorr91} is over 30 years old and has withstood the tests of time and cryptanalysis, so our proof that $\EdDSA$ is just as secure as $\Schnorr$ allows the former to inherit, and benefit from, this confidence. Second, our result formalizes and proves what was the intuition and belief in the first place~\cite{bernstein2012high}, namely that, despite the algorithmic differences, $\EdDSA$ is a sound hardening of $\Schnorr$.
\smallskip

\textbf{2. Accurate modeling of the hash function.} As noted above, BCJZ~\cite{SP:BCJZ21} assume the hash function $\HH$ is a random oracle, but
% , as Coron, Dodis, Malinaud and Puniya (CDMP)~\cite{C:CDMP05} explained, 
this, due to the extension attack, is not an accurate model for the MD-hash function $\SHAfive$ used by $\EdDSA$. We fill this gap by instead proving security in the $(\MD[\compF],\compF)$ model, where $\HH=\MD[\hh]$ is derived via the MD-transform~\cite{C:Merkle89a,C:Damgaard89b} and the compression function $\hh$ is a random oracle.
% We explain why this is \textit{not} obtained directly by combining the BCJZ result with indifferentiability or public indifferentiability. 
  

\begin{sloppypar}
\heading{Approach and broader contributions.} The above-mentioned results on $\EdDSA$ are obtained as a consequence of more general ones.
\end{sloppypar}
\smallskip
\textbf{3. The $\DRTransform$ transform and its soundness.}
We extend the hardening technique used in $\EdDSA$ to define a general transform that we call Derive-then-Derandomize ($\DRTransform$). It takes an \textit{arbitrary} signature scheme $\DS$, and with the aid of a PRG $\HH_1$ and a PRF $\HH_2$, constructs a hardened signature scheme $\fDS$. We provide (Theorem~\ref{th-dd}) a strong and general validation of $\DRTransform$, showing that $\fDS$ is UF-secure assuming $\DS$ is UF-secure. Moreover \textit{the reduction is tight} and the proof is simple. This shows that the $\EdDSA$ hardening method is generically sound.

\smallskip
\textbf{4. Indifferentiability of Shrink-MD.} It is well-known that $\MD$ is not indifferentiable~\cite{TCC:MauRenHol04} from a random oracle, but that
the $\ChopMD$~\cite{C:CDMP05}, which truncates the output of an an $\MD$ hash by some number of bits, is indifferentiable.
Unfortunately, we identified gaps in two prominent proofs of indifferentiability of $\ChopMD$~\cite{C:CDMP05,hfrobook}.
$\EdDSA$ uses a similar construction that reduces the $\MD$ hash output modulo a prime $\Prime$ sufficiently smaller than the size of the range of $\MD$, due to which we refer to this construction as $\ModMD$.
The $\ModMD$ construction has not been proven indifferentiable.
We simultaneously give new proofs of indifferentiability for $\ChopMD$ and $\ModMD$ as part of a more general class of constructions that we call $\ShrinkMD$ functors.
These are constructions of the form $\Out(\MD)$ where $\Out$ is some output-processing function, and we prove indifferentiability under certain ``shrinking'' conditions on $\Out$.
  

\smallskip
\textbf{5. Application to $\EdDSA$.} $\EdDSA$ is obtained as the result $\fDS$ of the $\DRTransform$ transform applied to the $\DS=\Schnorr$ signature scheme, and with the PRG and PRF defined via $\MD$, specifically $\HH_1(\sk) = \MD[\hh](\sk)$ and $\HH_2(e_2,M) = \MD[\hh](e_2\|M)\bmod\Prime$ where $\Prime$ is the prime order of the underlying group. Additionally, the hash function used in $\Schnorr$ is also $\HH_3(X) = \MD[\hh](X)\bmod\Prime$. Due to Theorem~\ref{th-dd} validating $\DRTransform$, we are left to show the PRG security of $\HH_1$, the PRF security of $\HH_2$ and the UF-security of $\Schnorr$, all with $\hh$ modeled as a random oracle. We do the first directly. We obtain the second as a consequence of the indifferentiability of $\ModMD$. (In principle it follows from the PRF security of AMAC~\cite{EC:BelBerTes16}, but we found it difficult to extract precise bounds via this route.) For the third, we again exploit indifferentiability of $\ModMD$, together with a technique from BCJZ~\cite{SP:BCJZ21} to handle clamping, to reduce to the UF security of regular $\Schnorr$, where the hash function is modeled as a random oracle. Putting all this carefully together yields our above-mentioned results for $\EdDSA$. We note that one delicate and important point is that the idealized compression function $\hh$ is \textit{the same} across $\HH_1,\HH_2$ and $\HH_3$, meaning these are not independent. This is handled through the building blocks in Theorem~\ref{th-dd} being functors~\cite{EC:BelDavGun20} rather than functions.




%\heading{Our answers, in brief.} We give affirmative answers to both questions above. The first element of our approach is that our reduction for $\EdDSA$, rather than being from $\DLP$ or some other algebraic problem, is directly from $\Schnorr$ itself, and is \textit{tight} up to a constant factor. This in one step answers the second question (tightness) discussed above because we immediately inherit, for $\EdDSA$, the guarantees of the known tight(er) proofs of $\Schnorr$~\cite{C:RotSeg21,INDOCRYPT:BelDai20,EC:FucPloSeu20}. Our proof is, moreover, in the $(\MD[\compF],\compF)$-model. To obtain it in a modular way, we introduce filtered indifferentiability (f-indiff), show that f-indiff of $\MD$ suffices to prove security of $\EdDSA$, and separately establish f-indiff of $\MD$ (despite its lack of indiff) using a combination of new techniques and results from the indifferentiability literature~\cite{EC:DodRisShr09,C:CDMP05}. This answers the first question above. We now look at all this in more detail.
%
%\heading{Reduction from $\Schnorr$.} Let's write $\probP\reducesTo\DS$ to mean that we prove UF security of signature scheme $\DS$ with a reduction from (i.e., assuming hardness of) problem $\probP$ in the ROM. (The notation says nothing about tightness, which will be discussed separately.) Thus, BCJZ~\cite{SP:BCJZ21} show that $\DLP\reducesTo\EdDSA$. We show instead that $\Schnorr\reducesTo\EdDSA$. That is, if $\Schnorr$ is secure, so is $\EdDSA$. Furthermore, while the BCJZ reduction is loose, ours is \textit{tight} up to a small constant factor. 
%
%The immediate dividend is that any (known, or even future) proof $\probP\reducesTo\Schnorr$ automatically yields, via our result, a proof $\probP\reducesTo\EdDSA$, with only a constant factor loss in tightness compared to the original proof. In particular we get a tighter $\DLP\reducesTo\EdDSA$ proof via~\cite{C:RotSeg21}, an almost tight $\MBDLP\reducesTo\EdDSA$ proof via~\cite{INDOCRYPT:BelDai20} and an up-to-constant tight $\DLP\reducesTo\EdDSA$ Algebraic Group Model proof via~\cite{EC:FucPloSeu20}. This answers the second question (tightness) above.
%
%%\authnote{Reviwer}{Not the biggest fan of this $X \reducesTo Y$ notation, which conceals both the model (e.g., ROM) and the tightness of the reduction, in a paper that focuses on both the model and the tightness of the reduction.   I'd really like this groundbreaking paper to be more precise/specific documenting the implications in that "Reduction from Schnorr" section.  }{red}
%
%But there are two further dividends. First, $\Schnorr$~\cite{JC:Schnorr91} is over 30 years old and has withstood the tests of time and cryptanalysis. Our proof that $\EdDSA$ is just as secure as $\Schnorr$ allows the former to inherit, and benefit from, this confidence. Second, our result formalizes and proves what was the intuition and belief for $\EdDSA$ in the first place, namely that, despite the algorithmic differences, it is ``the same'' as $\Schnorr$ in security.
%
%Our proof that $\Schnorr\reducesTo\EdDSA$ would be novel and interesting already even in the basic $(\HH,\HH)$ model. However, we actually prove this in the $(\MD[\compF],\compF)$ model, so that we also answer the first question above. We now turn to this.
%
%


%\heading{Outline of proof.} Our main result (Theorem~\ref{th-eddsa-md}) is a tight (up to a constant factor) $\Schnorr\reducesTo\EdDSA$ reduction in the $(\MD[\compF],\compF)$ model. (The constant is 16 when $k=256$.)  That is, we show UF-security of $\EdDSA$, assuming only UF-security of $\Schnorr$, even when the hash function used is an MD-style one like $\SHAfive$. The statement of this result does not involve f-indiff or filters; these arise only in the proof. The latter has a few steps that we now outline. A fuller explanation is in Section~\ref{sec-schemes} and a picture is in Figure~\ref{pic-indiff}.

% \authnote{Reviewer}{The repeated claim of being tight "up to a constant factor" for Theorem 2 has me curious what that constant is.  Why not be explicit?}{red}

%Let $\DS$ denote the target $\EdDSA$ scheme, whose security we consider in the UF game. The first step is to cast $\DS$ as an alternative scheme $\fDS$ (shown on the right in Figure~\ref{fig-eddsa}) whose security we consider in a filtered unforgeability (fUF) game that we define via Figure~\ref{fig:fUF}. The idea is that the role of the secret signing key is now played by the filter seed. The filter here is our $\FltEDDSA$ one discussed above, and Lemma~\ref{lm-fUF-eq-UF} says that the schemes have equivalent security. We now need to show fUF security of $\fDS$. 
%
%The next step, Theorem~\ref{th-use-findiff}, is an f-indiff composition theorem, in the vein of the indiff composition theorem of~\cite{TCC:MauRenHol04}. This reduces the task to two sub-tasks. The first is to show fUF security of $\fDS$ when the oracle called by the $\FltEDDSA$ filter is a random oracle rather than $\MD[\compF]$. That is, the scheme should be secure in an ``ideal,'' even if still filtered, setting. The second sub-task is to show that $\MD$ is f-indiff relative to the $\FltEDDSA$ filter.
%
%The first sub-task is handled by Theorem~\ref{th-ideal-eddsa}, and this is where $\Schnorr$ enters, the reduction being from the latter. As an abstraction boundary, we use a version of $\Schnorr$ in which the signing key is drawn, not uniformly, but via a key-generation algorithm that performs the $\EdDSA$ clamping. A technique of BCJZ~\cite{SP:BCJZ21} separately allows $\Schnorr$ itself to reduce to clamped $\Schnorr$ with the above constant factor loss in advantage. 
% (The constant is 16 when $k=256$.)

%The final and most technical step is Theorem~\ref{th-md-indiff}, showing that $\MD$ is f-indiff relative to the $\FltEDDSA$ filter. This first exploits the presence of the seed to reduce the task to showing a weak form of public indiff for $\MD$. It concludes by exploiting the public indifferentiability of $\MD$ as shown in DRS~\cite{EC:DodRisShr09}. An alternative proof would first generalize and extend the indiff of truncated $\MD$ shown in~\cite{C:CDMP05} to $\MD$ taken modulo a prime whose bit-length is half that of the $\MD$ output, and then reduce to this.

%\authnote{Reviewer}{Figure~\ref{fig-ge1} appears on page 17, but is really needed to understand the discussion here.  Also this page makes lots of other references (e.g., Lemma~\ref{lm-fUF-eq-UF}) that assume the reader is already familiar with the later main body material.  Can you make this more self-contained with pointers forward?}{red}

 
\heading{Discussion and related work.} Both BCJZ~\cite{SP:BCJZ21} and CGN~\cite{10.1007/978-3-030-64357-7_4} note that there are a few versions of $\EdDSA$ out there, the differences being in their verification algorithms. What Figure~\ref{fig-eddsa} shows is the most basic version of the scheme, but we will be able to cover the variants too, in a modular way, by reducing from $\Schnorr$ with the same verification algorithm.

BBT~\cite{EC:BelBerTes16} define the function $\AMAC[\compF]$ to take a key $e_2$ and message $M$, and return $\MD[\compF](e_2\|M)\bmod\Prime$. This is the $\HH_2$ in $\EdDSA$. We could exploit their results to conclude PRF security of $\HH_2$, but it requires putting together many different pieces from their work, and it is easier and more direct to establish PRF security of $\HH_2$ by using our lemma on the indifferentiability of $\ModMD$.

In the Generic Group Model (GGM)~\cite{EC:Shoup97}, it is possible to prove UF-security of $\Schnorr$ under standard (rather than random oracle) model assumptions on the hash functions~\cite{neven2009hash,C:CLMQ21}. But use of the GGM means the result applies to a limited class of adversaries. Our results, following the classical proofs for identification-based signatures~\cite{JC:PoiSte00,C:OhtOka98,EC:AABN02,C:KilMasPan16}, instead use the standard model for the group, while modeling the hash function (in our case, the compression function) as a random oracle. 

In an earlier version of this paper, our proofs had relied on a variant of indifferentiability that we had introduced. At the suggestion of a Crypto 2022 reviewer, this has been dropped in favor of a direct proof based on PRG and PRF assumptions on $\HH_1,\HH_2$. We thank the (anonymous) reviewer for this suggestion.

Theorem~\ref{th-dd} is in the standard model if the PRG, PRF and starting signature scheme $\DS$ are standard-model, hence can be viewed as a standard-model justification of the hardening template underlying $\EdDSA$. However, when we want to justify $\EdDSA$ itself, we need to consider the specific, $\MD$-based instantiations of the PRG, PRF and $\Schnorr$ hash function, and for these we use the model where the compression function is ideal.

Several works study de-randomization of signing by deriving the coins via a PRF applied to the message, considering different ways to key the PRF~\cite{C:Goldreich86a,SAC:MNPV98,AC:BelPoeSte16,EC:BelTac16}. We use their techniques in the proof of Theorem~\ref{th-dd}.

One might ask how to view the UF-security of $\Schnorr$ signatures as an assumption. What is relevant is not its form (it is interactive) but that (1) it can be seen as a hub from where one can bridge to other assumptions that imply it, such as DL (non-tightly)~\cite{JC:PoiSte00,EC:AABN02} or MBDL (tightly)~\cite{INDOCRYPT:BelDai20}, and (2) it is validated by decades of cryptanalysis.

Our results have been stated for UF but extend to SUF (Strong unforgeability), meaning our proofs also show SUF-security of $\EdDSA$ in the $(\MD[\compF],\compF)$ model assuming SUF security of $\Schnorr$, with a tight (up to the usual constant factor) reduction.

$\EdDSA$ could be used with other hash functions such as $\SHAKE{256}$. The extension attack does not apply to the latter, so the proof of BCJZ~\cite{SP:BCJZ21} applies, but gives a loose reduction from DL; our results still add something, namely a tight reduction from $\Schnorr$ and thus improved tightness in several ways as discussed above.

% Given the way our work uses code-based games, it could benefit from being cast in the state-separation~\cite{AC:BDFKK18} or constructive cryptography~\cite{ICS:MauRen11,TCC:MauRen16} frameworks. We leave these as directions for future work.










\section{Oracle Cloning in NIST PQC Candidates}
\label{sec-pqc}

\headingg{Notation.} A KEM scheme $\kemScheme$ specifies an encapsulation $\kemEnc$ that, on input a public encryption key $\pk$ returns a session key $K$, and a ciphertext $C^*$ encapsulating it, written $(C^*,K)\getsr \kemEnc(\pk)$. A PKE scheme $\pkeScheme$ specifies an encryption algorithm $\pkeEnc$ that, on input $\pk$, message $M\in\bits^{\pkeML}$ and randomness $R$, deterministically returns ciphertext $C\gets\pkeEnc(\pk,M;R)$. For neither primitive will we, in this section, be concerned with the key generation or decapsulation / decryption algorithm. We might write $\kemScheme[X_1,X_2,\ldots]$ to indicate that the scheme has oracle access to functions $X_1,X_2,\ldots$, and correspondingly then write $\kemEnc[X_1,X_2,\ldots]$, and similarly for $\pkeScheme$.

\subsection{Design process} 

The literature~\cite{TCC:HofHovKil17,IMA:Dent03,EC:SaiXagYam18,C:JZCWM18} provides many transforms that take a public-key encryption scheme $\pkeScheme$, assumed to meet some weaker-than-IND-CCA notion of security we denote $\secPropPKE$ (for example, OW-CPA, OW-PCA or IND-CPA), and, with the aid of some number of random oracles, turn $\pkeScheme$ into a KEM that is guaranteed (proven) to be IND-CCA \textit{assuming the ROs are independent.} We'll refer to such transforms as \textit{sound}. Many (most) KEMs submitted to the NIST Post-Quantum Cryptography standardization process were accordingly designed as follows:
\begin{bignewenum}
	\item First, they specify a $\secPropPKE$-secure public-key encryption scheme $\pkeScheme$. 
\item Second, they pick a sound transform $\pkeToKem$ and obtain KEM $\GkemScheme{4}[\aFunc{H}_1,\aFunc{H}_2,\aFunc{H}_3,\aFunc{H}_4] \allowbreak = \allowbreak \pkeToKem[\pkeScheme,\aFunc{H}_2,\aFunc{H}_3,\aFunc{H}_4]$. (The notation is from~\cite{TCC:HofHovKil17}. The transforms use up to three random oracles that we are denoting $\aFunc{H}_2,\aFunc{H}_3,\aFunc{H}_4$, reserving $\aFunc{H}_1$ for possible use by the PKE scheme.) We refer to $\GkemScheme{4}$ (the subscript refers to its using 4 oracles) as the \textit{base} KEM, and, as we will see, it differs across the transforms.
\item Finally ---the under-the-radar step that is our concern--- the ROs $\aFunc{H}_1,\ldots,\aFunc{H}_4$ are constructed from cryptographic hash functions to yield what we call the \textit{final} KEM $\GkemScheme{1}$. In more detail, the submissions make various choices of cryptographic hash functions $\aFunc{F}_1,\ldots,\aFunc{F}_m$ that we call the \textit{base functions}, and, for $i=1,2,3,4$, specify constructions $\construct{C}_i$ that, with oracle access to the base functions, define the $\aFunc{H}_i$, which we write as $\aFunc{H}_i \gets \construct{C}_i[\aFunc{F}_1,\ldots,\aFunc{F}_m]$. We call this process oracle cloning, and we call $H_i$ the \textit{final functions.} (Common values of $m$ are $1,2$.) The actual, submitted KEM $\GkemScheme{1}$ (the subscript because $m$ is usually 1) uses the final functions, so that its encapsulation algorithm can be written as: 
\begin{tabbing}
	1234\=123\=\kill
\>	\underline{$\GkemEnc{1}[\aFunc{F}_1,\ldots,\aFunc{F}_m](\pk)$} \\[2pt]
\> For $i=1,2,3,4$ do $\aFunc{H}_i \gets \construct{C}_i[\aFunc{F}_1,\ldots,\aFunc{F}_m]$ \\
\> $(C^*,K)\getsr \GkemEnc{4}[\aFunc{H}_1,\aFunc{H}_2,\aFunc{H}_3,\aFunc{H}_4](\pk)$ \\
\> Return $(C^*,K)$
\end{tabbing}
\end{bignewenum} 
The question now is whether the final $\GkemScheme{1}$ is secure. We will show that, for some submissions, it is not. This is true for the choices of base functions $\aFunc{F}_1,\ldots,\aFunc{F}_m$ made in the submission, but also if these are assumed to be ROs. It is true despite the soundness of the transform, meaning insecurity arises from poor oracle cloning, meaning choices of the constructions $\construct{C}_i$. We will then consider submissions for which we have not found an attack. In the latter analysis, we are willing to assume (as the submissions implicitly do) that $\aFunc{F}_1,\ldots,\aFunc{F}_m$ are ROs, and we then ask whether the final functions are ``close'' to independent ROs.

\begin{figure}[t]
	\oneCol{0.7}{
		\begin{algorithm-initial}{Algorithm $\GkemEnc{4}[\aFunc{H}_1,\aFunc{H}_2,\aFunc{H}_3,\aFunc{H}_4](\pk)$}
\item $M \getsr \bits^{\pkeML}$ ; 
 % $X\gets\XFunc[H_3,H_4](\pk,M)$ ; 
 $R\gets\emptystring$ 
			\item If ($\DFunc=\true$) then $R\concat K' \gets \aFunc{H}_2(X)$ \Comment{$|K'|=\KLenFunc$}
			\item $\pkeCiph \gets \pkeEnc[\aFunc{H}_1](\pk,M;R)$ % ; $Y\gets\YFunc[H_3](X,M)$ 
			\item $\kemCiph \gets \pkeCiph\concat Y$  
			% ; $Z \gets \ZFunc[H_3,H_4](\pk,R,M,C,K',Y)$ 
			\item $K \gets \aFunc{H}_4(Z)$
			; Return $(\kemCiph, K)$ \smallskip			
		\end{algorithm-initial} 

	}
	\begin{center}
		\begin{tabular}{|c||c|c|c|c|c|c|}\hline
			 & $\DFunc$ & $\KLenFunc$  & $X$ & $Y$ & $Z$ & Used in  \\ \hline\hline
			\multirow{2}{*}{$\QpkeToKem_1$} & \multirow{2}{*}{$\true$} & \multirow{2}{*}{$0$} & \multirow{2}{*}{$M$} & \multirow{2}{*}{$\emptystring$} & \multirow{2}{*}{$M$} & \pqcnameRoundOne{LIMA}, \\
			& & & & & & \pqcnameRoundOne{Odd Manhattan}\\ \hline
			$\QpkeToKem_2$ & $\true$ & $0$ & $\pk\|M$ & $\emptystring$ & $\pk\|M$ & \pqcnameRoundTwo{ThreeBears}\\ \hline
			\multirow{2}{*}{$\QpkeToKem_3$} & \multirow{2}{*}{$\true$} & \multirow{2}{*}{$0$} & \multirow{2}{*}{$M$} & \multirow{2}{*}{$\emptystring$} & \multirow{2}{*}{$M\|\pkeCiph$}  & \pqcnameRoundTwo{BIKE-1-CCA}\\
			& & & & & &\pqcnameRoundTwo{BIKE-3-CCA}, \pqcnameRoundTwo{LAC} \\ \hline
			$\QpkeToKem_4$ & $\true$ &$0$ & $M\|\pk$ & $\emptystring$ & $M\|\pkeCiph$ & \pqcnameRoundTwo{SIKE}\\ \hline
			$\QpkeToKem_5$ & $\true$ & $0$ & $M$ & $\aFunc{H}_3(X)$ & $M\|\pkeCiph$ & \pqcnameRoundTwo{HQC}, \pqcnameRoundTwo{RQC}, \pqcnameRoundTwo{ROLLO-II}, \pqcnameRoundOne{LOCKER} \\ \hline
			$\QpkeToKem_6$ &$\true$ & $>0$ & $M\|\aFunc{H}_3(\pk)$ & $\emptystring$ & $K'\|\pkeCiph$  & \pqcnameRoundTwo{SABER} \\ \hline
			$\QpkeToKem_7$ & $\true$ &$>0$ & $\aFunc{H}_3(\pk)\|\aFunc{H}_3(M)$ & $\emptystring$ & $K'\|\aFunc{H}_3(\pkeCiph)$ & \pqcnameRoundTwo{CRYSTALS-Kyber} \\ \hline
			\multirow{1}{*}{$\QpkeToKem_8$} & \multirow{1}{*}{$\true$} & \multirow{1}{*}{$0$} & \multirow{1}{*}{$M$} & \multirow{1}{*}{$\aFunc{H}_3(X)$} & \multirow{1}{*}{$M$} & \pqcnameRoundOne{DAGS}, \pqcnameRoundOne{NTRU-HRSS-KEM} \\ \hline
			\multirow{2}{*}{$\QpkeToKem_9$} & \multirow{2}{*}{$\true$} & \multirow{2}{*}{$0$} & \multirow{2}{*}{$M$} & \multirow{2}{*}{$\aFunc{H}_3(X)$} & \multirow{2}{*}{$M\|\pkeCiph\|Y$} & \pqcnameRoundOne{BIG QUAKE}, \pqcnameRoundOne{EMBLEM},\\
			& & & & & & \pqcnameRoundOne{Lizard}, \pqcnameRoundOne{Titanium} \\ \hline
			$\QpkeToKem_{10}$ &$\true$ & $>0$ &   $\aFunc{H}_4(M)\|\aFunc{H}_4(\pk)$  & $\aFunc{H}_3(X)$ & $K'\|\aFunc{H}_4(\pkeCiph\|Y)$ & \pqcnameRoundTwo{NewHope} \\ \hline
			\multirow{2}{*}{$\QpkeToKem_{11}$} & \multirow{2}{*}{$\true$} & \multirow{2}{*}{$>0$} & \multirow{2}{*}{$M\|\pk$} & \multirow{2}{*}{$\aFunc{H}_3(X)$} & \multirow{2}{*}{$K'\|\pkeCiph\|Y$} & \pqcnameRoundOne{FrodoKEM}, \pqcnameRoundOne{Round2}\\
			& & & & & & \pqcnameRoundTwo{Round5} \\ \hline
			$\QpkeToKem_{12}$ &$\true$ & $>0$ & $\pk \| M$ & $\aFunc{H}_3(X)$ & $K'\|\pkeCiph$ & \pqcnameRoundOne{KCL} \\ \hline
			$\QpkeToKem_{13}$ &$\true$ & $>0$ & $\aFunc{H}_3(\pk)\|M$ & $\emptystring$ & $\pkeCiph\|K'$  & \pqcnameRoundTwo{FrodoKEM} \\ \hline
			$\QpkeToKem_{14}$ &$\false$ & $0$ & $\bot$ & $\aFunc{H}_3(M)$ & $M\|\pkeCiph\|Y$ & \pqcnameRoundTwo{Classic McEliece}\\ \hline
			$\QpkeToKem_{15}$ &$\true$ & $0$ & $M$ & $\emptystring$ & $R\|M$ & \pqcnameRoundTwo{NTS-KEM} \\ \hline
			$\QpkeToKem_{16}$ & $\false$ & $0$ & $\bot$ & $\aFunc{H}_3(M\| \pk)$ & $M \| C \| Y$ & \pqcnameRoundTwo{Streamlined NTRU Prime} \\ \hline
			$\QpkeToKem_{17}$ & $\true$ & $0$ & $M$ & $\aFunc{H}_3(M\|\pk)$ & $M \| C \| Y$ & \pqcnameRoundTwo{NTRU LPRime} \\ \hline
		\end{tabular}
	\end{center}
	\caption{%
	\textbf{Top:} Encapsulation algorithm of the base KEM scheme produced by our parameterized transform. \textbf{Bottom:} Choices of parameters $X,Y,Z,\DFunc,\KLenFunc$ resulting in specific transforms used by the NIST PQC submissions. Second-round submissions are in \pqcnameRoundTwo{bold}, first-round submissions in \pqcnameRoundOne{gray}. Submissions using different transforms in the two rounds appear twice.}\label{fig:pqc-kems}
	\hrulefill
\end{figure}
%\hnote{ The syntax here, where we give KE four distinct 1-argument oracle, is inconsistent with the KEM syntax in prelims (where we have one two-argument oracle).}

\subsection{The base KEM} 

We need first to specify the base  $\GkemScheme{4}$ (the result of the sound transform, from step~(2) above). The NIST PQC submissions typically cite one of HHK~\cite{TCC:HofHovKil17}, Dent~\cite{IMA:Dent03}, SXY~\cite{EC:SaiXagYam18} or JZCWM~\cite{C:JZCWM18} for the sound transform they use, but our examinations show that the submissions have embellished, combined or modified the original transforms. The changes do \textit{not} (to best of our knowledge) violate soundness (meaning the used transforms still yield an IND-CCA $\GkemScheme{4}$ if $\aFunc{H}_2,\aFunc{H}_3,\aFunc{H}_4$ are independent ROs and $\pkeScheme$ is $\secPropPKE$-secure) but they make a succinct exposition challenging. We address this with a framework to unify the designs via a single, but parameterized, transform, capturing the submission transforms by different parameter choices. 

Figure~\ref{fig:pqc-kems} (top) shows the encapsulation algorithm $\GkemEnc{4}$ of the KEM that our parameterized transform associates to $\pkeScheme$ and $H_1,H_2,H_3,H_4$. The parameters are the variables $X,Y,Z$ (they will be functions of other quantities in the algorithms), a boolean $\DFunc$, and an integer $\KLenFunc$. When choices of these are made, one gets a fully-specified transform and corresponding base KEM $\GkemScheme{4}$. Each row in the table in the same Figure shows one such choice of parameters, resulting in 15 fully-specified transforms. The final column shows the submissions that use the transform. 

The encapsulation algorithm at the top of Figure~\ref{fig:pqc-kems} takes input a public key $\pk$ and has oracle access to functions $H_1,H_2,H_3,H_4$. At line~1, it picks a random seed $M$ of length the message length of the given PKE scheme. Boolean $\DFunc$ being $\true$ (as it is except in two cases) means $\pkeEnc$ is randomized. In that case, line~2 applies $H_2$ to $X$ (the latter, determined as per the table, depends on $M$ and possibly also on $\pk$) and parses the output to get coins $R$ for $\pkeEnc$ and possibly (if the parameter $\KLenFunc\neq 0$) an additional string $K'$. At line~3, a ciphertext $\pkeCiph$ is produced by encrypting the seed $M$ using $\pkeEnc$ with public key $\pk$ and coins $R$. In some schemes, a second portion of the ciphertext, $Y$, often called the ``confirmation", is derived from $X$ or $M$, using $\aFunc{H}_3$, as shown in the table, and line~4 then defines $\kemCiph$. Finally, $\aFunc{H}_4$ is used as a key derivation function to extract a symmetric key $K$ from the parameter $Z$, which varies widely among transforms. 

In total, 26 of the 39 NIST PQC submissions which target KEMs in either the first or second round use transforms which fall into our framework. The remaining schemes do not use more than one random oracle, construct KEMs without transforming PKE schemes, or target security definitions other than \INDCCA. 

%\heading{Oracle cloning in NIST PQC submissions.}
%In schemes like the above which use up to four random oracles, instantiation with concrete functions is a nontrivial problem. 
%Some designers chose to implement their multiple random oracles with multiple concrete hash functions; i.e. $H_1 = \SHAKE{256}$, $H_2 = \mathsf{TupleHash256}$, etc.
%
%Most, however, following NIST's guidance on choosing symmetric primitives~\cite{NIST-PQC-FAQ17}, preferred to rely on just one or two cryptographic hash functions to implement all the ROs.
%Implementing multiple ROs via a single hash function implicitly defines a method of implementing multiple ROs via a single RO. 
%This is the task we call random oracle cloning. 
%As we discuss in the following, the various submissions differ significantly in the care taken when cloning random oracles. 
%In the best cases, the specification already is very explicit and commendably describes how implementations shall domain-separate the different oracles.
%In the most critical cases, a failure to separate domains lead to outright key-recovery attacks.
%We group the submissions into four groups with increasing success at maintaining independence through the oracle cloning steps.
%The steps a scheme takes to clone random oracles may be found in either a scheme's specificiation or its implementation.
%If they are found in the implementation, they are independent of specific implementation choices or particular concrete hash functions.

\subsection{Submissions we break}

We present attacks on \pqcnameRoundOne{BIG QUAKE}~\cite{nistpqc:BIGQUAKE}, \pqcnameRoundOne{DAGS}~\cite{nistpqc:DAGS}, and \pqcnameRoundOne{Round2}~\cite{nistpqc:Round2}. These attacks succeed in full or partial recovery of the encapsulated KEM key from a ciphertext, and are extremely fast. We have implemented the attacks to verify them.

Although none of these schemes progressed to Round~2 of the competition without significant modification, to the best of our knowledge, none of the attacks we described were pointed out during the review process. Given the attacks' superficiality, this is surprising and suggests to us that more attention should be paid to oracle cloning methods and their vulnerabilities during review. 

% All three schemes implement the three functions $H_2,H_3,H_4$ using a single function $F_1$, with no domain separation. 

\heading{Randomness-based decryption.} The PKE schemes used by \pqcnameRoundOne{BIG QUAKE} and \pqcnameRoundOne{Round2} have the property that given a ciphertext $C \gets \pkeEnc(\pk,M;R)$ and also given the coins $R$, it is easy to recover $M$, even without knowledge of the secret key. We formalize this property, saying $\pkeScheme$ allows randomness-based decryption, if there is an (efficient) algorithm $\pkeDecR$ such that $\pkeDecR(\pk,\allowbreak \pkeEnc(\pk, \allowbreak M; R),\allowbreak R) = M$ for any public key $\pk$, coins $R$ and message $m$. This will be used in our attacks.






\heading{Attack on \pqcnameRoundOne{BIG QUAKE}.} The base KEM $\kemScheme_1[H_1,H_2,H_3,H_4]$ is given by the transform $\QpkeToKem_9$ in the table of Figure~\ref{fig:pqc-kems}. The final KEM $\kemScheme_2[F]$ uses a single function $F$ to instantiate the random oracles, which it does as follows. It sets $H_3=H_4=F$ and $H_2=W[F]\circ F$ for a certain function $W$ (the rejection sampling algorithm) whose details will not matter for us. The notation $W[F]$ meaning that $W$ has oracle access to $F$. The following attack (explanations after the pseudocode) recovers the encapsulated KEM key~$K$ from  ciphertext $C^* \getsr \GkemEnc{1}[F](\pk)$---

\begin{tabbing}
	123\=123\=\kill
	\underline{Adversary $\advA[F](\pk,C^*)$} \Comment{Input public key and ciphertext, oracle for $F$} \\[2pt]
	1. \> $C\|Y\gets C^*$ \Comment{Parse $C^*$ to get PKE ciphertext $C$ and $Y=H_3(M)$} \\
	2. \> $R\gets W[F](Y)$ \Comment{Apply function $W[F]$ to $Y$ to recover coins $R$} \\
	3. \> $M\gets \pkeDecR(\pk,C,R)$ \Comment{Use randomness-based decryption for $\pkeScheme$} \\
	4. \> $K \gets F(M)$ ; Return $K$
\end{tabbing}

\noindent As per $\QpkeToKem_9$ we have $Y = H_3(M) = F(M)$. The coins for $\pkeEnc$ are $R = H_2(M) = (W[F]\circ F)(M) = W[F](F(M)) = W[F](Y)$. Since $Y$ is in the ciphertext, the coins $R$ can be recovered as shown at line~2. The PKE scheme allows randomness-based decryption, so at line~3 we can recover the message $M$ underlying $C$ using algorithm $\pkeDecR$. But $K = H_4(M) = F(M)$, so $K$ can now be recovered as well. In conclusion, the specific cloning method chosen by \pqcnameRoundOne{BIG QUAKE} leads to complete recovery of the encapsulated key from the ciphertext.


%\begin{figure}
%	\twoCols{0.44}{0.46}{
%	\begin{oracle}{Adversary $\advA^F_{\pqcname{BIG QUAKE}}$}
%		\item $\params, \pk, C^* \gets \Initialize()$
%		\item $C,Y \gets C^*$
%		\item $R \gets W[F](Y)$
%		\item $M \gets \pkeDecR(C,R)$
%		\item $K \gets F(M)$
%		\item Return $K$ 
%	\end{oracle}
%}{	
%\begin{oracle}{Adversary $\advA^F_{\pqcname{Round2}}$}
%		\item $\params, \pk, C^* \gets \Initialize()$
%		\item $C,Y \gets C^*$
%		\item $R \gets F(Y)$
%		\item $M \gets \pkeDecR(C,R)$
%		\item $K \gets F(M)$
%		\item Return $K$ 
%	\end{oracle}
%
%}
%	\label{fig-atk-bq}
%	\caption{Attackers against \pqcname{BIG QUAKE} (left) and \pqcname{Round2} (right) \fg{missing?}.}.
%\end{figure}
%In Figure~\ref{fig-atk-bq}, we present an attacker $\advA$ which can win game $\ngameOWCPA{\pqcname{BIGQUAKE},F}$ with probability $1$. This attacker takes advantage of the fact that in transform $\QpkeToKem_9$, the variable $Y=H_3(M) = F(M)$ is included in the ciphertext, and the PKE's coins $R = H_2(M) = W[F](F(M)) = W[F](Y)$ can be computed from $Y$. Then, because \pqcname{BIG QUAKE} allows decryption using the coins of encryption, $\advA$ can correctly derive the random seed $M$ and the encapsulation key $K$.
%The specific cloning method of the random oracles $\aFunc{H}_2$ $\aFunc{H}_3$, and~$\aFunc{H}_4$ chosen by \pqcname{BIG QUAKE} hence leads to complete key recovery.% of the $\QpkeToKem_9$ transform proven secure in the quantum random oracle model.

\heading{Attack on \pqcnameRoundOne{Round2}.} The base KEM $\kemScheme_1[H_2,H_3,H_4]$ is given by the transform $\QpkeToKem_{11}$ in the table of Figure~\ref{fig:pqc-kems}. The final KEM $\kemScheme_2[F]$ uses a single base function $F$ to instantiate the final functions, which it does as follows. It sets $H_4=F$. The specification and reference implementation differ in how $H_2,H_3$ are defined: In the former, $\aFunc{H}_2(x) = F(F(x))\concat F(x)$ and $\aFunc{H}_3(x) = F(F(F(x)))$, while, in the latter, $\aFunc{H}_2(x) = F(F(F(x))) \concat F(x)$ and $\aFunc{H}_3(x) = F(F(X))$. These differences arise from differences in the way the output of a certain function $W[F]$ is parsed.


Our attack is on the reference-implementation version of the scheme. We need to also know that the scheme sets $\KLenFunc$ so that $R\|K'\gets H_2(X)$ with $H_2(X) = F(F(F(X)))\|F(X)$ results in $R = F(F(F(X)))$. But $Y=H_3(X) = F(F(X))$, so $R=F(Y)$ can be recovered from the ciphertext. Again exploiting the fact that the PKE scheme allows randomness-based decryption, we obtain the following attack that recovers the encapsulated KEM key~$K$ from  ciphertext $C^* \allowbreak \getsr \allowbreak \GkemEnc{1}[F](\pk)$--- 

\begin{tabbing}
	123\=123\=\kill
	\underline{Adversary $\advA[F](\pk,C^*)$} \Comment{Input public key and ciphertext, oracle for $F$} \\[2pt]
	1. \> $C\|Y\gets C^*$; $R \gets F(Y)$\\
	2. \> $M\gets \pkeDecR(\pk,C,R)$ ;
	 $K \gets F(M)$ ; Return $K$
\end{tabbing}

%\cite{nistpqc:Round2} sets $\aFunc{H}_4 = F$ for a random oracle $F$.
%It defines $\aFunc{H}_2$ and $\aFunc{H}_3$, which are queried on the same input, as substrings of the output of a fixed-length function $W[F]$ based on $F$ such that that for all bitstrings $x$, we have $W[F](x) = F(x) \concat F(F(x)) \concat F(F(F(x)))$.
%In the specification, $W[F](x)$ is parsed such that $\aFunc{H}_2(x) = F(F(x))\concat F(x)$ and $\aFunc{H}_3(x) = F(F(F(x)))$.
%However, in the reference implementation, $\aFunc{H}_2(x) = F(F(F(x))) \concat F(x)$ and $\aFunc{H}_3(x) = F(F(X))$.
%As a consequence, $R = F(F(F(X))) = F(Y)$ can be computed from $Y$.
%Similar to \pqcname{BIG QUAKE}, the PKE scheme underlying \pqcname{Round2} allows to efficently decrypt~$\pkeCiph$ with knowledge of~$R$, enabling a full recovery of the encapsulated key from the KEM ciphertext.

\noindent This attack exploits the difference between the way $H_2,H_3$ are defined across the specification and implementation, which may be a bug in the implementation with regard to the parsing of $W[F](x)$. However, the attack also exploits dependencies between $\aFunc{H}_2$ and $\aFunc{H}_3$, which ought not to exist when instantiating what are required to be distinct random oracles. 

\pqcnameRoundOne{Round2} was incorporated into the second-round submission \pqcnameRoundTwo{Round5}, which specifies a different base function and cloning functor (the latter of which uses the secure method we call ``output splitting") to instantiate oracles $H_2$ and $H_3$. This attack therefore does not apply to \pqcnameRoundTwo{Round5}.
%When \pqcname{Round2} was merged into the second-round submission~\pqcname{Round5}~\cite{nistpqc:Round5},
%a variable-length output hash function %SHAKE
%replaced both~$W$ and~$F$, and the resulting scheme falls into Group~3. 
%

\heading{Attack on DAGS.} If $x$ is a byte string we let $x[i]$ be its $i$-th byte, and if $x$ is a bit string we let $x_i$ be its $i$-th bit. We say that a function $V$ is an extendable output function if it takes input a string $x$ and an integer $\ell$ to return an $\ell$-byte output, and $\ell_1 \leq \ell_2$ implies that $V(x,\ell_1)$ is a prefix of $V(x,\ell_2)$. If $v = v_1v_2v_3v_4v_5v_6v_7v_8$ is a byte then let $Z(v) = 00v_3v_4v_5v_6v_7v_8$ be obtained by zeroing out the first two bits. If $y$ is a string of $\ell$ bytes then let $Z'(y) = Z(y[1])\| \cdots \| Z(y[\ell])$. Now let $V'(x,\ell) = Z'(V(x,\ell))$. 

The base KEM $\kemScheme_1[H_1,H_2,H_3,H_4]$ is given by the transform $\QpkeToKem_{8}$ in the table of Figure~\ref{fig:pqc-kems}. The final KEM $\kemScheme_2[V]$ uses an extendable output function $V$ to instantiate the random oracles, which it does as follows. It sets $\aFunc{H}_2(x) = V'(x,512)$ and $\aFunc{H}_3(x) = V'(x,32)$. It sets $\aFunc{H}_4(x) = V(x,64)$. 

As per $\QpkeToKem_8$ we have $K = H_4(M)$ and $Y = H_3(M)$. Let $L$ be the first 32 bytes of the 64-byte $K$. Then $Y = Z'(L)$. So $Y$ reveals $32\cdot 6 = 192$ bits of $K$. Since $Y$ is in the ciphertext, this results in a partial encapsulated-key recovery attack. The attack reduces the effective length of $K$ from $64\cdot 8 = 512$ bits to $512-192 = 320$ bits, meaning $37.5\%$ of the encapsulated key is recovered. Also $R = H_2(M)$, so $Y$, as part of the ciphertext, reveals 32 bytes of $R$, which does not seem desirable, even though it is not clear how to exploit it for an attack.


\subsection{Submissions with unclear security}

For the scheme \pqcnameRoundTwo{NewHope}~\cite{nistpqc:NewHope}, we can give neither an attack nor a proof of security. However, we can show that the final functions $H_2, H_3, H_4$ produced by the cloning functor $\construct{F}_{\pqcnameRoundTwo{NewHope}}$ with oracle access to a single extendable-output function $V$ are differentiable from independent random oracles. The cloning functor $\construct{F}_{\pqcnameRoundTwo{NewHope}}$ sets $H_1(x)=V(x,128)$ and $H_4 = V(x,32)$. It computes $H_2$ and $H_3$ from $V$ using the output splitting cloning functor. Concretely, $\kemScheme_2$ parses $V(x,96)$ as $H_2(x)\concat H_3(x)$, where $H_2$ has output length 64 bytes and $H_3$ has output length 32 bytes. Because $V$ is an extendable-output function, $H_4(x)$ will be a prefix of $H_2(x)$ for any string $x$.

We do not know how to exploit this correlation to attack the \INDCCA security of the final KEM scheme $\kemScheme_2[V]$, and we conjecture that, due to the structure of~$\QpkeToKem_{10}$, no efficient attack exists. 
We can, however, attack the rd-indiff security of functor $\construct{F}_{\pqcnameRoundTwo{NewHope}}$, showing that that the security proof for the base KEM $\kemScheme_1[H_2,H_3,H_4]$ does not naturally transfer to $\kemScheme_2[V]$.
Therefore, in order to generically extend the provable security results for $\kemScheme_1$ to $\kemScheme_2$, it seems advisable to instead apply appropriate oracle cloning methods.

\subsection{Submissions with provable security but ambiguous specification}

In their reference implementations, these submissions use cloning functors which we can and do validate via our framework, providing provable security in the random oracle model for the final KEM schemes. However, the submission documents do not clearly specify a secure cloning functor, meaning that variant implementations or adaptations may unknowingly introduce weaknesses. 
The schemes
\pqcnameRoundTwo{BIKE}~\cite{nistpqc:BIKE},
\pqcnameRoundOne{KCL}~\cite{nistpqc:KCL},
\pqcnameRoundOne{LAC}~\cite{nistpqc:LAC},
\pqcnameRoundOne{Lizard}~\cite{nistpqc:Lizard},
\pqcnameRoundOne{LOCKER}~\cite{nistpqc:LOCKER},
\pqcnameRoundOne{Odd Manhattan}~\cite{nistpqc:OddM},
\pqcnameRoundTwo{ROLLO-II}~\cite{nistpqc:ROLLO},
\pqcnameRoundTwo{Round5}~\cite{nistpqc:Round5},
\pqcnameRoundTwo{SABER}~\cite{nistpqc:SABER} and
\pqcnameRoundOne{Titanium}~\cite{nistpqc:Titanium}
fall into this group.

%When we define the base KEM $\kemScheme_1[H_1\cab H_2\cab H_3\cab H_4]$ via a transform taken from the submission document, and we define the final KEM $\kemScheme_2$ by applying the functors in the reference implementation to $\kemScheme_1$, we can reduce the \INDCCA security of $\kemScheme_2$ from the \INDCCA security of $\kemScheme_1$ via our formalism.
%This validates the security of the reference implementations. 

%We however make a distinction between these schemes and schemes with clear provable security. Here, the cloning functors are taken from the reference implementations and are often dependent on specific choices of parameters. 
%These choices may diverge from the submission document, they may vary between implementations, or they may not appear in the submission document at all.
%All of these ambiguities make it difficult for developers to know which set of parameters provides security and to avoid unknowingly introducing weaknesses by modifying security-critical parameters. 

\heading{Length differentiation.} Many of these schemes use the ``identity" functor in their reference implementations, meaning that they set the final functions $H_1 = H_2 = H_3 = H_4 = F$ for a single base function $F$. 
If the scheme $\kemScheme_1[H_1,H_2,H_3,H_4]$ never queries two different oracles on inputs of a single length, the  domains of $H_1,\ldots,H_4$ are implicitly separated.
Reference implementations typically enforce this separation by fixing the input length of every call to $F$. 
Our formalism calls this query restriction "length differentiation" and proves its security as an oracle cloning method. We also generalize it to all methods which prevent the scheme from querying any two distinct random oracles on a single input. 
%Concretely, variant implementations with altered parameters or modifications to the schemes' use in different transforms may modify the parameters of the schemes in ways leading the oracles' domains to no longer be disjoint, in which case weaknesses may be introduced unknowingly.

In the following, we discuss two schemes from the group, \pqcnameRoundTwo{ROLLO-II} and \pqcnameRoundOne{Lizard}, where ambiguity about cloning methods between the specification and reference implementation jeopardizes the security of applications using these schemes. It will be important that, like \pqcnameRoundOne{BIG QUAKE} and \pqcnameRoundOne{RoundTwo}, the PKE schemes defined by \pqcnameRoundTwo{ROLLO-II} and \pqcnameRoundOne{Lizard} allow randomness-based decryption. 


The scheme \pqcnameRoundTwo{ROLLO-II}
\cite{nistpqc:ROLLO} defines its base KEM  $\kemScheme_1[H_1,H_2,H_3,H_4]$ using the $\QpkeToKem_5$ transform from Figure~\ref{fig:pqc-kems}. The submission document states that $\aFunc{H}_1$, $\aFunc{H}_2$, $\aFunc{H}_3$, and $\aFunc{H}_4$ are ``typically" instantiated with a single fixed-length hash function~$F$, but does not 
describe the cloning functors used to do so. 
If the identity functor is used, so that $H_1 = H_2 = H_3 = H_4 = F$, (or more generally, any functor that sets $H_2=H_3$), an attack is possible.
In the transform $\QpkeToKem_5$, both $\aFunc{H}_2$ and $\aFunc{H}_3$ are queried on the same input $M$. Then $Y = H_3(M) = F(M) = H_2(M) = R$ leaks the PKE's random coins, so the following attack will allow total key recovery via the randomness-based decryption. 
\begin{tabbing}
	123\=123\=\kill
	\underline{Adversary $\advA[F](\pk,C^*)$} \Comment{Input public key and ciphertext, oracle for $F$} \\[2pt]
	1. \> $C\|Y\gets C^*$ ; $M\gets \pkeDecR(\pk,C,Y)$ \Comment ($Y=R$ is the coins) \\
	2. \> $K \gets F(M\concat C \concat Y)$ ; Return $K$
\end{tabbing}
%The public key encryption scheme defines two deterministic functions $W$ and $W'$, and defines the ciphertext to be the value $\pkeCiph = (M \xor \aFunc{H}_1(W(R)))\concat W'(R,\pk)$.
%Since under this instantiation $R = Y$ is published in the KEM ciphertext, the adversary can compute $\aFunc{H}_1(W(R))$ and use it to extract $M$ from the first part of the PKE ciphertext, allowing total key recovery.
In the reference implementation of \pqcnameRoundTwo{ROLLO-II}, however, $\aFunc{H}_2$ is instantiated using a second, independent function $V$ instead of $F$, which prevents the above attack. 
Although the random oracles $H_1,H_3$ and $H_4$ are instantiated using the identity functor, they are never queried on the same input thanks to length differentiation.
As a result, the reference implementation of \pqcnameRoundTwo{ROLLO-II} is provably secure, though alternate implementations could be both compliant with the submission document and completely insecure. 
The relevant portions of both the specification and the reference implementation were originally found in the corresponding first-round submission (\pqcnameRoundOne{LOCKER}).%, which was one of the contributors to \pqcnameRoundTwo{ROLLO-II} with \pqcnameRoundOne{LAKE} and \pqcnameRoundOne{Rank-Ouroboros}.

\medskip

\pqcnameRoundOne{Lizard}
\cite{nistpqc:Lizard}
follows transform~$\QpkeToKem_9$ to produce its base KEM $\kemScheme_1[H_2\cab H_3\cab H_4]$. Its submission document suggests instantiation with a single function $F$ as follows: it sets $H_3 = H_4 = F$, and it sets $H_2 = W \circ F$ for some postprocessing function $W$ whose details are irrelevant here.
Since, in $\QpkeToKem_9$, $Y = \aFunc{H}_3(M) = F(M)$ and $R = \aFunc{H}_2(M) = W\circ F (M) = W(Y)$, the randomness $R$ will again be leaked through $Y$ in the ciphertext, permitting a key-recovery attack using randomness-based decryption much like the others we have described. This attack is prevented in the reference implementation of \pqcnameRoundOne{Lizard}, which instantiates $H_3$ and $H_4$ using an independent function $G$. The domains of $H_3$ and $H_4$ are separated by length differentiation. This allows us to prove the security of the final KEM $\kemScheme_2[G,F]$, as defined by the reference implementation.

However, the length differentiation of $\aFunc{H}_3$ and $\aFunc{H}_4$ breaks down in the chosen-ciphertext-secure PKE variant specification of \pqcnameRoundOne{Lizard}, which transforms $\kemScheme_1$. The PKE scheme, given a plaintext $P$, chooses a random message $M$, computes $R=H_2(M)$ and $Y=H_3(M)$ according to $\QpkeToKem_9$, but it computes $K = H_4(M)$, then includes the value $B = K \xor P$ as part of the ciphertext $C^*$.
Both the identity functor and the functor used by the KEM reference implementation set $H_3 = H_4$, so the following attack will extract the plaintext from any ciphertext--
\begin{tabbing}
	123\=123\=\kill
	\underline{Adversary $\advA(\pk,C^*)$} \Comment{Input public key and ciphertext} \\[2pt]
	1. \> $C\|B\|Y\gets C^*$ \Comment{Parse $C^*$ to get $Y$ and $B = P \xor K$}\\ 
	2. \> $P\gets Y \xor B$ ; Return $P$ \Comment{$Y = H_3(M) = H_4(M) = K$ is the mask.}
\end{tabbing}

The reference implementation of the public-key encryption schemes prevents the attack by cloning $\aFunc{H}_3$ and $\aFunc{H}_4$ from $G$ via a third cloning functor, this one using the output splitting method. 
Yet, the inconsistency in the choice of cloning functors between the specification and both implementations underlines that ad-hoc cloning functors may easily ``get lost'' in modifications or adaptations of a scheme.

\subsection{Submissions with clear provable security}
Here we place schemes which explicitly discuss their methods for domain separation and follow good practice in their implementations:
\pqcnameRoundTwo{Classic McEliece}~\cite{nistpqc:ClassicMcEliece},
\pqcnameRoundTwo{CRYSTALS-Kyber}~\cite{nistpqc:CRYSTALSKyber},
\pqcnameRoundOne{EMBLEM}~\cite{nistpqc:EMBLEM},
\pqcnameRoundTwo{FrodoKEM}~\cite{nistpqc:FrodoKEM},
\pqcnameRoundTwo{HQC}~\cite{nistpqc:HQC},
\pqcnameRoundOne{LIMA}~\cite{nistpqc:LIMA},
\pqcnameRoundOne{NTRU-HRSS-\allowbreak{}KEM}~\cite{nistpqc:NTRU-HRSS-KEM},
\pqcnameRoundTwo{NTRU Prime}~\cite{nistpqc:NTRUPrime},
\pqcnameRoundTwo{NTS-KEM}~\cite{nistpqc:NTS-KEM},
\pqcnameRoundTwo{RQC}~\cite{nistpqc:RQC},
\pqcnameRoundTwo{SIKE}~\cite{nistpqc:SIKE} and
\pqcnameRoundTwo{ThreeBears}~\cite{nistpqc:ThreeBears}.
These schemes are careful to account for dependencies between random oracles that are considered to be independent in their security models.
When choosing to clone multiple random oracles from a single primitive, the schemes in this group use padding bytes, deploy hash functions designed to accommodate domain separation, or restrictions on the length of the inputs which are codified in the specification.
These explicit domain separation techniques can be cast in the formalism we develop in this work.

% \pqcheading{HQC and RQC}
\pqcnameRoundTwo{HQC} and \pqcnameRoundTwo{RQC}
are unique among the PQC KEM schemes in that their specifications warn that the identity functor admits key-recovery attacks. As protection, they recommend that $\aFunc{H}_2$ and $\aFunc{H}_3$ be instantiated with unrelated primitives. %They do not provide an alternative cloning functor, nor discuss how to instantiate $H_4$.

%\heading{The others.}
%For completeness, we here briefly mention those submissions for \INDCCA KEM schemes which use transforms similar to the ones we discuss, but stand outside our framework.
%(Another 6 first-round submissions do not target \INDCCA.)
%\pqcname{KINDI}'s~\cite{nistpqc:KINDI} unique decryption interface returns the ephemeral randomness as well as the message, encapsulating the key in the former rather than the latter.
%\pqcname{LEDAcrypt}~\cite{nistpqc:LEDAcrypt} uses a Niederreiter~\cite{PCIT:Nieder86}-inspired scheme which uses only one hash function.
%\pqcname{Lepton}~\cite{nistpqc:Lepton} uses a transformation similar in structure to~$\QpkeToKem_{11}$, but it transforms an IND-CPA KEM scheme.
%\pqcname{LOTUS}~\cite{nistpqc:LOTUS} uses the Fujisaki--Okamoto transform~\cite{C:FujOka99}, additionally employing a symmetric encryption scheme which we do not capture in our framework.
%\pqcname{NTRU}~\cite{nistpqc:NTRU}, which builds upon the first-round submission \pqcname{NTRU-HRSS-KEM}, relies on a deterministic public-key encryption scheme, and its model includes only one hash function.
%\pqcname{QC-MDPC}~\cite{nistpqc:QCMDPC} encrypts a deterministic function of its message and the encryption's randomness, which is outside of our framework for transforms, but otherwise its transformation equals $\QpkeToKem_8$; it also consciously treats domain separation.
%\pqcname{RLCE}~\cite{nistpqc:RLCE} deviates from Fujisaki--Okamoto-style transforms (and hence our framework) in which the randomness is derived from the same seed as the message.
%Many of these schemes face the same instantiation hurdles as those in our framework, and have as pressing a need for domain separation when cloning multiple oracles from a single primitive.

\heading{Signatures.}
Although the main focus of this paper is on domain separation in KEMs, we wish to note that these issues are not unique to KEMs.
At least one digital signature scheme in the second round of the NIST PQC competition, \pqcnameRoundTwo{MQDSS}~\cite{nistpqc:MQDSS}, models multiple hash functions as independent random oracles in its security proof, then clones them from the same primitive without explicit domain separation.
We have not analyzed the NIST PQC digital signature schemes' security to see whether more subtle domain separation is present, or whether oracle collisions admit the same vulnerabilities to signature forgery as they do to session key recovery.
This does, however, highlight that the problem of random oracle cloning is pervasive among more types of cryptographic schemes.% than KEMs in our specific framework.






% !TEX root = main.tex

% \section{Notational Preliminaries}
\section{Preliminaries}\label{sec-prelims}

%We switch gears to provide a theoretical framework that enables us to grasp and confirm methods for instantiating multiple random oracles by a single one, establishing domain separation as a conceptual goal.
%In order to do so, let us first pause to introduce some helpful notation and concepts.

\heading{Basic notation.} By $[i..j]$ we abbreviate the set $\{i,\ldots,j\}$, for integers $i \leq j$. If $\vecxx$ is a vector then $|\vecxx|$ is its length (the number of its coordinates), $\vecxx[i]$ is its $i$-th coordinate and $[\vecxx]=\set{\vecx[i]}{i\in [1..|\vecxx|]}$ is the set of its coordinates. The empty vector is denoted $()$. If $S$ is a set, then $S^*$ is the set of vectors over $S$, meaning the set of vectors of any (finite) length with coordinates in $S$. Strings are identified with vectors over $\bits$, so that if $x \in \bits^*$ is a string then $|x|$ is its length, $x[i]$ is its $i$-th bit, and $x[i..j]$ is the substring from its $i$-th to its $j$-th bit (including), for $i \leq j$. The empty string is $\emptystring$.  
%We let $x[i..j] = x[i]\ldots x[j]$ be the concatenation of bits $i$ through $j$ of $x$ if $i\leq j$, and $\emptystring$ otherwise. 
If $x,y$ are strings then we write $x \prefix y$ to indicate that $x$ is a prefix of $y$. If $S$ is a finite set then $|S|$ is its size (cardinality). A set $S\subseteq\bits^*$ is \textit{length closed} if $\bits^{|x|}\subseteq S$ for all $x\in S$. 

We let $y \gets A[\Oracle_1, \ldots](x_1,\ldots ; r)$ denote executing algorithm $A$ on inputs $x_1,\ldots$ and coins $r$, with access to oracles $\Oracle_1, \ldots$, and letting $y$ be the result. We let $y \getsr A[\Oracle_1, \ldots ](x_1,\ldots)$ be the resulting of picking $r$ at random and letting $y \gets A[\Oracle_1, \ldots ](x_1,\ldots;r)$. We let $\algOutput(A[\Oracle_1, \ldots ](x_1,\ldots))$ denote the set of all possible outputs of algorithm $A$ when invoked with inputs $x_1,\ldots$ and access to oracles $\Oracle_1, \ldots$. Algorithms are randomized unless otherwise indicated. Running time is worst case.
% ``PT'' stands for ``polynomial-time,'' whether for randomized algorithms or deterministic ones.
An adversary is an algorithm.

We use the code-based game-playing framework of~\cite{EC:BelRog06}. A game $\Gm$ (see Figure~\ref{fig:cc-indiff} for an example) starts with an $\Initialize$ procedure, followed by a non-negative number of additional procedures, and ends with a $\Finalize$ procedure. Procedures are also called oracles. Execution of adversary $\advA$ with game $\Gm$ consists of running $\advA$ with oracle access to the game procedures, with the restrictions that $\advA$'s first call must be to $\Initialize$, its last call must be to $\Finalize$, and it can call these two procedures at most once. The output of the execution is the output of $\Finalize$. 
%By $\Pr[\Gm(\advA)\Rightarrow y]$ we denote the probability that the execution of game $\Gm$ with adversary $\advA$ results in this output being $y$, and write just $\Pr[\Gm(\advA)]$ when $y=\true$, meaning  $\Pr[\Gm(\advA)]$ is the probability that the execution of game $\Gm$ with adversary $\advA$ results in the output of the execution being the boolean $\true$. 
We write $\Pr[\Gm(\advA)]$ to denote the probability that the execution of game $\Gm$ with adversary $\advA$ results in the output being the boolean $\true$.
Note that our adversaries have no output. The role of what in other treatments is the adversary output is, for us, played by the query to $\Finalize$. We adopt the convention that the running time of an adversary is the worst-case time to execute the game with the adversary, so the time taken by game procedures (oracles) to respond to queries is included.

\heading{Functions.} As usual $g\Colon\domain\to\rangeSet$ indicates that $g$ is a function taking inputs in the domain set $\domain$ and returning outputs in the range set $\rangeSet$. We may denote these sets by $\GGenroSpDom{g}$ and $\GGenroSpRng{g}$, respectively.

We say that $g\Colon \GGenroSpDom{g} \to \GGenroSpRng{g}$ has output length $\ell$ if $\GGenroSpRng{g}=\bits^{\ell}$. We say that $g$ is a single output-length (sol) function if there is some $\ell$ such that $g$ has output length $\ell$ and also the set $\domain$ is length closed. We let $\AllSOLFuncs{\domain}{\ell}$ denote the set of all sol functions $g\Colon\domain\to\bits^{\ell}$. 

We say $g$ is an extendable output length (xol) function if the following are true: (1) $\GGenroSpRng{g}=\bits^*$ (2) there is a length-closed set $\GGenroSpDomP{g}$ such that $\GGenroSpDom{g} = \GGenroSpDomP{g} \cross\N$ (3) $|g(x,\ell)|=\ell$ for all $(x,\ell)\in\GGenroSpDom{g}$, and (4) $g(x,\ell)\prefix g(x,\ell')$ whenever $\ell\leq\ell'$. We let $\AllXOLFuncs{\domain}$ denote the set of all xol functions $g\Colon\domain\to\bits^{*}$. 


% The image $\Fimage(f)$ of $f$ is the set $\{f(d):d \in \GGenroSpDom{f}\}$, which is a subset of $\GGenroSpRng{f}$. If the image of $f$ includes $\bot$, we let the support of $f$ be the set $\{d: d \in \GGenroSpDom{f}\text{ and } f(d) \neq \bot\}$.
%We say that $f$ has $n$ inputs if the members of $\domain$ are $n$-tuples, in this case writing as usual $f(d_1,\ldots,d_n)$ ---rather than the possibly more pedantically correct $f((d_1,\ldots,d_n))$--- for the output of $f$ on input $(d_1,\ldots,d_n)$.
%We let $\AllFuncs{\domain}{\rangeSet}$ denote the set of all functions $f\Colon \domain\to \rangeSet$.
%When $\domain,\rangeSet$ are finite, the size of this set is $|\rangeSet|^{|\domain|}$.
%A \textit{function space} $\roSp$ with domain $\domain$ and range $\rangeSet$ is simply a subset of $\AllFuncs{\domain}{\rangeSet}$.
%By $\aFunc{f} \getsr \roSp$ we mean that function $f$ is picked at random from $\roSp$.
%The distribution is uniform unless otherwise indicated.
%The domain and range of function space $\roSp$ are denoted $\GGenroSpDom{\roSp}$ and $\GGenroSpRng{\roSp}$, respectively.
%
%
%We capture providing oracle access to multiple functions $f_1,\ldots,f_n$ as providing an oracle for a single, two-input function $f$, with $f(i,\cdot)$ playing the role of $f_i$. We say that a function space $\roSp$ with domain $\domain$ and range $\rangeSet$ has \textit{arity $n$} if there are function spaces $\roSp_1, \ldots \roSp_n$ with domains  $\domain_1,\ldots,\domain_n$ such that $\domain = \Aset{(i,x)}{x\in\domain_i\mbox{ and }i\in [1..n]}$ and $f(i,\cdot) \in \roSp_i$ for all $i \in [1..n]$. % This captures the restriction that, regardless of the choice of $\aFunc{f}$ from $\roSp$, the domain of $\aFunc{f}(i,\cdot)$ should depend only on the index $i$.
%We refer to $\roSp_1,\ldots, \roSp_n$ as the sub-spaces of $\roSp$ and to $\domain_1,\ldots,\domain_n$ as the sub-domains of $\roSp$. If $f\in\roSp$ then we let functions $f_i\Colon\domain_i \to \rangeSet$ be defined by $f_i(\cdot) = f(i,\cdot)$, and refer to them as the sub-functions of $f$.
%
%We say that a function space $\FuncSp{RFS}$ with domain $\domain$, range $\rangeSet$, and arity $1$ is a \textit{full random function space} \TODO{better name} if $\FuncSp{RFS} = \AllFuncs{\domain}{\rangeSet}$. Equivalently, this means that when $f$ is drawn randomly from $\FuncSp{RFS}$, for any $x \in \domain$, $f(x)$ is uniformly distributed on $\rangeSet$, and $f(x)$ is independent of $f(y)$ for all $x,y \in \domain$ such that $x \neq y$. 
%We say that a function space $\FuncSp{XOFS}$ with arity $1$, domain $\domain \times \mathbb{N}$, and range $\bits^*$ is an \textit{extendable-output random function space} if
%\begin{multline*}
%\FuncSp{XOFS} = \{ f \in \AllFuncs{\domain \times \mathbb{N}}{\bits^*} \Colon\\
%(|f(x, n)| = n) \wedge (f(x,n) \prefix f(x,n+1))\forall(x,n) \in \domain \times \mathbb{N}\}.
%\end{multline*}
%When $f$ is drawn uniformly at random from $\FuncSp{XOFS}$, the value of $f(x,n)$ is independent of the value of $f(y,n')$ for all $x,y \in \domain$ and $n,n' \in \mathbb{N}$. 
%We say that a function space $\FuncSp{FS}$ with arity $n$ is a \textit{random function space} if and only if each of its sub-spaces is either a full random function space or an extendable-output random function space. 


% Informally, we talk about ``oracles" in a function space $\roSp$ of arity $n> 1$, we mean the $n$ oracles that compute these restrictions of the randomly drawn function $\aFunc{F} \in \roSp$. 



\section{Read-only indifferentiability of translating functors}\label{sec-cc-indiff}

We define read-only indifferentiability (rd-indff) of functors. Then we define a class of functors called translating, and give general results about their rd-indiff security. Later we will apply this to analyze the security of cloning functors, but the treatment in this section is broader and, looking ahead to possible future applications, more general than we need for ours.



\subsection{Functors and read-only indifferentiability} 

A random oracle, formally, is a function drawn at random from a certain space of functions. A construction (functor) is a mapping from one such space to another. We start with definitions for these.

\heading{Function spaces and functors.} A function space $\roSp$ is simply a set of functions, with the requirement that all functions in the set have the same domain $\Dom(\roSp)$ and the same range $\Rng(\roSp)$. Examples are $\AllSOLFuncs{\domain}{\ell}$ and $\AllXOLFuncs{\domain}$. Now $f\getsr\roSp$ means we pick a function uniformly at random from the set $\roSp$. 

Sometimes (but not always) we want an extra condition called input independence. It asks that the values of $f$ on different inputs are identically and independently distributed when $f\getsr\roSp$. More formally, let $\domain$ be a set and let $\FSOutSet$ be a function that associates to any $W \in \domain$ a set $\FSOutSet(W)$. Let $\FSOutSet(\domain)$ be the union of the sets $\FSOutSet(W)$ as $W$ ranges over $\domain$. Let $\AllFuncs{\domain}{\FSOutSet}$ be the set of all functions $f\Colon\domain\to\FSOutSet(\domain)$ such that $f(W)\in \FSOutSet(W)$ for all $W\in\domain$. We say that $\roSp$ provides input independence if there exists such a $\FSOutSet$ such that $\roSp = \AllFuncs{\Dom(\roSp)}{\FSOutSet}$. Put another way, there is a bijection between $\roSp$ and the set $S$ that is the cross product of the sets $\FSOutSet(W)$ as $W$ ranges over $\Dom(\roSp)$. (Members of $S$ are $|\Dom(\roSp)|$-vectors.) As an example the function space $\AllSOLFuncs{\domain}{\ell}$ satisfies input independence, but $\AllXOLFuncs{\domain}$ does \textit{not} satisfy input independence.

Let $\GenroSp{\functionInSet}$ be a function space that we call the starting space. Let $\GenroSp{\functionOutSet}$ be another function space that we call the ending space. We imagine that we are given a function $\functionIn\in\GenroSp{\functionInSet}$ and want to construct a function $\functionOut\in\GenroSp{\functionOutSet}$. We refer to the object doing this as a functor. Formally a \textit{functor} is a deterministic algorithm $\construct{F}$ that, given as oracle a function $\aFunc{\functionIn}\in\GenroSp{\functionInSet}$, returns a function $\construct{F}[\aFunc{\functionIn}]\in\GenroSp{\functionOutSet}$. We write $\construct{F}\Colon \GenroSp{\functionInSet}\to \GenroSp{\functionOutSet}$ to emphasize the starting and ending spaces of functor $\construct{F}$. 



 
 
 % \hd{Can we give some intuition here as to what this means? It's what makes our random oracles random.}

\heading{Rd-indiff.} We want the ending function to ``emulate'' a random function from $\GenroSp{\functionOutSet}$. Indifferentiability is a way of defining what this means. The original definition of MRH~\cite{TCC:MauRenHol04} has been followed by many variants~\cite{C:CDMP05,EC:RisShaShr11,EC:DGHM13,EC:Mittelbach14}. 
Here we give ours, called read-only indifferentiability, which implies composition not just for single-stage games, but even for multi-stage ones~\cite{EC:RisShaShr11,EC:DGHM13,EC:Mittelbach14}. 

Let $\GenroSp{\functionOutSet}$ and $\GenroSp{\functionInSet}$ be function spaces, and let $\construct{F}\Colon \GenroSp{\functionInSet}\to 
\GenroSp{\functionOutSet}$ be a functor. Our variant of indifferentiability mandates a particular, strong simulator, which can read, but not write, its (game-maintained) state, so that this state is a static quantity. 
% \fg{Why do we use the word ``state'' here? It suggests being mutable.}
% Not sure what else to use or what mutable means.
Formally a \textit{read-only simulator}~$\simulator$ for
% a primitive oracle space~$\GenroSp{\functionInSet}$ from a target oracle space~$\GenroSp{\functionOutSet}$
$\construct{F}$ specifies a \textit{setup algorithm}  $\SimgenCC$ which outputs the state, and a  deterministic \textit{evaluation algorithm} $\Simeval$ that, given as oracle a function~$\aFunc{\functionOut} \in \GenroSp{\functionOutSet}$, and given a string $\commoncoins \in \algOutput(\SimgenCC)$ (the read-only state), defines a function $\Simeval[\aFunc{\functionOut}](\commoncoins,\cdot) \Colon\GenroSpDom{\functionInSet}\to\GenroSpRng{\functionInSet}$. 

\begin{figure}[t]
	\twoColsNoDivide{0.35}{0.4}{
		\ExperimentHeader{Game $\ngameCCINDIFF{\construct{F},\GenroSp{\functionInSet},\GenroSp{\functionOutSet},\workDom,\simulator}$}
		
		\ExptSepSpace
		
		\begin{oracle}{$\Initialize$}
			\item $\aFunc{\functionIn} \getsr \GenroSp{\functionInSet}$
			\item $\aFunc{\functionOut}_1 \gets \construct{F}[\aFunc{\functionIn}]$ ; $\aFunc{\functionOut}_0 \getsr \GenroSp{\functionOutSet}$
			\item $b \getsr \bits$
			\item $\commoncoins \getsr \SimgenCC()$
			% Should the following be there? \hd{No}
			%	\item Return $\commoncoins$
		\end{oracle}
		
		\ExptSepSpace
		
	}{
		\begin{oracle}{$\priv(W)$}
			\item If $W \in\workDom$ then return $\aFunc{\functionOut}_b(W)$
			\item Else return $\bot$
		\end{oracle}
		\ExptSepSpace
		
		\begin{oracle}{$\pub(U)$}
			\item if $(b=1)$ then return $\aFunc{\functionIn}(U)$
			\item else return $\Simeval[\functionOut_0](\commoncoins, U)$
		\end{oracle}
		
		\ExptSepSpace
		
		\begin{oracle}{$\Finalize(b')$}
			\item return $(b = b')$
		\end{oracle}
	}
	
	\caption{Game defining read-only indifferentiability.}
	\label{fig:cc-indiff}
	\hrulefill
\end{figure}

The intent is that $\Simeval[\aFunc{\functionOut}](\commoncoins,\cdot)$  play the role of a starting function $\aFunc{\functionIn} \in \GenroSp{\functionInSet}$ satisfying $\construct{F}[\functionIn] = \functionOut$. To formalize this, consider the read-only indifferentiability game $\ngameCCINDIFF{\construct{F},\GenroSp{\functionInSet},\GenroSp{\functionOutSet},\workDom,\simulator}$ of Figure~\ref{fig:cc-indiff}, where $\workDom\subseteq\GenroSpDom{\functionOutSet}$ is called the working domain. The adversary $\advA$ playing this game is called a distinguisher. Its advantage is defined as
\[
\AdvCCINDIFF{\construct{F},\GenroSp{\functionInSet},\GenroSp{\functionOutSet},\workDom,\simulator}{\advA} =
2 \cdot \Pr\left[ \ngameCCINDIFF{\construct{F},\GenroSp{\functionInSet},\GenroSp{\functionOutSet},\workDom,\simulator}(\advA)\right] - 1.
\]
To explain, in the game, $b$ is a challenge bit that the distinguisher is trying to determine.
Function $\functionOut_b$ is a random member of the ending space $\GenroSp{\functionOutSet}$ if $b=0$ and is $\construct{F}[\aFunc{\functionIn}](\cdot)$ if $b=1$. The query $W$ to oracle $\priv$ is required to be in $\GenroSpDom{\functionOutSet}$. The oracle returns the value of $\functionOut_b$ on $W$, but only if $W$ is in the working domain, otherwise returning $\bot$. The query $U$ to oracle $\pub$ is required to be in $\GenroSpDom{\functionInSet}$. The oracle returns the value of $\aFunc{\functionIn}$ on $U$ in the $b=1$ case, but when $b=0$, the simulator evaluation algorithm $\Simeval$ must answer the query with access to an oracle for $\aFunc{\functionOut}_0$. The distinguisher ends by calling $\Finalize$ with its guess $b'\in\bits$ of $b$ and the game returns $\true$ if $b'=b$ (the distinguisher's guess is correct) and $\false$ otherwise.

The working domain $\workDom\subseteq\GenroSpDom{\functionOutSet}$, a parameter of the definition, is included as a way to allow the notion of read-only indifferentiability to provide results for oracle cloning methods like length differentiation whose security depends on domain restrictions.  

The $\Simeval$ algorithm is given direct access to $\aFunc{\functionOut}_0$, rather than access to $\priv$ as in other definitions, to bypass the working domain restriction, meaning it may query $\aFunc{\functionOut}_0$ at points in $\GenroSpDom{\functionOutSet}$ that are outside the working domain.

All invocations of $\Simeval[\functionOut_0]$ are given the same (static, game-maintained) state $\commoncoins$ as input, but $\Simeval[\functionOut_0]$ cannot modify this state, which is why it is called read-only. 
Note $\Initialize$ does not return $\commoncoins$, meaning the state is not given to the distinguisher. 

%In the $b=0$ case, both $\advA$ and $\Simeval$ make queries to $\priv = \functionOut_0$, but while those of $\advA$ are required to be in the subset $\workDom$ of $\GenroSpDom{\functionOutSet}$, those of $\Simeval$ are not so constrained, meaning can range over all of $\GenroSpDom{\functionOutSet}$. 
%This is why having working domains is not the same as restricting the domain of $\functionOutSet$ from the start.
%In Section~\ref{sec-kem} we will discuss how the working domain should be chosen; here we leave it as an arbitrary set. 


\heading{Discussion.} To compare rd-indiff to other indiff notions, we set $\workDom = \GenroSpDom{\functionOutSet}$, because prior notions do not include working domains. Now, rd-indiff differs from prior indiff notions because it requires that the simulator state be just the immutable string chosen at the start of the game. In this regard, rd-indiff falls somewhere between the original MRH-indiff~\cite{TCC:MauRenHol04} and reset indiff~\cite{EC:RisShaShr11} in the sense that our simulator is more restricted than in the first and less than in the second. A construction (functor) that is reset-indiff is thus rd-indiff, but not necessarily vice-versa, and a construct that is rd-indiff is MRH-indiff, but not necessarily vice-versa. Put another way, the class of rd-indff functors is larger than the class of reset-indiff ones, but smaller than the class of MRH-indiff ones. Now, RSS's proof~\cite{EC:RisShaShr11} that reset-indiff implies security for multi-stage games extends to rd-indiff, so we get this for a potentially larger class of functors. This larger class includes some of the cloning functors we have described, which are not necessarily reset-indiff.

\subsection{Translating functors}

\heading{Translating functors.} We focus on a class of functors that we call translating. This class includes natural and existing oracle cloning methods, in particular all the effective methods used by NIST KEMs, and we will be able to prove general results for translating functors that can be applied to the cloning methods. 

A translating functor $\construct{T} \Colon \GenroSp{\functionInSet}\to \GenroSp{\functionOutSet}$ is a functor that, with oracle access to $\functionIn$ and on input $W \in\GenroSpDom{\functionOutSet}$, non-adaptively calls $\functionIn$ on a fixed number of inputs, and computes its output $\construct{T}[\functionIn](W)$ from the responses and $W$. Its operation can be split into three phases which do not share state: (1) a pre-processing phase which chooses the inputs to $\aFunc{\functionIn}$ based on $W$ alone (2) the calls to $\aFunc{\functionIn}$ to obtain responses (3) a post-processing phase which uses $W$ and the responses collected in phase 2 to compute the final output value $\construct{T}[\functionIn](W)$.

Proceeding to the definitions, let $\GenroSp{\functionInSet},\GenroSp{\functionOutSet}$ be function spaces. A $(\FuncSp{\functionInSet},\GenroSp{\functionOutSet})$-\textit{query translator} is a function (deterministic algorithm) $\QuT\Colon \GenroSpDom{\functionOutSet} \to \GenroSpDom{\functionInSet}^*$, meaning it takes a point $W$ in the domain of the ending space and returns a vector of points in the domain of the starting space. This models the pre-processing. A $(\FuncSp{\functionInSet},\FuncSp{\functionOutSet})$-\textit{answer translator} is a function (deterministic algorithm) $\AnT\Colon  \GenroSpDom{\functionOutSet} \times \GenroSpRng{\functionInSet}^* \to \GenroSpRng{\functionOutSet}$, meaning it takes the original $W$, and a vector of points in the range of the starting space, to return a point in the range of the ending space. This models the post-processing. To the pair $(\QuT, \AnT)$, we associate the functor $\construct{TF}_{\QuT,\AnT}\Colon \GenroSp{\functionInSet}\to\GenroSp{\functionOutSet}$, defined as follows:
\begin{tabbing}
	1234\=123\=\kill
	\> \underline{Algorithm $\construct{TF}_{\QuT,\AnT}[\aFunc{\functionIn}](W)$} \comment{Input $W \in \GenroSpDom{\functionOutSet}$ and oracle $\aFunc{\functionIn} \in \GenroSp{\functionInSet}$} \\[2pt]
	\> $\vecU\gets\QuT(W)$ \\
	\> For $j=1,\ldots,|\vecU|$ do $\vecV[j] \gets \aFunc{\functionIn}(\vecU[j])$ \comment{$\vecU[j]\in\Dom(\GenroSp{\functionInSet})$} \\
	\> $Y \gets \AnT(W,\vecV)$ ; Return $Y$
\end{tabbing}
The above-mentioned calls of phase~(2) are done in the second line of the code above, so that this implements a translating functor as we described. Formally we say that a functor $\construct{F} \Colon \GenroSp{\functionInSet}\to \GenroSp{\functionOutSet}$ 
% on $\workDom$ 
is \textit{translating} if there exists a $(\GenroSp{\functionInSet},\GenroSp{\functionOutSet})$-query translator $\QuT$ and a $(\GenroSp{\functionInSet},\GenroSp{\functionOutSet})$-answer translator $\AnT$ such that $\construct{F} = \construct{TF}_{\QuT,\AnT}$.
%\fg{Where does $\workDom$ come from here? Is it needed, is it used later like that?}

\heading{Inverses.}
So far, query and answer translators may have  just seemed an unduly complex way to say that a translating oracle construction is one that makes non-adaptive oracle queries. The purpose of making the query and answer translators explicit is to define \textit{invertibility}, which determines rd-indiff security.
 
Let $\GenroSp{\functionInSet}$ and $\GenroSp{\functionOutSet}$ be function spaces. Let $\QuTInv$ be a function (deterministic algorithm) that takes an input $U \in \GenroSpDom{\functionInSet}$ and returns a vector $\vecW$ over $\GenroSpDom{\functionOutSet}$. We allow $\QuTInv$ to return the empty vector $()$, which is taken as an indication of failure to invert. Define the \textit{support} of $\QuTInv$, denoted $\supportQuT{\QuTInv}$, to be the set of all $U\in\GenroSpDom{\functionInSet}$ such that $\QuTInv(U)\neq ()$. Say that $\QuTInv$ has \textit{full support} if $\supportQuT{\QuTInv} = \GenroSpDom{\functionInSet}$, meaning there is no $U \in \GenroSpDom{\functionInSet}$ such that $\QuTInv(U)=()$. Let $\AnTInv$ be a function (deterministic algorithm) that takes $U\in\GenroSpDom{\functionInSet}$ and a vector $\vecY$ over $\GenroSpRng{\functionOutSet}$ to return an output in $\GenroSpRng{\functionInSet}$. Given a function $\aFunc{\functionOut} \in \GenroSp{\functionOutSet}$, we define the function $\mathrm{P}[\aFunc{\functionOut}]_{\QuTInv,\AnTInv} \Colon \GenroSpDom{\functionInSet}\to\GenroSpRng{\functionInSet}$ by 
\begin{tabbing}
	123456\=\kill
	\> \underline{Function $\mathrm{P}[\aFunc{\functionOut}]_{\QuTInv,\AnTInv}(U)$} \comment{$U  \in % \supportQuT{\QuTInv} \subseteq 
	\GenroSpDom{\functionInSet}$} \\[2pt]
	\> $\vecW \gets \QuTInv(U)$ ; $\vecY\gets \aFunc{\functionOut}(\vecW)$ ;
	$V\gets \AnTInv(U,\vecY)$ ; Return $V$
\end{tabbing}
Above, $\aFunc{\functionOut}$ is applied to a vector component-wise, meaning $\aFunc{\functionOut}(\vecW)$ is defined as the vector $(\aFunc{\functionOut}(\vecW[1]),\allowbreak \ldots,\allowbreak \aFunc{\functionOut}(\vecW[|\vecW|]))$. 
% As the comment indicates, the input $U$ to the function is in $\supportQuT{\QuTInv} \subseteq \GenroSpDom{\functionInSet}$.

We require that the function $\mathrm{P}[\aFunc{\functionOut}]_{\QuTInv,\AnTInv}$ belong to the starting space $\GenroSp{\functionInSet}$. Now let $\QuT$ be a $(\GenroSp{\functionInSet},\GenroSp{\functionOutSet})$-query translator and $\AnT$ a $(\GenroSp{\functionInSet},\GenroSp{\functionOutSet})$-answer translator. Let $\workDom\subseteq\GenroSpDom{\functionOutSet}$ be a working domain. We say that \textit{$\QuTInv,\AnTInv$ are inverses of $\QuT,\AnT$ over $\workDom$} if two conditions are true. The first is that for all  $\aFunc{\functionOut}\in\GenroSp{\functionOutSet}$ and all $W \in \workDom$ we have
	\begin{equation}
	\construct{TF}_{\QuT,\AnT}
	[\mathrm{P}[\aFunc{\functionOut}]_{\QuTInv,\AnTInv}](W) = \aFunc{\functionOut}(W) \;.  \label{eq-invertible-def}
	\end{equation}
This equation needs some parsing. Fix a function $\aFunc{\functionOut}\in\GenroSp{\functionOutSet}$ in the ending space. Then $\aFunc{\functionIn} = \mathrm{P}[\aFunc{\functionOut}]_{\QuTInv,\AnTInv}$ is in $\GenroSp{\functionInSet}$. Recall that the functor $\construct{F} = \construct{TF}_{\QuT,\AnT}$ takes a function $\aFunc{\functionIn}$ in the starting space as an oracle and defines a function $\functionOut' = \construct{F}[\aFunc{\functionIn}]$ in the ending space. Equation~(\ref{eq-invertible-def}) is asking that $\functionOut'$ is identical to the original function $\functionOut$, on the working domain $\workDom$. The second condition (for invertibility) is that if $U\in \set{\QuT(W)[i]}{W\in\workDom}$ ---that is, $U$ is an entry of the vector $\vecU$ returned by $\QuT$ on some input $W$--- then $\QuTInv(U)\neq ()$. Note that if $\QuTInv$ has full support then this condition is already true, but otherwise it is an additional requirement.

% the restriction of some function in $\GenroSp{\functionInSet}$ to inputs in $\supportQuT{\QuTInv}$. 

We say that $(\QuT,\AnT)$ is invertible over $\workDom$ if there exist $\QuTInv,\AnTInv$ such that $\QuTInv,\AnTInv$ are inverses of $\QuT,\AnT$ over $\workDom$, and we say that a translating functor $\construct{TF}_{\QuT,\AnT}$ is invertible over $\workDom$ if $(\QuT,\AnT)$ is invertible over $\workDom$.

In the rd-indiff context, function $\mathrm{P}[\aFunc{\functionOut}]_{\QuTInv,\AnTInv}$ will be used by the simulator. Roughly, we try to set $ \Simeval[\functionOut](\commoncoins, U) = \mathrm{P}[\aFunc{\functionOut}]_{\QuTInv,\AnTInv}(U)$. But we will only be able to successfully do this for $U \in \supportQuT{\QuTInv}$. The state $\commoncoins$ is used by $\Simeval$ to provide replies when $U \not\in \supportQuT{\QuTInv}$. 

\begin{figure}[t]
	\oneCol{0.5}{
		\ExperimentHeader{Game $\ngameTI{\GenroSp{\functionInSet},\GenroSp{\functionOutSet},\QuTInv,\AnTInv}$}
		
	%	\ExptSepSpace
		
		\begin{oracle}{$\Initialize$}
			\item $b\getsr\bits$ ; $\aFunc{\functionOut} \getsr \GenroSp{\functionOutSet}$ 
			\item $\functionIn_1\getsr \GenroSp{\functionInSet}$ ; 
					$\functionIn_0 \gets \mathrm{P}[\aFunc{\functionOut}]_{\QuTInv,\AnTInv}$ 
		\end{oracle}
		
		\ExptSepSpace
		

		\begin{oracleC}{$\pub(U)$}{$U\in\GenroSpDom{\functionInSet}$}
			\item If $\QuTInv(U)=()$ then return $\bot$
			\item return $\functionIn_b(U)$
		\end{oracleC}
		\ExptSepSpace
		
		
		\begin{oracle}{$\Finalize(b')$}
			\item return $(b = b')$
		\end{oracle}\smallskip
	}
	
	\caption{Game defining translation indistinguishability.}
	\label{fig:translating-condition}
	\hrulefill
\end{figure}



Equation~(\ref{eq-invertible-def}) is a correctness condition. There is also a security metric. Consider the \textit{translation indistinguishability} game  $\ngameTI{\GenroSp{\functionInSet},\GenroSp{\functionOutSet},\QuTInv,\AnTInv}$ of Figure~\ref{fig:translating-condition}. Define the ti-advantage of adversary $\advB$ via
\[
\AdvTI{\GenroSp{\functionInSet},\GenroSp{\functionOutSet},\QuTInv,\AnTInv}{\advB} =
2 \cdot \Pr\left[ \ngameTI{\GenroSp{\functionInSet},\GenroSp{\functionOutSet},\QuTInv,\AnTInv}(\advB)\right] - 1.
\]
%	\begin{center}\begin{tabular}{cc|cc}
%			\begin{minipage}{2in}
%				\begin{tabbing}
%					123\=\kill
%					$\aFunc{\functionIn} \getsr \GenroSp{\functionInSet}$ \\
%					For all $U\in\GenroSpDom{\functionInSet}$ do \\
%					\> If $\QuTInv(U)=()$ then $\aFunc{\functionIn}(u)\gets\bot$ \\
%					Return $\aFunc{\functionIn}$
%			\end{tabbing} \end{minipage} &   \hspace{2pt}     &   \hspace{2pt}      &
%			\begin{minipage}{2in}
%				\begin{tabbing}
%					123\=\kill
%					$\aFunc{\functionOut} \getsr \GenroSp{\functionOutSet}$ ;
%					$\aFunc{\functionIn} \gets \mathrm{P}[\aFunc{\functionOut}]_{\QuTInv,\AnTInv}$ \\
%					For all $U\in\GenroSpDom{\functionInSet}$ do \\
%					\> If $\QuTInv(U)=()$ then $\aFunc{\functionIn}(u)\gets\bot$ \\
%					Return $\aFunc{\functionIn}$
%				\end{tabbing}
%			\end{minipage}
%	\end{tabular}\end{center}
In reading the game, recall that $()$ is the empty vector, whose return by $\QuTInv$ represents an inversion error. TI-security is thus asking that if $\aFunc{\functionOut}$ is randomly chosen from the ending space, then the output of $\mathrm{P}[\aFunc{\functionOut}]_{\QuTInv,\AnTInv}$ on an input $U$ is distributed like the output on $U$ of a random function in the starting space, \textit{but only as long as $\QuTInv(U)$ was non-empty}. We will see that the latter restriction creates some challenges in simulation whose resolution exploits using read-only state. We say that $(\QuTInv,\AnTInv)$ provides perfect translation indistinguishability if $\AdvTI{\GenroSp{\functionInSet},\GenroSp{\functionOutSet},\QuTInv,\AnTInv}{\advB} = 0$ for all $\advB$, regardless of the running time of $\advB$. 

%\begin{lemma}\label{th-qt-invertible} Suppose $\GenroSp{\functionInSet},\GenroSp{\functionOutSet}$ are function spaces. Suppose $\QuT$ is a $(\GenroSp{\functionInSet},\GenroSp{\functionOutSet})$-query translator and $\AnT$ is a $(\GenroSp{\functionInSet},\GenroSp{\functionOutSet})$-answer translator. Suppose the length of the vectors returned by $\QuT$ and $\AnT$ are one. Suppose $\QuTInv,\AnTInv$ are full inverses of $\QuT,\AnT$ over $\workDom$. 
%

%Let $\aFunc{\functionInSet}$ and $\aFunc{\functionOutSet}$ be function spaces such that $\GenroSpRng{\functionInSet} = \GenroSpRng{\functionOutSet}$. Let $\workDom \subseteq \GenroSpDom{\functionOutSet}$. Let $\QuT\Colon \GenroSpDom{\functionOutSet} \to \GenroSpDom{\functionInSet}$ be a query translation for $\GenroSp{\functionInSet},\GenroSp{\functionOutSet},\workDom$ with inverse $\QuTInv$. 
%	Let $\AnT \Colon \GenroSpDom{\functionOutSet}\cross \GenroSpRng{\functionInSet} \to \GenroSpRng{\functionOutSet}$ be defined such that $\AnT(x,y) = y$ for all $x \in \GenroSpDom{\functionOutSet}$ and $y \in \GenroSpRng{\functionInSet}$.
%	Then $(\QuT,\AnT)$ is invertible.
%\end{lemma}



Additionally we of course ask that the functions $\QuT,\AnT,\QuTInv,\AnTInv$ all be efficiently computable. In an asymptotic setting, this means they are polynomial time. In our concrete setting, they show up in the running-time of the simulator or constructed adversaries. (The latter, as per our conventions, being the time for the execution  of the adversary with the overlying game.) 

\subsection{Rd-indiff of translating functors} 

We now move on to showing that invertibility of a pair $(\QuT,\AnT)$ implies rd-indifferentiability of the translating functor $\construct{TF}_{\QuT,\AnT}$.
%  We indicated above that simulation is harder when $\QuTInv$ may return the empty vector $()$. We will distinguish this case. Define the \textit{support} of $\QuTInv$, denoted $\supportQuT{\QuTInv}$, to be the set of all $U\in\GenroSpDom{\functionInSet}$ such that $\QuTInv(U)\neq ()$. Say that $\QuTInv$ has \textit{full support} if $\supportQuT{\QuTInv} = \GenroSpDom{\functionInSet}$, meaning there is no $U \in \GenroSpDom{\functionInSet}$ such that $\QuTInv(U)=()$.
% In this case, the second condition in the invertibility requirement above simplifies, with $p$, in both experiments, never returning $\bot$. 
We start with the case that $\QuTInv$ has full support.


\begin{figure}[t]
	\begin{twoCols}{0.25}{0.6}{
			\begin{algorithm}{Algorithm $\SimgenCC$}
				\item Return $\emptystring$
			\end{algorithm}
		}
		{
			\begin{algorithm}{Algorithm $\Simeval[\aFunc{\functionOut}](\commoncoins,U)$}
				\item $\vecW\gets\QuTInv(U)$ ; $\vecY \gets \aFunc{\functionOut}(\vecW)$ ; $V\gets\AnTInv(U,\vecY)$
				\item Return $V$ \smallskip
			\end{algorithm}
		}
	\end{twoCols}
	\begin{twoCols}{0.25}{0.6}{
			\begin{algorithm}{Algorithm $\SimgenCC$}
				\item $\commoncoins \getsr \bits^{\prfKl}$
				\item Return $\commoncoins$
			\end{algorithm}
		}
		{
			\begin{algorithm}{Algorithm $\Simeval[\aFunc{\functionOut}](\commoncoins,U)$}
				\item $\vecW\gets\QuTInv(U)$
				\item If $\vecW=()$ then return $\prf_{\commoncoins}[{\functionOut}](U)$
				\item $\vecY \gets \aFunc{\functionOut}(\vecW)$ ; $V\gets\AnTInv(U,\vecY)$
				\item Return $V$
			\end{algorithm}\smallskip
		}
	\end{twoCols}
	\caption{Simulators for Theorem~\ref{th-cc-indiff-invertible} (top) and Theorem~\ref{th-cc-indiff-invertible-gen} (bottom).}
	\label{fig-th-cc-indiff-invertible}
	\label{fig:cc-indiff-sims}
\end{figure}

\begin{theorem}\label{th-cc-indiff-invertible}  Let $\GenroSp{\functionInSet}$ and $\GenroSp{\functionOutSet}$ be function spaces. Let $\workDom$ be a subset of $\GenroSpDom{\functionOutSet}$.
	Let $\QuT,\AnT$ be $(\GenroSp{\functionInSet},\GenroSp{\functionOutSet})$ query and answer translators, respectively. Let $\QuTInv,\AnTInv$ be inverses of $\QuT,\AnT$ over $\workDom$. Assume $\QuTInv$ has full support. Define read-only simulator $\simulator$ as per the top panel of Figure~\ref{fig-th-cc-indiff-invertible}.
	Let $\construct{F} = \construct{TF}_{\QuT,\AnT}$. Let $\advA$ be any distinguisher.  Then we construct a ti-adversary $\advB$ such that
	\begin{align*}
	\AdvCCINDIFF{\construct{F},\GenroSp{\functionInSet},\GenroSp{\functionOutSet},\workDom,\simulator}{\advA}  & \leq \AdvTI{\GenroSp{\functionInSet},\GenroSp{\functionOutSet},\QuTInv,\AnTInv}{\advB} \;.
	\end{align*}
	Let $\ell$ be the maximum output length of $\QuT$. 
	If $\advA$ makes $q_{\priv},q_{\pub}$ queries to its $\priv,\pub$ oracles, respectively, then $\advB$ makes $\ell\cdot q_{\priv}+q_{\pub}$ queries to its $\pub$ oracle. The running time of $\advB$ is about that of $\advA$. 
\end{theorem}
\begin{figure}[t]
	\twoCols{0.48}{0.38}{
		\ExperimentHeader{Games $\Gm_0$, $\Gm_1$} 
		
		\begin{oracle}{$\Initialize$}
			\item $\aFunc{\functionIn} \getsr \GenroSp{\functionInSet}$\Comment{Game $\Gm_0$} 
			\item $\aFunc{\functionOut}_0 \getsr \GenroSp{\functionOutSet}$ ; 
			$\functionIn \gets \mathrm{P}[\aFunc{\functionOut}_0]_{\QuTInv,\AnTInv}$ \ccomment{Game $\Gm_1$}
		\end{oracle}
		
		\ExptSepSpace
		
		\begin{oracle}{$\priv(W)$}
			\item If $W\in\workDom$ then return $\construct{F}[\aFunc{\functionIn}](W)$
			\item Else return $\bot$
		\end{oracle}
		\ExptSepSpace
		
		\begin{oracle}{$\pub(U)$}
			\item return $\aFunc{\functionIn}(U)$
		\end{oracle}
		
		\ExptSepSpace
		
		\begin{oracle}{$\Finalize(b')$}
			\item return $(b' = 1)$
		\end{oracle}
		
		%	\ExperimentHeader{Game $\Gm_1$} 
		%
		%		\begin{oracle}{$\Initialize$}
		%			
		%			\item $\aFunc{\functionOut}_1 \gets \construct{F}[\aFunc{\functionIn}]$ 	
		%		%	\item $\commoncoins \getsr \SimgenCC()$
		%		\end{oracle}
		%		
		%		\ExptSepSpace
		%
		%		
		%		\begin{oracle}{$\priv(X)$}
		%			\item If $X\not\in\workDom$ then 
		%			\item \hindent return $\bot$ 
		%			\item return $\aFunc{\functionOut}_1(X)$
		%		\end{oracle}
		%		\ExptSepSpace
		%		
		%		\begin{oracle}{$\pub(U)$}
		%			\item return $\aFunc{\functionIn}(U)$
		%		\end{oracle}
		%		
		%		\ExptSepSpace
		%		
		%		\begin{oracle}{$\Finalize(b')$}
		%			\item return $(b' = 1)$
		%		\end{oracle}
	}
	{
		\ExperimentHeader{Game $\Gm_2$} 
		
		\begin{oracle}{$\Initialize$}
			\item $\aFunc{\functionOut}_0 \getsr \GenroSp{\functionOutSet}$ 
			\item  $\functionIn \gets \mathrm{P}[\aFunc{\functionOut}_0]_{\QuTInv,\AnTInv}$
			% \item $\aFunc{\functionOut}_1 \gets \construct{F}[\aFunc{\functionIn}]$ 	
			% \item $\commoncoins \getsr \SimgenCC()$
		\end{oracle}
		
		\ExptSepSpace
		
		
		\begin{oracle}{$\priv(W)$}
			\item If $W\in\workDom$ then return $\aFunc{\functionOut}_0(W)$
			\item Else return $\bot$ 
		\end{oracle}
		\ExptSepSpace
		
		\begin{oracle}{$\pub(U)$}
			\item return $\aFunc{\functionIn}(U)$
		\end{oracle}
		
		\ExptSepSpace
		
		\begin{oracle}{$\Finalize(b')$}
			\item return $(b' = 1)$
		\end{oracle}
	}
	\twoColsNoDivide{0.38}{0.5}{
		\begin{algorithm-initial}{Adversary $\advB$}
			\item $\Initialize()$
			\item $\advA[\Init', \pub', \priv', \Fin']()$
		\end{algorithm-initial} 
		
		\ExptSepSpace
		\begin{algorithm-subsequent}{$\Init'$}
			\item Return
		\end{algorithm-subsequent}
		\ExptSepSpace
		\begin{algorithm-subsequent}{$\pub'(U)$}
			\item return $\pub(U)$
		\end{algorithm-subsequent}
	}{
		\ExptSepSpace
		\begin{algorithm-subsequent}{$\priv'(W)$}
			\item if $W \not \in \workDom$ then return $\bot$
			\item $\vecU \gets \QuT(W)$
			\item For $j=1,\ldots,|\vecU|$ do 
			$\vecV[j] \gets \pub(\vecU[j])$
			\item return $\AnT(W,\vecV)$
		\end{algorithm-subsequent}
		\ExptSepSpace
		\begin{algorithm-subsequent}{$\Fin'(b')$}
			\item $\Finalize(b')$
		\end{algorithm-subsequent}
	}
	\caption{Top: Games for proof of Theorem~\ref{th-cc-indiff-invertible}. Bottom: Adversary for proof of Theorem~\ref{th-cc-indiff-invertible}.}
	\label{fig:th-cc-indiff-invertible-games}
	\hrulefill
\end{figure}
\begin{proof}[Theorem~\ref{th-cc-indiff-invertible}] Consider the games of Figure~\ref{fig:th-cc-indiff-invertible-games}. In the left panel, line 1 is included only in $\Gm_0$ and line 2 only in $\Gm_1$, and this is the only way the games differ. Game $\Gm_0$ is the real game, meaning the case $b=1$ in game $\ngameCCINDIFF{\construct{F},\GenroSp{\functionInSet},\GenroSp{\functionOutSet},\workDom,\simulator}$. In game $\Gm_2$, oracle $\priv$ is switched to a random function $\functionOut_0$. From the description of the simulator in Figure~\ref{fig:cc-indiff-sims} we see that
	\begin{align*}
	\Simeval[\functionOut_0](\emptystring,U) &= \mathrm{P}[\aFunc{\functionOut}_0]_{\QuTInv,\AnTInv}(U)
	\end{align*}
	for all $U\in \GenroSpDom{\functionInSet}$ and all $\functionOut_0\in\GenroSp{\functionOutSet}$, so that oracle $\pub$ in game $\Gm_2$ is responding according to the simulator based on $\functionOut_0$. So game $\Gm_2$ is the case $b=0$ in game $\ngameCCINDIFF{\construct{F},\GenroSp{\functionInSet},\GenroSp{\functionOutSet},\workDom,\simulator}$. Thus
	\begin{align*}
	\AdvCCINDIFF{\construct{F},\GenroSp{\functionInSet},\GenroSp{\functionOutSet},\workDom,\simulator}{\advA} & =  \Pr[\Gm_0(\advA)] - \Pr[\Gm_2(\advA)] \\
	&= (\Pr[\Gm_0(\advA)] - \Pr[\Gm_1(\advA)])+(\Pr[\Gm_1(\advA)] - \Pr[\Gm_2(\advA)]) \;.
	\end{align*}
	We define translation-indistinguishability adversary $\advB$ in Figure~\ref{fig:th-cc-indiff-invertible-games} so that 
	\begin{align*}
	\Pr[\Gm_0(\advA)] - \Pr[\Gm_1(\advA)] & \leq \AdvTI{\GenroSp{\functionInSet},\GenroSp{\functionOutSet},\QuTInv,\AnTInv}{\advB} \;.
	\end{align*}
	Adversary $\advB$ is playing game  $\ngameTI{\GenroSp{\functionInSet},\GenroSp{\functionOutSet},\QuTInv,\AnTInv}$. Using its $\pub$ oracle, it presents the interface of $\Gm_0$ and $\Gm_1$ to $\advA$. In order to simulate the $\priv$ oracle, $\advB$ runs $\construct{TF}_{\QuT,\AnT}[\pub]$. This is consistent with $\Gm_0$ and $\Gm_1$. If $b=1$ in $\ngameTI{\GenroSp{\functionInSet},\GenroSp{\functionOutSet},\QuTInv,\AnTInv}$, then $\advB$ perfectly simulates $\Gm_0$ for $\advA$. If $b=1$, then $\advB$ correctly simulates $\Gm_1$ for $\advA$.
	To complete the proof we claim that  
	\begin{align*}
	\Pr[\Gm_1(\advA)] &= \Pr[\Gm_2(\advA)]  \;.
	\end{align*}
	This is true by the correctness condition. The latter says that if $\functionIn \gets \mathrm{P}[\aFunc{\functionOut}_0]_{\QuTInv,\AnTInv}$ then $\construct{F}[\aFunc{\functionIn}]$ is just $\functionOut_0$ itself. So $\functionOut_1$ in game $\Gm_1$ is the same as $\functionOut_0$ in game $\Gm_2$, making their $\priv$ oracles identical. And their $\pub$ oracles are identical by definition. \qed
\end{proof}

\medskip

\begin{figure}[t]
	\twoColsNoDivide{0.3}{0.3}{
		\ExperimentHeader{$\ngamePRF{\prf,\GenroSp{\functionInSet},\GenroSp{\functionOutSet}}$}
		
		\begin{oracle}{$\Initialize()$}
			\item $b \getsr \bits$
			\item $\functionOut \getsr \GenroSp{\functionOutSet}$
			\item $\commoncoins \getsr \bits^{\prfKl}$
			\item $\functionIn_1 \gets \prf[\functionOut](\commoncoins,\cdot)$ 
			\item $\functionIn_0 \getsr \GenroSp{\functionInSet}$
		\end{oracle}
	}{
		\begin{oracle}{$\RO(W)$}
			\item Return $\functionOut(W)$
		\end{oracle}
		\ExptSepSpace
		\begin{oracle}{$\FnO(U)$}
			\item $V \gets \functionIn_b(U)$ 
			\item Return $V$
		\end{oracle}
		\ExptSepSpace
		\begin{oracle}{$\Finalize(b')$}
			\item Return $(b' = b)$
		\end{oracle}
	}
	
	\caption{Game to define PRF security of $(\GenroSp{\functionInSet},\GenroSp{\functionOutSet})$-oracle aided PRF $\prf$.}\label{fig-prf}
	\hrulefill
\end{figure}


\noindent The simulator in Theorem~\ref{th-cc-indiff-invertible} is stateless, so when $\workDom$ is chosen to be $\GenroSpDom{\functionOutSet}$ the theorem is establishing reset indifferentiability~\cite{EC:RisShaShr11} of $\construct{F}$. 

For translating functors where $\QuTInv$ does not have full support, we need an auxiliary primitive that we call a $(\GenroSp{\functionInSet},\GenroSp{\functionOutSet})$-oracle aided PRF. Given an oracle for a function $\functionOut\in\GenroSp{\functionOutSet}$, an $(\GenroSp{\functionInSet},\GenroSp{\functionOutSet})$-oracle aided PRF $\prf$ defines a function $\prf[\functionOut]\Colon\bits^{\prfKl}\cross\GenroSpDom{\functionInSet}\to\GenroSpRng{\functionInSet}$. The first input is a key. For $\advC$ an adversary, let $\prfAdv{\prf,\GenroSp{\functionInSet},\GenroSp{\functionOutSet}}{\advC} = 2\Pr[\ngamePRF{\prf,\GenroSp{\functionInSet},\GenroSp{\functionOutSet}}(\advC)]-1$, where the game is in Figure~\ref{fig-prf}. The simulator uses its read-only state to store a key $\commoncoins$ for $\prf$, then using $\prf(\commoncoins,\cdot)$ to answer queries outside the support $\supportQuT{\QuTInv}$. 

We introduce this primitive because it allows multiple instantiations. The simplest is that it is a PRF, which happens when it does not use its oracle. In that case the simulator is using a computational primitive (a PRF) in the indifferentiability context, which seems novel. Another instantiation prefixes $\commoncoins$ to the input and then invokes $e$ to return the output. This works for certain choices of $\GenroSp{\functionOutSet}$, but not always. Note $\prf$ is used only by the simulator and plays no role in the functor. 








% We will use this prefix as the key of a pseudorandom function $\prf \Colon \bits^{\prfKl} \times \GenroSpDom{\functionInSet}\to \GenroSpRng{\functionOutSet}$, which may query the random oracle $\aFunc{\functionOut}_0$.







% We will also need some assumptions on the function spaces and the query translator, for which we give some definitions.



% This is now assumed already in the definitions, no special name needed.
%Let $(\QuT,\AnT)$ be invertible with inverses $\QuTInv$ and $\AnTInv$. We say that $\QuTInv$ is a full inverse of $\QuT$ if $\QuTInv$ never returns $()$ on the image of $\QuT$; that is, if $\QuTInv$ never fails to invert a query in $\GenroSpDom{\functionInSet}$ that could be output by $\QuT$. 


\begin{figure}[t]
	\twoCols{0.44}{0.42}{
		\ExperimentHeader{Games $\Gm_0$, $\Gm_1$} 
		
		\begin{oracle}{$\Initialize$}
			\item $\aFunc{\functionIn}_1 \getsr \GenroSp{\functionInSet}$ 
			\item $\aFunc{\functionIn}_2 \getsr \GenroSp{\functionInSet}$ \Comment{Game $\Gm_1$}
			\item $\aFunc{\functionOut}_1 \gets \construct{F}[\aFunc{\functionIn}_1]$
			
		\end{oracle}
		
		\ExptSepSpace
		
		\begin{oracle}{$\priv(W)$}
			\item If $W\in\workDom$ then return $\aFunc{\functionOut}_1(W)$
			\item Else return $\bot$
		\end{oracle}
		\ExptSepSpace
		
		\begin{oracle}{$\pub(U)$}
			\item if $\QuTInv(U) = ()$ then
			\item \hindent return $\aFunc{\functionIn}_2(U)$ \Comment{Game $\Gm_1$}
			\item return $\aFunc{\functionIn}_1(U)$
		\end{oracle}
		
		\ExptSepSpace
		
		\begin{oracle}{$\Finalize(b')$}
			\item return $(b' = 1)$
		\end{oracle}
	}
	{
		\ExperimentHeader{Game $\Gm_2,\Gm_3$} 
		
		\begin{oracle}{$\Initialize$}
			\item $\aFunc{\functionOut}_0 \getsr \GenroSp{\functionOutSet}$ 
			\item  $\functionIn_1 \gets \mathrm{P}[\aFunc{\functionOut}_0]_{\QuTInv,\AnTInv}$ 
			\item $\aFunc{\functionIn}_2 \getsr \GenroSp{\functionInSet}$ \ccomment{Game $\Gm_2$}
			\item $\functionOut_1 \gets \construct{F}[\aFunc{\functionIn}_1]$
			% \item $\aFunc{\functionOut}_1 \gets \construct{F}[\aFunc{\functionIn}]$ 	
			\item $\commoncoins \getsr \SimgenCC()$ \ccomment{Game $\Gm_3$}
		\end{oracle}
		
		\ExptSepSpace
		
		
		\begin{oracle}{$\priv(W)$}
			\item If $W\in\workDom$ then return $\aFunc{\functionOut}_1(W)$
			\item Else return $\bot$ 
		\end{oracle}
		\ExptSepSpace
		
		\begin{oracle}{$\pub(U)$}
			\item if $\QuTInv(U) = ()$ then
			\item \hindent return $\aFunc{\functionIn}_2(U)$ \ccomment{Game $\Gm_2$}
			\item \hindent return $\prf_{\commoncoins}[\functionOut_0](U)$ \ccomment{Game $\Gm_3$}
			\item return $\aFunc{\functionIn}_1(U)$
		\end{oracle}
		
		\ExptSepSpace
		
		\begin{oracle}{$\Finalize(b')$}
			\item return $(b' = 1)$
		\end{oracle}
	}
	\twoColsNoDivide{0.44}{0.42}
	{
		\ExperimentHeader{Game $\Gm_4$} 
		
		\begin{oracle}{$\Initialize$}
			\item $\aFunc{\functionOut}_0 \getsr \GenroSp{\functionOutSet}$ 
			\item  $\functionIn_1 \gets \mathrm{P}[\aFunc{\functionOut}_0]_{\QuTInv,\AnTInv}$
			% \item $\aFunc{\functionOut}_1 \gets \construct{F}[\aFunc{\functionIn}]$ 	
			\item $\commoncoins \getsr \SimgenCC()$
		\end{oracle}
		
		\ExptSepSpace
		
		\begin{oracle}{$\priv(W)$}
			\item If $W\in\workDom$ then return $\functionOut_0(W)$
			\item Else return $\bot$ 
		\end{oracle}
	}{
		
		\begin{oracle}{$\pub(U)$}
			\item if $\QuTInv(U) = ()$ then
			\item \hindent return $\prf[\functionOut_0]_{\commoncoins}(U)$
			\item return $\aFunc{\functionIn}_1(U)$
		\end{oracle}
		
		\ExptSepSpace
		
		\begin{oracle}{$\Finalize(b')$}
			\item return $(b' = 1)$
		\end{oracle}
	}
	
	\caption{Games for proof of Theorem~\ref{th-cc-indiff-invertible-gen}.}
	\label{fig:th-cc-indiff-invertible-gen-games}
	\hrulefill
\end{figure}

\begin{figure}[t]
	\twoCols{0.44}{0.42}{
		\begin{algorithm-initial}{Adversary $\advB$}
			\item $\Initialize()$
			\item $\advA[\Init', \pub', \priv', \Fin']()$
		\end{algorithm-initial} 
		
		\ExptSepSpace
		\begin{algorithm-subsequent}{$\Init'$}
			\item Return
		\end{algorithm-subsequent}
		\ExptSepSpace
		\begin{algorithm-subsequent}{$\pub'(U)$}
			\item if $T[U] \neq \bot$ then return $T[U]$
			\item $W \gets \pub(U)$
			\item if $W = \bot$ then
			\item \hindent $(i,X) \gets U$
			\item \hindent $T[U] \getsr \FSOutSet(U)$
			\item \hindent $W \gets T[U]$
			\item return $W$
		\end{algorithm-subsequent}
		\ExptSepSpace
		\begin{algorithm-subsequent}{$\priv'(W)$}
			\item if $W \in \workDom$ then return $\construct{F}[\pub](W)$
			\item Else return $\bot$
		\end{algorithm-subsequent}
		\ExptSepSpace
		\begin{algorithm-subsequent}{$\Fin'(b')$}
			\item $\Finalize(b')$
		\end{algorithm-subsequent}
	}{	
		\begin{algorithm-initial}{Adversary $\advC$}
			\item $\Initialize()$
			\item $\advA[\Init', \pub', \priv', \Fin']()$
		\end{algorithm-initial} 
		\ExptSepSpace
		\begin{algorithm-subsequent}{$\Init'$}
			\item Return
		\end{algorithm-subsequent}
		\ExptSepSpace
		\begin{algorithm-subsequent}{$\pub'(U)$}
			\item if $\QuTInv(U) = ()$ then
			\item \hindent return $\FnO(U)$
			\item return $\mathrm{P}[\RO]_{\QuTInv,\AnTInv}(U)$
		\end{algorithm-subsequent}
		\ExptSepSpace
		\begin{algorithm-subsequent}{$\priv'(W)$}
			\item If $W \in \workDom$ then 
			\item \hindent return $\construct{F}[\pub'](W)$
			\item Else return $\bot$
		\end{algorithm-subsequent}
		\ExptSepSpace
		\begin{algorithm-subsequent}{$\Fin'(b')$}
			\item $\Finalize(b')$
		\end{algorithm-subsequent}
	}
	\caption{Adversaries for proof of Theorem~\ref{th-cc-indiff-invertible-gen}.}
	\label{fig:th-cc-indiff-invertible-gen-adversaries}
	\hrulefill
\end{figure}

% Let $\domain,\rangeSet$ be non-empty, finite sets. Recall that a function space $\GenroSp{H} \subseteq \AllFuncs{\domain}{\rangeSet}$ is $q$-wise independent if for all distinct $U_1,\ldots,U_q\in\domain$ and all $V_1,\ldots,V_q\in\rangeSet$ we have
%\begin{align*}
%\Pr[\aFunc{h}(U_1)=V_1\mbox{ and }\cdots\mbox{ and }\aFunc{h}(U_q)=V_q] &= %\frac{1}{|\rangeSet|^q} \;,
%\end{align*}
%where the probability is over $\aFunc{h} \getsr \GenroSp{H}$.

\begin{theorem}\label{th-cc-indiff-invertible-gen}  Let $\GenroSp{\functionInSet}$ and $\GenroSp{\functionOutSet}$ be function spaces, and assume they provide input independence. Let $\workDom$ be a subset of $\GenroSpDom{\functionOutSet}$.
	Let $\QuT,\AnT$ be $(\GenroSp{\functionInSet},\GenroSp{\functionOutSet})$ query and answer translators, respectively. Let $\QuTInv,\AnTInv$ be inverses of $\QuT,\AnT$ over~$\workDom$.
	% This is now the default
	% , and assume that $\QuTInv$ is a full inverse of $\QuT$.
	Define read-only simulator $\simulator$ as per the bottom panel of Figure~\ref{fig-th-cc-indiff-invertible}.
	Let $\construct{F} = \construct{TF}_{\QuT,\AnT}$. Let $\advA$ be any distinguisher.  Then we construct a ti-adversary $\advB$ and a prf-adversary $\advC$ such that
	\begin{align*}
	\AdvCCINDIFF{\construct{F},\GenroSp{\functionInSet},\GenroSp{\functionOutSet},\workDom,\simulator}{\advA}  & \leq \AdvTI{\GenroSp{\functionInSet},\GenroSp{\functionOutSet},\QuTInv,\AnTInv}{\advB} +\prfAdv{\prf,\GenroSp{\functionInSet}}{\advC} \;.
	\end{align*}
	Let $\ell$ be the maximum output length of $\QuT$ and $\ell'$ the maximum output length of $\QuTInv$.
	If $\advA$ makes $q_{\priv},q_{\pub}$ queries to its $\priv,\pub$ oracles, respectively, 
	then $\advB$ makes $\ell\cdot q_{\priv}+q_{\pub}$ queries to its $\pub$ oracle 
	and $\advC$ makes at most $\ell\cdot \ell'\cdot q_{\priv}+ q_{\pub}$ queries to its $\RO$ oracle 
	and at most $q_{\pub}+\ell\cdot q_{\priv}$ queries to its $\FnO$ oracle. The running times of $\advB,\advC$ are about that of $\advA$.
	\end{theorem}

	\begin{proof}[Theorem~\ref{th-cc-indiff-invertible-gen}] We will rely on the sequence of games in Figure~\ref{fig:th-cc-indiff-invertible-gen-games}. The first game $G_0$ is the real game, meaning the case $b=1$ in game $\ngameCCINDIFF{\construct{F},\GenroSp{\functionInSet},\GenroSp{\functionOutSet},\workDom,\simulator}$. 
	Game $\Gm_1$ differs from $\Gm_0$ because it samples an additional function $\aFunc{\functionIn}_2$ from the starting space. When an inversion error occurs in the $\pub$ oracle, game $\Gm_1$ answers using $\aFunc{\functionIn}_2$ instead of $\aFunc{\functionIn}_1$. Since the starting space $\GenroSp{\functionInSet}$ provides input independence, both $\aFunc{\functionIn}_1$ and $\aFunc{\functionIn}_2$ are drawn from $\AllFuncs{\GenroSpDom{\functionInSet}}{\FSOutSet}$ for some $\FSOutSet$. Then on any input $U$, the outputs of $\aFunc{\functionIn}_1$ and $\aFunc{\functionIn}_2$ are identically and independently distributed. The adversary can therefore only tell that queries outside the support of $\QuTInv$ are not being answered by $\aFunc{\functionIn}_1$ if the $\pub$ oracle becomes inconsistent with the $\priv$ oracle. This happens only if the $\priv$ oracle, while computing $\construct{F}[\aFunc{\functionIn}_1] = \construct{TF}_{\QuT,\AnT}[\aFunc{\functionIn}_1]$, queries $\aFunc{\functionIn}_1$ on some point outside the support of $\QuTInv$, which is impossible by the first condition in the definition of invertibility.  Hence
	\[\Pr[\Gm_0(\advA)] = \Pr[\Gm_1(\advA)].\]
	Between games $\Gm_1$ and $\Gm_2$, we draw a function $\functionOut_0$ from the ending space and replace $\aFunc{\functionIn}_1$ with $\mathrm{P}_{\QuTInv,\AnTInv}[\functionOut_0]$.
	We construct the translation-indistinguishability adversary $\advB$ in Figure~\ref{fig:th-cc-indiff-invertible-gen-games} so that 
	\[ 	\Pr[\Gm_1(\advA)]-\Pr[\Gm_2(\advA)] \leq \genAdv{ti}{\GenroSp{\functionInSet},\GenroSp{\functionOutSet},\QuTInv,\AnTInv}{\advB}. \]
	This adversary simulates the interface of $\Gm_1$ and $\Gm_2$ for $\advA$, using its $\pub$ oracle to implement $\aFunc{\functionIn}_1$ and check for inversion errors. It lazily samples $\aFunc{\functionIn}_2$, which is consistent with $\Gm_1$ and $\Gm_2$ by the input independence of $\GenroSp{\functionInSet}$. Its $\priv'$ oracle runs $\construct{F}[\pub]$, which is consistent. 
	When the challenge bit $b = 1$ in game $\nGame{ti}{\GenroSp{\functionInSet},\GenroSp{\functionOutSet},\QuTInv,\AnTInv}$, adversary $\advB$ simulates game $\Gm_1$ perfectly, and when $b=0$ it perfectly simulates game $\Gm_2$.
	
	In game $\Gm_3$, we replace $\aFunc{\functionIn}_2$ with an $(\GenroSp{\functionInSet},\GenroSp{\functionOutSet})$-oracle-aided pseudorandom function $\prf$ and sample a PRF key $\commoncoins$ in the $\Initialize$ oracle. We construct an adversary $\advC$ in Figure~\ref{fig:th-cc-indiff-invertible-gen-games} against the PRF-security of $\prf$. This adversary plays game $\nGame{prf}{\GenroSp{\functionInSet},\GenroSp{\functionOutSet},\prf}$ and simulates the interface of games $\Gm_2$ and $\Gm_3$ for $\advA$. It uses its $\RO$ oracle to simulate $\aFunc{\functionOut}_0$, and it uses its $\FnO$ oracle to answer $\pub$ queries outside the support of $\QuTInv$. When $b=0$ in game $\nGame{prf}{\GenroSp{\functionInSet},\GenroSp{\functionOutSet},\prf}$, the adversary perfectly simulates $\Gm_2$ for $\advA$, and when $b=1$ it perfectly simulates $\Gm_3$.
	Therefore
	\[\Pr[\Gm_2(\advA)]-\Pr[\Gm_3(\advA)] \leq \genAdv{prf}{\GenroSp{\functionInSet},\GenroSp{\functionOutSet},\prf}{\advC}.\]
	In Game $\Gm_4$, we answer $\priv$ queries with $\functionOut_0$ directly, instead of with $\construct{F}[\mathrm{P}_{\QuTInv,\AnTInv}[\functionOut_0]]$. By the correctness condition of invertibility, these two functions are identical, so
	\[\Pr[\Gm_3(\advA)] = \Pr[\Gm_4(\advA)].\]
	
	Looking at the pseudocode for simulator $\simulator$ in the bottom panel of Figure~\ref{fig:cc-indiff-sims}, we see that $\Simeval[\functionOut]$ first runs $\QuTInv$ on its input $U$. If $\QuTInv(U)=()$, then it returns $\prf_{\commoncoins}[\functionOut](U)$. Otherwise, it runs $\mathrm{P}[\functionOut]_{\QuTInv,\AnTInv}(U)$ and returns the output. This is identical to lines 6-8 of game $\Gm_4$, so $\advA$ wins $\Gm_4$ if and only if it loses the ideal game (meaning the case $b=0$), of the rd-indiff game $\nGame{rd-indiff}{\construct{F},\GenroSp{\functionInSet},\GenroSp{\functionOutSet},\workDom,\simulator}$. Thus
	\begin{align*} \advRINDIFF{\construct{F},\GenroSp{\functionInSet},\GenroSp{\functionOutSet},\workDom,\simulator}{\advA} &= \Pr[\Gm_0(\advA)]-\Pr[\Gm_4(\advA) \\
	&= \Pr[\Gm_1(\advA)] - \Pr[\Gm_3(\advA)]\\
	&= (\Pr[\Gm_1(\advA)] - \Pr[\Gm_2(\advA)] )+( \Pr[\Gm_2(\advA)] - \Pr[\Gm_3(\advA))]\\
	&\leq \genAdv{ti}{\GenroSp{\functionInSet},\GenroSp{\functionOutSet},\QuTInv,\AnTInv}{\advB} + \genAdv{prf}{\GenroSp{\functionInSet},\GenroSp{\functionOutSet},\prf}{\advC} .
	\end{align*}
	This completes the proof. \qed
\end{proof}
\section{Analysis of cloning functors}
\label{sec-framework}\label{sec-domain-separation}

Section~\ref{sec-cc-indiff} defined the rd-indiff metric of security for functors and give a framework to prove rd-indiff of translating functors. We now apply this to derive security results about particular, practical cloning functors. 



\heading{Arity-$n$ function spaces.} The cloning functors apply to function spaces where a function specifies sub-functions, corresponding to the different random oracles we are trying to build. 
%In practice the sub-functions tend to be either sol or xol (see Section~\ref{sec-prelims} for definitions). The latter can however be captured (although with some impact on efficiency) via sol functions with some large, maximum output length. This is done by seeing the xol function as part of the functor, in the sense that an xol call $x,\ell$ can be rendered as the functor calling the sol function on $x$ and then using the $\ell$-bit prefix of the response. Accordingly, we restrict attention to sol when considering cloning functors. 
%
%\mb{The above actually gives a reduction that allows us to capture methods using xol functions in our framework of sol functions, by how we define the functor. It may be worth formalizing this and flushing it out more. We could do it for some NIST example like NewHope.}
Formally, a function space $\roSp$ is said to have arity $n$ if its members are two-argument functions $f$ whose first argument is an integer $i \in \allowbreak [1..n]$. For $i\in \allowbreak [1..n]$ we let $f_i = f(i,\cdot)$ and $\roSp_i \allowbreak = \allowbreak \set{f_i}{f\in\roSp}$, and refer to the latter as the $i$-th subspace of $\roSp$. We let $\Dom_i(\roSp)$ be the set of all $X$ such that $(i,X) \allowbreak \in \allowbreak \GenroSpDom{\roSp}$. 

We say that $\roSp$ has sol subspaces if $\roSp_i$ is a set of sol functions with domain $\Dom_i(\roSp)$, for all $i \allowbreak \in \allowbreak [1..n]$. More precisely, there must be integers $\OL_1(\roSp), \allowbreak \ldots, \allowbreak \OL_n(\roSp)$ such that $\roSp_i \allowbreak = \allowbreak \AllSOLFuncs{\Dom_i(\roSp)}{\OL_i(\roSp)}$ for all $i\in [1..n]$. In this case, we let $\Rng_i(\roSp) \allowbreak = \allowbreak \bits^{\OL_i(\roSp)}$. This is the most common case for practical uses of ROs.


%The second choice is that it is the set of all xol functions having some domain, again denoted $\Dom_i(\roSp)$, meaning $\roSp_i =  \AllXOLFuncs{\Dom_i(\roSp)}$. In this case we let $\Rng_i(\roSp) = \bits^{*}$ and refer to $\roSp_i$ as an xol subspace. 
%We can now define the domain of $\roSp$ as $\GenroSpDom{\roSp} \allowbreak = \allowbreak \set{(i,X)}{i\in [1..n]\allowbreak  \mbox{ and }\allowbreak  X \allowbreak \in \allowbreak  \Dom_i(\roSp)}$, and the range of $\roSp$ as $\GenroSpRng{\roSp} \allowbreak = \allowbreak \Rng_1(\roSp)\cup\cdots\cup\Rng_n(\roSp)$.



%We say that $\roSp$ is a function space if $\roSp = \roSp_1\cross\cdots\cross\roSp_n$ for some sets $\roSp_1,\ldots,\roSp_n$ (called the subspaces) having the property that, for each $i\in [1..n]$ there is a set $\Dom_i(\roSp)$ (called the domain of $\roSp_i$) such that one of the following holds:
%\begin{newitemize}
%	\item $\roSp_i =  \AllSOLFuncs{\Dom_i(\roSp)}{\ell_i}$ for some $\ell_i$, meaning $\roSp_i$ is a set of sol functions. In this case, let $\Rng_i(\roSp) = \bits^{\ell_i}$.
%	\item $\roSp_i =  \AllXOLFuncs{\Dom_i(\roSp)}$, meaning $\roSp_i$ is a set of xol functions. In this case, let $\Rng_i(\roSp) = \bits^{*}$.
%\end{newitemize}
%We identify a tuple of functions $(f_1,\ldots,f_n)\allowbreak  \in \allowbreak  \roSp$ with the single, two-argument function $f$ defined by $f(i,X)=f_i(X)$ for all $i\in [1..n]$ and all $X\allowbreak  \in \allowbreak  \Dom_i(\roSp)$. The domain of $\roSp$ is accordingly defined as $\GenroSpDom{\roSp} \allowbreak = \allowbreak \set{(i,X)}{i\in [1..n]\allowbreak  \mbox{ and }\allowbreak  X \allowbreak \in \allowbreak  \Dom_i(\roSp)}$. The range of $\roSp$ is $\GenroSpRng{\roSp} \allowbreak = \allowbreak \Rng_1(\roSp)\cup\cdots\cup\Rng_n(\roSp)$. 

To explain, access to $n$ random oracles is modeled as access to a two-argument function $f$ drawn at random from $\roSp$, written $f \getsr\roSp$. If $\roSp$ has sol subspaces, then for each $i$, the function $f_i$ is a sol function, with a certain domain and output length depending only on $i$.
% or an xol function (with a certain domain depending only on $i$).  In either case, 
All such functions are included. This ensures input independence as we defined it earlier. Thus if $f \getsr\roSp$, then for each $i$ and any distinct inputs to $f_i$, the outputs are independently distributed. Also functions $f_1,\ldots,f_n$ are independently distributed when $f \getsr\roSp$. Put another way, we can identify $\roSp$ with $\roSp_1\cross\cdots\cross\roSp_n$. 

% We say that $\roSp$ is a sol (respectively, xol) function space if $\roSp_1,\ldots,\roSp_n$ are all sets of sol (respectively, xol) functions, meaning the first (respectively, second) case above holds for all $i$. Note that $\roSp$ may be neither sol nor xol, which happens when some subspaces are sol and others are xol. When $n=1$ we may write $f(X)$ in place of $f_1(X)=f(1,X)$, identifying $\GenroSpDom{\roSp}$ with $\Dom_1(\roSp)$. 




\heading{Domain-separating functors.} We can now formalize the domain separation method by seeing it as defining a certain type of (translating) functor. 

Let the ending space $\GenroSp{\functionOutSet}$ be an arity $n$ function space. Let $\construct{F}  \Colon \GenroSp{\functionInSet}\to \GenroSp{\functionOutSet}$ be a translating functor and $\QuT,\AnT$ be its query and answer translations, respectively. Assume $\QuT$ returns a vector of length~1 and that $\AnT((i,X),\vecV)$ simply returns $\vecV[1]$. We say that $\construct{F}$ is \textit{domain separating} if the following is true: $\QuT(i_1,X_1)\neq \QuT(i_2,X_2)$ for any $(i_1,X_1),(i_2,X_2) \in \GenroSpDom{\functionOutSet}$ that satisfy $i_1\neq i_2$. 

To explain, recall that the ending function is obtained as $\functionOut \gets \construct{F}[\functionIn]$, and defines $\functionOut_i$ for $i\in [1..n]$. Function $\functionOut_i$ takes input $X$, lets $(u)\gets\QuT(i,X)$ and returns $\functionIn(u)$. The domain separation requirement is that if $(u_i)\gets\QuT(i,X_i)$ and $(u_j)\gets\QuT(j,X_j)$, then $i\neq j$ implies $u_i\neq u_j$, regardless of $X_i,X_j$. Thus if $i\neq j$ then the inputs to which $\functionIn$ is applied are always different. The domain of $\functionIn$ has been ``separated'' into disjoint subsets, one for each $i$. 

% \label{sec-ds-methods}
\heading{Practical cloning functors.}
We show that many popular methods for oracle cloning in practice, including ones used in NIST KEM submissions, can be cast as translating functors. 


%\figref{fig:prac-domsep} shows the pseudocode for their query and answer translators and notes in which NIST submissions they appear.
% We then discuss each approach individually, explaining its parameters, its advantages and disadvantages, and the intuition justifying its security.

In the following, the starting space $\GenroSp{\functionInSet} = \AllSOLFuncs{\bits^*}{\OL(\GenroSp{\functionInSet})} $ is assumed to be a sol function space with domain $\bits^*$ and an output length denoted $\OL(\GenroSp{\functionInSet})$. The ending space $\GenroSp{\functionOutSet}$ is an arity $n$ function spaces that has sol subspaces. 

%If $\functionIn \in \GenroSp{\functionInSet}$ we write $\functionIn(X)$ in place of $\functionIn(1,X)$ or $\functionIn_1(X)$. 

%In this discussion, we will assume that the goal is to build $n > 1$ independent random oracles from a single random oracle.
%Formally, we consider the ``target'' function space $\FuncSp{\functionOutSet}$ having arity $\GenroSpCardinality{\functionOutSet}= n> 1$ and domain $\GenroSpDom{\functionOutSet} = [1..n] \times \bits^*$,
%and the ``starting'' function space $\FuncSp{\functionInSet}$ having arity $1$ and domain $\bits^*$.
%We will further assume that $\GenroSpRng{\functionOutSet} = \GenroSpRng{\functionInSet}$ and that this range is finite. 

%\begin{figure}[ht]
%	\centering
%	
%	\begin{tabular}{|p{2cm}||p{2.9cm}|p{2.9cm}|p{3.8cm}|}
%		\hline
%		Method		& \textbf{Prefixing}	& \textbf{Length\newline differentiating}	& \textbf{Output splitting} \\ \hline \hline
%		
%		Requirements	&
%			vector $\pfvec$ of distinct prefixes
%			&
%			distinct $|x|$ per~$i$
%			&
%			fixed length-$l$ output
%			\\ \hline
%		
%		Query\newline Translator
%			&
%			\begin{algorithm}{$\FixedprefixqueryRO{\pfvec}(i,x)$}	
%				\item Return $(\pfvec[i]\concat x)$ 
%			\end{algorithm}
%			&
%			\begin{algorithm}{$\LengthqueryRO(i,x)$}
%				\item Return $(x)$
%			\end{algorithm}
%			&
%			\begin{algorithm}{$\SplittingqueryRO(i,x)$}
%				\item Return $(x)$
%			\end{algorithm}
%			\\ \hline
%		
%		Answer\newline Translator
%			&
%			\begin{algorithm}{$\FixedprefixanswerRO(s,y)$}
%				\item Return $y$
%			\end{algorithm}
%			&
%			\begin{algorithm}{$\LengthanswerRO(s,y)$}
%				\item Return $y$
%			\end{algorithm}
%			&
%			\begin{algorithm}{$\SplittinganswerRO(s,y)$}
%				\item  $(i,x) \gets  s$
%				\item Return $y[(i-1)l+1..il]$
%			\end{algorithm}
%			\\ \hline
%		
%		Applied in\newline NIST KEMs
%			&
%			\pqcnameRoundTwo{ClassicMcEliece}, \pqcnameRoundTwo{Frodo}, \pqcnameRoundOne{LIMA}, \pqcnameRoundTwo{NTRU Prime}, \pqcnameRoundOne{SIKE}, \pqcnameRoundOne{QC-MDPC},  \pqcnameRoundTwo{ThreeBears}
%			&
%			\pqcnameRoundOne{EMBLEM}, \pqcnameRoundTwo{HQC}, \pqcnameRoundTwo{RQC}, \pqcnameRoundTwo{LAC}, \pqcnameRoundOne{LOCKER}, \pqcnameRoundTwo{NTS-KEM}, \pqcnameRoundTwo{SABER},\hspace{2cm}\pqcnameRoundOne{Round2}, \pqcnameRoundTwo{Round5}
%			&
%			\pqcnameRoundTwo{FrodoKEM}, \pqcnameRoundOne{NTRU-HRSS-KEM}, \pqcnameRoundOne{QC-MDPC}, \pqcnameRoundOne{Round2}, \pqcnameRoundTwo{Round5}
%			\\ \hline
%	\end{tabular}
%	
%	\medskip
%	
%	\caption{%
%		Query and answer translators for practical oracle cloning methods, and the NIST KEMs that employ them.
%		(Round 1 submissions are in \pqcnameRoundOne{gray}, and Round 2 submissions are in \pqcnameRoundTwo{black}.)
%	}
%	\label{fig:prac-domsep}
%\end{figure}
%We have shown that invertible translating oracle constructions can achieve the goal of constructing multiple random oracles from a single random oracle.
%Here we show that many popular methods of domain separating hash functions which are used to clone random oracles are in fact invertible translating constructions, justifying their effectiveness.

%In this discussion, we will assume that the goal is to build $n>1$ independent random oracles from a single random oracle. Formally, let $\GenroSp{\functionInSet}$ be a function space with arity $n$ and let $\GenroSp{\functionOutSet}$ be a funciton space with arity one.
%We also assume, in order to stay close to the specifications of current cryptographic hash functions, that $\GenroSpDom{\functionInSet} = [1..n]\times \bits^*$, and $\GenroSpDom{\functionOutSet} = \bits^*$, meaning that the domain of all random oracles drawn from these function spaces will be $\bits^*$. 


\heading{Prefixing.}
Here we formalize the canonical method of domain separation.
Prefixing is used in the following NIST PQC submissions: \pqcnameRoundTwo{ClassicMcEliece}, \pqcnameRoundTwo{FrodoKEM}, \pqcnameRoundOne{LIMA}, \pqcnameRoundTwo{NTRU Prime}, \pqcnameRoundOne{SIKE}, \pqcnameRoundOne{QC-MDPC},  \pqcnameRoundTwo{ThreeBears}.

Let $\pfvec$ be a vector of strings. We require that it be \textit{prefix-free}, by which we mean that $i\neq j$ implies that $\pfvec[i]$ is not a prefix of $\pfvec[j]$. Entries of this vector will be used as prefixes to enforce domain separation. One example is that the entries of $\pfvec$ are distinct strings all of the same length. Another is that a $\pfvec[i]=\mathrm{E}(i)$ for some prefix-free code $\mathrm{E}$ like a Huffman code. 

Assume $\OL_i(\GenroSp{\functionOutSet})=\OL(\GenroSp{\functionInSet})$ for all $i\in [1..n]$, meaning all ending functions have the same output length as the starting function. % Assume $\GenroSpDom{\functionInSet}=\bits^*$. 
The functor $\pfFunctor{\pfvec} \Colon \GenroSp{\functionInSet} \to \GenroSp{\functionOutSet}$ corresponding to $\pfvec$ is defined by $\pfFunctor{\pfvec}[\functionIn](i\cab X) \allowbreak = \allowbreak \functionIn(\pfvec[i]\| X)$. To explain, recall that the ending function is obtained as $\functionOut \gets \pfFunctor{\pfvec}[\functionIn]$, and defines $\functionOut_i$ for $i\in [1..n]$. Function $\functionOut_i$ takes input $X$, prefixes $\pfvec[i]$ to $X$ to get a string $X'$, applies the starting function $\functionIn$ to $X'$ to get $Y$, and returns $Y$ as the value of $\functionOut_i(X)$. 

We claim that $\pfFunctor{\pfvec}$ is a translating functor that is also a domain-separating functor as per the definitions above. To see this, define query translator~$\FixedprefixqueryRO{\pfvec}$ by $\FixedprefixqueryRO{\pfvec}(i,X)= (\pfvec[i]\| X)$, the 1-vector whose sole entry is $\pfvec[i]\| X$. The answer translator $\FixedprefixanswerRO{\pfvec}$, on input $(i,X),\vecV$, returns $\vecV[1]$,  
% if $\vecV\neq ()$ ---
meaning it ignores $i,X$ and returns the sole entry in its 1-vector $\vecV$.
% --- and otherwise ---if $\vecV=()$--- returns the default value $0^{\OL(\GenroSp{\functionInSet})}$. 

% If $\GenroSp{\functionInSet}$ is a xol function space then we define the functor $\pfFunctor{\pfvec} \Colon \GenroSp{\functionInSet} \to \GenroSp{\functionOutSet}$ by $\pfFunctor{\pfvec}[\functionIn](i,(x,\ell)) = \functionIn(\pfvec[i]\| X)$.

%\begin{theorem}\label{thm-prefix} Let $m$, $n$, $l$, and $\ell$ be positive integers, let $\FuncSp{\functionOutSet}$ be a function space with arity $n$, domain $[1..n]\times\bits^*$, and range $\bits^l$, and let $\FuncSp{\functionInSet}$ be a function space with arity $1$, domain $\bits^*$, and range $\bits^l$. Let $\vecX$ be a vector with $n$ entries in $\bits^m$.  Let $\FixedprefixqueryRO_{\vecX}$ and $\answerRO$ and their  inverses $\FixedprefixqueryRO_{\vecX}^{-1}$ and $\answerInv$ be as defined in~\figref{fig:prac-domsep}. Let $q\geq 1$ be an integer and let $\GenroSp{H} \subseteq \AllFuncs{\cup_{i=1}^\ell \bits^i}{\rangeSet}$ be a $q$-wise independent function space.
%	Define read-only simulator $\Gensimulator{q}$ as per the bottom panel of Figure~\ref{fig-th-cc-indiff-invertible}. Let $\construct{F} = startin\construct{TF}_{\FixedprefixqueryRO_{\vecX},\answerRO}$.
%	Let $\advA$ be any distinguisher making no queries to its $\priv$ oracle outside of $\workDom$ and at most $q$ queries to its $\pub$ oracle outside of $\supportQuT(\FixedprefixqueryRO_{\vecX}^{-1})$. Assume that $\ell$ is the maximum length of any query $\advA$ makes to its $\pub$ oracle. Then
%		\[
%	\AdvCCINDIFF{\construct{F},\GenroSp{\functionInSet},\GenroSp{\functionOutSet},\Gensimulator{q}}{\advA} = 0 \;.
%	\]
%\end{theorem}
%\begin{proof} 
%\qed
%\end{proof}

We proceed to the inverses, which are defined as follows:
\begin{center}\begin{tabular}{c|c}
\begin{minipage}{2in}\begin{tabbing}
	123\=123\=\kill
	\underline{Algorithm $\FixedprefixqueryInv{\pfvec}(U)$} \\[2pt]
	$\vecW \gets ()$ \\
	For $i=1,\ldots,n$ do \\
	\> If $\pfvec[i]\prefix U$ then $\pfvec[i]\|X\gets U$ ; $\vecW[1]\gets (i,X)$ \\
	Return $\vecW$
\end{tabbing}\end{minipage}
&
\begin{minipage}{2in}\begin{tabbing}
	123\=123\=\kill
	\underline{Algorithm $\FixedprefixanswerInv{\pfvec}(U,\vecY)$} \\[2pt]
	If $\vecY\neq ()$ then $V\gets \vecY[1]$  \\
	Else $V \gets 0^{\OL(\GenroSp{\functionInSet})}$ \\
	Return $V$ \\
\end{tabbing}\end{minipage}
\end{tabular}\end{center}
The working domain is the full one: $\workDom = \GenroSpDom{\functionOutSet}$. We now verify Equation~(\ref{eq-invertible-def}). Let $\QuT,\QuTInv,\AnT,\AnTInv$ be $\FixedprefixqueryRO{\pfvec},\FixedprefixqueryInv{\pfvec},\FixedprefixanswerRO{\pfvec},\FixedprefixanswerInv{\pfvec}$, respectively. Then for all $W = (i,X) \in \GenroSpDom{\functionOutSet}$, we have:
\begin{align*}
\construct{TF}_{\QuT,\AnT}
	[\mathrm{P}[\aFunc{\functionOut}]_{\QuTInv,\AnTInv}](W) &= \mathrm{P}[\aFunc{\functionOut}]_{\QuTInv,\AnTInv}(\pfvec[i]\|X) \\
&= \AnTInv(\pfvec[i]\|X,(\functionOut(i,X))) \\ 	
&= \aFunc{\functionOut}(i,X) \;.
\end{align*}
We observe that $(\FixedprefixqueryInv{\pfvec},\FixedprefixanswerInv{\pfvec})$ provides perfect translation indistinguishability. Since $\FixedprefixqueryInv{\pfvec}$ does not have full support, we can't use Theorem~\ref{th-cc-indiff-invertible}, but we can conclude rd-indiff via Theorem~\ref{th-cc-indiff-invertible-gen}.

%\mb{Some conclusion or result is needed here with regard to rd-indiff obtained via our Theorems. It need not be a formal Theorem, but perhaps some text summary of what one gets.}

\heading{Identity.} Many NIST PQC submissions simply let $\functionOut_i(X)  = \functionIn(X)$, meaning the ending functions are identical to the starting one. This is captured by the identity functor $\idFunctor
\Colon \GenroSp{\functionInSet} \to \GenroSp{\functionOutSet}$, defined by $\idFunctor[\functionIn](i,X) = \functionIn(X)$. This again assumes $\OL_i(\GenroSp{\functionOutSet})=\OL(\GenroSp{\functionInSet})$ for all $i\in [1..n]$, meaning all ending functions have the same output length as the starting function. This functor is translating, via $\IdqueryRO(i,X)=X$ and $\IdanswerRO((i,X),\vecV)=\vecV[1]$.
%  if $\vecV\neq ()$, and  $0^{\OL(\GenroSp{\functionInSet})}$ otherwise. 
It is however \textit{not}, at least in general, domain separating.

Clearly, this functor is not, in general, rd-indiff. To make secure use of it nonetheless, applications can restrict the inputs to the ending functions to enforce a virtual domain separation, meaning, for $i\neq j$, the schemes never query $\functionOut_i$ and $\functionOut_j$ on the same input. One way to do this is length differentiation. Here, for $i\in [1..n]$, the inputs to which $\functionOut_i$ is applied all have the same length $l_i$, and $l_1,\ldots,l_n$ are distinct. Length differentiation is used in the following NIST PQC submissions: \pqcnameRoundTwo{BIKE},\pqcnameRoundOne{EMBLEM}, \pqcnameRoundTwo{HQC}, \pqcnameRoundTwo{RQC}, \pqcnameRoundTwo{LAC}, \pqcnameRoundOne{LOCKER}, \pqcnameRoundTwo{NTS-KEM}, \pqcnameRoundTwo{SABER}, \pqcnameRoundOne{Round2}, \pqcnameRoundTwo{Round5},\pqcnameRoundOne{Titanium}. There are, of course, many other similar ways to enforce the virtual domain separation.

There are two ways one might capture this with regard to security. One is to restrict the domain $\GenroSpDom{\functionOutSet}$ of the ending space. For example, for length differentiation, we would require that there exist distinct $l_1,\ldots,l_n$ such that for all $(i,X)\in \GenroSpDom{\functionOutSet}$ we have $|X|=l_i$. For such an ending space, the identity functor would provide security. The approach we take is different. We don't restrict the domain of the ending space, but instead define security with respect to a subdomain, which we called the working domain, where the restriction is captured. This, we believe, is better suited for practice, for a few reasons. One is that a single implementation of the ending functions can be used securely in different applications that each have their own working domain. Another is that implementations of the ending functions do not appear to enforce any restrictions, leaving it up to applications to figure out how to securely use the functions. In this context, highlighting the working domain may help application designers think about what is the working domain in their application and make this explicit, which can reduce error. 

But we warn that the identity functor approach is more prone to misuse and in the end more dangerous and brittle than some others. 

As per the above, inverses can only be given for certain working domains. Let us say that $\workDom\subseteq\GenroSpDom{\functionOutSet}$ separates domains if for all $(i_1,X_1),(i_2,X_2)\in\workDom$ satisfying $i_1\neq i_2$, we have $X_1\neq X_2$. Put another way, for any $(i,X)\in \workDom$ there is at most one $j$ such that $X\in \Dom_j(\GenroSp{\functionOutSet})$. We assume an efficient inverter for~$\workDom$. This is a deterministic algorithm $\WDInv_{\workDom}$ that on input $X\in\bits^*$ returns the unique $i$ such that $(i,X)\in\workDom$ if such an $i$ exists, and otherwise returns $\bot$. (The uniqueness is by the assumption that $\workDom$ separates domains.) 

As an example, for length differentiation, we pick some \textit{distinct} integers $l_1,\ldots,l_n$ such that $\bits^{l_i}\subseteq \Dom_i(\GenroSp{\functionOutSet})$ for all $i\in [1..n]$. We then let $\workDom = \set{(i,X)\in\GenroSpDom{\functionOutSet}}{|X|=l_i}$. This separates domains. Now we can define $\WDInv_{\workDom}(X)$ to return the unique $i$ such that $|X| = l_i$ if $|X| \in \{l_1,\ldots,l_n\}$, otherwise returning $\bot$.

The inverses are then defined using $\WDInv_{\workDom}$, as follows, where $U\in \GenroSpDom{\functionInSet}=\bits^*$:
\begin{center}\begin{tabular}{c|c}
\begin{minipage}{2in}\begin{tabbing}
	123\=123\=\kill
	\underline{Algorithm $\IdqueryInv(U)$} \\[2pt]
	$\vecW \gets ()$ ;
	$i\gets \WDInv_{\workDom}(U)$ \\
	If $i\neq\bot$ then $\vecW[1]\gets (i,U)$ \\
	Return $\vecW$
\end{tabbing}\end{minipage}
&
\begin{minipage}{2in}\begin{tabbing}
	123\=123\=\kill
	\underline{Algorithm $\IdanswerInv(U,\vecY)$} \\[2pt]
	If $\vecY\neq ()$ then $V\gets \vecY[1]$  \\
	Else $V \gets 0^{\OL(\GenroSp{\functionInSet})}$ \\
	Return $V$
\end{tabbing}\end{minipage}
\end{tabular}\end{center}
The correctness condition of Equation~(\ref{eq-invertible-def}) over $\workDom$ is met, and since $\WDInv_{\workDom}(X)$ never returns $\bot$ for $X \in \workDom$, the second condition of invertibility is also met. $(\IdqueryInv,\IdanswerInv)$ provides perfect translation indistinguishability. Since $\IdqueryInv$ does not have full support, we can't use Theorem~\ref{th-cc-indiff-invertible}, but we can conclude rd-indiff via Theorem~\ref{th-cc-indiff-invertible-gen}.


%\mb{Are the above claims true?}
%
%\mb{Some conclusion or result is needed here with regard to rd-indiff obtained via our Theorems. It need not be a formal Theorem, but perhaps some text summary of what one gets.}

%\begin{theorem}\label{thm-ldiff} Let $n$, $l$, and $\ell$ be integers. 
%Let $\FuncSp{\functionOutSet}$ be a function space with arity $n$, domain $[1..n]\times\bits^*$, and range $\bits^l$, and let $\FuncSp{\functionInSet}$ be a function space with arity $1$, domain $\bits^*$, and range $\bits^l$. 
%Let $\workDom\subset [1..n]\times\bits^*$ have the property that for all $x\in \bits^*$ and all $i,i'\in [1..n]$, $(i,x) \in \workDom$ and $(i',x) \in \workDom$ only if $i = i'$. 
%Let $\LengthqueryRO_\workDom$ and $\answerRO$ be the query and answer translations described in~\figref{fig:prac-domsep}, and let their inverses $\LengthqueryRO^{-1}_\workDom$ and $\AnTInv$ be as defined in~\figref{fig:prac-domsep}. 
%Let $q\geq 1$ be an integer and let $\GenroSp{H} \subseteq \AllFuncs{\cup_{i=1}^\ell \bits^i}{\rangeSet}$ be a $q$-wise independent function space.
%Define read-only simulator $\Gensimulator{q}$ as per the bottom panel of Figure~\ref{fig-th-cc-indiff-invertible}. Let $\construct{F} = startin\construct{TF}_{\LengthqueryRO_\workDom,\answerRO}$.
%Let $\advA$ be any distinguisher making no queries to its $\priv$ oracle outside of $\workDom$, and, to its $\pub$ oracle, at most $q$ queries which are not in $\supportQuT(\LengthqueryRO_\workDom^{-1})$ and let $\ell$ be the maximum length of query  $\advA$ to $\pub$. Then
%		\[
%	\AdvCCINDIFF{\construct{F},\GenroSp{\functionInSet},\GenroSp{\functionOutSet},\Gensimulator{q}}{\advA} = 0 \;.
%	\]
%\end{theorem}
%\begin{proof}
%
%	\qed
%\end{proof}

\heading{Output-splitting.} We formalize another method that we call output splitting. It is used in the following NIST PQC submissions: \pqcnameRoundTwo{FrodoKEM}, \pqcnameRoundOne{NTRU-HRSS-KEM}, \pqcnameRoundOne{Odd Manhattan},\pqcnameRoundOne{QC-MDPC}, \pqcnameRoundOne{Round2}, \pqcnameRoundTwo{Round5}.

 Let $\ell_i = \OL_1(\GenroSp{\functionOutSet}) + \cdots + \OL_{i}(\GenroSp{\functionOutSet})$ for $i\in [1..n]$. Let $\ell = \OL(\GenroSp{\functionInSet})$ be the output length of the sol functions $\functionIn \in  \GenroSp{\functionInSet}$, and assume $\ell = \ell_n$. The output-splitting functor $\splFunctor \Colon \GenroSp{\functionInSet} \to \GenroSp{\functionOutSet}$ is defined by $\splFunctor[\functionIn](i,X) = \functionIn(X)[\ell_{i-1}\!+\!1 .. \ell_{i}]$. That is, if $\functionOut \gets \splFunctor[\functionIn]$, then $\functionOut_i(X)$ lets $Z \gets \functionIn(X)$ and then returns bits $\ell_{i-1}\!+\!1$ through $\ell_{i}$ of $Z$. This functor is translating, via $\SplittingqueryRO(i,X)=X$ and $\SplittinganswerRO((i,X),\vecV)=\vecV[1][\ell_{i-1}\!+\!1 .. \ell_{i}]$. It is however \textit{not} domain separating. 

The inverses are defined as follows, where $U\in \GenroSpDom{\functionInSet}=\bits^*$:
\begin{center}\begin{tabular}{c|c}
\begin{minipage}{2in}\begin{tabbing}
	123\=123\=\kill
	\underline{Algorithm $\SplittingqueryInv(U)$} \\[2pt]
	For $i=1,\ldots,n$ do $\vecW[i] \gets (i,U)$ \\
	Return $\vecW$
\end{tabbing}\end{minipage}
&
\begin{minipage}{2in}\begin{tabbing}
	123\=123\=\kill
	\underline{Algorithm $\SplittinganswerInv(U,\vecY)$} \\[2pt]
	$V\gets \vecY[1]\|\cdots \|\vecY[n]$  \\
	Return $V$
\end{tabbing}\end{minipage}
\end{tabular}\end{center}
The correctness condition of Equation~(\ref{eq-invertible-def}) over $\workDom = \GenroSp{\functionOutSet}$ is met, and $(\SplittingqueryInv,\SplittinganswerInv)$ provides perfect translation indistinguishability. Since $\SplittingqueryInv$ has full support, we can conclude rd-indiff via Theorem~\ref{th-cc-indiff-invertible}.
%\mb{Are the above claims true?}
%
%\mb{Some conclusion or result is needed here with regard to rd-indiff obtained via our Theorems. It need not be a formal Theorem, but perhaps some text summary of what one gets.}
%\begin{theorem}\label{thm-outspl} Let $n$ and $l$ be integers. 
%Let $\FuncSp{\functionOutSet}$ be the arity-$n$ function space 
%$ \AllFuncs{[1..n]\times\bits^*}{\bits^l}$.
%Let $\workDom = \GenroSpDom{\functionOutSet}$.
%Let $\FuncSp{\functionInSet}$ be the function space $\AllFuncs{\bits^*}{\bits^{nl}}$.
%Let $\SplittingqueryRO$ and $\SplittinganswerRO$ be the query and answer translations described in~\figref{fig:prac-domsep}, and let their inverses $\SplittingqueryInv$ and $\SplittinganswerInv$ be as defined in~\figref{fig:prac-domsep}.
%Let $q\geq 1$ be an integer and let $\GenroSp{H} \subseteq \AllFuncs{\bits^l}{\rangeSet}$ be a $q$-wise independent function space.
%Define read-only simulator $\Gensimulator{q}$ as per the bottom panel of Figure~\ref{fig-th-cc-indiff-invertible}. 
%Let $\construct{F} = \construct{TF}_{\SplittingqueryRO,\SplittinganswerRO}$.
%Let $\advA$ be any distinguisher making at most $q$ queries, to its $\pub$ oracle which are not in $\supportQuT(\SplittingqueryInv)$ and let $\ell$ be the maximum length of query  $\advA$ to $\pub$. Then
%		\[
%	\AdvCCINDIFF{\construct{F},\GenroSp{\functionInSet},\GenroSp{\functionOutSet},\Gensimulator{q}}{\advA} = 0 \;.
%	\]
%\end{theorem}
%\begin{proof}
%	We will show that $(\SplittingqueryRO,\SplittinganswerRO)$ is invertible with partial inverses $\SplittingqueryInv$ and $\SplittinganswerInv$.
%	First consider the construction $\mathrm{P}[\aFunc{\functionOut}]_{\SplittingqueryInv,\SplittinganswerInv}(U)$, which is defined in Section~\ref{sec-cc-indiff}. This function, on an input $U \in \bits^*$ calls $\SplittingqueryInv$, which returns a vector $\vecX$ of length $n$ in which the $i^{\text{th}}$ entry is the query $(i,U)$. Then $\mathrm{P}[\aFunc{\functionOut}]_{\SplittingqueryInv,\SplittinganswerInv}(U)$ queries its oracle $\aFunc{\functionOut}$ on each of the entries of $\vecX$, returning a vector $\vecY$ of length $n$ whose entries are elements of $\bits^l$. It then calls $\SplittinganswerInv(U,\vecY)$, which concatenates these entries into a bitstring of length $n*l$.
%	
%	To satisfy the first condition of invertibility, we must show that for all $U \in \GenroSpDom{\functionOutSet}$ and all $\aFunc{\functionOut} \in \FuncSp{\functionOutSet}$, \[\construct{F}[\mathrm{P}[\aFunc{\functionOut}]_{\SplittingqueryInv,\SplittinganswerInv}](U) = \aFunc{\functionOut}(U).\]
%	We separate $U$ into the ordered pair $(i,x)$. The oracle construction on the left-hand side truncates $i$, and queries its oracle on only $x$. The oracle, as discussed above, calls $\aFunc{\functionOut}(j,x)$ for every $j\in [1..n]$, and concatenates the results into a single $n*l$-bit string $Y$. Then the oracle construction calls $\SplittinganswerRO(U,Y)$, which returns the $i^{\text{th}}$ substring of $l$ bits. Of course, this is simply the result of the function call $\aFunc{\functionOut}(i,x) = \aFunc{\functionOut}(U)$, so the first condition of invertibility holds.
%	
%	The second condition of invertibility requires that $\mathrm{P}[\aFunc{\functionOut}]_{\SplittingqueryInv,\SplittinganswerInv}$ and $\Func{\functionIn}$ are distributed identically when $\aFunc{\functionOut}$ and $\aFunc{\functionIn}$ are drawn from $\FuncSp{\functionOutSet}$ and $\FuncSp{\functionInSet}$, respectively. 
%	Since both function spaces include all possible functions from their domains to their ranges, the outputs of both $\aFunc{\functionOut}$ and $\aFunc{\functionIn}$ on any input are uniformly random bitstrings. Therefore the output of $\aFunc{\functionInSet}$ is a uniformly random $n*l$-bit string, and the output of $\mathrm{P}[\aFunc{\functionOut}]_{\SplittingqueryInv,\SplittinganswerInv}$ is the concatentation of $n$ uniformly random $l$-bit strings, which is itself a uniformly random $n*l$-bit string. Therefore $(\SplittingqueryRO,\SplittinganswerRO)$ is invertible, and the theorem follows from Theorem~\ref{th-cc-indiff-invertible-gen}.
%\end{proof}
%
%\heading{Further methods.}
%In \lncsorfull{Appendix~\ref{apx:intrusive}}, we will discuss a further oracle cloning method which is specific to KEMs. This method, like length differentiation, does not require any change to a scheme's existing hash function calls, but can be secure even when a scheme must query two distinct oracles on the same input.
%We give the query and answer translator partial inverses for each of the concrete oracle cloning methods introduced in Section~\ref{sec-domain-separation}, along with brief justifications of their invertibility.
%\begin{figure}
%	\begin{tabular}{|p{3cm}|p{3.5cm}|p{5.5cm}|} \hline
%		Method name & Query translator inverse & Answer translator inverse\\ \hline
%		Prefixing & 
%		\begin{algorithm}{ $\FixedprefixqueryRO{\pfvec}^{-1}(U)$}
%			\item for $i$ in $[1..n]$
%			\item \hindent if $\pfvec[i]\prefix x$ then return $(i,x)$
%			\item return $\bot$
%		\end{algorithm} &
%		\begin{algorithm}{$\answerInv(S,\vecY)$}
%			\item Return $\vecY[0]$
%		\end{algorithm} \\ \hline
%		Length differentiating &
%		\begin{algorithm}{$\LengthqueryRO^{-1}[\workDom(U)]$}
%			\item for $i$ in $[1..n]$
%			\item \hindent if $(i,S) \in \workDom$ then return $(i,S)$.
%		\end{algorithm} & 
%		\begin{algorithm}{$\answerInv(S,\vecY)$}
%			\item Return $\vecY[0]$
%		\end{algorithm} \\ \hline
%		Output splitting &
%		\begin{algorithm}{$\SplittingqueryInv(U)$}
%			\item Return $((i,S))_{i \in [1..n]}$
%		\end{algorithm}& 
%		\begin{algorithm}{$\SplittinganswerInv(S,\vecY)$}
%			\item Return $\vecY[1]\| \vecY[2] \|\cdots\|\vecY[n]$
%		\end{algorithm}\\ \hline
%	\end{tabular}
%	\caption{Query and answer translator inverses for the functions of~\figref{fig:prac-domsep}.}
%	\label{fig:qt-inverses}
%\end{figure}
%
%
%First consider the construction $\mathrm{P}[\aFunc{\functionOut}]_{\SplittingqueryInv,\SplittinganswerInv}(U)$, which is defined in Section~\ref{sec-cc-indiff}. This function, on an input $S \in \bits^*$ calls $\SplittingqueryInv$, which returns a vector $\vecW$ of length $n$ in which the $i^{\text{th}}$ entry is the query $(i,S)$. Then $\mathrm{P}[\aFunc{\functionOut}]_{\SplittingqueryInv,\SplittinganswerInv}(U)$ queries its oracle $\aFunc{\functionOut}$ on each of the entries of $\vecW$, returning a vector $\vecY$ of length $n$ whose entries are elements of $\bits^l$. It then calls $\SplittinganswerInv(S,\vecY)$, which concatenates these entries into a bitstring of length $n*l$.
%
%To satisfy the first condition of invertibility, we must show that for all $S \in \GenroSpDom{\functionOutSet}$ and all $\aFunc{\functionOut} \in \FuncSp{\functionOutSet}$, \[\construct{F}[\mathrm{P}[\aFunc{\functionOut}]_{\SplittingqueryInv,\SplittinganswerInv}](U) = \aFunc{\functionOut}(U).\]
%We separate $U$ into the ordered pair $(i,x)$. The oracle construction on the left-hand side truncates $i$, and queries its oracle on only $x$. The oracle, as discussed above, calls $\aFunc{\functionOut}(j,x)$ for every $j\in [1..n]$, and concatenates the results into a single $n*l$-bit string $Y$. Then the oracle construction calls $\SplittinganswerRO(S,Y)$, which returns the $i^{\text{th}}$ substring of $l$ bits. Of course, this is simply the result of the function call $\aFunc{\functionOut}(i,x) = \aFunc{\functionOut}(U)$, so the first condition of invertibility holds.
%
%The second condition of invertibility requires that $\mathrm{P}[\aFunc{\functionOut}]_{\SplittingqueryInv,\SplittinganswerInv}$ and $\Func{\functionIn}$ are distributed identically when $\aFunc{\functionOut}$ and $\aFunc{\functionIn}$ are drawn from $\FuncSp{\functionOutSet}$ and $\FuncSp{\functionInSet}$, respectively. 
%Since both function spaces include all possible functions from their domains to their ranges, the outputs of both $\aFunc{\functionOut}$ and $\aFunc{\functionIn}$ on any input are uniformly random bitstrings. Therefore the output of $\aFunc{\functionInSet}$ is a uniformly random $n*l$-bit string, and the output of $\mathrm{P}[\aFunc{\functionOut}]_{\SplittingqueryInv,\SplittinganswerInv}$ is the concatentation of $n$ uniformly random $l$-bit strings, which is itself a uniformly random $n*l$-bit string. Therefore $(\SplittingqueryRO,\SplittinganswerRO)$ is invertible, and the theorem follows from Theorem~\ref{th-cc-indiff-invertible-gen}.
%
%Invertibility is a minimal requirement for a translating oracle construction to be rd-indifferentiable. However, for many practical oracle constructions we can simplify this definition and give a sufficient condition for invertibility and, via the theorem that follows, rd-indifferentiabililty.
%
%We say that the \textit{inverse} of a query translation $\QuT$ for $\GenroSp{\functionInSet}$,$\GenroSp{\functionOutSet}$,$\workDom$ is a function $\QuTInv:\GenroSpDom{\functionInSet}\to\workDom\cup\{\bot\}$,
%which is injective on its support, which for all $x \in \workDom$ has the property that $\QuTInv \circ \QuT(x)=x$.
%\begin{lemma}\label{th-qt-invertible} Let $\aFunc{\functionInSet}$ and $\aFunc{\functionOutSet}$ be function spaces such that $\GenroSpRng{\functionInSet} = \GenroSpRng{\functionOutSet}$. Let $\workDom \subseteq \GenroSpDom{\functionOutSet}$. Let $\QuT\Colon \GenroSpDom{\functionOutSet} \to \GenroSpDom{\functionInSet}$ be a query translation for $\GenroSp{\functionInSet},\GenroSp{\functionOutSet},\workDom$ with inverse $\QuTInv$. 
%	Let $\AnT \Colon \GenroSpDom{\functionOutSet}\cross \GenroSpRng{\functionInSet} \to \GenroSpRng{\functionOutSet}$ be defined such that $\AnT(x,y) = y$ for all $x \in \GenroSpDom{\functionOutSet}$ and $y \in \GenroSpRng{\functionInSet}$.
%	Then $(\QuT,\AnT)$ is invertible.
%\end{lemma}
%\begin{proof}
%	We can define the partial inverses $\QuTInv_p$ and $\AnTInv$ of $\QuT$ and $\AnT$, respectively.
%	
%	Define $\QuTInv_p\Colon \GenroSpDom{\functionOutSet} \to \workDom^*$ as the function with the same support as $\QuTInv$ that on all inputs $x$ in its support, returns a vector of length $1$ whose only entry is $\QuTInv(x)$.
%	For any $U\in\GenroSpDom{\functionInSet}$ and any vector $\vecY$ over $\GenroSpRng{\functionOutSet}$, define $\AnTInv(U,\vecY)=\vecY[1]$. Since $\GenroSpRng{\functionOutSet} = \GenroSpRng{\functionInSet}$, $\AnTInv$ has the appropriate domain and range to be a partial inverse of $\AnT$.
%	
%	To simplify the first condition of invertibility, we evaluate $\AnTInv$, then expand the pseudocode of  $\construct{TF}_{\QuT,\AnT}$.
%	\[ 	\construct{TF}_{\QuT,\AnT}
%	[\mathrm{P}[\aFunc{\functionOut}]_{\QuTInv,\AnTInv}](x) = \construct{TF}_{\QuT,\AnT}
%	[\aFunc{\functionOut}\circ\QuTInv_p](x)= \AnT(x,\aFunc{\functionOut}\circ\QuTInv_p\circ \QuT(x))\;.\]
%	We note that since $\aFunc{\functionOut}$ is evaluated component-wise on vectors, and $\QuTInv_p$ always returns a vector of length at most $1$, we can replace $\QuTInv_p$ with $\QuTInv$. Then the latter two functions cancel each other out for all inputs in $\workDom$, leaving the right-hand-side Since $\QuTInv$ is the inverse of $\QuT$, these functions cancel each other out. Evaluating $\AnT$ shows that this expression equals $\aFunc{\functionOut}(x)$, and the first condition of invertibility holds.
%	
%	Secondly,  we require that the functions $\mathrm{P}[\aFunc{\functionOut}]_{\QuTInv_p,\AnTInv}$ and $\aFunc{\functionIn}$ are identically distributed on the support of $\QuTInv_p$.
%	Evaluating $\AnTInv$ on the left-hand side, we see that this is true if and only if $\aFunc{\functionOut}\circ\QuTInv_p$ is distributed identically to $\aFunc{\functionIn}$ on the support of $\QuTInv_C$. As before, we replace $\QuTInv_p$ with $\QuTInv$. 
%	Since both $\aFunc{\functionOut}$ and $\aFunc{\functionIn}$ are sampled uniformly at random from all functions with their domain and range, they both are distributed identically on distinct queries, unless a collision occurs under $\QuTInv$. However, since $\QuTInv$ is injective on its support, no collisions exist, and the second condition holds.
%	It follows that $(\QuT,\AnT)$ is invertible.
%\end{proof}
%
%
%To demonstrate the use of this theorem, we show invertibility for the prefixing and length-differentiating oracle constructions. 
%
%Since any bitstring may be prefixed by at most one entry of $\pfvec$, the inverse of $\FixedprefixqueryRO{\pfvec}$ is injective on its support. For any  $(i,x) \in [1..n]\times\bits^*$, it is clear that $\FixedprefixqueryRO{\pfvec}^{-1}\circ\FixedprefixqueryRO (i,x)=(i,x)$. Then $\FixedprefixqueryRO{\pfvec}^{-1}$ is the inverse of $\FixedprefixqueryRO{\pfvec}$. By its definition, $\AnT(x,y) = y$ for any $(x,y) \in \GenroSpDom{\functionOutSet}\times\GenroSpRng{\functionInSet}$. Setting $\workDom= [1..n]\times \bits^*$.
%
%Because there is at most one $i$ for any string $x$ such that $(i,S)\in \workDom$, it is clear that $\LengthqueryRO[\workDom(U)]^{-1}\circ \LengthqueryRO[\workDom(U)]$ is the identity function on $\workDom$. Since the entire input to $\LengthqueryRO[\workDom(U)]^{-1}$ is included in its output, it is clearly injective on its support.  Therefore $\LengthqueryRO[\workDom(U)]^{-1}$ is the inverse of $\LengthqueryRO[\workDom(U)]$, and the theorem follows from Corollary~\ref{th-cc-indiff-qt-only}. Note that the function $\LengthqueryRO^{-1}[\workDom]$, and therefore the simulator, requires checking membership in $\workDom$. 

% leave blank space below

\heading{Rd-indiff of \pqcnameRoundTwo{NewHope}.}
We next demonstrate how read-only indifferentiability can highlight subpar methods of oracle cloning, using the example of \pqcnameRoundTwo{NewHope}~\cite{nistpqc:NewHope}.
The base KEM $\kemScheme_1$ defined in the specification of \pqcnameRoundTwo{NewHope} relies on just two random oracles, $G$ and $H_4$. (The base scheme defined by transform $\QpkeToKem_{10}$, which uses 3 random oracles $H_2$, $H_3$, and $H_4$, is equivalent to $\kemScheme_1$ and can be obtained by applying the output-splitting cloning functor to instantiate $H_2$ and $H_3$ with $G$. \pqcnameRoundTwo{NewHope}'s security proof explicitly claims this equivalence~\cite{nistpqc:NewHope}.)

\begin{figure}[t]
\oneCol{0.65}{
	\underline{Adversary $\advA^{\Init,\pub,\priv,\Fin}$}\\
	\quad $\Init()$ \\
	\quad $y \gets \pub(0)$ ;
	 $d \getsr \{1,2\}$ ; $y_d \gets \priv(d,0)$\\
	\quad If ($y_d[1..256]) = y[1..256]$ then $\Fin(1)$ else 
	 $\Fin(0)$
}
	\caption{Adversary against the rd-indiff security of $\construct{F}_{\pqcnameRoundTwo{NewHope}}$.}
	\label{fig-newhope-adv}
	\hrulefill
\end{figure}

The final KEM $\kemScheme_2$ instantiates these two functions through~$\SHAKE{256}$ without explicit domain separation, setting $\aFunc{H}_4(X) = \SHAKE{256}(X,32)$ and $\aFunc{G}_(X) = \SHAKE{256}(X,96)$.
For consistency with our results, which focus on sol function spaces, we model $\SHAKE{256}$ as a random member of a sol function space $\GenroSp{\functionInSet}$ with some very large output length $L$, and assume that the adversary does not request more than $L$ bits of output from  $\SHAKE{256}$ in a single call. We let $\GenroSp{\functionOutSet}$ be the arity-2 sol function space defining sub-functions $G$ and $H_4$.
In this setting, the cloning functor $\construct{F}_{\pqcnameRoundTwo{NewHope}}: \GenroSp{\functionInSet} \to \GenroSp{\functionOutSet}$ used by $\pqcnameRoundTwo{NewHope}$ is defined by $\construct{F}_{\pqcnameRoundTwo{NewHope}}[\functionIn](1,X)= s(X)[1..256]$ and $\construct{F}_{\pqcnameRoundTwo{NewHope}}[\functionIn](2,X) = s(X)[1..768]$.
We will show that this functor cannot achieve rd-indiff for the given oracle spaces and the working domain $\workDom=\bits^*$. In Figure~\ref{fig-newhope-adv}, we give an adversary~$\advA$ which has high advantage in the rd-indiff game $\ngameCCINDIFF{\construct{F}_{\pqcnameRoundTwo{NewHope}},\GenroSp{\functionInSet},\GenroSp{\functionOutSet},\workDom,\simulator}$ for any indifferentiability simulator~$\simulator$. When $b=1$ in game $\ngameCCINDIFF{\construct{F}_{\pqcnameRoundTwo{NewHope}},\GenroSp{\functionInSet},\GenroSp{\functionOutSet},\workDom,\simulator}$, we have that
\[ y_d[1..256] = \construct{F}_{\pqcnameRoundTwo{NewHope}}[s](d,0)[1..256] =  \aFunc{\functionIn}(0)[1..256] = y[1..256],\]
so adversary $\advA$ will always call $\Fin$ on the bit $1$ and win.
When $b=0$ in game $\ngameCCINDIFF{\construct{F}_{\pqcnameRoundTwo{NewHope}},\GenroSp{\functionInSet},\GenroSp{\functionOutSet},\workDom,\simulator}$, the two strings $y_1 =\aFunc{\functionOut}_0(1,X)$ and $y_2 = \aFunc{\functionOut}_0(2,X)$ will have different $256$-bit prefixes, except with probability $\epsilon = 2^{-256}$. 
Therefore, when $\advA$ queries $\pub(0)$, the simulator's response $y$ can share the prefix of most one of the two strings $y_1$ and $y_2$. 
Its response must be independent of $d$, which is not chosen until after the query to $\pub$, so $\Pr[y[1..256] = y_d[1..256]] \leq 1/2+\epsilon$, regardless of the behavior of $\simulator$.
Hence, $\advA$ breaks the indifferentiability of~$\queryRO^{\pqcnameRoundTwo{NewHope}}$ with probability roughly~$1/2$, rendering \pqcnameRoundTwo{NewHope}'s random oracle functor differentiable.

The implication of this result is that \pqcnameRoundTwo{NewHope}'s implementation differs noticeably from the model in which its security claims are set, even when $\SHAKE{256}$ is assumed to be a random oracle.
This admits the possibility of hash function collisions and other sources of vulnerability that are not eliminated by the security proof. 
To claim provable security for \pqcnameRoundTwo{NewHope}'s implementation, further justification is required to argue that these potential collisions are rare or unexploitable. 
We do not claim that an attack on read-only indifferentiability implies an attack on the IND-CCA security of \pqcnameRoundTwo{NewHope}, but it does highlight a gap that needs to be addressed. 
Read-only indifferentiability constitutes a useful tool for detecting such gaps and measuring the strength of various oracle cloning methods. 

% !TEX root = main-full.tex

\section{Oracle Cloning in KEMs}
\label{sec-kem}

Having shown rd-indiff of various practical cloning functors, we'd like to come back around and apply this to show IND-CCA security of KEMs (as the target primitive of the NIST PQC submissions) that use these functors. At one level, this may seem straightforward and unnecessary, for it is a special case of a general indifferentiability composition theorem, which says that once indifferentiability of a functor has been shown, ``all'' uses of it are secure. In particular, the composition theorems of~\cite{TCC:MauRenHol04,EC:RisShaShr11} for MRH-indefferentiability apply also to rd-indiff and guarantee security when the latter is measured via a single-stage game, which is true for IND-CCA KEMs. This, however, fails to account for working domains, which are not present in prior indifferentiability formulations; the existing composition results only guarantee security when the working domain is the full domain of the ending space. But this fails to be the case for some oracle cloning methods like length differentiation that are used in NIST PQC KEMs. We want a composition theorem that can allow us to conclude security of such usages.

For this, we first must ask what is the meaning or definition of the working domain in the context of the application, here IND-CCA KEMs. Below, we define this. Then we give a working-domain-conscious composition theorem for IND-CCA KEMs that allows us to draw the conclusions mentioned above. The starting point for this treatment is to enhance the syntax of KEMs to allow them to say precisely what types of ROs they want and use.

%The starting point of this treatment is to adapt KEM syntax to use a single function drawn from a function space instead of multiple random oracles with custom interfaces. This allows us to define a single \INDCCA security game that works for any KEM scheme, as opposed to prior notions. We then define what makes a set $\workDom$ a working domain of a KEM scheme.
%
%With these in place, we can give a working-domain-conscious rd-indiff composition theorem, Theorem~\ref{thm:kem-query-translation}. This theorem states that if a set $\workDom$ is a working domain of an \INDCCA secure KEM $\kemScheme_1$ with function space $\GenroSp{\functionOutSet}$ and $\construct{F}: \GenroSp{\functionInSet} \to \GenroSp{\functionOutSet}$ is an rd-indiff functor, then the natural KEM $\kemScheme_2$ running $\kemScheme_1$ with cloning functor $\construct{F}$ is also \INDCCA secure.
%An rd-indiff composition theorem can be shown 

%We apply the framework and results of prior sections to treat oracle cloning in KEMs. 

%The intent is to illustrate the use of working domains. We define these for KEMs and then give a working-domain-conscious composition theorem for KEMs.

%that (unlike ones for prior indifferentiability notions) takes working domains into account. 



\heading{KEM syntax.} In the formal version of the ROM in~\cite{CCS:BelRog93}, there is a single random oracle that has some fixed domain and range, for example mapping $\bits^*$ to $\bits$. Schemes, however, often want multiple random oracles, and also want their oracles to have particular domains and ranges that depend on the scheme. To capture this, we have the scheme syntax include a specification of the desired function space from which the random oracle is then drawn by games defining security. We suggest that schemes specified in standards include a specification of this space, to avoid errors.

Formally, a key-encapsulation mechanism (KEM) $\kemScheme$ specifies the following. First is 
% an arity-$n$ 
a function space $\kemRoSp$. Now as usual there is a key-generation algorithm $\kemKg$ that, given access to an oracle $\aFunc{H} \in \kemRoSp$, returns a public encryption key and matching secret decryption key, $(\pk,\allowbreak \dk)\getsr\kemKg[\aFunc{H}]$. Next there is an encapsulation algorithm $\kemEnc$ that, given input $\pk$, and given oracle $\aFunc{H}$, returns a symmetric key $K \in \bits^{\kemKl}$ and a ciphertext $C$ encapsulating it, $(C,K) \getsr\kemEnc[\aFunc{H}](\pk)$, where $\kemKl$ is the symmetric-key length of $\kemScheme$. The randomness length of $\kemEnc$ is denoted $\kemRl$. Finally, there is a deterministic decapsulation algorithm $\kemDec$ that, given inputs $\dk,C$, and given oracle $\aFunc{H}$, returns $\kemDec[\aFunc{H}](\dk,C) \allowbreak \in \allowbreak\bits^{\kemKl}\cup\{\bot\}$.



%A public-key encryption scheme $\pkeScheme$ also specifies a function space $\pkeRoSp$ and function $\aFunc{H}$ is drawn from this function space in the manner described for KEMs. There is a randomized key-generation algorithm, which, given oracle access to $\aFunc{H}$ returns a public encryption key and matching secret decryption key $\(ek,\allowbreak \dk) \getsr \pkeKg[\aFunc{H}]()$. There is a message space, which we assume to be the set of all bitstrings of a certain length $\pkeML$. There is an encryption algorithm, that, given inputs $\pk$ and message $m\in \bits^\pkeML$, returns a ciphertext $C$, $(C) \getsr \kemEnc{\aFunc{H}](\pk,m)$. Finally, there is a 

\begin{figure}[t]
	\twoColsNoDivide{0.36}{0.3}{
		\ExperimentHeader{Game %$\ngameINDCPA{\kemScheme,\roSp}$,
			 $\ngameINDCCA{\kemScheme}$}
		
		\begin{oracle}{$\Initialize$}
			\item $\aFunc{H}\getsr\kemRoSp$ ; $b\getsr\bits$
			\item $(\pk,\dk)\getsr\kemKg[\RO]$
			\item $(C^*,K^*_1)\getsr\kemEnc[\RO](\pk)$
			\item $K^*_0\getsr\bits^{\kemKl}$
			\item return $\pk,C^*,K^*_b$
		\end{oracle}
		
		\ExptSepSpace
	}{
		\begin{oracle}{$\DecO(C)$}%{Game $\ngameINDCCA{\kemScheme,\roSp}$}
			\item If ($C=C^*$) then return $\bot$
			\item $K \gets \kemDec[\RO](\dk,C)$
			\item return $K$
		\end{oracle}	
		\ExptSepSpace	
		\begin{oracle}{$\RO(W)$}
			\item return $\aFunc{H}(W)$
		\end{oracle}
		
		\ExptSepSpace
		
		\begin{oracle}{$\Finalize(b')$}
			\item return $(b=b')$ \vspace{4pt}
		\end{oracle}
	}
	
%	\twoCols{0.44}{0.46}{
%		\ExperimentHeader{Games $\ngameOWCPA{\kemScheme,\roSp}$, $\ngameOWPCA{\kemScheme,\roSp}$}
%		
%		\begin{oracle}{$\Initialize$}
%			\item $\aFunc{H}\getsr\roSp$
%			\item $\params\getsr\kemPg[\RO]$
%			\item $(\pk,\dk)\getsr\kemKg[\RO]$
%			\item $(C^*,K^*)\getsr\kemEnc[\RO](\pk)$
%			\item return $\pk,C^*$
%		\end{oracle}
%		
%		\ExptSepSpace
%	}{
%		\begin{oracleC}{$\PCO(C, K)$}{Game $\ngameOWPCA{\kemScheme,\roSp}$}
%			\item If ($K = \kemDec[\aFunc{H}](\dk,C)$) then return $1$
%			\item Else return $0$
%		\end{oracleC}	
%		\ExptSepSpace	
%		\begin{oracle}{$\RO(i,X)$}
%			\item return $\aFunc{H}(i,X)$
%		\end{oracle}
%		
%		\ExptSepSpace
%		
%		\begin{oracle}{$\Finalize(K)$}
%			\item return $(K = K^*)$ \vspace{4pt}
%		\end{oracle}
%	}
	
% 	\twoCols{0.44}{0.46}{
% 		\ExperimentHeader{$\ngameOWCPA{\pkeScheme,\roSp}$}
% 		
% 		\begin{oracle}{$\Initialize()$}
% 			\item $\aFunc{H}\getsr\roSp$
% 			\item $(\pk,\dk)\getsr\pkeKg[\RO]()$
% 			\item $m^* \getsr \bits^{\pkeML}$
% 			\item $c^* \getsr \pkeEnc(\pk,m)$
% 			\item return $ \pk, c^*$
% 		\end{oracle}
% 		
% 		\ExptSepSpace
% 			
% 		\begin{oracle}{$\RO(i,X)$}
% 			\item return $\aFunc{H}(i,X)$
% 		\end{oracle}
% 		
% 		\ExptSepSpace
% 		
% 		\begin{oracle}{$\Finalize(m)$}
% 			\item return $(m=m^*)$ \vspace{4pt}
% 		\end{oracle}
% 		
% 	}{
% 		\ExperimentHeader{Game $\ngameINDCPA{\pkeScheme,\roSp}$}
% 
% 		\begin{oracle}{$\Initialize()$}
% 			\item $\aFunc{H}\getsr\roSp$
% 			\item $(\pk,\dk)\getsr\pkeKg[\RO]()$
% 			\item $b \getsr \bits$
% 			\item return $\pk$
% 		\end{oracle}
% 
% 		\ExptSepSpace
% 		\begin{oracle}{$\EncO(m_0,m_1)$}
% 			\item $C \gets \pkeEnc(\pk,m_b)$
% 			\item return $C$
% 		\end{oracle}
% 		
% 		\ExptSepSpace
% 
% 		\begin{oracle}{$\RO(i,X)$}
% 			\item return $\aFunc{H}(i,X)$
% 		\end{oracle}
% 
% 		\ExptSepSpace
% 
% 		\begin{oracle}{$\Finalize(m)$}
% 			\item return $(m=m^*)$ \vspace{4pt}
% 		\end{oracle}
% 	}
% 	\caption{Top: KEM security games for indistinguishability under chosen-plaintext (resp.\ chosen-ciphertext) attacks (left) and one-wayness under chosen-plaintext (resp.\ plaintext-checking) attacks (right). Bottom: PKE security games for one-wayness (left) and indistinguishability (right) security under chosen-plaintext attacks.}
	\caption{%
		KEM security game for indistinguishability under  chosen-ciphertext attacks. % (top) and one-wayness under chosen-plaintext (resp.\ plaintext-checking) attacks (bottom).
	}
	\label{fig:KEM}
	\hrulefill
\end{figure}


\heading{Security definitions.}
We cast the standard security notion of indistinguishability under chosen-ciphertext attack (IND-CCA) for KEMs~\cite{CraSho03} in our extended syntax in Figure~\ref{fig:KEM}.
% Our treatment is in the multi-user setting, where the adversary upon initialization can choose the number of scheme instances~$u$.
%The security games are parameterized by a function space~$\roSp$;
%by default, this is the KEM scheme's function space~$\kemRoSp$ (and then may be omitted).
Adversary~$\advA$ gets a challenge ciphertext $C^*$ and a challenge key $K_b^*$ that is either the key $K_1^*$ underlying $C^*$ or a random key $K_0^*$, and, to win, must determine $b$. Decapsulation oracle $\DecO$ allows it to decapsulate any non-challenge ciphertext of its choice. We let
\begin{newmath}
	\indccaAdv{\kemScheme}{\advA} = 2 \Pr[\ngameINDCCA{\kemScheme}] - 1
\end{newmath}%
to be the ind-cca advantage of adversary~$\advA$. 


%(and analogously for \INDCCA).

%One-wayness asks from an adversary~$\advA$ to output the key encapsulated in a given ciphertext. %(for any of the users $1,\dots,u$).
%In the \OWPCA variant, $\advA$ is additionally given a plaintext-checking oracle telling whether a given key is the decapsulation of a given ciphertext.
%We let
%\begin{newmath}
%	\genAdv{OWCPA}{\kemScheme,\roSp}{\advA} = \Pr\big[ \ngameOWCPA{\kemScheme,\roSp} \big]
%\end{newmath}%
%to be the advantage of adversary~$\advA$ in the \OWCPA game (and analogously for \OWPCA).



\heading{Working domain of a KEM.} Let $\kemScheme$ be a KEM. Let $\workDom \subseteq \GenroSpDom{\kemRoSp}$ be a subset of $\GenroSpDom{\kemRoSp}$. Consider game $\ngameDOM{\kemScheme,\workDom}$ in Figure~\ref{fig:DOM}. The intent is that, at the end of the game, the set $\usedDomain$ contains all queries made to $\RO$ by the scheme algorithms, while excluding ones made by the adversary $\advA$ but not by scheme algorithms. Boolean flag $\SchemeQuery$ controls when a query $W$ to $\RO$ is to be put in $\usedDomain$ in accordance with this policy. (We do assume all queries to $\RO$ are in $\GenroSpDom{\kemRoSp}$.) The adversary wins if it can make the scheme algorithms query a point outside the working domain. Its wdom-advantage is $\wdomAdv{\kemScheme,\workDom}{\advA} \allowbreak = \allowbreak \Pr[\ngameDOM{\kemScheme,\workDom}(\advA)]$. We say that \textit{$\workDom$ is a working domain of $\kemScheme$} if $\wdomAdv{\kemScheme,\workDom}{\advA}=0$ for all adversaries $\advA$, regardless of the running time and number of oracle queries of~$\advA$. 
% (This condition can be relaxed to a computational one, and the latter suffices for Theorem~\ref{th-kem}, but in practice we are not aware of this relaxation being useful, so make the definition as we do.) 

The set $\GenroSpDom{\kemRoSp}$ is always a working domain of $\kemScheme$. The interesting case is when one can specify a subset of it that is a working domain.


%\hd{The computational condition is not sufficient.}
 

 





\begin{figure}[tp]
	\oneCol{0.6}{
	\ExperimentHeader{Game $\ngameDOM{\kemScheme,\workDom}$}

	\begin{oracle}{$\Initialize$}
		\item $\aFunc{H}\getsr\kemRoSp$
		; $\SchemeQuery\gets\true$
		\item $(\pk,\dk)\getsr\kemKg[\RO]$ ; 
		 $(C,K)\getsr\kemEnc[\RO](\pk)$
		\item $\SchemeQuery\gets\false$
		; Return $\pk,C,K$
	\end{oracle}


	\ExptSepSpace

	\begin{oracle}{$\DecO(C)$}
		\item $\SchemeQuery\gets\true$
		; $K \gets \kemDec[\RO](\dk,C)$
		; $\SchemeQuery\gets\false$
		; Return $K$
	\end{oracle}

	\ExptSepSpace

	\begin{oracle}{$\RO(W)$}
	\item If $\SchemeQuery$ then $\usedDomain\gets \usedDomain\cup\{W\}$
		\item return $\aFunc{H}(W)$
	\end{oracle}

	\ExptSepSpace

	\begin{oracle}{$\Finalize$}
		\item return $(\usedDomain\not\subseteq\workDom)$ \vspace{4pt}
	\end{oracle}
	}

	\caption{%
		Game to determine the working domain~$\workDom$ of a KEM~$\kemScheme$.
	}
	\label{fig:DOM}
	\hrulefill
\end{figure}

%In other words, having working domain~$\workDom$ means that all queries made by scheme algorithms to $\aFunc{H}$ are always (with probability one) in the set $\workDom$. This is across all inputs and coins for the algorithms. Note that the condition is only on queries made by scheme algorithms; nothing is imposed on queries made directly by the adversary but never made by scheme algorithms.
%(The flag $\SchemeQuery$ in the game~$\ngameDOM{\kemScheme,\workDom}(\advA)$ is $\true$ when the oracle queries are made by scheme algorithms, and only these are considered for the winning condition.




%We say that $\kemScheme$ has \textit{disjoint working sub-domains} if the sub-domains $\workDom_1,\dots,\workDom_n$ are pairwise disjoint, meaning scheme algorithms never query two different random oracles at the same input.


\heading{Composition.} Let $\kemScheme$ be a given KEM that we assume is IND-CCA secure. Let $\construct{F} \Colon \GenroSp{\functionInSet} \to \kemRoSp$ be a functor. We associate to them the KEM $\FkemScheme = \construct{F}(\kemScheme)$ that is defined as follows. Its function space is $\FkemRoSp = \GenroSp{\functionInSet}$, the starting space of the functor. The algorithms of $\FkemScheme$, given an oracle for $\functionIn$, run the corresponding algorithm of $\kemScheme$ with oracle $\functionOut = \construct{F}[\functionIn]$. Let $\workDom$ be a working domain for $\kemScheme$ and assume $\construct{F}$ is rd-indiff over $\workDom$. Then Theorem~\ref{th-kem}, below, says that $\FkemScheme$ is IND-CCA as well.

The application to NIST PQC KEMs is as follows. Let $\kemScheme$ be a base KEM from one of the submissions, as discussed in Section~\ref{sec-pqc}, so that $\kemRoSp$ is an arity-4 function space. We know (or are willing to assume) that $\kemScheme$ is IND-CCA. Now, we want to instantiate the four oracles of $\kemScheme$ by a single one, say drawn from the sol function space $\GenroSp{\functionInSet} = \AllSOLFuncs{\bits^*}{\ell}$ for some given value of $\ell$ like $\ell=256$. We pick a cloning functor $\construct{F}\Colon \GenroSp{\functionInSet}\to \kemRoSp$ that determines a function for the base KEM from one of the given functions. The example of interest is that this is the identity cloning functor, which is not rd-indiff over its full domain. Instantiating the oracles of $\kemScheme$, via the functor applied to an oracle of the starting space, yields the KEM $\FkemScheme$. This is what, in Section~\ref{sec-pqc}, we called the final KEM, and the question is whether it is IND-CCA. Employing length differentiation corresponds to the base KEM having the corresponding working domain. From Section~\ref{sec-framework} we know that the identify functor is rd-indiff over this working domain. Now Theorem~\ref{th-kem} says that the final KEM is IND-CCA.



\begin{theorem}\label{th-kem} Let $\kemScheme$ be a KEM. Let $\construct{F} \Colon \GenroSp{\functionInSet} \to \kemRoSp$ be a functor. Let $\FkemScheme = \construct{F}(\kemScheme)$ be the KEM associated to them as above. Let $\workDom$ be a working domain for $\kemScheme$, and let $\simulator$ be a read-only simulator for $\construct{F}$. Let $\advA$ be an ind-cca adversary. Then we construct  adversaries~$\advB$, and $\advD$ such that
	\begin{align*}
		\indccaAdv{\FkemScheme}{\advA}
		& \leq
		\indccaAdv{\kemScheme}{\advB}
		%+ 2\cdot \wdomAdv{\kemScheme,\workDom}{\advC}
		+ 2 \cdot \AdvCCINDIFF{\construct{F},\GenroSp{\functionInSet},\kemRoSp,\workDom,\simulator}{\advD} \;.
\end{align*}
The running time of $\advD$ is about that of $\advA$. If $\advA$ makes $q$ queries to $\RO$, then the running time of $\advB$ is about that of $\advA$ plus $q$ times the running time of $\simulator$. 
\end{theorem}

\begin{figure}[tp]
	\twoCols{0.44}{0.46}{
		\ExperimentHeader{Games $\Gm_0, \Gm_1$} 
				
		\begin{oracle}{$\Initialize$}
			\item $\aFunc{\functionIn}\getsr\GenroSp{\functionInSet}$ ; $\functionOut \gets \construct{F}[\functionIn]$ \Comment{Game $\Gm_0$}
			\item $\commoncoins \getsr \SimgenCC()$ ; $\functionOut \getsr \kemRoSp$ \Comment{Game $\Gm_1$}
			\item $b\getsr\bits$
				\item $(\pk,\dk)\getsr\kemKg[\functionOut]$ 
			\item $(C^*,K^*_1)\getsr\kemEnc[\functionOut](\pk)$ 
						\item $K^*_0\getsr\bits^{\kemKl}$
			\item return $\pk,C^*,K^*_b$
		\end{oracle}
		
		\ExptSepSpace
		\begin{oracle}{$\DecO(C)$}%{Game $\ngameINDCCA{\kemScheme,\roSp}$}
			\item If ($C=C^*$) then return $\bot$
			\item $K \gets \kemDec[\functionOut](\dk,C)$ 			\item return $K$
		\end{oracle}	
		\ExptSepSpace	
		
		\begin{oracle}{$\RO(U)$} 
			\item return $\functionIn(U)$ \Comment{Game $\Gm_0$}
			\item return $\Simeval[\functionOut](\commoncoins, U)$ \Comment{Game $\Gm_1$}
		\end{oracle}
\ExptSepSpace
		
		\begin{oracle}{$\Finalize(b')$}
			\item return $(b=b')$ \vspace{4pt}
		\end{oracle}
		\ExptSepSpace
		
	}{
		\ExperimentHeader{Games $\Gm_2$, $\Gm_3$} 
		
			\begin{oracle}{$\Initialize$}
			\item $\aFunc{\functionIn} \getsr \GenroSp{\functionInSet}$ ; $\functionOut_1
			 \gets \construct{F}[\aFunc{\functionIn}]$ 
			 \item $\commoncoins \getsr \SimgenCC()$ ; $\functionOut_0 \getsr \kemRoSp$ 
			\item $c \getsr \bits$
		\end{oracle}
		
		\ExptSepSpace
		
			\begin{oracle}{$\priv(W)$}
			\item If $W \not\in\workDom$ then
			\item \hindent $\bad\gets\true$ 
			\item \hindent return $\bot$ \Comment{Game $\Gm_3$} 
			\item return $\aFunc{\functionOut}_c(W)$
		\end{oracle}
		\ExptSepSpace
		
		\begin{oracle}{$\pub(U)$}
			\item if $(c=1)$ then return $\aFunc{\functionIn}(U)$
			\item else return $\Simeval[\functionOut_0](\commoncoins, U)$
		\end{oracle}
		
		\ExptSepSpace
		
		\begin{oracle}{$\Finalize(c')$}
			\item return $(c = c')$
		\end{oracle}
	
		


}
\caption{Games for the proof of Theorem~\ref{th-kem}.}
\label{fig-kem-games}
\hrulefill
\end{figure}


\begin{figure}[tp]
	\twoCols{0.44}{0.42}{
		\begin{algorithm-initial}{Adversary $\advD$} 
		\item $\advA^{\Initialize', \DecO', \RO', \Finalize'}()$
		\end{algorithm-initial}

		\ExptSepSpace	
		
		\begin{algorithm-subsequent}{$\Initialize'$}
			\item $b\getsr\bits$
			\item $(\pk,\dk)\getsr\kemKg[\priv]$
			\item $(C^*,K^*_1)\getsr\kemEnc[\priv](\pk)$
			\item $K^*_0\getsr\bits^{\kemKl}$
			\item return $\pk,C^*,K^*_b$
		\end{algorithm-subsequent}
		
		\ExptSepSpace
		
		\begin{algorithm-subsequent}{$\DecO'(C)$}
			\item If ($C=C^*$) then return $\bot$
			\item $K \gets \kemDec[\priv](\dk,C)$
			\item return $K$
		\end{algorithm-subsequent}	
		
		\ExptSepSpace	
		
		\begin{algorithm-subsequent}{$\RO'(U)$}
			\item return $\pub(U)$
		\end{algorithm-subsequent}
		
		\ExptSepSpace
		
		\begin{algorithm-subsequent}{$\Finalize'(b')$}
			\item if $(b = b')$ then $\Finalize(1)$
			\item else $\Finalize(0)$\vspace{4pt}
		\end{algorithm-subsequent}
	}{		
		\begin{algorithm-initial}{Adversary $\advB$}
			\item $\commoncoins \getsr \SimgenCC()$
			\item $\advA^{\Initialize', \DecO', \RO', \Finalize'}()$
		\end{algorithm-initial}
	\ExptSepSpace
	
	\begin{algorithm-subsequent}{$\Initialize'$}
		\item $(\pk,C^*,K^*_b) \gets \Initialize()$
		\item return $\pk,C^*,K^*_b$
	\end{algorithm-subsequent}
	\ExptSepSpace
	
	\begin{algorithm-subsequent}{$\DecO'(C)$}%{Game 
		\item return $\DecO(C)$
	\end{algorithm-subsequent}	
	
	\ExptSepSpace	
	
	\begin{algorithm-subsequent}{$\RO'(W)$}
		\item return $\Simeval[\RO](\commoncoins,W)$
	\end{algorithm-subsequent}
	
	\ExptSepSpace
	
	\begin{algorithm-subsequent}{$\Finalize'(b')$}
		\item $\Finalize(b')$\vspace{4pt}
	\end{algorithm-subsequent}

	}
	\caption{Adversaries for the proof of Theorem~\ref{th-kem}.}
	\label{fig-kem-advs}
	\hrulefill
\end{figure}


\begin{proof} Consider the games in Figure~\ref{fig-kem-games}. We have
\begin{align*}
	\indccaAdv{\FkemScheme}{\advA} &= 2\Pr[\Gm_0(\advA)]-1 \\
	&= 2\Pr[\Gm_1(\advA)]-1 + 2(\Pr[\Gm_0(\advA)]-\Pr[\Gm_1(\advA)]) \;.
\end{align*}
Let adversary $\advB$ be as shown in Figure~\ref{fig-kem-advs}. Then
\begin{align*}
 2\Pr[\Gm_1(\advA)]-1 \leq \indccaAdv{\kemScheme}{\advB} \;.
\end{align*}
Game $\Gm_3$ is game $\ngameCCINDIFF{\construct{F},\GenroSp{\functionInSet},\kemRoSp,\workDom,\simulator}$. Game $\Gm_2$ drops the working domain check at line~4. Let adversary $\advD$ be as shown in Figure~\ref{fig-kem-advs}. Then
\begin{align*}
	\Pr[\Gm_0(\advA)]-\Pr[\Gm_1(\advA)] &\leq 2\Pr[\Gm_2(\advD)]-1 \;.
\end{align*}
Games $\Gm_2,\Gm_3$ are identical-until-$\bad$ so by the Fundamental Lemma of Game Playing~\cite{EC:BelRog06} we have
\begin{align*}
	2\Pr[\Gm_2(\advD)]-1 &= 2\Pr[\Gm_3(\advD)]-1 + 
	2(\Pr[\Gm_2(\advD)]-\Pr[\Gm_3(\advD)]) \\
	&\leq 2\Pr[\Gm_3(\advD)]-1 + 2\Pr[\Gm_2(\advD)\mbox{ sets }\bad] \;.
\end{align*}
Now we have
\begin{align*}
	2\Pr[\Gm_3(\advD)]-1 &= \AdvCCINDIFF{\construct{F},\GenroSp{\functionInSet},\kemRoSp,\workDom,\simulator}{\advD}  \;.
\end{align*}
Adversary $\advD$ invokes its $\priv$ oracle only on points queried by scheme algorithms, and, regardless of the challenge bit $c$, the function underlying $\priv$ is a member of $\kemRoSp$. Because $\workDom$ is a working domain for $\kemScheme$, we have
\begin{align*}
	\Pr[\Gm_2(\advD)\mbox{ sets }\bad] &=0  \;.
\end{align*}
This concludes the proof. \qed
\end{proof}
%
%
%	We prove the theorem with a sequence of games, with the first, $\Gm_0$ being the  $\ngameINDCCA{\FkemScheme,\GenroSp{\functionInSet}}(\advA)$ game with the algorithms of $\FkemScheme$ unrolled; meaning they explicitly run the algorithms of $\FkemScheme$ with $\construct{F}[\aFunc{\functionIn}]$ as an oracle.
%	In Game $\Gm_1$, we replace the random oracle $\aFunc{\functionIn}$ of the previous game with two identical oracles $\priv_1 =\pub_1 = \aFunc{\functionIn}$, allowing the scheme to use the internal oracle $\priv_1$ in place of its random oracle $\aFunc{\functionIn}$, and giving the adversary access to $\pub_1$. 
%	This does not change the behavior of the random oracle or the values output by any oracle, so 
%	\[\Pr[\Gm_1(\advA)]= \Pr[\Gm_0(\advA)].\]
%	In Game $\Gm_2$, we set $\priv = \construct{F}[\aFunc{\functionIn}]$ and let the oracles of $\kemScheme$ call $\priv$ directly. The adversary cannot tell whether $\construct{F}$ is computed inside or outside the $\priv$ oracle, so this change is unobservable, and
%	\[\Pr[\Gm_2(\advA)] = \Pr[\Gm_1(\advA)]. \]
%	In Game $\Gm_3$, we set a $\bad$ flag if the internal oracle $\priv$ is called on a point outside of $\workDom$. This is just bookkeeping, so 
%	\[\Pr[\Gm_3(\advA)] = \Pr[\Gm_2(\advA)].\]
%	In Game $\Gm_4$, we return $\bot$ whenever the $\bad$ flag is set. By the identical-until-bad lemma, 
%	\[ \Pr[\Gm_4(\advA)] - \Pr[\Gm_3(\advA)] \leq \Pr[\Gm_3(\advA)\text{ sets }\bad].\]
%	Game $\Gm_3$ sets the bad flag only if $\kemScheme$ calls its oracle $\priv$ on a point outside of $\workDom$. However, $\priv = \construct[F][\aFunc{\functionIn}] \in \GenroSp{\functionOutSet}$ and $\workDom$ is a working domain of $\kemScheme$. Therefore no series of $\Initialize$ and $\DecO$ queries can make $\kemScheme$ query $\priv$ outside of $\workDom$, and $\Pr[\Gm_3(\advA)\text{ sets }\bad]=0$, so
%	\[ \Pr[\Gm_4(\advA)]= \Pr[\Gm_3(\advA)].\]
%	In Game $\Gm_5$, we draw a function $\aFunc{\functionOut}$ from $\GenroSp{\functionOutSet}$, and we let $\priv(i,X)$ return $ =\aFunc{\functionOut}(i,X)$ for all $(i,X) \in \workDom$. We also set $\pub_4 = \Simeval[\aFunc{\functionOut}](\commoncoins, \cdot)$, where $\commoncoins \getsr \SimgenCC()$ is chosen at the outset of the game. 
%	We construct an adversary $\advD$ in Figure~\ref{fig-kem-advs} such that 
%	 \[
%	 \Pr\big[ \Gm_{4} \big] - \Pr\big[ \Gm_{5} \big]
%	 \leq \AdvCCINDIFF{\construct{F},\GenroSp{\functionInSet},\GenroSp{\functionOutSet},\workDom,\simulator}{\advD}.
%	 \]
%	The adversary $\advD$ plays game $\ngameCCINDIFF{\construct{F},\GenroSp{\functionInSet},\GenroSp{\functionOutSet},\workDom,\simulator}$. It perfectly simulates either Game~$\Gm_4$ or Game~$\Gm_5$ for $\advA$ by relaying queries to the $\pub$ and $\priv$ oracles to its own $\pub$ and $\priv$ oracles, sampling the challenge bit~$b$ itself.
%	If $\advA$ wins the simulated game, $\advD$ guesses that $b=1$ in the rd-indiff game; otherwise $\advD$ guesses that $b=0$.
%	
%	In game $\Gm_6$, we set $\pub = \aFunc{\functionOut}$.
%	Next, we define a wrapper adversary $\advB$ with four oracles $\Initialize$, $\DecO$, $\RO$, and $\Finalize$. This adversary runs $\SimgenCC$ to get a string $\commoncoins$, then runs $\advA$. Whenever $\advA$ makes a query to $\pub$, the adversary $\advB$ answers it with $\Simeval[\RO](\commoncoins,\cdot)$.  This change simply moves the execution of $\simulator$ from the $\pub$ oracle to the adversary, so the sequence of operations in $\Gm_6(\advB)$ is identical to that of $\Gm_5(\advA)$, and
%	\[ \Pr[\Gm_6(\advB)] =\Pr[\Gm_5(\advA)] .\]
%	The final game $\Gm_7$, eliminates the internal $\priv$ oracle and gives the algorithms of $\kemScheme$ access to $\RO$ instead. This change is unobservable unless $\kemScheme$ queries $\priv$ on a point outside of $\workDom$. This is impossible because $\workDom$ is a working domain of $\kemScheme$, so 
%	\[\Pr[\Gm_7(\advB)] = \Pr[\Gm_6(\advB)].\]  
%	 Looking at game $\Gm_7$, we see that it is identical to the game $\genAdv{arg1}{arg2}{arg3}\ngameINDCCA{\kemScheme,\GenroSp{\functionOutSet}}(\advB)$, so 
%	 \[\Pr[\Gm_7(\advB)] = \Pr[\ngameINDCCA{\kemScheme,\GenroSp{\functionOutSet}}(\advB)].\]
%	 
%	 Then we have the bound 
%	 \begin{align*}	 \Pr[\ngameINDCCA{\kemScheme[\construct{F}],\GenroSp{\functionInSet}}(\advA)] - \Pr[\ngameINDCCA{\kemScheme,\GenroSp{\functionOutSet}}(\advB)] &= \Pr[\Gm_0(\advA)] - \Pr[\Gm_7(\advB)]\\
%	 &= \Pr[\Gm_4\advA)] - \Pr[\Gm_5(\advA)]\\
%	 &\leq \AdvCCINDIFF{\construct{F},\GenroSp{\functionInSet},\GenroSp{\functionOutSet},\workDom,\simulator}{\advD}
%	 \end{align*}
%	 The theorem statement follows. 
%\end{proof}
%
%


%\subsection{Concrete KEMs with oracle cloning}
%\label{sec-kem-practicaldomsep} %% this will just map to the top section

%\TODO{left over, need summary and wrap up here.}
%
%Coming back to our observations on the NIST PQC submissions, let us now study some of the ways practical schemes achieve domain separation with different techniques.
%As an immediate consequence of this theorem and Corollary~\ref{th-concrete-rd-indiff}, we have shown that the security proof of every scheme in Figure~\ref{fig-domsep-kems} applies to its instantiation. This justifies our hypothesis that the schemes in Group 3 and 4 preserve provable IND-CCA security through the instantiation step. 
%We will consider \pqcname{Classic McEliece}, a member of Group~4 from Section~\ref{sec-pqc} that discusses domain separation as part of its design, as well as \pqcname{SABER}, an example from Group~3 whose schemes achieve domain separation without explicitly discussing it in the design, as we can establish through our formalism.

%\pqcheading{Classic McEliece},
%in its specification, instantiates all of its oracles through $\SHAKE{256}$, which it always calls with a requested output length of $32$ bytes.
%Its KEM transform, $\QpkeToKem_{14}$, uses two random oracles to derive a key confirmation ($Y$ in our framework) and the session key.
%The security proof additionally requires a third independent random oracle to derive an independently random session key in the case that
%decryption fails.
%To clone three independent random oracles, \pqcname{Classic McEliece} uses a simple prefixing query translation, with single-byte prefixes for each oracle: $\texttt{0x00}$ for $\aFunc{H}_4$, $\texttt{0x01}$ for $\aFunc{H}_1$, and $\texttt{0x02}$ for $\aFunc{H}_3$.
%This query translation is $\FixedprefixqueryRO_{\vecX}$,
%\fg{In the submission version, this was written $\FixedprefixqueryRO_{8, \vecX}$ (for $l = 8$ bits).}
%where $\vecX$ is the vector $(\texttt{0x00},\texttt{0x01},\texttt{0x02})$.
%By Corollary~\ref{th-concrete-rd-indiff}, this construction is rd-indifferentiable, so 
%Theorem~\ref{thm:kem-query-translation} gives that the security proof of \pqcname{Classic McEliece} applies to the instantiated scheme. Then \pqcname{Classic McEliece}'s instantiation is \INDCCA-secure.

%\pqcheading{SABER}
%\cite{nistpqc:SABER}
%uses a length-differentiating construction to achieve disjoint working domains.
%\pqcname{SABER} instantiates its three random oracles $\aFunc{H}_1$, $\aFunc{H}_2$, and $\aFunc{H}_3$ with two primitives, namely $\aFunc{H}_1$ with $\SHAA{3}{512}$ and $\aFunc{H}_3$ and $\aFunc{H}_4$ with $\SHAA{3}{256}$.%
%\footnote{Starting to count the random oracles from~$1$, $\aFunc{H}_i$ here corresponds to $\aFunc{H}_{i+1}$ in the transform description from Section~\ref{sec-pqc}.}
%In the working domain~$\workDom$ of \pqcname{SABER}, queries to $\aFunc{H}_3$ always have a length of $64$~bytes and queries to $\aFunc{H}_2$ never have a length of $64$~bytes.
%Therefore, \pqcname{SABER} uses a length-differenting oracle construction to clone $\aFunc{H}_3$ and $\aFunc{H}_4$ from the same random oracle representing $\SHAA{3}{256}$. 
%By Corollary~\ref{th-concrete-rd-indiff}, length-differentiation oracle constructions are rd-indifferentiable, and the inclusion of another independent random oracle in both the primitive and target function spaces does not impact this result. 
%Then by Theorem~\ref{thm:kem-query-translation}, we know that the security result for \pqcname{SABER} in the arity-$3$ oracle space carries over to its instantiation when modeling $\SHAA{3}{256}$ and $\SHAA{3}{512}$ as random oracles, so the instantiation has \INDCCA security as desired.

\mathversion{normal}
\section*{Acknowledgments}

We thank Dan Bernstein for comments and corrections. We thank the Eurocrypt 2020 reviewers for their comments.
 
\chapter{Tighter Bounds for the TLS 1.3 and SIGMA Key Exchange Protocols}\label{chap:tight-ake}
\mathversion{normal2}
\section{Introduction}
\label{sec:introduction}

The Transport Layer Security (TLS) protocol~\cite{rfc8446} is responsible for securing billions of Internet connections every day.
Usage statistics for Google Chrome%
\fullonly{\footnote{\url{https://transparencyreport.google.com/https/}}}		%% last checked 2020-09-03: 76% -- 98%
and Mozilla Firefox%
\fullonly{\footnote{\url{https://telemetry.mozilla.org/}}}				%% last checked 2020-09-03: 89%  (HTTP_PAGELOAD_IS_SSL)
report that $76$--$98$\% of all web page accesses are encrypted.%
At the heart of TLS is an authenticated key exchange (AKE) protocol, the so-called handshake protocol, responsible for providing the parties (client and server) with a shared, symmetric key that is fresh, private and authenticated.
The ensuing record layer secures data using this key.
The AKE protocol of TLS is based on the \SIGMA (``SIGn-and-MAc'') design of Krawczyk~\cite{C:Krawczyk03} for the Internet Key Exchange (IKE) protocol~\cite{rfc2409} of IPsec~\cite{rfc2401},
which generically augments an unauthenticated, ephemeral Diffie--Hellman (DH) key exchange with authenticating signatures and MACs.

Naturally, the \SIGMA AKE protocol and its incarnation in TLS have been the recipients of proofs of security.
We contend that these largely justify the AKE protocols in principle, but not in practice,
meaning not for the parameters in actual use and at the desired or expected level of security.
Our work takes steps towards filling this gap.


\iffull
\subsection{Qualitative and Quantitative Bounds}
\else
\subsubsection*{Qualitative and quantitative bounds.}
\fi

Let us expand on this.
The protocols~$\KE$ we consider are built from
a cyclic group~$\group$ in which some DH problem~$\mathsf{P}$ is assumed to be hard,
a pseudorandom function~$\PRF$ and unforgeable signature and MAC schemes~$\SIGScheme$ and~$\MACScheme$.
The target for~$\KE$ is session-key security with explicit authentication as originating from~\cite{C:BelRog93,EC:BelPoiRog00,EC:CanKra01}.
A proof of security has both a qualitative and quantitative dimension.
Qualitatively, a proof of security for the AKE protocol~$\KE$ says that $\KE$ meets its target definition assuming the building blocks meet theirs,
where, in either case, meeting the definition means any poly-time adversary has negligible advantage in violating it.

The quantitative dimension associates to each adversary in the security game of~$\KE$ a set of resources~$r$,
representing its runtime and attack surface (e.g., the number of users and executed protocol sessions the adversary has access to).
It then relates the maximum advantage of any $r$-resource adversary in breaking $\KE$'s security to likewise advantage functions for the building blocks
through an equation of the (simplified) form
\[
	\Adv_{\KE}(r) \leq %f(r) + 
	f_\group \cdot \Adv^{\mathsf{P}}_{\group}(r_\group) + f_{\SIGScheme} \cdot \Adv^{\EUFCMA}_{\SIGScheme}(r_{\SIGScheme}) + \dots,
\]
deriving quantitative factors~$f_\mathsf{X}$ and resources~$r_\mathsf{X}$ for the advantage of each building block~$\mathsf{X}$.

Speaking asymptotically again, when $f_\mathsf{X}$ and $r_\mathsf{X}$ are polynomial functions in~$r$,
then $\Adv_{\KE}(r)$ is negligible whenever all building blocks' advantages are.
Due to the complexity of key exchange models and the challenging task of combining the right components in a secure manner,
key exchange analyses (including prior work on \SIGMA~\cite{C:CanKra02} and TLS~1.3~\fullelse{\cite{CCS:DFGS15,EuroSP:KraWee16,EPRINT:DFGS16,EuroSP:FisGue17,JC:DFGS21}}{\cite{CCS:DFGS15,EuroSP:KraWee16,EuroSP:FisGue17,JC:DFGS21}}) indeed often remain abstract and consider only qualitative, asymptotic security bounds.

Standardized protocols like TLS in contrast have to define concrete choices for each cryptographic building block.
This involves considering reasonable estimates for adversarial resources (like runtime~$t$ and number of key-exchange model queries~$q$) and specific instances and parameters for the underlying components~$\mathsf{X}$.
One would hope that key exchange proofs can provide guidance in making sound choices that result in the desired overall security level.
Unfortunately, AKE security bounds regularly are highly non-tight, meaning that $f_\mathsf{X}$ and/or $r_\mathsf{X}$ for some components~$\mathsf{X}$ are so large that reasonable stand-alone parameters for~$\mathsf{X}$ yield vacuous key exchange advantages for practical parameters.
While the asymptotic bound tells us that scaling up the parameters for~$\mathsf{X}$ (say, the DDH problem~\cite{Boneh98}) will at some point result in a secure overall advantage,
this causes efficiency concerns (e.g., doubling elliptic curve DH security parameters means quadrupling the cost for group operations) and hence does not happen in practice.
\begin{table}[t]
	\centering
	\small
	
	\renewcommand{\arraystretch}{0.001}
	\renewcommand{\tabcolsep}{0.05cm}
	\begin{tabular}{@{}lllllllllll@{}}
	\toprule
	\multicolumn{3}{c}{Adv.\ resources}		&&&		& \multicolumn{2}{c}{\SIGMA}	& \hspace{0.2cm} & \multicolumn{2}{c}{TLS~1.3} \\
	\cmidrule{1-3} \cmidrule{7-8} \cmidrule{10-11}
	$t$~~~~~~	& $\#U$~~	& $\#S$ && Curve~~~~~~~	& Target~	& CK\,{\scriptsize\cite{C:CanKra02}}~	& Us~{\scriptsize(Thm.~\ref{thm:SIGMAI})}	&& DFGS\,{\scriptsize\cite{JC:DFGS21}}~	& Us~{\scriptsize(Thm.~\ref{thm:tls})} \\
	\midrule
$2^{60}$	&$2^{20}$	&$2^{35}$	&&\texttt{secp256r1} 	&$2^{-68}$	&\cellcolor{red!25}$\approx 2^{-61}$	&$\approx 2^{-116}$	&& \cellcolor{red!25}$\approx 2^{-64}$	&$\approx 2^{-116}$	 \\
$2^{60}$	&$2^{30}$	&$2^{55}$	&&\texttt{secp256r1}	&$2^{-68}$	&\cellcolor{red!25}$\approx 2^{-21}$	&$\approx 2^{-106}$	&& \cellcolor{red!25}$\approx 2^{-24}$	&$\approx 2^{-106}$	 \\
\midrule
$2^{60}$	&$2^{20}$	&$2^{35}$	&&\texttt{x25519}	&$2^{-68}$	&\cellcolor{red!25}$\approx 2^{-57}$	&$\approx 2^{-112}$	&& \cellcolor{red!25}$\approx 2^{-60}$	&$\approx 2^{-112}$	 \\
$2^{60}$	&$2^{30}$	&$2^{55}$	&&\texttt{x25519}	&$2^{-68}$	&\cellcolor{red!25}$\approx 2^{-17}$	&$\approx 2^{-102}$	&& \cellcolor{red!25}$\approx 2^{-20}$	&$\approx 2^{-102}$	 \\
% $2^{60}$	&$2^{20}$	&$2^{35}$	&&\texttt{secp384r1}	&$2^{-132}$	&$\approx 2^{-189}$	& $\approx 2^{-244}$	&& $\approx 2^{-192}$	& $\approx 2^{-244}$	 \\
% $2^{60}$	&$2^{30}$	&$2^{55}$	&&\texttt{secp384r1}	&$2^{-132}$	&$\approx 2^{-149}$	& $\approx 2^{-234}$	&& $\approx 2^{-152}$	& $\approx 2^{-234}$	  \\
\midrule
\midrule
$2^{80}$	&$2^{20}$	&$2^{35}$	&&\texttt{secp256r1}	&$2^{-48}$	&\cellcolor{red!25}$\approx 2^{-21}$	&$\approx 2^{-76}$	&& \cellcolor{red!25}$\approx 2^{-24}$	&$\approx 2^{-76}$	 \\
$2^{80}$	&$2^{30}$	&$2^{55}$	&&\texttt{secp256r1}	&$2^{-48}$	&\cellcolor{red!25}1			&$\approx 2^{-66}$	&& \cellcolor{red!25}1			&$\approx 2^{-66}$	 \\
\midrule
$2^{80}$	&$2^{20}$	&$2^{35}$	&&\texttt{x25519}	&$2^{-48}$	&\cellcolor{red!25}$\approx 2^{-17}$	&$\approx 2^{-72}$	&& \cellcolor{red!25}$\approx 2^{-20}$	&$\approx 2^{-72}$	 \\
$2^{80}$	&$2^{30}$	&$2^{55}$	&&\texttt{x25519}	&$2^{-48}$	&\cellcolor{red!25}1			&$\approx 2^{-62}$	&& \cellcolor{red!25}1			&$\approx 2^{-62}$	 \\
\midrule
$2^{80}$	&$2^{20}$	&$2^{35}$	&&\texttt{secp384r1}	&$2^{-112}$	&$\approx 2^{-149}$	& $\approx 2^{-204}$	&& $\approx 2^{-152}$	& $\approx 2^{-204}$	 \\
$2^{80}$	&$2^{30}$	&$2^{55}$	&&\texttt{secp384r1}	&$2^{-112}$	&\cellcolor{red!25}$\approx 2^{-109}$	&$\approx 2^{-194}$	&& \cellcolor{orange!25}$\approx 2^{-112}$	& $\approx 2^{-194}$	 \\
	\bottomrule
	\end{tabular}
	
	\medskip
	
	\caption{%
		Exemplary concrete advantages of a key exchange adversary with given resources $t$ (running time), $\#U$ (number of users), $\#S$ (number of sessions), in breaking the security of the \SIGMA and TLS~1.3 protocols
		when instantiated with curve \texttt{secp256r1}, \texttt{secp384r1}, or \texttt{x25519},
		based on the prior bounds by Canetti-Krawczyk~\cite{C:CanKra02} resp.\ Dowling et al.~\cite{JC:DFGS21}, and the bounds we establish (Theorem~\ref{thm:SIGMAI} and~\ref{thm:tls}).
		Target indicates the maximal advantage~$t/2^b$ tolerable when aiming for the respective curve's security level ($b = 128$ resp.\ $192$ bits);
		entries in red-shaded cells miss that target.
		See Section~\ref{sec:evaluation} %and Appendix~\ref{apx:evaluation} 
		for full details and curves \texttt{secp521r1} and~\texttt{x448}.
	}
	\label{tbl:bounds-overview}
\end{table}

We illustrate in Table~\ref{tbl:bounds-overview} the effects of the non-tight bounds for \SIGMA and TLS~1.3
when instantiating the protocols with NIST curves \texttt{secp256r1}, \texttt{secp384r1}~\cite{NIST:FIPS-186-4}, or curve \texttt{x25519}~\cite{rfc7748} and idealizing the protocols' other components (see Section~\ref{sec:evaluation} for full details).
Following the curves' security, we aim at a security level of~$128$~bits, resp.~$192$~bits, meaning the ratio of an adversary's runtime to its advantage should be bounded by~$2^{-128}$, resp.~$2^{-192}$.
When considering the advantage of key exchange adversaries running in time~$t$, interacting in the security game with $\#U$ users and $\#S$ sessions,
we can see that previous security bounds fail to meet the targeted security level
for real-world--scale parameters ($\#U$ ranging in $2^{20}$--$2^{30}$ based on $2^{27}$ active certificates on the Internet%
\fullonly{\footnote{\url{https://letsencrypt.org/stats/}}}%		%% last checked 2020-09-03: 136M active certs, 227M fully-qualified domains certified
, $\#S$ ranging in $2^{35}$--$2^{55}$ based on $2^{32}$ Internet users and $2^{33}$ daily Google searches%
\fullonly{\footnote{\url{https://www.internetlivestats.com/}}}%		%% last checked 2020-09-03: 85479 searches per second
).
In the security analysis by Canetti and Krawczyk~\cite{C:CanKra02} (CK) for \SIGMA, the factor associated to the decisional Diffie--Hellman problem is $f_{\DDH}(t,\#U,\#S) = \#U \cdot \#S$,
where $\#U$ and $\#S$ again are the number of users, resp.\ sessions, accessible by the adversary.
The analysis by Dowling et al.~\cite{JC:DFGS21} (DFGS) for TLS~1.3 reduces to the strong Diffie--Hellman problem~\cite{RSA:AbdBelRog01}---via the PRF-ODH assumption~\cite{C:JKSS12,C:BFGJ17}---with factor $f_{\strongDH}(t,\#U,\#S) = (\#S)^2$.
In contrast, we reduce to the strong Diffie--Hellman problem with a constant factor for both \SIGMA and TLS~1.3.

Let us discuss three data points from Table~\ref{tbl:bounds-overview}:
\begin{enumerate}
	\item Already with medium-sized resources, investing time~$t = 2^{60}$ and interacting with a million users ($\#U = 2^{20}$) and a few billion sessions ($\#S = 2^{35}$), the CK~\cite{C:CanKra02} and DFGS~\cite{JC:DFGS21} advantage bounds for \SIGMA and TLS~1.3 with curves \texttt{secp256r1} and \texttt{x25519} fall $6$--$11$~bits below the target of~$2^{-68}$ for $128$-bit security.
	
	\item When considering a more powerful, global-scale adversary ($t = 2^{80}$, $\#U = 2^{30}$, $\#S = 2^{55}$), both CK and DFGS bounds for \texttt{secp256r1}/\texttt{x25519} become fully vacuous;
	the upper bound on the probability of the adversary breaking the protocol is~$1$.
	% (More precisely, the bounds yield an advantage of~$2^{16}$ resp.\ $2^{8}$.)
	We stress that \texttt{secp256r1} is the mandatory-to-implement curve for TLS~1.3;
	\texttt{secp256r1} and \texttt{x25519} together make up for 90\% of the TLS~1.3 ECDHE handshakes reported through Firefox Telemetry.

	\item Finally, and notably, even switching to the higher-security curve \texttt{secp384r1} helps only marginally in the latter case:
	the resulting advantage against \SIGMA falls $3$~bits short of the $192$-bit security target of~$2^{-112}$,
	and the TLS advantage bound only barely meets that target.
\end{enumerate}
For all curves and choices of parameters, our bounds do better. 


\iffull
\subsection{Contributions}
\else
\subsubsection*{Contributions\lncsdot}
\fi

Most prior results in tightly secure key exchange (e.g., \cite{TCC:BHJKL15,C:GjoJag18}) apply only to bespoke protocols, carefully designed to allow for tighter proof techniques, at the cost of requiring more complex primitives which, in the end, eat up the gained practical efficiency.
\iffull
Recently, Cohn-Gordon et al.~\cite{C:CCGJJ19,EPRINT:CCGJJ19} established a proof strategy for a simple and efficient DH key exchange with reasonable tightness loss (only linear in the number of users~$\#U$), achieving implicit authentication through static DH keys through careful key derivation via a random oracle~\cite{CCS:BelRog93} with an optional explicit-authentication step.

\fi
Our work in contrast establishes tight security for standardized AKE protocols.
We give tight reductions for the security of \SIGMA and TLS~1.3 to the strong Diffie--Hellman problem~\cite{RSA:AbdBelRog01},
which in addition we prove is as hard as the discrete logarithm problem in the generic group model (GGM)~\cite{EC:Shoup97,IMA:Maurer05}.
Instantiating our bounds shows that, with standardized real-world parameters, we achieve the intended security levels even when considering powerful, globally-scaled attackers.


\iffull
\paragraph{Code-based security model and proofs}
For our proofs, we provide detailed proof steps and reductions using the code-based game-playing framework of Bellare and Rogaway~\cite{EC:BelRog06}.
Our security model is similar to the one applied by Cohn-Gordon et al.~\cite{C:CCGJJ19},
%considering in particular compromises of long-term secrets and session keys (but not internal state or randomness),
but formalized also as a code-based game (in Section~\ref{sec:ake-model}) and stronger in that it captures explicit authentication and regular (``perfect'') forward secrecy (instead of only weak forward secrecy in~\cite{C:CCGJJ19}).
\else

\fi


\paragraph{Tighter security proof of SIGMA(-I)}
We establish fully quantitative security bounds for \SIGMA and its identity-protecting variant~\SIGMAI~\cite{C:Krawczyk03} in Sections~\ref{sec:sigma} and~\ref{sec:sigma-proof}.
Our result is for BR-like~\cite{C:BelRog93} key exchange security and gives a tight reduction to the strong Diffie--Hellman problem~\cite{RSA:AbdBelRog01} in the used DH group, and to the multi-user (mu) security of the employed pseudorandom function (PRF), signature scheme, and MAC scheme, adapting the techniques by Cohn-Gordon et al.~\cite{C:CCGJJ19} in the random oracle model~\cite{CCS:BelRog93}.
The latter mu-security bounds are essentially equivalent to the corresponding bounds by CK~\cite{C:CanKra02}.
Our improvement comes from shaving off a factor of $\#U \cdot \#S$ (number of users times number of sessions) on the DH problem advantage compared to CK.
While we move to the interactive strong Diffie--Hellman problem (compared to \fullelse{the decisional DH (DDH) problem~\cite{Boneh98} used in~\cite{C:CanKra02}}{DDH~\cite{Boneh98} used in~\cite{C:CanKra02}}),
we prove (in Appendix~\ref{apx:strongDHproof}) that the strong DH problem, like DDH, is as hard as solving discrete logarithms in the generic group model~\cite{EC:Shoup97,IMA:Maurer05}%
\fullonly{, reflecting the (only generic) algorithms known for solving discrete logarithms in elliptic curve groups}.


\paragraph{Tighter security proof for the TLS~1.3 DH handshake}
We likewise establish fully quantitative security bounds for the key exchange of the recently standardized newest version of the Transport Layer Security protocol, TLS~1.3~\cite{rfc8446}, in Sections~\ref{sec:tls} and~\ref{sec:tls-proof}.
The main quantitative improvement in our reduction is again a tight reduction to the strong DH problem, whereas prior bounds by DFGS~\cite{JC:DFGS21} incurred a quadratic loss to the PRF-ODH assumption~\cite{C:JKSS12,C:BFGJ17}, a loss which translates directly to strong DH~\cite{C:BFGJ17}.
While TLS~1.3 roughly follows the \SIGMAI design, its cascading key schedule impedes the precise technique of Cohn-Gordon et al.~\cite{C:CCGJJ19} and a direct application of our results on \SIGMAI, as no single function (to be modeled as a random oracle) binds the Diffie--Hellman values to the session context.
We therefore have to carefully adapt the proof to accommodate the more complex key schedule and other core variations in TLS~1.3's key exchange, achieving conceptually similar tightness results as for \SIGMAI.
% This is reflected in our concrete security bounds for TLS~1.3 based on standardized components:
% for real-world resource parameters (cf.\ Table~\ref{tbl:bounds-overview} and Section~\ref{sec:evaluation}) they meet the targeted security levels of the mandatory-to-implement curve~\texttt{secp256r1} as well as \texttt{secp384r1} and \texttt{x25519},
% and improve upon the DFGS bounds by up to $82$~bits of security.


\paragraph{Evaluation}
In Section~\ref{sec:evaluation}, we evaluate the concrete security implications of our improved bounds for \SIGMA and TLS~1.3 for a wide range of real-world resource parameters and all five elliptic curves \fullonly{(\texttt{secp256r1}, \texttt{secp384r1}, \texttt{secp521r1},\texttt{x25519}, \texttt{x448}) }standardized for use in TLS~1.3~\cite{rfc8446},
a summary of which is displayed in Table~\ref{tbl:bounds-overview}.
\iffull
Leveraging our GGM bound for the strong Diffie--Hellman problem, we focus on the hardness of solving discrete logarithms in the respective elliptic curve groups, instantiating signatures based on ECDSA~\cite{NIST:FIPS-186-4} resp.\ EdDSA~\cite{CHES:BDLSY11}.
We idealized the symmetric PRF, MAC, and hash function primitives (in two variants, with key and output sizes twice as large as the curve's security level, or fixed at $256$~bits corresponding to the choice in most TLS~1.3 cipher suites).

\fi
We report that our tighter proofs indeed materialize for a wide range of real-world resource parameters%
\fullonly{ (adversary runtime~$t \in \{2^{40},2^{60},2^{80}\}$, number of users~$\#U \in \{2^{20},2^{30}\}$, and number of sessions~$\#S \in \{2^{35},2^{45},2^{55}\}$)}.
The resulting attacker advantages meet the targeted security levels of all five curves.
% The resulting attacker advantages meet the targeted security levels of curves~\texttt{secp256r1} (mandatory to implement for TLS~1.3) as well as \texttt{secp384r1} and \texttt{x25519}.
% (For higher-security curves \texttt{secp521r1} and \texttt{x448} and high-end adversary parameters, the idealized mu-security PRF and MAC loss becomes the dominating component, requiring key/output sizes larger than $256$~bits.)
In comparison to the prior CK~\cite{C:CanKra02} \SIGMA and DFGS~\cite{JC:DFGS21} TLS~1.3 bounds,
our results improve the obtained security across these real-world parameters by up to~$85$~bits for \SIGMA and $92$~bits for TLS~1.3, respectively.


\iffull

\iffull
\subsection{Optimizations, Limitations, and Possible Extensions}
\else
\subsubsection*{Optimizations, limitations, and possible extensions\lncsdot}
\fi
\SIGMA being a generic AKE design, the signature, PRF, and MAC schemes may be instantiated with primitives optimized for multi-user security.
While we focus on standardized and deployed schemes in our evaluation without assuming tight mu-security, our \SIGMA bound allows to directly leverage such optimization.
For PRFs and MACs,  efficient candidates exist (e.g., AMAC~\cite{EC:BelBerTes16}).
For signatures, tight mu-security is more challenging~\cite{EC:BJLS16} and often involves computationally much more expensive constructions~\cite{TCC:BHJKL15}.

Like Cohn-Gordon et al.~\cite{C:CCGJJ19}, our key exchange security model considers exposure of long-term secrets and session keys,
but does not allow revealing internal session state or randomness (as in the (e)CK model~\cite{EC:CanKra01,PROVSEC:LaMLauMit07}).
This is appropriate for protocols like TLS~1.3 not aiming to protect against such threats.
The original \SIGMA proof~\cite{C:CanKra02} did establish security in the CK model~\cite{EC:CanKra01} allowing exposure of session state;
in that sense our results are qualitatively weaker.
In recent work, Jager et al.~\cite{EC:JKRS21} give a tightly secure protocol which uses symmetric state encryption to protect against ephemeral state reveals.
Establishing a tight security reduction for a SIGMA-style DH-based AKE protocol which can handle adaptive compromises of session state (including DH exponents) remains a challenging open problem.

In our proofs, we crucially rely on the ability to observe and program a random oracle used for key derivation in the AKE protocol, borrowing from~\cite{C:CCGJJ19}.
Notably, the approach of Cohn-Gordon et al.\ is tailored to an AKE protocol achieving authenticity implicitly through mixing long-term DH keys into the key derivation.
Our proofs can hence be seen as translating and adapting their technique to the setting of \SIGMA and TLS~1.3, where an unauthenticated ephemeral DH exchange is explicitly authenticated through signatures and MACs,
confirming that the generic \SIGMA design as well as the standardized TLS~1.3 protocol bind enough context to their DH shares for this proof technique to work.
Leveraging the random oracle model~\cite{CCS:BelRog93} is another qualitative difference compared to the original \SIGMA proof~\cite{C:CanKra02} in the standard model.
Interestingly, this distinction vanishes in comparison to the provable security results for the TLS~1.3 handshake protocol~\cite{CCS:DFGS15,EPRINT:DFGS16,EuroSP:FisGue17,JC:DFGS21} which employ the PRF-ODH assumption~\cite{C:JKSS12,C:BFGJ17},
an interactive assumption which plausibly can only be instantiated in the random oracle model (from the strong DH assumption).
\fi

% \old{%
% The DFGS analyses of TLS~1.3 establish security in a multi-stage key exchange (MSKE) model~\cite{CCS:FisGue14}, proving security not only of the final session key, but also of intermediate handshake encryption keys and further secrets.
% While our proofs (for both \SIGMA and TLS~1.3) establish security of the intermediate (handshake) encryption key, too,
% we do not treat them as first-class keys available to the adversary (e.g., through revealing them).
% We expect that our results extend to a MSKE treatment, leaving this extension to possible future work.
% }

\iffull
\subsection{Concurrent Work}
\else
\subsubsection*{Concurrent work\lncsdot}
\fi

In concurrent and independent work, Diemert and Jager (DJ)~\cite{JC:DieJag20} studied the tight security of the main TLS~1.3 handshake.
Their work also tightly reduces the security of TLS~1.3 to the strong Diffie--Hellman problem by extending the technique of Cohn-Gordon et al.~\cite{C:CCGJJ19}, and their bounds and ours are similarly tight.
When instantiated with real-world parameters, both bounds are dominated by the same terms, as we will demonstrate in Section~\ref{sec:evaluation}.
Our proof differs from theirs in two key ways:
We use an incomparable security model that is weaker in some ways and stronger in others, and we approximate the TLS~1.3 key schedule with fewer random oracles.
We also contextualize our results quite differently than the DJ work, with a detailed numerical analysis that is enabled by our fully parameterized, concrete bounds.
Uniquely to this work, we treat the more generic \SIGMAI protocol and justify our use of the strong DH problem with new bounds in the generic group model.
Diemert and Jager~\cite{JC:DieJag20} in turn study tight composition with the TLS record protocol. 

The DJ analysis is carried out in the multi-stage key exchange model~\cite{CCS:FisGue14}, proving security not only of the final session key, but also of intermediate handshake encryption keys and further secrets.
While our proof does show security of these intermediate keys, we do not treat them as first-class keys accessible to the adversary through dedicated queries in the security model.
Unlike either the DJ or Cohn-Gordon et al.\ works, our model addresses explicit authentication, which we prove via HMAC's unforgeability.

To tackle the challenge that TLS~1.3's key schedule does not bind DH values and session context in one function, DJ model the full cascading derivation of each intermediate key monolithically as an independent, programmable random oracle (cf.~\cite[Theorem~6]{JC:DieJag20}). 
We instead model the key schedule's inner HKDF~\cite{C:Krawczyk10} extraction and expansion functions as two individual random oracles, carefully connected via efficient look-up tables, yielding a slightly less extensive use of random oracles and compensating for the existence of shared computations in the derivation of multiple keys.
This approach produces more compact bounds and allows our analysis to stay closer to the use of HKDF in TLS~1.3, where the output of one extraction call is used to derive multiple keys.


%%% 2020-11-12
% In concurrent and independent work, Diemert and Jager~\cite{JC:DieJag20} studied the tight security of the main TLS~1.3 handshake.
% Despite the use of different security models, their bounds and ours provide similarly tight reductions to the strong Diffie--Hellman problem
% \acnsreplace{}{ with the same dominating terms for real-world parameters, as we will discuss in Section~\ref{sec:evaluation}}.
% Their analysis is carried out in the multi-stage key exchange model~\cite{CCS:FisGue14}, proving security not only of the final session key, but also of intermediate handshake encryption keys and further secrets.
% To tackle the challenge that TLS~1.3's key schedule does not bind DH values and session context in one function, they model the full cascading derivation of each intermediate key monolithically as an independent, programmable random oracle (cf.~\cite[Theorem~6]{JC:DieJag20}). 
% We instead model the key schedule's inner HKDF~\cite{C:Krawczyk10} extraction and expansion functions as two individual random oracles, carefully connected via efficient look-up tables, \acnsreplace{}{yielding a slightly less extensive use of random oracles} and compensating for the existence of shared computations in the derivation of multiple keys. .
% This approach produces more compact bounds and allows our analysis to stay closer to the use of HKDF in TLS~1.3, where the output of one extraction call is used to derive multiple keys.
% \acnsreplace{}{Beyond the strong DH problem, both their and our proofs further reduce to multi-user security of signatures and PRFs, applying random oracle bounds for the latter (cf.~\cite[Section~5]{JC:DieJag20} and our \fullelse{Section~\ref{sec:components:muPRF}}{Appendix~\ref{apx:components:muPRF}}).}
% In addition, our model captures explicit authentication \acnsreplace{}{(which we prove via HMAC's unforgeability)} and our bounds are fully parameterized enabling the evaluation of concrete practical advantages (cf.~Section~\ref{sec:evaluation}).
% Finally, while our work additionally treats the more generic \SIGMAI protocol and proves GGM bounds for the strong DH problem,
% \iffull
% Diemert and Jager~\cite{JC:DieJag20} further study composition of the TLS~1.3 handshake with the nonce-randomized AES-GCM encryption in the record protocol,
% connecting tight multi-user bounds for the latter~\cite{C:BelTac16,CCS:HoaTesThi18} with a tighter version of prior MSKE composition results~\cite{thesis:Guenther18} they establish.
% \else
% Diemert and Jager~\cite{JC:DieJag20} in turn study tight composition with the TLS record protocol.
% \fi
% \acnsreplace{}{We will further discuss and compare our technical results with those of Diemert and Jager~\cite{JC:DieJag20} in more detail throughout the paper.}


%% AC20 rebuttal
% This indeed is independent and concurrent work, published after the AC deadline. While DJ and our proofs use different security models, the results are essentially consistent. We’ll add a detailed comparison; briefly the main differences are:
%   * DJ use the DFGS [22,..] model that allows Reveal/Test queries on intermediate keys; we simplify presentation by limiting these queries to the final session keys (as you said, @R1). Our proof strategy would however easily allow to branch out the PRF/KDF-steps to show security of intermediate keys.
%   * DJ model the derivation of each intermediate key as an independent, programmable RO (cf. DJ Thm. 6). We instead model HKDF.Extract/Expand as two small, individual ROs (carefully connected via look-up tables), yielding more compact and slightly tighter bounds. This better captures the use of HKDF in TLS 1.3, where the output of one Extract call is used to derive multiple keys. 
%   * DJ prove tight multi-user PRF bounds for HMAC/HKDF in the ROM. We likewise apply ROM mu-PRF bounds (cf. Apx B.2), but additionally prove explicit authentication via HMAC’s EUF-CMA security.
%   * We give fully parameterized bounds and concrete practical advantages, and prove a GGM bound for StrongDH.
%   * We also analyze the more generic SIGMA(-I) case.


%% AC20 older, longer version
% This indeed is independent and concurrent work, published after the AC deadline. DJ and our bounds are essentially consistent: Our bound contains the same tight bounds for symmetric primitives (HMAC/HKDF), it gains from a more fine-grained KDF modeling and includes a MAC term for explicit authentication. We’ll add a detailed comparison, the main differences are:
% 
% * DJ:
%   - show security of intermediate keys via the multi-stage KE model (like prior work by DFGS [22,..]); we focus on the main keys and only conjecture intermediate keys’ security (making the models different--correct @R1).
%   - model compound steps for each key's derivation, yielding a higher-level KDF with inputs directly bound to final key derivation, at the cost of 4+1 ROs and several StrongDH proof steps (cf. DJ Thm. 6)
%   - give slightly tighter MSKE composition and connect to mu-AEAD bounds
% 
% * We:
%   - model only HKDF Extract/Expand as two small, individual ROs (carefully connected via look-up tables), yielding more compact and slightly tighter bounds (1 StrongDH step, 2 ROs); the same approach can be applied for multi-stage KE
%   - also analyze the more generic SIGMA(-I) case
%   - show explicit authentication
%   - give fully parameterized bounds and concrete practical advantages
%   - prove a GGM bound for StrongDH, enabling comparison between assumptions    




\iffalse
\newpage
\section*{\color{Red}Old/Draft Introduction}





Authenticated key exchange (AKE), allowing two parties to establish a shared secret over an insecure communication channel,
is one of the most widely used cryptographic components in today's world,
securing billions of Internet connections every day. \hd{Mihir's comment: Needs more specificity. Exactly which protocols are widely used? What does widely mean?}
\hd{The TLS protocol for secure Internet communication is used by x\% of the top Y00K Alexa sites. Its handshake protocol is an authenticated key exchange (AKE), which allows two parties to establish a shared secret over an insecure communication channel.
AKE is also a crucial component of other major Internet security protocols like IKE and the Signal messaging protocol. }
 
\hd{Here is my super rough intro outline, focusing more on what the problem we solve is and nuuumbers. All notation is absurd shorthand and not remotely final.}
	
\hd{	One thing we desire of an authenticated key exchange protocol KE is provable security. What does this actually mean? It means a theorem gives an upper bound for the advantage of an adversary attacking KE. This upper bound usually depends on the hardness of a well-known problem, such as the Decisional Diffie--Hellman problem or the discrete logarithm problem. Intuitively, if the discrete log problem is hard, then an adversary should not be able to break KE. }

\hd{Of course, the hardness of any problem is dependent on its size and the resources of the adversary. Discrete logarithms are easy to compute in small-order groups, and any AKE scheme should be easy to break if its key length is one bit. Bounds on an adversary's advantage therefore depend on the runtime of the adversary and the size of its attack surface. The latter is measured by the number of queries the adversary makes in a security game; each query represents some interaction of the adversary with its environment. }

\hd{A typical bound has the form Adv(KE)(t,queries) $\leq$ F(queries, params) + G(queries,params)*Adv(problem)(params,T(params,queries,t),queries). Here, F is a negligible function, and G and T are polylog functions. If problem is hard, then we assume that Adv(problem)(params, t,queries) is negligible. From an asymptotic perspective, Adv(KE)(t,queries) is negligible, which is good enough. In a concrete setting where we pick values for t, queries, and params, the story is more complicated. }
\fg{Maybe one sample bound, with just one subproblem. Use DH Problem with ballpark numbers.}

\hd{We choose t and queries based on realistic assumptions about the computational resources of a potential adversary. We also assume the hardness of problem based on similar computational resources. If T(params, queries, t) is large, it may exceed reasonable assumptions about computational resources. In this case, Adv(problem)(params, T(params,queries,t) may be high. Similarly, if G(queries, params) is large, the bound on KE may not prevent viable adversaries even if Adv(problem) is small.  Of course, since Adv(problem) is negligible, we can always scale up the parameters until the upper bound on the adversary is sufficiently small. In practice, this causes efficiency concerns and does not happen.}

\hd{A tight security reduction to problem is one for which G and T are small polynomials. In order to use a security bound to exclude realistic attacks, we need two things: first, we need problem to be hard for realistic adversaries. Second, we need the reduction to problem to be tight. }

\hd{TLS 1.3 does not have a tight security proof. For realistic parameters and standardized groups, existing bounds do not provide the target level of security. Numbersnumbersnumbersnumbers. }

\hd{Some existing tight AKE reductions to hard problems already exist, but they target implicit authentication. Many major Internet security protocols like TLS and IKE target explicit authentication because why? They accomplish this by following the design of the SIGMA protocol.}

\hd{We extend Cohn-Gordon's techniques to the explicitly authenticated AKE protocols SIGMA and TLS1.3. We give tight reductions to the stDH assumption, which we prove is as hard as discrete log in the generic group model. Instantiating our bounds shows that with standardized parameters, we achieve target security levels even considering a globally-scaled attacker. Now on to contributions subsubsection}

At the heart of most AKE protocols \hd{Mihir's comment: Most in what pool of protocols?}%in most cases 
is a Diffie--Hellman-style key exchange~\cite{DifHel76}, leveraging the versatility, efficiency, and security of this elegant primitive. \hd{Mihir's comment: Say DH is used because of forward secrecy instead of praising it.}

To ensure (explicit) authentication, many practical key exchange designs including major Internet security protocols like TLS~\cite{rfc5246,rfc8446} and the IKE protocol~\cite{rfc2409} of IPsec~\cite{rfc2401} follow the \SIGMA (``SIGn-and-MAc'') key exchange design put forward by Krawczyk~\cite{C:Krawczyk03} which augments Diffie--Hellman (DH) key exchange with authenticating signatures and MACs.

When it comes to determining which concrete security parameters for the protocol building blocks to use when deploying key exchange protocols, parameters should ideally be both theoretically sound (in the sense of providing meaningful security guarantees based on the proof) as well as reasonably efficient. \hd{Mihir's comment: the scheme, not the parameters, should be sound an efficient; and this needs to be more specific.}
\hd{My intepretation: numbers would be helpful here: we want to choose parameters so that the scheme can run in x time and have a guaranteed security level of y. This naturally leads into the tradeoff of non-tight reductions.}
One would hope that reductionist security proofs provide sound guidelines for deriving such parameters. 
Unfortunately, the proof techniques applied in key exchange security results most often suffer from (highly) non-tight security reductions.
The available reductions incur security losses from the protocol primitives' security that are linear, or sometimes even quadratic, in the number of protocol sessions considered and parties being involved. % due to guessing steps performed in the reduction.
For example, the security proof for \SIGMA~\cite{C:CanKra02} has a linear loss in the number of sessions, and proofs of the (conceptually more complex) TLS protocol in version~1.2~\cite{C:JKSS12,C:KraPatWee13,C:BFKPSZ14} as well as the newest version~1.3~\cite{CCS:DFGS15,EPRINT:DFGS16} incur a quadratic loss in the number of sessions. \hd{Mihir's comment: There is always a tight reduction from SOME assumption; so talking about the loss without this context isn't clear.}
Considering that Google alone securely serves several billion search requests every day\footnote{\url{https://www.internetlivestats.com}, retrieved 2020-02-10},
the number of key exchange sessions in major Internet security protocols may well be on the order of $2^{50}$ across the Internet over a longer period of time. \hd{Mihir's comment: how many searches are done per session key? What is the source of the $2^{50}$ number?}
Taking such numbers into account, theoretically sound parameters for key exchange components would need to provision for about a $100$-bit security loss. \hd{Mihir's comment: What is a loss and how do you measure it in bits?}
This leaves deployed protocols like~TLS in the unfortunate situation that the gap between theoretically sound parameters and those actually deployed is too big for concrete security bounds to be at all meaningful for the real-world deployment.
\hd{Mihir's comment: What is a theoretically sound parameter? Those actually deployed: we don't deploy parameters. We deploy schemes. What is a bound, and what does it mean for it to be meaningful?}
\hd{My interpretation: I think we should focus more on the environment of our paper: A scheme is proven secure, meaning it has a parametrized reduction bounding the advantage of an adversary by the hardness of a well-known problem. We want to pick large parameters that make the advantage prohibitively low. We also want to pick small parameters that make the scheme fast. If we pick parameters that are too small, the parametrized equation may allow adversaries that have high advantage. }
The security losses seen in security proofs of many key exchange protocols has led to explorations of protocol designs with \emph{tight} security proofs.
Results in this direction include the works by Bader et al.~\cite{TCC:BHJKL15} as well as Gj\o{}steen and Jager~\cite{C:GjoJag18} which achieve fully tight security (i.e., with very small security loss in the parameters),
but at the cost of requiring more complex primitives which, in the end, eat up the gained efficiency (even compared to standard key exchange protocols instantiated with parameters accounting for the non-tight losses).
More recently, Cohn-Gordon et al.~\cite{C:CCGJJ19,EPRINT:CCGJJ19} managed to achieve a reasonable trade-off between tightness and efficiency,
putting forward a nifty proof strategy for a simple and efficient, implicitly authenticated key exchange protocol.
Their main protocol (called~$\Pi$) uses a simple ephemeral Diffie--Hellman exchange, combining both ephemeral and---for authentication---static DH shares in a random-oracle--based key derivation, which contains sufficient context information to enable an elegant proof that is tight in the number of sessions and only loses a factor of the number of parties involved.
In practice, the number of parties in a key exchange is clearly much smaller than the number of sessions, meaning this approach provides a practical trade-off for reasonable efficiency based on theoretically sound parameters.
Indeed Cohn-Gordon et al.\ show that a loss in the number of parties is optimal for a certain class of protocols and underlying assumptions.

The protocols for which Cohn-Gordon et al.~\cite{C:CCGJJ19} establish their tighter security results provide implicit authentication, and their proof strategy heavily relies on this aspect when programming the random oracle.
While implicit authentication has recently seen adoption in popular new protocols like Signal~\cite{Signal} or Noise~\cite{Noise},
key exchange protocols in many other Internet security protocols aim at explicit authentication~\cite{C:BelRog93}, guaranteeing presence of a communication partner upon protocol acceptance.
While Cohn-Gordon et al.\ show that explicit authentication can be generically added to their implicitly authenticated key exchange protocols through a follow-up compiler step, this unfortunately reduces efficiency and means the result does not apply to deployed real-world protocols following a more direct path to explicit authentication.
Additionally, the tighter protocol designs rely on long-term Diffie--Hellman keys for implicit authentication, which are challenging to deploy in practice due to lacking support for according certificates in the web public-key infrastructure, which already barred adoption of the DH-based OPTLS design~\cite{EuroSP:KraWee16} in the recent TLS~1.3 standardization.
This leads to the question:
\begin{center}
	\emph{Can we achieve similarly tighter security proofs for deployed key exchange protocols,
	aiming at explicitly authenticated Diffie--Hellman key exchange?}
\end{center}


\subsubsection*{Contributions.}
In this work, we answer that question positively, providing the first tight (in the number of sessions) security proof for SIGMA-style (explicitly) authenticated key exchange protocols, covering both the basic protocol variant as well as \SIGMAI with added privacy for parties' identities.
We furthermore translate our results to the \SIGMA-based TLS~1.3 key exchange design, overcoming technical hurdles introduced by that protocol's significantly higher complexity.


\paragraph{Code-based security model and proofs.}
For our proofs, we provide detailed proof steps and reductions using the code-based game-playing framework of Bellare and Rogaway~\cite{EC:BelRog06}.
Our security model is similar to the one applied by Cohn-Gordon et al.~\cite{C:CCGJJ19},
%considering in particular compromises of long-term secrets and session keys (but not internal state or randomness),
but formalized also as a code-based game (in Section~\ref{sec:ake-model}) and stronger in that it captures explicit authentication and regular (``perfect'') forward secrecy (instead of only weak forward secrecy in~\cite{C:CCGJJ19}).


\paragraph{Tighter security proof of SIGMA(-I).}
In terms of tightness, our security proof of SIGMA(-I) in Sections~\ref{sec:sigma} and~\ref{sec:sigma-proof} provides a tight reduction to the strong Diffie--Hellman assumption~\cite{RSA:AbdBelRog01} in the used DH group, and to multi-user (mu) security definitions of the employed pseudorandom function (PRF), signature scheme, and MAC scheme.
Notably, while all these assumptions are in principle stronger than those used in the original proof for \SIGMA~\cite{C:CanKra02} (namely, the decisional Diffie--Hellman (DDH) assumption and single-user PRF, signature, and MAC security),
overall, we still gain in terms of efficiency when instantiating the protocol with theoretically sound parameters.
Most importantly, while being an interactive assumption compared to non-interactive DDH, no better algorithm for solving the strong DH problem (or generically the gap DH problem~\cite{PKC:OkaPoi01}) is known than to actually solve the computational DH (CDH) problem.
\fg{Add that in AGM, stDH reduces to Dlog (though with a non-tight \#queries factor)~\cite{C:FucKilLos18}.}
\hd{Verify Mihir's GGM conjecture that stDH is as hard as CDH and include}
One would hence in practice instantiate both DDH and strong DH assumptions with the same groups.
\SIGMA being a generic design, the PRF and MAC scheme can be instantiated with efficient mu-secure primitives (like, e.g., AMAC~\cite{EC:BelBerTes16}).
For any signature scheme, mu-security furthermore reduces to regular single-user (su) security with a factor of the number of users---which in our setting corresponds to the number of parties running the key exchange protocol.
This means that, here, we achieve the same level of tightness obtained by Cohn-Gordon et al.~\cite{C:CCGJJ19},
with only a loss in the number of parties, but not in the number of sessions.
Our results can be seen as confirming their insights (and translating them to the explicit authentication setting),
in that protocols binding enough context to their DH secrets in a (programmable) random-oracle--based key derivation
can achieve tight(er) security bounds in an appropriate security model.

\paragraph{Tighter security proof for the TLS~1.3 DH handshake.}
We exemplify the impact of extending the techniques from~\cite{C:CCGJJ19} to \SIGMA-style explicitly authenticated key exchange protocols
by translating our \SIGMAI result to the recently standardized newest version of the Transport Layer Security protocol, TLS~1.3~\cite{rfc8446} in Sections~\ref{sec:tls} and~\ref{sec:tls-proof}.
So far, the only reductionist security proofs known for the TLS~1.3 key exchange (the so-called handshake protocol) incur a highly non-tight security bound losing a quadratic factor in the number of sessions~\cite{CCS:DFGS15,EPRINT:DFGS16,EuroSP:FisGue17}.
While TLS~1.3 at its core follows the \SIGMAI design, its key schedule in particular is substantially more complicated, preventing the direct application of our results on \SIGMAI.
We are however able to give a carefully adapted proof which accommodates the more complex key schedule and other core variations in TLS~1.3's key exchange, achieving conceptually the same tightness results as for \SIGMAI.
Since TLS~1.3, in contrast to \SIGMA, fixes a specific set of components it deploys (esp.\ HMAC~\cite{C:BelCanKra96} as the KDF and MAC building block, for which we are not aware of a tight mu-security result),
our results do not reach the same tightness level possible for \SIGMA (instantiated with optimized components like~AMAC).
Nevertheless, our analysis still substantially improves over previous ones, as it incurs only a (linear) loss in the number of sessions for HMAC's EUF-CMA security, while the reduction to the (strong) Diffie--Hellman assumption is tight.
This way, our result provides a more theoretically sound confirmation of the practical scheme parameters deployed in TLS~1.3.


\subsubsection*{Discussion, limitations, and possible extensions.}

Like Cohn-Gordon et al.~\cite{C:CCGJJ19}, our key exchange security model considers exposure of long-term secrets and session keys, but does not allow revealing internal session state or randomness (as in the (e)CK model~\cite{EC:CanKra01,PROVSEC:LaMLauMit07}).
This is appropriate for protocols like TLS~1.3 not aiming at such levels of security.
The original \SIGMA proof~\cite{C:CanKra02} did establish security in the CK model~\cite{EC:CanKra01} allowing exposure of session state; in that sense our results are qualitatively weaker.
It is however unclear how a tight reduction for many challenged sessions down to a single DH problem instance could be obtained that at the same time allows to adaptively reveal internal session state (including DH exponents).

Our proof technique crucially relies on the ability to observe and program a random oracle that used for key derivation in the AKE protocol, borrowing from~\cite{C:CCGJJ19}.
Despite the random oracle model~\cite{CCS:BelRog93} having established itself as a versatile tool to reason about practical security, this is a noteworthy qualitative difference compared to the original \SIGMA proof~\cite{C:CanKra02} carried out in the standard model.
Interestingly, this distinction vanishes in comparison to the provable security results for the TLS~1.3 handshake protocol~\cite{CCS:DFGS15,EPRINT:DFGS16,EuroSP:FisGue17} that employ the PRF-ODH assumption~\cite{C:JKSS12,C:BFGJ17},
an interactive assumption which plausibly can only be instantiated in the random oracle model (from the strong DH assumption).

One reason for previous TLS~1.3 analyses requiring the PRF-ODH assumption is that they establish TLS~1.3's security in an multi-stage key exchange (MSKE) model~\cite{CCS:FisGue14}, proving security not only of the final session key, but also of intermediate handshake encryption keys and further secrets.
This enables, e.g., a clearer argument about the enhanced privacy obtained by encryption part of the key exchange in the style of~\SIGMAI.
While our proofs (for both \SIGMAI and TLS~1.3) establish security of the intermediate (handshake) encryption key, too,
we do not treat those keys as first-class keys available to the adversary (e.g., through revealing them) as in a multi-stage model.
We expect that our techniques similarly apply to a MSKE treatment, leaving this extension to possible future work.

%%%
%%% integrated the following already
%%%
% \begin{itemize}
% 	\item KE most widely deployed crypto ``primitive'' in the real world
% 	\hd{Really? I would have expected AEAD or digital signatures or something like that. If you can support it, this is a really good opener if only because it's surprising.}
% 	\fg{What I was aiming at was ``one of the most'', which is easier to write and argue.}
% 	\item often DH-based, for forward secrecy and efficiency
% 	\item many variants, but major Internet protocols (TLS, IKE ..) following SIGMA-style approach of explicit authentication via signatures and MACs
% 	\item proof techniques for KE often suffer from (highly) non-tight security reduction, including both number of sessions and number of users,
% 	e.g. TLS~1.2 quadratic \cite{C:JKSS12,C:KraPatWee13,C:BFKPSZ14}, TLS~1.3 quadratic \cite{CCS:DFGS15,EPRINT:DFGS16} (what about OPTLS? \fg{they don't say in their proof, but at least have to guess one session}), \cite{C:CanKra02} \SIGMA proof has \#sessions loss (not quadratic)
% 	\item while numbers of users is somewhat managable, number of sessions over a reasonable time span easily reaches orders of magnitutes that substantially affect security bounds of practically deployed protocols
% 	\item --- give some concrete numbers for the bounds
% 	\item this all rather seems to be artifacts of proof techniques, and the bounds are not met by any practical attacks or cryptanalysis
% 	\item indeed, C:19 work (and prior) \cite{C:GjoJag18,C:CCGJJ19} managed to overcome non-tight security results, esp. C:19~\cite{C:CCGJJ19} putting forward a nifty strategy for a simple and efficient, implicitly authenticated KE protocol
% 	\item their strategy heavily relies on RO programming and authentication being implicit; they provide explicit authentication only through a follow-up compiler step, which unfortunately reduces efficiency and means it does not apply to many real-world protocols
% 	\item in particular, it relies on long-term DH keys, which are hard to get certificates on in practice (cite something from the TLS standardization?) which was already seen in TLS standardization, where a signature-less OPTLS design~\cite{EuroSP:KraWee16} was discarded for deployment reasons
% 	
% 	
% 	\medskip
% 	
% 	\item in this work we ask: can we achieve similar tightness improvements for deployed DH designs, explicitly those doing DH with explicit, signature-based authentication?
% 	\item we answer positively, providing the first tight (in the number of sessions) security proof for SIGMA-style AKE protocols in a BR-like security model, covering both basic and \SIGMAI variant with identity hiding
% 	\item core modeling insights: all-real-or-random model, multiple test queries, in protocols important to bind session identifiers together with DH shares in (programmable) random oracle computation -- while given for SIGMA, will see how this works out for the more complex TLS~1.3 key schedule
% 	\item we provide a detailed proof using the code-based game-playing framework of Bellare and Rogaway~\cite{EC:BelRog06}
% 	\item we exemplify the impact of extending C:19~\cite{C:CCGJJ19} technique to SIGMA-style protocols by translating our results to the novel TLS~1.3 protocol, for which computational security results so far incured a highly-non-tight, quadratic security bound (DFGS15,DFGS16 -- what about OPTLS, do they give a concrete bound, do they maybe get away with only guessing one side?)
% 	\item we do not get to the same tightness level possible for SIGMA (based on components chosen for tight efficiency like AMAC), as TLS~1.3 uses HMAC for which we don't know (???) a tight mu-security result,
% 	\item but we still improve over previous bounds, esp. because the \#session loss is only incurred for HMAC EUF-CMA security, while the reduction to strong-DH is tight
% 	
% 	\medskip
% 	
% 	\item Limitations and future work:
% 	\begin{itemize}
% 		\item Do a BR-like model, not CK; in particular we don't (and do not know how to) treat exposure of state in our proofs.
% 		\item Only consider single session key derived, which is a simplification of TLS~1.3 doing a MSKE~\cite{CCS:FisGue14} which shines through already in the \SIGMAI idenity-hiding variant. We expect our results can be extended to MSKE setting.
% 		\item \hd{We use the identity-hiding variant but don't talk about identity-hiding security or specify any requirement for encryption security.}
% 		\item HMAC mu-security unclear (???), so variants of TLS deploying mu-optimized PRFs/MACs could be envisioned.
% 		\item As C:19~\cite{C:CCGJJ19}, our proof crucially relies on the RO technique~\cite{CCS:BelRog93}.
% 		Similarly OPTLS, and while DFGS15,DFGS16 apply the PRF-ODH assumption, there is strong indication that this assumption can only be instantiated via a random oracle~\cite{C:BFGJ17}.
% 	\end{itemize}
% \end{itemize}
% 
% \hd{This seems like a very strong storyline to me. The only thing I'd recommend adding is that there is an implicit tradeoff here: in order to get tighter security bounds, we're relying on a stronger, less standard DH assumption. It's reasonable to think that an adversary would have a higher advantage against strongDH than against plain DH. What's our argument that this tradeoff is actually beneficial? If I had to give such an argument, I'd say that this assumption is weaker than GapDH, which is a fairly well-studied problem that is conjectured to be hard. For this reason, we think that an adversary's strongDH advantage would still be very small for reasonable bounds, and the \# sessions factor has a greater impact. If we make this point, it probably becomes even more important to put StrongDH in the prelims.}
% 
% \fg{Good point, we should discuss the previous security results for both SIGMA and TLS~1.3 (in terms of tightness and assumptions).}

\newpage
\fi

\section{AKE Security Model\fullelse{}{ and Multi-User Building Blocks}}
\label{sec:ake-model}

We provide our results in a game-based key exchange model formalized in Figure~\ref{fig:AKE-security}, at its core following the seminal work by Bellare and Rogaway~\cite{C:BelRog93} considering an active network adversary that controls all communication (initiating sessions and determining their next inputs through $\Send$ queries) and is able to corrupt long-term secrets ($\RevLongTermKey$) as well as session keys ($\RevSessionKey$).
The adversary's goal is then to
(a) distinguish the established shared \emph{session key} in a ``fresh'' (not trivially compromised, captured through a $\Fresh$ predicate) session from a uniformly random key obtained through $\Test$ queries (breaking \emph{key secrecy}),
or (b) make a session accept without matching communication partner (breaking \emph{explicit authentication}).

Following Cohn-Gordon et al.~\cite{C:CCGJJ19}, we formalize our model in a real-or-random version (following Abdalla, Fouque, and Pointcheval~\cite{PKC:AbdFouPoi05} with added forward secrecy~\cite{SP:AbdBenMac15}) with \emph{many} $\Test$ queries which all answer with a real or uniformly random session key based on the \emph{same} random bit~$b$.
We focus on the security of the \emph{main} session key established.
While our proofs (for both \SIGMA and TLS~1.3) establish security of the intermediate encryption and MAC keys, too,
we do not treat them as first-class keys available to the adversary through $\Test$ and $\RevSessionKey$ queries.
We expect that our results extend to a multi-stage key exchange (MSKE~\cite{CCS:FisGue14}) treatment
and refer to the concurrent work by Diemert and Jager~\cite{JC:DieJag20} for tight results for TLS~1.3 in a MSKE model.

In contrast to the work by Cohn-Gordon et al.~\cite{C:CCGJJ19} and Diemert and Jager~\cite{JC:DieJag20}, our model additionally captures explicit authentication through the $\ExplicitAuth$ predicate in Figure~\ref{fig:AKE-security}, ensuring sessions with non-corrupted peer accept with an honest partner session.
We and~\cite{JC:DieJag20} further treat protocols where the communication partner's identity of a session may be unknown at the outset and only learned during the protocol execution; this setting of ``post-specified peers''~\cite{C:CanKra02} particularly applies to the \SIGMA protocol family~\cite{C:Krawczyk03} as well as TLS~1.3~\cite{rfc8446}.


\subsection{Key Exchange Protocols}

\label{sec:ake-syntax}
We begin by formalizing the syntax of key exchange protocols.
A key exchange protocol $\KE$ consists of three algorithms~$(\KEKGen\cab \KEActivate\cab \KERun)$
and an associated key space~$\KEkeyspace$ (where most commonly $\KEkeyspace = \bits^n$ for some $n \in \NN$).
The key generation algorithm~$\KEKGen() \tor (\pk\cab \sk)$ generates new long-term public/secret key pairs.
In the security model, we will associate key pairs to distinct \emph{users} (or \emph{parties}) with some identity~$u \in \NN$ running the protocol,
and log the public long-term keys associated with each user identity in a list $\peerpk$.
(The adversary will be in control of initializing new users, identified by an increasing counter, and we assume it only references existing user identities.)
The activation algorithm~$\KEActivate(\id, \sk, \peerid, \peerpk, \role) \tor (\st', m')$ initiates a new session for a given user identity~$\id$ (and associated long-term secret key~$\sk$) acting in a given role~$\role \in \{\initiator\cab \responder\}$ and aiming to communicate with some peer user identity~$\peerid$.
$\KEActivate$ also takes as input the list $\peerpk$ of all users' public keys; protocols may use this list to look up their own and their peers' public keys. 
We provide the entire list instead of just the user's and peers' public keys to accommodate protocols with post-specified peer. 
These protocols may leave $\peerid$ unspecified at the time of session activation; when the peer identity is set at some later point, the list can be used to find the corresponding long-term key. 
Activation outputs a session state and (if $\role = \initiator$) first protocol message~$m'$, and will be invoked in the security model to create a new session~$\pi_u^i$ at a user~$u$ (where the label~$i$ distinguishes different sessions of the same user).
Finally, $\KERun(\id, \sk, \st,\peerpk, m) \tor (\st', m')$ delivers the next incoming key exchange message $m$ to the session of user~$\id$ with secret key~$\sk$ and state $\st$, resulting in an updated state~$\st'$ and a response message~$m'$. Like $\KEActivate$, it relies on the list $\peerpk$ to look up its own and its peer's long-term keys. 

The state of each session in a key exchange protocol contains at least the following variables, beyond possibly further, protocol-specific information:
\begin{description}
	\setlength{\itemsep}{0.25em} % little more space
	
	\item[$\peerid \in \NN$.]
	Reflects the (intended) partner identity of the session;
	\fullelse{in protocols with post-specified peers this is learned and set (once) by the session during the protocol execution.}
	{if post-specified, this is learned and set (once) during protocol execution.}
	
	\item[$\role \in \{\initiator,\responder\}$.]
	The session's role, determined upon activation.
	
	\item[$\status \in \{\running,\accepted,\rejected\}$.]
	The session's status;
	initially $\status = \running$,
	a session accepts when it switches to $\status = \accepted$ (once).
	
	\item[$\skey \in \KEkeyspace$.]
	The derived session key (in\fullonly{ the protocol-specific key space~}$\KEkeyspace$), set upon acceptance.
	
	\item[$\sid$.]
	The session identifier used to define partnered session in the security model;
	initially unset, $\sid$ is determined (once) during protocol execution.
\end{description}


\subsection{Key Exchange Security}


We formalize our key exchange security game~$G^{\KESEC}_{\KE,\advA}$ in Figure~\ref{fig:AKE-security},
based on the concepts introduced above in Figure~\ref{fig:AKE-security}
and following the framework for code-based game playing by Bellare and Rogaway~\cite{EC:BelRog06}.
After initializing the game, %generating public/secret keys for~$n$ users,
the adversary~$\advA$ is given access to queries
$\NewUser$ (generating a new user's public/secret key pair),
$\Send$ (controlling activation and message processing of sessions),
$\RevSessionKey$ (revealing session keys),
$\RevLongTermKey$ (corrupting user's long-term secrets),
and~$\Test$ (providing challenge real-or-random session keys),
as well as a $\Finalize$ query to which it will submit its guess~$b'$ for the challenge bit~$b$, ending the game.
\begin{figure}[tp]
	\begin{minipage}[t]{0.5\textwidth}
		\NewExperiment[$G^{\KESEC}_{\KE,\advA}$]
		
		\begin{oracle}{$\Initialize$}
			\item $\time \gets 0$; $\users \gets 0$
			\item $b \getsr \bits$
		\end{oracle}
		
		\ExptSepSpace
		
		\begin{oracle}{$\NewUser$}
			\item $\users \gets \users + 1$
			\item $(pk_\users, sk_\users) \getsr \KEKGen()$
			\item $\revltk_\users \gets \infty$
			\item $\peerpk[\users] \gets \pk_{\users}$
			\item return $pk_\users$
		\end{oracle}
		
		\ExptSepSpace
		
		\begin{oracle}{$\Send(u, i, m)$}
			\item if $\pi_u^i = \bot$ then
			\item \hindent $(\peerid,\role) \gets m$
			\item \hindent $(\pi_u^i, m') \getsr \KEActivate(u\cab \sk_u\cab \peerid\cab \peerpk\cab \role)$
			\item \hindent $\pi_u^i.\taccepted \gets 0$
			
			\item else
			\item \hindent $(\pi_u^i, m') \getsr \KERun(u, \sk_u, \pi_u^i, \peerpk, m)$
			
			\item if $\pi_u^i.\status = \accepted$ then
			\item \hindent $\time \gets \time + 1$
			\item \hindent $\pi_u^i.\taccepted \gets \time$
			
			\item return $m'$
		\end{oracle}
		
		\ExptSepSpace
		
		\begin{oracle}{$\RevSessionKey(u, i)$}
			\item if $\pi_u^i = \bot$ or $\pi_u^i.\status \neq \accepted$ then
			\item \hindent return $\bot$
			
			\item $\pi_u^i.\revealed \gets \true$
			\item return $\pi_u^i.\skey$
		\end{oracle}
		
		\ExptSepSpace
		
		\begin{oracle}{$\RevLongTermKey(u)$}
			\item $\time \gets \time + 1$
			\item $\revltk_u \gets \time$
			\item return $sk_u$
		\end{oracle}
		
		\ExptSepSpace
		
		\begin{oracle}{$\Test(u, i)$}
			\item if $\pi_u^i = \bot$ or $\pi_u^i.\status \neq \accepted$ or $\pi_u^i.\tested$ then
			\item \hindent return $\bot$
			
			\item $\pi_u^i.\tested \gets \true$
			
			\item $T \gets T \cup \{\pi_u^i\}$
			\item $k_0 \gets \pi_u^i.\skey$
			\item $k_1 \sample \KEkeyspace$
			\item return $k_b$
		\end{oracle}
	\end{minipage}
	\begin{minipage}[t]{0.49\textwidth}
		\ExptSepSpace
		
		\begin{oracle}{$\Finalize(b')$}
			\item if $\neg \Sound$ then
			\iffull\item \hindent\fi return $1$
			
			\item if $\neg \ExplicitAuth$ then
			\iffull\item \hindent\fi return $1$
			
			\item if $\neg \Fresh$ then
			\iffull\item \hindent\fi $b' \gets 0$
			
			\item return $[[b = b']]$
		\end{oracle}
		
		\ExptSepSpace
		
		\begin{algorithm}{$\Sound$}
			%%% no triple sid match
			\item if $\exists$ distinct $\pi_u^i$, $\pi_v^j$, $\pi_w^k$ with $\pi_u^i.\sid = \pi_v^j.\sid = \pi_w^k.\sid$ then
			\comment{no triple sid match}
			\item \hindent return $\false$
			
			%%% same sid ==> same key
			\item if $\exists \pi_u^i, \pi_v^j$ with \newline
				\null\hindent $\pi_u^i.\status = \pi_v^j.\status = \accepted$ \newline
				\null\hindent and $\pi_u^i.\sid = \pi_v^j.\sid$ \newline
				\null\hindent and $\pi_u^i.\peerid = v$ and $\pi_v^j.\peerid = u$ \newline
				\null\hindent and $\pi_u^i.\role \neq \pi_v^j.\role$, but $\pi_u^i.\skey \neq \pi_v^j.\skey$ then  \comment{partnering implies same key}
			\item \hindent return $\false$
			
			\item return $\true$
		\end{algorithm}
		
		\ExptSepSpace
		
		\begin{algorithm}{$\ExplicitAuth$}
			\item return \newline
			\null \hindent  $\forall \pi_u^i : \pi_u^i.\status = \accepted$ \newline 
			\null \hindent \hindent \hindent \hindent and $\pi_u^i.\taccepted < \revltk_{\pi_u^i.\peerid}$ \newline \comment{all sessions accepting with a non-corrupted peer \dots} \newline
			\null\hindent \hindent  $\implies \exists \pi_v^j : \pi_u^i.\peerid = v$ \newline
			\null \hindent \hindent \hphantom{$\implies$} and $\pi_u^i.\sid = \pi_v^j.\sid$ \newline
			\null \hindent \hindent \hphantom{$\implies$} and $\pi_u^i.\role \neq \pi_v^j.\role$
			\newline 	\comment{\dots\ have a partnered session \dots} \newline
			\null \hindent \hphantom{$\implies$} and $(\pi_v^j.\status = \accepted \!\implies\! \pi_v^j.\peerid = u)$
				\comment{\dots\ agreeing on the peerid (upon acceptance)}
		\end{algorithm}
		
		\ExptSepSpace
		
		\begin{algorithm}{$\Fresh$}
			\item for each $\pi_u^i \in T$
			\item \hindent if $\pi_u^i.\revealed$ then
			\item \hindent \hindent return $\false$
			\comment{tested session may not be revealed}
			
			\item \hindent if $\exists \pi_v^j \neq \pi_u^i : \pi_v^j.sid = \pi_u^i.sid$ 
			\newline \null \hindent \hindent and ($\pi_v^j.\tested$ or $\pi_v^j.\revealed$) then
			\item \hindent \hindent return $\false$
			\comment{tested session's partnered session may not be tested or revealed}
			
			\item \hindent if $\revltk_{\pi_u^i.\peerid} < \pi_u^i.\taccepted$ then
			\item \hindent \hindent return $\false$
			\comment{tested session's peer may not be corrupted prior to acceptance}
			
			\item return $\true$
		\end{algorithm}
	\end{minipage}
	
	\caption{%
		Key exchange security game.
	}
	\label{fig:AKE-security}
\end{figure}

The game~$G^{\KESEC}_{\KE,\advA}$ then (in $\Finalize$) determines whether~$\advA$ was successful through the following three predicates,
formalized in pseudocode in Figure~\ref{fig:AKE-security}:
\iffull
\begin{description}
	\setlength{\itemsep}{0.5em} % little more space
	
	\item[$\Sound$.]
	The soundness predicate~$\Sound$ checks that (a) no three session identifiers collide (hence the session identifier properly serves to identify two partnered sessions).
	Furthermore, it ensures that (b) accepted sessions with the same session identifier, agreeing partner identities, and distinct roles derive the same session key.
	\iffull
	The adversary breaks soundness if it violates either of these properties.
	\fi
	
	\item[$\ExplicitAuth$.]
	The predicate~$\ExplicitAuth$ captures explicit authentication in that it requires that for any session of some user~$\id$ that accepted while its partner~$\peerid$ was not corrupted (captured through logging relative acceptance time~$\taccepted$ and long-term reveal time~$\revltk_{\peerid}$) has
	(a) a partnered session run by the intended peer identity and in an opposite role,
	and (b) if that partnered session accepts, it will do so with peer identity~$\id$.
	\iffull
	The adversary breaks explicit authentication if this predicate evaluates to false.
	\fi
	
	\item[$\Fresh$.]
	Finally, to capture key secrecy, we have to restrict the adversary to testing only so-called \emph{fresh} sessions in order to exclude trivial attacks, which the freshness predicate~$\Fresh$ ensures.
	A tested session is non-fresh, if
	(a) its session key has been revealed (in which case~$\advA$ knows the real key),
	(b) its partnered session (through~$\sid$) has been revealed or tested (in which case~$\advA$ knows the real key or may see two different random keys),
	or (c) its intended peer identity was compromised prior to accepting (in which case~$\advA$ may fully control the communication partner).
	\iffull
	If the adversary violates freshness, we invalidate its guess by overwriting~$b' \gets 0$.
	\fi
\end{description}
\else
$\Sound$ ensures session identifiers are set in a sound manner (non-colliding, ensuring agreement on session keys).
$\ExplicitAuth$ encodes explicit authentication, requiring that accepted sessions agree on the intended peer (if non-corrupted).
Finally, to capture key secrecy, we have to restrict the adversary to testing only \emph{fresh} (i.e., not trivially compromised) sessions in order to exclude trivial attacks; this is ensured through~$\Fresh$.
\fi

We call two distinct sessions~$\pi_u^i$ and~$\pi_v^j$ \emph{partnered} if $\pi_u^i.\sid = \pi_v^j.\sid$.
We refer to sessions generated by $\KEActivate$ (i.e., controlled by the game) as \emph{honest} sessions
to reflect that their behavior is determined honestly by the game and not the adversary.
The long-term key of an honest session may still be corrupted, or its session key may be revealed without affecting this notion of ``honesty''.

%%% old, non-parameterized version
% {\color{gray}
% \begin{definition}[Key exchange security]
% 	Let $\KE$ be a key exchange protocol
% 	and $\advA$ an adversary interacting in the key exchange security game~$G^{\KESEC}_{\KE,\advA}$ defined in Figure~\ref{fig:AKE-security}.
% 	We call $\KE$ secure in our model if the advantage
% 	\[
% 		\Adv^{\KESEC}_{\KE,\advA} := 2 \cdot \Pr \left[ \Gm^{\KESEC}_{\KE,\advA} \Rightarrow 1 \right] - 1
% 	\]
% 	of winning the game is small.
% \end{definition}
% }

\begin{definition}[Key exchange security]
	\label{def:KE-security}
	Let $\KE$ be a key exchange protocol and~$G^{\KESEC}_{\KE,\advA}$ be the key exchange security game defined in Figure~\ref{fig:AKE-security}.
	We define
	\[
		\Adv^{\KESEC}_{\KE}(t, \qNewUser, \qSend, \qRevSessionKey, \qRevLongTermKey, \qTest) := 2 \cdot \max_\advA \Pr \left[ \Gm^{\KESEC}_{\KE,\advA} \Rightarrow 1 \right] - 1,
	\]
	where the maximum is taken over all adversaries, denoted \emph{$(t, \qNewUser, \qSend, \qRevSessionKey, \qRevLongTermKey, \qTest)$-$\KESEC$-adversaries}, running in time at most~$t$ and making at most $\qNewUser$, $\qSend$, $\qRevSessionKey$, $\qRevLongTermKey$, resp.\ $\qTest$ queries to their oracles $\NewUser$, $\Send$, $\RevSessionKey$, $\RevLongTermKey$, resp.\ $\Test$.
\end{definition}


\subsection{Security Properties}

Let us briefly revisit some core security properties captured in our key exchange security model.

First, we capture regular \emph{key secrecy} of the main session key through $\Test$ queries, incorporating \emph{forward secrecy} (sometimes called ``perfect'' forward secrecy) by allowing the adversary to corrupt any user as long as all tested sessions accept prior to corrupting their respective intended peer.
This strengthens our model compared to that of Cohn-Gordon et al.~\cite{C:CCGJJ19} which only captures weak forward secrecy where the adversary has to be passive in sessions where it corrupts long-term secrets.
Diemert and Jager~\cite{JC:DieJag20} additionally treat the security of intermediate keys and further secrets beyond the main session key in a multi-stage approach~\cite{CCS:FisGue14}, but without capturing explicit authentication.

Our model encodes \emph{explicit authentication} (via $\ExplicitAuth$), a strengthening compared to the implicit-authentication model in~\cite{C:CCGJJ19}.

Like~\cite{C:CCGJJ19,JC:DieJag20}, our model captures \emph{key-compromise impersonation} attacks by allowing the session owner's secret key of tested sessions to be corrupted at any point in time.
Similarly, we do \emph{not} capture \emph{session-state or randomness reveals}~\cite{EC:CanKra01,PROVSEC:LaMLauMit07} or \emph{post-compromise security}~\cite{CSF:CohCreGar16}.



% \TODO{%
% To include in intro of model/discussion:
% \begin{itemize}
% 	\item BR93-like model~\cite{C:BelRog93} in terms of active adversary, able to corrupt long-term secrets and session keys; we don't consider session state or randomness reveal like in CK/eCK models~\cite{EC:CanKra01,PROVSEC:LaMLauMit07}
% 	\item like \cite{C:CCGJJ19} we adopt a real-or-random definition from~\cite{PKC:AbdFouPoi05} with forward secrecy~\cite{SP:AbdBenMac15}, allowing multiple $\Test$ queries all answered based on the same challenge bit~$b$
% 	\item extending the work of Cohn-Gordon~\cite{C:CCGJJ19} that was restricted to implicitly authenticated key exchange, our model however consideres \emph{explicitly authenticated} key exchange protocols
% 	\item (similar to \cite{C:CCGJJ19} paper:) describe concepts of sessions and their state (variables)
% 	\item KE definition
% 	\item define partnering / matching
% 	\item describe attacker model and oracles in Figure~\ref{fig:AKE-security}
% 	\item KE security definition
% 	\item discussion of the model: repeat real-or-random, multiple test queries, describe purpose of $\Sound$, $\ExplicitAuth$, and $\Fresh$; we do regular (``sometimes called `perfect' '') forward secrecy in constrast to only weak fs in \cite{C:CCGJJ19}; like them also capture key compromise impersonation 
% \end{itemize}
% }



\section{Components}
\label{sec:components}

\subsection{Multi-User Unforgeability with Adaptive Corruptions}

\begin{figure}[t]
	\centering
	
	%%% MACs mu-EUF-CMA
	\begin{minipage}[t]{0.2\textwidth}
		\NewExperiment[$\Gm^{\muEUFCMA}_{\abstractMACScheme,\advA}$]
		
		\begin{oracle}{$\Initialize$}
			\item $Q \gets \emptyset$
			\item $\setfont{C} \gets \emptyset$
			\item $u \gets 0$
		\end{oracle}
		
		\ExptSepSpace
		
			\begin{oracle}{$\Corrupt(i)$}
			\item $\setfont{C} \gets \setfont{C} \cup \{i\}$
			\item return $K_i$
		\end{oracle}
		\ExptSepSpace
	
			\end{minipage}
	\begin{minipage}[t]{0.3\textwidth}
		\vspace*{\iffull0.4cm\else0cm\fi}
		\begin{oracle}{$\New$}
			\item $u \gets u + 1$
			\item $K_u \getsr \abstractMACKGen()$
			\item[]
		\end{oracle}
		\ExptSepSpace
		\begin{oracle}{$\OTag(i, m)$}
			\item $\tau \getsr \abstractMACTag(K_i, m)$
			\item $Q \gets Q \cup \{(i, m)\}$
			\item return $\tau$
		\end{oracle}
		
		
		\ExptSepSpace
			\end{minipage}
	\begin{minipage}[t]{0.47\textwidth}
		\vspace*{.4cm}
		
		\begin{oracle}{$\OVerify(i, m, \tau)$}
			\item $d \gets \abstractMACVerify(K_i, m, \tau)$
			\item return $d$
			\item[]
		\end{oracle}
		\ExptSepSpace
		
		\begin{oracle}{$\Finalize(i^*, m^*, \tau^*)$}
			\item $d^* \gets \abstractMACVerify(K_{i^*}, m^*, \tau^*)$
			\item return $[[ d^* = 1 \land i^* \notin \setfont{C} \land (i^*\!, m^*) \notin Q ]]$
		\end{oracle}
	\end{minipage}
	
	\caption{%
		Multi-user existential unforgeability ($\muEUFCMA$) of MAC schemes.
	}
	\label{fig:muEUFCMA}
\end{figure}

\begin{definition}[MAC $\muEUFCMA$ security]
	\label{def:MAC-muEUFCMA}
	Let $\abstractMACScheme$ be a MAC scheme
	and $\Gm^{\muEUFCMA}_{\abstractMACScheme,\advA}$ be the game for MAC multi-user existential unforgeability under chosen-message attacks with adaptive corruptions defined as in Figure~\ref{fig:muEUFCMA}.
	We define
	\[
		\Adv^{\muEUFCMA}_{\abstractMACScheme}(t\cab \qNew\cab \qTag\cab\qTagU\cab \qVerify\cab \qVerifyU\cab \qCorrupt) := \max_\advA \Pr \left[ \Gm^{\muEUFCMA}_{\abstractMACScheme,\advA} \Rightarrow 1 \right]
	\]
	where the maximum is taken over all adversaries, denoted \emph{$(t\cab \qNew\cab \qTag\cab\qTagU\cab \qVerify\cab\qVerifyU\cab \qCorrupt)$-$\muEUFCMA$-adversaries}, running in time at most~$t$ and making at most $\qNew$, $\qTag$, $\qVerify$, resp.\ $\qCorrupt$ queries to their $\New$, $\OTag$, $\OVerify$, resp.\ $\Corrupt$ oracle, and making at most $\qTagU$ queries $\OTag(i, \cdot)$, resp.\ $\qVerifyU$ queries $\OVerify(i, \cdot)$ for any user~$i$.
	
	\fg{Do we want to call it $\muEUFCMA$ or $\muEUFCMA^{\mathsf{corr}}$?}
\end{definition}


\iffalse
 %% 2021-07-26: This is long replaced by our indifferentiability approach.
\subsection{Key Schedule Function Security}

\fg{This is an attempt at carving out an overall function representing the TLS key schedule, in order to define some form of ``PRF-ODH--like''~\cite{C:JKSS12,C:BFGJ17} for it.}

\begin{itemize}
	\item Describe the TLS key schedule as a function
	\[
		\TLSKDF(\psk, \dhe, DHx, DHy, label) := K
	\]
	
	\item Instances would be
	\begin{align*}
		\ets \assign &\TLSKDF(\psk, \bot, \bot, \bot, \labelETS \concat \hash(\CH))\\
		\cats \assign &\TLSKDF(\psk, \dhe, g^\clientExponent, g^\serverExponent, \labelClientATS \concat \hash(\CH \concat \dotsb \concat \SF))\\
	\end{align*}
	
	\item Maybe ``$label$'' needs to be futher split into label and context.
\end{itemize}


\begin{figure}[t]
	\begin{minipage}[t]{3.5cm}
		\NewExperiment[$mu\old{PRF}^\RO_{\Func^\RO, \advA}$]
		
		\begin{oracle*}{$\Initialize()$}
			\item $b \sample \bits$
			\item $u \gets 0$
			\item $\inconsistent \gets \false$
		\end{oracle*}
		
		\begin{oracle*}{$\New()$}
			\item $u \gets u + 1$
			\item $K_u \sample \bits^k$
			\item $c_u \gets \false$
			\item return $u$
		\end{oracle*}
		
		\begin{oracle}{$\RO(x)$}
			\item if $R[x] = \bot$ then
			\item \hindent $R[x] \sample \bits^r$
			\item return $R[x]$
		\end{oracle}
	\end{minipage}
	%
	\begin{minipage}[t]{5cm}
		\begin{oracle*}{$\Fn(i, x)$}
			\item if $i < u$ or $\corr_i$ then return $\bot$
			
			\item if $F[i, x] = \bot$ then
			
			\item \hindent if $b = 1$ then
			\item \hindent \hindent $F[i, x] \gets \Func^{\RO(\cdot)}(K_i, x)$
			
			\item \hindent else
			\item \hindent \hindent $F[i, x] \gets \Func^{\SimRO(i, \cdot)}(K_i, x)$
			
			\item return $F[i, x]$
		\end{oracle*}
		
		\begin{oracle*}{$\Corrupt(i)$}
			\item if $\corr_i$ then return $\bot$
			\item $\corr_i \gets \true$
			\item if $b = 0$
			\item \hindent For all $(x, y) \in S[i]$ do
			\item \hindent \hindent if $R[x] \neq \bot$ then
			\item \hindent \hindent \hindent $\inconsistent \gets \true$
			\item \hindent \hindent $R[x] \gets y$ \comment{program $\RO$; if $R[x]$ was already set, we're now inconsistent} 
			\item return $K_i$
		\end{oracle*}
		
		\begin{oracle}{$\Finalize(b')$}
			\item return $[[b = b' \text{ or } \inconsistent]]$
		\end{oracle}
	\end{minipage}
	%
	\begin{minipage}[t]{3.5cm}
		\begin{oracle}{$\SimRO(i, x)$}
% 			\item if $S[i, x] \neq \bot$\\ or $\RO[x] \neq \bot$ then
% 			\item \hindent $\inconsistent \gets \true$ \comment{repeated $\RO$ queries, we'll be inconsistent from now on}
			\item $y \sample \bits^r$
			\item $S[i] \gets S[i] \cup \{(x,y)\}$
			\item return $y$
			\TODO{This approach doesn't work, $\SimRO$ has to be consistent at least for some queries, think of the extraction of $\psk$ in TLS, which will be a repeated query by $\TLSKDF$ to $\SimRO$. It's the \emph{final output values} that $\SimRO$ should be assigning independent random values, but unclear which RO step that is, generically.}
			\fg{We can probably fix this by making SimRO consistent and requiring that the last key derivation step is injective, so that the last output is indeed independent. (See below.)}
		\end{oracle}
	\end{minipage}
	
	\caption{%
		Key schedule security -- PSK-only version.
		$\SimRO$ is not an oracle and cannot be invoked by $\advA$ directly.
	}
	\label{fig:key-schedule-security_PSK}
\end{figure}

Let's start with a security notion focusing only on the $\psk$ secret part, leaving out the $\dhe, DHx, DHy$ components.
We want some form of ``multi-user PRF security with adaptive corruptions'', however for the latter we need [can we formalize this] some form of programmable random oracle component.
An attempt is formalized in Figure~\ref{fig:key-schedule-security_PSK}, for a function $\Func^\RO$ with $\psk \in \bits^k$ based on a random oracle with output in $\bits^r$.

The intuition we're trying to capture here is that an adversary cannot distinguish the real-world execution of~$\Func^\RO$ under some user key~$K_i$ (using a random oracle, which is ``programmed'' accordingly on any queries made by~$\Func$) from the random-world execution of $\Func$ under~$K_i$ with a \emph{simulating random oracle}~$\SimRO$, which simply samples independent random outputs but does not ``program'' them into~$\RO$,
\emph{as long as} the secret key~$K_i$ is not corrupted.
Once $K_i$ is corrupted, all simulated random oracle queries to $\SimRO$ for evaluations under~$K_i$ are retroactively programmed into~$\RO$ to ensure consistency with any future $\RO$ queries that $\advA$ can now make itself due to knowledge of~$K_i$.
This programming may introduce inconsistencies:
if $\advA$ managed to query (guess) any of the simulated queries to~$\RO$ \emph{beforehand}, we declare it successful by setting the $\inconsistent$ flag.
Note that this in particular covers cases where $\SimRO$ would be called on repeated inputs~$x$ (for the same or different user keys~$K_i$).%
\footnote{\fg{This doesn't affect the PSK handshake, but in the DHE-only handshake, the early extracted values~$\es$, $\des$ are constants, so that would need some different treatment (e.g., just saying that~$\des$ is a fixed constant which is not computed through the random oracle.}}

Why is $\SimRO$ drawing independent random outputs, not even ensuring consistency with itself?
We will be interested in the function~$\TLSKDF$ which can be formalized as $\TLSKDF(\dots) := \RO(\TLSKDF'(\dots))$, i.e., the final output is the direct output of a random oracle call.
Together with $\SimRO$ drawing independently at random, this means that all outputs of $\TLSKDF$ (i.e., responses of the~$\Fn$ oracle) under uncorrupted keys, in the random world, are independent random values (however consistent on the $\TLSKDF$ level, as the $\Fn$ oracle caches prior values $F[i,x]$ computed on $\Fn(i,x)$.
\fg{We'll stipulate that $\TLSKDF'(\psk, \dots)$ is injective, i.e., with $\psk$ fixed, two inputs will never lead to the same final derivation step due to distinct labels.
IMPORTANT: The final step is the second level of the key schedule, where all keys have distinct labels. From there, we can do further derivations individually (e.g., on finished keys, handshake traffic keys, etc.).}

\fg{Core issue remaining:
Even if we do all this and replace all second-level TLS keys via the $\TLSKDF$ security with random ones, we still need to pull in the RO programming into the main key exchange game,
i.e., the SimRO and RO parts as well as programming upon PSK corruption need to be done in the KE game.
This kind of breaks with the modularity aimed at by this security notion in the first place, leaving it unclear what the overall value is.}
\fi

% \iffull
% \section{Proof of Theorem~\ref{thm-ggm-bound}}
% \else
\section{Proof of the Strong Diffie--Hellman GGM Bound (Theorem~\ref{thm-ggm-bound})}

% \fi
\label{apx:strongDHproof}
\begin{proof}
We begin by giving a code-based game for the strong Diffie--Hellman problem in the generic group model. First, we establish some preliminaries, using the setting and notation of Bellare and Dai~\cite{INDOCRYPT:BelDai20}.  Let $\group$ be an arbitrary set of strings with prime order $p$, and let $\E: \ZZ_p \to \group$ be a bijection, called the encoding function. For any two strings $A, B \in \group$, we define the operation $A \mathop{\OP_\E} B = \E(\E^{-1}(A) + \E^{-1}(B) \mod p)$. The set $\group$ is a group with respect to this operation, and it is isomorphic to $\ZZ_p$. Therefore, $\group$ has the identity $\E(0)$, and it is generated by $\E(1)$. 

In the generic group model, we wish for the adversary to compute group operations only through an oracle $\OP$. We accomplish this by picking the encoding function $\E$ at random and keeping it secret; then providing oracle access to $\OP_\E$ through $\OP$. In this model, we can give a sequence of games bounding the advantage of any adversary $\advA$ that makes $t$ queries to the $\OP$ oracle and $q$ queries to the $\stDH$ oracle.

\startproof{GGM}
\proofngame[ggm-start]
This first game formalizes the strong Diffie--Hellman problem in the generic group model. Note that for any $a \in \ZZ_p$, $a$ is the discrete logarithm of the group element $\E(a)$. 

It follows that 
\[
	\Adv^{\strongDH}_{\group}(t, \qstDH) = \Pr[\curGm* \Rightarrow 1].
\]
\begin{figure}[tp]
	\begin{minipage}[t]{0.49\textwidth}
		\NewExperiment[$\lblGm{ggm-start}$]
			
			\begin{oracle}{$\Initialize()$}
				\item $p \gets |\group|$; $\E \getsr \Bijections(\ZZ_p, \group)$
				\item $\one \gets \E(0)$; $\generator\gets \E(1)$
				\item $x,y \getsr \ZZ_p^*$; $\X \gets \E(\x)$; $\Y\gets E(\y)$
				\item $\GL \gets \{\one,\generator,\x,\y\}$
				\item return $(\one,\generator,\x,\y)$
			\end{oracle}
			\ExptSepSpace
			\begin{oracle}{$\OP(A,B,\sgn)$}
				\item if $A \not \in \GL$ or $B \not \in \GL$ then return $\bot$
				\item $c \gets \E^{-1}(A) \sgn \E^{-1}(B) \mod p$
				\item $C \gets \E(c)$; $\GL \gets \GL \cup \{C\}$
				\item return $C$
			\end{oracle}
			\end{minipage}
		\begin{minipage}[t]{0.49\textwidth}
		\ExptSepSpace
			\begin{oracle}{$\stDH(A,B)$}
				\item if $A \not \in \GL$ or $B \not \in \GL$ then return $\bot$
				\item $z \gets x \cdot E^{-1}(A) \mod p$
				\item $Z \gets E(z)$
				\item return $[[Z=B]]$
			\end{oracle}
			\ExptSepSpace
			\begin{oracle}{$\Finalize(Z)$}
				\item if $Z \not \in \GL$ then return $\false$
				\item $z \gets \x \cdot \y \mod p$; return $[[Z = \E(z)]]$
			\end{oracle}
		\end{minipage}
		
		\caption[]{%
			Game $\lblGm{ggm-start}$ of the $\stDH$ proof.
		}
		\label{fig:GGM-proof:game:start}
	\end{figure}
\proofngame[ggm-vector]
In Game $\curGm*$, we change the internal notation of the game.
First, for clarity and without loss of generality, we assume the adversary queries its $\OP$ and $\stDH$ oracles only on valid inputs (meaning their inputs are valid group elements in $\GL$).
Instead of representing each element of~$\group$ with an element of~$\ZZ_p$, we use a vector over $\ZZ_p^3$.
We define the basis vectors $\vec{e_1} := (1,0,0)$, $\vec{e_2} := (0,1,0)$, and $\vec{e_3} := (0,0,1)$.
We map  $\ZZ_p^{3}$ to $\ZZ_p$ by taking the inner product with the vector $(1,x,y)$.
(Effectively, we are representing each element of $\ZZ_p$ as a linear combination modulo $p$ of $1$, $x$, and $y$.)
We cache the map from $\ZZ_p^3$ to $\group$ induced by this transformation in a table $\TV$ and its inverse map in a table $\TI$.
% From this point forward, we will assume without loss ofG generality that all queries to the $\OP$ and $\stDH$ oracles are valid (meaning their inputs are valid group elements in $\GL$).

Although one element of $\group$ may now have multiple representations, the bilinearity of the inner product ensures that the view of the adversary is not changed, and $\Pr[\curGm* ] = \Pr[\prevGm* ].$

\begin{figure}[tp]
	\begin{minipage}[t]{0.49\textwidth}
		\NewExperiment[$\lblGm{ggm-vector}$]
		
		\ExptSepSpace
		
		\begin{oracle}{$\Initialize()$}
			\item $p \gets |\group|$; $\E \getsr \Bijections(\ZZ_p, \group)$
			\item $k\gets 0$; $\one \gets \VE(\vec{0})$; $\generator\gets \VE(\vec{e_1})$
			\item $\x,\y \getsr \ZZ_p^*$; $\vec{x} \gets {\one,\x,\y}$
			\item $\X \gets \VE(\vec{e_2})$; $\Y\gets \VE(\vec{e_3})$
			\item return $(\one,\generator,\x,\y)$
		\end{oracle}
		\ExptSepSpace
		\begin{oracle}{$\OP(A,B,\sgn)$}
			\item $\vec{c} \gets \VE^{-1}(A) \sgn \VE^{-1}(B) \mod p$
			\item $C \gets \VE(\vec{c})$; return $C$
		\end{oracle}
	\ExptSepSpace
		\begin{algorithm}{$\VE(\vec{t})$}
			\item if $\TV[\vec{t}] \neq \bot$ then return $\TV[\vec{t}]$
			\item $k \gets k+1$; $\vec{t_k} \gets \vec{t}$
			\item $v \gets \langle \vec{t},\vec{x} \rangle$; $C \gets E(v)$;  $\GL \gets \GL \cup \{C\}$
			\item  $\TV[\vec{t}] \gets C$; $\TI[C] \gets \vec{t}$
			\item return $\TV[\vec{t}]$
		\end{algorithm}
	\end{minipage}
	\begin{minipage}[t]{0.49\textwidth}
		\ExptSepSpace
		\begin{oracle}{$\stDH(A,B)$}
			\item $\vec{a} \gets \VE^{-1}(A)$; $\vec{b} \gets \VE^{-1}(B)$
			\item return $[[\VE(x \vec{a}) = B]]$
		\end{oracle}
		\ExptSepSpace
		\begin{oracle}{$\Finalize(Z)$}
			\item return $[[\VE(x \vec{e_3}) = Z]]$
		\end{oracle}
	\ExptSepSpace
		\begin{algorithm}{$\VE^{-1}(C)$}
			\item return $\TI[C]$
		\end{algorithm}
	\end{minipage}
	
	\caption[]{%
		Game $\lblGm{ggm-vector}$ of the $\stDH$ proof.
	}
	\label{fig:GGM-proof:game:vector}
\end{figure}

\proofngame[ggm-lazysample]
Next, we replace the random encoding function $\E$ with a lazily sampled encoding represented by table $\TV$ for the forward direction and $\TI$ for the backward direction. Because we want our encoding to be one-to-one, we sample from the set $\group\setminus \GL$. This assigns a unique element of $\group$ to each vector $\vec{t}$. However, as we've noted, each integer in $\ZZ_p$ has multiple representations in $\ZZ_p^3$. If two representations of the same integer are submitted to the encoding algorithm $\VE$, we set a $\bad$ flag and program the encoding table to maintain consistency.

We also change the format of the check in the $\stDH$ oracle. Since $\VE(x\vec{a}) = B = \VE(\vec{b})$ if and only if $\langle x \vec{a}, \vec{x} \rangle = \langle \vec{b}, \vec{x} \rangle$, we return $\true$ if the latter condition holds and $\false$ otherwise.  These two conditions are equivalent, so 
$\Pr[\curGm*] = \Pr[\prevGm*]$.

\proofngame[ggm-noprgm]
In this game, we stop programming the encoding table after the bad flag is set. Let $F_1$ denote the event that $\curGm*$ sets the $\bad$ flag at any point. 
By the fundamental lemma of game playing, $\Pr[\prevGm*] \leq  \Pr[\curGm*\text{ and }\overline{F_1} + \Pr[F_1]$. 

\proofngame[ggm-newfin] 
We remove the now-redundant $\bad$ flag, but the $\Finalize$ oracle now returns $\true$ if at any point in game $\prevGm*$ the $\bad$ flag would have been set (i.e. if event $F_1$ occurs). Otherwise, all oracles behave exactly as they did in $\prevGm*$. It follows that 
$\Pr[\prevGm* \text{ and } \overline{F_1}] + \Pr[F_1] \leq \Pr[\curGm*]$.

Additionally, in the $\stDH$ oracle, we separate out checking for trivial queries: if the adversary computed $A = \generator^a$ and $B = X^a$ for an integer $a$ of their choosing. If this is so, then $\vec{a} = a \vec{e_1}$ and $\vec{b} = a \vec{e_2}$, so $\langle x \vec{a}, \vec{x} \rangle = x a = \langle \vec{b},\vec{x} \rangle$, so may return $\true$. If the query is nontrivial but should still return true according to our previous condition, we set a $\bad[2]$ flag. This does not change the oracle's response to any query, so the above bound still holds. 

\proofngame[ggm-nostdh]
In Game $\curGm*$, we no longer return  $\true$ in the $\stDH$ oracle after the $\bad[2]$ flag is set. This makes the second check redundant and has the effect that the $\stDH$ oracle's behavior is no longer dependent on the value of either $x$ or $\vec{x}$. Let event $F_2$ denote the event that $\curGm*$ sets the $\bad[2]$ flag. By the fundamental lemma of game playing, $\Pr[\prevGm*]\leq \Pr[\curGm* \text{ and } \overline{F_2}] + \Pr[F_2]$. 


\proofngame[ggm-quadfin]
In Game $\curGm*$, we remove the redundant check and $\bad$ flag from the $\stDH$ oracle, and in the $\Finalize$ oracle we return $\true$ whenever the $\bad[2]$ flag would have been set in $\prevGm*$. Otherwise all oracles behave precisely as they did in $\prevGm*$. It follows that $\Pr[\prevGm* \text{ and } \overline{F_2}]+\Pr[F_2] \leq \Pr[\curGm*]$.  We also move the initialization of variables $x$, $y$, and $\vec{x}$ from $\Initialize$ to $\Finalize$. Since these variables are not used by any oracle but $\Finalize$, this does not change the view of the adversary. 

\begin{figure}[tp]
	\begin{minipage}[t]{0.49\textwidth}
		\NewExperiment[\frame{$\lblGm{ggm-lazysample}$}, $\lblGm{ggm-noprgm}$]
		
		\ExptSepSpace
		
		\begin{oracle}{$\stDH(A,B)$}
			\item $\vec{a} \gets \VE^{-1}(A)$; $\vec{b} \gets \VE^{-1}(B)$
			\item \gamechange{if $\langle x \vec{a},\vec{x} \rangle = \langle \vec{b}, \vec{x} \rangle$ then return $\true$}{}
			\item return \gamechange{$\false$}{}
		\end{oracle}
		
		\ExptSepSpace
		\begin{algorithm}{$\VE(\vec{t})$}
			\item if $\TV[\vec{t}] \neq \bot$ then return $\TV[\vec{t}]$
			\item \gamechange{$C \gets \group \setminus \GL$}{}
			\item \gamechange{if $(\exists \vec{s} : \TV[\vec{s}] \neq \bot$ and $\langle \vec{t},\vec{x} \rangle = \langle \vec{s}, \vec{x} \rangle )$ }
			\item \quad \gamechange{then $\bad\gets \true$; \frame{$C \gets \TV[\vec{s}]$}}
			\item $k \gets k+1$; $\vec{t_k} \gets \vec{t}$
			\item $\GL \gets \GL \cup \{C\}$
			\item  $\TV[\vec{t}] \gets C$; $\TI[C] \gets \vec{t}$
			\item return $\TV[\vec{t}]$
		\end{algorithm}
	\end{minipage}
	\begin{minipage}[t]{0.49\textwidth}
		\NewExperiment[\frame{$\lblGm{ggm-newfin}$}, $\lblGm{ggm-nostdh}$]
		
		\ExptSepSpace
		
		\begin{oracle}{$\stDH(A,B)$}
			\item $\vec{a} \gets \VE^{-1}(A)$; $\vec{b} \gets \VE^{-1}(B)$; \gamechange{$a \gets \vec{a}[1]$}{}
			\item \gamechange{if $\vec{a} = a\vec{e_1}$ and $\vec{b} = a \vec{e_2}$ then return $\true$}{}
			\item if $\langle x \vec{a},\vec{x} \rangle = \langle \vec{b}, \vec{x} \rangle$ then \gamechange{$\bad[2] \gets \true$}{}; \frame{return $\true$}
			\item return $\false$
		\end{oracle}
		
		\ExptSepSpace
		\begin{algorithm}{$\VE(\vec{t})$}
			\item if $\TV[\vec{t}] \neq \bot$ then return $\TV[\vec{t}]$
			\item $C \gets \group \setminus \GL$
			\item $k \gets k+1$; $\vec{t_k} \gets \vec{t}$
			\item $\GL \gets \GL \cup \{C\}$
			\item  $\TV[\vec{t}] \gets C$; $\TI[C] \gets \vec{t}$
			\item return $\TV[\vec{t}]$
		\end{algorithm}
		
		\ExptSepSpace
		\begin{oracle}{$\Finalize(Z)$}
			\item \gamechange{if $\exists i,j : 1 \leq i < j \leq k$ and $\langle \vec{t_i}-\vec{t_j},\vec{x} \rangle = 0$}
			\item \quad \gamechange{then return $\true$}
			\item return $[[\VE(x \vec{e_3}) = Z]]$ 
		\end{oracle}
	\end{minipage}
	\begin{minipage}[t]{0.49\textwidth}
		\NewExperiment[$\lblGm{ggm-quadfin}$]
		
		\ExptSepSpace
		\begin{oracle}{$\Initialize()$}
			\item $p \gets |\group|$;
			\item $k\gets 0$;$\one \gets \VE(\vec{0})$; $\generator\gets \VE(\vec{e_1})$
			\item $\X \gets \VE(\vec{e_2})$; $\Y\gets \VE(\vec{e_3})$
			\item return $(\one,\generator,\x,\y)$
		\end{oracle}
		\ExptSepSpace
		\begin{oracle}{$\Finalize(Z)$}
			\item $x,y \getsr \ZZ_p^*$; $\vec{x} \gets (\one, x, y)$
			\item if $\exists i,j : 1 \leq i < j \leq k$ and $\langle \vec{t_i}-\vec{t_j},\vec{x} \rangle = 0$)
			\item \quad then return $\true$
			\item \gamechange{if $\exists i,j : 1 \leq i < j \leq k$}
			\item[] \gamechange{and $\langle x\vec{t_i}-\vec{t_j},\vec{x} \rangle = 0$ or $\langle x \vec{t_j}-\vec{t_i},\vec{x} \rangle 0$}
			\item \quad \gamechange{then return $\true$}
			\item return $[[\VE(x \vec{e_3}) = Z]]$ 
		\end{oracle}
	\end{minipage}
	\begin{minipage}[t]{0.49\textwidth}
		\ExptSepSpace
		
		\begin{oracle}{$\stDH(A,B)$}
			\item $\vec{a} \gets \VE^{-1}(A)$; $\vec{b} \gets \VE^{-1}(B)$;$a \gets \vec{a}[1]$
			\item if $( \vec{a} =a\vec{e_1}$ and $\vec{b} = a \vec{e_2})$ then return $\true$
			\item return $0$
		\end{oracle}
		
		
	\end{minipage}
	
	\caption[]{%
		Top left: Games $\lblGm{ggm-lazysample}$ (changes highlighted in \gamechange{gray}) and $\lblGm{ggm-noprgm}$ (changes highlighted in \frame{frames}) of the strong Diffie--Hellman proof. Top right: Games $\lblGm{ggm-newfin}$ and $\lblGm{ggm-nostdh}$. Bottom: Game $\lblGm{ggm-quadfin}$ (changes highlighted in \gamechange{gray}) of the strong Diffie--Hellman proof.
	}
	\label{fig:GGM-proof:game:lazysample}
	\label{fig:GGM-proof:game:quadfin}
\end{figure}
At this point, we can collect the bounds from each gamehop to see that
\shortlongeqn[.]{
	\Adv^{\strongDH}_{\group}(t, \qstDH)\leq \Pr[\curGm*]
}
Therefore we analyze the advantage of an adversary in game $\curGm*$.
  
We can separately analyze each condition of $\Finalize$. We know that $x$ and $y$ are sampled independently of the $t + 4$ entries of $\TV$. For each index $i \in [1\ldots t+4]$, let $F_i$ be the bivariate linear polynomial over $\ZZ_p$ whose coefficients are given by the vector $\vec{t_i}$. Then for any pair  of vectors $(\vec{t_i}, \vec{t_j})$, the condition $\langle \vec{t_i} - \vec{t_j} \rangle = 0$ holds only if $(1,x,y)$ is a root of $F_i-F_j$. Using Lemma 1 of~\cite{EC:Shoup97} and a union bound over all pairs, the probability of this event is at most $(t+4)^2/p$. 

For the second condition; we see that for any $(\vec{t_i}, \vec{t_j})$, it is true that $\langle x \vec{t_i} - \vec{t_j} \rangle = 0$ only if $(1,x,y)$ is a root of $XF_i - F_j$, which is a bivariate quadratic polynomial over $\ZZ_p$. Again Using Lemma 1 and a union bound, this occurs with probability at most $2(t+4)^2/p$.

If neither event occurs, then the adversary wins only if $[\VE(x\vec{e_3}) = Z]$. Because the second condition failed, we know that $(x\vec{e_3})$ is not an entry in table $\TV$. Therefore the response to $\VE(x \vec{e_3})$ will be sampled uniformly at random, and it will equal $Z$ with probability $1/p$.
%I'm lying here, technically it's $1/(p-t)$. I do not care. 
Then by the union bound, $\Pr[\curGm*] \leq (3(t+4)^2+1)/p$. Collecting the bounds gives the theorem statement for all $t >25$.
\end{proof}
\section{The SIGMA Protocol}
\label{sec:sigma}

The \SIGMA family of key exchange protocols introduced by Krawczyk~\cite{C:Krawczyk03,SIGMA-fullversion} describes several variants for building authenticated Diffie--Hellman key exchange using the ``SIGn-and-MAc'' approach.
Its design has been adopted in several Internet security protocols, including, e.g., the Internet Key Exchange protocol~\cite{rfc2409,rfc4306} as part of the IPsec Internet security protocol~\cite{rfc2401} and the newest version~1.3 of the Transport Layer Security (TLS) protocol~\cite{rfc8446}.

Beyond the basic \SIGMA design, we are particularly interested in the \SIGMAI variant which forms the basis of the TLS~1.3 key exchange and aims at hiding the protocol participants' identities as additional feature.
We here present an augmented version of the basic \SIGMA/\SIGMAI protocols which includes explicit exchange of session-identifying random numbers (nonces) to be closer to SIGMA(-like) protocols in practice,
somewhat following the ``full-fledged'' \SIGMA variant~\cite[Appendix~B]{SIGMA-fullversion}.
We illustrate these protocol flows in Figure~\ref{fig:sigma-protocol}. %
\iffull
 and Figure~\ref{fig:sigma-formal} formalizes both as key exchange protocols according to the syntax of Section~\ref{sec:ake-syntax}.
\else
% ; we refer to Appendix~\ref{apx:sigma} for their full formalization as key exchange protocols according to Definition~\ref{def:KE-protocol}.
\fi

The \SIGMA and \SIGMAI protocols make use of
a signature scheme~$\SIGScheme = (\SIGKGen\cab \SIGSign\cab \SIGVerify)$,
a MAC scheme~$\MACScheme = (\MACKGen\cab \MACTag\cab \MACVerify)$,
a pseudorandom function~$\PRF$,
and a function~$\RO$ which we model as a random oracle.
The parties' long-term secret keys consist of one signing key, i.e., $\KE.\KEKGen = \SIGScheme.\SIGKGen$.
The protocols consists of three messages exchanged and accordingly two steps performed by both initiator and responder,
which we describe in more detail now.

\begin{figure}[t]
	\centering
	
	% \begin{minipage}[t]{0.9\textwidth}
\iffull\else\resizebox{11cm}{!}{\fi %% resizebox in lncs
\begin{tikzpicture}
	% Set the X coordinates of the client, server, and arrows
	\edef\ClientX{0}
	\edef\ArrowLeft{4.5}
	\edef\ArrowRight{10.5}
	\edef\ServerX{15}
	
	\newcommand{\sigmaY}{\protvarstyle{Y}}
	
	% Set the starting Y coordinate
	\edef\Y{0}

	% Draw header boxes
	\node [rectangle,draw,inner sep=5pt,right] at (\ClientX,\Y) {\textbf{Initiator} $I$};
	\node [rectangle,draw,inner sep=5pt,left] at (\ServerX,\Y) {\textbf{Responder} $R$};

	% shared info
	\node [inner sep=5pt] at ($(\ClientX,\Y) ! 0.5 ! (\ServerX,\Y)$) {cyclic group $\group = \langle g \rangle$ of prime order~$p$};

	\NextLine[2]
	\ClientAction{\underline{$\RunInitI(I, \sk_I, \st)$}}
	\ServerAction{\underline{$\RunRespI(R, \sk_R, \st, \peerpk, m = (\nonce_I, \X))$}}
	\NextLine
	\ClientAction{$\x \getsr \ZZ_p$, $\X \gets g^\x$}
	\ServerAction{$\y \getsr \ZZ_p$, $\sigmaY \gets g^\y$}
	\NextLine
	
	\ClientAction[name=RunInitI-last]{$\nonce_I \getsr \bits^{nl}$}
	\ServerAction{$\nonce_R \getsr \bits^{nl}$}
	
	%%%%%%%%%%%%%%%%%%%%%%%%%%%%%%%%%%%%%%%%%%%%%%%%%%%%%%%%%%%%%%%%%
	\ClientToServer{$\nonce_I, \X$}{}
	\NextLine[0.25]
	%%%%%%%%%%%%%%%%%%%%%%%%%%%%%%%%%%%%%%%%%%%%%%%%%%%%%%%%%%%%%%%%%
	\ServerAction{$\sid \gets (\nonce_I, \nonce_R,\X,\sigmaY)$}
	\NextLine
	\ServerAction{$\mk \gets \RO(\nonce_I,\nonce_R,\X,\sigmaY,\X^\y)$}
	\NextLine
	\ServerAction{$\ks / \kt / \fbox{$\ke$} \gets \PRF(\mk,0 / 1 / 2)$}
	\NextLine
	
	\ServerAction{$\sigma \gets \SIGScheme.\SIGSign(\sk_R, \labelrs\|\nonce_I\|\nonce_R\| \X \| \sigmaY)$}
	\NextLine
	\ServerAction{$\tau \gets \MACScheme.\MACTag(\kt, \labelrm\|\nonce_I\|\nonce_R\|R)$}
	\NextLine
	\ServerAction[name=RunRespI-last]{$c \gets (R, \sigma,\tau)$ \fbox{$c \gets \text{Enc}_{\ke}(R,\sigma,\tau)$}}
	
	\NextLine[-1]
	\ClientAction[name=RunInitII]{\underline{$\RunInit(I, \sk_I, \st, \peerpk, m = (\nonce_R,\sigmaY,c))$}}
	{\NextLine[1]}
	%%%%%%%%%%%%%%%%%%%%%%%%%%%%%%%%%%%%%%%%%%%%%%%%%%%%%%%%%%%%%%%%%
	\ServerToClient{$\nonce_R, \sigmaY, c$}{}
	{\NextLine[-0.5]}
	\ClientAction{$\sid \gets (\nonce_I,\nonce_R,\X,\sigmaY)$}% \fg{Why do we set $\sid$ here already (there's aborts below), not upon acceptance?} \fg{Because we actually need sid be set for explicit auth.}
	\NextLine
	%%%%%%%%%%%%%%%%%%%%%%%%%%%%%%%%%%%%%%%%%%%%%%%%%%%%%%%%%%%%%%%%%

	\ClientAction{$\mk \gets \RO(\nonce_I,\nonce_R,\X,\sigmaY,\sigmaY^{\x})$}
	\NextLine
	\ClientAction{$\ks / \kt / \fbox{$\ke$} \gets \PRF(\mk,0 / 1 / 2)$}
	\NextLine
	\ClientAction{$(R, \sigma,\tau) \gets c$ \fbox{$(R, \sigma, \tau) \gets \text{Dec}_{\ke}(c)$}}
	\NextLine
	
	\ClientAction{\textbf{abort} if $\neg \SIGScheme.\SIGVerify(\peerpk[R], \labelrs\|\nonce_I\|\nonce_R\| \X \| \sigmaY, \sigma)$}
	\NextLine
	\ClientAction{\textbf{abort} if $\neg \MACScheme.\MACVerify(\kt, \labelrm\|\nonce_I\|\nonce_R\|R, \tau)$}
	\NextLine
	
	\ClientAction{$\status \gets \accepted$; $\peerid \gets R$}
	\NextLine
	\ClientAction{$\sigma' \gets \SIGScheme.\SIGSign(\sk_I, \labelis\|\nonce_I\|\nonce_R\|\X\|\sigmaY)$} 
	\NextLine
	\ClientAction{$\tau' \gets \MACScheme.\MACTag(\kt, \labelim\|\nonce_I\|\nonce_R\|I)$}
	\NextLine
	\ClientAction[name=RunInitII-last]{$c' \gets (I,\sigma', \tau')$ \fbox{$c' \gets \text{Enc}_{\ke}(I,\sigma',\tau')$}}
	

	\ServerAction[name=RunRespII]{\underline{$\RunRespII(\id,\sk,\st,\peerpk, m = c')$}}
	{\NextLine[0.25]}
	%%%%%%%%%%%%%%%%%%%%%%%%%%%%%%%%%%%%%%%%%%%%%%%%%%%%%%%%%%%%%%%%%
	{\NextLine[0.5]}
	%%%%%%%%%%%%%%%%%%%%%%%%%%%%%%%%%%%%%%%%%%%%%%%%%%%%%%%%%%%%%%%%%
	\ServerAction{$(I, \sigma',\tau') \gets c'$\fbox{$(I, \sigma', \tau') \gets \text{Dec}_{\ke}(c')$}}
	\NextLine
	\ServerAction{\textbf{abort} if  $\neg \SIGScheme.\SIGVerify(\peerpk[I], \labelis\|\nonce_I\|\nonce_R\|\X\|\sigmaY,\sigma')$}
	\NextLine
	\ServerAction{\textbf{abort} if $\neg \MACScheme.\MACVerify(\kt, \labelim\|\nonce_I\|\nonce_R\|I, \tau')$}
	\NextLine	
	\ServerAction{$\status \gets \accepted$; $\peerid \gets I$}
	
	\NextLine[1.5]
	
	\SharedAction{\textbf{accept} with key~$\skey = \ks$ and session identifier~$\sid = (\nonce_I, \nonce_R, \X, \sigmaY)$}
	
	
	%
	% state passing
	%
	\draw [thick,gray,dashed,-latex] (RunInitI-last.south west) -- (RunInitII.north west) node [midway,right,darkgray] {$\st.\state \gets (\nonce, \X, \x)$};
	\draw [thick,gray,dashed,-latex] (RunRespI-last.south east) -- (RunRespII.north east) node [midway,left,darkgray] {$\st.\state \gets (\nonce, \nonce',\X,\sigmaY,\ks,\kt,\fbox{$\ke$})$};
\end{tikzpicture}
\iffull\else}\fi %% resizebox in lncs
% \end{minipage}


	
	\caption{%
		The \SIGMA/\SIGMAI protocol flow diagram.
		\fbox{Boxed} code is only performed in the \SIGMAI variant.
		Values~$\inputlabel[x]$ indicate label strings (distinct per~$x$).
	}
	\label{fig:sigma-protocol}
\end{figure}
\definecollection{SIGMAformal}
\begin{collect*}{SIGMAformal}{}{}{}{} %%% ===== COLLECT BEGIN =====
\begin{figure}[tp]
  	\begin{minipage}[t]{0.49\textwidth}
		\begin{algorithm}{$\KEActivate(\id, \sk, \peerid,\peerpk,\role)$}
			\item $\st'.\role \gets \role$
			\item $\st'.\status \gets \running$
			\item if $\role = \initiator$ then
			\item \hindent $(\st',m') \gets \RunInitI(\id,\sk,\st')$
			\item else $m' \gets \bot$
			\item return $(\st',m')$
  		\end{algorithm}

  		\ExptSepSpace

		\begin{algorithm}{$\KERun(\id,\sk,\st,\peerpk,m)$}
			\item if $\st.\status \neq \running $ then
			\item return $\bot$
			\item if $\st.\role = \initiator$ then
			\item \hindent $(\st',m') \gets \RunInit(\id,\sk,\st,\peerpk,m)$
			\item else if $\st.\sid = \bot$
			\item \hindent $(\st',m') \gets \RunRespI(\id,\sk,\st,\peerpk,m)$
			\item else
			\item \hindent $(\st',m') \gets \RunRespII(\id,\sk,\st,\peerpk,m)$
			\item return $(\st',m')$
		\end{algorithm}
		
		\ExptSepSpace
		
		\begin{algorithm}{$\RunInitI(\id,\sk,\st)$}
			\item $\nonce_I \sample \bits^{\nl}$
			\item $\x \sample \ZZ_p$
			\item $\X \gets g^{\x}$
			\item $\st'.\state \gets (\nonce_I, \X, \x)$
			\item $m' \gets (\nonce_I, \X)$
			\item return $(\st',m')$
		\end{algorithm}
		
		\ExptSepSpace
		
		\begin{algorithm}{$\RunRespI(\id,\sk,\st,\peerpk,m)$}
			\item $(\nonce_I,\X) \gets m$ 
			\item $\nonce_R \sample \bits^{\nl}$
			\item $\y \sample \ZZ_p$
			\item $\Y \gets g^{\y}$
			\item $\st'.\sid \gets (\nonce_I, \nonce_R,\X,\Y)$
			\item $\sigma \gets \SIGScheme.\SIGSign(\sk,\labelrs\|\nonce_I\|\nonce_R\|\X\|\Y)$
			\item $\mk \gets \RO(\nonce_I\|\nonce_R\|\X\|\Y\|\X^{\y})$
			\item $\ks \gets \PRF(\mk,0)$
			\item $\kt \gets \PRF(\mk,1)$
			\item \frame{$\ke \gets \PRF(\mk,2)$}
			\item $\tau \gets \MACScheme.\MACTag(\kt, \labelrm\|\nonce_I\|\nonce_R\|\id)$
			\item $\st'.\state \gets (\nonce_I,\nonce_R,\X,\Y,\ks,\kt)$ \newline
				\frame{$\st'.\state \gets (\nonce_I, \nonce_R,\X,\Y,\ks,\kt,\ke)$}
			\item $m' \gets (\nonce_R, \Y, \id, \sigma, \tau)$ \newline
				\frame{$m' \gets (\nonce_R, \Y, \ENCEnc(\ke,(\id,\sigma,\tau)))$}
			\item return $(\st', m')$
		\end{algorithm}
	\end{minipage}
	\begin{minipage}[t]{0.49\textwidth}
		\begin{algorithm}{$\RunInit(\id,\sk,\st,\peerpk,m)$}
			\item $(\nonce_R,\Y,\peerid,\sigma, \tau) \gets m$ \newline
				\frame{$(\nonce_R,\Y,\ciph) \gets m$}
			\item $(\nonce_I,\X,\x) \gets \st.\state$
			\item $\st'.\sid \gets (\nonce_I,\nonce_R,\X,\Y)$
			
			\item $\mk \gets \RO(\nonce_I\|\nonce_R\|\X\|\Y\|\Y^{\x})$
			\item $\ks \gets \PRF(\mk,0)$
			\item $\kt \gets \PRF(\mk,1)$
			\item \frame{$\ke \gets \PRF(\mk,2)$}
			
			\item \frame{$(\peerid,\sigma,\tau) \gets \ENCDec(\ke,\ciph)$}
			\item $\st'.\peerid \gets \peerid$
			
			\item if $\SIGScheme.\SIGVerify(\peerpk[\peerid], \labelrs\|\nonce_I\|\nonce_R\|\X\|\Y, \sigma)$\\
				and $\MACScheme.\MACVerify(\kt, \labelrm\|\nonce_I\|\nonce_R\|\peerid, \tau)$ then
			\item \hindent $\st'.\status \gets \accepted$
			\item \hindent $\st'.\skey \gets \ks$
			\item \hindent $\sigma' \gets \SIGScheme.\SIGSign(\sk, \labelis\|\nonce_I\|\nonce_R\|\X\|\Y)$
			\item \hindent $\tau' \gets \MACScheme.\MACTag(\kt, \labelim\|\nonce_I\|\nonce_R\|\id)$
			\item \hindent $m' \gets (\id, \sigma', \tau')$ \newline
				\null\hindent \frame{$m' \gets \ENCEnc(\ke,(\id,\sigma',\tau'))$}
			\item else
			\item \hindent $m' \gets \bot$
			\item \hindent $\st'.\status \gets \rejected$
			\item return $(\st', m')$
		\end{algorithm}

		\ExptSepSpace

		\begin{algorithm}{$\RunRespII(\id,\sk,\st,\peerpk,m)$}
			\item $(\nonce_I,\nonce_R,\X,\Y,\ks,\kt) \gets \st.\state$ \newline
				\frame{$(\nonce_I,\nonce_R,\X,\Y,\ks,\kt,\ke) \gets st.\state$}
			\item $(\peerid,\sigma',\tau')\gets m$ \newline
				\frame{$(\peerid,\sigma',\tau')\gets \ENCDec(\ke,m)$}
			\item $\st'.\peerid\gets \peerid$
			\item if $\SIGScheme.\SIGVerify(\peerpk[\peerid], \labelis\|\nonce_I\|\nonce_R\|\X\|\Y, \sigma')$\\
				and $\MACScheme.\MACVerify(\kt, \labelim\|\nonce_I\|\nonce_R\|\peerid, \tau')$ then
			\item \hindent $\st'.\status \gets \accepted$
			\item \hindent $\st'.\skey \gets \ks$
			\item else $\st'.\status \gets \rejected$
			\item $m' \gets \emptystring$
			\item return $(\st', m')$
		\end{algorithm}
	\end{minipage}
  	\caption{%
		The formalized \SIGMA/\SIGMAI key exchange protocols (cf.\ Section~\ref{sec:ake-syntax}).
		\fbox{Boxed} code is only performed in the \SIGMAI variant.
  	}
  	\label{fig:sigma-formal}
\end{figure}
\end{collect*}

%\SIGMAformal

\begin{description}
	\item[Initiator Step~1.]
	The initiator picks a Diffie--Hellman exponent~$\x \sample \ZZ_p$ and a random nonce~$\nonce_I$ of length~$nl$ and sends $\nonce_I$ and~$g^\x$.
	
	
	\item[Responder Step~1.]
	The responder also picks a random DH exponent~$\y$ and a random nonce~$\nonce_R$.
	It then derives a master key as~$\mk \gets \RO(\nonce_I\cab \nonce_R\cab \X\cab \Y\cab \X^\y)$ from nonces, DH shares, and the joint DH secret~$g^{\x\y} = (g^\x)^\y$.
	From~$\mk$, keys are derived via $\PRF$ with distinct labels:
	the session key~$\ks$,
	the MAC key~$\kt$,
	and (only in \SIGMAI) the encryption key~$\ke$.
	
	The responder computes a signature~$\sigma$ with~$\sk_R$ over nonces and DH shares (and a unique label~$\labelrs$) and a MAC value~$\tau$ under key~$\kt$ over the nonces and its identity~$R$ (and unique label~$\labelrm$).
	It sends $\nonce_I$, $g^\y$, as well as $R$, $\sigma$, and~$\tau$ to the initiator.
	In \SIGMAI the last three elements are encrypted using~$\ke$ to conceal the responder's identity against passive adversaries.
	
	
	\item[Initiator Step~2.]
	The initiator also computes~$\mk$ and keys~$\ks$, $\kt$, and (in \SIGMAI, used to decrypt the second message part) $\ke$.
	It ensures both the received signature~$\sigma$ and MAC~$\tau$ verify, and aborts otherwise.
	
	It computes its own signature~$\sigma'$ under~$\sk_I$ on nonces and DH shares (with a different label~$\labelis$)
	and a MAC~$\tau'$ under~$\kt$ over the nonces and its identity~$I$ (with yet another label~$\labelim$).
	It sends $I$, $\sigma'$, and~$\tau'$ to the responder (in \SIGMAI encrypted under~$\ke$)
	and accepts with session key~$\ks$ using the nonces and DH shares $(\nonce_I, \nonce_R, \X, \Y)$ as session identifier.
	
	\item[Responder Step~2.]
	The responder finally checks the initiator's signature~$\sigma'$ and MAC~$\tau'$ (aborting if either fails)
	and then accepts with session key~$\skey = \ks$ and session identifier~$\sid = (\nonce_I, \nonce_R, \X, \Y)$.
\end{description}

\section{Tighter Security Proof for SIGMA-I}
\label{sec:sigma-proof}

We now come to our first main result, a tighter security proof for the \SIGMAI protocol.
Note that by omitting message encryption our proof similarly applies to the basic \SIGMA protocol.

%%% old, non-parameterized version
% {\color{gray}
% \begin{theorem}
% 	Let $\advA$ be a key exchange security adversary against the \SIGMAI protocol as specified in Figure~\ref{fig:sigma-formal},
% 	and let~$\RO$ in the protocol be modeled as a random oracle.
% 	Then there exist algorithms~$\advB_1$, $\advB_2$, $\advB_3$, and $\advB_4$ with running time close to that of $\advA$, given in the proof, such that
% 	\begin{align*}
% 		\Adv^{\KESEC}_{\KE,\advA} \leq
% 			~&\frac{q_{\Send}^2}{2^{\nl+1}} + \frac{q_{\Send}^2}{2q}	% Game 2
% 			+ \genAdv{stDH}{\group}{\advB_1}				% Game 6
% 			+ \genAdv{mu{\minus}PRF}{\PRF}{\advB_2}\\			% Game 7
% 			&+ \genAdv{mu{\minus}EUF{\minus}CMA}{\SIGScheme}{\advB_3}	% Game 9
% 			+ \genAdv{mu{\minus}EUF{\minus}CMA}{\MACScheme}{\advB_4},	% Game 11
% 	\end{align*}
% 	where
% 	$\nl$ is the nonce length in \SIGMAI,
% 	$\group$ is the used Diffie--Hellman group of prime order~$p$,
% 	and $\advA$ makes at most $q_{\Send}$ queries to its $\Send$ oracle.
% \end{theorem}
% }

\begin{theorem}
	\label{thm:SIGMAI}
	Let the \SIGMAI protocol be as specified in Figure~\fullelse{\ref{fig:sigma-formal}}{\ref{fig:sigma-protocol}} based on a group~$\group$ of prime order~$p$, a PRF~$\PRF$, a signature scheme~$\SIGScheme$, and a MAC~$\MACScheme$, and let $\RO$ in the protocol be modeled as a random oracle.
	For any $(t\cab \qNewUser\cab \qSend\cab \qRevSessionKey\cab \qRevLongTermKey\cab \qTest)$-$\KESEC$-adversary against~\SIGMAI making at most $\qRO$ queries to $\RO$,
	we give algorithms~$\advB_1$, $\advB_2$, $\advB_3$, and $\advB_4$ in the proof,
	with running times~$t_{\advB_1} \approx t + 2\qRO \log_2 p$ and $t_{\advB_i}\approx~t$ (for $i = 2,\dots,4$) close to that of~$\advA$,
	such that
	\begin{collectinmacro}{\SIGMABound}{}{} %%% ===== COLLECT BEGIN =====
	\begin{align*}
		\Adv&^{\KESEC}_{\mSIGMAI}(t, \qNewUser, \qSend, \qRevSessionKey, \qRevLongTermKey, \qTest)\\
			\leq
% 			&~\frac{\qSend^2}{2^{\nl+1}} + \frac{\qSend^2}{2p}	+ \frac{\qSend^2}{2^{\nl}\cdot p}	% Game 2 -- old sid collision bounds
			&~\frac{3\qSend^2}{2^{\nl+1}\cdot p}								% Game 2
			+ \Adv^{\strongDH}_{\group}(t_{\advB_1}, \qRO)							% Game 6
			+ \Adv^{\muPRFSEC}_{\PRF}(t_{\advB_2}, \qSend, 3\qSend,3) \\					% Game 7
			&+ \Adv^{\muEUFCMA}_{\SIGScheme}(t_{\advB_3}, \qNewUser, \qSend, \qSend, \qRevLongTermKey)	% Game 9
			+ \Adv^{\muEUFCMA}_{\MACScheme}(t_{\advB_4}, \qSend, \qSend,1, \qSend,1, 0).			% Game 11 
	\end{align*}
	\end{collectinmacro}
	\SIGMABound
	Here,
	$\nl$ is the nonce length in \SIGMAI
	and $\group$ is the used Diffie--Hellman group of prime order~$p$.
\end{theorem}

In terms of multi-user security for the employed primitives,
multi-user PRF and MAC security can be obtained tightly, e.g., via the efficient AMAC construction~\cite{EC:BelBerTes16},
and multi-user signature security can be generically reduced to single-user security of any signature scheme with a loss in the number of users, here parties (not sessions) in the key exchange game.

\iffull

\begin{proof}
\startproof{SIGMA}

Our proof of key exchange security for \SIGMAI proceeds via a sequence of code-based games~\cite{EC:BelRog06}.
\iffull\else
Due to space limitations, we will refer to Appendix~\ref{apx:sigma} for some of the game hops' detailed code description.
\fi
For the first half, the proof conceptually follows the strategy put forward by Cohn-Gordon et al.~\cite{C:CCGJJ19}.


\proofngame[initial]
The initial game, $\curGm$, is the key exchange security game played by~$\advA$ (cf.\ Figure~\ref{fig:AKE-security}),
using the $\KEKGen$, $\KEActivate$, and $\KERun$ routines of \SIGMAI defined in Figure~\ref{fig:sigma-formal}.
Therefore,
\shortlongeqn[.]{
	\Pr[ \curGm* \Rightarrow 1 ] = \Pr[ \Gm^{\KESEC}_{\KE,\advA} \Rightarrow 1 ]
}

\proofngame[tracking]
Between $\prevGm$ and $\curGm$ (Figure~\ref{fig:SIGMAI-proof:game:tracking}\iffull\else{ in the appendix on page~\pageref{apxfig:SIGMAI-proof:tracking--collisions}}\fi), we make internal changes to the record-keeping of the game, namely we track the nonces and group elements chosen and received by honest sessions.
Whenever two honest sessions pick the same nonce or group element, we set a flag~$\bad[C]$.
Whenever an honest responder session picks a nonce and group element that has already been received by an initiator session, we set a flag~$\bad[O]$. 
This change is unobservable by the adversary, hence
\shortlongeqn[.]{
	\Pr[ \prevGm* \Rightarrow 1 ] = \Pr[ \curGm* \Rightarrow 1]
}


\begin{collectinmacro}{\SIGMIProofTrackingCollisions}{}{} %%% ===== COLLECT BEGIN =====
\begin{figure}[tp]
	\begin{minipage}[t]{0.49\textwidth}
	\NewExperiment[$\lblGm{tracking}$, \frame{$\lblGm{collisions}$}]

	\begin{oracle}{$\RunInitI(\id, \sk,\st)$}
		\item $\nonce_I \sample \bits^{\nl}$
		\item $\x \sample \ZZ_p$
		\item $\X \gets g^{\x}$
		\item \gamechange{if $(\nonce_I, \X) \in \N$ then $\bad[C] \gets \true$} \frame{; abort}
		\item \gamechange{$\N \gets \N \cup \{(\nonce_I, \X)\}$}
		\item $\st'.\state \gets (\nonce_I,\X,\x)$
		\item $m' \gets (\nonce_I, \X)$
		\item return $(\st',m')$
	\end{oracle}

	\ExptSepSpace
\begin{oracle}{$\RunInit(\id, \sk, \st, \peerpk, m)$}
	\item $(\nonce_R,\Y,\ciph) \gets m$
	\item \gamechange{$\Recv \gets \Recv \cup \{(\nonce_R,\Y)\}$}
	\item $(\nonce_I,\X,\x) \gets \st.\state$
	%		\item $\st'.\sid \gets (\nonce_I,\nonce_R,\X,\Y)$
	%		
	%		\item $\mk \gets \RO(\nonce_I\|\nonce_R\|\X\|\Y\|\Y^{\x})$
	%		\item $\ks \gets \PRF(\mk,0)$
	%		\item $\kt \gets \PRF(\mk,1)$
	%		\item $\ke \gets \PRF(\mk,2)$
	%		
	%		\item $(\peerid,\sigma,\tau) \gets \ENCDec(\ke,\ciph)$
	%		\item $\st'.\peerid \gets \peerid$
	\item \ldots
\end{oracle}
	\end{minipage}
	%
	\begin{minipage}[t]{0.49\textwidth}
	\ExptSepSpace
	\begin{oracle}{$\RunRespI(\id,\sk,\st,\peerpk,m)$}
		\item $(\nonce_I,\X) \gets m$
		\item $\nonce_R \sample \bits^{\nl}$
		\item $\y \sample \ZZ_p$
		\item $\Y \gets g^{\x}$
		\item \gamechange{if $(\nonce_R,\Y) \in \Recv$ then $\bad[O] \gets\true$} \frame{; abort}
		\item \gamechange{if $(\nonce_R,\Y) \in \N$ then $\bad[C] \gets \true$}\frame{; abort}
		\item \gamechange{$\N \gets \N \cup \{ (\nonce_R, \Y) \}$}
		\item $\st'.\sid \gets (\nonce_I, \nonce_R,\X,\Y)$
		\item \ldots
%		\item $\sigma \gets \SIGScheme.\SIGSign(\sk,\labelrs\|\nonce_I\|\nonce_R\|\X\|\Y)$
%		\item $\mk \gets \RO(\nonce_I\|\nonce_R\|\X\|\Y\|\X^{\y})$
%		\item $\ks \gets \PRF(mk,0)$
%		\item $\kt \gets \PRF(\mk,1)$
%		\item $\ke \gets \PRF(\mk,2)$
%		\item $\tau \gets \MACScheme.\MACTag(\kt, \labelrm\|\nonce_I\|\nonce_R\|\id)$
%		\item $\st'.\state \gets (\nonce_I,\nonce_R,\X,\Y,\ks,\kt, \ke)$
%		\item $m' \gets (\nonce_R, \Y, \id, \sigma, \tau)$
	\end{oracle}
	\ExptSepSpace

\begin{algorithm}[start=101]{$\RO(m)$}
	\item if $H[m] = \bot$ then $H[m] \sample \bits^{\kl}$
	\item return $H[m]$
\end{algorithm}
	\end{minipage}

	\caption[]{%
		Games $\lblGm{tracking}$ (changes highlighted in \gamechange{gray}) and $\lblGm{collisions}$ (changes highlighted in \frame{frames}) of the \SIGMAI proof;
		with the explicit (lazy-sampled) random oracle~$\RO$.
	}
	\label{fig:SIGMAI-proof:game:tracking}
	\label{fig:SIGMAI-proof:game:collisions}

  	\iffull\else %% appendix label if deferred to appendix
	\label{apxfig:SIGMAI-proof:tracking--collisions}
	\fi
\end{figure}
\end{collectinmacro}

\iffull %% deferred to appendix in non-full version
\SIGMIProofTrackingCollisions
\fi


\proofngame[collisions]

In Game~$\curGm$ (Figure~\ref{fig:SIGMAI-proof:game:collisions}\iffull\else{ in the appendix on page~\pageref{apxfig:SIGMAI-proof:tracking--collisions}}\fi), we abort whenever nonces and group elements collide among honest sessions (i.e., the~$\bad[C]$ flag is set), or whenever an honest responder session chooses a nonce and group element already submitted by the adversary to an initiator (i.e., the~$\bad[O]$ flag is set).
By the identical-until-bad lemma~\cite{EC:BelRog06},
\shortlongeqn[.]{
	\Pr[ \prevGm* \Rightarrow 1] - \Pr[ \curGm* \Rightarrow 1] \leq \Pr[ \bad[C]\text{ or }\bad[O] \gets \true \text{ in } \prevGm* ]
}
In all of the calls to $\RunInitI$ and $\RunRespI$, up to $\qSend$ pairs of nonces and group elements are chosen uniformly at random. By the birthday bound, the probability of a collision between two of these pairs setting the $\bad[C]$ flag is at most $\frac{\qSend^2}{2^{\nl+1}\cdot p}$ (where $\nl$ is the nonce length and $p$ the order of the Diffie--Hellman group).
There are at most $\qSend$ pairs received by initiator sessions, so the probability that a responder session randomly chooses one of these pairs is at most $\frac{\qSend}{2^{\nl}\cdot p}$; then by the union bound we have that 
$\Pr[\bad[O] \gets \true \text{ in } \prevGm*] \leq \frac{\qSend^2}{2^{\nl}\cdot p}$.
Since each of $\RunInitI$ and $\RunRespI$ is called at most once per $\Send$ query, if an adversary makes $\qSend$ queries to its $\Send$ oracle, then 
\shortlongeqn[.]{
	\Pr[ \prevGm* \Rightarrow 1] - \Pr[ \curGm* \Rightarrow 1] \leq \frac{3\qSend^2}{2^{\nl+1}\cdot p}
}
In all subsequent games, we are now sure that each honest session chooses a unique nonce and group element.
Since the session identifier~$\sid = (\nonce_I, \nonce_R, \X, \Y)$ contains exactly one initiator and one responder nonce,
this furthermore implies that when two honest sessions are partnered, they must have different roles.


\begin{collectinmacro}{\SIGMIProofRecordKeysInitiatorsCopy}{}{} %%% ===== COLLECT BEGIN =====
\begin{figure}[tp]
  \begin{minipage}[t]{0.49\textwidth}
    \NewExperiment[$\lblGm{record-keys}$, \frame{$\lblGm{initiators-copy}$}]

    \begin{oracle}{$\RunInitI(\id, \sk, \st)$}
      \item $\nonce_I \sample \bits^{\nl}$
      \item $x \sample \ZZ_p$
      \item $\X \gets g^{\x}$
      \item if $(\nonce_I, \X) \in \N$ then abort
      \item $\N \gets \N \cup \{(\nonce_I,\X)\}$
      \item $\st'.\state \gets (\nonce_I,\X,x)$
      \item $m' \gets (\nonce_I, \X)$
      \item \gamechange{$\Sent \gets \Sent \cup {m'}$}
      \item return $(\st',m')$
    \end{oracle}
    \ExptSepSpace
    \begin{oracle}{$\RunInit(\id,\sk,\st,\peerpk,m)$}
      \item $(\nonce_R,\Y,\ciph) \gets m$
      \item $\Recv \gets \Recv \cup \{(\nonce_R,\Y)\}$
      \item $(\nonce_I,\X,x) \gets \st.\state$
      \item $\st'.\sid \gets (\nonce_I,\nonce_R,\X,\Y)$
      \item \gamechange{if $\S[\st'.\sid] \neq \bot$ then }
      \item \hindent $\mk \gets \RO(\nonce_I\|\nonce_R\|\X\|\Y\|\Y^{\x})$
      \item \hindent $\ks \gets \PRF(\mk,0)$
      \item \hindent $\kt \gets \PRF(\mk,1)$
      \item \hindent $\ke \gets \PRF(\mk,2)$
      \item \hindent \frame{$\ks,\kt,\ke \gets \S[\st'.\sid]$}
      \item \gamechange{else }
      \item \hindent $\mk \gets \RO(\nonce_I\|\nonce_R\|\X\|\Y\|\Y^{\x})$
      \item \hindent $\ks \gets \PRF(\mk,0)$
      \item \hindent $\kt \gets \PRF(\mk,1)$
      \item \hindent $\ke \gets \PRF(\mk,2)$
      \item $(\peerid,\sigma, \tau) \gets \ENCDec(\ke,\ciph)$
      \item $\st'.\peerid \gets \peerid$
      \item \ldots
%      \item if $\SIGScheme.\SIGVerify(\labelrs\|\nonce_I\|\nonce_R\|\X\|\Y, \sigma, \pk_{\peerid})$ and $\MACScheme.\MACVerify(\kt, \labelrm\|\nonce_I\|\nonce_R\|\peerid)$ then
%      \item \hindent $\st'.\status \gets \accepted$
%      \item \hindent $\st'.\skey \gets \ks$
%      \item \hindent $\sigma' \gets \SIGScheme.\SIGSign(\sk, \labelis\|\nonce_I\|\nonce_R\|\X\|\Y)$
%      \item \hindent $\tau' \gets \MACScheme.\MACTag(\kt, \labelim\|\nonce_I\|\nonce_R\|\id)$
%      \item \hindent $m' \gets (\id, \sigma', \tau')$
%      \item else
%      \item \hindent $m' \gets \bot$
%      \item \hindent $\st'.\status \gets \rejected$
    \end{oracle}
    \end{minipage}
    \begin{minipage}[t]{0.49\textwidth}
    \ExptSepSpace
    \begin{oracle}{$\RunRespI(\id,\sk,\st,\peerpk,m)$}
      \item $(\nonce_I,\X) \gets m$
      \item $\nonce_R \sample \bits^{\nl}$
      \item $\y \sample \ZZ_p$
      \item $\Y \gets g^{\x}$
      \item if $(\nonce_R,\Y) \in \Recv$ then abort
      \item if $(\nonce_R, \Y) \in \N$ then abort
      \item $\N \gets \N \cup \{(\nonce_R,\Y)\}$
      \item $\st'.\sid \gets (\nonce_I, \nonce_R,\X,\Y)$
      \item $\sigma \gets \SIGScheme.\SIGSign(\sk,\labelrs\|\nonce_I\|\nonce_R\|\X\|\Y)$
      \item $\mk \gets \RO(\nonce_I\|\nonce_R\|\X\|\Y\|\X^{\y})$
      \item $\ks \gets \PRF(\mk,0)$
      \item $\kt \gets \PRF(\mk,1)$
      \item $\ke \gets \PRF(\mk,2)$
      \item \gamechange{if $m \in \Sent$ then }
      \item \hindent \gamechange{$\S[\st'.\sid] \gets (\ks,\kt,\ke)$}
      \item $\tau \gets \MACScheme.\MACTag(\kt, \labelrm\|\nonce_I\|\nonce_R\|\id)$
      \item $\st'.\state \gets (\nonce_I,\nonce_R,\X,\Y,\ks,\kt)$
      \item $m' \gets (\nonce_R, \Y, \ENCEnc(\ke,(\id,\sigma,\tau)))$
      \item return $(\st',m')$
    \end{oracle}
  \end{minipage}
	\caption[]{%
		Games $\lblGm{record-keys}$ (changes highlighted in \gamechange{gray}) and $\lblGm{initiators-copy}$ (changes highlighted in \frame{frames}) of the \SIGMAI proof.
	}
	\label{fig:SIGMAI-proof:game:record-keys}
	\label{fig:SIGMAI-proof:game:initiators-copy}

  	\iffull\else %% appendix label if deferred to appendix
	\label{apxfig:SIGMAI-proof:record-keys--initiators-copy}
	\fi
\end{figure}
\end{collectinmacro}


\proofngame[record-keys]

In Game~$\curGm$ (Figure~\ref{fig:SIGMAI-proof:game:record-keys}\iffull\else{ in the appendix on page~\pageref{apxfig:SIGMAI-proof:record-keys--initiators-copy}}\fi), we remove the now superfluous collision flags~$\bad[C]$ and~$\bad[O]$ and add additional bookkeeping.
All honest initiator sessions now log their outgoing messages in an internal table $\Sent$. 
Honest responder sessions use this table to check if the message they received was sent by an honest initiator session. 
If so, they log their keys $\kt$, $\ke$, and $\ks$ in a second internal table, $\S$, indexed by their session identifier. 
These changes are unobservable by the adversary, so
\shortlongeqn[.]{
	\Pr[ \prevGm* \Rightarrow 1 ] = \Pr[ \curGm* \Rightarrow 1]
}


\proofngame[initiators-copy]

In Game~$\curGm$ (Figure~\ref{fig:SIGMAI-proof:game:initiators-copy}\iffull\else{ in the appendix on page~\pageref{apxfig:SIGMAI-proof:record-keys--initiators-copy}}\fi), we require that initiator sessions whose key material has already been computed by an honest partner session simply copy their partners' key material.
When an honest initiator session $\pi_u^i$ with nonce $n$ and group element $\X$ receives a message $m~\gets~(\nonce_R,\Y, \ciph)$, it sets its session identifier $\sid \gets (\nonce_I,\nonce_R,\X,\Y)$. 
It then checks if $\S[\sid] \neq \bot$ (which is only the case if $\pi_u^i$ has an honest partner),
and if so uses the stored key material~$\ks,\kt,\ke \gets \S[\st'.\sid]$. 

Recall that both partnered sessions agree on the DH shares~$\X$ and~$\Y$ as components of $\sid$.
They hence also agree on the shared DH secret $Z = g^{\x\y}$ and thus on the master key derived as~$\RO(\nonce_I \conc \nonce_R \conc \X \conc \Y \conc Z)$ as well as the derived key~$\ks$, $\kt$, and~$\ke$.
For the adversary~$\advA$ it is hence unobservable if initiators with honest partner actually compute their keys themselves or copy their partners' key material in Game~$\curGm$, so
\shortlongeqn[.]{
	\Pr[ \prevGm* \Rightarrow 1 ] = \Pr[ \curGm* \Rightarrow 1]
}

\iffull %% deferred to appendix in non-full version
\SIGMIProofRecordKeysInitiatorsCopy
\fi


\proofngame[uniform-mk]

In Game~$\curGm$ (Figure~\ref{fig:SIGMAI-proof:game:uniform-mk}),
all honest sessions sample their master keys uniformly at random (as long as the random oracle has not been been queried on the corresponding input) and program the random oracle to that value (through $\RO$'s internal table~$H[\nonce_I\|\nonce_R\|\X\|\Y\|\Y^{\x}] \gets \mk$).
This is equivalent to $\RO$ performing the same checks and uniform sampling, and hence undetectable for~$\advA$:
\shortlongeqn[.]{
	\Pr[ \prevGm* \Rightarrow 1 ] = \Pr[ \curGm* \Rightarrow 1]
}


\proofngame[responders-stop-programming]

In Game~$\curGm$ (Figure~\ref{fig:SIGMAI-proof:game:uniform-mk}), responder sessions whose first message came from an honest initiator stop programming the random oracle on their uniformly chosen master key~$\mk$.
This is undetectable for adversary~$\advA$ unless it makes a query $\RO(\nonce_I\|\nonce_R\|\X\|\Y\|\Zz)$,
where $\sid = (\nonce_I,\nonce_R,\X,\Y)$ is the session identifier shared by two honest partnered sessions, and $\Zz$ is the Diffie--Hellman secret corresponding to the pair $(\X,\Y)$.
We call this event~$F$, and bound the probability of $F$ by giving a reduction~$\advB_1$ (specified in Figure~\ref{fig:SIGMAI-proof:game:advB1}) to the strong Diffie--Hellman assumption in the DH group~$\group$. The reduction makes at most as many queries to its $\stDH$ oracle as $\advA$ makes to its~$\RO$ oracle, as follows.

Given its strong DH challenge $(A = g^a, B = g^b)$ and having access to an oracle~$\stDH_a(U, V)$ which outputs~1 if $U^a = V$ and 0 otherwise,
$\advB_1$ simulates $\curGm$ for an adversary $\advA$ as follows.
In all honest initiator sessions, $\advB_1$ embeds its challenge into the sent DH value as~$\X \gets A \cdot g^\r$, where~$\r \in \ZZ_p$ is sampled uniformly at random for each session.
Furthermore, in all responder sessions receiving their first message from an honest initiator, $\advB_1$ embeds its challenge as $\Y \gets B \cdot g^{\r'}$, where~$\r' \in \ZZ_p$ is sampled uniformly at random for each session.

Let us first observe that if event~$F$ occurs, then the value~$\Zz$ in the random oracle query $\RO(\nonce_I\|\nonce_R\|\X\|\Y\|\Zz)$ will equal $g^{(a+\r)(b+\r')}$ for some  $\r, \r' \in \ZZ_p$ chosen by $\advB_1$, and consequently
\shortlongeqn[.]{
	\Zz \cdot \Y^{-\r} = g^{(a+\r)(b+\r')-(b+\r') \cdot \r} = g^{a(b+\r')} = \Y^a
}
This equality can be tested for by~$\advB_1$ by calling its~$\stDH_a$ oracle on the pair $(\Y, \Zz \cdot \Y^{-\r})$.
We let $\advB_1$ do so whenever $\advA$ queries~$\RO$ on some value~$(\nonce_I\|\nonce_R\|\X\|\Y\|\Zz)$ where $(\nonce_I, \X = A \cdot g^\r)$ was output by an honest initiator session and $(\nonce_R, \Y = g^{(b+\r')})$ was output by a responder session with an honest initiator; the responder stores $(\nonce_I,\nonce_R,\X,\Y)$ in a look-up table $\Q$ so this can be checked efficiently.
If $\stDH_a(\Y, \Zz \cdot \Y^{-\r}) = 1$ on such occasion, i.e., event~$F$ occurs,
$\advB_1$ stops with output $\Zz \cdot \Y^{-\r} \cdot A^{-\r'} = g^{(a+\r)(b+\r')} \cdot g^{-(b + \r') \cdot \r} \cdot g^{-a\r'} = g^{ab}$ and wins.
Therefore,
\shortlongeqn[.]{
	\Pr[F] \leq \genAdv{stDH}{\group}{\advB_1}
}

One subtlety in this step is ensuring that~$\advB_1$ can correctly simulate answers to $\RevSessionKey$ queries to any initiator or responder session.
We do so by accordingly programming the random oracle on the sampled master key, where needed.
First of all observe that responder sessions without honest initiator keep picking their own~$\Y$ share and compute $\mk$ regularly.
Initiator and responder sessions with honest partner have the challenge embedded and sample an independent master key which is not programmed to the random oracle.
However, $\advB_1$ stops and wins (as described above) if~$\advA$ ever queries the random oracle on the correct DH secret;
i.e., $\advA$ never sees the (inconsistent) random oracle output for these master keys.
The interesting case is when an initiator session (which always embeds the challenge in its DH share as $\X = A \cdot g^r$) obtains a message~$(\nonce_R, \Y, \ciph)$ \emph{not} originating from an honest responder:
Here, $\Y$ may well have been picked by the adversary who could furthermore have corrupted the initiator's peer and hence make the initiator accept---with a master key it cannot compute itself.

We therefore let $\advB_1$ attempt to copy the adversary's master key, if it has been computed.
The $\RO$ oracle logs all queries it receives by their putative session id $(\nonce_I,\nonce_R,\X,\Y)$ in a look-up table~$H'$, so $\advB_1$ can efficiently access all $\Zz$ such that $(\nonce_I,\nonce_R,\X,\Y,\Zz)$ has been queried to $\RO$. 
Since the DH secret corresponding to the pair $(\X,\Y)$ equals $\Y^{a+r}$, if $\Zz$ is this DH secret, then 
\shortlongeqn[.]{
	\Zz \cdot \Y^{-r} = \Y^{(a+r)-r} = \Y^a
}
The reduction can check this equality using its $\stDH_a$ oracle and in that case use the response to $\RO(\nonce_I,\nonce_R,\X,\Y,\Zz)$ as~$\mk$.
Otherwise, $\advB_1$ samples~$\mk$ at random and stores it in the table~$\Q$ (Line~\ref{line:SIGMA-proof:game:advB1:initiator-remember-Q-mk} of Figure~\ref{fig:SIGMAI-proof:game:advB1}), indicating it should be programmed in the random oracle later if queried on a matching~$\Zz$ value (Line~\ref{line:SIGMA-proof:game:advB1:program-Q-mk}).
This ensures all responses to $\RevSessionKey$ queries are consistent with $\advA$'s queries to the random oracle~$\RO$.

\begin{figure}[tp]
    \begin{minipage}[t]{0.49\textwidth}
      \NewExperiment[$\lblGm{uniform-mk}$, \frame{$\lblGm{responders-stop-programming}$}]

        \begin{oracle}{$\RunInit(\id,\sk,\st,\peerpk,m)$}
        \iffull
          \item $(\nonce_R,\Y,\ciph) \gets m$
          \item $\Recv \gets \Recv \cup \{(\nonce_R,\Y)\}$
          \item $(\nonce_I,\X,\x) \gets st.\state$
          \item $\st'.\sid \gets (\nonce_I,\nonce_R,\X,\Y)$
        \else
          \item \dots
        \fi
          \item if $\S[\st'.\sid] \neq \bot$ then
          \item \hindent $\ks,\kt,\ke \gets \S[\st'.\sid]$
          \item else
          \item \hindent \gamechange{$\mk\sample \bits^{\kl}$}
          \item \hindent \gamechange{if $H[\nonce_I\|\nonce_R\|\X\|\Y\|\Y^{\x}]\neq \bot$}
          \item \hindent \hindent \gamechange{$\mk \gets H[\nonce_I\|\nonce_R\|\X\|\Y\|\Y^{\x}]$}
          \item \hindent \gamechange{$H[\nonce_I\|\nonce_R\|\X\|\Y\|\Y^{\x}] \gets \mk$}
          \item \hindent $\ks \gets \PRF(\mk,0)$
          \item \hindent $\kt \gets \PRF(\mk,1)$
          \item \hindent $\ke \gets \PRF(\mk,2)$
        \iffull
          \item $(\peerid,\sigma, \tau) \gets \ENCDec(\ke,\ciph)$
          \item $\st'.\peerid \gets \peerid$
          \item if $\SIGScheme.\SIGVerify(\peerpk[\peerid], \labelrs\|\nonce_I\|\nonce_R\|\X\|\Y, \sigma)$\\
		and $\MACScheme.\MACVerify(\kt, \labelrm\|\nonce_I\|\nonce_R\|\peerid)$ then
          \item \hindent $\st'.\status \gets \accepted$
          \item \hindent $\st'.\skey \gets \ks$
          \item \hindent $\sigma' \gets \SIGScheme.\SIGSign(\sk, \labelis\|\nonce_I\|\nonce_R\|\X\|\Y)$
          \item \hindent $\tau' \gets \MACScheme.\MACTag(\kt, \labelim\|\nonce_I\|\nonce_R\|\id)$
          \item \hindent $m' \gets \ENCEnc(\ke,(\id, \sigma', \tau'))$
          \item else
          \item \hindent $m' \gets \bot$
%           \item \hindent 
          ; $\st'.\status \gets \rejected$
          \item return $(\st', m')$
        \else
          \item \dots
        \fi
        \end{oracle}
        \end{minipage}
        \begin{minipage}[t]{0.49\textwidth}
          \ExptSepSpace
        \begin{oracle}{$\RunRespI(\id,\sk,\st,\peerpk,m)$}
        \iffull
          \item $(\nonce_I,\X) \gets m$
          \item $\nonce_R \sample \bits^{\nl}$
          \item $\y \sample \ZZ_p$
          \item $\Y \gets g^{\x}$
          \item if $(\nonce_R, \Y) \in \Recv$ then abort
          \item if $(\nonce_R, \Y) \in \N$ then abort
          \item $\N \gets \N \cup \{(\nonce_R,\Y)\}$
          \item $\st'.\sid \gets (\nonce_I, \nonce_R,\X,\Y)$
          \item $\sigma \gets \SIGScheme.\SIGSign(\sk,\labelrs\|\nonce_I\|\nonce_R\|\X\|\Y)$
        \else
          \item \dots
        \fi
          \item \gamechange{$\mk\sample \bits^{\kl}$}
          \item \frame{if $m \not\in \Sent$ then}
          \item \hindent \gamechange{if $H[\nonce_I\|\nonce_R\|\X\|\Y\|\X^{\y}]\neq \bot$}
          \item \hindent \hindent \gamechange{$\mk \gets H[\nonce_I\|\nonce_R\|\X\|\Y\|\X^{\y}]$}
          \item \hindent \gamechange{$H[\nonce_I\|\nonce_R\|\X\|\Y\|\X^{\y}] \gets  \mk$}
          \item $\ks \gets \PRF(\mk,0)$
          \item $\kt \gets \PRF(\mk,1)$
          \item $\ke \gets \PRF(\mk,2)$
          \item if $m \in \Sent$ then
          \item \hindent $\S[\st'.\sid] \gets (\ks,\kt,\ke)$
        \iffull
          \item $\tau \gets \MACScheme.\MACTag(\kt, \labelrm\|\nonce_I\|\nonce_R\|\id)$
          \item $\st'.\state \gets (\nonce_I,\nonce_R,\X,\Y,\ks,\kt)$
          \item $m' \gets (\nonce_R, \Y, \id, \sigma, \tau)$
          \item return $(\st', m')$
        \else
          \item \dots
        \fi
        \end{oracle}
    \end{minipage}
  	\caption[]{%
		Games $\lblGm{uniform-mk}$ (changes highlighted in \gamechange{gray}) and $\lblGm{responders-stop-programming}$ (changes highlighted in \frame{frames}) of the \SIGMAI proof.
  	}
  	\label{fig:SIGMAI-proof:game:uniform-mk}
  	\label{fig:SIGMAI-proof:game:responders-stop-programming}
\end{figure}

\begin{figure}[tp]
  \centering
  \scalebox{0.9}{
  \begin{minipage}[t]{0.49\textwidth}
    \NewExperiment[$\advB_1(A,B)^{\stDH_a(\cdot,\cdot)}$]

      \begin{oracle}{$\RunInitI(\id, \sk, \st)$}
        \item $\nonce_I \sample \bits^{\nl}$
        \item \gamechange{$\r \sample \ZZ_p$}
        \item \gamechange{$\X \gets A \cdot g^{\r}$}
        \item if $(\nonce_I, \X) \in \N$ then abort
        \item $\N \gets \N \cup \{(\nonce_I,\X)\}$
        \item \gamechange{$\st'.\state \gets (\nonce_I,\X,\r)$}
        \item $m' \gets (\nonce_I, \X)$
        \item \gamechange{$\Sent[m'] \gets x$}
        \item return $(\st',m')$
      \end{oracle}

    \ExptSepSpace

      \begin{oracle}{$\RunInit(\id,\sk,\st,\peerpk,m)$}
      \iffull
        \item $(\nonce_R,\Y,\ciph) \gets m$
        \item $\Recv \gets \Recv \cup \{(\nonce_R,\Y)\}$
        \item $(\nonce_I,\X,\r) \gets \st.\state$
        \item $\st'.\sid \gets (\nonce_I,\nonce_R,\X,\Y)$
      \else
        \item \dots
      \fi
        \item if $\S[\st'.\sid] \neq \bot$ then
        \item \hindent $\ks,\kt,\ke \gets \S[\st'.\sid]$
        \item else
        \item \hindent $\mk\sample \bits^{\kl}$
        \item \hindent \gamechange{for each $\Zz \in H'[\nonce_I\|\nonce_R\|\X\|\Y]$}
        \item \hindent \hindent \gamechange{if $\stDH_a(\Y,\Zz \cdot \Y^{-\r}) = 1$ then}
        \item \hindent \hindent \hindent \gamechange{$\mk \gets H[\nonce_I\|\nonce_R\|\X\|\Y\|\Zz]$}
        \item \hindent \gamechange{$\Q[\st'.\sid] \gets (\r,\bot,\mk)$}
        \item \hindent $\ks \gets \PRF(\mk,0)$
        \item \hindent $\kt \gets \PRF(\mk,1)$
        \item \hindent $\ke \gets \PRF(\mk,2)$
      \iffull
        \item $(\peerid,\sigma, \tau) \gets \ENCDec(\ke, \ciph)$ 
        \item $\st'.\peerid \gets \peerid$
        \item if $\SIGScheme.\SIGVerify(\labelrs\|\nonce_I\|\nonce_R\|\X\|\Y, \sigma, \pk_{\peerid})$\\
		and $\MACScheme.\MACVerify(\kt, \labelrm\|\nonce_I\|\nonce_R\|\peerid)$ then
        \item \hindent $\st'.\status \gets \accepted$
        \item \hindent $\st'.\skey \gets \ks$
        \item \hindent $\sigma' \gets \SIGScheme.\SIGSign(\sk, \labelis\|\nonce_I\|\nonce_R\|R\|W)$
        \item \hindent $\tau' \gets \MACScheme.\MACTag(\kt, \labelim\|\nonce_I\|\nonce_R\|\id)$
        \item \hindent $m' \gets \ENCEnc(\ke,(\id, \sigma', \tau'))$
        \item else
        \item \hindent $m' \gets \bot$
        \item \hindent $\st'.\status \gets \rejected$
        \item return $(\st', m')$
      \else
        \item \dots
      \fi
      \end{oracle}

    \end{minipage}
    \begin{minipage}[t]{0.49\textwidth}
      \begin{oracle}{$\RunRespI(\id,\sk,\st,\peerpk,m)$}
        \item $(\nonce_I,\X) \gets m$
        \item $\nonce_R \sample \bits^{\nl}$
        \item \gamechange{$\r' \sample \ZZ_p$}
        \item $\mk\sample \bits^{\kl}$
        \item \gamechange{if $m \in \Sent$ then}
        \item \hindent \gamechange{$\r \gets \Sent[m]$}
        \item \hindent \gamechange{$\Y \gets B \cdot g^{r'}$}
        \item \hindent \gamechange{$\st'.\sid \gets (\nonce_I,\nonce_R,\X,\Y)$}
        \item \hindent \gamechange{for each $\Zz \in H'[\nonce_I\|\nonce_R\|\X\|\Y]$}
        \item \hindent \hindent \gamechange{if $\stDH_a(\Y,\Zz \cdot \Y^{-\r}) = 1$ then}
        \item \hindent \hindent \hindent \gamechange{$\procfont{Finalize}(\Zz \cdot \Y^{-\r} \cdot A^{-\r'}$)}
        \item \hindent \gamechange{$\Q[\st'.\sid] \gets (\r,\r',\mk)$}
		\label{line:SIGMA-proof:game:advB1:initiator-remember-Q-mk}
        \item else 
        \item \hindent $\Y \gets g^{\r'}$
        \item \hindent $\st'.\sid \gets (\nonce_I, \nonce_R,\X,\Y)$
        \item \hindent if $H[\nonce_I\|\nonce_R\|\X\|\Y\|\X^\y]\neq \bot$
        \item \hindent \hindent$\mk \gets H[\nonce_I\|\nonce_R\|\X\|\Y\|\X^{\r'}]$
        \item \hindent $H[\nonce_I\|\nonce_R\|\X\|\Y\|\X^\y] \gets \mk$
        \item if $(\nonce_R, \Y) \in \Recv$ then abort
        \item if $(\nonce_R, \Y) \in \N$ then abort
        \item $\N \gets \N \cup \{(\nonce_R,\Y)\}$
        \item $\sigma \gets \SIGScheme.\SIGSign(\sk,\labelrs\|\nonce_I\|\nonce_R\|\X\|\Y)$
        \item $\ks \gets \PRF(\mk,0)$
        \item $\kt \gets \PRF(\mk,1)$
        \item $\ke \gets \PRF(\mk,2)$
        \item if $m \in \Sent$ then
        \item \hindent $\S[\st'.\sid] \gets (\ks,\kt,\ke)$
        \item $\tau \gets \MACScheme.\MACTag(\kt, \labelrm\|\nonce_I\|\nonce_R\|\id)$
        \item $\st'.\state \gets (\nonce_I,\nonce_R,\X,\Y,\ks,\kt, \ke)$
        \item $m' \gets (\nonce_R, \Y, \ENCEnc(\ke,(\id, \sigma, \tau))$
        \item return $(\st', m')$
      \end{oracle}
      
      \ExptSepSpace
      
      \begin{oracle}{$\RO(m)$}
      \item if $H[m] = \bot$ then
      \item \hindent $H[m] \sample \bits^{\kl}$
      \item \hindent \gamechange{parse $\nonce_I \conc \nonce_R \conc \X \conc \Y \conc \Zz \gets m$}
      \item \hindent \gamechange{$H'[\nonce_I\|\nonce_R\|\X\|\Y] \gets H'[\nonce_I\|\nonce_R\|\X\|\Y] \cup \{\Zz\}$}
      \item \hindent \gamechange{if $\Q[(\nonce_I,\nonce_R,\X,\Y)] \neq \bot$ then}
      \item \hindent \hindent \gamechange{$(\r,\r',\mk)\gets \Q[\nonce_I,\nonce_R,\X,\Y]$}
      \item \hindent \hindent \gamechange{if $\stDH_a(\Y,\Zz \cdot \Y^{-\r}) = 1$ then}
      \item \hindent \hindent \hindent \gamechange{if $\r' = \bot$ then $H[m] \gets \mk$}
	\label{line:SIGMA-proof:game:advB1:program-Q-mk}
      \item \hindent \hindent \hindent \gamechange{else $\procfont{Finalize}(\Zz \cdot \Y^{-\r} \cdot A^{-\r'})$}
      \item return $H[m]$
    \end{oracle}
  \end{minipage}
  }

  \caption[]{%
      		Reduction~$\advB_1$ to the strong Diffie--Hellman assumption of the \SIGMAI proof.
      		Sections highlighted in \gamechange{gray} have been significantly altered compared to Game~$\lblGm{responders-stop-programming}$.
  }
  \label{fig:SIGMAI-proof:game:advB1}
\end{figure}

Observe that, in all this, $\advB_1$ calls its $\stDH$ oracle at most once for each entry~$H[\nonce_I \conc \nonce_R \conc \X \conc \Y \conc \Zz] = \mk$ in the $\RO$ table~$H$. 
In $\RO$, $\stDH$ is called (once) only for entries that were not present when $\Q[(\nonce_I, \nonce_R, \X, \Y)]$ was set, then~$H'$ is set.
In $\RunInit$ and $\RunRespI$, $\stDH$ is called only for matching $H'$ entries established prior to setting~$\Q$.
Therefore, if $\stDH$ is called in $\RO$ for an entry, it was not called in either $\RunInit$ or $\RunRespI$. 
If $\stDH$ is called on an entry in $\RunRespI$, then the responder session is partnered, so its partner will copy its keys in $\RunInit$ and not call $\stDH$. Furthermore, due to uniqueness of nonces and DH shares (by Game~$\lblGm{collisions}$), no $\RunInit$ or $\RunRespI$ call makes $\stDH$ be invoked twice for the same $H'$ entry. 

Since the total time to iterate through the for loops over all $\KERun$ and $\RO$ queries is at most $O(\qRO)$, the running time of $\advB_1$ is roughly that of $\advA$, plus the time needed to compute the arguments of the $\stDH$ queries. 
Each of these arguments requires one group operation and one exponentiation. (All other operations performed by $\advB_1$ add only a small constant amount of time per $\Send$ query, which is dominated by the runtime of $\advA$.) The exponentiation can be computed using $2\log_2 p$ group operations using the square-and-multiply (or double-and-add) algorithm, so $t_{\advB_1} \approx t + 2\qRO\log_2 p$. The runtime $t$ of $\advA$ already includes the computation of $2\qSend\log_2 p$ group operations, so this is a significant but not prohibitive increase in runtime.

Having $\advB_1$ perfectly simulate Game~$\prevGm$ for~$\advA$ up to the point when~$F$ happens,
and $\curGm$ and~$\prevGm$ differing only when~$F$ happens,
we have
\shortlongeqn[,]{
	\Pr[ \prevGm* \Rightarrow 1 ]
	= \Pr[ \curGm* \Rightarrow 1] + \Pr[F]
	\leq \Pr[ \curGm* \Rightarrow 1] + \genAdv{\stDH}{\group}{}(t_{\advB_1}, \qRO)
}
and $t_{\advB_1} \approx t+2\qRO \log_2 p$. 


\proofngame[random-prf]

In Game~$\curGm$ (Figure~\ref{fig:SIGMAI-proof:game:random-prf}\iffull\else{ in the appendix on page~\pageref{apxfig:SIGMAI-proof:PRF}}\fi), responder oracles responding to honest messages samples session, MAC, and encryption keys~$\ks$, $\kt$, and~$\ke$ randomly instead of computing them through a PRF.
(Initiator oracles partnered with an honest responder will continue to copy those, now randomly sampled keys.)

\begin{collectinmacro}{\SIGMIProofPRF}{}{} %%% ===== COLLECT BEGIN =====
\begin{figure}[t]
    \begin{minipage}[t]{0.49\textwidth}
      \NewExperiment[$\lblGm{random-prf}$]

      \begin{oracle}{$\RunRespI(\id,\sk,\st,\peerpk,m)$}
          \item $(\nonce_I,\X) \gets m$
          \item $\nonce_R \sample \bits^{\nl}$
          \item $\y \sample \ZZ_p$
          \item $\Y \gets g^{\y}$
          \item if $(\nonce_R, \Y) \in \Recv$ then abort
          \item if $(\nonce_R, \Y) \in \N$ then abort
		  \item $\N \gets \N \cup \{(\nonce_R,\Y)\}$
          \item $\st'.\sid \gets (\nonce_I, \nonce_R,\X,\Y)$
          \item $\sigma \gets \SIGScheme.\SIGSign(\sk,\labelrs\|\nonce_I\|\nonce_R\|\X\|\Y)$
          \item $\mk\sample \bits^{\kl}$
          \item if $m \not\in \Sent$ then
          \item \hindent if $H[\Y^{\x}\|\nonce_I\|\nonce_R\|\X\|\Y]\neq \bot$
          \item \hindent \hindent$\mk \gets H[\Y^{\x}\|\nonce_I\|\nonce_R\|\X\|\Y]$
          \item $\ks \gets \PRF(\mk,0)$
          \item $\kt \gets \PRF(\mk,1)$
          \item $\ke \gets \PRF(\mk,2)$
          \item \gamechange{if $m \in \Sent$}
          \item \hindent \gamechange{$\ks \sample \bits^{kl}$}
          \item \hindent \gamechange{$\kt \sample \bits^{kl}$}
          \item \hindent \gamechange{$\ke \sample \bits^{kl}$}
	      \item \hindent $\S[\st'.\sid] \gets (\ks,\kt,\ke)$
          \item $\tau \gets \MACScheme.\MACTag(\kt, \labelrm\|\nonce_I\|\nonce_R\|\id)$
          \item $\st'.\state \gets (\nonce_I,\nonce_R,\X,\Y,\ks,\kt)$
          \item $m' \gets (\nonce_R, \Y, \ENCEnc(\ke,(\id, \sigma, \tau))$
          \item return $(\st', m')$
      \end{oracle}
    \end{minipage}
    \begin{minipage}[t]{0.49\textwidth}
      \NewExperiment[$\advB_2^{\PRFfn(\cdot,\cdot)}$]


      \begin{oracle}{$\RunRespI(\id,\sk,\st,\peerpk,m)$}
        \item $(\nonce_I,\X) \gets m$
        \item $\nonce_R \sample \bits^{\nl}$
        \item $\y \sample \ZZ_p$
        \item $\Y \gets g^{\y}$
        \item if $(\nonce_R, \Y) \in \Recv$ then abort
        \item if $(\nonce_R, \Y) \in \N$ then abort
        \item $\N \gets \N \cup \{(\nonce_R,\Y)\}$
        \item $\st'.\sid \gets (\nonce_I, \nonce_R,\X,\Y)$
        \item $\sigma \gets \SIGScheme.\SIGSign(\sk,\labelrs\|\nonce_I\|\nonce_R\|\X\|\Y)$
        \item $\mk\sample \bits^{\kl}$
        \item if $m \not\in \Sent$ then
        \item \hindent if $H[\nonce_I\|\nonce_R\|\X\|\Y\|\X^{\y}]\neq \bot$
        \item \hindent \hindent$\mk \gets H[\nonce_I\|\nonce_R\|\X\|\Y\|\X^{\y}]$
        \item $\ks \gets \PRF(\mk,0)$
        \item $\kt \gets \PRF(\mk,1)$
        \item $\ke \gets \PRF(\mk,2)$. 
        \item if $m \in \Sent$
        \item \hindent \gamechange{$\procfont{New}()$; $i++$}
        \item \hindent \gamechange{$\ks \gets \procfont{Fn}(i,0)$}
        \item \hindent \gamechange{$\kt \gets \procfont{Fn}(i,1)$}
        \item \hindent \gamechange{$\ke \gets \procfont{Fn}(i,2)$}
        \item \hindent $\S[\st'.\sid] \gets (\ks,\kt,\ke)$
        \item $\tau \gets \MACScheme.\MACTag(\kt, \labelrm\|\nonce_I\|\nonce_R\|\id)$
        \item $\st'.\state \gets (\nonce_I,\nonce_R,\X,\Y,\ks,\kt)$
        \item $m' \gets (\nonce_R, \Y,  \ENCEnc(\ke,(\id, \sigma, \tau))$
        \item return $(\st', m')$
        \end{oracle}
    \end{minipage}
  	\caption[]{%
		Game $\lblGm{random-prf}$ and reduction~$\advB_2$ to PRF security of the \SIGMAI proof.
		Changes from~$\lblGm{responders-stop-programming}$ resp. compared to~$\lblGm{random-prf}$ highlighted in \gamechange{gray}.
  	}
  	\label{fig:SIGMAI-proof:game:random-prf}
  	\label{fig:SIGMAI-proof:game:advB2}
  	
  	\iffull\else %% appendix label if deferred to appendix
	\label{apxfig:SIGMAI-proof:PRF}
	\fi
\end{figure}
\end{collectinmacro}

\iffull %% deferred to appendix in non-full version
\SIGMIProofPRF
\fi

Since the PRF key~$\mk$ in this case is sampled independently of the random oracle and the rest of the game,
this reduces straightforwardly to the multi-user security
% \footnote{While generically, multi-user security of PRFs reduces to single-user security of PRFs with a factor in the number of users via a hybrid argument~\cite{FOCS:BelCanKra96},
% simple and efficient constructions like AMAC~\cite{EC:BelBerTes16} achieve tight multi-user security.}
of the PRF via the reduction~$\advB_2$ we give in Figure~\ref{fig:SIGMAI-proof:game:advB2}.
The adversary $\advB_2$ makes one $\New$ and two $\FUNC$ queries for each $\RunRespI$ query, or three $\FUNC$ queries in $\SIGMAI$. Notably, it makes at most three $\FUNC$ queries per user, and no $\Corrupt$ queries because $\mk$ is never revealed to the adversary. Outside of the oracle calls, its running time exactly equals that of $\advA$ in Game $~\prevGm$, as their pseudocode is identical, so $t_{\advB_2}\approx t$. 
Using its~$\procfont{Fn}$ oracle of the PRF game, $\advB_2$ perfectly simulates~$\prevGm$ if the oracle gives real-PRF answers and $\curGm$ if it returns uniformly random values.
Therefore,
\shortlongeqn[.]{
	\Pr[ \prevGm* \Rightarrow 1 ]
	\leq \Pr[ \curGm* \Rightarrow 1] + \genAdv{mu{\minus}PRF}{\PRF}{}(t_{\advB_2},\qSend,3\qSend,3,0)
}

Observe that from now on, session and MAC keys of responder oracles that received honest initiator's messages are chosen independently at random,
and that initiator oracles with matching~$\sid$ will copy those keys.
Notably, this is the case even for sessions whose (own or peer's) long-term secret have been revealed to the adversary.
We will use these properties in the following to argue authentication of sessions
as well as forward security of the session keys.

\medskip

Our final game hops are concerned with the explicit authentication performed through signatures and MACs in the \SIGMAI protocol,
and as such extend those proof steps for implicit authentication of the main protocols in~\cite{C:CCGJJ19}.
\iffull\else
Due to space limitations, we give the code-based game hops for the remaining four games in the appendix in Figure~\ref{apxfig:SIGMAI-proof:signature-mac-games} on page~\pageref{apxfig:SIGMAI-proof:signature-mac-games}.
\fi


\proofngame[signature-bookkeeping]

In Game~$\curGm$ (Figure~\ref{fig:SIGMAI-proof:game:signature-bookkeeping}), we log all messages for which signatures are generated by an honest session,
and set a bad flag~$\bad[S]$ if the adversary submits a valid signature under an uncorrupted signing key for a message which was not produced by an honest session.
This internal bookkeeping does not affect the adversary's advantage, so
\shortlongeqn[.]{
	\Pr[ \prevGm* \Rightarrow 1 ] = \Pr[ \curGm* \Rightarrow 1]
}


\proofngame[signature-forgeries]

In Game~$\curGm$ (Figure~\ref{fig:SIGMAI-proof:game:signature-forgeries}), we abort if the $\bad[S]$ flag is set.
By the identical-until-bad lemma, the difference in advantage between $\prevGm$ and $\curGm$ is bounded by the probability that this event occurs,
which we reduce via an algorithm~$\advB_3$ to the multi-user security of the digital signature scheme~$\SIGScheme$.%
% \footnote{%
% Multi-user EUF-CMA security of signature schemes (with adaptive corruptions) can be reduced to classical, single-user EUF-CMA security by a hybrid argument~\cite{TCC:BHJKL15,DCC:MenSma04}, losing a factor of number of users, which correspond to the number of parties (not sessions) in our setting.
% In many cases, such loss is indeed unavoidable~\cite{EC:BJLS16}.
% }


\begin{collectinmacro}{\SIGMIProofSignatureMAC}{}{} %%% ===== COLLECT BEGIN =====
\begin{figure}[tp]
	\begin{minipage}[t]{0.49\textwidth}
		\NewExperiment[$\lblGm{signature-bookkeeping}$, \frame{$\lblGm{signature-forgeries}$}]

		\begin{oracle}{$\RunInit(\id,\sk,\st,\peerpk,m)$}
% 			\item $(\nonce_R,\Y,\peerid,\sigma, \tau) \gets m$
% 			\item $(\nonce_I,\X,x) \gets st.\state$
% 			\item $\st'.sid \gets (\nonce_I,\nonce_R,\X,\Y)$
% 			\item $\st'.\peerid \gets \peerid$
% 			\item if $S[\st'.sid] \neq \bot$ then
% 			\item \hindent $\ks,\kt,\ke \gets S[\st'.sid]$
% 			\item else
% 			\item \hindent $\mk\sample \bits^{hl}$
% 			\item \hindent if $H[\nonce_I\|\nonce_R\|\X\|\Y\|\Y^{\x}]\neq \bot$
% 			\item \hindent \hindent $\mk \gets H[\nonce_I\|\nonce_R\|\X\|\Y\|\Y^{\x}]$
% 			\item \hindent $H[\nonce_I\|\nonce_R\|\X\|\Y\|\Y^{\x}] \gets mk$
% 			\item \hindent $\ks \gets \PRF_{\mk}(0)$
% 			\item \hindent $\kt \gets \PRF_{\mk}(1)$
%			\item \hindent $\ke \gets \PRF_{\mk}(2)$
			\item \dots
			\item if $\SIGScheme.\SIGVerify(\peerpk[\peerid], \labelrs\|\nonce_I\|\nonce_R\|\X\|\Y, \sigma)$\\
				and $\MACScheme.\MACVerify(\kt, \labelrm\|\nonce_I\|\nonce_R\|\peerid, \tau)$ then
			\item \hindent \gamechange{if $\revltk_{\peerid} = \infty$ and}\newline
				\null \hindent\hindent \gamechange{$(\peerid, \labelrs\|\nonce_I\|\nonce_R\|\X\|\Y) \notin \Q_{S}$ then}
			\item \hindent \hindent \gamechange{$\bad[S] \gets \true$} \frame{; abort}
			\skipline
			\item[]
			\skipline
			\item \hindent $\st'.\status \gets \accepted$
			\item \hindent $\st'.\skey \gets \ks$
			\item \hindent $\sigma' \gets \SIGScheme.\SIGSign(\sk, \labelis\|\nonce_I\|\nonce_R\|\X\|\Y)$
			\item \hindent \gamechange{$\Q_{S} \gets \Q_{S} \cup \{(\id, \labelis\|\nonce_I\|\nonce_R\|\X\|\Y)\}$}
			\item \hindent $\tau' \gets \MACScheme.\MACTag(\kt, \labelim\|\nonce_I\|\nonce_R\|\id)$
			\skipline
			\skipline
			\item \dots
% 			\item \hindent $m' \gets (\id, \sigma', \tau')$
% 			\item else
% 			\item \hindent $m' \gets \bot$
% 			\item \hindent $\st'.\status \gets \rejected$
		\end{oracle}

		\ExptSepSpace

		\begin{oracle}{$\RunRespI(\id,\sk,\st,\peerpk,m)$}
% 			\item $(\nonce_I,\X) \gets m$
% 			\item $\nonce_R \sample \bits^{nl}$
% 			\item if $\nonce_I \in \N$ then $\bad \gets \true $; abort
% 			\item $N \gets N \cup \{\nonce_I\}$
% 			\item $y \sample \ZZ_p$
% 			\item $\Y \gets g^{\x}$
% 			\item if $\Y \in E$ then $\bad \gets \true$; abort
% 			\item $E \gets E \cup \{\Y\}$
%           \item if $\nonce_R,\Y \in \Recv$ then abort
% 			\item $\st'.sid \gets (\nonce_I, \nonce_R,\X,\Y)$
			\item \dots
			\item $\sigma \gets \SIGScheme.\SIGSign(\sk,\labelrs\|\nonce_I\|\nonce_R\|\X\|\Y)$
			\item \gamechange{$\Q_{S} \gets \Q_{S} \cup \{(\id, \labelrs\|\nonce_I\|\nonce_R\|\X\|\Y)\}$}
			\item \dots
% 			\item $mk\sample \bits^{hl}$
% 			\item if $m \not\in Sent$ then
% 			\item \hindent if $H[\Y^{\x}\|\nonce_I\|\nonce_R\|\X\|\Y]\neq \bot$
% 			\item \hindent \hindent$mk \gets H[\Y^{\x}\|\nonce_I\|\nonce_R\|\X\|\Y]$
% 			\item \hindent $Q[\st'.sid] \gets {\X^y, mk}$
% 			\item \hindent $\ks \gets \PRF_{mk}(0)$
% 			\item \hindent $\kt \gets \PRF_{mk}(1)$
% 			\item else
% 			\item \hindent $\ks \sample \bits^{kl}$
% 			\item \hindent $\kt \sample \bits^{kl}$
			\item $\tau \gets \MACScheme.\MACTag(\kt, \labelrm\|\nonce_I\|\nonce_R\|\id)$
			\skipline
			\item \dots
% 			\item $\st'.\state \gets (\nonce_I,\nonce_R,\X,\Y,\ks,\kt)$
% 			\item $m' \gets (\nonce_R, \Y, \id, \sigma, \tau)$
% 			\item if $m \in Sent$ then
% 			\item \hindent $S[\st'.sid] \gets (\ks,\kt)$
			\item[] %% for alignment
		\end{oracle}

		\ExptSepSpace

		\begin{oracle}{$\RunRespII(\id,\sk,\st,\peerpk,m)$}
% 			\item $(\nonce_I,\nonce_R,\X,\Y,\ks,\kt,\ke) \gets \st.\state$
% 			\item $(\peerid,sigma',\tau')\gets \ENCDec(\ke,m)$
% 			\item $\st'.\peerid\gets \peerid$
			\item \dots
			\item if $\SIGScheme.\SIGVerify(\peerpk[\peerid], \labelis\|\nonce_I\|\nonce_R\|\X\|\Y, \sigma')$\\
				and $\MACScheme.\MACVerify(\kt, \labelim\|\nonce_I\|\nonce_R\|\peerid, \tau')$ then
			\item \hindent \gamechange{if $\revltk_{\peerid} = \infty$ and}\newline
				\null \hindent\hindent \gamechange{$(\peerid, \labelrs\|\nonce_I\|\nonce_R\|\X\|\Y) \notin \Q_{S}$ then}
			\item \hindent \hindent \gamechange{$\bad[S] \gets \true$} \frame{; abort}
			\skipline
			\item[]
			\skipline
			\item \hindent $\st'.\status \gets \accepted$
			\item \hindent $\st'.\skey \gets \ks$
			\item else $\st'.\status \gets \rejected$
			\item return $(\st', m')$
		\end{oracle}
	\end{minipage}
	\begin{minipage}[t]{0.49\textwidth}
		\NewExperiment[$\lblGm{mac-bookkeeping}$, \frame{$\lblGm{mac-forgeries}$}]

		\begin{oracle}{$\RunInit(\id,\sk,\st,\peerpk,m)$}
% 			\item $(\nonce_R,\Y,\peerid,\sigma, \tau) \gets m$
% 			\item $(\nonce_I,\X,x) \gets st.\state$
% 			\item $\st'.sid \gets (\nonce_I,\nonce_R,\X,\Y)$
% 			\item $\st'.\peerid \gets \peerid$
% 			\item if $S[\st'.sid] \neq \bot$ then
% 			\item \hindent $\ks,\kt \gets S[\st'.sid]$
% 			\item else
% 			\item \hindent $mk\sample \bits^{hl}$
% 			\item \hindent if $H[\nonce_I\|\nonce_R\|\X\|\Y\|\Y^{\x}]\neq \bot$
% 			\item \hindent \hindent $mk \gets H[\nonce_I\|\nonce_R\|\X\|\Y\|\Y^{\x}]$
% 			\item \hindent $H[\nonce_I\|\nonce_R\|\X\|\Y\|\Y^{\x}] \gets mk$
% 			\item \hindent $\ks \gets \PRF_{mk}(0)$
% 			\item \hindent $\kt \gets \PRF_{mk}(1)$
			\item \dots
			\item if $\SIGScheme.\SIGVerify(\peerpk[\peerid], \labelrs\|\nonce_I\|\nonce_R\|\X\|\Y, \sigma)$\\
				and $\MACScheme.\MACVerify(\kt, \labelrm\|\nonce_I\|\nonce_R\|\peerid)$ then
			\item \hindent if $\revltk_{\peerid} = \infty$ and \newline
				\null \hindent \hindent $(\peerid, \labelrs\|\nonce_I\|\nonce_R\|\X\|\Y) \notin \Q_{S}$ then
			\item \hindent \hindent abort
			\item \hindent \gamechange{if $\S[\st'.\sid] \neq \bot$ and}\newline
				\null \hindent \hindent \gamechange{$(\st'.\sid, \labelrm\|\nonce_I\|\nonce_R\|\peerid) \notin \Q_M$ then}
			\item \hindent \hindent \gamechange{$\bad[M] \gets \true$} \frame{; abort}
			\item \hindent $\st'.\status \gets \accepted$
			\item \hindent $\st'.\skey \gets \ks$
			\item \hindent $\sigma' \gets \SIGScheme.\SIGSign(\sk, \labelis\|\nonce_I\|\nonce_R\|\X\|\Y)$
			\item \hindent $\Q_{S} \gets \Q_{S} \cup \{(\id, \labelis\|\nonce_I\|\nonce_R\|\X\|\Y)\}$
			\item \hindent $\tau' \gets \MACScheme.\MACTag(\kt, \labelim\|\nonce_I\|\nonce_R\|\id)$
			\item \hindent \gamechange{if $\S[\st'.\sid] \neq \bot$ then}
			\item \hindent \hindent \gamechange{$\Q_M \gets \Q_M \cup \{(\st'.\sid, \labelim\|\nonce_I\|\nonce_R\|\id)\}$}
			\item \dots
% 			\item \hindent $m' \gets (\id, \sigma', \tau')$
% 			\item else
% 			\item \hindent $m' \gets \bot$
% 			\item \hindent $\st'.\status \gets \rejected$
		\end{oracle}

		\ExptSepSpace

		\begin{oracle}{$\RunRespI(\id,\sk,\st,\peerpk,m)$}
% 			\item $(\nonce_I,\X) \gets m$
% 			\item $\nonce_R \sample \bits^{nl}$
% 			\item if $\nonce_I \in \N$ then $\bad \gets \true $; abort
% 			\item $N \gets N \cup \{\nonce_I\}$
% 			\item $y \sample \ZZ_p$
% 			\item $\Y \gets g^{\x}$
% 			\item if $\Y \in E$ then $\bad \gets \true$; abort
% 			\item $E \gets E \cup \{\Y\}$
% 			\item $\st'.sid \gets (\nonce_I, \nonce_R,\X,\Y)$
			\item \dots
			\item $\sigma \gets \SIGScheme.\SIGSign(\sk,\labelrs\|\nonce_I\|\nonce_R\|\X\|\Y)$
			\item $\Q_{S} \gets \Q_{S} \cup \{(\id, \labelrs\|\nonce_I\|\nonce_R\|\X\|\Y)\}$
			\item \dots
% 			\item $mk\sample \bits^{hl}$
% 			\item if $m \not\in Sent$ then
% 			\item \hindent if $H[\Y^{\x}\|\nonce_I\|\nonce_R\|\X\|\Y]\neq \bot$
% 			\item \hindent \hindent$mk \gets H[\Y^{\x}\|\nonce_I\|\nonce_R\|\X\|\Y]$
% 			\item \hindent $Q[\st'.sid] \gets {\X^y, mk}$
% 			\item \hindent $\ks \gets \PRF_{mk}(0)$
% 			\item \hindent $\kt \gets \PRF_{mk}(1)$
% 			\item else
% 			\item \hindent $\ks \sample \bits^{kl}$
% 			\item \hindent $\kt \sample \bits^{kl}$
			\item $\tau \gets \MACScheme.\MACTag(\kt, \labelrm\|\nonce_I\|\nonce_R\|\id)$
			\item \gamechange{if $\S[\st'.\sid] \neq \bot$ then}
			\item \hindent \gamechange{$\Q_M \gets \Q_M \cup \{(\st'.\sid, \labelrm\|\nonce_I\|\nonce_R\|\id)\}$}
			\item \dots
% 			\item $\st'.\state \gets (\nonce_I,\nonce_R,\X,\Y,\ks,\kt)$
% 			\item $m' \gets (\nonce_R, \Y, \id, \sigma, \tau)$
% 			\item if $m \in Sent$ then
% 			\item \hindent $S[\st'.sid] \gets (\ks,\kt)$
		\end{oracle}

		\ExptSepSpace

		\begin{oracle}{$\RunRespII(\id,\sk,\st,\peerpk,m)$}
% 			\item $(\nonce_I,\nonce_R,\X,\Y,\ks,\kt,\ke) \gets \st.\state$
% 			\item $(\peerid,sigma',\tau')\gets \ENCDec(\ke,m)$
% 			\item $\st'.\peerid\gets \peerid$
			\item \dots
			\item if $\SIGScheme.\SIGVerify(\peerpk[\peerid], \labelis\|\nonce_I\|\nonce_R\|\X\|\Y, \sigma')$\\
				and $\MACScheme.\MACVerify(\kt, \labelim\|\nonce_I\|\nonce_R\|\peerid, \tau')$ then
			\item \hindent if $\revltk_{\peerid} = \infty$ and \newline
				\null \hindent \hindent \hindent $(\peerid, \labelis\|\nonce_I\|\nonce_R\|\X\|\Y) \notin \Q_{S}$ then
			\item \hindent \hindent abort
			\item \hindent \gamechange{if $S[\st'.\sid] \neq \bot$ and}\newline
				\null \hindent \hindent \gamechange{$(\st'.\sid, (\peerid, \labelim\|\nonce_I\|\nonce_R\|\peerid) \notin \Q_M$ then}
			\item \hindent \hindent \gamechange{$\bad[M] \gets \true$} \frame{; abort}
			\item \hindent $\st'.\status \gets \accepted$
			\item \hindent $\st'.\skey \gets \ks$
			\item else $\st'.\status \gets \rejected$
			\item return $(\st', m')$
		\end{oracle}
	\end{minipage}

	\caption[]{%
		Games $\lblGm{signature-bookkeeping}$, $\lblGm{signature-forgeries}$, $\lblGm{mac-bookkeeping}$, and~$\lblGm{mac-forgeries}$ of the \SIGMAI proof.
		Changes in $\lblGm{signature-bookkeeping}$ and $\lblGm{mac-bookkeeping}$ are highlighted in \gamechange{gray},
		changes in $\lblGm{signature-forgeries}$ and $\lblGm{mac-forgeries}$ are highlighted in \frame{frames}.
	}
	\label{fig:SIGMAI-proof:game:signature-bookkeeping}
	\label{fig:SIGMAI-proof:game:signature-forgeries}
	\label{fig:SIGMAI-proof:game:mac-bookkeeping}
	\label{fig:SIGMAI-proof:game:mac-forgeries}
	
	\iffull\else %% appendix label if deferred to appendix
	\label{apxfig:SIGMAI-proof:signature-mac-games}
	\fi
\end{figure}
\end{collectinmacro}

\iffull %% deferred to appendix in non-full version
\SIGMIProofSignatureMAC
\fi

In the reduction, $\advB_3$ obtains all long-term public keys from the multi-user signature game and uses its signing oracles for any honest signature to be produced. It therefore makes $\qNewUser$ queries to $\New$ and one $\SIGSign$ query for each call to $\RunRespI$ or $\RunInit$, for at most $\qSend$ such queries.
It relays $\RevLongTermKey$ queries as corruptions in its multi-user game, making $\qRevLongTermKey$ corruption queries in total. 
When $\bad[S]$ is triggered, $\advB_3$ submits the triggering message and signature under the targeted (uncorrupted) public key as its forgery.
As the triggering message was not signed before under the corresponding secret key (and hence not queried to the signing oracle by~$\advB_3$), the forgery is valid and $\advB_3$ wins if $\bad[S]$ is set.
It follows that
\shortlongeqn[.]{
	\Pr[ \prevGm* \Rightarrow 1 ]
	\leq \Pr[ \curGm* \Rightarrow 1] + \genAdv{mu{\minus}EUF{\minus}CMA}{\SIGScheme}{\advB_3}(t_{\advB_3},\qNewUser,\qSend,\qSend,\qRevLongTermKey)
}
Except for the replacement of key generation, signatures, corruptions with oracle queries, the pseudocode of $\advB_3$ is identical to that of $\advA$ in game $\prevGm*$, so $t_{\advB_3} \approx t$. 

\proofngame[mac-bookkeeping]

In Game~$\curGm$ (Figure~\ref{fig:SIGMAI-proof:game:mac-bookkeeping}), we remove the now redundant $\bad[S]$ flag again,
and log all MAC tags generated by honest sessions with honest partners in a list $\Q_{M}$ (using, as before, the table $\S$ to determine whether a session has an honest partner).
We set a flag~$\bad[M]$ if a session with an honest partner receives a valid MAC tag which was not computed by any honest oracle. 
This bookkeeping is similar to the changes from $\lblGm{random-prf}$ to $\lblGm{signature-bookkeeping}$, but noting MAC tags instead of signatures.
As before, the bookkeeping itself does not affect the adversary's advantage:
\shortlongeqn[.]{
	\Pr[ \prevGm* \Rightarrow 1 ] = \Pr[ \curGm* \Rightarrow 1]
}


\proofngame[mac-forgeries]

In Game~$\curGm$ (Figure~\ref{fig:SIGMAI-proof:game:mac-forgeries}), we abort if the $\bad[M]$ flag is set to true.
Again applying the identical-until-bad lemma, we need to bound the probability of $\bad[M]$ being set in $\prevGm$,
which we do via the following reduction~$\advB_4$ to the multi-user EUF-CMA security of the MAC scheme~$\MACScheme$.%
% \footnote{%
% As for PRF security, multi-user EUF-CMA security for MACs reduces via a hybrid argument to the single-user setting,
% but specific constructions (like AMAC~\cite{EC:BelBerTes16}) achieve tight multi-user security.}

The reduction $\advB_4$ simulates~$\prevGm$ truthfully, except that for any session with honest origin partner (i.e., session with state~$\st$ where $\S[\st.\sid] \neq \bot$), $\advB_4$ does not compute~$\kt$ itself, but instead assigns an incremented user identifier~$i$ to this session's $\sid$ and computes any calls to $\MACTag$ or $\MACVerify$ using its corresponding oracles for user~$i$.
There is at most one query to $\NewUser$, and one each to $\MACTag$ and $\MACVerify$ for each of $\advA$'s queries to $\Send$. Hence $\advB_4$ makes at most $\qSend$ queries to each of these three oracles, and at most one query to $\MACTag$ and $\MACVerify$ per user in the mu-EUF-CMA game.
When $\bad[M]$ is triggered, $\advB_4$ submits the triggering message and MAC tag under user identifier~$i$ as its forgery.
In the simulation, sessions will share a user identifier $i$ if and only if they are partnered and would share keys in Game~$\prevGm$.
These keys are furthermore unique to one initiator and one responder session only, so consistency is maintained.
Furthermore, $\kt$ cannot be exposed (by $\RevLongTermKey$ or $\RevSessionKey$) to adversary~$\advA$, hence implicitly replacing it with the MAC game's oracles is sound, and $\advB_4$ makes no $\Corrupt$ queries. 
Except for oracle replacements, the pseudocode of $\advB_4$ is identical to that of $\advA$ in $\prevGm*$, so $t_{\advB_4}\approx t$.

If $\bad[M]$ is triggered, then $S[\st'.\sid]\neq \bot$, so $\st'.\sid$ corresponds to some user identifier~$i$ in the multi-user EUF-CMA game.
Additionally, a tag $\tau$ for message $m$ was verified under identity~$i$, and $(\st'.\sid,m)$ was not logged in $\Q_M$. Since $\advB_4$ logs $(\st'.\sid,m)$ every time it calls its $\MACTag$ oracle on the pair $(i,m)$, this call cannot have occurred.
Then $\tau$ is a valid forgery on $m$, which $\advB_4$ will output for user~$i$ to win the EUF-CMA game.
Thus,
\shortlongeqn[.]{
	\Pr[ \prevGm* \Rightarrow 1 ]
	\leq \Pr[ \curGm* \Rightarrow 1] + \genAdv{mu{\minus}EUF{\minus}CMA}{\MACScheme}{}(t_{\advB_4},\qSend,\qSend,1,\qSend,1,0)
}

\medskip

% ===========================================================================
% ===========================================================================
%
% Proof steps for explicit authentication
%
% ===========================================================================
% ===========================================================================
We can now consider the final advantage of an adversary playing Game~$\curGm$.
Adversary~$\advA$ has a non-zero advantage if in the final oracle query~$\Finalize(b')$%
\iffull
\begin{enumerate}
	\item $\Sound$ is false,
	\item $\ExplicitAuth$ is false, or
	%   \item $\Fresh$ is false, and $b = 0$.
	\item $\Fresh$ is true and $b' = b$.%
	\footnote{If $\Fresh$ is false, $b = b' = 0$ happens with probability~$\frac{1}{2}$, so $\advA$'s advantage is~$0$.}
\end{enumerate}
\else
,	$\Sound$ is false,
	$\ExplicitAuth$ is false, or
	$\Fresh$ is true and $b' = b$.%
	\footnote{If $\Fresh$ is false, $b = b' = 0$ happens with probability~$\frac{1}{2}$, so $\advA$'s advantage is~$0$.}
\fi

\paragraph{Soundness}
The flag $\Sound$ is set if (1) three honest sessions hold the same session identifier,
or if (2) two partnered sessions hold different session keys.

For (1):
No three honest sessions can share the same session identifiers,
as this would require a collision in either the contained initiator or responder nonce,
which is excluded by Game~$\lblGm{collisions}$.

For (2):
The session identifier includes both nonces~$\nonce$ and~$\nonce_R$ and DH shares~$\X$ and~$\Y$,
which together determine the derived master key~$\mk = \RO(\nonce_I \| \nonce_R \| \X \| \Y \| \Zz)$ (where $\Zz$ is the DH secret from $\X$ and $\Y$) and thus the session key.
Agreement on the session identifier hence implies deriving the same session key.

Hence, in Game~$\curGm$, $\Sound$ is always true.


\paragraph{Explicit authentication}
The predicate $\ExplicitAuth$ requires that for any session~$\pi_u^i$ accepting with a non-compromised peer~$v$,
there exists a partnered session~$\pi_v^j$ of user~$v$ with opposite role
which, if it accepts, has $u$ set as its peer.

The session $\pi_u^i$, prior to accepting, obtained a valid signature on $\pi^i_u.\sid$ and a label corresponding to a role $r \neq \pi_u^i.\role$. Due to Game~$\lblGm{signature-forgeries}$,  this signature must have been issued by an honest session~$\pi_v^j$ (since~$v$ was not compromised at this point).
All honest sessions sign their own~$\sid$ and a label corresponding to their own role, so $\pi_u^i.\sid = \pi_v^j.\sid$ and $\pi_u^i.\role = r \neq \pi_v^j.\role$ are satisfied.

Furthermore, when $\pi_v^j$ accepts, it must have received a valid MAC tag~$\tau$ on a label identifying an opposite-role session and that session's user identity, as well as their shared nonces.
Due to Game~$\lblGm{mac-forgeries}$, this MAC value must have been computed by an honest session holding the same nonces, as $\pi_v^j$ has an honest partner session and therefore $\S[\pi_v^j.\sid] \neq \bot$. 
Furthermore, by Game~$\lblGm{collisions}$, nonces do not collide and hence that session must have been~$\pi_u^i$, thus computing the MAC on user identity~$u$, which $\pi_v^j$ accordingly sets as peer identity.

Therefore $\ExplicitAuth$ is always true in $\curGm$.
Note that we did not require that the long-term key of user $u$ was uncorrupted, and we allow the adversary to continue interacting with sessions after compromise; hence covering key compromise impersonation attacks.


\paragraph{Guessing the challenge bit}

Finally, we have to consider $\advA$'s chance of guessing the challenge bit~$b$,
which it may only learn through~$\Test$ queries such that all tested sessions are fresh (i.e., $\Fresh$ is true).

The $\Fresh$ predicate being true ensures that all tested sessions (those in~$T$) accepted prior to their respective partner being corrupt.
Then, as $\ExplicitAuth$ is true, we have that for each tested session there exists an honest session with the same $\sid$ and different roles.
This session, by $\Fresh$, was not tested or revealed.
Being partnered, the first message~$(\nonce_I, \X)$ between these two honest sessions was not tampered with,
so in the responder session, whether it was the tested session or its partner, the master and session keys are sampled uniformly at random (due to Games~$\lblGm{responders-stop-programming}$ and~$\lblGm{random-prf}$).
Since the initiator session holds the same~$\sid$, it copied the responder's random session key (due to Game~$\lblGm{initiators-copy}$).
This random session key was not revealed in either of the two sessions (by $\Fresh$),
and hence from $\advA$'s perspective is a uniformly random and independent value.
In all $\Test$ oracle responses, $k_0$ and $k_1$ are hence identically distributed and so $\curGm$ is fully independent of~$b$.
It follows that the adversary~$\advA$ has no better than a $\frac{1}{2}$ probability of choosing $b'$ equal to $b$,
so
\shortlongeqn[,]{
	\Pr[\curGm\Rightarrow 1] = \frac{1}{2}
}
which concludes the proof.
\end{proof}

\else

%%%
%%% Proof sketch for non-full version
%%%

\begin{proof}[Proof outline]
We defer the detailed game-based description of the proof to the full version~\cite{EPRINT:DavGun20} and only outline its core and novel technical steps here.
We give a more detailed proof for our TLS~1.3 bound in Section~\ref{sec:tls-proof} which requires careful handling of the more complex key schedule, but is still structurally close.

The heart of the proof is the reduction to the strong DH problem.
In prior analyses of SIGMA and TLS~1.3, this reduction embeds a DH challenge into a single tested session.
This technique incurs a loss in the number of sessions because the reduction must guess in advance which session will be tested.
Translating techniques from Cohn-Gordon et al.~\cite{C:CCGJJ19}, we instead use the random self-reducibility of DH to embed a single challenge into every session which could possibly accept and be tested without violating the $\Fresh$ predicate.

We can divide all sessions into two categories:
(A) those who receive nonces or DH shares that have been tampered with by an adversary and
(B) those who receive unaltered nonces and DH shares from an honest peer.
Embedding a DH challenge into each of these types of sessions must be addressed differently.

If an adversary controls the DH share received by an honest session (category~(A)), it can compute that session's DH secret, from which are derived master key, session key, and MAC key.
If such a session has an embedded challenge, the simulator cannot honestly produce the proper master key.
Instead, it uses the strong DH oracle to detect if the adversary ever makes an RO query containing the session's nonces, DH shares, and the corresponding DH secret, and it programs the response to this query to maintain consistency.
The reduction also cannot produce the proper master key for sessions in category~(B); however, it can again use the strong DH oracle to detect RO queries containing a valid DH secret that would output the proper master key.
This secret can be used to extract the challenge secret and hence win the strong DH game.
One particular nuance here is that checking each RO query for every session's DH secret would lead to a quadratic loss in the number of strong DH oracle queries.
We maintain tightness by instead using the nonces and group elements in the RO query to identify the relevant sessions and efficiently program responses.

For sessions in category~(B), the master key is now chosen uniformly at random.
Invoking PRF security allows the session, traffic encryption, and MAC keys to be selected at random as well.
Each accepting session must receive a valid signature and MAC tag on its nonces and group elements.
Excluding the small probability that nonces and group elements collide between honest sessions, the adversary can only produce these by corrupting a long-term key or by forgery.
The former approach violates the $\Fresh$ predicate; the latter violates the EUF-CMA security of either the signature or MAC scheme.
Therefore, these sessions will accept only if they complete an entire protocol execution without tampering with an honest peer holding the same master key and thus same session key.

For sessions in category~(A), the master key may be known to the adversary.
However, these sessions still must receive a valid signature to accept.
Since the nonces and group elements were tampered with, no honest session will produce this signature.
Again, the adversary must resort to either corruption or forgery, hence violating either freshness or signature EUF-CMA security.
\end{proof}
\fi

% The game sequence of the proof is as follows:
% \begin{enumerate}
% \item Ensure no honest \nonces or group elements collide. (birthday attack probability)
% \item Initiator oracles copy their partner's session key if one exists instead of
% generating the key themselves. (this is undetectable).
% \item Responder oracles set the PRF key at random instead of $RO(g^{\x}y)$, unless $RO(g^{\x}y)$
% has already been queried. The H oracle then programs later $RO(g^{\x}y)$ queries to respond
% with the chosen PRF key. (this is undetectable)
% \item Initiator oracles with no partner set the PRF key at random if that is consistent
%  and program later $RO(g^{\x}y)$ queries as in step 3.
% \item Responder oracles with honest initiatiors choose the PRF key at random, without
%  programming. This is undetectable unless H receives a query with the correct DH value of
%  an honest initiator/responder pair. Call this event $F$.
%
%
% We can limit the probability of $F$ with an adversary against the strong DH assumption.
% The adversary, which we call B, given a challenge (\X,\Y) runs this simulation faithfully, except that every
% initiator sends a first message $\X*g^p$, where $g$ is the generator and $p$ is randomly
% chosen, and every initator responds with $\Y*g^q$, where $q$ is randomly chosen. If an
% initiator run by B receives a message $g^r$ from the adversary against the KE protocol, the adversary can derive the
% correct DH value and B cannot. However, B can still program the oracle by recognizing
%  queries RO(z) using the strong DH oracle. If the query $stDH_x(g^r,z*(g^r)^{-p})$ returns \true, then
%  $z$ is the DH value that the initiator with first message $X*g^p$ should have derived,
%  and B can return the PRF key chosen by $k$.
%
%  Then, if event $F$ occurs, and $RO(z)$ is queried, where
%  $z=g^{(x+p)(y+q)}$, B can recognize this using the strong DH oracle.
%  B queries $stDH_x(Y*g^q, z*Y^{-p} *g^{-pq})$, and the former equality holds if and only if this query returns \true.
%  Then the probability of $F$ is limited by the advan\taue of $B$.
%  \item Responders to honest initiators choose all of their intermediate keys (session keys, MAC keys)
%  at random instead of using a randomly-keyed PRF. This can be reduced directly to the multi-user security of the PRF, \hd{Is there a good source for multi-user PRF security?}
%
%  \end{enumerate}
%
%  For SIGMA-I, only the last step changes, where the encryption key is also chosen at random instead of
%  using a PRF. \hd{Do we need a separate game for the confidentiality of identities? We probably do.}



%%%% put out all figures
% \TODO{solve figure placement without \texttt{\textbackslash{}clearpage}.}
% \clearpage

\section{The TLS 1.3 Handshake Protocol}
\label{sec:tls}

The Transport Layer Security (TLS) protocol in version~1.3~\cite{rfc8446} bases its key exchange design (the so-called handshake protocol) on a variant of \SIGMAI.
Following the core \SIGMA design, the TLS~1.3 main handshake is an ephemeral Diffie--Hellman key exchange, authenticated through a combination of signing and MAC-ing the (full, hashed) communication transcript.%
\footnote{TLS~1.3 also specifies an abbreviated resumption-style handshake based on pre-shared keys; we focus on the main DH-based handshake in this work.}
Additionally, and similar to~\SIGMAI, beyond establishing the main (application traffic) session key, handshake traffic keys are derived and used to encrypt part of the handshake.

Beyond additional protocol features like negotiating the cryptographic algorithms to be used, communicating further information in extensions, etc.---which we do not capture here---, TLS~1.3 however deviates in two core cryptographic aspects from the more simplistic and abstract SIGMA(-I) design:
it hashes the communication transcript when deriving keys and computing signatures and MACs,
and it uses a significantly more complicated key schedule.
In this section we revisit the TLS~1.3 handshake and discuss the careful technical changes and additional assumptions needed to translate our tight security results for \SIGMAI to TLS~1.3's main key exchange mode.


\iffull
\subsection{Protocol Description}
\else
\subsubsection*{Protocol description\lncsdot}
\fi

We focus on a slightly simplified version of the handshake encompassing all essential cryptographic aspects for our tightness results.
% Due to the substantial complexity of the TLS~1.3 handshake, we focus here on a slightly simplified version of the protocol which encompasses all essential cryptographic aspects but abstracts away further technical protocol details that are not relevant for our analysis.
In particular, we only consider mutual authentication and security of the main application traffic keys%
\fullonly{ (see \cite{CCS:DFGS15,EPRINT:DFGS16,EuroSP:FisGue17,JC:DFGS21} for full computational, multi-stage key exchange analyses of the different modes with varying authentication)} and accordingly leave out some computations and additional messages.
\fullonly{To ease linking back to the underlying \SIGMAI structure, we describe the protocol in the following referencing back to the latter (cf.\ Section~\ref{sec:sigma}).}
We illustrate the handshake protocol and its accompanying key schedule in Figure~\ref{fig:tls-protocol}, the latter deriving keys in the extract-then-expand paradigm of the HKDF key derivation function~\cite{C:Krawczyk10}.%
\footnote{%
\fullonly{We follow the standard HKDF syntax: }%
$\HKDF.\Extract(\textit{XTS}\cab \textit{SKM})$ on input salt~$\textit{XTS}$ and source key material~$\textit{SKM}$ outputs a pseudorandom key~$\textit{PRK}$.
$\HKDF.\Expand(\textit{PRK}\cab \textit{CTXinfo})$ on input a pseudorandom key~$\textit{PRK}$ and context information~$\textit{CTXinfo}$ outputs pseudorandom key material~$\textit{KM}$.}

\begin{figure}[t!]
	\centering
	
	\begin{minipage}[t]{\iffull0.54\else0.6\fi\textwidth}
\scalebox{0.7}{
\begin{tikzpicture}
	% Set the X coordinates of the client, server, and arrows
	\edef\ClientX{0}
	\edef\ArrowLeft{0}
	\iffull
	\edef\ArrowRight{12}
	\edef\ServerX{12}
	\else
	\edef\ArrowRight{9.5}
	\edef\ServerX{9.5}
	\fi
	% Set the starting Y coordinate
	\edef\Y{0}

	% Draw header boxes
	\node [rectangle,draw,inner sep=5pt,right] at (\ClientX,\Y) {\textbf{Client}};
	\node [rectangle,draw,inner sep=5pt,left] at (\ServerX,\Y) {\textbf{Server}};

	\NextLine[2]
	
	\ClientAction{\TLSmsg{$\CHELO$}: $\nonce_C \sample \{0,1\}^{\nl}$, $X \gets g^x$}
% 	\NextLine
% 	\ClientAction{\TLSmsg{+~$\CKEYS$}: $X \gets g^x$}
	
	%%%%%%%%%%%%%%%%%%%%%%%%%%%%%%%%%%%%%%%%%%%
% 	\NextLine
% 	\SharedAction{$\ES \gets \HKDF.\Extract(0, 0)$}
% 	\NextLine
% 	\SharedAction{$\dES \gets \HKDF.\Expand(\ES, \inputlabel[3], \Hash(\texttt{""}))$}
	%%% we simplify this into a constant
	
	%%%%%%%%%%%%%%%%%%%%%%%%%%%%%%%%%%%%%%%%%%%%%%%%%%%%%%%%%%%%%%%%%%%%%%%%%%%%%%%%%%%%%%%%%%%%%%%%%%%%%%%
	\NextLine[0.75]
	\ClientToServer{\TLSmsg{$\CHELO$}}{}%, \TLSmsg{$\CKEYS$}}{}
	\NextLine[0.75]
	%%%%%%%%%%%%%%%%%%%%%%%%%%%%%%%%%%%%%%%%%%%%%%%%%%%%%%%%%%%%%%%%%%%%%%%%%%%%%%%%%%%%%%%%%%%%%%%%%%%%%%%
	
	\ServerAction{\TLSmsg{$\SHELO$}: $\nonce_S \sample \{0,1\}^{\nl}$, $Y \gets g^y$}
% 	\NextLine
% 	\ServerAction{\TLSmsg{+~$\SKEYS$}: $Y \gets g^y$}
	
	%%%%%%%%%%%%%%%%%%%%%%%%%%%%%%%%%%%%%%%%%%%%%%%%%%%%%%%%%%%%%%%%%%%%%%%%%%%%%%%%%%%%%%%%%%%%%%%%%%%%%%%
	\NextLine[0.75]
	\ServerToClient{\TLSmsg{$\SHELO$}}{}%, \TLSmsg{$\SKEYS$}}{}
	\NextLine[0.75]
	%%%%%%%%%%%%%%%%%%%%%%%%%%%%%%%%%%%%%%%%%%%%%%%%%%%%%%%%%%%%%%%%%%%%%%%%%%%%%%%%%%%%%%%%%%%%%%%%%%%%%%%
	
	\ClientAction{$\DHE \gets Y^x$}
	\ServerAction{$\DHE \gets X^y$}
% 	\NextLine
	\SharedAction{$\HS \gets \HKDF.\Extract(\constant[1], \DHE)$}
	\NextLine
	
% 	\SharedAction{$\CHTS \gets \HKDF.\Expand(\HS, \inputlabel[1], \Hash(\sCHELO \conc \sSHELO))$}
% 	\NextLine
% 	\SharedAction{$\SHTS \gets \HKDF.\Expand(\HS, \inputlabel[2], \Hash(\sCHELO \conc \sSHELO))$}
% 	\NextLine
	\SharedAction{$\CHTS/\SHTS \gets \HKDF.\Expand(\HS, \inputlabel[1]/\inputlabel[2], \Hash(\sCHELO \conc \sSHELO))$}
	\NextLine
	\SharedAction{$\dHS \gets \HKDF.\Expand(\HS, \inputlabel[3], \Hash(\texttt{""}))$}
	\NextLine
	
% 	\SharedAction{$\tkchs \gets \HKDF.\Expand(\CHTS, \inputlabel[4], \Hash(\texttt{""}))$}
% 	\NextLine
% 	\SharedAction{$\tkshs \gets \HKDF.\Expand(\SHTS, \inputlabel[4], \Hash(\texttt{""}))$}
	\SharedAction{$\tkchs/\tkshs \gets \HKDF.\Expand(\CHTS/\SHTS, \inputlabel[4], \Hash(\texttt{""}))$}
	\NextLine[0.5]
	
	%\NextLine
	%\SharedAction{$\tkchs, iv_{chs} \gets \HKDF.\kExp(\CHTS, \inputlabel[n], H_n)$}
	%\NextLine
	%\SharedAction{$\tkshs, iv_{shs} \gets \HKDF.\kExp(\SHTS, \inputlabel[n], H_n)$}
	%%%%%%%%%%%%%%%%%%%%%%%%%%%%%%%%%%%%%%%


% 	\NextLine
% 	\ServerAction{\TLSmsg{$\{\ENCEX\}$}: $\vec{ext}_S$}
% 	\NextLine
% 	\ServerAction{\TLSmsg{$\{\CERTR\}^*$}: $\inputlabel[n]$}
	\NextLine
	\ServerAction{\TLSmsg{$\mSCERT$}: $\pk_S$}
	\NextLine
	\ServerAction{\TLSmsg{$\mSCERTV$}: $\sSCERTV\gets\SIGScheme.\SIGSign(\sk_S, \inputlabel[5] \conc \Hash(\sCHELO \conc \ldots \conc \sSCERT))$}
	\NextLine
	\ServerAction{$\SFK \gets \HKDF.\Expand(\SHTS, \inputlabel[6],\Hash(\texttt{""}))$}
	\NextLine
	\ServerAction{\TLSmsg{$\SFIN$}: $\sSFIN \gets \HMAC(\SFK, \Hash(\sCHELO \conc \ldots \conc \sSCERTV))$}

	%%%%%%%%%%%%%%%%%%%%%%%%%%%%%%%%%%%%%%%%%%%%%%%%%%%%%%%%%%%%%%%%%%%%%%%%%%%%%%%%%%%%%%%%%%%%%%%%%%%%%%%
	\NextLine
	\ServerToClient{$\{\TLSmsg{\mSCERT}, \TLSmsg{\mSCERTV}, \TLSmsg{\SFIN}\}_{\tkshs}$}{}
	\NextLine
	%%%%%%%%%%%%%%%%%%%%%%%%%%%%%%%%%%%%%%%%%%%%%%%%%%%%%%%%%%%%%%%%%%%%%%%%%%%%%%%%%%%%%%%%%%%%%%%%%%%%%%%
	
	\ClientAction{\textbf{abort} if $\neg \SIGScheme.\SIGVerify(\pk_S, \inputlabel[5] \conc \Hash(\sCHELO \conc \ldots \conc \sSCERT), \sSCERTV)$}
	\NextLine
	\ClientAction{\textbf{abort} if $\sSFIN \neq \HMAC(\SFK, \Hash(\sCHELO \conc \ldots \conc \sSCERTV))$}
	\NextLine[1.5]
	
	%%%%%%%%%%%%%%%%%%%%%%%%%%%%%%%%%
	
% 	\SharedAction{$\MS \gets \HKDF.\Extract(\dHS, 0)$}
% 	\NextLine
	
% 	\AcceptStage{3}{$\CATS \gets \HKDF.\Expand(\MS, \inputlabel[xxx], \Hash(\sCHELO \conc \ldots \conc \sSFIN))$}
% 	%\NextLine
% 	%\SharedAction{$\tkcapp, iv_{capp} \gets \HKDF.\kExp(\CATS, \inputlabel[n], H_n)$}
% 	\NextLine[1.5]
% 	
% 	\AcceptStage{4}{$\SATS \gets \HKDF.\Expand(\MS, \inputlabel[xxx], \Hash(\sCHELO \conc \ldots \conc \sSFIN))$}
% 	%\NextLine
% 	%\SharedAction{$\tksapp, iv_{sapp} \gets \HKDF.\kExp(\SATS, \inputlabel[n], H_n)$}
% 	\NextLine[1.5]
% 	
% 	\AcceptStage{5}{$\EMS \gets \HKDF.\Expand(\MS, \inputlabel[xxx], \Hash(\sCHELO \conc \ldots \conc \sSFIN))$}
% 	\NextLine[1.5]
	
% 	\Encryption[<-,dashed]{record layer, AEAD-encrypted with key $\tksapp$ (optional)}{}
% 	\NextLine[1]
	
	%%%%%%%%%%%%%%%%%%%%%%%%%%%%%%%%%
	
	\ClientAction{\TLSmsg{$\mCCERT$}: $\pk_C$}
	\NextLine
	\ClientAction{\TLSmsg{$\mCCERTV$}: $\sCCERTV \gets \SIGScheme.\SIGSign(\sk_C, \inputlabel[7] \conc \Hash(\sCHELO \conc \ldots \conc \sCCERT))$}
	\NextLine
	\ClientAction{$\CFK \gets \HKDF.\Expand(\CHTS, \inputlabel[6],\Hash(\texttt{""}))$}
	\NextLine
	\ClientAction{\TLSmsg{$\CFIN$}: $\sCFIN \gets \HMAC(\CFK,  \Hash(\sCHELO \conc \ldots \conc \sCCERTV))$}

	%%%%%%%%%%%%%%%%%%%%%%%%%%%%%%%%%%%%%%%%%%%%%%%%%%%%%%%%%%%%%%%%%%%%%%%%%%%%%%%%%%%%%%%%%%%%%%%%%%%%%%%
	\NextLine
	\ClientToServer{$\{\TLSmsg{\mCCERT}, \TLSmsg{\mCCERTV}, \TLSmsg{\CFIN}\}_{\tkchs}$}{}
	\NextLine
	%%%%%%%%%%%%%%%%%%%%%%%%%%%%%%%%%%%%%%%%%%%%%%%%%%%%%%%%%%%%%%%%%%%%%%%%%%%%%%%%%%%%%%%%%%%%%%%%%%%%%%%

	\ServerAction{\textbf{abort} if $\neg \SIGScheme.\SIGVerify(pk_C, \inputlabel[7] \conc \Hash(\sCHELO \conc \ldots \conc \sCCERT), \sCCERTV)$}
	\NextLine
	\ServerAction{\textbf{abort} if $\sCFIN \neq \HMAC(\CFK, \Hash(\sCHELO \conc \ldots \conc \sCCERTV))$}
	\NextLine
	
	
	%% abstract 'application traffic secret' being derived here
	\NextLine[0.5]
	\SharedAction{$\MS \gets \HKDF.\Extract(\dHS, 0)$}
	\NextLine
	\SharedAction{$\ATS \gets \HKDF.\Expand(\MS, \inputlabel[8], \Hash(\sCHELO \conc \ldots \conc \sSFIN))$}
	
% 	\AcceptStage{6}{$\RMS \gets \HKDF.\Expand(\MS, \inputlabel[xxx], \Hash(\sCHELO \conc \ldots \conc \sCFIN))$}
% 	\NextLine[1.5]
	
% 	\Encryption[->,dashed]{record layer, AEAD-encrypted using key $\tkcapp$}{}
% 	\NextLine[1]
% 	\Encryption[<-,dashed]{record layer, AEAD-encrypted with key $\tksapp$}{}
	
	\NextLine[1.5]
	\SharedAction{\textbf{accept} with key~$\skey = \ATS$, session identifier~$\sid = (\nonce_C, \nonce_S, X, Y)$}
\end{tikzpicture}
}
\end{minipage}

	\hspace{-0.5cm}
	\begin{minipage}[t]{\iffull0.44\else0.35\fi\textwidth}
\iffull\else\resizebox{5cm}{!}{\fi %% resizebox in lncs
\begin{tikzpicture}[on grid]
	\tikzstyle{extract}=[Blue,draw,rectangle]
	\tikzstyle{expand}=[Red,draw,rectangle,rounded corners]
	\tikzstyle{context}=[latex-,dashed,font=\scriptsize]
	
	%
	% main secrets
	%
	\node (DHE) {$\DHE = g^{xy}$};
	\node [below=3 of DHE] (HS) {$\HS$};
	\node [below=3 of HS] (dHS) {$\dHS$};
	\node [below=3 of dHS] (MS) {$\MS$};
	
	%
	% main secret derivation
	%
	\node [extract,below=1.5 of DHE] (ext-HS) {$\sExtract$};
	\node [expand,below=1.5 of HS] (exp-dHS) {$\sExpand$};
	\node [extract,below=1.5 of dHS] (ext-MS) {$\sExtract$};
	
	\begin{scope}[-latex]
		\draw (DHE) -- (ext-HS);
		\draw (ext-HS) -- (HS);
		\draw (HS) -- (exp-dHS);
		\draw (exp-dHS) -- (dHS);
		\draw (dHS) -- (ext-MS);
		\draw (ext-MS) -- (MS);
	\end{scope}
	
	%
	% CHTS secrets and derivation
	%
	\node [right=2.75 of HS] (CHTS) {$\CHTS$};
	\node [right=2.5 of CHTS] (tkchs) {$\tkchs$};
	\node [below right=1 and 2.5 of CHTS] (CFK) {$\CFK$};
	
	\node [expand,right=1.5 of HS] (exp-CHTS) {$\sExpand$};
	\draw [context] (exp-CHTS) -- ++(0,0.75) node [above] {$\Hash(\sCHELO \conc \sSHELO)$};
	
	\node [expand,right=1.25 of CHTS] (exp-tkchs) {$\sExpand$};
	\node [expand,below right=1 and 1.25 of CHTS] (exp-CFK) {$\sExpand$};
	
	\begin{scope}[-latex]
		\draw (HS) -- (exp-CHTS);
		\draw (exp-CHTS) -- (CHTS);
		\draw (CHTS) -- (exp-tkchs);
		\draw (exp-tkchs) -- (tkchs);
		\draw (CHTS) |- (exp-CFK);
		\draw (exp-CFK) -- (CFK);
	\end{scope}
	
	%
	% SHTS secrets and derivation
	%
	\node [below right=2 and 2.75 of HS] (SHTS) {$\SHTS$};
	\node [right=2.5 of SHTS] (tkshs) {$\tkshs$};
	\node [below right=1 and 2.5 of SHTS] (SFK) {$\SFK$};
	
	\node [expand,below right=2 and 1.5 of HS] (exp-SHTS) {$\sExpand$};
	\draw [context] (exp-SHTS) -- ++(0,0.75) node [above] {$\Hash(\sCHELO \conc \sSHELO)$};
	
	\node [expand,right=1.25 of SHTS] (exp-tkshs) {$\sExpand$};
	\node [expand,below right=1 and 1.25 of SHTS] (exp-SFK) {$\sExpand$};
	
	\begin{scope}[-latex]
		\draw (HS) -- ++(0.75,0) |- (exp-SHTS);
		\draw (exp-SHTS) -- (SHTS);
		\draw (SHTS) -- (exp-tkshs);
		\draw (exp-tkshs) -- (tkshs);
		\draw (SHTS) |- (exp-SFK);
		\draw (exp-SFK) -- (SFK);
	\end{scope}
	
	%
	% ATS secret and derivation
	%
	\node [right=3 of MS] (ATS) {$\ATS$};
	
	\node [expand,right=1.5 of MS] (exp-ATS) {$\sExpand$};
	\draw [context] (exp-ATS) -- ++(0,0.75) node [above] {$\Hash(\sCHELO \conc \dots \conc \sSFIN)$};
	
	\begin{scope}[-latex]
		\draw (MS) -- (exp-ATS);
		\draw (exp-ATS) -- (ATS);
	\end{scope}
	
\end{tikzpicture}
\iffull\else}\fi %% resizebox in lncs
\end{minipage}

	
	%
	% Legend
	%
	\begin{minipage}{0.95\textwidth}
	\vspace{0.25cm}%
	\scriptsize%
	\begin{tabular}{lllll}
		\multicolumn{2}{l}{\textbf{Protocol flow legend}}
			&~~~& \multicolumn{2}{l}{\textbf{Message Abbreviations}} \\
		{\TLSmsg{$\mathtt{MSG}$}:~$Z$}	& message $\mathtt{MSG}$ sent, containing $Z$
			&& \TLSmsg{$\sCHELO$}	& \TLSmsg{$\CHELO$} \\
		$\{\TLSmsg{\mathtt{MSG}}\}_K$	& message AEAD-encrypted with~$K = \tkshs / \tkchs$
			&& \TLSmsg{$\sSHELO$}	& \TLSmsg{$\SHELO$} \\
			&&& \TLSmsg{$\sCCERT/\sSCERT$}~	& \TLSmsg{$\mathtt{Client/}\mSCERT$} \\
			&&& \TLSmsg{$\sCCERTV/\sSCERTV$}	& \TLSmsg{$\mathtt{Client/}\mSCERTV$} \\
			&&& \TLSmsg{$\sCFIN/\sSFIN$}	& \TLSmsg{$\mathtt{Client/}\SFIN$} \\
	\end{tabular}
% 	\begin{tabular}{ll}
% 		\TLSmsg{+~$\mathtt{MSG}$}	& message sent as extension within previous message \\
% 		\TLSmsg{$\mathtt{MSG}^*$}	& message only sent for intended client authentication \\
% 	\end{tabular}
	\end{minipage}
	
	\caption{%
		The simplified TLS~1.3 main Diffie--Hellman handshake protocol (left) and key schedule (right).
		Values~$\inputlabel[i]$ and~$\constant[i]$ indicate bitstring labels, resp.\ constant values, (distinct per~$i$).
		Boxes $\sExtract$ and $\sExpand$ denote $\HKDF$ extraction resp.\ expansion, dashed inputs to~$\sExpand$ indicating context information (see protocol figure for detailed computations).
	}
	\label{fig:tls-protocol}
\end{figure}

In the TLS~1.3 handshake, the client acts as initiator and the server as responder.
Within $\HELO$ messages, both send nonce values~$\nonce_C$ resp.\ $\nonce_S$ together with ephemeral Diffie--Hellman shares~$g^x$ resp.\ $g^y$.
Based on these values, both parties extract a handshake secret~$\HS$ from the shared DH value~$\DHE = g^{xy}$ using $\HKDF.\Extract$ with a constant salt input.%
\fullelse{\footnote{This salt input becomes relevant for pre-shared key handshakes, but in the full handshake takes the constant value~$\constant[1] = \Expand(\Extract(0,0), \texttt{"derived"}, \Hash(\texttt{""}))$.}}{ }
In a second step, client and server derive their respective handshake traffic keys~$\tkchs$, $\tkshs$ and MAC keys~$\CFK$, $\SFK$ through two levels of $\HKDF.\Expand$ steps from the handshake secret~$\HS$, including in the first level distinct labels and the hashed communication transcript~$\Hash(\sCHELO \conc \sSHELO)$ so far as context information.

The handshake traffic keys are then used to encrypt the remaining handshake messages.
First the server, then the client send their certificate (carrying their identity and public key), a signature over the hashed transcript up to including their certificate\fullonly{ ($\Hash(\sCHELO \conc \dots \conc \sSCERT)$, resp.\ $\Hash(\sCHELO \conc \dots \conc \sCCERT)$)}, as well as a MAC over the (hashed) transcript up to incl.\ their signatures\fullonly{ ($\Hash(\sCHELO \conc \dots \conc \sSCERTV)$, resp.\ $\Hash(\sCHELO \conc \dots \conc \sCCERTV)$)}.
Note the similarity to \SIGMAI here:
each party signs both nonces and DH values (within $\sCHELO \conc \sSHELO$, modulo transcript hashing) together with a unique label,
and then MACs both nonces and their own identity (the latter being part of their certificate).%
\fullelse{\footnote{Instead of using distinct labels for the client and server MAC computations, TLS~1.3 employs distinct MAC keys for client and server, achieving separation between the two MAC values this way.}}{ }
The application traffic secret~$\ATS$---which we treat as the session key~$\skey$, unifying secrets of both client and server---is then derived from the master secret~$\MS$ through $\HKDF.\Expand$ with handshake context up to the $\SFIN$ message.
The master secret in turn is derived through (context-less) $\Expand$ and $\Extract$ from the handshake secret~$\HS$.


\iffull
\subsection{Handling the TLS~1.3 Key Schedule}
\else
\subsubsection*{Handling the TLS~1.3 key schedule\lncsdot}
\fi

\iffull
As mentioned before, the message flow of the TLS~1.3 handshake relatively closely follows the \SIGMAI design~\cite{C:Krawczyk03,SIGMA-fullversion} (cf.\ Figure~\ref{fig:sigma-protocol}):
after exchanging nonces and DH shares (in $\HELO$) from both sides, the remaining (encrypted) messages carry identities ($\CERT$), signatures over the nonces and DH shares ($\CERTV$), and MACs over the nonces and identities ($\FIN$).

What crucially differentiates the TLS~1.3 handshake from the basic \SIGMAI design (beyond putting more under the respective signatures and MACs, which does not negatively affect the key exchange security we are after) is the way keys are derived.
While \SIGMAI immediately derives a master key through a random oracle with input \emph{both} the shared DH secret \emph{and} the session identifying nonces and DH shares,
TLS~1.3 separates them in its HKDF-based extract-then-expand key schedule:
The core secrets---handshake secret ($\HS$) and master secret ($\MS$)---are derived without further context purely from the shared DH secret~$\DHE = g^{xy}$ (beyond other constant inputs).
Only when deriving the specific-purpose secrets---handshake traffic keys ($\tkchs$, $\tkshs$), MAC keys ($\CFK$, $\SFK$), and session key ($\ATS$)---is context added to the key derivation, including in particular the nonces and DH shares identifying the session.
To complicate matters even further, this context is hashed before entering key derivation (or signature and MAC computation), and the final session key~$\ATS$ depends on more messages than just the session-identifying ones.
Since our tighter security proof for the SIGMA(-I) protocol (cf.\ Section~\ref{sec:sigma-proof}) heavily makes use of (exactly) the session identifiers being input together with DH secrets to a random oracle when programming the latter,
the question arises how to treat the TLS~1.3 key schedule when aiming at a similar proof strategy.

In their concurrent work, Diemert and Jager~\cite{JC:DieJag20} satisfy this requirement by modeling the full derivation of each stage key in their multi-stage treatment as a separate random oracle.
This directly connects inputs to keys, but results in a monolithic random oracle treatment of the key schedule which loses the independence of the intermediate $\HKDF.\Extract$ and $\HKDF.\Expand$ steps in translation.

We overcome the technical obstacle of this linking while staying closer to the structure of TLS~1.3's key schedule.
First of all, we directly model both $\HKDF.\Extract$ and $\HKDF.\Expand$ as individual (programmable) random oracles,
which leads to a slightly less excessive use of the random oracle technique.
We then have to carefully orchestrate the programming of intermediate secrets and session keys in a two-level approach,
connecting them through constant-time look-ups,
and taking into account that inputs to deriving the session keys depend on values established through the intermediate secrets (namely, the server's $\FIN$ MAC).
Along the way, we separately ensure that we recognize any hashed inputs of interest that the adversary might query to the random oracle, without modeling the hash function~$\Hash$ as a random oracle itself.
By tracking intermediate programming points (especially $\HS$ and~$\MS$) in the random oracles, we recover the needed capability of linking sessions and their session identifiers and DH shares exchanged to the corresponding session keys.
This finally allows us to again (efficiently) determine when and on what input to query the strong Diffie--Hellman oracle when programming challenge DH shares into the TLS~1.3 key exchange execution during the proof.

\else

What crucially differentiates the TLS~1.3 handshake from the basic \SIGMAI design is the way keys are derived.
While \SIGMAI derives its master key through a random oracle with input \emph{both} the shared DH secret \emph{and} the session identifying nonces and DH shares,
TLS~1.3 separates them in its HKDF-based extract-then-expand key schedule:
The core $\HS$ and $\MS$ secrets are derived \emph{without} further context purely from the shared DH secret~$\DHE = g^{xy}$.
Only when deriving the specific-purpose secrets---handshake traffic keys, MAC keys, and the session key~$\ATS$---are the nonces and DH shares add as session-identifying context.
To complicate matters even further, this context is hashed and the final session key~$\ATS$ depends on more messages than just the session-identifying ones.
Recall that the original techniques by Cohn-Gordon et al.~\cite{C:CCGJJ19} heavily relies on (exactly) the session identifiers being input together with DH secrets to a random oracle when programming the latter,
impeding a more direct application like for \SIGMAI.
In their concurrent work, Diemert and Jager~\cite{JC:DieJag20} satisfy this requirement by modeling the full derivation of each stage key in their multi-stage treatment as a separate random oracle.
This directly connects inputs to keys, but results in a monolithic random oracle treatment of the key schedule which loses the independence of the intermediate $\HKDF.\Extract$ and $\HKDF.\Expand$ steps in translation.
As we will show next, we overcome the technical obstacle of this linking while directly modeling $\HKDF.\Extract$ and $\HKDF.\Expand$ as individual random oracles,
carefully orchestrating the programming of intermediate secrets and session keys and connecting them through constant-time look-ups.
This leads to a slightly less excessive use of the random oracle technique and allows us to stay much closer to the structure of TLS~1.3's key schedule.

\fi


% \hrulefill
%
%\subsection*{\color{Red}Problems with the TLS key schedule compared to SIGMA}
%
%\begin{itemize}
%	\item Proof strategy needs $(n_C, n_S, g^x, g^y, g^{xy})$ in the RO query to (efficiently!) do StrongDH look-ups.
%	\item In TLS, we have $HS \gets Extract(0, g^{xy})$ which doesn't allow for this; the handshake transcript (ClientHello+ServerHello), which include $(n_C, g^x)$ resp.\ $(n_S, g^y)$ only enter key derivations in Expand calls.
%	\item This captures the derivation of $tk_{hs}$ ($k_e$ in \SIGMAI notation) and the MAC keys ($k_t$ in SIGMA), but not the expansion of~$dHS$ towards the master secret is \emph{without} transcript.
%	\item However, the final session key $tk_{app}$ ($k_s$ in SIGMA) will be derived from $MS$ \emph{with} transcript.
%\end{itemize}
%
%
%\subsection*{\color{Red}Updated proof Idea}
%
%\begin{itemize}
%	\item We will have two random oracles:
%	\begin{enumerate}
%		\item $RO_1 = \Extract$ will map $\DHE$ values to $\HS$ (and later derived $\dHS$ to $\MS$), we keep (two) tables mapping $\HS$ and $\MS$ values back to~$\DHE$ ($Q1[\HS] = \DHE$, resp.\ $Q2[\MS] = \DHE$), so that we know which $\DHE$ inputs to $\Expand$ originated from.
%		
%		\item $RO_2 = \Expand$ will capture expansion from $\HS$ and $\MS$. Here is where context information ($g^x, g^y$ through $\CHELO$, $\SHELO$ message inputs) enters the key derivation.
%	\end{enumerate}
%	
%	\item Comparing with SIGMA, we treat HS as the master key. While this is not derived from sid, we can figure out what sid it belongs to because immediately afterwards it is used within Expand together with the sid.
%	
%% 	\item $MS \gets Extract(dHS, 0)$ is computed regularly from $dHS = RO_1(Z, ...)$ (and we immediately do this when a $dHS$ call in $RO_1$ happens), but we will store the intermediary mapping of values $Z$ to $MS$ as $Q[MS] = Z$. (Read as: $MS$ was derived from $Z$ in $RO_1$.)
%	
%	\item Note: $\Hash$ is \emph{not} a random oracle (but just collision resistant). In order to map hash inputs to $RO_2$ back to the session identifiers, we keep tables mapping all hashed transcripts, $H_i = \Hash(\sCHELO \conc \dots)$, of sessions to their inputs (or at least to the nonces and $g^x, g^y$), so we know when some $H_i$ is used for which we need to lookup the DH share inputs, and the sid it corresponds to.
%	
%	\item When the adversary calls $RO_2(X,  l)$ for some $l$ containing (the hash of) $g^x, g^y$, we can detect if we've seen $X$ (as key to $Q1$/$Q2$) being derived before, and then can call our StrongDH oracle on $(g^x, g^y, Q[X])$.
%	
%	\item Adversary asking~$RO_2$ on some $X$ that results from an encoded challenge \emph{before} the challenge is encoded---bound by probability of $|RO_1|$, but also: isn't it still fine if we detect this later and call our StrongDH oracle then? \fg{Probably not, later detecting means an adversary might have been able to detect this already?}
%	
%	\item Probability that $X$ of any $RO_2$ is later hit by a $RO_1$ output should be birthday-bounded by \#queries to $RO_1$ squared / $|RO_1|$.
%\end{itemize}
%
%
%

\section{Tighter Security Proof for the TLS~1.3 Handshake}
\label{sec:tls-proof}

We now give our second main result, the tighter-security bound \fullelse{for the TLS~1.3 handshake protocol}{for TLS~1.3}.

\begin{theorem}
	\label{thm:tls}
	Let $\advA$ be a key exchange security adversary against the TLS~1.3 handshake protocol as specified in Figure~\ref{fig:tls-protocol} based on a hash function~$\Hash$, a signature scheme~$\SIGScheme$, and a group~$\group$ of prime order~$p$,
	and let the $\HKDF$ functions~$\Extract$ and $\Expand$ in the protocol be modeled as (independent) random oracles~$\RO_1$, resp.~$\RO_2$. 
	For any $(t, \qNewUser, \qSend, \qRevSessionKey, \qRevLongTermKey, \qTest)$-$\KESEC$-adversary against \SIGMAI making at most $\qRO$ queries to the random oracle,
	we give algorithms~$\advB_1$, $\advB_2$, $\advB_3$, and $\advB_4$ in the proof,
	with running times~$t_{\advB_i} \approx t$ (for $i = 1,3,4$) and $t_{\advB_2} \approx t + 2\qRO \log_2 p$ close to that of $\advA$, such that
	\begin{collectinmacro}{\TLSBound}{}{} %%% ===== COLLECT BEGIN =====
	\begin{align*}
		\Adv&^{\KESEC}_{\mTLS}(t, \qNewUser,\qSend, \qRevSessionKey,\qRevLongTermKey,\qTest)
			\leq
% 			\frac{\qSend^2}{2^{\nl}} + \frac{\qSend^2}{2p} + \frac{\qSend^2}{2^{\nl}\cdot p} % sid collisions -- old game hops
			\frac{3\qSend^2}{2^{\nl+1}\cdot p} % sid collisions
			+ \Adv^{\COLL}_{\Hash}(t_{\advB_1})\\ % hash collisions
			&+  2 \cdot \Adv^{\strongDH}_{\group}(t_{\advB_2},\qRO)
			+\frac{\qRO\cdot \qSend}{2^{\kl-1}}
			+ \Adv^{\muEUFCMA}_{\SIGScheme}(t_{\advB_3},\qNewUser,\qSend,\qSend,\qRevLongTermKey)\\
			&+ \Adv^{\muEUFCMA}_{\HMAC}(t_{\advB_4}, \qSend,\qSend,1, \qSend,1,0).
	\end{align*}
	\end{collectinmacro}
	\TLSBound
	Here,
	$\nl = 256$ is the nonce length in TLS~1.3,
	$\kl$ is the output length of~$\RO_2 = \HKDF.\Expand$,
	$\group$ is the used Diffie--Hellman group of prime order~$p$,
	and $\qSend \cdot \qRO \leq 2^{\kl-3}$.%
	\footnote{We simplify the factor on $\Adv^{\strongDH}_{\group}$ to~$2$ by assuming $\qSend \cdot \qRO \leq 2^{\kl-3}$, which is true for any reasonable real-world parameters.
	See the proof for the exact bound.}
\end{theorem}

\iffull
\else

%%%%%%%%%%%%%%%%%%%%%%%%%%%%%%%%%%%%%%%%%%%%%%%%%%%%%%%%%%%%%%%%%%%%%%%%%%%%%%%%%%%%%%%%%%%%%%%%
%% TLS Proof VERY short version for lncs / non-full
%%%%%%%%%%%%%%%%%%%%%%%%%%%%%%%%%%%%%%%%%%%%%%%%%%%%%%%%%%%%%%%%%%%%%%%%%%%%%%%%%%%%%%%%%%%%%%%%

% \begin{proof}[Proof outline]
\subsubsection*{Proof idea\lncsdot}
Let us first outline the core and novel technical steps, before we give some more proof details below;
for space reasons we defer the full proof to the full version~\cite{EPRINT:DavGun20}.
We note that as all keys in the \SIGMA exchange are derived from the master key $mk$, which is itself derived from the shared Diffie--Hellman secret, all intermediate keys in TLS~1.3 are derived from the handshake secret $\HS$, which is derived directly from the shared Diffie--Hellman secret $\DHE$.
Embedding a DH challenge into all sessions robs the reduction of the ability to compute $\HS$;
as in the \SIGMA proof, we will need to use the strong DH oracle to detect and program queries that would output an inconsistent value of $\HS$.
Since $\HS$ is derived without context, a naive method would have to check every input to $\HKDF.\Extract$ against the DH shares received by each session, which would however result in a non-tight, quadratic runtime loss. 

We instead leverage that the handshake secret~$\HS$ is an internal value, not exposed by any oracle.
The adversary hence cannot detect an inconsistent $\HS$ value until it makes the entire chain of queries leading to one of the keys $\tkshs$, $\tkchs$, $\CFK$, $\SFK$, or $\ATS$ used in $\Send$, $\RevSessionKey$, and $\Test$ responses.
Our reduction prudently sets up a separate bidirectional lookup table for each ``link'' in that chain.
The adversary can make the RO queries in the chain in any order; we need only program the last one for consistency, at which time we have seen the session's DH secret, nonces, and group elements as query inputs.
Linking the output of one key-derivation step to the input of the next this way, the reduction can identify the relevant sessions using only constant time and linear space.
Together with a careful argument that the attacker is unlikely to guess an intermediate chain value, this allows us to treat $\HKDF.\Extract$ and $\HKDF.\Expand$ as two individual random oracles.
Thereby, we stay close to how HKDF is used in TLS~1.3 and obtain two compact strong-DH bounds.
%
% One subtlety that is unique to our proof is that the reduction to the strong DH problem no longer maintains perfect consistency.
% Because key derivation happens as a multi-stage process, it is possible, though unlikely, that the adversary will randomly guess one of the intermediate values of key derivation without making the corresponding RO query.
% If this happens, the reduction does not have a complete ``chain'' and cannot identify which sessions' shared DH secret has been discovered.
% It therefore can neither program the ROs appropriately nor extract the challenge secret from the shared secret, because it does not know which keys are known or which randomness was used by the compromised sessions.
% This flaw leads to a loss of $\frac{2^{kl}}{2^{kl} - \qRO \cdot \qSend}$ in the reduction.
% We assume, as is true for any reasonable choices of the symmetric parameter $kl$, that $\qRO \cdot \qSend \leq 2^{kl -2}$.
% This allows us to upper bound this factor with a much simpler constant.
% \end{proof}

\bigskip

Now we give a more precise view of the structure of our proof, with a particular focus on nonstandard techniques % and the points at which the proof differs from that for \SIGMAI due to aspects unique to TLS~1.3.
and the critical random oracle programming in the reduction step to the strong Diffie--Hellman problem, handling the complexity of TLS~1.3's key schedule.

\begin{proof}
	\let\proofsep\medskip %% tighter spacing in here
	
	\startproof{TLS-short}
	
	We develop the bound via a series of code-based game hops.
	
	\proofngame[initial-game]
	
	The first game $\curGm$ is the key exchange security game (cf.\ Figure~\ref{fig:AKE-security}) for the TLS~1.3 handshake protocol (Figure~\ref{fig:tls-protocol}).
	%, where the formal algorithms~$\KEActivate$ and $\KERun$ execute the steps shown in Figure~\ref{fig:tls-protocol} as well as appropriate maintenance of the session state.
	%
	%execute e capture the two protocol steps for each role in formal algorithms~$\KEActivate$, $\RunRespI$, $\RunInit$, and $\RunRespII$.
	% We briefly define the formal algorithms $\KEActivate$, $\RunRespI$, $\RunInit$, and $\RunRespII$ (following the formal algorithm structure of \SIGMAI from Figure~\ref{fig:sigma-formal}) as follows.
	% Let $\KEActivate$ include all of the client computation up to its first sent message, and $\RunInit$ cover the rest of the client's actions.
	% Let the $\RunRespI$ algorithm perform all of the server's computation up to the generation of its second message,
	% and let the $\RunRespII$ algorithm include all of the server's remaining computation.
	% After acceptance, servers set their peer id to that contained in~$\CCERT$, clients set their peer identity to that in~$\SCERT$, and all sessions set their session key $\skey \gets \ATS$.
	% With these definitions in place,
	So,
	$\Pr[\curGm*\Rightarrow 1] = \Pr[ \Gm^{\KESEC}_{\TLS,\advA} \Rightarrow 1 ]$.
	
	
	\proofngames[collisions]{4}
	
	Over the next four games we ensure the uniqueness of each session's protocol transcript by aborting if an honest session chooses a nonce and DH share that have already been sent or received by another honest session, or if a collision occurs in the hash function $\Hash$. We limit the probability of nonce and DH share collisions using a union bound, and give a simple reduction $\advB_1$ to the collision resistance of the hash function~$\Hash$. We also lazily sample the random oracles $\RO_1$ and $\RO_2$ using internal tables $H_1$ and $H_2$.
	Excluding collisions, we obtain the bound
	$\Pr[\prevGm*\Rightarrow 1] - \Pr[\curGm\Rightarrow 1] \leq \frac{3\qSend^2}{2^{\nl+1}\cdot p}% sid collisions
	+ \Adv^{\COLL}_{\Hash}(t_1)$.
	
	
	\proofngames[copy-keys]{2}
	
	Following the technique of~\cite{C:CCGJJ19}, we let initiator sessions in category~(A) copy session, MAC, and traffic encryption keys from their partners via a table indexed by session IDs. 
	In TLS~1.3, there are two encryption keys $\tkshs$ and $\tkchs$, and two MAC keys $\SFK$ and $\CFK$ to copy. One significant difference from both~\cite{C:CCGJJ19} and our \SIGMAI proof is that the session key $\ATS$ now depends on the messages $\sSCERT$, $\sSCERTV$, and $\sSFIN$. We have not yet ensured that partnered sessions agree on these values. Therefore honest initiators will only copy $\ATS$ from their partners if they received the exact same $(\sSCERT\cab \sSCERTV\cab \sSFIN)$ sent by their partner, which they check via an internal look-up table. Otherwise, $\ATS$ is still computed as in previous games. 
	Since keys are only copied when partners agree on all of the information entering the key derivation function, this change is unobservable to~$\advA$, hence
	$\Pr[\curGm*\Rightarrow 1] = \Pr[\prevGm*\Rightarrow 1]$.
	
	
	\proofngames[programming]{2}
	
	These two games contain both the most critical step and the one that diverges the most from the \SIGMAI proof.
	We let all category~(A) sessions that are not already copying their keys pick the handshake traffic keys $\SHTS$ and $\CHTS$, and the session key $\ATS$ uniformly at random, checking for consistency with the random oracle $\RO_2$ and retroactively programming it when necessary.
	(Category (A) initiator sessions who do not copy $\ATS$ due to tampering sample only $\ATS$.) 
	Then, we eliminate the consistency check and let these sessions' handshake traffic keys and session key be uniformly random and inconsistent with the adversary's queries to $\RO_2$. 
	We argue that the adversary can only detect this inconsistency if it queries $\RO_2$ on the correct input to derive one of $\SHTS$, $\CHTS$, or $\ATS$ for a category~(A) session, an event we refer to as event~$F$. 
	
	We give a reduction $\advB_2$ to the strong DH assumption in group~$\group$ which wins with high probability if event $F$ occurs. 
	Given a challenge $C,D$, algorithm $\advB_2$ simulates Game $7$. It embeds $C$ in the DH shares of all initiators and $D$ in the DH shares of all category~(A) responders. 
	Because $\advB_2$ cannot compute the DH secret for embedded sessions, it uses its $\stDH$ oracle to catch and program all queries to $\RO_2$ which are dependent on this secret. When event~$F$ occurs, $\advB_2$ uses its own randomness to extract the challenge DH secret from the DH secret contained in the query that triggered event~$F$. 
	In addition to the details covered in Section~\ref{sec:tls-proof}, the reduction has a few nuances:
	\begin{enumerate}
		\item If for some category~(A) session, $\advA$ can guess without making the corresponding query any of the intermediate values $\HS = \RO_1(\constant[1],\DHE)$, $\dHS = \RO_2(\HS,\inputlabel[3],\Hash(``"))$, or $\MS = \RO_1(\dHS,0)$, where $\DHE$ is the DH secret associated to some pair of embedded shares $(\X,\Y)$ chosen by honest sessions, then it can trigger event $F$ without ever submitting $\DHE$ to an oracle.
		Without knowing $\DHE$, $\advB_2$ cannot detect this query, so it does not program $\RO_2$ appropriately and the simulation fails.
		$\advB_2$ does not itself compute $\HS$, $\dHS$, or $\MS$ for category~(A) sessions, so if $\advA$ does not make the appropriate queries than all three values are uniformly random and each can be guessed with probability at most~$\frac{\qRO \cdot \qSend}{2^{\kl}}$.  
		
		\item In TLS~1.3, the context string including the Diffie--Hellman shares is hashed with $\Hash$ before it enters the key derivation, so $\advB_2$ cannot directly associate an $\RO_2$ query with an honest~$\sid$.
		We address this by logging hash computations of honest sessions in a reverse look-up table~$R$.
		Then in the $\RO_2$ oracle, $\advB_2$ can use $R$ to efficiently find the context associated with a particular query.
		
	\end{enumerate}
	When $\qRO \cdot \qSend \leq 2^{\kl - 3}$, we obtain the bound
	$\Pr[\prevGm*\Rightarrow 1 ] - \Pr[\curGm*\Rightarrow 1] \leq 2\cdot \Adv^{\strongDH}_{\group}(t_{\advB_2},\qRO) +\frac{\qRO \cdot \qSend}{2^{\kl}}$.
	
	The reduction $\advB_2$ queries the $\stDH$ oracle at most once for each query to $\RO_2$ query and once more when event $F$ occurs.
	Computing the input to each $\stDH$ query requires 1 multiplication and one exponentiation in the base group, which can be done using $1+2\log_2 p$ total group operations. In our runtime analysis, we count each group operation as $1$ step, so $t_{\advB_2} \approx t + 2 \qRO\log_2 p$.
	
	\proofngame[uniform-keys] 
	In game $\curGm$, category~(A) sessions sample all encryption and MAC keys uniformly at random. This is distinguishable only if the adversary can query $\RO$ on a string containing one of the random values $\SHTS$ or $\CHTS$, so by the birthday bound
	$\Pr[\prevGm*\Rightarrow 1 ] - \Pr[\curGm*\Rightarrow 1] \leq \frac{\qRO\cdot \qSend}{2^{\kl}}$.
	
	\proofngames[sigs-and-macs]{4}
	In the remaining games, we eliminate signature and MAC forgeries via straightforward reductions $\advB_3$ and $\advB_4$ to the multi-user EUF-CMA security of $\SIGScheme$ and $\MACScheme$. This gives the bound 
	$
	\Pr[\prevGm*\Rightarrow 1 ] - \Pr[\curGm*\Rightarrow 1] \leq
	\Adv^{\muEUFCMA}_{\SIGScheme}(t_{\advB_3\cab \qNew\cab \qSend\cab \qSend\cab \qRevLongTermKey})
	+ \Adv^{\muEUFCMA}_{\MACScheme}(t_{\advB_4}\cab \qSend\cab \qSend\cab 1\cab \qSend\cab 1\cab 0).
	$
	
	Finally, we argue that $\advA$ has advantage $0$ in game $\curGm$, using logic similar to that in our $\SIGMAI$ proof, with two slight differences: 1. Partnered sessions no longer use labels to distinguish their MAC tags; instead we note that  messages tagged by initiator sessions are strictly longer than messages tagged by responder sessions. 2. We cannot immediately conclude that partnered sessions agree on the same session key because the session key $\ATS$ relies on values that are not contained in the session identifier. However, since we have excluded MAC forgeries, all the information entering the derivation of $\ATS$ is authenticated by the responder session's MAC tag.
\end{proof}

\fi

\iffull
\begin{collectinmacro}{\TLSProofFull}{}{} %%% ===== COLLECT BEGIN =====
\startproof{TLS}
We prove our bound by making an incremental series of changes to the key exchange security game and limiting the amount that each change affects the success probability of $\advA$. 
\newcounter{advB-TLS}

\proofngame[tls-start]

The initial game, Game $\curGm$, is the key exchange security game for TLS played by $\advA$, using the implicit $\KEKGen$, $\KEActivate$, and $\KERun$ routines defined by the TLS protocol specification on the left side of Figure~\ref{fig:tls-protocol}.
(In this game, $\HKDF.\Extract$ and $\HKDF.\Expand$ are modeled by random oracles $\RO_1$ and $\RO_2$ respectively.)
By definition, 
\[\Pr[\curGm\Rightarrow 1] = \Pr[ \Gm^{\KESEC}_{\TLS,\advA} \Rightarrow 1 ].\]

\proofngame[tls-log-nonces]

In game $\prevGm$, we start logging the nonces and group elements chosen by honest sessions. Whenever two honest sessions choose the same nonces or group elements, we set a flag $\bad[C]$. Whenever an honest responder session chooses a nonce and group element that have already been received by another session, we set a flag $\bad[O]$. We also make both random oracles $\RO_1$ and $\RO_2$ lazily sampled using internal tables $H_1$ and $H_2$. These changes only affect the values of the game's internal state, and the view of the adversary remains the same as in $\prevGm$, so 
\[ \Pr[\curGm\Rightarrow 1] = \Pr[\prevGm \Rightarrow 1]. \]

\proofngame[tls-sid-collisions]

Starting with $\curGm$, we abort whenever two honest sessions sample the same nonce or group element and whenever an honest responder samples a nonce and group element that are already in use. Since this happens only after one of the flags $\bad[C]$ and $\bad[O]$ is set, by the identical-until-bad lemma,
\[\Pr[\prevGm*\Rightarrow 1] - \Pr[\curGm*\Rightarrow 1] \leq \Pr[\bad[C] \gets \true\text{ or }\bad[O]\gets \true \text{ in }\prevGm].\]
One nonce and one group element is chosen in each $\RunInitI$ call and each $\RunRespI$ call, so at most one nonce and one group element is chosen for each of the~$\qSend$ queries the adversary makes to its $\Send$ oracle.
We use the birthday bound to limit the probability of a collision (flag $\bad[C]$) in either the set of honest sessions' nonces or the set of honest sessions' DH shares to $\frac{\qSend^2}{2^{\nl+1}\cdot p}$. Every time a responder session chooses a nonce and group element, there are at most $\qSend$ values have already been chosen, so by the union bound $\bad[O]$ is set with probability at most $\frac{\qSend^2}{2^{\nl}\cdot p}$.  Therefore
\[
	\Pr[ \prevGm* \Rightarrow 1] - \Pr[ \curGm* \Rightarrow 1] \leq \frac{3\qSend^2}{2^{\nl+1}\cdot p}.
\]
%%%technically the multiplier could be 3/2, not 2. 
\proofngame[tls-log-sidhashes]

Next, we must ensure that partial transcripts between honest sessions do not collide under the hash function $\Hash$. This is a step unique to the $\TLS$ proof, which hashes all of its context with a collision-resistant hash function before it is input into key-derivation. In $\curGm$, honest sessions will log all of their hash outputs in a look-up table $T$: whenever an honest session computes $d = \Hash(s)$ for some string $s$, it sets $T[d] \gets s$ if $T[d]$ has not already been defined.
If $T[d]$ is not empty, then some prior honest session has computed $d = \Hash(s')$ for some string $s'$. The session will set a flag $\bad[\Hash]$ if $s' \neq s$, noting that a collision has occurred. We also remove the now superfluous $\bad[C]$ flag. These administrative changes do not affect the view of the adversary, so 
\[\Pr[\curGm*\Rightarrow 1 ] = \Pr[\prevGm* \Rightarrow 1]. \]

\proofngame[tls-hash-collisions]

\stepcounter{advB-TLS}
In Game $\curGm$, we abort whenever hashes computed by honest sessions collide (i.e. the $\bad[\Hash]$ flag is set). By the identical-until-bad lemma,
\[ \Pr[\prevGm \Rightarrow 1] - \Pr[\curGm\Rightarrow 1] \leq \Pr[\bad[\Hash] \gets \true\text{ in }\prevGm].\]
We bound the probability that $\bad[\Hash]$ is set via a reduction $\curadvB$ to the collision-resistance security of $\Hash$. 
The reduction simulates $\prevGm$ honestly for the adversary $\advA$. 
If the flag $\bad[\Hash]$ is set, then the reduction has obtained strings $s$, $s'$, and $d$ such that $s' \neq s$, and $\Hash(s) = \Hash(s') = d$. Then $\curadvB$ outputs $(s,s')$ and wins the collision-resistance game, so $\genAdv{cr}{\Hash}{\curadvB} \geq \Pr[\bad[\Hash] \gets \true\text{ in } \prevGm].$
The runtime~$t_{\curadvB}$ of $\curadvB$ approximately equals the runtime of $\advA$ in $\prevGm$.
It follows that 
\[ \Pr[\prevGm* \Rightarrow 1] - \Pr[\curGm* \Rightarrow 1] \leq \Adv^{\COLL}_{\Hash}(t_{\curadvB}).\]

\proofngame[tls-log-keys]

In Game $\curGm$, we remove the superfluous $\bad[\Hash]$ flag and make additional internal changes to the behavior of honest sessions. 
As in the \SIGMAI proof, all honest initatior sessions now log the first message they send in a set $\Sent$, and honest responder sessions use this set to check whether their first received message came from an honest session without tampering. If so, we say the responder session has an ``honest origin partner."
In the \SIGMAI protocol, partnering between honest sessions was sufficient to ensure agreement on the derived master key and all subsequently computed keys, since partners are guaranteed to hold the same nonces and group elements. 
In TLS~1.3, partnering also ensures agreement on the handshake traffic secrets $\SHTS$ and $\CHTS$, but it does not ensure agreement on the session key $\ATS$. 
Therefore the responder only logs the handshake traffic keys $\SFK, \CFK, \tkshs,$ and $\tkchs$ in a look-up table $\S$ under its session identifier. 
In addition to the session identifier, the application traffic secret $\ATS$ depends on the server's identity $\sSCERT$, signature $\sSCERTV$, and MAC tag $\sSFIN$. 
These values are not necessarily shared by partner sessions in Game $\curGm$, so two partnered sessions may derive different values of $\ATS$.   
The responder session therefore logs its session key $\ATS$ in a second look-up table $\S'$ indexed by all of the dependencies of the session key: $\sid, \sSCERT,\sSCERTV$, and $\sSFIN$. 
All of this is just bookkeeping, so 
\[ \Pr[\curGm*\Rightarrow 1] = \Pr[\prevGm* \Rightarrow1].\]

\proofngame[tls-copy-keys]

Going forward from Game $\curGm$, honest initiators copy their key material from tables $\S$ and $\S'$ if it is consistent for them to do so.
In the case where the adversary has tampered with the values of $\sSCERT$, $\sSCERTV$, or $\sSFIN$, the partner's session key depends on the untampered values and should not be copied.
Therefore honest initiators always copy encryption and MAC keys from the table $\S$ if they have an honest partner session, but they only copy $\ATS$ when the $\sSCERT, \sSCERTV$, and $\sSFIN$ messages they received match the ones sent by their partner.
The initiator session can check whether tampering occurred using the table $\S'$, which will contain a session key $\ATS$ at index $\sid \conc \sSCERT \conc \sSCERTV \conc \sSFIN$ if and only if the honest partner session computed and sent $\sSCERT$, $\sSCERTV$, and $\sSFIN$.

We argue that all copied keys are consistent with the keys that would be derived in $\prevGm$.
Recall that partnered sessions agree on the nonces and the DH shares $\X$ and $\Y$ as components of $\sid$, so they also agree on the shared DH secret $\Zz$ associated with the pair $(\X,\Y)$. 
Partnered sessions therefore agree on the handshake secret $\HS$, which is derived from $\Zz$ without context, and on the handshake traffic secrets, which are derived with the session identifier as context.
Thus partnered sessions agree on the values of the handshake traffic keys $\SFK,\CFK,\tkshs$, and $\tkchs$ which are derived from the handshake traffic secrets. 
For the adversary it is hence unobservable if honest sessions compute the handshake traffic keys themselves, or copy the keys from their partners.
By agreeing on the handshake secret $\HS$, partnered sessions will also agree on the master secret $\MS$, which is derived from $\HS$ without context. 
The if $\sSCERT$, $\sSCERTV$, and $\sSFIN$ are left untampered, both sessions will derive the session key as $\RO_2(\MS, \inputlabel[8],\Hash(\sid \conc \sSCERT \conc \sSCERTV \conc \sSFIN)])$. Hence it is again unobservable whether an honest initiator derives $\ATS$ itself or copies $\ATS$ from an honest partner which agrees on the values of $\sSCERT$, $\sSCERTV,\sSFIN$; consequently
\[ \Pr[\curGm* \Rightarrow 1] = \Pr[\prevGm* \Rightarrow 1].\]

%\proofngame[tls-log-guessing]
%While the \SIGMAI protocol uses only one random oracle to derive its master key, the TLS protocol's key schedule uses both $\HKDF.\Extract$ and $\HKDF.\Expand$ multiple times to obtain the final keys. We therefore need to ensure that the adversary $\advA$ is not able to guess intermediate values in the key derivation process without making the corresponding queries to $\RO_1$ and $\RO_2$. To this end, in game $\curGm$ we make random oracle $\RO_2$ log the first input $s$ of each query in a list $W$. Whenever $\RO_1$ or $\RO_2$ samples a response to a new query (including both adversarial queries and queries made by the game), they set a flag $\bad[W]$ if the response was the input to a prior $\RO_2$ query, using the list $W$ to efficiently check the condition. This is again just bookkeeping, so 
%\[ \Pr[\curGm* \Rightarrow 1] = \Pr[\prevGm* \Rightarrow 1].\]

%\proofngame[tls-no-guessing]
%We now eliminate the possibility of guessing the intermediate values of key derivation. In Game $\curGm$, the game aborts when an $\RO_1$ or $\RO_2$ would return a value the adversary has already queried to $\RO_2$ (i.e., when the $\bad[W]$ flag is set). By the identical-until bad lemma,
%\[ \Pr[\prevGm \Rightarrow 1] - \Pr[\curGm\Rightarrow 1] \leq \Pr[\bad[W] \gets \true\text{ in } \prevGm].\]
%We bound the right-hand side of this equation.  Let $hl_1$ be the output length of $\RO_1$, and let $hl_2$ be the output length of $\RO_2$. Since the responses to random oracle queries are sampled randomly, they are independent of the contents of the list $W$ of prior inputs to $\RO_2$.
% Each $\RunInit$ or $\RunRespI$ call queries $\RO_2$ on at most $4$ different inputs (these inputs are $\HS, \SHTS, \CHTS$, and $\MS$), so if the adversary makes $\qRO$ total queries to $\RO_1$ and $\RO_2$, and $\qSend$ queries to the $\Send oracle$, then $W$ has at most $\qRO + 4\qSend$ entries. 
%By the birthday bound and the union bound, the probability that one of the responses from $\RO_1$ or $\RO_2$ collides with an element of $W$ is at most$\bad[R]$ is at most $\frac{(\qRO)(\qRO + 4 \qRO \qSend)}{2^{hl_1}}+ \frac{(\qRO)(\qRO + 4 \qRO \qSend)}{2^{hl_2}}$, hence 
%\[ \Pr[\prevGm \Rightarrow 1] - \Pr[\curGm\Rightarrow 1] \leq \frac{(\qRO)(\qRO + 4 \qRO \qSend)}{2^{hl_1}}+ \frac{(\qRO)(\qRO + 4 \qRO \qSend)}{2^{hl_2}} .\]

\proofngame[tls-uniform-keys]

In Game $\curGm$, all responders sample $\ATS$, $\SHTS$ and $\CHTS$ randomly (unless their values have already been fixed by queries to random oracle $\RO_2$ on the corresponding input), then retroactively programs random oracle $\RO_2$ by setting its internal table $H_2$ on the appropriate input.
%They also update the list $W$ to include the inputs $\HS$ and $\MS$. 
Partnered initiator sessions which have not copied $\ATS$ (i.e., those who received tampered $\sSCERT$, $\sSCERTV$, and $\sSFIN$) also sample $\ATS$ randomly and program $\RO_2$ when necessary.
We choose to program $\ATS$, $\SHTS$, and $\CHTS$, as opposed to only $\mk$ in the \SIGMAI proof, because these three keys are derived with context. 
Most importantly, the DH shares $\X$ and $\Y$ indirectly enter the key derivation for these keys, which will be critical for the reduction in the next step.
This simply moves the lazy sampling process from $\RO_2$ to $\RunRespI$ and $\RunInit$ for certain queries, which is unobservable to the adversary; therefore
\[ \Pr[\curGm* \Rightarrow 1] = \Pr[\prevGm* \Rightarrow 1]. \]

\proofngame[tls-stop-programming]

The step between $\prevGm$ and $\curGm$ is most technically involved step of this proof, and it is also the most significantly altered from the corresponding step in the proof of \SIGMAI. 
In $\curGm$, partnered initiators and responder sessions with honest origin partners will stop maintaining the consistency of their keys $\ATS$, $\SHTS$, and $\CHTS$ with the random oracle $\RO_2$. 
Specifically, responders with honest origin partners sample $\ATS$, $\SHTS$, and $\CHTS$ uniformly at random even if $\RO_2$ has already been queried on the string $\HS,\inputlabel,\sidhash$ for the appropriate label and hash, and they do not retroactively program $\RO_2$. Partnered initiator sessions which have not copied $\ATS$ from their partner also sample $\ATS$ uniformly without checking or programming $\RO_2$. 
These keys are therefore completely random, and they will be inconsistent with any random oracle queries made before or after the keys are sampled.

In order to detect this inconsistency, the adversary must make a query to $\RO_2$ that would, in $\prevGm$, return one of the unprogrammed keys.
Which queries are these? 
They are the queries that an honest responder session with honest origin partner would use to derive $\SHTS$, $\CHTS$, and $\ATS$, and the queries that an honest partnered initiator which received a tampered message would use to derive $\ATS$. 
Formally, let $\sid = (\nonce, \nonce',\X,\Y)$ be the session ID held by some honest responder session with honest origin partner, and let $\sSCERT$, $\sSCERTV$, $\sSFIN$ be the identity, signature, and MAC tag sent by this session. 
Let $\DHE$ be the DH secret corresponding to the pair $(\X,\Y)$.
Then the adversary $\advA$ can detect an inconsistency (in derviations of honest responders) in game $\curGm$ if at any point during the game $\advA$ queries $\RO_2$ on one of the tuples
\[
	(\RO_1(\constant[1],\DHE), \inputlabel,\Hash(\sid))
	\qquad \text{or} \qquad
	(\MS,\inputlabel[8],\Hash(\sid\conc \sSCERT \conc \sSCERTV \conc \sSFIN)),
\]
where $\inputlabel \in \{\inputlabel[1], \inputlabel[2]\}$
and where for some $\HS$, $\dHS$, we have that $\HS = \RO_1(\constant[1],\DHE)$, that $\dHS = \RO_2(\HS,\inputlabel[3],\Hash(\texttt{""}))$, and that $\MS = \RO_1(\dHS,0)$.
Otherwise (for derviations of honest initiators), let $\sid$ be the session ID held by an honest partnered initiator session, and let $\sSCERT$, $\sSCERTV$, and $\sSFIN$ be the identity, signature, and MAC tag received by that session.
For initiator sessions that do not copy~$\ATS$, at least one of these values was not sent by the honest partner.
Then the adversary $\advA$ can detect an inconsistency in game $\curGm$ if at any point it queries $\RO_2$ on the tuple
\[
	(\MS,\inputlabel[8],\Hash(\sid\conc \sSCERT \conc \sSCERTV \conc \sSFIN)),
\]
where for some $\HS$, $\dHS$, we have that $\HS = \RO_1(\constant[1],\DHE)$, that $\dHS = \RO_2(\HS\cab \inputlabel[3]\cab \Hash(\texttt{""}))$, and that $\MS = \RO_1(\dHS,0)$.
Let event $F$ denote the event that the adversary $\advA$ makes at least one of the above queries.
If event $F$ does not occur, then $\ATS$, $\SHTS$, and $\CHTS$ are chosen uniformly at random in both $\prevGm$ and $\curGm$, hence
\[\Pr[\prevGm \Rightarrow 1] - \Pr[\curGm]\Rightarrow 1] \leq \Pr[F\text{ occurs in }\prevGm]. \]
\stepcounter{advB-TLS}%
We bound the probability of event $F$ via a reduction $\curadvB$ to the strong Diffie--Hellman assumption in group~$\group$.
The reduction will make no more queries to its $\stDH$ oracle than $\advA$ makes to its $\RO_2$ oracles.

Given its strong DH challenge $(A = g^a, B= g^b)$ and having access to the strong Diffie--Hellman oracle $\stDH_a$, $\curadvB$ simulates $\prevGm$ for an adversary $\advA$ in the following manner: In all honest initiator sessions, $\curadvB$ samples $r$ uniformly at random from $\ZZ_p$ and sets the session's DH share $\X \gets A \cdot g^r$. In all honest responder sessions with honest origin partner, $\curadvB$ samples $r'$ uniformly from $\ZZ_p$ and sets the session's DH share $\Y \gets B \cdot g^{r'}$. Both of these DH shares are still distributed uniformly over $\ZZ_p$ as long as $p$ is prime and $A$ and $B$ are not the identity.  To extract $g^{ab}$ when event $F$ occurs, the reduction $\curadvB$ will follow the same general strategy as the reduction $\advB_1$ in the proof of \SIGMAI, with four major points of divergence. We address these points first, before giving a full description of $\curadvB$.
\begin{enumerate}
	\item Since $\curadvB$ no longer knows $x$ or $y$ such that $\X = g^x$ or $\Y=g^y$, it cannot compute the Diffie--Hellman secret $\DHE$ or the derived handshake secret $\HS$, so it samples $\HS$ randomly for honest responder sessions with honest origin partners and for honest partnered initiator sessions. The adversary can only tell that $\HS$ was not correctly computed if it notices that $\SHTS$, $\CHTS$, or $\dHS$ are derived from an incorrect value of $\HS$. 
	The former two cases require the adversary to make a query that triggers event $F$. In the latter case, $\dHS$ is not revealed to the adversary through any oracle, so the adversary must notice that $\ATS$, which is derived indirectly from $\dHS$ via the master secret, is derived from an incorrect value of $\HS$. This also requires $\advA$ to make a query that triggers event $F$. Therefore, until event $F$ occurs, this change is unobservable to the adversary.
	\item In the TLS protocol, the context string, including the Diffie--Hellman shares $\X$ and $\Y$, is hashed with $\Hash$ before it enters key derivation, so $\curadvB$ cannot directly associate a query to $\RO_2$ with the honest session(s) whose session ID is being used. The reduction addresses this by having each honest responder with honest origin partner and each honest partnered initiator, log the hash of its context in a reverse look-up table $R$.
	(The context does not include the handshake or master secrets.)
	Then in the $\RO_2$ oracle, $\advB_2$ can use $R$ to efficiently check whether the hash $\sidhash$ of a query is used to derive a handshake or application traffic key.
	\item Due to TLS's complex key schedule, no one random oracle query contains both a pair of Diffie--Hellman shares and the DH secret associated with that pair. Instead, $\curadvB$ will augment the $\RO_1$ and $\RO_2$ oracles to log in a reverse look-up table $T$ the DH secret associated with each of the intermediate values $\HS$, $\dHS$, and $\MS$. The DH secret for $\dHS = \RO_2(\HS,\inputlabel[3],\Hash(\texttt{""}))$ simply be copied from $T[\HS]$, and the DH secret for $\MS = \RO_1(\dHS,0)$ will be copied from $T[\dHS]$. For each query to $\RO_2$ with secret $s$, the reduction can efficiently check using $T$ whether $s$ was derived from some DH secret via $\RO_1$. 
	\item The TLS key schedule uses multiple random oracle queries (if we model $\HKDF.\Extract$ and $\HKDF.\Expand$ as random oracles) whereas the \SIGMAI protocol uses only one. If $\advA$ can guess the intermediate value $\HS = \RO_1(\constant[1]\cab \DHE)$, where $\DHE$ is the DH secret associated to some pair of embedded shares $(\X,\Y)$ chosen by honest sessions, then it can trigger event $F$ without ever submitting $\DHE$ to an oracle. In this case, $\advA$ can trigger event $F$, but $\curadvB$ cannot win the Strong DH game. However, if $\RO_1(\constant[1],\DHE)$ is never queried, then it is uniformly random, and the probability that $\advA$ guesses correctly is bounded by $\frac{\qRO \cdot \qSend}{2^{\kl}}$ by the birthday bound. 
\end{enumerate}

To compute the correct handshake and application traffic keys, $\curadvB$ needs to be able to correctly program $\CHTS$, $\SHTS$, and $\ATS$.
When these keys are chosen by an honest responder with honest origin partner or a partnered initiator, $\curadvB$ uses its strong DH oracle to check whether $\RO_2$ has already received the query that the adversary needs to make to generate these keys. If the query has already been made, $\curadvB$ can look up the DH secret using $T$ and win the game. 
Otherwise, $\curadvB$ hashes the session's context and logs it in $R$, so that future $\RO_2$ queries can identify this session for retroactive programming.
It also logs the session's randomness in a look-up table $\Q$, to be used if event $F$ is triggered relative to this session by a future $\RO_2$ query.

Like in the \SIGMAI proof, $\curadvB$ must be able to correctly compute handshake and application traffic keys for unpartnered initiator sessions. Because all initiator sessions have embedded DH shares, $\curadvB$ cannot compute the DH secret $\DHE$ for these sessions. However, it can use its StrongDH oracle to check whether the adversary has queried such a secret and copy the expected keys to preserve consistency in this case. If no query has been made, the keys are selected randomly and the initiator session stores its context, randomness, and keys in $R$.  In future queries to the $\RO_2$ oracle, $\curadvB$ will use $R$ to efficiently check whether a query should output one of the initiator session's keys. If so, it retroactively programs the oracle using the keys from $R$. 

Therefore, if event $F$ occurs, reduction $\curadvB$ wins the strong Diffie--Hellman game except with probability $\frac{\qRO\cdot \qSend}{2^{\kl}}$, resulting in
$\Adv^{\strongDH}_{\group}(t_{\curadvB},\qRO) \geq (1-\frac{\qRO\cdot \qSend}{2^{\kl}})\cdot \Pr[F]$. 
Then $\Pr[F] \leq \frac{2^{\kl}}{2^{\kl}-\qRO\cdot \qSend} \cdot \Adv^{\strongDH}_{\group}(t_{\curadvB},\qRO)$. 
Otherwise, the reduction simulates $\prevGm$ perfectly except with probability $\frac{\qRO \cdot \qSend}{2^{\kl}}$.

\begin{align*}
\Pr[ \prevGm* \Rightarrow 1 ]
&= \Pr[ \curGm* \Rightarrow 1] + \Pr[F] + (1-\Pr[F])\cdot \frac{\qRO\cdot \qSend}{2^{\kl}} \\
&\leq \Pr[ \curGm* \Rightarrow 1] + \frac{2^{\kl}+\qRO\cdot \qSend}{2^{\kl}-\qRO \cdot \qSend} \cdot \Adv^{\strongDH}_{\group}(t_{\curadvB},\qRO)+\frac{\qRO \cdot \qSend}{2^{\kl}} \\
&\leq \Pr[ \curGm* \Rightarrow 1] + 2 \cdot \Adv^{\strongDH}_{\group}(t_{\curadvB},\qRO)+\frac{\qRO\cdot \qSend}{2^{\kl}},
\end{align*}
where the last simplification step assumes that $\qSend \cdot \qRO \leq 2^{\kl-2}$, which is true for any reasonable real-world parameters.

\proofngame[tls-uniform-traffic-keys]

In Game $\curGm$, honest responders with honest origin partners sample $\SFK$, $\CFK$, $\tkchs$ and $\tkshs$ uniformly at random, so these keys are no longer consistent with $\RO_2$. The adversary can distinguish this change if and only if it queries $\RO_2$ on a string $\SHTS,\inputlabel,\Hash(\texttt{""})$, or $\CHTS,\inputlabel,\Hash(\texttt{""})$, where $\inputlabel \in \{\inputlabel[4],\inputlabel[6]\}$, and $\SHTS$ and $\CHTS$ are chosen by an honest responder sessions with honest origin partner. Call this event $E$. In these sessions, $\SHTS$ and $\CHTS$ are chosen uniformly at random by $\prevGm$, and they are never revealed by any oracle. Therefore the probability of event $E$ is at most $\frac{\qRO \cdot \qSend}{2^{\kl}}$ by the birthday bound, hence 
\[ \Pr[\prevGm] \leq \Pr[\curGm] + \frac{\qRO \cdot \qSend}{2^{\kl}}.\]

Note that this step in the \SIGMAI proof introduced a multi-user PRF security bound due to final keys being derived through a PRF, not the random oracle.
Modeling $\HKDF.\Expand$ as random oracle~$\RO_2$, we here instead incur a birthday bound under the random oracle instead of a multi-user PRF security bound for~$\HKDF.\Expand$.

\medskip

The remaining game hops are identical to those in the proof of \SIGMAI, so we discuss them only briefly. 


\proofngame[tls-signature-bookkeeping]

In Game $\curGm$, we log all messages signed by an honest session in a look-up table $\Q_S$, and we set a flag $\bad[S]$ whenever a partnered session verifies a signature with an uncorrupted public key on a message that was not in $\Q_S$. This is just administrative, so 
\[\Pr[\curGm*\Rightarrow 1] = \Pr[\prevGm* \Rightarrow 1]. \]

\proofngame[tls-signature-forgery]
\stepcounter{advB-TLS}
In Game $\curGm$, we abort if the flag $\bad[S]$ is set. In this case, an honest partnered session received a signature which was not computed by an honest session, and which was verified by an uncorrupted public key. We can give a straightforward reduction $\curadvB$ to the multi-user EUF-CMA security of the signature scheme that wins whenever $\bad[S]$ is set and has runtime approximately equal to that of $\advA$ in $\prevGm*$. By the identical-until-bad lemma,
\[Pr[\prevGm* \Rightarrow 1] - \Pr[\curGm* \Rightarrow 1] \leq \Adv^{\muEUFCMA}_{\SIGScheme}(t_{\curadvB},\qNewUser,\qSend,\qSend,\qRevLongTermKey).
\]

Interestingly and in contrast to the \SIGMAI proof, soundness is still not guaranteed after this game hop, because we do not require the signature scheme to be strongly unforgeable. Therefore the adversary may be able to produce a new signature on a message that had been signed by an honest session, allowing it to tamper with $\sSCERTV$ without setting the $\bad[S]$ flag.

\proofngame[tls-mac-bookkeeping]

In Game $\curGm$ we log all messages for which an honest session computed a MAC tag in a look-up table $\Q_{M}$. We remove the $\bad[S]$ flag and instead set a flag $\bad[M]$ if an honest partnered session verifies a MAC on a message that is not in $\Q_{M}$. Again, this is only bookkeeping and does not impact the view of $\advA$, hence
\[\Pr[\curGm*\Rightarrow 1] = \Pr[\prevGm* \Rightarrow 1]. \]

\proofngame[tls-mac-forgery]

\stepcounter{advB-TLS}
Finally, in Game $\curGm$, we abort if an honest session with an honest partner verifies a MAC tag on a message which was not tagged by any honest session; i.e if the $\bad[M]$ flag is set. We can give a simple reduction $\curadvB$ to multi-user MAC security. The reduction $\curadvB$ assigns a pair of indices $i,i+1$ to each session identifier held by an honest session with honest origin partner. When an honest session with honest origin partner needs to compute a server MAC tag, $\curadvB$ finds the pair $(i,i+1)$ using the session identifier and calls its $\MACTag$ oracle with user identity $i$. When the session needs a client MAC tag $\curadvB$ calls $\MACTag$ with user identity $i+1$. The reduction calls its $\MACTag$ oracle no more than twice for every query $\advA$ makes to $\Send$ (once to generate a tag, and once to verify a tag).  Since by Game $\lblGm{tls-uniform-traffic-keys}$ all honest sessions with honest origin partners sample their MAC keys $\SFK$ and $\CFK$ uniformly at random, the keys implicitly generated by the MAC security game are consistent with the expected operation of Game $\curGm$. When the flag $\bad[M]$ is set, a partnered session has received a valid tag on a message which was never logged in $\Q_{M}$. The reduction can look up the pair $(i,i+1)$ using the session identifier of whichever session set $\bad[M]$. Since $\curadvB$ logs every message for which it calls its $\MACTag$ oracle, this is a valid forgery for either identity $i$ or identity $i+1$, and $\curadvB$ will win. Then 
\[ \Pr[\prevGm* \Rightarrow 1] - \Pr[\curGm* \Rightarrow 1] \leq  \Adv^{\muEUFCMA}_{\MACScheme}(t_{\curadvB}, \qSend,\qSend,1, \qSend,1,0). \]
The runtime of $\curadvB$ is about that of $\advA$ in $\prevGm*$. 

\medskip

We can now finally argue that the advantage of~$\advA$ in~$\curGm$ is zero.
The adversary $\advA$ would win game $\curGm$ with probability better than $\frac{1}{2}$ in one of three ways:
\begin{enumerate}
	\item $\Sound$ is false,
	\item $\ExplicitAuth$ is false, or
	\item $\Fresh$ is true and $b' = b$.
\end{enumerate}

\paragraph{Soundness}
The flag $\Sound$ is set if (1) three honest sessions hold the same session identifier, or if (2) two partnered sessions accept with different session keys. 
By Game $\lblGm{tls-sid-collisions}$, each session identifier is held by at most one session of each role. There are only two roles so case (1) never occurs.
If two partnered sessions $\pi_1$ and $\pi_2$ accept, the initiator session $\pi_1$ verified a MAC tag $\tau$ on the message $m = \nonce\conc \nonce' \conc \X \conc \Y\conc \sSCERT \conc \sSCERTV$. Because $\tau$ was verified by an honsest partnered session, by Game $\lblGm{tls-mac-forgery}$, this message was tagged by an honest session.
Honest sessions only tag strings including their own nonce, and by Game $\lblGm{tls-sid-collisions}$, the only honest session with nonce $\nonce'$ is $\pi_2$. Then $\pi_2$ must have tagged the message $m$, so $\pi_1$ and $\pi_2$ agree on both $\tau$ and $m$. Since the DH shares $\X$ and $\Y$ are components of $m$, $\pi_1$ and $\pi_2$ also agree on the DH secret $\DHE$ associated with the pair $(\X,\Y)$. Consequently, $\pi_1$ and $\pi_2$ will agree on any value derived deterministically from $m$, $\tau$, and $\DHE$, including the session key $\ATS$. 
Then the flag $\Sound$ is always true in $\curGm$.

\paragraph{Explicit authentication}
The flag $\ExplicitAuth$ is set if there exists a session $\pi_u^i$ that accepts with uncorrupted peer identity $v$, and either (1) no honest session $\pi_v^j$ is partnered with $\pi_u^i$, or (2) a session $\pi_v^j$ is partnered with $\pi_u^i$ but accepts with peer identity $w \neq u$. 
To have accepted with peer identity $v$, the session $\pi_u^i$ must have received and verified a signature $\sigma$ using the public key of identity $v$ on a message $m$ containing the session identifier of $\pi_u^i$. As $v$ was uncorrupted at the time that $\pi_u^i$ accepted, by Game $\lblGm{tls-signature-forgery}$, the message $m$ must have been signed by some honest session $\pi_v^j$. As honest sessions only sign messages containing their own session identifiers, $\pi_v^j.\sid = \pi_u^i.\sid$, so $\pi_v^j$ and $\pi_u^i$ are partnered.
If case (2) occurs, $\pi_v^j$ must have accepted a MAC tag $\tau$ on message $m'$ containing its session ID and the identity $w$ of its peer. We know that $\pi_v^j$ is a partnered session, so by $\lblGm{tls-mac-forgery}$, $m'$ was tagged by some honest session. Honest sessions tag only messages containing their own session identifiers, so by $\lblGm{tls-sid-collisions}$, the message $m'$ must have been tagged by either $\pi_u^i$ or $\pi_v^j$. In \SIGMAI, the messages tagged by these two sessions are differentiated by there labels. Here, they are differentiated by their length: one role signs a message including values $\sSFIN$, $\sCCERT$, and $\sCCERTV$, while the other signs a message which does not contain these values. For this reason $\pi_v^j$ will not verify the tag on a message it signed itself. Therefore $m'$ must have been tagged by $\pi_u^i$, so $m'$ contains the identity $u$. This contradicts the assumption that $w \neq u$, so case (2) never occurs, and the flag $\ExplicitAuth$ is always false in~$\curGm$.

\paragraph{Guessing the challenge bit}
Now the adversary can only win with advantage better than zero is by guessing the correct value of $b$ when the $\Fresh$ flag is set to true. 
This requirement ensures that all tested sessions accepted with uncorrupted peer identities.
Since $\ExplicitAuth$ is true, each tested session must therefore have an honest session with which it is partnered, and by $\Sound$, this session holds the same session key.
Then by $\lblGm{tls-copy-keys}$, each tested initiator session copies the session key of its partner.
By $\lblGm{tls-stop-programming}$ each tested responder session, and each responder session partnered with a tested initiator session chooses its session key uniformly at random.
By $\Fresh$, the partners of tested sessions were not tested or revealed. 
Then the session keys of all tested sessions are sampled uniformly and never revealed to the adversary by any oracle. Therefore the key returned by each $\Test$ query is uniformly random and independent of the bit $b$.
The adversary's view is independent of the bit $b$, so it will win $\curGm$ with probability $\frac{1}{2}$, and consequently its advantage is $0$.

\medskip

Collecting the bounds across all game hops gives the theorem statement.
\end{collectinmacro}

\iffull
\begin{proof}
\TLSProofFull
\end{proof}
\else

%%%%%%%%%%%%%%%%%%%%%%%%%%%%%%%%%%%%%%%%%%%%%%%%%%%%%%%%%%%%%%%%%%%%%%%%%%%%%%%%%%%%%%%%%%%%%%%%
%% TLS Proof short version for lncs / non-full
%%%%%%%%%%%%%%%%%%%%%%%%%%%%%%%%%%%%%%%%%%%%%%%%%%%%%%%%%%%%%%%%%%%%%%%%%%%%%%%%%%%%%%%%%%%%%%%%

Due to space restrictions, we only give a high-level summary of the proof here, with a particular focus on those points where the proof differs from that for \SIGMAI (in Section~\ref{sec:sigma-proof}) due to aspects unique to TLS~1.3.
The full proof can be found in Appendix~\ref{apx:tls-proof-full}.

\begin{proof}[Proof summary]
\let\proofsep\medskip %% tighter spacing in here

\startproof{TLS-short}

Similar to the \SIGMAI proof, we will develop the bound via a series of incrementally changing code-based games. 


\proofngame[initial-game]

The first game $\curGm$ is the key exchange security game (cf.\ Figure~\ref{fig:AKE-security}) for the TLS~1.3 handshake protocol (Figure~\ref{fig:tls-protocol}).
Similar to~\SIGMAI (cf.\ Figure~\ref{fig:sigma-formal}), we capture the two protocol steps for each role in formal algorithms~$\KEActivate$, $\RunRespI$, $\RunInit$, and $\RunRespII$.
% We briefly define the formal algorithms $\KEActivate$, $\RunRespI$, $\RunInit$, and $\RunRespII$ (following the formal algorithm structure of \SIGMAI from Figure~\ref{fig:sigma-formal}) as follows.
% Let $\KEActivate$ include all of the client computation up to its first sent message, and $\RunInit$ cover the rest of the client's actions.
% Let the $\RunRespI$ algorithm perform all of the server's computation up to the generation of its second message,
% and let the $\RunRespII$ algorithm include all of the server's remaining computation.
% After acceptance, servers set their peer id to that contained in~$\CCERT$, clients set their peer identity to that in~$\SCERT$, and all sessions set their session key $\skey \gets \ATS$.
% With these definitions in place,
This way,
$\Pr[\curGm*\Rightarrow 1] = \Pr[ \Gm^{\KESEC}_{\TLS,\advA} \Rightarrow 1 ]$.


\proofngames[collisions]{4}

Over the next four games we start to abort if any collisions arise among the nonces, Diffie--Hellman shares, or context hashes computed by honest sessions. We also abort if any responder session chooses a nonce and group element that have already been received by another session. We use the general strategy of logging all values of interest, then setting a bad flag and aborting if we compute a value that has already been logged. We limit the probability of nonce and DH share collisions using the birthday bound, and give a simple reduction $\advB_1$ to the collision resistance of the hash function~$\Hash$. We also lazily sample the random oracles $\RO_1$ and $\RO_2$ using internal tables $H_1$ and $H_2$.
Excluding collisions, we obtain the bound
$\Pr[\prevGm*\Rightarrow 1] - \Pr[\curGm\Rightarrow 1] \leq \frac{3\qSend^2}{2^{\nl+1}\cdot p}% sid collisions
+ \Adv^{\COLL}_{\Hash}(t_1)$.


\proofngames[copy-keys]{2}

As in the \SIGMAI proof, we will let partnered initiator sessions copy key material from their honest partners via a table indexed by their session identifiers.
In TLS~1.3, there are two encryption keys $\tkshs$ and $\tkchs$, and two MAC keys $\SFK$ and $\CFK$ to copy. One significant difference from the \SIGMAI proof is that the session key $\ATS$ now depends on the messages $\sSCERT$, $\sSCERTV$, and $\sSFIN$. We have not ensured that partnered sessions agree on these values. Therefore honest initiators will only copy $\ATS$ from their partners if they received the exact same $(\sSCERT\cab \sSCERTV\cab \sSFIN)$ sent by their partner. In particular, equality is checked for the unencrypted values via an internal look-up table. 
Since keys are only copied when partners agree on all of the information entering the key derivation function, this change is unobservable to~$\advA$, hence
$\Pr[\curGm*\Rightarrow 1] = \Pr[\prevGm*\Rightarrow 1]$.


\proofngames[programming]{2}

These two games contain both the most critical step and the one that diverges the most from the proof for~\SIGMAI.
Let a responder session with honest origin partner be one whose first message was sent by an honest initiator session. 
First, we let all responder sessions with honest origin partners pick the handshake traffic keys $\SHTS$ and $\CHTS$, and the session key $\ATS$ uniformly at random, checking for consistency with the random oracle $\RO_2$ and retroactively programming it when necessary. 
Partnered initiator sessions who cannot copy their session key $\ATS$ due to tampering with their partner's second server messages $(\sSCERT\cab \sSCERTV\cab \sSFIN)$ will also pick $\ATS$ at random, ensuring consistency with~$\RO_2$.
Then, these sessions eliminate the consistency check and let their handshake traffic keys and session key be uniformly random and inconsistent with the adversary's queries to $\RO_2$. 
We argue that the adversary can only detect this inconsistency if it queries $\RO_2$ on the correct input to derive one of $\SHTS$, $\CHTS$, or $\ATS$ for an honest session with an honest origin partner, an event we refer to as event $F$. 

As in the \SIGMAI proof, we give a reduction $\advB_2$ to the strong DH assumption in group~$\group$ which wins with high probability if event $F$ occurs. 
This reduction follows roughly the same strategy: it embeds its challenges in the DH shares of all initiators and all responders with honest origin partners. 
Because $\advB_2$ cannot compute the DH secret for embedded sessions, it uses its $\stDH$ oracle to catch and program all queries to $\RO_2$ which are dependent on this secret. When event~$F$ occurs, $\advB_2$ uses its own randomness to extract the challenge DH secret from the DH secret contained in the query that triggered event $F$.
However, compared to the \SIGMAI proof, $\advB_2$'s strategy is customized for the TLS protocol in a few significant ways: 
\begin{enumerate}
	\item The TLS key schedule uses multiple random oracles where $\SIGMAI$ uses only one.
	If $\advA$ can guess the intermediate value $\HS = \RO_1(\constant[1],\DHE)$, where $\DHE$ is the DH secret associated to some pair of embedded shares $(\X,\Y)$ chosen by honest sessions, then it can trigger event $F$ without ever submitting $\DHE$ to an oracle.
	In this case, $\advA$ triggers event $F$, but $\advB_2$ can neither win the Strong DH game nor simulate $\prevGm$ correctly.
	However, if $\RO_1(\constant[1],\DHE)$ is never queried it remains uniformly random, and by the birthday bound $\advA$ succeeds at guessing $\HS$ with probability at most~$\frac{\qRO \cdot \qSend}{2^{\kl}}$. 

	\item In the TLS protocol, the context string including the Diffie--Hellman shares is hashed with $\Hash$ before it enters the key derivation, so $\advB_2$ cannot directly associate a query to $\RO_2$ with an honest~$\sid$.
	We address this by logging hash computations of honest sessions in a reverse look-up table~$R$.
	Then in the $\RO_2$ oracle, $\advB_2$ can use $R$ to efficiently find the context associated with a particular query.
	
	\item Due to TLS's complex key schedule, no one random oracle query contains both a pair of Diffie--Hellman shares and the DH secret associated with that pair. Instead, $\advB_1$ must link queries to $\RO_1$, which contain the DH secret with queries to $\RO_2$, which contain the context. It does this by augmenting the $\RO_1$ and $\RO_2$ oracles.
	When each of the intermediate values $\HS$, $\dHS$, and $\MS$ is derived, the associated DH secret is logged in a reverse look-up table $T$.
% 	The handshake $\HS$ is derived directly from the DH secret;
% 	$\dHS$ and $\MS$ copy the DH secret associated with $\HS$ when they are derived.
	This allows~$\advB_2$ efficient look-ups to check association of DH secrets and context through its strong DH oracle.
	
\end{enumerate}
When $\qRO \cdot \qSend \leq 2^{\kl - 2}$, we obtain the bound
$\Pr[\prevGm*\Rightarrow 1 ] - \Pr[\curGm*\Rightarrow 1] \leq 2 \cdot \Adv^{\strongDH}_{\group}(t_{\advB_2},\qRO) +\frac{\qRO \cdot \qSend}{2^{\kl}}$.

Similar to $\advB_1$ in the proof of $\SIGMA$, $\advB_2$ queries the $\stDH$ oracle at most once for each entry in $H'$. Although one call to $\RunInit$ or $\RunRespI$ may cause up to three $\stDH$ queries, each of these queries will have a unique label and a unique entry in $H'$.
Computing the input to each $\stDH$ query requires 1 multiplication and one exponentiation in the base group, which can be done using $1+2\log_2 p$ total group operations. In our runtime analysis, we count each group operation as $1$ step, so $t_{\advB_2} \approx t + 2 \qRO\log_2 p$.

\proofngame[uniform-keys] 
In game $\curGm$, sessions with honest origin partners sample all encryption and MAC keys uniformly at random. This is distinguishable only if the adversary can query $\RO$ on a string containing one of the random values $\SHTS$ or $\CHTS$, so by the birthday bound
$\Pr[\prevGm*\Rightarrow 1 ] - \Pr[\curGm*\Rightarrow 1] \leq \frac{\qRO\cdot \qSend}{2^{\kl}}$.

\proofngames[sigs-and-macs]{4}
In the remaining games, we eliminate signature and MAC forgeries via reductions $\advB_3$ and $\advB_4$ to the multi-user EUF-CMA security of $\SIGScheme$ and $\MACScheme$, precisely as we did in the proof of $\SIGMAI$. This gives the bound 
$
	\Pr[\prevGm*\Rightarrow 1 ] - \Pr[\curGm*\Rightarrow 1] \leq
		\Adv^{\muEUFCMA}_{\SIGScheme}(t_{\advB_3\cab \qNew\cab \qSend\cab \qSend\cab \qRevLongTermKey})
		+ \Adv^{\muEUFCMA}_{\MACScheme}(t_{\advB_4}\cab \qSend\cab \qSend\cab 1\cab \qSend\cab 1\cab 0).
$

Finally, we argue that $\advA$ has advantage $0$ in game $\curGm$ using similar logic to the proof of $\SIGMAI$, with two substantive differences. 
First, we must make a slightly more involved argument about soundness because the session key $\ATS$ relies on values that are not contained in the session identifier. 
For $\curGm$, two partnered sessions must still hold the same session key because the information $\sid\conc\sSCERT\conc\sSCERTV\conc\sSFIN$ is authenticated by the responder session's MAC tag. 
Second, MAC tags are no longer labeled by role. However, messages tagged by initiator sessions are strictly longer than messages tagged by responder sessions, so we can still differentiate the two easily. 
\end{proof}

\fi
\fi

\iffull
	% no short table
\else

	\begin{table}[t]
		\centering
		\small
		
		\renewcommand{\arraystretch}{0.001}
		\renewcommand{\tabcolsep}{0.15cm}
		\begin{tabular}{@{}llllllllllll@{}}
		\toprule
		 & \multicolumn{4}{c}{Adversary resources}	& & & & \multicolumn{2}{c}{Security bound}	\\
		 \cmidrule{2-5} \cmidrule{9-10} \\
		$b$ & $t$~~~~~	& $\#N$	& $\#S$ & $\#RO$ & Target  &&  Mode~~~~~~~ & DFGS\,{\scriptsize\cite{JC:DFGS21}}	& Us (Cor.~\ref{cor:full-psk-ecdhe-ke},~\ref{cor:psk-ke})	\\
		\midrule
	128 & $2^{60}$ & $2^{25}$ & $2^{35}$ & $2^{50}$ & $2^{-68}$ && PSK-only & \cellcolor{green!25}$\approx 2^{-119}$	& \cellcolor{green!25}$\approx 2^{-152}$	\\ 
	128 & $2^{80}$ & $2^{35}$ & $2^{55}$ & $2^{70}$ & $2^{-48}$ && PSK-only & \cellcolor{green!25}$\approx 2^{-59~}$	& \cellcolor{green!25}$\approx 2^{-112}$	\\ 
	\midrule
	128 & $2^{60}$ & $2^{25}$ & $2^{35}$ & $2^{50}$ & $2^{-68}$ && \texttt{secp256r1} & $\approx 2^{-61}$	& \cellcolor{green!25}$\approx 2^{-132}$	\\ 
	128 & $2^{80}$ & $2^{35}$ & $2^{55}$ & $2^{70}$ & $2^{-48}$ && \texttt{secp256r1} & $1$	& \cellcolor{green!25}$\approx 2^{-92~}$	\\ 
	\midrule
	128 & $2^{60}$ & $2^{25}$ & $2^{35}$ & $2^{50}$ & $2^{-68}$ && \texttt{x25519} & $\approx 2^{-57}$	& \cellcolor{green!25}$\approx 2^{-128}$	\\ 
	128 & $2^{80}$ & $2^{35}$ & $2^{55}$ & $2^{70}$ & $2^{-48}$ && \texttt{x25519} & $1$	&\cellcolor{green!25}$\approx 2^{-88~}$	\\ 
	\midrule
	192 & $2^{60}$ & $2^{25}$ & $2^{35}$ & $2^{50}$ & $2^{-132}$ && \texttt{secp384r1} & \cellcolor{green!25}$\approx 2^{-189}$ 	&\cellcolor{green!25}$\approx 2^{-259}$ 	\\ 
	192 & $2^{80}$ & $2^{35}$ & $2^{55}$ & $2^{70}$ & $2^{-112}$ && \texttt{secp384r1} & $\approx 2^{-108}$	&\cellcolor{green!25}$\approx 2^{-219}$ 	\\ 
	\midrule
	224 & $2^{60}$ & $2^{25}$ & $2^{35}$ & $2^{50}$ & $2^{-164}$ && \texttt{x448} & $\cellcolor{green!25}\approx 2^{-200}$ 	&\cellcolor{green!25}$\approx 2^{-280}$ 	\\ 
	224 & $2^{80}$ & $2^{35}$ & $2^{55}$ & $2^{70}$ & $2^{-144}$ && \texttt{x448} & $\approx 2^{-110}$	&\cellcolor{green!25}$\approx 2^{-240}$ 	\\ 
	\midrule
	256 & $2^{60}$ & $2^{25}$ & $2^{35}$ & $2^{50}$ & $2^{-196}$ && \texttt{secp521r1} & $\cellcolor{green!25}\approx 2^{-200}$ 	&\cellcolor{green!25}$\approx 2^{-280}$ 	\\ 
	256 & $2^{80}$ & $2^{35}$ & $2^{55}$ & $2^{70}$ & $2^{-176}$ && \texttt{secp521r1} & $\approx 2^{-110}$	&\cellcolor{green!25}$\approx 2^{-240}$ 	\\ 
	%
	% $2^{60}$	&$2^{20}$	&$2^{35}$	& \texttt{x25519}	&$2^{-68}$  & $\approx 2^{-119}$	& $\approx 2^{-152}$	&& $\approx 2^{-57}$	&$\approx 2^{-129}$ \\	 \midrule 
	% $2^{80}$	&$2^{30}$	&$2^{55}$	& \texttt{secp256r1}	&$2^{-48}$  & $\approx 2^{-59}$	& $\approx 2^{-112}$	&& 1			&$\approx 2^{-93}$ \\	 
	% $2^{80}$	&$2^{30}$	&$2^{55}$	& \texttt{x25519}	&$2^{-48}$  & $\approx 2^{-59}$	& $\approx 2^{-112}$	&& 1			&$\approx 2^{-89}$ \\
	% $2^{80}$	&$2^{30}$	&$2^{55}$	& \texttt{secp384r1}	&$2^{-112}$ & $\approx 2^{-146}$	& ---	&& $\approx 2^{-109}$	&$\approx 2^{-220}$	\\
		\bottomrule
		\end{tabular}

		\medskip
		
		\caption{%
			Exemplary concrete advantages of a key exchange adversary with given resources $t$ (running time), $\#N$ (number of pre-shared keys), $\#S$ (number of sessions), and $\#RO$ (number of random oracle queries) in breaking the security of the TLS~1.3 PSK handshake protocols.
			%
			Numbers based on the prior bounds by Dowling et al.~\cite{JC:DFGS21}
			and our bounds for PSK-(EC)DHE and PSK-only (in Corollaries~\ref{cor:full-psk-ecdhe-ke} resp.~\ref{cor:psk-ke}).
			``Target'' indicates the maximal advantage~$t/2^b$ tolerable for a given bound on $t$ when aiming for the respective curve's (or hash function's, in case of PSK-only mode) bit security level~$b$;
			entries in \colorbox{green!25}{green}-shaded cells meet that target.
			Mode indicates PSK-only mode (with \SHA{384}) or otherwise PSK-(EC)DHE mode with the given curve \texttt{secp256r1}, \texttt{x25519} (with \SHA{256}), or \texttt{secp384r1}, \texttt{x448}, \texttt{secp521r1} (with \SHA{384}).
		}
		\label{tbl:bounds-overview}
	\end{table}
\fi


\section{Evaluation}
\label{sec:evaluation}
Asymptotically, our tighter security bounds improve on prior analysis of TLS~1.3 by a quadratic factor.
We evaluate ours and prior bounds over a wide range of fully concrete resource parameters, following the approach of Davis and Günther~\cite{ACNS:DavGun21}.
\iffull
	The
full range of evaluated parameters is given in 
	Tables~\ref{tbl:bounds-full-psk-only} and~\ref{tbl:bounds-full-psk-dhe} 
		below,
along with reasoning for how we chose the various ranges of resource parameters.
The tables show that while the prior PSK-(EC)DHE bound by Dowling et al.~\cite{JC:DFGS21} meets the target security goals in a number of configurations,
there are at least some settings for all elliptic-curve groups in which the targeted security is not met.
Our bounds do significantly better than the target in all configurations we considered.
The gap for the PSK-only handshake is less significant as the loosest prior reduction for TLS~1.3 was to the Diffie--Hellman problem.

Overall, our bounds improve on previous analyses of the PSK-only handshake by~$15$ to~$53$ bits of security, and those of the PSK-(EC)DHE handshake by~$60$ to~$131$ bits of security, across all our parameters evaluated.

	\def\EvalTitle{Evaluation Details}
\iffull
	\subsection{\EvalTitle}
\else
	\newpage
	\section{\EvalTitle}
\fi
\label{app:evaluation}

In the following, we will briefly explain the reasoning behind each of our specific resource parameter estimates. 
An adversary in the MSKE game (cf.\ Definition~\ref{def:MSKE-security}) is limited in its runtime~$t$, the number of pre-shared keys~$\#N$, and distinct protocol sessions~$\#S$ it can observe or interact with, and the number of random oracle queries~$\#RO$ it can make.
This last quantity captures offline work the adversary spends on computing the hash function~$\Hash$, which in our analysis we model as random oracle.
The choice of ciphersuite enters the bound through the length of the symmetric session keys and pre-shared keys.
For the PSK-(EC)DHE handshake, the bound also depends on the underlying Diffie--Hellman group.


\paragraph{Runtime $t \in \{2^{40}, 2^{60}, 2^{80}\}$.}
We consider a range of adversarial runtimes from easily achievable ($2^{40}$ operations) to state-scaled computational power ($2^{80}$ operations). 

\paragraph{Random oracle queries $\#RO \in \{2^{40}, 2^{60}, 2^{80}\}$.}
The number of random oracle queries models the number of hash function computations an adversary is capable of computing. Accordingly, we scale the number of RO queries with the runtime, always setting $\#RO = t/2^{10}$.

\paragraph{Number of pre-shared keys $\#N \in \{2^{25}, 2^{35}\}$.}
The world's largest certificate authority Let's~Encrypt reports $\approx 2^{27.5}$ active certificates for fully-qualified domains.%
\footnote{\url{https://letsencrypt.org/stats/}} %% last checked 2021-09-29: 193M active certs, 252M fully-qualified domains certified
While not every \emph{user} of TLS~1.3 will perform resumption, our model counts the number of \emph{pre-shared keys},
where typically users may hold many pre-shared keys, with servers regularly issuing several PSKs per full-handshake connection for later resumption.
We hence estimate that the number of pre-shared keys accessible to a globally-scaled adversary may well exceed the reported number of (server) certificates.

\paragraph{Number of sessions $\#S \in \{2^{35}, 2^{45}, 2^{55}\}$.}
We use the same estimates as Davis and G{\"u}nther~\cite{ACNS:DavGun21}, based on Google's and Firefox's usage reports.%
\footnote{\url{https://transparencyreport.google.com/}, \url{https://telemetry.mozilla.org/}}
With a daily browser user base of $2$ billion ($\approx 2^{31}$) and an HTTPS traffic encryption rate in the range of $76$--$98\%$,
we estimate an adversary could encounter up to~$2^{55}$ distinct sessions over an extended time period.
Note that although the PSK handshakes are less commonly used by browsers than the full TLS~1.3 handshake, they are frequently used by embedded and low-powered devices which do not appear in these reports.
Naturally, we do not allow the number of sessions to exceed the adversary's runtime $t$.

\paragraph{Diffie--Hellman groups.}
There are ten Diffie--Hellman groups standardized for use with the PSK-(EC)DHE handshake: five elliptic-curve groups and five finite-field groups. 
We reduce to the security of the strong Diffie--Hellman assumption in each of these groups.
Davis and Günther gave a proof of hardness in the generic group model (GGM) for the strong DH problem.
This result is a good heuristic for elliptic-curve groups, but not for finite-field ones because they are vulnerable to index-calculus based attacks not covered by the GGM.
The elliptic-curve groups are more efficient and more widely used than finite-field groups, so we restrict our analysis to these groups:
\texttt{secp256r1}, \texttt{x25519}, \texttt{secp384r1}, \texttt{x448}, \texttt{secp521r1}.
For each group, we give in Table~\ref{tbl:bounds-full-psk-dhe} the order~$p$ and the expected security level~$b$ in bits.
We use the security level $b$ to determine the choice of hash function and the target security level for the entire PSK-(EC)DHE handshake.

\paragraph{Ciphersuite and symmetric lengths.}
Our bounds reduce to the collision resistance of the random oracle $\ROthash$, which models the handshake's hash function.
The choice of hash function also determines the length of the session and resumption keys.
TLS~1.3 has five ciphersuites, all of which set the hash function to be either $\SHA{256}$ or $\SHA{384}$.
For PSK-(EC)DHE mode, we select $\SHA{256}$ as the hash function whenever a curve with $128$-bit security is used and we select $\SHA{384}$ for higher-security curves.
As our results of Section~\ref{sec:ks-indiff} only apply to PSK-only mode when $\SHA{256}$ is the hash function, we always use $\SHA{256}$ and a target-security level of $128$ bits.

\begin{table}[t]
	\centering
% 	\fontsize{4.5}{5}\selectfont % smaller than \tiny
	\footnotesize
	\renewcommand{\arraystretch}{0.01}
	\renewcommand{\tabcolsep}{0.15cm}
	\vspace{-0.3cm} %% a little higher
	\begin{tabular}{@{}lllllll@{}}
		\toprule
		\multicolumn{4}{c}{Adversary resources}		&		& \multicolumn{2}{c}{PSK-only}	\\
		\cmidrule{1-4} \cmidrule{6-7}
		$t$	& $\#N$	& $\#S$ & $\#RO$ & Target $t/2^b$	& DFGS\,{\cite{JC:DFGS21}}~	& Us~{(Cor.~\ref{cor:psk-ke})} \\
		\midrule	
$2^{40}$	&$2^{25}$	&$2^{35}$	&$2^{30}$	&$2^{-88}$	&\cellcolor{green!25}$\approx 2^{-158}$	&\cellcolor{green!25}$\approx 2^{-173}$	\\
$2^{40}$	&$2^{35}$	&$2^{35}$	&$2^{30}$	&$2^{-88}$	&\cellcolor{green!25}$\approx 2^{-150}$	&\cellcolor{green!25}$\approx 2^{-173}$	\\
\midrule
$2^{60}$	&$2^{25}$	&$2^{35}$	&$2^{50}$	&$2^{-68}$	&\cellcolor{green!25}$\approx 2^{-119}$	&\cellcolor{green!25}$\approx 2^{-152}$	\\
$2^{60}$	&$2^{25}$	&$2^{45}$	&$2^{50}$	&$2^{-68}$	&\cellcolor{green!25}$\approx 2^{-109}$	&\cellcolor{green!25}$\approx 2^{-151}$	\\
$2^{60}$	&$2^{25}$	&$2^{55}$	&$2^{50}$	&$2^{-68}$	&\cellcolor{green!25}$\approx 2^{-99}$	&\cellcolor{green!25}$\approx 2^{-133}$	\\
$2^{60}$	&$2^{35}$	&$2^{35}$	&$2^{50}$	&$2^{-68}$	&\cellcolor{green!25}$\approx 2^{-119}$	&\cellcolor{green!25}$\approx 2^{-152}$	\\
$2^{60}$	&$2^{35}$	&$2^{45}$	&$2^{50}$	&$2^{-68}$	&\cellcolor{green!25}$\approx 2^{-109}$	&\cellcolor{green!25}$\approx 2^{-151}$	\\
$2^{60}$	&$2^{35}$	&$2^{55}$	&$2^{50}$	&$2^{-68}$	&\cellcolor{green!25}$\approx 2^{-99}$	&\cellcolor{green!25}$\approx 2^{-133}$	\\
\midrule
$2^{80}$	&$2^{25}$	&$2^{35}$	&$2^{70}$	&$2^{-48}$	&\cellcolor{green!25}$\approx 2^{-79}$	&\cellcolor{green!25}$\approx 2^{-112}$	\\
$2^{80}$	&$2^{25}$	&$2^{45}$	&$2^{70}$	&$2^{-48}$	&\cellcolor{green!25}$\approx 2^{-69}$	&\cellcolor{green!25}$\approx 2^{-112}$	\\
$2^{80}$	&$2^{25}$	&$2^{55}$	&$2^{70}$	&$2^{-48}$	&\cellcolor{green!25}$\approx 2^{-59}$	&\cellcolor{green!25}$\approx 2^{-112}$	\\
$2^{80}$	&$2^{35}$	&$2^{35}$	&$2^{70}$	&$2^{-48}$	&\cellcolor{green!25}$\approx 2^{-79}$	&\cellcolor{green!25}$\approx 2^{-112}$	\\
$2^{80}$	&$2^{35}$	&$2^{45}$	&$2^{70}$	&$2^{-48}$	&\cellcolor{green!25}$\approx 2^{-69}$	&\cellcolor{green!25}$\approx 2^{-112}$	\\
$2^{80}$	&$2^{35}$	&$2^{55}$	&$2^{70}$	&$2^{-48}$	&\cellcolor{green!25}$\approx 2^{-59}$	&\cellcolor{green!25}$\approx 2^{-112}$	\\
	\bottomrule
	\end{tabular}
	
	\medskip

	\caption{%
		Concrete advantages of a key exchange adversary with given resources $t$ (running time), $\#N$ (number of pre-shared keys), $\#S$ (number of sessions), and $\#RO$ (number of random oracle queries) in breaking the security of the TLS~1.3 PSK-only handshake protocol with a ciphersuite targeting $128$-bit security.
		%
		Numbers based on the prior bounds by Dowling et al.~\cite{JC:DFGS21}
		and our bound for PSK-only in Corollary~\ref{cor:psk-ke}.
		``Target'' indicates the maximal advantage~$t/2^b$ tolerable for a given bound on $t$ when aiming for the bit security level~$b = 128$;
		entries in \colorbox{green!25}{green}-shaded cells meet that target.
		We assume that the ciphersuite uses $\SHA{256}$ as its hash function (see Appendix~\ref{app:domsep} for further explanation).
	}
\label{tbl:bounds-full-psk-only}
\end{table}

\begin{table}[p]
	\centering
% 	\fontsize{4.5}{5}\selectfont % smaller than \tiny
	% \tiny
	\renewcommand{\arraystretch}{0.01}
	\renewcommand{\tabcolsep}{0.15cm}
	%\vspace{-0.5cm} %% a little higher
	\resizebox{!}{.42\textheight}{%
	\begin{tabular}{@{}llllllll@{}}
		\toprule
		\multicolumn{4}{c}{Adversary resources}		&&		& \multicolumn{2}{c}{PSK-(EC)DHE}	\\[-0.5mm]
		\cmidrule{1-4} \cmidrule{7-8}
		$t$	& $\#N$	& $\#S$ & $\#RO$ & Curve (\fullelse{bit security~$b$, group order~$p$}{bit sec.\!~$b$,\! order~$p$})	& Target $t/2^b$	& DFGS\,{\cite{JC:DFGS21}}~	& Us~{(Cor.~\ref{cor:full-psk-ecdhe-ke})} \\[-0.5mm]
		\midrule
$2^{40}$	&$2^{25}$	&$2^{35}$	&$2^{30}$	&\texttt{secp256r1} ($b \!=\! 128$, \! $p \!\approx\! 2^{256}$)	&$2^{-88}$	&\cellcolor{green!25}$\approx 2^{-92}$	&\cellcolor{green!25}$\approx 2^{-167}$	\\
$2^{40}$	&$2^{35}$	&$2^{35}$	&$2^{30}$	&\texttt{secp256r1} ($b \!=\! 128$, \! $p \!\approx\! 2^{256}$)	&$2^{-88}$	&$\approx 2^{-82}$	& \cellcolor{green!25}$\approx 2^{-167}$	\\
\midrule
$2^{40}$	&$2^{25}$	&$2^{35}$	&$2^{30}$	&\texttt{x25519} ($b \!=\! 128$, \! $p \!\approx\! 2^{252}$)	&$2^{-88}$	&\cellcolor{green!25}$\approx 2^{-92}$	&\cellcolor{green!25}$\approx 2^{-163}$	\\
$2^{40}$	&$2^{35}$	&$2^{35}$	&$2^{30}$	&\texttt{x25519} ($b \!=\! 128$, \! $p \!\approx\! 2^{252}$)	&$2^{-88}$	&$\approx 2^{-82}$	& \cellcolor{green!25}$\approx 2^{-163}$	\\
\midrule
$2^{40}$	&$2^{25}$	&$2^{35}$	&$2^{30}$	&\texttt{secp384r1} ($b \!=\! 192$, \! $p \!\approx\! 2^{384}$)	&$2^{-152}$	&\cellcolor{green!25}$\approx 2^{-220}$	&\cellcolor{green!25}$\approx 2^{-294}$	\\
$2^{40}$	&$2^{35}$	&$2^{35}$	&$2^{30}$	&\texttt{secp384r1} ($b \!=\! 192$, \! $p \!\approx\! 2^{384}$)	&$2^{-152}$	&\cellcolor{green!25}$\approx 2^{-210}$	&\cellcolor{green!25}$\approx 2^{-294}$	\\
\midrule
$2^{40}$	&$2^{25}$	&$2^{35}$	&$2^{30}$	&\texttt{x448} ($b \!=\! 224$, \! $p \!\approx\! 2^{446}$)	&$2^{-184}$	&\cellcolor{green!25}$\approx 2^{-220}$	&\cellcolor{green!25}$\approx 2^{-301}$	\\
$2^{40}$	&$2^{35}$	&$2^{35}$	&$2^{30}$	&\texttt{x448} ($b \!=\! 224$, \! $p \!\approx\! 2^{446}$)	&$2^{-184}$	&\cellcolor{green!25}$\approx 2^{-210}$	&\cellcolor{green!25}$\approx 2^{-301}$	\\
\midrule
$2^{40}$	&$2^{25}$	&$2^{35}$	&$2^{30}$	&\texttt{secp521r1} ($b \!=\! 256$, \! $p \!\approx\! 2^{521}$)	&$2^{-216}$	&\cellcolor{green!25}$\approx 2^{-220}$	&\cellcolor{green!25}$\approx 2^{-301}$	\\
$2^{40}$	&$2^{35}$	&$2^{35}$	&$2^{30}$	&\texttt{secp521r1} ($b \!=\! 256$, \! $p \!\approx\! 2^{521}$)	&$2^{-216}$	&$\approx 2^{-210}$	& \cellcolor{green!25}$\approx 2^{-301}$	\\
\midrule[1pt]
$2^{60}$	&$2^{25}$	&$2^{35}$	&$2^{50}$	&\texttt{secp256r1} ($b \!=\! 128$, \! $p \!\approx\! 2^{256}$)	&$2^{-68}$	&$\approx 2^{-61}$	& \cellcolor{green!25}$\approx 2^{-132}$	\\
$2^{60}$	&$2^{25}$	&$2^{45}$	&$2^{50}$	&\texttt{secp256r1} ($b \!=\! 128$, \! $p \!\approx\! 2^{256}$)	&$2^{-68}$	&$\approx 2^{-40}$	& \cellcolor{green!25}$\approx 2^{-132}$	\\
$2^{60}$	&$2^{25}$	&$2^{55}$	&$2^{50}$	&\texttt{secp256r1} ($b \!=\! 128$, \! $p \!\approx\! 2^{256}$)	&$2^{-68}$	&$\approx 2^{-12}$	& \cellcolor{green!25}$\approx 2^{-127}$	\\
$2^{60}$	&$2^{35}$	&$2^{35}$	&$2^{50}$	&\texttt{secp256r1} ($b \!=\! 128$, \! $p \!\approx\! 2^{256}$)	&$2^{-68}$	&$\approx 2^{-60}$	& \cellcolor{green!25}$\approx 2^{-132}$	\\
$2^{60}$	&$2^{35}$	&$2^{45}$	&$2^{50}$	&\texttt{secp256r1} ($b \!=\! 128$, \! $p \!\approx\! 2^{256}$)	&$2^{-68}$	&$\approx 2^{-32}$	& \cellcolor{green!25}$\approx 2^{-132}$	\\
$2^{60}$	&$2^{35}$	&$2^{55}$	&$2^{50}$	&\texttt{secp256r1} ($b \!=\! 128$, \! $p \!\approx\! 2^{256}$)	&$2^{-68}$	&$\approx 2^{-2}$	& \cellcolor{green!25}$\approx 2^{-127}$	\\
\midrule
$2^{60}$	&$2^{25}$	&$2^{35}$	&$2^{50}$	&\texttt{x25519} ($b \!=\! 128$, \! $p \!\approx\! 2^{252}$)	&$2^{-68}$	&$\approx 2^{-57}$	& \cellcolor{green!25}$\approx 2^{-128}$	\\
$2^{60}$	&$2^{25}$	&$2^{45}$	&$2^{50}$	&\texttt{x25519} ($b \!=\! 128$, \! $p \!\approx\! 2^{252}$)	&$2^{-68}$	&$\approx 2^{-37}$	& \cellcolor{green!25}$\approx 2^{-128}$	\\
$2^{60}$	&$2^{25}$	&$2^{55}$	&$2^{50}$	&\texttt{x25519} ($b \!=\! 128$, \! $p \!\approx\! 2^{252}$)	&$2^{-68}$	&$\approx 2^{-12}$	& \cellcolor{green!25}$\approx 2^{-123}$	\\
$2^{60}$	&$2^{35}$	&$2^{35}$	&$2^{50}$	&\texttt{x25519} ($b \!=\! 128$, \! $p \!\approx\! 2^{252}$)	&$2^{-68}$	&$\approx 2^{-57}$	& \cellcolor{green!25}$\approx 2^{-128}$	\\
$2^{60}$	&$2^{35}$	&$2^{45}$	&$2^{50}$	&\texttt{x25519} ($b \!=\! 128$, \! $p \!\approx\! 2^{252}$)	&$2^{-68}$	&$\approx 2^{-32}$	& \cellcolor{green!25}$\approx 2^{-128}$	\\
$2^{60}$	&$2^{35}$	&$2^{55}$	&$2^{50}$	&\texttt{x25519} ($b \!=\! 128$, \! $p \!\approx\! 2^{252}$)	&$2^{-68}$	&$\approx 2^{-2}$	& \cellcolor{green!25}$\approx 2^{-123}$	\\
\midrule
$2^{60}$	&$2^{25}$	&$2^{35}$	&$2^{50}$	&\texttt{secp384r1} ($b \!=\! 192$, \! $p \!\approx\! 2^{384}$)	&$2^{-132}$	&\cellcolor{green!25}$\approx 2^{-189}$	&\cellcolor{green!25}$\approx 2^{-259}$	\\
$2^{60}$	&$2^{25}$	&$2^{45}$	&$2^{50}$	&\texttt{secp384r1} ($b \!=\! 192$, \! $p \!\approx\! 2^{384}$)	&$2^{-132}$	&\cellcolor{green!25}$\approx 2^{-168}$	&\cellcolor{green!25}$\approx 2^{-259}$	\\
$2^{60}$	&$2^{25}$	&$2^{55}$	&$2^{50}$	&\texttt{secp384r1} ($b \!=\! 192$, \! $p \!\approx\! 2^{384}$)	&$2^{-132}$	&\cellcolor{green!25}$\approx 2^{-140}$	&\cellcolor{green!25}$\approx 2^{-254}$	\\
$2^{60}$	&$2^{35}$	&$2^{35}$	&$2^{50}$	&\texttt{secp384r1} ($b \!=\! 192$, \! $p \!\approx\! 2^{384}$)	&$2^{-132}$	&\cellcolor{green!25}$\approx 2^{-188}$	&\cellcolor{green!25}$\approx 2^{-259}$	\\
$2^{60}$	&$2^{35}$	&$2^{45}$	&$2^{50}$	&\texttt{secp384r1} ($b \!=\! 192$, \! $p \!\approx\! 2^{384}$)	&$2^{-132}$	&\cellcolor{green!25}$\approx 2^{-160}$	&\cellcolor{green!25}$\approx 2^{-259}$	\\
$2^{60}$	&$2^{35}$	&$2^{55}$	&$2^{50}$	&\texttt{secp384r1} ($b \!=\! 192$, \! $p \!\approx\! 2^{384}$)	&$2^{-132}$	&$\approx 2^{-130}$	& \cellcolor{green!25}$\approx 2^{-254}$	\\
\midrule
$2^{60}$	&$2^{25}$	&$2^{35}$	&$2^{50}$	&\texttt{x448} ($b \!=\! 224$, \! $p \!\approx\! 2^{446}$)	&$2^{-164}$	&\cellcolor{green!25}$\approx 2^{-200}$	&\cellcolor{green!25}$\approx 2^{-280}$	\\
$2^{60}$	&$2^{25}$	&$2^{45}$	&$2^{50}$	&\texttt{x448} ($b \!=\! 224$, \! $p \!\approx\! 2^{446}$)	&$2^{-164}$	&\cellcolor{green!25}$\approx 2^{-170}$	&\cellcolor{green!25}$\approx 2^{-279}$	\\
$2^{60}$	&$2^{25}$	&$2^{55}$	&$2^{50}$	&\texttt{x448} ($b \!=\! 224$, \! $p \!\approx\! 2^{446}$)	&$2^{-164}$	&$\approx 2^{-140}$	& \cellcolor{green!25}$\approx 2^{-261}$	\\
$2^{60}$	&$2^{35}$	&$2^{35}$	&$2^{50}$	&\texttt{x448} ($b \!=\! 224$, \! $p \!\approx\! 2^{446}$)	&$2^{-164}$	&\cellcolor{green!25}$\approx 2^{-190}$	&\cellcolor{green!25}$\approx 2^{-280}$	\\
$2^{60}$	&$2^{35}$	&$2^{45}$	&$2^{50}$	&\texttt{x448} ($b \!=\! 224$, \! $p \!\approx\! 2^{446}$)	&$2^{-164}$	&$\approx 2^{-160}$	& \cellcolor{green!25}$\approx 2^{-279}$	\\
$2^{60}$	&$2^{35}$	&$2^{55}$	&$2^{50}$	&\texttt{x448} ($b \!=\! 224$, \! $p \!\approx\! 2^{446}$)	&$2^{-164}$	&$\approx 2^{-130}$	& \cellcolor{green!25}$\approx 2^{-261}$	\\
\midrule
$2^{60}$	&$2^{25}$	&$2^{35}$	&$2^{50}$	&\texttt{secp521r1} ($b \!=\! 256$, \! $p \!\approx\! 2^{521}$)	&$2^{-196}$	&\cellcolor{green!25}$\approx 2^{-200}$	&\cellcolor{green!25}$\approx 2^{-280}$	\\
$2^{60}$	&$2^{25}$	&$2^{45}$	&$2^{50}$	&\texttt{secp521r1} ($b \!=\! 256$, \! $p \!\approx\! 2^{521}$)	&$2^{-196}$	&$\approx 2^{-170}$	& \cellcolor{green!25}$\approx 2^{-279}$	\\
$2^{60}$	&$2^{25}$	&$2^{55}$	&$2^{50}$	&\texttt{secp521r1} ($b \!=\! 256$, \! $p \!\approx\! 2^{521}$)	&$2^{-196}$	&$\approx 2^{-140}$	& \cellcolor{green!25}$\approx 2^{-261}$	\\
$2^{60}$	&$2^{35}$	&$2^{35}$	&$2^{50}$	&\texttt{secp521r1} ($b \!=\! 256$, \! $p \!\approx\! 2^{521}$)	&$2^{-196}$	&$\approx 2^{-190}$	& \cellcolor{green!25}$\approx 2^{-280}$	\\
$2^{60}$	&$2^{35}$	&$2^{45}$	&$2^{50}$	&\texttt{secp521r1} ($b \!=\! 256$, \! $p \!\approx\! 2^{521}$)	&$2^{-196}$	&$\approx 2^{-160}$	& \cellcolor{green!25}$\approx 2^{-279}$	\\
$2^{60}$	&$2^{35}$	&$2^{55}$	&$2^{50}$	&\texttt{secp521r1} ($b \!=\! 256$, \! $p \!\approx\! 2^{521}$)	&$2^{-196}$	&$\approx 2^{-130}$	& \cellcolor{green!25}$\approx 2^{-261}$	\\
\midrule[1pt]
$2^{80}$	&$2^{25}$	&$2^{35}$	&$2^{70}$	&\texttt{secp256r1} ($b \!=\! 128$, \! $p \!\approx\! 2^{256}$)	&$2^{-48}$	&$\approx 2^{-21}$	& \cellcolor{green!25}$\approx 2^{-92}$	\\
$2^{80}$	&$2^{25}$	&$2^{45}$	&$2^{70}$	&\texttt{secp256r1} ($b \!=\! 128$, \! $p \!\approx\! 2^{256}$)	&$2^{-48}$	&$\approx 2^{-1}$	& \cellcolor{green!25}$\approx 2^{-92}$	\\
$2^{80}$	&$2^{25}$	&$2^{55}$	&$2^{70}$	&\texttt{secp256r1} ($b \!=\! 128$, \! $p \!\approx\! 2^{256}$)	&$2^{-48}$	&$\approx 2^{19}$	& \cellcolor{green!25}$\approx 2^{-92}$	\\
$2^{80}$	&$2^{35}$	&$2^{35}$	&$2^{70}$	&\texttt{secp256r1} ($b \!=\! 128$, \! $p \!\approx\! 2^{256}$)	&$2^{-48}$	&$\approx 2^{-21}$	& \cellcolor{green!25}$\approx 2^{-92}$	\\
$2^{80}$	&$2^{35}$	&$2^{45}$	&$2^{70}$	&\texttt{secp256r1} ($b \!=\! 128$, \! $p \!\approx\! 2^{256}$)	&$2^{-48}$	&$\approx 2^{-1}$	& \cellcolor{green!25}$\approx 2^{-92}$	\\
$2^{80}$	&$2^{35}$	&$2^{55}$	&$2^{70}$	&\texttt{secp256r1} ($b \!=\! 128$, \! $p \!\approx\! 2^{256}$)	&$2^{-48}$	&$\approx 2^{20}$	& \cellcolor{green!25}$\approx 2^{-92}$	\\
\midrule
$2^{80}$	&$2^{25}$	&$2^{35}$	&$2^{70}$	&\texttt{x25519} ($b \!=\! 128$, \! $p \!\approx\! 2^{252}$)	&$2^{-48}$	&$\approx 2^{-17}$	& \cellcolor{green!25}$\approx 2^{-88}$	\\
$2^{80}$	&$2^{25}$	&$2^{45}$	&$2^{70}$	&\texttt{x25519} ($b \!=\! 128$, \! $p \!\approx\! 2^{252}$)	&$2^{-48}$	&$\approx 2^{3}$	& \cellcolor{green!25}$\approx 2^{-88}$	\\
$2^{80}$	&$2^{25}$	&$2^{55}$	&$2^{70}$	&\texttt{x25519} ($b \!=\! 128$, \! $p \!\approx\! 2^{252}$)	&$2^{-48}$	&$\approx 2^{23}$	& \cellcolor{green!25}$\approx 2^{-88}$	\\
$2^{80}$	&$2^{35}$	&$2^{35}$	&$2^{70}$	&\texttt{x25519} ($b \!=\! 128$, \! $p \!\approx\! 2^{252}$)	&$2^{-48}$	&$\approx 2^{-17}$	& \cellcolor{green!25}$\approx 2^{-88}$	\\
$2^{80}$	&$2^{35}$	&$2^{45}$	&$2^{70}$	&\texttt{x25519} ($b \!=\! 128$, \! $p \!\approx\! 2^{252}$)	&$2^{-48}$	&$\approx 2^{3}$	& \cellcolor{green!25}$\approx 2^{-88}$	\\
$2^{80}$	&$2^{35}$	&$2^{55}$	&$2^{70}$	&\texttt{x25519} ($b \!=\! 128$, \! $p \!\approx\! 2^{252}$)	&$2^{-48}$	&$\approx 2^{23}$	& \cellcolor{green!25}$\approx 2^{-88}$	\\
\midrule
$2^{80}$	&$2^{25}$	&$2^{35}$	&$2^{70}$	&\texttt{secp384r1} ($b \!=\! 192$, \! $p \!\approx\! 2^{384}$)	&$2^{-112}$	&\cellcolor{green!25}$\approx 2^{-149}$	&\cellcolor{green!25}$\approx 2^{-219}$	\\
$2^{80}$	&$2^{25}$	&$2^{45}$	&$2^{70}$	&\texttt{secp384r1} ($b \!=\! 192$, \! $p \!\approx\! 2^{384}$)	&$2^{-112}$	&\cellcolor{green!25}$\approx 2^{-129}$	&\cellcolor{green!25}$\approx 2^{-219}$	\\
$2^{80}$	&$2^{25}$	&$2^{55}$	&$2^{70}$	&\texttt{secp384r1} ($b \!=\! 192$, \! $p \!\approx\! 2^{384}$)	&$2^{-112}$	&$\approx 2^{-109}$	& \cellcolor{green!25}$\approx 2^{-219}$	\\
$2^{80}$	&$2^{35}$	&$2^{35}$	&$2^{70}$	&\texttt{secp384r1} ($b \!=\! 192$, \! $p \!\approx\! 2^{384}$)	&$2^{-112}$	&\cellcolor{green!25}$\approx 2^{-149}$	&\cellcolor{green!25}$\approx 2^{-219}$	\\
$2^{80}$	&$2^{35}$	&$2^{45}$	&$2^{70}$	&\texttt{secp384r1} ($b \!=\! 192$, \! $p \!\approx\! 2^{384}$)	&$2^{-112}$	&\cellcolor{green!25}$\approx 2^{-129}$	&\cellcolor{green!25}$\approx 2^{-219}$	\\
$2^{80}$	&$2^{35}$	&$2^{55}$	&$2^{70}$	&\texttt{secp384r1} ($b \!=\! 192$, \! $p \!\approx\! 2^{384}$)	&$2^{-112}$	&$\approx 2^{-108}$	& \cellcolor{green!25}$\approx 2^{-219}$	\\
\midrule
$2^{80}$	&$2^{25}$	&$2^{35}$	&$2^{70}$	&\texttt{x448} ($b \!=\! 224$, \! $p \!\approx\! 2^{446}$)	&$2^{-144}$	&\cellcolor{green!25}$\approx 2^{-180}$	&\cellcolor{green!25}$\approx 2^{-240}$	\\
$2^{80}$	&$2^{25}$	&$2^{45}$	&$2^{70}$	&\texttt{x448} ($b \!=\! 224$, \! $p \!\approx\! 2^{446}$)	&$2^{-144}$	&\cellcolor{green!25}$\approx 2^{-150}$	&\cellcolor{green!25}$\approx 2^{-240}$	\\
$2^{80}$	&$2^{25}$	&$2^{55}$	&$2^{70}$	&\texttt{x448} ($b \!=\! 224$, \! $p \!\approx\! 2^{446}$)	&$2^{-144}$	&$\approx 2^{-120}$	& \cellcolor{green!25}$\approx 2^{-240}$	\\
$2^{80}$	&$2^{35}$	&$2^{35}$	&$2^{70}$	&\texttt{x448} ($b \!=\! 224$, \! $p \!\approx\! 2^{446}$)	&$2^{-144}$	&\cellcolor{green!25}$\approx 2^{-170}$	&\cellcolor{green!25}$\approx 2^{-240}$	\\
$2^{80}$	&$2^{35}$	&$2^{45}$	&$2^{70}$	&\texttt{x448} ($b \!=\! 224$, \! $p \!\approx\! 2^{446}$)	&$2^{-144}$	&$\approx 2^{-140}$	& \cellcolor{green!25}$\approx 2^{-240}$	\\
$2^{80}$	&$2^{35}$	&$2^{55}$	&$2^{70}$	&\texttt{x448} ($b \!=\! 224$, \! $p \!\approx\! 2^{446}$)	&$2^{-144}$	&$\approx 2^{-110}$	& \cellcolor{green!25}$\approx 2^{-240}$	\\
\midrule
$2^{80}$	&$2^{25}$	&$2^{35}$	&$2^{70}$	&\texttt{secp521r1} ($b \!=\! 256$, \! $p \!\approx\! 2^{521}$)	&$2^{-176}$	&\cellcolor{green!25}$\approx 2^{-180}$	&\cellcolor{green!25}$\approx 2^{-240}$	\\
$2^{80}$	&$2^{25}$	&$2^{45}$	&$2^{70}$	&\texttt{secp521r1} ($b \!=\! 256$, \! $p \!\approx\! 2^{521}$)	&$2^{-176}$	&$\approx 2^{-150}$	& \cellcolor{green!25}$\approx 2^{-240}$	\\
$2^{80}$	&$2^{25}$	&$2^{55}$	&$2^{70}$	&\texttt{secp521r1} ($b \!=\! 256$, \! $p \!\approx\! 2^{521}$)	&$2^{-176}$	&$\approx 2^{-120}$	& \cellcolor{green!25}$\approx 2^{-240}$	\\
$2^{80}$	&$2^{35}$	&$2^{35}$	&$2^{70}$	&\texttt{secp521r1} ($b \!=\! 256$, \! $p \!\approx\! 2^{521}$)	&$2^{-176}$	&$\approx 2^{-170}$	& \cellcolor{green!25}$\approx 2^{-240}$	\\
$2^{80}$	&$2^{35}$	&$2^{45}$	&$2^{70}$	&\texttt{secp521r1} ($b \!=\! 256$, \! $p \!\approx\! 2^{521}$)	&$2^{-176}$	&$\approx 2^{-140}$	& \cellcolor{green!25}$\approx 2^{-240}$	\\
$2^{80}$	&$2^{35}$	&$2^{55}$	&$2^{70}$	&\texttt{secp521r1} ($b \!=\! 256$, \! $p \!\approx\! 2^{521}$)	&$2^{-176}$	&$\approx 2^{-110}$	& \cellcolor{green!25}$\approx 2^{-240}$	\\
\bottomrule
	\end{tabular}}
	
	\medskip

	\caption{%
		Concrete advantages of a key exchange adversary with given resources $t$ (running time), $\#N$ (number of pre-shared keys), $\#S$ (number of sessions), and $\#RO$ (number of random oracle queries) in breaking the security of the TLS~1.3 PSK-(EC)DH handshake protocol.
		%
		Numbers based on the prior bounds by Dowling et al.~\cite{JC:DFGS21}
		and our bound for PSK-(EC)DHE in Corollary~\ref{cor:full-psk-ecdhe-ke}.
		``Target'' indicates the maximal advantage~$t/2^b$ tolerable for a given bound on $t$ when aiming for the respective curve's bit security level~$b$;
		entries in \colorbox{green!25}{green}-shaded cells meet that target.
		See Section~\ref{sec:evaluation} and Appendix~\ref{app:evaluation} for further details.
	}
\label{tbl:bounds-full-psk-dhe}
\end{table}
\iffull
\else
	\clearpage
\fi




	

%%% Local Variables:
%%% mode: latex
%%% TeX-master: "main"
%%% End:

\def\EvalTitle{Evaluation Details}
\iffull
	\subsection{\EvalTitle}
\else
	\newpage
	\section{\EvalTitle}
\fi
\label{app:evaluation}

In the following, we will briefly explain the reasoning behind each of our specific resource parameter estimates. 
An adversary in the MSKE game (cf.\ Definition~\ref{def:MSKE-security}) is limited in its runtime~$t$, the number of pre-shared keys~$\#N$, and distinct protocol sessions~$\#S$ it can observe or interact with, and the number of random oracle queries~$\#RO$ it can make.
This last quantity captures offline work the adversary spends on computing the hash function~$\Hash$, which in our analysis we model as random oracle.
The choice of ciphersuite enters the bound through the length of the symmetric session keys and pre-shared keys.
For the PSK-(EC)DHE handshake, the bound also depends on the underlying Diffie--Hellman group.


\paragraph{Runtime $t \in \{2^{40}, 2^{60}, 2^{80}\}$.}
We consider a range of adversarial runtimes from easily achievable ($2^{40}$ operations) to state-scaled computational power ($2^{80}$ operations). 

\paragraph{Random oracle queries $\#RO \in \{2^{40}, 2^{60}, 2^{80}\}$.}
The number of random oracle queries models the number of hash function computations an adversary is capable of computing. Accordingly, we scale the number of RO queries with the runtime, always setting $\#RO = t/2^{10}$.

\paragraph{Number of pre-shared keys $\#N \in \{2^{25}, 2^{35}\}$.}
The world's largest certificate authority Let's~Encrypt reports $\approx 2^{27.5}$ active certificates for fully-qualified domains.%
\footnote{\url{https://letsencrypt.org/stats/}} %% last checked 2021-09-29: 193M active certs, 252M fully-qualified domains certified
While not every \emph{user} of TLS~1.3 will perform resumption, our model counts the number of \emph{pre-shared keys},
where typically users may hold many pre-shared keys, with servers regularly issuing several PSKs per full-handshake connection for later resumption.
We hence estimate that the number of pre-shared keys accessible to a globally-scaled adversary may well exceed the reported number of (server) certificates.

\paragraph{Number of sessions $\#S \in \{2^{35}, 2^{45}, 2^{55}\}$.}
We use the same estimates as Davis and G{\"u}nther~\cite{ACNS:DavGun21}, based on Google's and Firefox's usage reports.%
\footnote{\url{https://transparencyreport.google.com/}, \url{https://telemetry.mozilla.org/}}
With a daily browser user base of $2$ billion ($\approx 2^{31}$) and an HTTPS traffic encryption rate in the range of $76$--$98\%$,
we estimate an adversary could encounter up to~$2^{55}$ distinct sessions over an extended time period.
Note that although the PSK handshakes are less commonly used by browsers than the full TLS~1.3 handshake, they are frequently used by embedded and low-powered devices which do not appear in these reports.
Naturally, we do not allow the number of sessions to exceed the adversary's runtime $t$.

\paragraph{Diffie--Hellman groups.}
There are ten Diffie--Hellman groups standardized for use with the PSK-(EC)DHE handshake: five elliptic-curve groups and five finite-field groups. 
We reduce to the security of the strong Diffie--Hellman assumption in each of these groups.
Davis and Günther gave a proof of hardness in the generic group model (GGM) for the strong DH problem.
This result is a good heuristic for elliptic-curve groups, but not for finite-field ones because they are vulnerable to index-calculus based attacks not covered by the GGM.
The elliptic-curve groups are more efficient and more widely used than finite-field groups, so we restrict our analysis to these groups:
\texttt{secp256r1}, \texttt{x25519}, \texttt{secp384r1}, \texttt{x448}, \texttt{secp521r1}.
For each group, we give in Table~\ref{tbl:bounds-full-psk-dhe} the order~$p$ and the expected security level~$b$ in bits.
We use the security level $b$ to determine the choice of hash function and the target security level for the entire PSK-(EC)DHE handshake.

\paragraph{Ciphersuite and symmetric lengths.}
Our bounds reduce to the collision resistance of the random oracle $\ROthash$, which models the handshake's hash function.
The choice of hash function also determines the length of the session and resumption keys.
TLS~1.3 has five ciphersuites, all of which set the hash function to be either $\SHA{256}$ or $\SHA{384}$.
For PSK-(EC)DHE mode, we select $\SHA{256}$ as the hash function whenever a curve with $128$-bit security is used and we select $\SHA{384}$ for higher-security curves.
As our results of Section~\ref{sec:ks-indiff} only apply to PSK-only mode when $\SHA{256}$ is the hash function, we always use $\SHA{256}$ and a target-security level of $128$ bits.

\begin{table}[t]
	\centering
% 	\fontsize{4.5}{5}\selectfont % smaller than \tiny
	\footnotesize
	\renewcommand{\arraystretch}{0.01}
	\renewcommand{\tabcolsep}{0.15cm}
	\vspace{-0.3cm} %% a little higher
	\begin{tabular}{@{}lllllll@{}}
		\toprule
		\multicolumn{4}{c}{Adversary resources}		&		& \multicolumn{2}{c}{PSK-only}	\\
		\cmidrule{1-4} \cmidrule{6-7}
		$t$	& $\#N$	& $\#S$ & $\#RO$ & Target $t/2^b$	& DFGS\,{\cite{JC:DFGS21}}~	& Us~{(Cor.~\ref{cor:psk-ke})} \\
		\midrule	
$2^{40}$	&$2^{25}$	&$2^{35}$	&$2^{30}$	&$2^{-88}$	&\cellcolor{green!25}$\approx 2^{-158}$	&\cellcolor{green!25}$\approx 2^{-173}$	\\
$2^{40}$	&$2^{35}$	&$2^{35}$	&$2^{30}$	&$2^{-88}$	&\cellcolor{green!25}$\approx 2^{-150}$	&\cellcolor{green!25}$\approx 2^{-173}$	\\
\midrule
$2^{60}$	&$2^{25}$	&$2^{35}$	&$2^{50}$	&$2^{-68}$	&\cellcolor{green!25}$\approx 2^{-119}$	&\cellcolor{green!25}$\approx 2^{-152}$	\\
$2^{60}$	&$2^{25}$	&$2^{45}$	&$2^{50}$	&$2^{-68}$	&\cellcolor{green!25}$\approx 2^{-109}$	&\cellcolor{green!25}$\approx 2^{-151}$	\\
$2^{60}$	&$2^{25}$	&$2^{55}$	&$2^{50}$	&$2^{-68}$	&\cellcolor{green!25}$\approx 2^{-99}$	&\cellcolor{green!25}$\approx 2^{-133}$	\\
$2^{60}$	&$2^{35}$	&$2^{35}$	&$2^{50}$	&$2^{-68}$	&\cellcolor{green!25}$\approx 2^{-119}$	&\cellcolor{green!25}$\approx 2^{-152}$	\\
$2^{60}$	&$2^{35}$	&$2^{45}$	&$2^{50}$	&$2^{-68}$	&\cellcolor{green!25}$\approx 2^{-109}$	&\cellcolor{green!25}$\approx 2^{-151}$	\\
$2^{60}$	&$2^{35}$	&$2^{55}$	&$2^{50}$	&$2^{-68}$	&\cellcolor{green!25}$\approx 2^{-99}$	&\cellcolor{green!25}$\approx 2^{-133}$	\\
\midrule
$2^{80}$	&$2^{25}$	&$2^{35}$	&$2^{70}$	&$2^{-48}$	&\cellcolor{green!25}$\approx 2^{-79}$	&\cellcolor{green!25}$\approx 2^{-112}$	\\
$2^{80}$	&$2^{25}$	&$2^{45}$	&$2^{70}$	&$2^{-48}$	&\cellcolor{green!25}$\approx 2^{-69}$	&\cellcolor{green!25}$\approx 2^{-112}$	\\
$2^{80}$	&$2^{25}$	&$2^{55}$	&$2^{70}$	&$2^{-48}$	&\cellcolor{green!25}$\approx 2^{-59}$	&\cellcolor{green!25}$\approx 2^{-112}$	\\
$2^{80}$	&$2^{35}$	&$2^{35}$	&$2^{70}$	&$2^{-48}$	&\cellcolor{green!25}$\approx 2^{-79}$	&\cellcolor{green!25}$\approx 2^{-112}$	\\
$2^{80}$	&$2^{35}$	&$2^{45}$	&$2^{70}$	&$2^{-48}$	&\cellcolor{green!25}$\approx 2^{-69}$	&\cellcolor{green!25}$\approx 2^{-112}$	\\
$2^{80}$	&$2^{35}$	&$2^{55}$	&$2^{70}$	&$2^{-48}$	&\cellcolor{green!25}$\approx 2^{-59}$	&\cellcolor{green!25}$\approx 2^{-112}$	\\
	\bottomrule
	\end{tabular}
	
	\medskip

	\caption{%
		Concrete advantages of a key exchange adversary with given resources $t$ (running time), $\#N$ (number of pre-shared keys), $\#S$ (number of sessions), and $\#RO$ (number of random oracle queries) in breaking the security of the TLS~1.3 PSK-only handshake protocol with a ciphersuite targeting $128$-bit security.
		%
		Numbers based on the prior bounds by Dowling et al.~\cite{JC:DFGS21}
		and our bound for PSK-only in Corollary~\ref{cor:psk-ke}.
		``Target'' indicates the maximal advantage~$t/2^b$ tolerable for a given bound on $t$ when aiming for the bit security level~$b = 128$;
		entries in \colorbox{green!25}{green}-shaded cells meet that target.
		We assume that the ciphersuite uses $\SHA{256}$ as its hash function (see Appendix~\ref{app:domsep} for further explanation).
	}
\label{tbl:bounds-full-psk-only}
\end{table}

\begin{table}[p]
	\centering
% 	\fontsize{4.5}{5}\selectfont % smaller than \tiny
	% \tiny
	\renewcommand{\arraystretch}{0.01}
	\renewcommand{\tabcolsep}{0.15cm}
	%\vspace{-0.5cm} %% a little higher
	\resizebox{!}{.42\textheight}{%
	\begin{tabular}{@{}llllllll@{}}
		\toprule
		\multicolumn{4}{c}{Adversary resources}		&&		& \multicolumn{2}{c}{PSK-(EC)DHE}	\\[-0.5mm]
		\cmidrule{1-4} \cmidrule{7-8}
		$t$	& $\#N$	& $\#S$ & $\#RO$ & Curve (\fullelse{bit security~$b$, group order~$p$}{bit sec.\!~$b$,\! order~$p$})	& Target $t/2^b$	& DFGS\,{\cite{JC:DFGS21}}~	& Us~{(Cor.~\ref{cor:full-psk-ecdhe-ke})} \\[-0.5mm]
		\midrule
$2^{40}$	&$2^{25}$	&$2^{35}$	&$2^{30}$	&\texttt{secp256r1} ($b \!=\! 128$, \! $p \!\approx\! 2^{256}$)	&$2^{-88}$	&\cellcolor{green!25}$\approx 2^{-92}$	&\cellcolor{green!25}$\approx 2^{-167}$	\\
$2^{40}$	&$2^{35}$	&$2^{35}$	&$2^{30}$	&\texttt{secp256r1} ($b \!=\! 128$, \! $p \!\approx\! 2^{256}$)	&$2^{-88}$	&$\approx 2^{-82}$	& \cellcolor{green!25}$\approx 2^{-167}$	\\
\midrule
$2^{40}$	&$2^{25}$	&$2^{35}$	&$2^{30}$	&\texttt{x25519} ($b \!=\! 128$, \! $p \!\approx\! 2^{252}$)	&$2^{-88}$	&\cellcolor{green!25}$\approx 2^{-92}$	&\cellcolor{green!25}$\approx 2^{-163}$	\\
$2^{40}$	&$2^{35}$	&$2^{35}$	&$2^{30}$	&\texttt{x25519} ($b \!=\! 128$, \! $p \!\approx\! 2^{252}$)	&$2^{-88}$	&$\approx 2^{-82}$	& \cellcolor{green!25}$\approx 2^{-163}$	\\
\midrule
$2^{40}$	&$2^{25}$	&$2^{35}$	&$2^{30}$	&\texttt{secp384r1} ($b \!=\! 192$, \! $p \!\approx\! 2^{384}$)	&$2^{-152}$	&\cellcolor{green!25}$\approx 2^{-220}$	&\cellcolor{green!25}$\approx 2^{-294}$	\\
$2^{40}$	&$2^{35}$	&$2^{35}$	&$2^{30}$	&\texttt{secp384r1} ($b \!=\! 192$, \! $p \!\approx\! 2^{384}$)	&$2^{-152}$	&\cellcolor{green!25}$\approx 2^{-210}$	&\cellcolor{green!25}$\approx 2^{-294}$	\\
\midrule
$2^{40}$	&$2^{25}$	&$2^{35}$	&$2^{30}$	&\texttt{x448} ($b \!=\! 224$, \! $p \!\approx\! 2^{446}$)	&$2^{-184}$	&\cellcolor{green!25}$\approx 2^{-220}$	&\cellcolor{green!25}$\approx 2^{-301}$	\\
$2^{40}$	&$2^{35}$	&$2^{35}$	&$2^{30}$	&\texttt{x448} ($b \!=\! 224$, \! $p \!\approx\! 2^{446}$)	&$2^{-184}$	&\cellcolor{green!25}$\approx 2^{-210}$	&\cellcolor{green!25}$\approx 2^{-301}$	\\
\midrule
$2^{40}$	&$2^{25}$	&$2^{35}$	&$2^{30}$	&\texttt{secp521r1} ($b \!=\! 256$, \! $p \!\approx\! 2^{521}$)	&$2^{-216}$	&\cellcolor{green!25}$\approx 2^{-220}$	&\cellcolor{green!25}$\approx 2^{-301}$	\\
$2^{40}$	&$2^{35}$	&$2^{35}$	&$2^{30}$	&\texttt{secp521r1} ($b \!=\! 256$, \! $p \!\approx\! 2^{521}$)	&$2^{-216}$	&$\approx 2^{-210}$	& \cellcolor{green!25}$\approx 2^{-301}$	\\
\midrule[1pt]
$2^{60}$	&$2^{25}$	&$2^{35}$	&$2^{50}$	&\texttt{secp256r1} ($b \!=\! 128$, \! $p \!\approx\! 2^{256}$)	&$2^{-68}$	&$\approx 2^{-61}$	& \cellcolor{green!25}$\approx 2^{-132}$	\\
$2^{60}$	&$2^{25}$	&$2^{45}$	&$2^{50}$	&\texttt{secp256r1} ($b \!=\! 128$, \! $p \!\approx\! 2^{256}$)	&$2^{-68}$	&$\approx 2^{-40}$	& \cellcolor{green!25}$\approx 2^{-132}$	\\
$2^{60}$	&$2^{25}$	&$2^{55}$	&$2^{50}$	&\texttt{secp256r1} ($b \!=\! 128$, \! $p \!\approx\! 2^{256}$)	&$2^{-68}$	&$\approx 2^{-12}$	& \cellcolor{green!25}$\approx 2^{-127}$	\\
$2^{60}$	&$2^{35}$	&$2^{35}$	&$2^{50}$	&\texttt{secp256r1} ($b \!=\! 128$, \! $p \!\approx\! 2^{256}$)	&$2^{-68}$	&$\approx 2^{-60}$	& \cellcolor{green!25}$\approx 2^{-132}$	\\
$2^{60}$	&$2^{35}$	&$2^{45}$	&$2^{50}$	&\texttt{secp256r1} ($b \!=\! 128$, \! $p \!\approx\! 2^{256}$)	&$2^{-68}$	&$\approx 2^{-32}$	& \cellcolor{green!25}$\approx 2^{-132}$	\\
$2^{60}$	&$2^{35}$	&$2^{55}$	&$2^{50}$	&\texttt{secp256r1} ($b \!=\! 128$, \! $p \!\approx\! 2^{256}$)	&$2^{-68}$	&$\approx 2^{-2}$	& \cellcolor{green!25}$\approx 2^{-127}$	\\
\midrule
$2^{60}$	&$2^{25}$	&$2^{35}$	&$2^{50}$	&\texttt{x25519} ($b \!=\! 128$, \! $p \!\approx\! 2^{252}$)	&$2^{-68}$	&$\approx 2^{-57}$	& \cellcolor{green!25}$\approx 2^{-128}$	\\
$2^{60}$	&$2^{25}$	&$2^{45}$	&$2^{50}$	&\texttt{x25519} ($b \!=\! 128$, \! $p \!\approx\! 2^{252}$)	&$2^{-68}$	&$\approx 2^{-37}$	& \cellcolor{green!25}$\approx 2^{-128}$	\\
$2^{60}$	&$2^{25}$	&$2^{55}$	&$2^{50}$	&\texttt{x25519} ($b \!=\! 128$, \! $p \!\approx\! 2^{252}$)	&$2^{-68}$	&$\approx 2^{-12}$	& \cellcolor{green!25}$\approx 2^{-123}$	\\
$2^{60}$	&$2^{35}$	&$2^{35}$	&$2^{50}$	&\texttt{x25519} ($b \!=\! 128$, \! $p \!\approx\! 2^{252}$)	&$2^{-68}$	&$\approx 2^{-57}$	& \cellcolor{green!25}$\approx 2^{-128}$	\\
$2^{60}$	&$2^{35}$	&$2^{45}$	&$2^{50}$	&\texttt{x25519} ($b \!=\! 128$, \! $p \!\approx\! 2^{252}$)	&$2^{-68}$	&$\approx 2^{-32}$	& \cellcolor{green!25}$\approx 2^{-128}$	\\
$2^{60}$	&$2^{35}$	&$2^{55}$	&$2^{50}$	&\texttt{x25519} ($b \!=\! 128$, \! $p \!\approx\! 2^{252}$)	&$2^{-68}$	&$\approx 2^{-2}$	& \cellcolor{green!25}$\approx 2^{-123}$	\\
\midrule
$2^{60}$	&$2^{25}$	&$2^{35}$	&$2^{50}$	&\texttt{secp384r1} ($b \!=\! 192$, \! $p \!\approx\! 2^{384}$)	&$2^{-132}$	&\cellcolor{green!25}$\approx 2^{-189}$	&\cellcolor{green!25}$\approx 2^{-259}$	\\
$2^{60}$	&$2^{25}$	&$2^{45}$	&$2^{50}$	&\texttt{secp384r1} ($b \!=\! 192$, \! $p \!\approx\! 2^{384}$)	&$2^{-132}$	&\cellcolor{green!25}$\approx 2^{-168}$	&\cellcolor{green!25}$\approx 2^{-259}$	\\
$2^{60}$	&$2^{25}$	&$2^{55}$	&$2^{50}$	&\texttt{secp384r1} ($b \!=\! 192$, \! $p \!\approx\! 2^{384}$)	&$2^{-132}$	&\cellcolor{green!25}$\approx 2^{-140}$	&\cellcolor{green!25}$\approx 2^{-254}$	\\
$2^{60}$	&$2^{35}$	&$2^{35}$	&$2^{50}$	&\texttt{secp384r1} ($b \!=\! 192$, \! $p \!\approx\! 2^{384}$)	&$2^{-132}$	&\cellcolor{green!25}$\approx 2^{-188}$	&\cellcolor{green!25}$\approx 2^{-259}$	\\
$2^{60}$	&$2^{35}$	&$2^{45}$	&$2^{50}$	&\texttt{secp384r1} ($b \!=\! 192$, \! $p \!\approx\! 2^{384}$)	&$2^{-132}$	&\cellcolor{green!25}$\approx 2^{-160}$	&\cellcolor{green!25}$\approx 2^{-259}$	\\
$2^{60}$	&$2^{35}$	&$2^{55}$	&$2^{50}$	&\texttt{secp384r1} ($b \!=\! 192$, \! $p \!\approx\! 2^{384}$)	&$2^{-132}$	&$\approx 2^{-130}$	& \cellcolor{green!25}$\approx 2^{-254}$	\\
\midrule
$2^{60}$	&$2^{25}$	&$2^{35}$	&$2^{50}$	&\texttt{x448} ($b \!=\! 224$, \! $p \!\approx\! 2^{446}$)	&$2^{-164}$	&\cellcolor{green!25}$\approx 2^{-200}$	&\cellcolor{green!25}$\approx 2^{-280}$	\\
$2^{60}$	&$2^{25}$	&$2^{45}$	&$2^{50}$	&\texttt{x448} ($b \!=\! 224$, \! $p \!\approx\! 2^{446}$)	&$2^{-164}$	&\cellcolor{green!25}$\approx 2^{-170}$	&\cellcolor{green!25}$\approx 2^{-279}$	\\
$2^{60}$	&$2^{25}$	&$2^{55}$	&$2^{50}$	&\texttt{x448} ($b \!=\! 224$, \! $p \!\approx\! 2^{446}$)	&$2^{-164}$	&$\approx 2^{-140}$	& \cellcolor{green!25}$\approx 2^{-261}$	\\
$2^{60}$	&$2^{35}$	&$2^{35}$	&$2^{50}$	&\texttt{x448} ($b \!=\! 224$, \! $p \!\approx\! 2^{446}$)	&$2^{-164}$	&\cellcolor{green!25}$\approx 2^{-190}$	&\cellcolor{green!25}$\approx 2^{-280}$	\\
$2^{60}$	&$2^{35}$	&$2^{45}$	&$2^{50}$	&\texttt{x448} ($b \!=\! 224$, \! $p \!\approx\! 2^{446}$)	&$2^{-164}$	&$\approx 2^{-160}$	& \cellcolor{green!25}$\approx 2^{-279}$	\\
$2^{60}$	&$2^{35}$	&$2^{55}$	&$2^{50}$	&\texttt{x448} ($b \!=\! 224$, \! $p \!\approx\! 2^{446}$)	&$2^{-164}$	&$\approx 2^{-130}$	& \cellcolor{green!25}$\approx 2^{-261}$	\\
\midrule
$2^{60}$	&$2^{25}$	&$2^{35}$	&$2^{50}$	&\texttt{secp521r1} ($b \!=\! 256$, \! $p \!\approx\! 2^{521}$)	&$2^{-196}$	&\cellcolor{green!25}$\approx 2^{-200}$	&\cellcolor{green!25}$\approx 2^{-280}$	\\
$2^{60}$	&$2^{25}$	&$2^{45}$	&$2^{50}$	&\texttt{secp521r1} ($b \!=\! 256$, \! $p \!\approx\! 2^{521}$)	&$2^{-196}$	&$\approx 2^{-170}$	& \cellcolor{green!25}$\approx 2^{-279}$	\\
$2^{60}$	&$2^{25}$	&$2^{55}$	&$2^{50}$	&\texttt{secp521r1} ($b \!=\! 256$, \! $p \!\approx\! 2^{521}$)	&$2^{-196}$	&$\approx 2^{-140}$	& \cellcolor{green!25}$\approx 2^{-261}$	\\
$2^{60}$	&$2^{35}$	&$2^{35}$	&$2^{50}$	&\texttt{secp521r1} ($b \!=\! 256$, \! $p \!\approx\! 2^{521}$)	&$2^{-196}$	&$\approx 2^{-190}$	& \cellcolor{green!25}$\approx 2^{-280}$	\\
$2^{60}$	&$2^{35}$	&$2^{45}$	&$2^{50}$	&\texttt{secp521r1} ($b \!=\! 256$, \! $p \!\approx\! 2^{521}$)	&$2^{-196}$	&$\approx 2^{-160}$	& \cellcolor{green!25}$\approx 2^{-279}$	\\
$2^{60}$	&$2^{35}$	&$2^{55}$	&$2^{50}$	&\texttt{secp521r1} ($b \!=\! 256$, \! $p \!\approx\! 2^{521}$)	&$2^{-196}$	&$\approx 2^{-130}$	& \cellcolor{green!25}$\approx 2^{-261}$	\\
\midrule[1pt]
$2^{80}$	&$2^{25}$	&$2^{35}$	&$2^{70}$	&\texttt{secp256r1} ($b \!=\! 128$, \! $p \!\approx\! 2^{256}$)	&$2^{-48}$	&$\approx 2^{-21}$	& \cellcolor{green!25}$\approx 2^{-92}$	\\
$2^{80}$	&$2^{25}$	&$2^{45}$	&$2^{70}$	&\texttt{secp256r1} ($b \!=\! 128$, \! $p \!\approx\! 2^{256}$)	&$2^{-48}$	&$\approx 2^{-1}$	& \cellcolor{green!25}$\approx 2^{-92}$	\\
$2^{80}$	&$2^{25}$	&$2^{55}$	&$2^{70}$	&\texttt{secp256r1} ($b \!=\! 128$, \! $p \!\approx\! 2^{256}$)	&$2^{-48}$	&$\approx 2^{19}$	& \cellcolor{green!25}$\approx 2^{-92}$	\\
$2^{80}$	&$2^{35}$	&$2^{35}$	&$2^{70}$	&\texttt{secp256r1} ($b \!=\! 128$, \! $p \!\approx\! 2^{256}$)	&$2^{-48}$	&$\approx 2^{-21}$	& \cellcolor{green!25}$\approx 2^{-92}$	\\
$2^{80}$	&$2^{35}$	&$2^{45}$	&$2^{70}$	&\texttt{secp256r1} ($b \!=\! 128$, \! $p \!\approx\! 2^{256}$)	&$2^{-48}$	&$\approx 2^{-1}$	& \cellcolor{green!25}$\approx 2^{-92}$	\\
$2^{80}$	&$2^{35}$	&$2^{55}$	&$2^{70}$	&\texttt{secp256r1} ($b \!=\! 128$, \! $p \!\approx\! 2^{256}$)	&$2^{-48}$	&$\approx 2^{20}$	& \cellcolor{green!25}$\approx 2^{-92}$	\\
\midrule
$2^{80}$	&$2^{25}$	&$2^{35}$	&$2^{70}$	&\texttt{x25519} ($b \!=\! 128$, \! $p \!\approx\! 2^{252}$)	&$2^{-48}$	&$\approx 2^{-17}$	& \cellcolor{green!25}$\approx 2^{-88}$	\\
$2^{80}$	&$2^{25}$	&$2^{45}$	&$2^{70}$	&\texttt{x25519} ($b \!=\! 128$, \! $p \!\approx\! 2^{252}$)	&$2^{-48}$	&$\approx 2^{3}$	& \cellcolor{green!25}$\approx 2^{-88}$	\\
$2^{80}$	&$2^{25}$	&$2^{55}$	&$2^{70}$	&\texttt{x25519} ($b \!=\! 128$, \! $p \!\approx\! 2^{252}$)	&$2^{-48}$	&$\approx 2^{23}$	& \cellcolor{green!25}$\approx 2^{-88}$	\\
$2^{80}$	&$2^{35}$	&$2^{35}$	&$2^{70}$	&\texttt{x25519} ($b \!=\! 128$, \! $p \!\approx\! 2^{252}$)	&$2^{-48}$	&$\approx 2^{-17}$	& \cellcolor{green!25}$\approx 2^{-88}$	\\
$2^{80}$	&$2^{35}$	&$2^{45}$	&$2^{70}$	&\texttt{x25519} ($b \!=\! 128$, \! $p \!\approx\! 2^{252}$)	&$2^{-48}$	&$\approx 2^{3}$	& \cellcolor{green!25}$\approx 2^{-88}$	\\
$2^{80}$	&$2^{35}$	&$2^{55}$	&$2^{70}$	&\texttt{x25519} ($b \!=\! 128$, \! $p \!\approx\! 2^{252}$)	&$2^{-48}$	&$\approx 2^{23}$	& \cellcolor{green!25}$\approx 2^{-88}$	\\
\midrule
$2^{80}$	&$2^{25}$	&$2^{35}$	&$2^{70}$	&\texttt{secp384r1} ($b \!=\! 192$, \! $p \!\approx\! 2^{384}$)	&$2^{-112}$	&\cellcolor{green!25}$\approx 2^{-149}$	&\cellcolor{green!25}$\approx 2^{-219}$	\\
$2^{80}$	&$2^{25}$	&$2^{45}$	&$2^{70}$	&\texttt{secp384r1} ($b \!=\! 192$, \! $p \!\approx\! 2^{384}$)	&$2^{-112}$	&\cellcolor{green!25}$\approx 2^{-129}$	&\cellcolor{green!25}$\approx 2^{-219}$	\\
$2^{80}$	&$2^{25}$	&$2^{55}$	&$2^{70}$	&\texttt{secp384r1} ($b \!=\! 192$, \! $p \!\approx\! 2^{384}$)	&$2^{-112}$	&$\approx 2^{-109}$	& \cellcolor{green!25}$\approx 2^{-219}$	\\
$2^{80}$	&$2^{35}$	&$2^{35}$	&$2^{70}$	&\texttt{secp384r1} ($b \!=\! 192$, \! $p \!\approx\! 2^{384}$)	&$2^{-112}$	&\cellcolor{green!25}$\approx 2^{-149}$	&\cellcolor{green!25}$\approx 2^{-219}$	\\
$2^{80}$	&$2^{35}$	&$2^{45}$	&$2^{70}$	&\texttt{secp384r1} ($b \!=\! 192$, \! $p \!\approx\! 2^{384}$)	&$2^{-112}$	&\cellcolor{green!25}$\approx 2^{-129}$	&\cellcolor{green!25}$\approx 2^{-219}$	\\
$2^{80}$	&$2^{35}$	&$2^{55}$	&$2^{70}$	&\texttt{secp384r1} ($b \!=\! 192$, \! $p \!\approx\! 2^{384}$)	&$2^{-112}$	&$\approx 2^{-108}$	& \cellcolor{green!25}$\approx 2^{-219}$	\\
\midrule
$2^{80}$	&$2^{25}$	&$2^{35}$	&$2^{70}$	&\texttt{x448} ($b \!=\! 224$, \! $p \!\approx\! 2^{446}$)	&$2^{-144}$	&\cellcolor{green!25}$\approx 2^{-180}$	&\cellcolor{green!25}$\approx 2^{-240}$	\\
$2^{80}$	&$2^{25}$	&$2^{45}$	&$2^{70}$	&\texttt{x448} ($b \!=\! 224$, \! $p \!\approx\! 2^{446}$)	&$2^{-144}$	&\cellcolor{green!25}$\approx 2^{-150}$	&\cellcolor{green!25}$\approx 2^{-240}$	\\
$2^{80}$	&$2^{25}$	&$2^{55}$	&$2^{70}$	&\texttt{x448} ($b \!=\! 224$, \! $p \!\approx\! 2^{446}$)	&$2^{-144}$	&$\approx 2^{-120}$	& \cellcolor{green!25}$\approx 2^{-240}$	\\
$2^{80}$	&$2^{35}$	&$2^{35}$	&$2^{70}$	&\texttt{x448} ($b \!=\! 224$, \! $p \!\approx\! 2^{446}$)	&$2^{-144}$	&\cellcolor{green!25}$\approx 2^{-170}$	&\cellcolor{green!25}$\approx 2^{-240}$	\\
$2^{80}$	&$2^{35}$	&$2^{45}$	&$2^{70}$	&\texttt{x448} ($b \!=\! 224$, \! $p \!\approx\! 2^{446}$)	&$2^{-144}$	&$\approx 2^{-140}$	& \cellcolor{green!25}$\approx 2^{-240}$	\\
$2^{80}$	&$2^{35}$	&$2^{55}$	&$2^{70}$	&\texttt{x448} ($b \!=\! 224$, \! $p \!\approx\! 2^{446}$)	&$2^{-144}$	&$\approx 2^{-110}$	& \cellcolor{green!25}$\approx 2^{-240}$	\\
\midrule
$2^{80}$	&$2^{25}$	&$2^{35}$	&$2^{70}$	&\texttt{secp521r1} ($b \!=\! 256$, \! $p \!\approx\! 2^{521}$)	&$2^{-176}$	&\cellcolor{green!25}$\approx 2^{-180}$	&\cellcolor{green!25}$\approx 2^{-240}$	\\
$2^{80}$	&$2^{25}$	&$2^{45}$	&$2^{70}$	&\texttt{secp521r1} ($b \!=\! 256$, \! $p \!\approx\! 2^{521}$)	&$2^{-176}$	&$\approx 2^{-150}$	& \cellcolor{green!25}$\approx 2^{-240}$	\\
$2^{80}$	&$2^{25}$	&$2^{55}$	&$2^{70}$	&\texttt{secp521r1} ($b \!=\! 256$, \! $p \!\approx\! 2^{521}$)	&$2^{-176}$	&$\approx 2^{-120}$	& \cellcolor{green!25}$\approx 2^{-240}$	\\
$2^{80}$	&$2^{35}$	&$2^{35}$	&$2^{70}$	&\texttt{secp521r1} ($b \!=\! 256$, \! $p \!\approx\! 2^{521}$)	&$2^{-176}$	&$\approx 2^{-170}$	& \cellcolor{green!25}$\approx 2^{-240}$	\\
$2^{80}$	&$2^{35}$	&$2^{45}$	&$2^{70}$	&\texttt{secp521r1} ($b \!=\! 256$, \! $p \!\approx\! 2^{521}$)	&$2^{-176}$	&$\approx 2^{-140}$	& \cellcolor{green!25}$\approx 2^{-240}$	\\
$2^{80}$	&$2^{35}$	&$2^{55}$	&$2^{70}$	&\texttt{secp521r1} ($b \!=\! 256$, \! $p \!\approx\! 2^{521}$)	&$2^{-176}$	&$\approx 2^{-110}$	& \cellcolor{green!25}$\approx 2^{-240}$	\\
\bottomrule
	\end{tabular}}
	
	\medskip

	\caption{%
		Concrete advantages of a key exchange adversary with given resources $t$ (running time), $\#N$ (number of pre-shared keys), $\#S$ (number of sessions), and $\#RO$ (number of random oracle queries) in breaking the security of the TLS~1.3 PSK-(EC)DH handshake protocol.
		%
		Numbers based on the prior bounds by Dowling et al.~\cite{JC:DFGS21}
		and our bound for PSK-(EC)DHE in Corollary~\ref{cor:full-psk-ecdhe-ke}.
		``Target'' indicates the maximal advantage~$t/2^b$ tolerable for a given bound on $t$ when aiming for the respective curve's bit security level~$b$;
		entries in \colorbox{green!25}{green}-shaded cells meet that target.
		See Section~\ref{sec:evaluation} and Appendix~\ref{app:evaluation} for further details.
	}
\label{tbl:bounds-full-psk-dhe}
\end{table}
\iffull
\else
	\clearpage
\fi

\section*{Acknowledgements}
%\TODO{...}
We thank the anonymous reviewers of EUROCRYPT~2022 for their helpful comments.
%Felix Günther was supported in part by German Research Foundation (DFG) Research Fellowship grant~\mbox{GU~1859/1-1}.
%Tibor Jager was supported by the European Research Council (ERC) under the European Union's Horizon 2020 research and innovation programme, grant agreement 802823.
%Some of this work was done while Hannah Davis was visiting ETH Zurich.

\mathversion{normal}

\chapter{On the concrete security of TLS 1.3 PSK Mode}\label{chap:tight-ake-ext}
\mathversion{normal3}
\section{Introduction}
\label{sec:introduction}

The Transport Layer Security (TLS) protocol~\cite{rfc8446} is responsible for securing billions of Internet connections every day.
Usage statistics for Google Chrome%
\fullonly{\footnote{\url{https://transparencyreport.google.com/https/}}}		%% last checked 2020-09-03: 76% -- 98%
and Mozilla Firefox%
\fullonly{\footnote{\url{https://telemetry.mozilla.org/}}}				%% last checked 2020-09-03: 89%  (HTTP_PAGELOAD_IS_SSL)
report that $76$--$98$\% of all web page accesses are encrypted.%
At the heart of TLS is an authenticated key exchange (AKE) protocol, the so-called handshake protocol, responsible for providing the parties (client and server) with a shared, symmetric key that is fresh, private and authenticated.
The ensuing record layer secures data using this key.
The AKE protocol of TLS is based on the \SIGMA (``SIGn-and-MAc'') design of Krawczyk~\cite{C:Krawczyk03} for the Internet Key Exchange (IKE) protocol~\cite{rfc2409} of IPsec~\cite{rfc2401},
which generically augments an unauthenticated, ephemeral Diffie--Hellman (DH) key exchange with authenticating signatures and MACs.

Naturally, the \SIGMA AKE protocol and its incarnation in TLS have been the recipients of proofs of security.
We contend that these largely justify the AKE protocols in principle, but not in practice,
meaning not for the parameters in actual use and at the desired or expected level of security.
Our work takes steps towards filling this gap.


\iffull
\subsection{Qualitative and Quantitative Bounds}
\else
\subsubsection*{Qualitative and quantitative bounds.}
\fi

Let us expand on this.
The protocols~$\KE$ we consider are built from
a cyclic group~$\group$ in which some DH problem~$\mathsf{P}$ is assumed to be hard,
a pseudorandom function~$\PRF$ and unforgeable signature and MAC schemes~$\SIGScheme$ and~$\MACScheme$.
The target for~$\KE$ is session-key security with explicit authentication as originating from~\cite{C:BelRog93,EC:BelPoiRog00,EC:CanKra01}.
A proof of security has both a qualitative and quantitative dimension.
Qualitatively, a proof of security for the AKE protocol~$\KE$ says that $\KE$ meets its target definition assuming the building blocks meet theirs,
where, in either case, meeting the definition means any poly-time adversary has negligible advantage in violating it.

The quantitative dimension associates to each adversary in the security game of~$\KE$ a set of resources~$r$,
representing its runtime and attack surface (e.g., the number of users and executed protocol sessions the adversary has access to).
It then relates the maximum advantage of any $r$-resource adversary in breaking $\KE$'s security to likewise advantage functions for the building blocks
through an equation of the (simplified) form
\[
	\Adv_{\KE}(r) \leq %f(r) + 
	f_\group \cdot \Adv^{\mathsf{P}}_{\group}(r_\group) + f_{\SIGScheme} \cdot \Adv^{\EUFCMA}_{\SIGScheme}(r_{\SIGScheme}) + \dots,
\]
deriving quantitative factors~$f_\mathsf{X}$ and resources~$r_\mathsf{X}$ for the advantage of each building block~$\mathsf{X}$.

Speaking asymptotically again, when $f_\mathsf{X}$ and $r_\mathsf{X}$ are polynomial functions in~$r$,
then $\Adv_{\KE}(r)$ is negligible whenever all building blocks' advantages are.
Due to the complexity of key exchange models and the challenging task of combining the right components in a secure manner,
key exchange analyses (including prior work on \SIGMA~\cite{C:CanKra02} and TLS~1.3~\fullelse{\cite{CCS:DFGS15,EuroSP:KraWee16,EPRINT:DFGS16,EuroSP:FisGue17,JC:DFGS21}}{\cite{CCS:DFGS15,EuroSP:KraWee16,EuroSP:FisGue17,JC:DFGS21}}) indeed often remain abstract and consider only qualitative, asymptotic security bounds.

Standardized protocols like TLS in contrast have to define concrete choices for each cryptographic building block.
This involves considering reasonable estimates for adversarial resources (like runtime~$t$ and number of key-exchange model queries~$q$) and specific instances and parameters for the underlying components~$\mathsf{X}$.
One would hope that key exchange proofs can provide guidance in making sound choices that result in the desired overall security level.
Unfortunately, AKE security bounds regularly are highly non-tight, meaning that $f_\mathsf{X}$ and/or $r_\mathsf{X}$ for some components~$\mathsf{X}$ are so large that reasonable stand-alone parameters for~$\mathsf{X}$ yield vacuous key exchange advantages for practical parameters.
While the asymptotic bound tells us that scaling up the parameters for~$\mathsf{X}$ (say, the DDH problem~\cite{Boneh98}) will at some point result in a secure overall advantage,
this causes efficiency concerns (e.g., doubling elliptic curve DH security parameters means quadrupling the cost for group operations) and hence does not happen in practice.
\begin{table}[t]
	\centering
	\small
	
	\renewcommand{\arraystretch}{0.001}
	\renewcommand{\tabcolsep}{0.05cm}
	\begin{tabular}{@{}lllllllllll@{}}
	\toprule
	\multicolumn{3}{c}{Adv.\ resources}		&&&		& \multicolumn{2}{c}{\SIGMA}	& \hspace{0.2cm} & \multicolumn{2}{c}{TLS~1.3} \\
	\cmidrule{1-3} \cmidrule{7-8} \cmidrule{10-11}
	$t$~~~~~~	& $\#U$~~	& $\#S$ && Curve~~~~~~~	& Target~	& CK\,{\scriptsize\cite{C:CanKra02}}~	& Us~{\scriptsize(Thm.~\ref{thm:SIGMAI})}	&& DFGS\,{\scriptsize\cite{JC:DFGS21}}~	& Us~{\scriptsize(Thm.~\ref{thm:tls})} \\
	\midrule
$2^{60}$	&$2^{20}$	&$2^{35}$	&&\texttt{secp256r1} 	&$2^{-68}$	&\cellcolor{red!25}$\approx 2^{-61}$	&$\approx 2^{-116}$	&& \cellcolor{red!25}$\approx 2^{-64}$	&$\approx 2^{-116}$	 \\
$2^{60}$	&$2^{30}$	&$2^{55}$	&&\texttt{secp256r1}	&$2^{-68}$	&\cellcolor{red!25}$\approx 2^{-21}$	&$\approx 2^{-106}$	&& \cellcolor{red!25}$\approx 2^{-24}$	&$\approx 2^{-106}$	 \\
\midrule
$2^{60}$	&$2^{20}$	&$2^{35}$	&&\texttt{x25519}	&$2^{-68}$	&\cellcolor{red!25}$\approx 2^{-57}$	&$\approx 2^{-112}$	&& \cellcolor{red!25}$\approx 2^{-60}$	&$\approx 2^{-112}$	 \\
$2^{60}$	&$2^{30}$	&$2^{55}$	&&\texttt{x25519}	&$2^{-68}$	&\cellcolor{red!25}$\approx 2^{-17}$	&$\approx 2^{-102}$	&& \cellcolor{red!25}$\approx 2^{-20}$	&$\approx 2^{-102}$	 \\
% $2^{60}$	&$2^{20}$	&$2^{35}$	&&\texttt{secp384r1}	&$2^{-132}$	&$\approx 2^{-189}$	& $\approx 2^{-244}$	&& $\approx 2^{-192}$	& $\approx 2^{-244}$	 \\
% $2^{60}$	&$2^{30}$	&$2^{55}$	&&\texttt{secp384r1}	&$2^{-132}$	&$\approx 2^{-149}$	& $\approx 2^{-234}$	&& $\approx 2^{-152}$	& $\approx 2^{-234}$	  \\
\midrule
\midrule
$2^{80}$	&$2^{20}$	&$2^{35}$	&&\texttt{secp256r1}	&$2^{-48}$	&\cellcolor{red!25}$\approx 2^{-21}$	&$\approx 2^{-76}$	&& \cellcolor{red!25}$\approx 2^{-24}$	&$\approx 2^{-76}$	 \\
$2^{80}$	&$2^{30}$	&$2^{55}$	&&\texttt{secp256r1}	&$2^{-48}$	&\cellcolor{red!25}1			&$\approx 2^{-66}$	&& \cellcolor{red!25}1			&$\approx 2^{-66}$	 \\
\midrule
$2^{80}$	&$2^{20}$	&$2^{35}$	&&\texttt{x25519}	&$2^{-48}$	&\cellcolor{red!25}$\approx 2^{-17}$	&$\approx 2^{-72}$	&& \cellcolor{red!25}$\approx 2^{-20}$	&$\approx 2^{-72}$	 \\
$2^{80}$	&$2^{30}$	&$2^{55}$	&&\texttt{x25519}	&$2^{-48}$	&\cellcolor{red!25}1			&$\approx 2^{-62}$	&& \cellcolor{red!25}1			&$\approx 2^{-62}$	 \\
\midrule
$2^{80}$	&$2^{20}$	&$2^{35}$	&&\texttt{secp384r1}	&$2^{-112}$	&$\approx 2^{-149}$	& $\approx 2^{-204}$	&& $\approx 2^{-152}$	& $\approx 2^{-204}$	 \\
$2^{80}$	&$2^{30}$	&$2^{55}$	&&\texttt{secp384r1}	&$2^{-112}$	&\cellcolor{red!25}$\approx 2^{-109}$	&$\approx 2^{-194}$	&& \cellcolor{orange!25}$\approx 2^{-112}$	& $\approx 2^{-194}$	 \\
	\bottomrule
	\end{tabular}
	
	\medskip
	
	\caption{%
		Exemplary concrete advantages of a key exchange adversary with given resources $t$ (running time), $\#U$ (number of users), $\#S$ (number of sessions), in breaking the security of the \SIGMA and TLS~1.3 protocols
		when instantiated with curve \texttt{secp256r1}, \texttt{secp384r1}, or \texttt{x25519},
		based on the prior bounds by Canetti-Krawczyk~\cite{C:CanKra02} resp.\ Dowling et al.~\cite{JC:DFGS21}, and the bounds we establish (Theorem~\ref{thm:SIGMAI} and~\ref{thm:tls}).
		Target indicates the maximal advantage~$t/2^b$ tolerable when aiming for the respective curve's security level ($b = 128$ resp.\ $192$ bits);
		entries in red-shaded cells miss that target.
		See Section~\ref{sec:evaluation} %and Appendix~\ref{apx:evaluation} 
		for full details and curves \texttt{secp521r1} and~\texttt{x448}.
	}
	\label{tbl:bounds-overview}
\end{table}

We illustrate in Table~\ref{tbl:bounds-overview} the effects of the non-tight bounds for \SIGMA and TLS~1.3
when instantiating the protocols with NIST curves \texttt{secp256r1}, \texttt{secp384r1}~\cite{NIST:FIPS-186-4}, or curve \texttt{x25519}~\cite{rfc7748} and idealizing the protocols' other components (see Section~\ref{sec:evaluation} for full details).
Following the curves' security, we aim at a security level of~$128$~bits, resp.~$192$~bits, meaning the ratio of an adversary's runtime to its advantage should be bounded by~$2^{-128}$, resp.~$2^{-192}$.
When considering the advantage of key exchange adversaries running in time~$t$, interacting in the security game with $\#U$ users and $\#S$ sessions,
we can see that previous security bounds fail to meet the targeted security level
for real-world--scale parameters ($\#U$ ranging in $2^{20}$--$2^{30}$ based on $2^{27}$ active certificates on the Internet%
\fullonly{\footnote{\url{https://letsencrypt.org/stats/}}}%		%% last checked 2020-09-03: 136M active certs, 227M fully-qualified domains certified
, $\#S$ ranging in $2^{35}$--$2^{55}$ based on $2^{32}$ Internet users and $2^{33}$ daily Google searches%
\fullonly{\footnote{\url{https://www.internetlivestats.com/}}}%		%% last checked 2020-09-03: 85479 searches per second
).
In the security analysis by Canetti and Krawczyk~\cite{C:CanKra02} (CK) for \SIGMA, the factor associated to the decisional Diffie--Hellman problem is $f_{\DDH}(t,\#U,\#S) = \#U \cdot \#S$,
where $\#U$ and $\#S$ again are the number of users, resp.\ sessions, accessible by the adversary.
The analysis by Dowling et al.~\cite{JC:DFGS21} (DFGS) for TLS~1.3 reduces to the strong Diffie--Hellman problem~\cite{RSA:AbdBelRog01}---via the PRF-ODH assumption~\cite{C:JKSS12,C:BFGJ17}---with factor $f_{\strongDH}(t,\#U,\#S) = (\#S)^2$.
In contrast, we reduce to the strong Diffie--Hellman problem with a constant factor for both \SIGMA and TLS~1.3.

Let us discuss three data points from Table~\ref{tbl:bounds-overview}:
\begin{enumerate}
	\item Already with medium-sized resources, investing time~$t = 2^{60}$ and interacting with a million users ($\#U = 2^{20}$) and a few billion sessions ($\#S = 2^{35}$), the CK~\cite{C:CanKra02} and DFGS~\cite{JC:DFGS21} advantage bounds for \SIGMA and TLS~1.3 with curves \texttt{secp256r1} and \texttt{x25519} fall $6$--$11$~bits below the target of~$2^{-68}$ for $128$-bit security.
	
	\item When considering a more powerful, global-scale adversary ($t = 2^{80}$, $\#U = 2^{30}$, $\#S = 2^{55}$), both CK and DFGS bounds for \texttt{secp256r1}/\texttt{x25519} become fully vacuous;
	the upper bound on the probability of the adversary breaking the protocol is~$1$.
	% (More precisely, the bounds yield an advantage of~$2^{16}$ resp.\ $2^{8}$.)
	We stress that \texttt{secp256r1} is the mandatory-to-implement curve for TLS~1.3;
	\texttt{secp256r1} and \texttt{x25519} together make up for 90\% of the TLS~1.3 ECDHE handshakes reported through Firefox Telemetry.

	\item Finally, and notably, even switching to the higher-security curve \texttt{secp384r1} helps only marginally in the latter case:
	the resulting advantage against \SIGMA falls $3$~bits short of the $192$-bit security target of~$2^{-112}$,
	and the TLS advantage bound only barely meets that target.
\end{enumerate}
For all curves and choices of parameters, our bounds do better. 


\iffull
\subsection{Contributions}
\else
\subsubsection*{Contributions\lncsdot}
\fi

Most prior results in tightly secure key exchange (e.g., \cite{TCC:BHJKL15,C:GjoJag18}) apply only to bespoke protocols, carefully designed to allow for tighter proof techniques, at the cost of requiring more complex primitives which, in the end, eat up the gained practical efficiency.
\iffull
Recently, Cohn-Gordon et al.~\cite{C:CCGJJ19,EPRINT:CCGJJ19} established a proof strategy for a simple and efficient DH key exchange with reasonable tightness loss (only linear in the number of users~$\#U$), achieving implicit authentication through static DH keys through careful key derivation via a random oracle~\cite{CCS:BelRog93} with an optional explicit-authentication step.

\fi
Our work in contrast establishes tight security for standardized AKE protocols.
We give tight reductions for the security of \SIGMA and TLS~1.3 to the strong Diffie--Hellman problem~\cite{RSA:AbdBelRog01},
which in addition we prove is as hard as the discrete logarithm problem in the generic group model (GGM)~\cite{EC:Shoup97,IMA:Maurer05}.
Instantiating our bounds shows that, with standardized real-world parameters, we achieve the intended security levels even when considering powerful, globally-scaled attackers.


\iffull
\paragraph{Code-based security model and proofs}
For our proofs, we provide detailed proof steps and reductions using the code-based game-playing framework of Bellare and Rogaway~\cite{EC:BelRog06}.
Our security model is similar to the one applied by Cohn-Gordon et al.~\cite{C:CCGJJ19},
%considering in particular compromises of long-term secrets and session keys (but not internal state or randomness),
but formalized also as a code-based game (in Section~\ref{sec:ake-model}) and stronger in that it captures explicit authentication and regular (``perfect'') forward secrecy (instead of only weak forward secrecy in~\cite{C:CCGJJ19}).
\else

\fi


\paragraph{Tighter security proof of SIGMA(-I)}
We establish fully quantitative security bounds for \SIGMA and its identity-protecting variant~\SIGMAI~\cite{C:Krawczyk03} in Sections~\ref{sec:sigma} and~\ref{sec:sigma-proof}.
Our result is for BR-like~\cite{C:BelRog93} key exchange security and gives a tight reduction to the strong Diffie--Hellman problem~\cite{RSA:AbdBelRog01} in the used DH group, and to the multi-user (mu) security of the employed pseudorandom function (PRF), signature scheme, and MAC scheme, adapting the techniques by Cohn-Gordon et al.~\cite{C:CCGJJ19} in the random oracle model~\cite{CCS:BelRog93}.
The latter mu-security bounds are essentially equivalent to the corresponding bounds by CK~\cite{C:CanKra02}.
Our improvement comes from shaving off a factor of $\#U \cdot \#S$ (number of users times number of sessions) on the DH problem advantage compared to CK.
While we move to the interactive strong Diffie--Hellman problem (compared to \fullelse{the decisional DH (DDH) problem~\cite{Boneh98} used in~\cite{C:CanKra02}}{DDH~\cite{Boneh98} used in~\cite{C:CanKra02}}),
we prove (in Appendix~\ref{apx:strongDHproof}) that the strong DH problem, like DDH, is as hard as solving discrete logarithms in the generic group model~\cite{EC:Shoup97,IMA:Maurer05}%
\fullonly{, reflecting the (only generic) algorithms known for solving discrete logarithms in elliptic curve groups}.


\paragraph{Tighter security proof for the TLS~1.3 DH handshake}
We likewise establish fully quantitative security bounds for the key exchange of the recently standardized newest version of the Transport Layer Security protocol, TLS~1.3~\cite{rfc8446}, in Sections~\ref{sec:tls} and~\ref{sec:tls-proof}.
The main quantitative improvement in our reduction is again a tight reduction to the strong DH problem, whereas prior bounds by DFGS~\cite{JC:DFGS21} incurred a quadratic loss to the PRF-ODH assumption~\cite{C:JKSS12,C:BFGJ17}, a loss which translates directly to strong DH~\cite{C:BFGJ17}.
While TLS~1.3 roughly follows the \SIGMAI design, its cascading key schedule impedes the precise technique of Cohn-Gordon et al.~\cite{C:CCGJJ19} and a direct application of our results on \SIGMAI, as no single function (to be modeled as a random oracle) binds the Diffie--Hellman values to the session context.
We therefore have to carefully adapt the proof to accommodate the more complex key schedule and other core variations in TLS~1.3's key exchange, achieving conceptually similar tightness results as for \SIGMAI.
% This is reflected in our concrete security bounds for TLS~1.3 based on standardized components:
% for real-world resource parameters (cf.\ Table~\ref{tbl:bounds-overview} and Section~\ref{sec:evaluation}) they meet the targeted security levels of the mandatory-to-implement curve~\texttt{secp256r1} as well as \texttt{secp384r1} and \texttt{x25519},
% and improve upon the DFGS bounds by up to $82$~bits of security.


\paragraph{Evaluation}
In Section~\ref{sec:evaluation}, we evaluate the concrete security implications of our improved bounds for \SIGMA and TLS~1.3 for a wide range of real-world resource parameters and all five elliptic curves \fullonly{(\texttt{secp256r1}, \texttt{secp384r1}, \texttt{secp521r1},\texttt{x25519}, \texttt{x448}) }standardized for use in TLS~1.3~\cite{rfc8446},
a summary of which is displayed in Table~\ref{tbl:bounds-overview}.
\iffull
Leveraging our GGM bound for the strong Diffie--Hellman problem, we focus on the hardness of solving discrete logarithms in the respective elliptic curve groups, instantiating signatures based on ECDSA~\cite{NIST:FIPS-186-4} resp.\ EdDSA~\cite{CHES:BDLSY11}.
We idealized the symmetric PRF, MAC, and hash function primitives (in two variants, with key and output sizes twice as large as the curve's security level, or fixed at $256$~bits corresponding to the choice in most TLS~1.3 cipher suites).

\fi
We report that our tighter proofs indeed materialize for a wide range of real-world resource parameters%
\fullonly{ (adversary runtime~$t \in \{2^{40},2^{60},2^{80}\}$, number of users~$\#U \in \{2^{20},2^{30}\}$, and number of sessions~$\#S \in \{2^{35},2^{45},2^{55}\}$)}.
The resulting attacker advantages meet the targeted security levels of all five curves.
% The resulting attacker advantages meet the targeted security levels of curves~\texttt{secp256r1} (mandatory to implement for TLS~1.3) as well as \texttt{secp384r1} and \texttt{x25519}.
% (For higher-security curves \texttt{secp521r1} and \texttt{x448} and high-end adversary parameters, the idealized mu-security PRF and MAC loss becomes the dominating component, requiring key/output sizes larger than $256$~bits.)
In comparison to the prior CK~\cite{C:CanKra02} \SIGMA and DFGS~\cite{JC:DFGS21} TLS~1.3 bounds,
our results improve the obtained security across these real-world parameters by up to~$85$~bits for \SIGMA and $92$~bits for TLS~1.3, respectively.


\iffull

\iffull
\subsection{Optimizations, Limitations, and Possible Extensions}
\else
\subsubsection*{Optimizations, limitations, and possible extensions\lncsdot}
\fi
\SIGMA being a generic AKE design, the signature, PRF, and MAC schemes may be instantiated with primitives optimized for multi-user security.
While we focus on standardized and deployed schemes in our evaluation without assuming tight mu-security, our \SIGMA bound allows to directly leverage such optimization.
For PRFs and MACs,  efficient candidates exist (e.g., AMAC~\cite{EC:BelBerTes16}).
For signatures, tight mu-security is more challenging~\cite{EC:BJLS16} and often involves computationally much more expensive constructions~\cite{TCC:BHJKL15}.

Like Cohn-Gordon et al.~\cite{C:CCGJJ19}, our key exchange security model considers exposure of long-term secrets and session keys,
but does not allow revealing internal session state or randomness (as in the (e)CK model~\cite{EC:CanKra01,PROVSEC:LaMLauMit07}).
This is appropriate for protocols like TLS~1.3 not aiming to protect against such threats.
The original \SIGMA proof~\cite{C:CanKra02} did establish security in the CK model~\cite{EC:CanKra01} allowing exposure of session state;
in that sense our results are qualitatively weaker.
In recent work, Jager et al.~\cite{EC:JKRS21} give a tightly secure protocol which uses symmetric state encryption to protect against ephemeral state reveals.
Establishing a tight security reduction for a SIGMA-style DH-based AKE protocol which can handle adaptive compromises of session state (including DH exponents) remains a challenging open problem.

In our proofs, we crucially rely on the ability to observe and program a random oracle used for key derivation in the AKE protocol, borrowing from~\cite{C:CCGJJ19}.
Notably, the approach of Cohn-Gordon et al.\ is tailored to an AKE protocol achieving authenticity implicitly through mixing long-term DH keys into the key derivation.
Our proofs can hence be seen as translating and adapting their technique to the setting of \SIGMA and TLS~1.3, where an unauthenticated ephemeral DH exchange is explicitly authenticated through signatures and MACs,
confirming that the generic \SIGMA design as well as the standardized TLS~1.3 protocol bind enough context to their DH shares for this proof technique to work.
Leveraging the random oracle model~\cite{CCS:BelRog93} is another qualitative difference compared to the original \SIGMA proof~\cite{C:CanKra02} in the standard model.
Interestingly, this distinction vanishes in comparison to the provable security results for the TLS~1.3 handshake protocol~\cite{CCS:DFGS15,EPRINT:DFGS16,EuroSP:FisGue17,JC:DFGS21} which employ the PRF-ODH assumption~\cite{C:JKSS12,C:BFGJ17},
an interactive assumption which plausibly can only be instantiated in the random oracle model (from the strong DH assumption).
\fi

% \old{%
% The DFGS analyses of TLS~1.3 establish security in a multi-stage key exchange (MSKE) model~\cite{CCS:FisGue14}, proving security not only of the final session key, but also of intermediate handshake encryption keys and further secrets.
% While our proofs (for both \SIGMA and TLS~1.3) establish security of the intermediate (handshake) encryption key, too,
% we do not treat them as first-class keys available to the adversary (e.g., through revealing them).
% We expect that our results extend to a MSKE treatment, leaving this extension to possible future work.
% }

\iffull
\subsection{Concurrent Work}
\else
\subsubsection*{Concurrent work\lncsdot}
\fi

In concurrent and independent work, Diemert and Jager (DJ)~\cite{JC:DieJag20} studied the tight security of the main TLS~1.3 handshake.
Their work also tightly reduces the security of TLS~1.3 to the strong Diffie--Hellman problem by extending the technique of Cohn-Gordon et al.~\cite{C:CCGJJ19}, and their bounds and ours are similarly tight.
When instantiated with real-world parameters, both bounds are dominated by the same terms, as we will demonstrate in Section~\ref{sec:evaluation}.
Our proof differs from theirs in two key ways:
We use an incomparable security model that is weaker in some ways and stronger in others, and we approximate the TLS~1.3 key schedule with fewer random oracles.
We also contextualize our results quite differently than the DJ work, with a detailed numerical analysis that is enabled by our fully parameterized, concrete bounds.
Uniquely to this work, we treat the more generic \SIGMAI protocol and justify our use of the strong DH problem with new bounds in the generic group model.
Diemert and Jager~\cite{JC:DieJag20} in turn study tight composition with the TLS record protocol. 

The DJ analysis is carried out in the multi-stage key exchange model~\cite{CCS:FisGue14}, proving security not only of the final session key, but also of intermediate handshake encryption keys and further secrets.
While our proof does show security of these intermediate keys, we do not treat them as first-class keys accessible to the adversary through dedicated queries in the security model.
Unlike either the DJ or Cohn-Gordon et al.\ works, our model addresses explicit authentication, which we prove via HMAC's unforgeability.

To tackle the challenge that TLS~1.3's key schedule does not bind DH values and session context in one function, DJ model the full cascading derivation of each intermediate key monolithically as an independent, programmable random oracle (cf.~\cite[Theorem~6]{JC:DieJag20}). 
We instead model the key schedule's inner HKDF~\cite{C:Krawczyk10} extraction and expansion functions as two individual random oracles, carefully connected via efficient look-up tables, yielding a slightly less extensive use of random oracles and compensating for the existence of shared computations in the derivation of multiple keys.
This approach produces more compact bounds and allows our analysis to stay closer to the use of HKDF in TLS~1.3, where the output of one extraction call is used to derive multiple keys.


%%% 2020-11-12
% In concurrent and independent work, Diemert and Jager~\cite{JC:DieJag20} studied the tight security of the main TLS~1.3 handshake.
% Despite the use of different security models, their bounds and ours provide similarly tight reductions to the strong Diffie--Hellman problem
% \acnsreplace{}{ with the same dominating terms for real-world parameters, as we will discuss in Section~\ref{sec:evaluation}}.
% Their analysis is carried out in the multi-stage key exchange model~\cite{CCS:FisGue14}, proving security not only of the final session key, but also of intermediate handshake encryption keys and further secrets.
% To tackle the challenge that TLS~1.3's key schedule does not bind DH values and session context in one function, they model the full cascading derivation of each intermediate key monolithically as an independent, programmable random oracle (cf.~\cite[Theorem~6]{JC:DieJag20}). 
% We instead model the key schedule's inner HKDF~\cite{C:Krawczyk10} extraction and expansion functions as two individual random oracles, carefully connected via efficient look-up tables, \acnsreplace{}{yielding a slightly less extensive use of random oracles} and compensating for the existence of shared computations in the derivation of multiple keys. .
% This approach produces more compact bounds and allows our analysis to stay closer to the use of HKDF in TLS~1.3, where the output of one extraction call is used to derive multiple keys.
% \acnsreplace{}{Beyond the strong DH problem, both their and our proofs further reduce to multi-user security of signatures and PRFs, applying random oracle bounds for the latter (cf.~\cite[Section~5]{JC:DieJag20} and our \fullelse{Section~\ref{sec:components:muPRF}}{Appendix~\ref{apx:components:muPRF}}).}
% In addition, our model captures explicit authentication \acnsreplace{}{(which we prove via HMAC's unforgeability)} and our bounds are fully parameterized enabling the evaluation of concrete practical advantages (cf.~Section~\ref{sec:evaluation}).
% Finally, while our work additionally treats the more generic \SIGMAI protocol and proves GGM bounds for the strong DH problem,
% \iffull
% Diemert and Jager~\cite{JC:DieJag20} further study composition of the TLS~1.3 handshake with the nonce-randomized AES-GCM encryption in the record protocol,
% connecting tight multi-user bounds for the latter~\cite{C:BelTac16,CCS:HoaTesThi18} with a tighter version of prior MSKE composition results~\cite{thesis:Guenther18} they establish.
% \else
% Diemert and Jager~\cite{JC:DieJag20} in turn study tight composition with the TLS record protocol.
% \fi
% \acnsreplace{}{We will further discuss and compare our technical results with those of Diemert and Jager~\cite{JC:DieJag20} in more detail throughout the paper.}


%% AC20 rebuttal
% This indeed is independent and concurrent work, published after the AC deadline. While DJ and our proofs use different security models, the results are essentially consistent. We’ll add a detailed comparison; briefly the main differences are:
%   * DJ use the DFGS [22,..] model that allows Reveal/Test queries on intermediate keys; we simplify presentation by limiting these queries to the final session keys (as you said, @R1). Our proof strategy would however easily allow to branch out the PRF/KDF-steps to show security of intermediate keys.
%   * DJ model the derivation of each intermediate key as an independent, programmable RO (cf. DJ Thm. 6). We instead model HKDF.Extract/Expand as two small, individual ROs (carefully connected via look-up tables), yielding more compact and slightly tighter bounds. This better captures the use of HKDF in TLS 1.3, where the output of one Extract call is used to derive multiple keys. 
%   * DJ prove tight multi-user PRF bounds for HMAC/HKDF in the ROM. We likewise apply ROM mu-PRF bounds (cf. Apx B.2), but additionally prove explicit authentication via HMAC’s EUF-CMA security.
%   * We give fully parameterized bounds and concrete practical advantages, and prove a GGM bound for StrongDH.
%   * We also analyze the more generic SIGMA(-I) case.


%% AC20 older, longer version
% This indeed is independent and concurrent work, published after the AC deadline. DJ and our bounds are essentially consistent: Our bound contains the same tight bounds for symmetric primitives (HMAC/HKDF), it gains from a more fine-grained KDF modeling and includes a MAC term for explicit authentication. We’ll add a detailed comparison, the main differences are:
% 
% * DJ:
%   - show security of intermediate keys via the multi-stage KE model (like prior work by DFGS [22,..]); we focus on the main keys and only conjecture intermediate keys’ security (making the models different--correct @R1).
%   - model compound steps for each key's derivation, yielding a higher-level KDF with inputs directly bound to final key derivation, at the cost of 4+1 ROs and several StrongDH proof steps (cf. DJ Thm. 6)
%   - give slightly tighter MSKE composition and connect to mu-AEAD bounds
% 
% * We:
%   - model only HKDF Extract/Expand as two small, individual ROs (carefully connected via look-up tables), yielding more compact and slightly tighter bounds (1 StrongDH step, 2 ROs); the same approach can be applied for multi-stage KE
%   - also analyze the more generic SIGMA(-I) case
%   - show explicit authentication
%   - give fully parameterized bounds and concrete practical advantages
%   - prove a GGM bound for StrongDH, enabling comparison between assumptions    




\iffalse
\newpage
\section*{\color{Red}Old/Draft Introduction}





Authenticated key exchange (AKE), allowing two parties to establish a shared secret over an insecure communication channel,
is one of the most widely used cryptographic components in today's world,
securing billions of Internet connections every day. \hd{Mihir's comment: Needs more specificity. Exactly which protocols are widely used? What does widely mean?}
\hd{The TLS protocol for secure Internet communication is used by x\% of the top Y00K Alexa sites. Its handshake protocol is an authenticated key exchange (AKE), which allows two parties to establish a shared secret over an insecure communication channel.
AKE is also a crucial component of other major Internet security protocols like IKE and the Signal messaging protocol. }
 
\hd{Here is my super rough intro outline, focusing more on what the problem we solve is and nuuumbers. All notation is absurd shorthand and not remotely final.}
	
\hd{	One thing we desire of an authenticated key exchange protocol KE is provable security. What does this actually mean? It means a theorem gives an upper bound for the advantage of an adversary attacking KE. This upper bound usually depends on the hardness of a well-known problem, such as the Decisional Diffie--Hellman problem or the discrete logarithm problem. Intuitively, if the discrete log problem is hard, then an adversary should not be able to break KE. }

\hd{Of course, the hardness of any problem is dependent on its size and the resources of the adversary. Discrete logarithms are easy to compute in small-order groups, and any AKE scheme should be easy to break if its key length is one bit. Bounds on an adversary's advantage therefore depend on the runtime of the adversary and the size of its attack surface. The latter is measured by the number of queries the adversary makes in a security game; each query represents some interaction of the adversary with its environment. }

\hd{A typical bound has the form Adv(KE)(t,queries) $\leq$ F(queries, params) + G(queries,params)*Adv(problem)(params,T(params,queries,t),queries). Here, F is a negligible function, and G and T are polylog functions. If problem is hard, then we assume that Adv(problem)(params, t,queries) is negligible. From an asymptotic perspective, Adv(KE)(t,queries) is negligible, which is good enough. In a concrete setting where we pick values for t, queries, and params, the story is more complicated. }
\fg{Maybe one sample bound, with just one subproblem. Use DH Problem with ballpark numbers.}

\hd{We choose t and queries based on realistic assumptions about the computational resources of a potential adversary. We also assume the hardness of problem based on similar computational resources. If T(params, queries, t) is large, it may exceed reasonable assumptions about computational resources. In this case, Adv(problem)(params, T(params,queries,t) may be high. Similarly, if G(queries, params) is large, the bound on KE may not prevent viable adversaries even if Adv(problem) is small.  Of course, since Adv(problem) is negligible, we can always scale up the parameters until the upper bound on the adversary is sufficiently small. In practice, this causes efficiency concerns and does not happen.}

\hd{A tight security reduction to problem is one for which G and T are small polynomials. In order to use a security bound to exclude realistic attacks, we need two things: first, we need problem to be hard for realistic adversaries. Second, we need the reduction to problem to be tight. }

\hd{TLS 1.3 does not have a tight security proof. For realistic parameters and standardized groups, existing bounds do not provide the target level of security. Numbersnumbersnumbersnumbers. }

\hd{Some existing tight AKE reductions to hard problems already exist, but they target implicit authentication. Many major Internet security protocols like TLS and IKE target explicit authentication because why? They accomplish this by following the design of the SIGMA protocol.}

\hd{We extend Cohn-Gordon's techniques to the explicitly authenticated AKE protocols SIGMA and TLS1.3. We give tight reductions to the stDH assumption, which we prove is as hard as discrete log in the generic group model. Instantiating our bounds shows that with standardized parameters, we achieve target security levels even considering a globally-scaled attacker. Now on to contributions subsubsection}

At the heart of most AKE protocols \hd{Mihir's comment: Most in what pool of protocols?}%in most cases 
is a Diffie--Hellman-style key exchange~\cite{DifHel76}, leveraging the versatility, efficiency, and security of this elegant primitive. \hd{Mihir's comment: Say DH is used because of forward secrecy instead of praising it.}

To ensure (explicit) authentication, many practical key exchange designs including major Internet security protocols like TLS~\cite{rfc5246,rfc8446} and the IKE protocol~\cite{rfc2409} of IPsec~\cite{rfc2401} follow the \SIGMA (``SIGn-and-MAc'') key exchange design put forward by Krawczyk~\cite{C:Krawczyk03} which augments Diffie--Hellman (DH) key exchange with authenticating signatures and MACs.

When it comes to determining which concrete security parameters for the protocol building blocks to use when deploying key exchange protocols, parameters should ideally be both theoretically sound (in the sense of providing meaningful security guarantees based on the proof) as well as reasonably efficient. \hd{Mihir's comment: the scheme, not the parameters, should be sound an efficient; and this needs to be more specific.}
\hd{My intepretation: numbers would be helpful here: we want to choose parameters so that the scheme can run in x time and have a guaranteed security level of y. This naturally leads into the tradeoff of non-tight reductions.}
One would hope that reductionist security proofs provide sound guidelines for deriving such parameters. 
Unfortunately, the proof techniques applied in key exchange security results most often suffer from (highly) non-tight security reductions.
The available reductions incur security losses from the protocol primitives' security that are linear, or sometimes even quadratic, in the number of protocol sessions considered and parties being involved. % due to guessing steps performed in the reduction.
For example, the security proof for \SIGMA~\cite{C:CanKra02} has a linear loss in the number of sessions, and proofs of the (conceptually more complex) TLS protocol in version~1.2~\cite{C:JKSS12,C:KraPatWee13,C:BFKPSZ14} as well as the newest version~1.3~\cite{CCS:DFGS15,EPRINT:DFGS16} incur a quadratic loss in the number of sessions. \hd{Mihir's comment: There is always a tight reduction from SOME assumption; so talking about the loss without this context isn't clear.}
Considering that Google alone securely serves several billion search requests every day\footnote{\url{https://www.internetlivestats.com}, retrieved 2020-02-10},
the number of key exchange sessions in major Internet security protocols may well be on the order of $2^{50}$ across the Internet over a longer period of time. \hd{Mihir's comment: how many searches are done per session key? What is the source of the $2^{50}$ number?}
Taking such numbers into account, theoretically sound parameters for key exchange components would need to provision for about a $100$-bit security loss. \hd{Mihir's comment: What is a loss and how do you measure it in bits?}
This leaves deployed protocols like~TLS in the unfortunate situation that the gap between theoretically sound parameters and those actually deployed is too big for concrete security bounds to be at all meaningful for the real-world deployment.
\hd{Mihir's comment: What is a theoretically sound parameter? Those actually deployed: we don't deploy parameters. We deploy schemes. What is a bound, and what does it mean for it to be meaningful?}
\hd{My interpretation: I think we should focus more on the environment of our paper: A scheme is proven secure, meaning it has a parametrized reduction bounding the advantage of an adversary by the hardness of a well-known problem. We want to pick large parameters that make the advantage prohibitively low. We also want to pick small parameters that make the scheme fast. If we pick parameters that are too small, the parametrized equation may allow adversaries that have high advantage. }
The security losses seen in security proofs of many key exchange protocols has led to explorations of protocol designs with \emph{tight} security proofs.
Results in this direction include the works by Bader et al.~\cite{TCC:BHJKL15} as well as Gj\o{}steen and Jager~\cite{C:GjoJag18} which achieve fully tight security (i.e., with very small security loss in the parameters),
but at the cost of requiring more complex primitives which, in the end, eat up the gained efficiency (even compared to standard key exchange protocols instantiated with parameters accounting for the non-tight losses).
More recently, Cohn-Gordon et al.~\cite{C:CCGJJ19,EPRINT:CCGJJ19} managed to achieve a reasonable trade-off between tightness and efficiency,
putting forward a nifty proof strategy for a simple and efficient, implicitly authenticated key exchange protocol.
Their main protocol (called~$\Pi$) uses a simple ephemeral Diffie--Hellman exchange, combining both ephemeral and---for authentication---static DH shares in a random-oracle--based key derivation, which contains sufficient context information to enable an elegant proof that is tight in the number of sessions and only loses a factor of the number of parties involved.
In practice, the number of parties in a key exchange is clearly much smaller than the number of sessions, meaning this approach provides a practical trade-off for reasonable efficiency based on theoretically sound parameters.
Indeed Cohn-Gordon et al.\ show that a loss in the number of parties is optimal for a certain class of protocols and underlying assumptions.

The protocols for which Cohn-Gordon et al.~\cite{C:CCGJJ19} establish their tighter security results provide implicit authentication, and their proof strategy heavily relies on this aspect when programming the random oracle.
While implicit authentication has recently seen adoption in popular new protocols like Signal~\cite{Signal} or Noise~\cite{Noise},
key exchange protocols in many other Internet security protocols aim at explicit authentication~\cite{C:BelRog93}, guaranteeing presence of a communication partner upon protocol acceptance.
While Cohn-Gordon et al.\ show that explicit authentication can be generically added to their implicitly authenticated key exchange protocols through a follow-up compiler step, this unfortunately reduces efficiency and means the result does not apply to deployed real-world protocols following a more direct path to explicit authentication.
Additionally, the tighter protocol designs rely on long-term Diffie--Hellman keys for implicit authentication, which are challenging to deploy in practice due to lacking support for according certificates in the web public-key infrastructure, which already barred adoption of the DH-based OPTLS design~\cite{EuroSP:KraWee16} in the recent TLS~1.3 standardization.
This leads to the question:
\begin{center}
	\emph{Can we achieve similarly tighter security proofs for deployed key exchange protocols,
	aiming at explicitly authenticated Diffie--Hellman key exchange?}
\end{center}


\subsubsection*{Contributions.}
In this work, we answer that question positively, providing the first tight (in the number of sessions) security proof for SIGMA-style (explicitly) authenticated key exchange protocols, covering both the basic protocol variant as well as \SIGMAI with added privacy for parties' identities.
We furthermore translate our results to the \SIGMA-based TLS~1.3 key exchange design, overcoming technical hurdles introduced by that protocol's significantly higher complexity.


\paragraph{Code-based security model and proofs.}
For our proofs, we provide detailed proof steps and reductions using the code-based game-playing framework of Bellare and Rogaway~\cite{EC:BelRog06}.
Our security model is similar to the one applied by Cohn-Gordon et al.~\cite{C:CCGJJ19},
%considering in particular compromises of long-term secrets and session keys (but not internal state or randomness),
but formalized also as a code-based game (in Section~\ref{sec:ake-model}) and stronger in that it captures explicit authentication and regular (``perfect'') forward secrecy (instead of only weak forward secrecy in~\cite{C:CCGJJ19}).


\paragraph{Tighter security proof of SIGMA(-I).}
In terms of tightness, our security proof of SIGMA(-I) in Sections~\ref{sec:sigma} and~\ref{sec:sigma-proof} provides a tight reduction to the strong Diffie--Hellman assumption~\cite{RSA:AbdBelRog01} in the used DH group, and to multi-user (mu) security definitions of the employed pseudorandom function (PRF), signature scheme, and MAC scheme.
Notably, while all these assumptions are in principle stronger than those used in the original proof for \SIGMA~\cite{C:CanKra02} (namely, the decisional Diffie--Hellman (DDH) assumption and single-user PRF, signature, and MAC security),
overall, we still gain in terms of efficiency when instantiating the protocol with theoretically sound parameters.
Most importantly, while being an interactive assumption compared to non-interactive DDH, no better algorithm for solving the strong DH problem (or generically the gap DH problem~\cite{PKC:OkaPoi01}) is known than to actually solve the computational DH (CDH) problem.
\fg{Add that in AGM, stDH reduces to Dlog (though with a non-tight \#queries factor)~\cite{C:FucKilLos18}.}
\hd{Verify Mihir's GGM conjecture that stDH is as hard as CDH and include}
One would hence in practice instantiate both DDH and strong DH assumptions with the same groups.
\SIGMA being a generic design, the PRF and MAC scheme can be instantiated with efficient mu-secure primitives (like, e.g., AMAC~\cite{EC:BelBerTes16}).
For any signature scheme, mu-security furthermore reduces to regular single-user (su) security with a factor of the number of users---which in our setting corresponds to the number of parties running the key exchange protocol.
This means that, here, we achieve the same level of tightness obtained by Cohn-Gordon et al.~\cite{C:CCGJJ19},
with only a loss in the number of parties, but not in the number of sessions.
Our results can be seen as confirming their insights (and translating them to the explicit authentication setting),
in that protocols binding enough context to their DH secrets in a (programmable) random-oracle--based key derivation
can achieve tight(er) security bounds in an appropriate security model.

\paragraph{Tighter security proof for the TLS~1.3 DH handshake.}
We exemplify the impact of extending the techniques from~\cite{C:CCGJJ19} to \SIGMA-style explicitly authenticated key exchange protocols
by translating our \SIGMAI result to the recently standardized newest version of the Transport Layer Security protocol, TLS~1.3~\cite{rfc8446} in Sections~\ref{sec:tls} and~\ref{sec:tls-proof}.
So far, the only reductionist security proofs known for the TLS~1.3 key exchange (the so-called handshake protocol) incur a highly non-tight security bound losing a quadratic factor in the number of sessions~\cite{CCS:DFGS15,EPRINT:DFGS16,EuroSP:FisGue17}.
While TLS~1.3 at its core follows the \SIGMAI design, its key schedule in particular is substantially more complicated, preventing the direct application of our results on \SIGMAI.
We are however able to give a carefully adapted proof which accommodates the more complex key schedule and other core variations in TLS~1.3's key exchange, achieving conceptually the same tightness results as for \SIGMAI.
Since TLS~1.3, in contrast to \SIGMA, fixes a specific set of components it deploys (esp.\ HMAC~\cite{C:BelCanKra96} as the KDF and MAC building block, for which we are not aware of a tight mu-security result),
our results do not reach the same tightness level possible for \SIGMA (instantiated with optimized components like~AMAC).
Nevertheless, our analysis still substantially improves over previous ones, as it incurs only a (linear) loss in the number of sessions for HMAC's EUF-CMA security, while the reduction to the (strong) Diffie--Hellman assumption is tight.
This way, our result provides a more theoretically sound confirmation of the practical scheme parameters deployed in TLS~1.3.


\subsubsection*{Discussion, limitations, and possible extensions.}

Like Cohn-Gordon et al.~\cite{C:CCGJJ19}, our key exchange security model considers exposure of long-term secrets and session keys, but does not allow revealing internal session state or randomness (as in the (e)CK model~\cite{EC:CanKra01,PROVSEC:LaMLauMit07}).
This is appropriate for protocols like TLS~1.3 not aiming at such levels of security.
The original \SIGMA proof~\cite{C:CanKra02} did establish security in the CK model~\cite{EC:CanKra01} allowing exposure of session state; in that sense our results are qualitatively weaker.
It is however unclear how a tight reduction for many challenged sessions down to a single DH problem instance could be obtained that at the same time allows to adaptively reveal internal session state (including DH exponents).

Our proof technique crucially relies on the ability to observe and program a random oracle that used for key derivation in the AKE protocol, borrowing from~\cite{C:CCGJJ19}.
Despite the random oracle model~\cite{CCS:BelRog93} having established itself as a versatile tool to reason about practical security, this is a noteworthy qualitative difference compared to the original \SIGMA proof~\cite{C:CanKra02} carried out in the standard model.
Interestingly, this distinction vanishes in comparison to the provable security results for the TLS~1.3 handshake protocol~\cite{CCS:DFGS15,EPRINT:DFGS16,EuroSP:FisGue17} that employ the PRF-ODH assumption~\cite{C:JKSS12,C:BFGJ17},
an interactive assumption which plausibly can only be instantiated in the random oracle model (from the strong DH assumption).

One reason for previous TLS~1.3 analyses requiring the PRF-ODH assumption is that they establish TLS~1.3's security in an multi-stage key exchange (MSKE) model~\cite{CCS:FisGue14}, proving security not only of the final session key, but also of intermediate handshake encryption keys and further secrets.
This enables, e.g., a clearer argument about the enhanced privacy obtained by encryption part of the key exchange in the style of~\SIGMAI.
While our proofs (for both \SIGMAI and TLS~1.3) establish security of the intermediate (handshake) encryption key, too,
we do not treat those keys as first-class keys available to the adversary (e.g., through revealing them) as in a multi-stage model.
We expect that our techniques similarly apply to a MSKE treatment, leaving this extension to possible future work.

%%%
%%% integrated the following already
%%%
% \begin{itemize}
% 	\item KE most widely deployed crypto ``primitive'' in the real world
% 	\hd{Really? I would have expected AEAD or digital signatures or something like that. If you can support it, this is a really good opener if only because it's surprising.}
% 	\fg{What I was aiming at was ``one of the most'', which is easier to write and argue.}
% 	\item often DH-based, for forward secrecy and efficiency
% 	\item many variants, but major Internet protocols (TLS, IKE ..) following SIGMA-style approach of explicit authentication via signatures and MACs
% 	\item proof techniques for KE often suffer from (highly) non-tight security reduction, including both number of sessions and number of users,
% 	e.g. TLS~1.2 quadratic \cite{C:JKSS12,C:KraPatWee13,C:BFKPSZ14}, TLS~1.3 quadratic \cite{CCS:DFGS15,EPRINT:DFGS16} (what about OPTLS? \fg{they don't say in their proof, but at least have to guess one session}), \cite{C:CanKra02} \SIGMA proof has \#sessions loss (not quadratic)
% 	\item while numbers of users is somewhat managable, number of sessions over a reasonable time span easily reaches orders of magnitutes that substantially affect security bounds of practically deployed protocols
% 	\item --- give some concrete numbers for the bounds
% 	\item this all rather seems to be artifacts of proof techniques, and the bounds are not met by any practical attacks or cryptanalysis
% 	\item indeed, C:19 work (and prior) \cite{C:GjoJag18,C:CCGJJ19} managed to overcome non-tight security results, esp. C:19~\cite{C:CCGJJ19} putting forward a nifty strategy for a simple and efficient, implicitly authenticated KE protocol
% 	\item their strategy heavily relies on RO programming and authentication being implicit; they provide explicit authentication only through a follow-up compiler step, which unfortunately reduces efficiency and means it does not apply to many real-world protocols
% 	\item in particular, it relies on long-term DH keys, which are hard to get certificates on in practice (cite something from the TLS standardization?) which was already seen in TLS standardization, where a signature-less OPTLS design~\cite{EuroSP:KraWee16} was discarded for deployment reasons
% 	
% 	
% 	\medskip
% 	
% 	\item in this work we ask: can we achieve similar tightness improvements for deployed DH designs, explicitly those doing DH with explicit, signature-based authentication?
% 	\item we answer positively, providing the first tight (in the number of sessions) security proof for SIGMA-style AKE protocols in a BR-like security model, covering both basic and \SIGMAI variant with identity hiding
% 	\item core modeling insights: all-real-or-random model, multiple test queries, in protocols important to bind session identifiers together with DH shares in (programmable) random oracle computation -- while given for SIGMA, will see how this works out for the more complex TLS~1.3 key schedule
% 	\item we provide a detailed proof using the code-based game-playing framework of Bellare and Rogaway~\cite{EC:BelRog06}
% 	\item we exemplify the impact of extending C:19~\cite{C:CCGJJ19} technique to SIGMA-style protocols by translating our results to the novel TLS~1.3 protocol, for which computational security results so far incured a highly-non-tight, quadratic security bound (DFGS15,DFGS16 -- what about OPTLS, do they give a concrete bound, do they maybe get away with only guessing one side?)
% 	\item we do not get to the same tightness level possible for SIGMA (based on components chosen for tight efficiency like AMAC), as TLS~1.3 uses HMAC for which we don't know (???) a tight mu-security result,
% 	\item but we still improve over previous bounds, esp. because the \#session loss is only incurred for HMAC EUF-CMA security, while the reduction to strong-DH is tight
% 	
% 	\medskip
% 	
% 	\item Limitations and future work:
% 	\begin{itemize}
% 		\item Do a BR-like model, not CK; in particular we don't (and do not know how to) treat exposure of state in our proofs.
% 		\item Only consider single session key derived, which is a simplification of TLS~1.3 doing a MSKE~\cite{CCS:FisGue14} which shines through already in the \SIGMAI idenity-hiding variant. We expect our results can be extended to MSKE setting.
% 		\item \hd{We use the identity-hiding variant but don't talk about identity-hiding security or specify any requirement for encryption security.}
% 		\item HMAC mu-security unclear (???), so variants of TLS deploying mu-optimized PRFs/MACs could be envisioned.
% 		\item As C:19~\cite{C:CCGJJ19}, our proof crucially relies on the RO technique~\cite{CCS:BelRog93}.
% 		Similarly OPTLS, and while DFGS15,DFGS16 apply the PRF-ODH assumption, there is strong indication that this assumption can only be instantiated via a random oracle~\cite{C:BFGJ17}.
% 	\end{itemize}
% \end{itemize}
% 
% \hd{This seems like a very strong storyline to me. The only thing I'd recommend adding is that there is an implicit tradeoff here: in order to get tighter security bounds, we're relying on a stronger, less standard DH assumption. It's reasonable to think that an adversary would have a higher advantage against strongDH than against plain DH. What's our argument that this tradeoff is actually beneficial? If I had to give such an argument, I'd say that this assumption is weaker than GapDH, which is a fairly well-studied problem that is conjectured to be hard. For this reason, we think that an adversary's strongDH advantage would still be very small for reasonable bounds, and the \# sessions factor has a greater impact. If we make this point, it probably becomes even more important to put StrongDH in the prelims.}
% 
% \fg{Good point, we should discuss the previous security results for both SIGMA and TLS~1.3 (in terms of tightness and assumptions).}

\newpage
\fi

\section{The TLS~1.3 Pre-shared Key Handshake Protocol}
\label{sec:tls13-psk-protocol}


\paragraph{Overview.}
We consider the pre-shared key mode of TLS~1.3, used in a setting where both client and server already share a common secret, a so-called \emph{pre-shared key} (PSK).
A PSK is a cryptographic key which may either be manually configured, negotiated out-of-band, or (and most commonly) be obtained from a prior and possibly not PSK-based TLS session to enable fast \emph{session resumption}.
The TLS~1.3 PSK handshake comes in two flavors:
PSK-only, where security is established from the pre-shared key alone,
and PSK-(EC)DHE, which includes an (finite-field or \underline{e}lliptic-\underline{c}urve) Diffie--Hellman key exchange for added forward secrecy.
Both PSK handshakes essentially consist of two phases (cf.\ Figure~\ref{fig:tls-handshake}). 

\begin{enumerate}
	\item The client sends a random nonce and a list of offered pre-shared keys to the server, where each key is identified by a (unique) identifier~$\pskid$.%
	\footnote{%
	In this work, we do not consider negotiation of pre-shared keys in situations where client and server share multiple keys, but focus on the case where client and server share only one PSK and the client therefore offers only a single $\pskid$.
	However, we expect that our results extend to the general case as well.
	}
	The server then selects one $\pskid$ from the list, and responds with another random nonce and the selected $\pskid$.
	In PSK-(EC)DHE mode, client and server additionally perform a Diffie--Hellman key exchange, sending group elements along with the nonces and PSK identifiers. 
	%
	In both modes, the client also sends a so-called binder value,
	which applies a \textit{message authentication code} (MAC) to the client's nonce and $\pskid$ (and the Diffie--Hellman share in PSK-(EC)DHE mode)
	and binds the PSK handshake to the (potential) prior handshake in which the used pre-shared key was established (see~\cite{SP:CHSv16,CCS:Krawczyk16} for analysis rationale behind the binder value).

	\item Then client and server derive \emph{unauthenticated} cryptographic keys from the PSK and the established Diffie--Hellman key (the latter only in (EC)DHE mode, of course).
	This includes, for instance, the \emph{client} and \emph{server handshake traffic keys} ($\chtk$ and $\shtk$) used to encrypt the subsequent handshake messages, as well as \emph{finished keys} ($\cfk$ and $\sfk$) used to compute and exchange \emph{finished messages}.
	The finished messages are MAC tags over all previous messages, ensuring that client and server have received all previous messages exactly as they were sent.

	After verifying the finished messages, client and server ``accept'' \emph{authenticated} cryptographic keys, including the \emph{client} and \emph{server application traffic secret} ($\cats$ and $\sats$), the \emph{exporter master secret} ($\ems$), and the \emph{resumption master secret} ($\rms$) for future session resumptions.
\end{enumerate}




\begin{figure}[tp]
	\centering
	
	\scalebox{0.7}{%
\framebox{%
\begin{minipage}[t]{10.5cm}
\begin{tikzpicture}

	% Set the X coordinates of the client, server, and arrows
	\edef\ClientX{0}
	\edef\ArrowLeft{0}

		\edef\ArrowRight{9}
		\edef\ServerX{9}
	% Set the starting Y coordinate
	\edef\Y{0}

	% Draw header boxes
	\node [rectangle,draw,inner sep=5pt,right] at (\ClientX,\Y) {\textbf{Client}};
	\node [rectangle,draw,inner sep=5pt,left] at (\ServerX,\Y) {\textbf{Server}};

	\NextLine[1.5]
	
	\ClientAction{$\clientNonce \sample \bits^{\nl}$}
	\NextLine
	\ClientAction{$\clientExponent \sample \Z_p$, $\clientKeyShare \assign g^\clientExponent$}
	\NextLine
	\ClientAction{\TLSmsg{$\ClientHello$}: $\clientNonce$}
	\NextLine
	\ClientAction{\PSKECDHEonly{+~$\ClientKeyShare$: $\clientKeyShare$}}
	\NextLine
	
	\SharedAction{$\es \assign \abstractExtract(0, \psk)$}
	\NextLine
	\SharedAction{$\des \assign \abstractExpand(\es, \labelDerived \concat \Hash(\emptymessage))$}
	\NextLine
	\SharedAction{$\bk \assign \abstractExpand(\es, \labelExtBinder \concat \abstractHash(\emptymessage))$}
	\NextLine
	\SharedAction{$\bfk \assign \abstractExpand(\bk, \labelFinished)$}
	\NextLine
	
	\ClientAction{$\binder \assign \abstractMAC(\bfk, \abstractHash(\CH^{-}))$}
	\NextLine
	\ClientAction{\TLSmsg{+~$\ClientPreSharedKey$}: $\pskid, \binder$}
	
	%%%%%%%%%%%%%%%%%%%%%%%%%%%%%%%%%%%%%%%%%%%%%%%%%%%%%%%%%%%%%%%%%%%%%%%%%%%%%%%%%%%%%%%%%%%%%%%%%%%%%%%
	\ClientToServer{}{}
	\NextLine
	%%%%%%%%%%%%%%%%%%%%%%%%%%%%%%%%%%%%%%%%%%%%%%%%%%%%%%%%%%%%%%%%%%%%%%%%%%%%%%%%%%%%%%%%%%%%%%%%%%%%%%%
	
	\ServerAction{\textbf{abort} if $\binder \neq \abstractMAC(\bfk, \abstractHash(\CH^{-}))$}
	\NextLine

	\AcceptStage{1}{$\ets \assign \abstractExpand(\es, \labelETS \concat \abstractHash(\CH))$}
	\NextLine[1.5]
	\AcceptStage{2}{$\eems \assign \abstractExpand(\es, \labelEEMS \concat \abstractHash(\CH))$}
	\NextLine[1.5]
	
	\ServerAction{$\serverNonce \sample \bits^{\nl}$}
	\NextLine
	\ServerAction{$\serverExponent \sample \Z_p$, $\serverKeyShare \assign g^\serverExponent$}
	\NextLine
	\ServerAction{\TLSmsg{$\ServerHello$}: $\serverNonce$}
	\NextLine
	\ServerAction{\PSKECDHEonly{+~$\ServerKeyShare$: $\serverKeyShare$}}
	\NextLine
	\ServerAction{\TLSmsg{+~$\ServerPreSharedKey$}: $\pskid$}
	
	%%%%%%%%%%%%%%%%%%%%%%%%%%%%%%%%%%%%%%%%%%%%%%%%%%%%%%%%%%%%%%%%%%%%%%%%%%%%%%%%%%%%%%%%%%%%%%%%%%%%%%%
	\ServerToClient{}{}
	\NextLine
	%%%%%%%%%%%%%%%%%%%%%%%%%%%%%%%%%%%%%%%%%%%%%%%%%%%%%%%%%%%%%%%%%%%%%%%%%%%%%%%%%%%%%%%%%%%%%%%%%%%%%%%
	
	\ClientAction{\PSKECDHEonly{$\dhe \assign Y^x$}}
	\ServerAction{\PSKECDHEonly{$\dhe \assign X^y$}}
	\SharedAction{\PSKonly{$\dhe \assign 0$}}
	\NextLine
	
	\SharedAction{$\hs \assign \abstractExtract(\des, \dhe)$}
	\NextLine
	\SharedAction{$\chts \assign \abstractExpand(\hs, \labelClientHTS \concat \abstractHash(\CH \concat \SH))$}
	\NextLine
	\SharedAction{$\shts \assign \abstractExpand(\hs, \labelServerHTS \concat \abstractHash(\CH \concat \SH))$}
	\NextLine
	\SharedAction{$\dhs \assign \abstractExpand(\hs, \labelDerived \concat \abstractHash(\emptymessage))$}
	\NextLine
	\AcceptStage{3}{$\chtk \assign \DeriveTrafficKeys(\chts)$}
	\NextLine[1.5]
	\AcceptStage{4}{$\shtk \assign \DeriveTrafficKeys(\shts)$}
	\NextLine[1.5]
	
	\ServerAction{\TLSmsg{$\{ \EncryptedExtensions \}$}}
	\NextLine
	
	\SharedAction{$\sfk \assign \abstractExpand(\shts, \labelFinished)$}
	\NextLine
	
	\ServerAction{$\sfin \assign \abstractMAC(\sfk, \abstractHash(\CH \concat \dotsb \concat \EE ))$}
	\NextLine
	\ServerAction{\TLSmsg{$\{ \ServerFinished \}$}: $\sfin$}
	
	%%%%%%%%%%%%%%%%%%%%%%%%%%%%%%%%%%%%%%%%%%%%%%%%%%%%%%%%%%%%%%%%%%%%%%%%%%%%%%%%%%%%%%%%%%%%%%%%%%%%%%%
	\ServerToClient{}{}
	\NextLine
	%%%%%%%%%%%%%%%%%%%%%%%%%%%%%%%%%%%%%%%%%%%%%%%%%%%%%%%%%%%%%%%%%%%%%%%%%%%%%%%%%%%%%%%%%%%%%%%%%%%%%%%
	
	
	\ClientAction{\textbf{abort} if $\sfin \neq \HMAC(\sfk, \abstractHash(\CH \concat \dotsb \concat \EE))$}
	\NextLine
	\SharedAction{$\ms \assign \abstractExtract(\dhs, 0)$}
	\NextLine
	\AcceptStage{5}{$\cats \assign \abstractExpand(\ms, \labelClientATS \concat \abstractHash(\CH \concat \dotsb \concat \SF))$}
	\NextLine
	\AcceptStage{6}{$\sats \assign \abstractExpand(\ms, \labelServerATS \concat \abstractHash(\CH \concat \dotsb \concat \SF))$}
	\NextLine
	\AcceptStage{7}{$\ems \assign \abstractExpand(\ms, \labelEMS \concat \abstractHash(\CH \concat \dotsb \concat \SF))$}
	\NextLine[1.5]
	\SharedAction{$\cfk \assign \abstractExpand(\chts, \labelFinished)$}
	\NextLine
	
	\ClientAction{$\cfin \assign \abstractMAC(\cfk, \abstractHash(\CH \concat \dotsb \concat \SF))$}
	\NextLine
	\ClientAction{\TLSmsg{$\{ \ClientFinished \}$}: $\cfin$}
	
	%%%%%%%%%%%%%%%%%%%%%%%%%%%%%%%%%%%%%%%%%%%%%%%%%%%%%%%%%%%%%%%%%%%%%%%%%%%%%%%%%%%%%%%%%%%%%%%%%%%%%%%
	\ClientToServer{}{}
	\NextLine
	%%%%%%%%%%%%%%%%%%%%%%%%%%%%%%%%%%%%%%%%%%%%%%%%%%%%%%%%%%%%%%%%%%%%%%%%%%%%%%%%%%%%%%%%%%%%%%%%%%%%%%%
	
	
	\ServerAction{\textbf{abort} if $\cfin \neq \abstractMAC(\cfk, \abstractHash(\CH \concat \dotsb \concat \SF))$}
	\NextLine
	\AcceptStage{8}{$\rms \assign \abstractExpand(\ms, \labelRMS \concat \abstractHash(\CH \concat \dotsb \concat \CF))$}
\end{tikzpicture}
\end{minipage}
}
%
\framebox{%
\begin{minipage}[t]{11.5cm}
\begin{tikzpicture}

	% Set the X coordinates of the client, server, and arrows
	\edef\ClientX{0}
	\edef\ArrowLeft{0}
		\edef\ArrowRight{10}
		\edef\ServerX{10}

	% Set the starting Y coordinate
	\edef\Y{0}

	% Draw header boxes
	\node [rectangle,draw,inner sep=5pt,right] at (\ClientX,\Y) {\textbf{Client}};
	\node [rectangle,draw,inner sep=5pt,left] at (\ServerX,\Y) {\textbf{Server}};

	\NextLine[1.5]
	
	\ClientAction{$\clientNonce \sample \bits^{\nl}$}
	\NextLine
	\ClientAction{$\clientExponent \sample \Z_p$, $\clientKeyShare \assign g^\clientExponent$}
	\NextLine
	\ClientAction{\TLSmsg{$\ClientHello$}: $\clientNonce$}
	\NextLine
	\ClientAction{\PSKECDHEonly{+~$\ClientKeyShare$: $\clientKeyShare$}}
	\NextLine
	
% 	\SharedAction{\old{$\es \assign \HKDFExtr(0, \psk)$}}
% 	\NextLine
% 	\SharedAction{\old{$\des \assign \HKDFExpnd(\es, \labelDerived \concat \Thash(\emptymessage))$}}
% 	\NextLine
% 	\SharedAction{\old{$\bk \assign \HKDFExpnd(\es, \labelExtBinder \concat \Thash(\emptymessage))$}}
% 	\NextLine
% 	\SharedAction{\old{\TODO{Missing before!} $\bfk \assign \HKDFExpnd(\bk, \labelFinished)$}}
% 	\NextLine
	\NextLine[4]
	
	\ClientAction{$\binder \assign \TLSKDF_{\binder}(\psk, \abstractHash(\CH^{-}))$}
	\NextLine
	\ClientAction{\TLSmsg{+~$\ClientPreSharedKey$}: $\pskid, \binder$}
	
	%%%%%%%%%%%%%%%%%%%%%%%%%%%%%%%%%%%%%%%%%%%%%%%%%%%%%%%%%%%%%%%%%%%%%%%%%%%%%%%%%%%%%%%%%%%%%%%%%%%%%%%
	\ClientToServer{}{}
	\NextLine
	%%%%%%%%%%%%%%%%%%%%%%%%%%%%%%%%%%%%%%%%%%%%%%%%%%%%%%%%%%%%%%%%%%%%%%%%%%%%%%%%%%%%%%%%%%%%%%%%%%%%%%%
	
	\ServerAction{\textbf{abort} if $\binder \neq \TLSKDF_{\binder}(\psk, \abstractHash(\CH^{-}))$}
	\NextLine

	\AcceptStage{1}{$\ets \assign \TLSKDF_{\ets}(\psk, \abstractHash(\CH))$}
	\NextLine[1.5]
	\AcceptStage{2}{$\eems \assign \TLSKDF_{\eems}(\psk, \abstractHash(\CH))$}
	\NextLine[1.5]
	
	\ServerAction{$\serverNonce \sample \bits^{\nl}$}
	\NextLine
	\ServerAction{$\serverExponent \sample \Z_p$, $\serverKeyShare \assign g^\serverExponent$}
	\NextLine
	\ServerAction{\TLSmsg{$\ServerHello$}: $\serverNonce$}
	\NextLine
	\ServerAction{\PSKECDHEonly{+~$\ServerKeyShare$: $\serverKeyShare$}}
	\NextLine
	\ServerAction{\TLSmsg{+~$\ServerPreSharedKey$}: $\pskid$}
	
	%%%%%%%%%%%%%%%%%%%%%%%%%%%%%%%%%%%%%%%%%%%%%%%%%%%%%%%%%%%%%%%%%%%%%%%%%%%%%%%%%%%%%%%%%%%%%%%%%%%%%%%
	\ServerToClient{}{}
	\NextLine
	%%%%%%%%%%%%%%%%%%%%%%%%%%%%%%%%%%%%%%%%%%%%%%%%%%%%%%%%%%%%%%%%%%%%%%%%%%%%%%%%%%%%%%%%%%%%%%%%%%%%%%%
	
	\ClientAction{\PSKECDHEonly{$\dhe \assign Y^x$}}
	\ServerAction{\PSKECDHEonly{$\dhe \assign X^y$}}
	\SharedAction{\PSKonly{$\dhe \assign 0$}}
	\NextLine
	
% 	\SharedAction{\old{$\hs \assign \HKDFExtr(\des, \dhe)$}}
% 	\NextLine
% 	\SharedAction{\old{$\chts \assign \HKDFExpnd(\hs, \labelClientHTS \concat \Thash(\CH \concat \SH))$}}
% 	\NextLine
% 	\SharedAction{\old{$\shts \assign \HKDFExpnd(\hs, \labelServerHTS \concat \Thash(\CH \concat \SH))$}}
% 	\NextLine
% 	\SharedAction{\old{$\dhs \assign \HKDFExpnd(\hs, \labelDerived \concat \Thash(\emptymessage))$}}
% 	\NextLine
	\NextLine[4]
	
	\AcceptStage{3}{$\chtk \assign \TLSKDF_{\chtk}(\psk, \dhe, \abstractHash(\CH \concat \SH))$}
	\NextLine[1.5]
	\AcceptStage{4}{$\shtk \assign \TLSKDF_{\shtk}(\psk, \dhe, \abstractHash(\CH \concat \SH))$}
	\NextLine[1.5]
	
	\ServerAction{\TLSmsg{$\{ \EncryptedExtensions \}$}}
	\NextLine
	
% 	\SharedAction{\old{$\sfk \assign \HKDFExpnd(\shts, \labelFinished)$}}
% 	\NextLine
	\NextLine[1]
	
	\ServerAction{$\sfin \assign \TLSKDF_{\sfin}(\psk, \dhe, \abstractHash(\CH \concat \SH), \abstractHash(\CH \concat \dotsb \concat \EE ))$}
	\NextLine
	\ServerAction{\TLSmsg{$\{ \ServerFinished \}$}: $\sfin$}
	
	%%%%%%%%%%%%%%%%%%%%%%%%%%%%%%%%%%%%%%%%%%%%%%%%%%%%%%%%%%%%%%%%%%%%%%%%%%%%%%%%%%%%%%%%%%%%%%%%%%%%%%%
	\ServerToClient{}{}
	\NextLine
	%%%%%%%%%%%%%%%%%%%%%%%%%%%%%%%%%%%%%%%%%%%%%%%%%%%%%%%%%%%%%%%%%%%%%%%%%%%%%%%%%%%%%%%%%%%%%%%%%%%%%%%
	
	
	\ClientAction{\textbf{abort} if \scriptsize$\sfin \neq \TLSKDF_{\sfin}(\psk, \dhe, \abstractHash(\CH \concat \SH), \abstractHash(\CH \concat \dotsb \concat \EE ))$}
	\NextLine
	
% 	\SharedAction{\old{$\ms \assign \HKDFExtr(\dhs, 0)$}}
% 	\NextLine
	\NextLine[1]
	
	\AcceptStage{5}{$\cats \assign \TLSKDF_{\cats}(\psk, \dhe, \abstractHash(\CH \concat \dotsb \concat \SF))$}
	\NextLine
	\AcceptStage{6}{$\sats \assign \TLSKDF_{\sats}(\psk, \dhe, \abstractHash(\CH \concat \dotsb \concat \SF))$}
	\NextLine
	\AcceptStage{7}{$\ems \assign \TLSKDF_{\ems}(\psk, \dhe, \abstractHash(\CH \concat \dotsb \concat \SF))$}
	\NextLine[1.5]
	
% 	\SharedAction{\old{$\cfk \assign \HKDFExpnd(\chts, \labelFinished)$}}
% 	\NextLine
	\NextLine[1]
	
	\ClientAction{$\cfin \assign \TLSKDF_{\cfin}(\psk, \dhe, \abstractHash(\CH \concat \SH), \abstractHash(\CH \concat \dotsb \concat \SF))$}
	\NextLine
	\ClientAction{\TLSmsg{$\{ \ClientFinished \}$}: $\cfin$}
	
	%%%%%%%%%%%%%%%%%%%%%%%%%%%%%%%%%%%%%%%%%%%%%%%%%%%%%%%%%%%%%%%%%%%%%%%%%%%%%%%%%%%%%%%%%%%%%%%%%%%%%%%
	\ClientToServer{}{}
	\NextLine
	%%%%%%%%%%%%%%%%%%%%%%%%%%%%%%%%%%%%%%%%%%%%%%%%%%%%%%%%%%%%%%%%%%%%%%%%%%%%%%%%%%%%%%%%%%%%%%%%%%%%%%%
	
	
	\ServerAction{\textbf{abort} if \scriptsize$\cfin \neq \TLSKDF_{\cfin}(\psk, \dhe, \abstractHash(\CH \concat \SH), \abstractHash(\CH \concat \dotsb \concat \SF ))$}
	\NextLine
	\AcceptStage{8}{$\rms \assign \TLSKDF_{\rms}(\psk, \dhe, \abstractHash(\CH \concat \dotsb \concat \CF))$}
\end{tikzpicture}
\end{minipage}
}
}

	
	%
	% Legend
	%
	\begin{minipage}{0.95\textwidth}
	\vspace{0.25cm}
	\scriptsize
	\textbf{Legend} \\
	\begin{tabular}{ll}
		\TLSmsg{$\mathtt{MSG}$}:~$Y$	& message $\mathtt{MSG}$ sent, containing $Y$ \\
		\TLSmsg{+~$\mathtt{MSG}$}	& extension sent within previous message \\
		\TLSmsg{$\{\mathtt{MSG}\}$}	& $\mathtt{MSG}$ sent AEAD-encrypted with~$\chtk$/$\shtk$ \\
		\PSKECDHEonly{\dots}		& present only in PSK-(EC)DHE \\
		\PSKonly{\dots}			& present only in PSK \\
	\end{tabular}
	\begin{tabular}{ll}
		$\CH^{-}$			& partial $\ClientHello$ up to (incl.) $\pskid$ \\
		$\shortLabelCmd{x}$		& label value, distinct for distinct~$x$ \\
		&\\
		&\\
		&\\
	\end{tabular}
	\begin{tabular}{l}
		$\DeriveTrafficKeys(\mathit{HTS}) := \abstractExpand(\mathit{HTS}, \labelk \concat \Thash(\emptymessage), \hashlen) \concat \abstractExpand(\mathit{HTS}, \labeliv \concat \Thash(\emptymessage), \ivlen)$\\
		\quad(traffic key computation, deriving a $\hashlen$-bit key and a $\ivlen$-bit IV)
	\end{tabular}
	\end{minipage}
	
	\caption{%
		TLS~1.3 PSK and PSK-(EC)DHE handshake modes with (optional) 0-RTT keys (stages~1 and~2),
		with detailed key schedule (left) and our representation of the key schedule through functions~$\TLSKDF_{x}$ (right), explained in the text.
		Centered computations are executed by both client and server with their respective messages received, and possibly at different points in time.
		Dotted lines indicate the derivation of session (stage) keys together with their stage number.
		The labels $\shortLabelCmd{x}$ are distinct for distinct index~$x$, see
				Table~\ref{tab:labels}
		for their definition.
	}
	\label{fig:tls-handshake}
\end{figure}

	\begin{table}[htp]
	\centering
	\setlength{\tabcolsep}{4pt}
	\renewcommand{\arraystretch}{1.1}
	\begin{tabular}{@{}ll@{\qquad}ll@{}}
		\toprule
		\textbf{Value} & \textbf{Label}                                                  & \textbf{Value}                   & \textbf{Label}                        \\ \midrule
		$\des$         & $\labelDerived = \labelDerivedText$                             & $\chtk$ &    $\labelk = \labelkText$ \& $\labeliv= \labelivText$                         \\
		$\bk$          & $\labelExtBinder = \labelExtBinderText$ / $\labelResBinderText$ & $\shtk$                          & $\labelk = \labelkText$ \& $\labeliv= \labelivText$                                    \\
		$\bfk$         & $\labelFinished = \labelFinishedText$                           & $\sfk$                           & $\labelFinished = \labelFinishedText$ \\
		$\ets$         & $\labelETS = \labelETSText$                                     & $\cats$                          & $\labelClientATS = \labelClientATSText$         \\
		$\eems$        & $\labelEEMS = \labelEEMSText$                                   & $\sats$                          & $\labelServerATS = \labelServerATSText$         \\
		$\chts$        & $\labelClientHTS = \labelClientHTSText$                                   & $\ems$                           & $\labelEMS = \labelEMSText$           \\
		$\shts$        & $\labelServerHTS = \labelServerHTSText$                                   & $\cfk$                           & $\labelFinished = \labelFinishedText$ \\
		$\dhs$         & $\labelDerived = \labelDerivedText$                             & $\rms$                           & $\labelRMS = \labelRMSText$           \\ \bottomrule
	\end{tabular}
	\medskip
	\caption{%
		Definitions of the short labels used in \autoref{fig:tls-handshake}.
		We simplify the labeling of $\abstractExpand$ in our presentation.
		In the specification each $\abstractExpand$ is not only labeled by $\labelsym \concat H$ for some label $\labelsym$ and some hash $H$,
		but it is prefixed by the output length of the respecitive $\abstractExpand$ call and the constant label ``$\labelfont{tls13 }$''.
		As the output length for all of the above calls is equal (namely, the output length $\hashlen$ of $\abstractHash$), we leave this constant prefix out to reduce complexity.
		}
	\label{tab:labels}
\end{table}



\paragraph{Detailed specification.}
For our proofs we will need fully-specified descriptions for each of the TLS~1.3 PSK and PSK-(EC)DHE handshake protocols. 
Pseudocode for these protocols can be found in Figure~\ref{fig:tls-handshake}, where we let $(\G, p, g)$ be a cyclic group of prime order $p$ such that $\G = \langle g \rangle$.

The two descriptions on the left and right in Figure~\ref{fig:tls-handshake} show the same protocol, but they use different abstractions to highlight how we capture the complex way TLS~1.3 calls its hash function.
This one hash function is used in some places to condense transcripts, in others to help derive session keys, and in still others as part of a message authentication code. 
We call this function $\Hash$, and let its output length be $\hashlen$ bits so that we have $\abstractHash \colon \bits^* \to \bits^{\hashlen}$. 
Depending on the choice of ciphersuite, TLS~1.3 instantiates $\Hash$ with either $\SHA{256}$ or $\SHA{384}$ \cite{NIST:FIPS-180-4}. In our security analysis, we will model $\Hash$ as a random oracle.

On the left-hand side of Figure~\ref{fig:tls-handshake}, we distinguish four named subroutines of TLS~1.3 which use $\Hash$ for different purposes:
\begin{itemize}
%	\item Since we now only directly call a hash function on transcripts, we refer to this as the ``transcript hash''.

	\item A message authentication code $\abstractMAC  \colon \bits^{\hashlen} \times \bits^* \to \bits^{\hashlen}$, which  calls $\Hash$ via the $\HMAC$ function $\abstractMAC(K, M) := \HMAC[\Hash](K, M)$ where
\[
\HMAC[\Hash](K, M) := \Hash( (K \concat 0^{\blocklen - \hashlen}) \oplus \opad) \concat \Hash( (K \concat 0^{\blocklen - \hashlen} \oplus \ipad) \concat M )) 
\]
Here $\opad$ and $\ipad$ are $\blocklen$-bit strings,
where each byte of $\opad$ and $\ipad$ is set to the hexadecimal value \texttt{0x5c}, resp.\ \texttt{0x36}.
We have $\blocklen=512$ when $\SHA{256}$ is used and $\blocklen=512$ for $\SHA{384}$. When modeling $\SHA{256}$ resp. $\SHA{384}$ as a random oracle, we keep the corresponding value of $\blocklen$.
%HMAC assumes that $\Hash$ has a Merkle--Damg\r{a}rd structure with block length $\blocklen$;
%where $\SHA{256}$ has $\blocklen=512$ bits and $\SHA{384}$ has $\blocklen = 1024$ bits. \TODO{How do we defie HMAC if H is a RO?}
%Then functions $\abstractMAC$, $\abstractExtract$, and $\abstractExpand$, and $\abstractHash$ are defined as follows:


% 	\item $\replace{\abstractExtract}{\abstractExtract} \colon \bits^{\hashlen} \times \bits^* \to \bits^{\hashlen}$, a subroutine for \emph{extracting} key material in the key schedule. 
% 	\item $\replace{\abstractExpand}{\abstractExpand}  \colon \bits^{\hashlen} \times \bits^* \to \bits^{\hashlen}$, a subroutine for \emph{expanding} key material in the key schedule.
	\item $\abstractExtract, \abstractExpand \colon \bits^{\hashlen} \times \bits^* \to \bits^{\hashlen}$, two subroutines for \emph{extracting} and \emph{expanding} key material in the key schedule, following the HKDF key derivation paradigm of Krawczyk~\cite{C:Krawczyk10,rfc5869}. These functions are defined
\begin{itemize}
	\item $\abstractExtract(K, M) := \HKDFExtr(K,M) = \abstractMAC(K,M)$.
	
	\item $\abstractExpand(K, M) := \HKDFExpnd(K, M) = \abstractMAC(K, M \concat \texttt{0x01})$.%
	\footnote{$\HKDFExpnd$~\cite{rfc5869} is defined for any output length (given as third parameter).
	In TLS~1.3, $\abstractExpand$ always derives at most $\hashlen$ bits, which can be trimmed from a $\hashlen$-bit output; we hence in most places omit the output length parameter.}
\end{itemize} 
\end{itemize}
%We can establish the independence of these subroutines more easily by viewing them as named parameters within the TLS protocol.
Despite the new naming conventions, this abstraction closely mimics the TLS~1.3 standard:
$\abstractMAC$, $\abstractExtract$, and $\abstractExpand$ can be read as more generic ways of referring to the $\HMAC$, $\HKDFExtr$, and $\HKDFExpnd$ algorithms~\cite{rfc2104,rfc5869}. 


% \TODO{Define $\abstractMAC$ in terms of $\abstractHash$. (Cf. Theorem~\ref{thm:full-ks-indiff})}

% \begin{itemize}
% 	\item 
	

	
% 	\item $\abstractHash(M) := \Hash(M)$.
% \end{itemize}

The right-hand side of Figure~\ref{fig:tls-handshake} separates the key derivation functions for each first-class key as well as the binder and finished MAC values derived.
This way of modeling TLS~1.3 makes it easier to establish key independence for the many keys computed in the key schedule, as we will see in Section~\ref{sec:ks-indiff}.
We introduce $11$ functions $\TLSKDF_{\binder}$, $\TLSKDF_{\ets}$, $\TLSKDF_{\eems}$, $\TLSKDF_{\chtk}$, $\TLSKDF_{\cfin}$, $\TLSKDF_{\shtk}$, $\TLSKDF_{\sfin}$, $\TLSKDF_{\cats}$, $\TLSKDF_{\sats}$, $\TLSKDF_{\ems}$, and $\TLSKDF_{\rms}$ (indexed by the value they derive) and use them to abstract away many intermediate computations.
Note that we are not changing the protocol, though:
we define each $\TLSKDF$ function to capture the same steps it replaces.

Take as an example $\TLSKDF_{\sfin}$, the function used to derive the MAC in the $\ServerFinished$ message.
In the prior abstraction, a session would first use the key schedule to derive a finished key~$\sfk$ from the hashed transcript and the secrets~$\psk$ and~$\dhe$. 
It would then call $\abstractMAC$, keyed with $\sfk$, to generate the $\ServerFinished$ message authentication code on the hashed transcript and encrypted extensions.
Accordingly, we define $\TLSKDF_{\sfin} \colon \bits^{\hashlen} \times \G \times \bits^{\hashlen}\times \bits^{\hashlen} \to \bits^{\hashlen}$ as in Figure~\ref{fig:TLSKDF-sfin}.
In the protocol, $\TLSKDF_{\sfin}$ takes inputs the pre-shared key~$\psk$ and Diffie--Hellman secret~$\dhe$ and hash digests $\digest_1 = \Thash(\CH \concat \SH)$ and $\digest_2 = \Thash(\CH \concat \dotsb \concat \EE)$, and it outputs a $\MAC$ tag for the $\ServerFinished$ message.
%
The remaining key derivation functions are defined the same way; we give their signatures 
 
		below for completeness.
	{
\allowdisplaybreaks
\begin{align*}
1.~ &\TLSKDF_{\binder}[\ROhmac]	&&\colon \bits^{\hashlen} \times \bits^{\hashlen} \to \bits^{\hashlen} \\
2.~ &\TLSKDF_{\ets}[\ROhmac]	&&\colon \bits^{\hashlen} \times \bits^{\hashlen} \to \bits^{\hashlen} \\
3.~ &\TLSKDF_{\eems}[\ROhmac]	&&\colon \bits^{\hashlen} \times \bits^{\hashlen}  \to \bits^{\hashlen}\\
4.~ &\TLSKDF_{\chtk}[\ROhmac]	&&\colon \bits^{\hashlen}\times \G \times \bits^{\hashlen} \to \bits^{\hashlen + \ivlen} \\
5.~ &\TLSKDF_{\cfin}[\ROhmac]		&&\colon \bits^{\hashlen} \times \G \times \bits^{\hashlen} \times \bits^{\hashlen} \to \bits^{\hashlen} \\
6.~ &\TLSKDF_{\shtk}[\ROhmac]	&&\colon \bits^{\hashlen} \times \G \times \bits^{\hashlen} \to \bits^{\hashlen + \ivlen} \\
7.~ &\TLSKDF_{\sfin}[\ROhmac]		&&\colon \bits^{\hashlen} \times \G \times \bits^{\hashlen} \times \bits^{\hashlen} \to \bits^{\hashlen} \\
8.~ &\TLSKDF_{\cats}[\ROhmac]	&&\colon \bits^{\hashlen}\times \G \times \bits^{\hashlen} \to \bits^{\hashlen} \\
9.~ &\TLSKDF_{\sats}[\ROhmac] &&\colon \bits^{\hashlen} \times \G \times \bits^{\hashlen} \to \bits^{\hashlen} \\
10.~ &\TLSKDF_{\ems}[\ROhmac]	&&\colon \bits^{\hashlen} \times \G \times \bits^{\hashlen} \to \bits^{\hashlen} \\
11.~ &\TLSKDF_{\rms}[\ROhmac]	&&\colon \bits^{\hashlen} \times \G \times \bits^{\hashlen} \to \bits^{\hashlen}
\end{align*}
}


\begin{figure}[t]
	\centering
	\begin{minipage}{5.5cm}
	\begin{algorithm}{$\TLSKDF_{\sfin}(\psk, \dhe, \digest_1, \digest_2)$}
		\item $\es \gets \abstractExtract (0, \psk)$
		\item $\des \gets \abstractExpand(\es, \labelDerived \concat \Thash(\emptymessage))$
		\item $\hs \gets \abstractExtract(\des, \dhe)$
% 		\item $\shts \gets \abstractExpand(\hs, \labelServerHTS \concat \digest_1)$
% 		\item $\sfk \gets \abstractExpand(\shts, \labelFinished)$
% 		\item $\sfin \gets \abstractMAC(\sfk, \digest_2)$
% 		\item return $\sfin$
	\end{algorithm}
	\end{minipage}
	%
	\begin{minipage}{5.5cm}
	\begin{code}[start=4]
% 		\item $\es \gets \abstractExtract (0, \psk)$
% 		\item $\des \gets \abstractExpand(\es, \labelDerived \concat \Thash(\emptymessage))$
% 		\item $\hs \gets \abstractExtract(\des, \dhe)$
		\item $\shts \gets \abstractExpand(\hs, \labelServerHTS \concat \digest_1)$
		\item $\sfk \gets \abstractExpand(\shts, \labelFinished)$
		\item $\sfin \gets \abstractMAC(\sfk, \digest_2)$
		\item return $\sfin$
	\end{code}
	\end{minipage}
	
	\caption{Definition of $\TLSKDF_{\sfin}$, deriving the $\ServerFinished$ MAC.}
	\label{fig:TLSKDF-sfin}
\end{figure}


% 	\end{minipage}
% \caption{Pseudocode for function $\TLSKDF_{\sfin}$ to derive the $\ServerFinished$ message in the TLS~1.3 PSK and PSK-(EC)DHE (right) handshakes. (In the PSK-only handshake, $\dhe$ will always be the string $0^{\hashlen}$).}
% \label{fig:tlskdf-example}
% \end{figure}

Note that the definition of the 11 functions induces a lot of redundancy as we derive every value independently and therefore compute intermediate values (e.g., $\es$, $\des$, and $\hs$) multiple times over the execution of the handshake.
However, this is only conceptual.
Since the computations of these intermediate values are deterministic, the intermediate values will be the same for the same inputs and could be cached.




% \begin{figure}
% 	\centering
% 	
% 	\input{old-fig-tls-handshake}
% 	
% 	\caption{The TLS~1.3 PSK handshake without 0-RTT. Every TLS handshake message is denoted as ``$\msgfont{MSG} \colon C$'', where $C$ denotes the message's content. Similarly, an extension is denoted by ``$\msgfont{+ MSG} \colon C$''. Further, we denote by ``$\{ \msgfont{MSG} \} \colon C$'' messages containing $C$ and being AEAD-encrypted under the handshake traffic keys $\chtk$/$\shtk$. Messages and computations that are written in ``$[\dotsc]$'' only appear in the PSK-(EC)DHE handshake. Centered computations are executed by both client and server with their respective messages received, and possibly at different points in time. The labels are defined in \autoref{fig:labels}.
% 	%
% 	\fg{I'd suggest to introduce some color coding (akin to \cite[Fig.~1]{JC:DFGS21}) for messages, TLSKDF/RO-derived keys, etc.;
% 	this could match the to-be-pictured key schedule. (I can take care of both, if we want it.)}}
% 	\TODO{@fg: adapt, add stage numbers, ...}
% 	\TODO{@fg: left side: current figure, right side: figure with 11 RO-KDF + 1 RO $\Thash$, similar to \cite[Fig.~1]{JC:DFGS21}}
% 	\label{fig:tls13-psk}
% \end{figure}

%%% Local Variables:
%%% mode: latex
%%% TeX-master: "main"
%%% End:

\section{Code-based MSKE Model for PSK Modes}
\label{sec:ake-model}
We formalize security of the TLS~1.3 PSK modes in a game-based multi-stage key exchange (MSKE) model, adapted primarily from that of Dowling et al.~\cite{JC:DFGS21}. 
We fully specify our model in pseudocode in 
	Figures~\ref{fig:MSKE-model} and~\ref{fig:MSKE-preds}.
We adopt the explicit authentication property from the model of Davis and G{\"u}nther~\cite{ACNS:DavGun21} and capture forward secrecy by following the model of Schwabe et al.~\cite{CCS:SchSteWig20}.


\subsection{Key Exchange Syntax}

In our security model, the adversary interacts with \emph{sessions} executing a key exchange protocol $\KE$.
For the definition of the security experiment it will be useful to have a unified, generic interface to the algorithms implementing $\KE$, which can then be called from the various procedures defining the security experiment to run $\KE$. Therefore, we first formalize a general syntax for protocols.

We assume that pairs of users share long-term symmetric keys (pre-shared keys), which are chosen uniformly at random from a set $\KEpskeyspace$.%
\footnote{%
While our results can be generalized to any distribution on $\KEpskeyspace$ (based on its min-entropy),
for simplicity, we focus on the uniform distribution in this work.
}
We allow users to share multiple pre-shared keys, maintained in a list~$\pskeys$, and require that each user uses any key only in a fixed role (i.e., as client \emph{or} server) to avoid the Selfie attack~\cite{JC:DruGue21}.
We do not cover PSK negotiation;
each session will know at the start of the protocol which key it intends to use. 
% We do, however, wish to support the same user having multiple pre-shared keys it may use in a single role, we let the game maintain a list $\pskeys$ of all pre-shared keys that will be used in the game. 
% This list will be indexed by a tuple $(u, v, \pskid)$ containing the two users' IDs, and a unique string identifier.

New sessions are created via the algorithm $\KEActivate$.
This algorithm takes as input the new session's own user, identified by some ID~$u$, the user ID~$\peerid$ of the intended communication partner, a pre-shared key~$\psk$, and a role identifier---$\initiator$ (client) or $\responder$ (server)---that determines whether the session will send or receive the first protocol message.
It returns the new session $\pi_u^i$, which is identified by its user ID~$u$ and a unique index~$i$ so that a single user can execute many sessions.

Existing sessions send and receive messages by executing the algorithm $\KERun$.
The inputs to $\KERun$ are an existing session $\pi_u^i$ and a message $m$ it has received. 
The algorithm processes the message, updates the state of $\pi_u^i$, and returns the next protocol message $m'$ on behalf of the session. 
$\KERun$ also maintains the status of $\pi_u^i$, which can have one of three values: $\running$ when it is awaiting the next protocol message, $\accepted$ when it has established a session key, and $\rejected$ if the protocol has terminated in failure.

In a multi-stage protocol, sessions accept multiple session keys while running; we identify each with a numbered \emph{stage}.
A protocol may accept several stages/keys while processing a single message, and TLS~1.3 does this.
In order to handle each stage individually, our model adds artificial pauses after each acceptance to allow the adversary to interact with the sessions upon each stage accepting (beyond, as usual, each message exchanged).
When a session $\pi_u^i$ accepts in stage~$s$ while executing $\KERun$, we require $\KERun$ to set the status of $\pi_u^i$ to $\accepted_{s}$ and terminate.
We then define a special ``continue'' message. 
When session $\pi_u^i$ in state $\accepted_{s}$, receives this message it calls $\KERun$ again, updates its status to $\running_{s+1}$ and continues processing from the point where it left off.

% \iffull
% \tj{We do not yet say what happens when parsing of $(\peerid,\pskid,\role) \gets m$ fails (line 11 in the SEND query). Shall we ignore this?}
% \fg{If so, should say in the model; but currently I'd say ignore.}
% \fi

\subsection{Key Exchange Security}

We define key exchange security via a real-or-random security game, 
	formalized through Figures~\ref{fig:MSKE-model} and~\ref{fig:MSKE-preds}.

\paragraph{Game oracles.}
In this security game, the adversary~$\advA$ has access to seven oracles: $\Initialize$, $\NewSecret$, $\Send$, $\RevSessionKey$, $\RevLongTermKey$, $\Test$, and $\Finalize$, as well as any random oracles the protocol defines.
The game begins with a call to $\Initialize$, which samples a challenge bit $b$. 
It ends when the adversary calls $\Finalize$ with a guess $b'$ at the challenge bit.
We say the adversary ``wins'' the game if $\Finalize$ returns~$\true$.

The adversary can establish a random pre-shared key between two users by calling $\NewSecret$.%
\footnote{%
Our model stipulates that pre-shared keys are sampled uniformly random and honestly.
One could additionally allow the registration of biased or malicious PSKs, akin to models treating, e.g., the certification of public keys~\cite{ESORICS:BCFPPS13}.
While this would yield a theoretically stronger model, we consider a simpler model reasonable, because we expect most PSKs used in practice to be random keys established in prior protocol sessions.
Furthermore, we consider tightness as particularly interesting when ``good'' PSKs are used, since low-entropy PSKs might decrease the security below what is achieved by (non)-tight security proofs, anyway.
}
It can corrupt existing users' pre-shared keys via the oracle $\RevLongTermKey$.
The $\Send$ oracle creates new protocol sessions and processes protocol messages on the behalf of existing sessions.
The $\RevSessionKey$ oracle reveals a session's accepted session key.
Finally, the $\Test$ oracle servers as the challenge oracle:
it returns the real session key of a target session or an independent one sampled randomly from the session key space~$\KEkeyspace[s]$ of the respective stage $s$, depending on the value of the challenge bit~$b$.
%\tj{How about RegisterKey queries, where the adversary picks a (potentially biased) PSK and assigns it to parties? This is common in some pk models, and I think known to be strictly stronger, but on the other hand it often seems not to make a major difference (I think). I do not have a strong opinion, I think we discussed this briefly before, maybe we could talk about it in the next call and decide whether we want to leave a comment on this or not.}
%\fg{Added a footnote.}
	\begin{figure}[tp]
	\begin{minipage}[t]{0.55\textwidth}
	\NewExperiment[$G^{\KESEC}_{\KE,\advA}$]
	
		\begin{oracle}{$\Initialize$}
			\item $\time \gets 0$;
			\item $b \sample \bits$
% 			\item $\RO \sample \KE.\FSp$
% 				\TODO{either define function spaces in the main body or take out this line}
% 			\item for $u = 1$ to $n$
% 			\item \hindent $(pk_u, sk_u) \getsr \KEKGen()$
% 			\item \hindent $\revltk_u \gets \infty$
% 			\item return $\vec{pk} = (pk_1, \dots, pk_n)$
		\end{oracle}
		
		\ExptSepSpace
		
		\begin{oracle}{$\NewSecret(u, v, \pskid)$}
			\item $\time \gets \time + 1$
			\item if $\pskeys[(u, v, \pskid)] \neq \bot$
			\item \hindent return $\bot$
			\item $\pskeys[(u, v, \pskid)] \getsr \KEpskeyspace$ 
			\item $\revpsk_{(u,v,\pskid)} \gets \infty$
			\item return $\pskid$
		\end{oracle}
		
		\ExptSepSpace
		
		\begin{oracle}{$\Send(u, i, m)$}
			\item $\time \gets \time + 1$
			\item if $\pi_u^i = \bot$ then
		%	\item \hindent current\_cid $\gets \undef$
			\item \hindent $(\peerid,\pskid,\role) \gets m$ 
			\item \hindent if $\role = \initiator$
			\item \hindent \hindent then $\pskey \gets \pskeys[(u, \peerid, \pskid)]$
			\item \hindent \hindent else $\pskey \gets \pskeys[(\peerid, u, \pskid)]$
			\item \hindent $(\pi_u^i, m') \getsr \KEActivate(u\cab \peerid \cab \pskey  \cab \role)$
			\item else
			%\item \hindent current\_cid $\gets \pi_u^i.\cid[\pi_u^i.\stage][-1]$
			%last element, python notation may be unclear
			\item \hindent $(\pi_u^i, m') \getsr \KERun(u, \pi_u^i.\pskey, \pi_u^i, m)$
			\item if $\pi_u^i.\status = \accepted_{\pi_u^i.\stage}$ then
			\item \hindent $\stage \gets \pi_u^i.\stage$
			\item \hindent $\pi_u^i.\tacceptedtls[\stage] \gets \time$
			\item \hindent if $\reprogram[\pi_u^i.\sid[\stage]] \neq \bot$ then
			\item \hindent \hindent $\pi_u^i.\skey[\stage] \gets \reprogram[\pi_u^i.\sid[\stage]]$
			\item \hindent $\pi_u^i.\hascontpart[\stage] \gets \exists \pi_v^j$ with $\pi_v^j.\rolecid[\pi_u^i.\role][\stage] = \pi_u^i.\rolecid[\pi_u^i.\role][\stage]$
%%% old cid_I/R version
% 			\item \hindent $\pi_u^i.\hascontpart[\stage] \gets \exists \pi_v^j$ with $\pi_v^j.\initcid[\stage] = \pi_u^i.\initcid[\stage]$
% 			\item \hindent if $\pi_u^i.\role = \initiator$ then
% 			\item \hindent \hindent $\pi_u^i.\hascontpart[\stage] \gets (\pi_u^i.\hascontpart[\stage]$ AND $\exists \pi_v^j$ with $\pi_v^j.\respcid[\stage] = \pi_u^i.\respcid[\stage]$)
	
		%	\item if $\pi_u^i.\cid[\pi_u^i.\stage][-1] \neq$ current\_cid then
		%	\newline \comment{the cid has changed}
		%	\item \hindent $\pi_u^i.\tcid[$current\_cid$] \gets \time$
		%	\newline \comment{We log the time of all acceptances as well as the time of all updates to the contributive identifier.} 
			% For forward secrecy, we will need to identify whether a tested session's contributive partner existed at the time of the Test query (since the session should be fresh at that time). 
			% 
			\item return $m'$
		\end{oracle}
		
		\ExptSepSpace
		
		\begin{oracle}{$\RevSessionKey(u, i, s)$}
			\item $\time \gets \time+1$
			\item if $\pi_u^i = \bot$ or $\pi_u^i.\taccepted[s] = \infty$ then
			\item \hindent return $\bot$
			\item $\pi_u^i.\revealed[s] \gets \true$
			\item return $\pi_u^i.\skey[s]$
		\end{oracle}
		
		\ExptSepSpace
	\end{minipage}
%
	\begin{minipage}[t]{0.44\textwidth}
		\begin{oracle}{$\RevLongTermKey(u,v, \pskid)$}
			\item $\time \gets \time + 1$
			\item $\revpsk_{(u,v,\pskid)} \gets \time$
			\item return $\pskeys[(u,v,\pskid)]$
		\end{oracle}
		
		\ExptSepSpace

		\begin{oracle}{$\Test(u, i, s)$}
			\item $\time \gets \time+1$
			\item if $s \in \INT$ \newline
			\null \hindent \hindent  and $\exists \pi_v^j: \pi_v^j.\sid[s] = \pi_u^i.\sid[s]$ \newline
			\null \hindent \hindent  and $\pi_v^j.\taccepted[s] < \infty$ \newline
			\null \hindent \hindent and $\pi_v^j.\status \neq \accepted_{s}$ then
% 				\fullonly{\fg{We could make $\status$ ordered and check for $\status > \accepted_{s}$.}}
			\item \hindent return $\bot$\newline
			\null {\comment{can only test internal keys if all sessions having accepted that key have not moved on with the protocol}}
			\item if $\pi_u^i = \bot$ or $\pi_u^i.\taccepted[s] = \infty$ or $\neg \pi_u^i.\tested[s]$ then
			\item \hindent return $\bot$
			
			\item $\pi_u^i.\tested[s] \gets \time$
			
			\item $\testedsessions \gets \testedsessions \cup \{(\pi_u^i,s)\}$
			\item $k_0 \gets \pi_u^i.\skey[s]$
			\item $k_1 \sample \KEkeyspace[s]$
			\item if $s \in \INT$ then \newline
			\null \hindent $\forall \pi_v^j\colon \pi_v^j.\sid[s] = \pi_u^i.\sid[s]$ \newline
			\null \hindent \hindent \hindent and  $\pi_v^j.\status = \accepted_{s}$
			\item \hindent \hindent $\pi_v^j.\skey[s] \gets k_b$
			\item \hindent $\reprogram[\pi_u^i.\sid[s]] \gets k_b$
			\item return $k_b$
		\end{oracle}
		\ExptSepSpace
		
		\begin{oracle}{$\Finalize(b')$}
			\item if $\neg \Sound$ then
			\iffull\item \hindent\fi return $1$
			
			\item if $\neg \ExplicitAuth$ then
			\iffull\item \hindent\fi return $1$
			
			\item if $\neg \Fresh$ then
			\iffull\item \hindent\fi $b' \gets 0$
			
			\item return $[[b = b']]$
		\end{oracle}
		
		\ExptSepSpace
		
		\begin{oracle}{$\RO(i, \X)$}
			\item $\time \gets \time+1$
			\item return $\RO_i(\X)$
		\end{oracle}
		
	\end{minipage}
	
	\caption{%
		Multi-stage key exchange (MSKE) security game for a key exchange protocol~$\KE$ with pre-shared keys.
		Predicates $\Fresh$, $\ExplicitAuth$, and $\Sound$ are defined in Figure~\ref{fig:MSKE-preds}.
		The functions~$\RO_i$ correspond to the (independent) random oracles available to the adversary.
	}
	\label{fig:MSKE-model}
\end{figure}

%%% Local Variables:
%%% mode: latex
%%% TeX-master: "main"
%%% End:

	\begin{figure}[tp]
	\begin{minipage}[t]{0.49\textwidth}
		\begin{algorithm}{$\Fresh$}
		\item for each $(\pi_u^i,s) \in \testedsessions$
		\item \hindent $\ttest \gets \pi_u^i.\tested[s]$
		\item \hindent if $\pi_u^i.\revealed[s]$ then
		\item \hindent \hindent return $\false$
		\comment{tested session may not be revealed}
		
		\item \hindent if $\exists \pi_v^j \neq \pi_u^i : \pi_v^j.sid[s] = \pi_u^i.sid[s]$ 
		\newline \null \hindent \hindent and ($\pi_v^j.\tested[s]$ or $\pi_v^j.\revealed[s]$) then
		\item \hindent \hindent return $\false$
		\comment{tested session's partnered session may not be tested or revealed}
		
		%%%  forward secrecy
		\item \hindent if $\pi_u^i.\taccepted[\FS[s,\fs]]<\ttest$
		\item \hindent \hindent if $\revpsk_{(u, \pi_u^i.\peerid, \pi_u^i.\pskid)} < \pi_u^i.\taccepted[\FS[s,\fs]]$ and $\neg \pi_u^i.\hascontpart[\FS[s,\fs]]$ then %\hascontpart[s]?
		\item \hindent \hindent return $\false$
		\comment{Sessions with forward secrecy are fresh if they attained fs before their PSK was corrupted, or if they have a contributive partner (no tampering).}
		
		%%% Weak forward secrecy 2
		\item \hindent else if $\pi_u^i.\taccepted[\FS[s,\wfstwo]]<\ttest$
		\item \hindent \hindent if $\revpsk_{(u, \pi_u^i.\peerid, \pi_u^i.\pskid)}$ and $\neg \pi_u^i.\hascontpart[\FS[s, \wfstwo]]$ then %\hascontpart[s]?
		\item \hindent \hindent \hindent return $\false$
		\comment{Sessions with weak forward secrecy 2 are fresh if the PSK was never corrupted, or if they have a contributive partner.}
		
		%% no fs
		\item \hindent else if $\revpsk_{\{u, \pi_u^i.\peerid\}, \pi_u^i.\pskid}$ then 
		\item \hindent \hindent return $\false$
		\comment{Sessions with no forward secrecy are fresh if the PSK was never corrupted.}
		\item return $\true$
	\end{algorithm}
	
	\ExptSepSpace
	
		\begin{algorithm}{$\ExplicitAuth$}
			\item if $\forall \pi_u^i, s$:\\
			\hindent $s' \gets \EAUTH[\pi_u^i.\role, s]$ \\
			$\pi_u^i.\taccepted[s'] < \infty$ \\
			\null \hindent and $\pi_u^i.\taccepted[s] < \infty$ \\
			\null \hindent and $\pi_u^i.\taccepted[s'] <  \revpsk_{(u,\pi_u^i.\peerid,\pi_u^i.\pskid)}$ \\
			\null \hindent and $\pi_u^i.\taccepted[s'] <  \infty$ \\
			\comment{all sessions accepting in explicitly authenticated stages whose PSK was not corrupted before acceptance of the stage at which explicit authentication was (perhaps retroactively) established\dots\ }%
			%muahahaha
			\hindent \hindent  $\implies \exists \pi_v^j : \pi_u^i.\sid[s'] = \pi_v^j.\sid[s']$ \\
			\hindent \hindent \hphantom{$\implies$} and $\pi_u^i.\peerid = v$ \\
			\hindent \hindent \hphantom{$\implies$} and $\pi_u^i.\pskid = \pi_v^j.\pskid$ 
			%\newline
			%\null \hindent \hindent \hphantom{$\implies$} and $\pi_u^i.\role \neq \pi_v^j.\role$
			\newline
			\comment{\dots\ have a partnered session in that stage \dots\ }
			\comment{\dots\ agreeing on the peerid and pre-shared key\dots\ }
			\hindent \hphantom{$\implies$} and $(\pi_v^j.\taccepted[s] < \time \implies \pi_v^j.\sid[s] = \pi_u^i.\sid[s])$
			\newline
			\comment{\dots\ and partnered in stage $s$ (upon acceptance)}
			\item \hindent return $\true$
		\end{algorithm}
	\end{minipage}
%
	\begin{minipage}[t]{0.49\textwidth}
	\begin{algorithm}{$\Sound$}
		%%% no triple sid match
		\item if $\exists s$, distinct $\pi_u^i$, $\pi_v^j$, $\pi_w^k$ with $\pi_u^i.\sid[s] = \pi_v^j.\sid[s] = \pi_w^k.\sid[s]\neq\bot$ \newline
		and $\REPLAY[s] = \false$ then
		\item \hindent return $\false$
		\newline \comment{no triple sid match, except for replayable stages}
		%%% same sid ==> different roles (except replays)
		\item if $\exists \pi_u^i, \pi_v^j$, $s$ with \newline
		\null\hindent 
		$\pi_u^i.\sid[s] = \pi_v^j.\sid[s] \neq \bot$ and \newline
		\null\hindent $\pi_u^i.\role = \pi_v^j.\role$ and \newline
		\null\hindent ($\REPLAY[s] = \false$ or $\pi_u^i.\role = \initiator$) then 
		\item \hindent return $\false$
		\newline \comment{partnering implies different roles (except for responders in replayable stages)}
		
		%%% same sid ==> same cid
		\item if $\exists \pi_u^i, \pi_v^j$, $s$ with \newline
		\null \hindent
		$\pi_u^i.\sid[s] = \pi_v^j.\sid[s] \neq \bot$ and \newline $(\pi_u^i.\initcid[s] \neq \pi_v^j.\initcid[s]$ or $\pi_u^i.\respcid[s] \neq \pi_v^j.\respcid[s])$
		\item \hindent return $\false$
		\newline \comment{partnering implies matching cids}
		
		%%% different stages ==> different sids
		if $\exists \pi_u^i, \pi_v^j$ and $s \neq t$ such that
		\newline
		\null \hindent $\pi_u^i.\sid[s] = \pi_v^j.\sid[t]$
		\item \hindent return $\false$
		\newline \comment{different stages implies different sids}
		
		%%% partnering implies agreement on peer ID and PSKID
		\item if $\exists \pi_u^i, \pi_v^j$, $s$ with \newline
		\null\hindent $\pi_u^i.\sid[s] = \pi_v^j.\sid[s]\neq \bot $ \newline
		\null\hindent and $\pi_u^i.\peerid \neq v$ \newline
		\null\hindent or $\pi_v^j.\peerid \neq u$ or $\pi_u^i.\pskid \neq \pi_v^j.\pskid$ then
		\newline  \comment{partnering implies agreement on peer IDs and PSKs}
		%because all stages are at least implicitly authenticated.
		\item \hindent return $\false$
		
		%%% same sid ==> same key
		\item if $\exists \pi_u^i, \pi_v^j$, $s$ with \newline
		\null\hindent $\pi_u^i.\taccepted[s] < \time $ \newline
		\null\hindent and $ \pi_v^j.\taccepted[s] < \time$ \newline
		\null\hindent and $\pi_u^i.\sid[s] = \pi_v^j.\sid[s]\neq \bot$, \newline
		\null\hindent but $\pi_u^i.\skey[s] \neq \pi_v^j.\skey[s]$ then
		\newline  \comment{partnering implies same key}
		\item \hindent return $\false$
		\item return $\true$
	\end{algorithm}
	\end{minipage}
	
	\caption{%
		Predicates $\Fresh$, $\ExplicitAuth$, and $\Sound$ for the MSKE pre-shared key model.
	}
	\label{fig:MSKE-preds}
\end{figure}


\paragraph{Protocol properties.}
Keys established in different stages possess different security attributes, which are defined as part of the key exchange protocol: replayability, forward secrecy level, and authentication level. 
Certain stages, whose indices are tracked in a list $\INT$, produce ``internal'' keys intended for use only within the key exchange protocol; these keys may only be $\Test$ed at the time of acceptance of this particular key, but not later.
This is because otherwise such keys may be trivially distinguishable from random, e.g., via trial decryption, due to the fact that they are used within the protocol.
To avoid a trivial distinguishing attack, we force the rest of the protocol execution to be consistent with the result of such a $\Test$.
That is, a tested internal key is replaced in the protocol with whatever the $\Test$ returns to the adversary (which is either the real internal key or an independent random key).
%\ifdraft
%\tj{It seems we have not really specified this interface in the protocol syntax, but I think it is fine and we should not worry about this.}
%\fg{Don't understand in which way this relates to protocol syntax, let's discuss for full version.}
%\fi
The remaining stages produce ``external'' keys which may be tested at any time after acceptance. 

For some protocols, it may be possible that a trivial replay attack can achieve that several sessions agree on the same session key for stage $s$, but this is not considered an ``attack''.
For example, in TLS~1.3 PSK an adversary can always replay the $\ClientHello$ message to multiple sessions of the same server, which then all derive the same $\ets$ and $\eems$ keys (cf. Figure \ref{fig:tls-handshake}).
%\tj{Why does TLS~1.3 not use e.g. the server nonce here?}
%\fg{These are 0-RTT keys, so there is no server contribution (e.g., nonce) yet.}
%
To specify that such a replay is not considered a protocol weakness, and thus should not be considered a valid ``attack'', the protocol specification may define $\REPLAY[s]$ to $\true$ for a stage $s$. $\REPLAY[s]$ is set to $\false$ by default. 

As we focus on protocols which rely on (pre-authenticated) pre-shared keys,
our model encodes that all protocol stages are at least \emph{implicitly} mutually authenticated in the sense of Krawczyk~\cite{EPRINT:Krawczyk05},
i.e., a session is guaranteed that any established key can only be known by the intended partner.
Some stages will further be \emph{explicitly} authenticated, either immediately upon acceptance or retroactively upon acceptance of a later state. 
Additionally, the stage at which explicit authentication is achieved may differ between the initiator and responder roles.
For each stage~$s$ and role~$r$, the key exchange protocol specification states in~$\EAUTH[r, s]$ the stage~$t$ from whose acceptance stage~$s$ derives explicit authentication for the session in role~$r$.
Note that the stage-$s$ key is not authenticated until both stages~$s$ and~$\EAUTH[r, s]$ have been accepted. 
If the stage-$s$ key will never be explicitly authenticated for role~$r$, we set $\EAUTH[r,s] = \infty$.

We use a predicate $\ExplicitAuth$ 
	(cf.\ Figure~\ref{fig:MSKE-preds})
to require the existence of an honest partner for explicitly authenticated stages upon both parties' completion of the protocol, except when the session's pre-shared key was corrupted prior to accepting the explicitly-authenticating stage
(as in that case, we anticipate the adversary can trivially forge any authentication mechanism).

Motivated by TLS~1.3, it might be the case that initiator and responder sessions achieve slightly different guarantees of authentication.
While responders in TLS~1.3 are guaranteed the existence of an honest partner in any explicitly authenticated stage, initiators cannot guarantee that their partner has received their final message.
This issue was first raised by FGSW~\cite{SP:FGSW16} and led to their definitions of ``full'' and ``almost-full'' key confirmation; it was then extended to ``full'' and ``almost-full'' explicit authentication by DFW~\cite{CSF:deSFisWar20}. 
Our definitions for responders and initiators respectively resemble the latter two notions most closely, but we rely on session identifiers instead of ``key confirmation identifiers''.
%% We do not need rectified authentication, so let's not discuss it in that detail.
% The notion of \emph{rectified authentication} (as discussed in~\cite{JC:DFGS21}) captures the possibility that an adversary could impersonate an honest user during an unauthenticated stage, then later corrupt the pre-shared key in order to falsely authenticate as that user. 
% The keys agreed upon during the unauthenticated stage could not in this scenario retroactively be granted explicit authentication. However, in our model all stages will be at least implicitly mutually authenticated, so we do not need rectified authentication. 
% Instead, we use a predicate $\ExplicitAuth$ which will require the existence of an honest partner except when such a corruption occurs before acceptance. \tj{I did not understand this paragraph...}

We consider three levels of forward secrecy inspired by the KEMTLS work of Schwabe, Stebila, and Wiggers~\cite{CCS:SchSteWig20}: no forward secrecy, weak forward secrecy~2 (wfs2), %\fg{Can we find a better name for this?},
and full forward secrecy (fs). 
As for authentication, each stage may retroactively upgrade its level of forward secrecy upon the acceptance of later stages, and  forward secrecy may be established at different stages for each role.
For each stage $s$ and role $r$, the stage at which wfs2, resp.\ fs, is achieved is stated in~$\FS[r, s, \wfstwo]$, resp.\ $\FS[r, s, \fs]$, by the key exchange protocol.

The definition of weak forward secrecy~2 states that a session key with wfs2 should be indistinguishable as long as
(1) that session has received the relevant messages from an honest partner (formalized via matching contributive identifiers below, we say: ``has an honest contributive partner'')
or (2) the pre-shared key was never corrupted.
Full forward secrecy relaxes condition (2) to forbid corruption of the pre-shared key only before acceptance of the stage that retroactively provides full forward secrecy. 
We capture these notions of forward secrecy in a predicate $\Fresh$%
	(cf.\ Figure~\ref{fig:MSKE-preds})%
, which uses the log of events to check whether any tested session key is trivially distinguishable
(e.g., through the session or its partnered being revealed, or forward secrecy requirements violated).
With forward secrecy encoded in $\Fresh$, our long-term key corruption oracle ($\RevLongTermKey$), unlike in the model of~\cite{JC:DFGS21}, handles all corruptions the same way, regardless of forward secrecy. 

\paragraph{Session and game variables.}
Sessions~$\pi_u^i$ and the security game itself maintain several variables;
we indicate the former in $\mathit{italics}$, the latter in $\mathsf{sans{\mhyphen}serif}$ font.

% \fg{Shall we use unified font formatting for game variables? Right now some are \textsf{sans-serif}, other \textit{italics}.}
% \fg{Sans-serif is set by game, italics is set by protocol.}
The game uses a counter~$\time$, initialized to~$0$ and incremented with any oracle query the adversary makes, to order events in the game log for later analysis.
When we say that an event happens at a certain ``time'', we mean the current value of the $\time$ counter.
The list $\pskeys$ contains, as discussed above, all pre-shared keys, indexed by a tuple $(u, v, \pskid)$ containing the two users' IDs ($u$ using the key only in the initiator role, $v$ only in the reponder role), and a unique string identifier.
% $b$ is the challenge bit.
The list~$\revpsk$, indexed like~$\pskeys$, tracks the time of each pre-shared key corruption, initialized to~$\revpsk_{(u,v, \pskid)} \gets \infty$.
(In boolean expressions, we write $\revpsk_{(u,v, \pskid)}$ as a shorthand for $\revpsk_{(u,v,\pskid)} \neq \infty$.)
% \tj{It seems unclear to me how this list tracks the exact time. Shall we make explicit that we set $\revpsk_{(u,v, \pskid)} := \time$ when the PSK $(u, v, \pskid)$ is corrupted? }

Each session~$\pi_u^i$, identified by (adversarially chosen) user ID and a unique session ID, furthermore tracks the following variables:
\begin{itemize}
	\item $\status \in \{\running_s, \accepted_s, \rejected_s \mid s \in [1,\ldots,\STAGES]\}$, where $\STAGES$ is the total number of stages of the considered protocol.
	The status should be $\accepted_s$ immediately after the session accepts the stage-$s$ key, $\rejected_s$ after it rejects stage~$s$ (but may continue running; e.g., rejecting 0-RTT data), and $\running_s$ for some stage $s$ otherwise.
% 	Note: we do not define $\rejected$ flags for each stage as we are first considering TLS~1.3 PSK without 0-RTT.
% 	\fg{Wait, this is way outdated by now, no?}
% 	This can be added later.\tj{I do not understand the last two sencences.}
	
	\item $\peerid$.
	The identity of the session's intended communication partner.
	
	\item $\pskid$.
	The identifier of the session's pre-shared key.
	
	\item $\taccepted[s]$.
	For each stage $s$, the time (i.e., the value of the $\time$ counter) at which the stage $s$ key was accepted. Initialized to $\infty$.
	
	\item $\revealed[s]$.
	A boolean denoting whether the stage $s$ key has been leaked through a $\RevSessionKey$ query. Initialized to $\false$.
	
	\item $\tested[s]$.
	The time at which the stage $s$ key was tested. Initialized to $\infty$ before any Test query occurs. (In boolean expressions, we write $\tested[s]$ as a shorthand for $\tested[s] \neq \infty$.)
	
	\item $\sid[s]$.
	The session identifier for each stage~$s$, used to match honest communication partners within each stage.
	
	\item $\skey[s]$.
	The key accepted at each stage.
	
	\item $\initcid[s]$ and $\respcid[s]$.
	The contributive identifiers for each stage~$s$, where $\rolecid[\role][s]$ identifies the communication part that a session in role~$\role$ must have honestly received in order to be allowed to be tested in certain scenarios (cf.\ the freshness definition in the $\Fresh$ predicate).
	Unlike prior models, each session maintains a contributive identifiers for each role; one for itself and one for its intended partner.
	This enables more fine-grained testing of session stages in our model.
\end{itemize}
The predicate $\Sound$
	(cf.\ Figure~\ref{fig:MSKE-preds})
captures that variables are properly assigned, in particular that session identifiers uniquely identify a partner session (except for replayable stages)
and that partnering implies agreement on (distinct) roles, contributive identifiers, peer identities and the pre-shared key used, as well as the established session key.

\begin{definition}[Multi-stage key exchange security]
	\label{def:MSKE-security}
	Let $\KE$ be a key exchange protocol and~$G^{\KESEC}_{\KE,\advA}$ be the key exchange security game defined 
		in Figures~\ref{fig:MSKE-model} and~\ref{fig:MSKE-preds}.
	We define
	\[
		\Adv^{\KESEC}_{\KE}(t, \qNewSecret, \qSend, \qRevSessionKey, \qRevLongTermKey, \qTest, \qRO) := 2 \cdot \max_\advA \Pr \left[ \Gm^{\KESEC}_{\KE,\advA} \Rightarrow 1 \right] - 1,
	\]
	where the maximum is taken over all adversaries, denoted \emph{$(t\cab \qNewSecret\cab \qSend\cab \qRevSessionKey\cab \qRevLongTermKey\cab \qTest\cab \qRO)$-$\KESEC$-adversaries},
	running in time at most~$t$ and making at most $\qNewSecret$, $\qSend$, $\qRevSessionKey$, $\qRevLongTermKey$, $\qTest$, resp.\ $\qRO$ queries to their respective oracles $\NewSecret$, $\Send$, $\RevSessionKey$, $\RevLongTermKey$, $\Test$, and~$\RO$.
\end{definition}


%%% Local Variables:
%%% mode: latex
%%% TeX-master: "main"
%%% End:

In the random oracle model, we treat hash functions like $\SHA{256}$ as uniformly sampled random functions. 
Honest parties and adversaries alike access these functions via additional oracles in the security game. 
These are the \emph{random oracles}. 
These random functions will be sampled from a set called a \emph{function space} at the start of a security game.
Alternatively, the random oracle can \emph{lazily sample} responses to each query as they are needed. 
While we typically use the latter (lazily-sampled) model in key exchange security proofs, we will focus on the former conceptual view here. 

Let us give an example. 
When we model the TLS~1.3 protocol in the ROM, we will equip our protocol definition with a function space parameter $\FSp$. 
We set this parameter according to the portion of the protocol we wish to model as a random oracle. 
If we wish to replace the hash function $\hash$ with a random oracle $\ROhash$, then we would set $\FSp$ to be the set of all functions the set of all functions with domain $\bits^*$ and range $\bits^{\hashlen}$. 
The $\KE$ security game would sample $\ROhash$ from $\FSp$ in its $\Initialize$ routine, then provide oracle access to $\ROhash$ to all parties. 
This notation also captures protocols which use multiple random oracles.
If we wish to use two independent random oracles, say $\RO_1$ and $\RO_2$, then we would define an \emph{arity-$2$} function space $\FSp$, which is a set of tuples each containing two functions. 
Let $\FSp_1$, resp. $\FSp_2$ be the set from which $\RO_1$, $\RO_2$ should be drawn. 
Then we set $\FSp = \{(F_1, F_2): F_1 \in \FSp_1 \text{ and } F_2 \in \FSp_2\}$.
We call $\FSp_1$ and $\FSp_2$ the subspaces of $\FSp$. 
A security game provides access to $F_1$ and $F_2$ through a single oracle $\RO$ that takes two arguments; the first is the index of the function to be queried and the second is the contents of the query. 
So $\RO(i, \X)$ will return $F_i(\X)$.
We can also cast an arity-$1$ function space in this notation by identifying each function $F$ with the tuple $(F)$, but we will typically omit the parentheses and index argument when only one random oracle is used.


Indifferentiability was originally developed by Maurer, Renner, and Holenstein~\cite{TCC:MauRenHol04}, and it has been used to prove security for hash functions built from public compression functions.
More generally, it gives a framework to show the security of a transition between any two function spaces.
We'll call these spaces $\SSp$ (for ``starting space'') and $\ESp$ (for ``ending space'').
A \textit{construction} of $\ESp$ from $\SSp$ is an algorithm $\construct{C}$ that outputs elements of $\ESp$ given an oracle $\RO_{\SSp} \in \SSp$. 
We may use the notation $\construct{C}:\SSp \to \ESp$.
We then say that $\construct{C}$ is ``indifferentiable'' if for any function $\RO_{\SSp}$ sampled from $\SSp$, $\construct{C}[\RO]$ behaves indistinguishably from a function $\RO_{\ESp}$ sampled from $\ESp$.
Indifferentiability requires this behavior to hold even when the adversary can access \emph{both $\construct{C}[\RO_{\SSp}]$ and $\RO_{\SSp}$} without any restriction. 
%
Once we have an indifferentiable construction between two function spaces, we can use the indifferentiability ``composition theorem'' to prove that (almost) any protocol is as secure when it uses $\construct{C}[\RO_{\SSp}]$ as its random oracle as when it uses $\RO_{\ESp}$.%
\footnote{As Ristenpart, Shacham, and Shrimpton~\cite{EC:RisShaShr11} showed, indifferentiability composition does not cover what they call ``multi-stage games,'' meaning games in which the adversary is split into distinct algorithms with restricted communication. Our multi-stage AKE security game is actually a ``single-stage'' game in the RSS terminology; indifferentiability composition does apply to our results without issue.}

How do we check whether a construction $\construct{C}$ is indifferentiable?
From the earlier intuition, we set up a security game with two worlds.
In one world, often called the ``real world'', the adversary has oracle access to $\RO_{\SSp}$ (drawn from $\SSp$) and $\construct{C}[\RO_{\SSp}]$.
In the other, the ``ideal world'', it has oracle access to $\RO_{\ESp}$, a random oracle sampled from $\ESp$.
The adversary's task is then to return a bit indicating which world it is in.

This intuition is obviously incomplete:
the adversary can distinguish between worlds just by counting its oracles.
We need a second oracle in the ideal world.
This second oracle, $\PubO$, must behave indistinguishably from $\RO_{\SSp}$, but its responses must also be consistent with the view of $\RO_{\ESp}$ (accessed via the first oracle, $\PrivO$) as a construction of $\PubO$.
The algorithm that does this is called a ``simulator''.
Every construction requires a different simulator $\simulator$, so we make it a parameter of the definition.
We can now give pseudocode for the full indifferentiability security game, shown in Figure~\ref{fig:game-indiff}.

\begin{figure}[tp]
	\centering
	\begin{minipage}[t]{0.3\textwidth}
		\NewExperiment[Game $\Gindiff_{\construct{C}, \simulator,\SSp, \ESp}$]
		
		\begin{oracle}{$\Initialize()$}
			\item $b \getsr \bits$
			\item $\RO_{\SSp} \getsr \SSp$
			\item $\RO_{\ESp} \getsr \ESp$
			\item $\state \getsr \emptystring$
		\end{oracle}
		
		\ExptSepSpace
		
		\begin{oracle}{$\Finalize(b')$}
			\item return \smash{$b'$}
		\end{oracle}
	\end{minipage}
	\begin{minipage}[t]{0.49\textwidth}
		\vphantom{\underline{Game $\Gindiff_{\construct{C}, \simulator,\SSp, \ESp}$}}
		\ExptSepSpace
		
		\begin{oracle}{$\PubO(i,\Y)$}
			\item if $b = 0$ then
			\item \quad  $(z,\state) \gets \simulator[\PrivO](i,\Y,\state)$
			\item \quad return $z$
			\item else return $\RO_{\SSp}(i,\Y)$
		\end{oracle}
		
		\ExptSepSpace
		
		\begin{oracle}{$\PrivO(i,\X)$}
			\item if $b = 0$ then return $\RO_{\SSp}(i,\X)$
			\item else return $\construct{C}[\RO_{\ESp}](i, \X)$
		\end{oracle}
	\end{minipage}
	\vspace{5pt}
	\caption{The game  $\Gindiff_{\construct{C}, \simulator, \SSp, \ESp}$ measuring indifferentiability of a construct $\construct{C}$ that transforms function space $\SSp$ into $\ESp$. The game is parameterized by a simulator $\simulator$.}
	\label{fig:game-indiff}
\end{figure}


%We can now give a formal security definition for indifferentiability.
\begin{definition}[Indifferentiability]
	Let $\SSp$ and $\ESp$ be function spaces, and let $\construct{C}$ be a construction of $\ESp$ from $\SSp$. Then for any simulator $\simulator$ and any adversary $\advD$ which makes $\qPriv$ queries to the $\PrivO$ oracle and $\qPub$ queries to the $\PubO$ oracle, the indifferentiability advantage of $\advD$ is
	\[\genAdv{\indiff}{\construct{C},\simulator, \qPriv, \qPub}{\advD} := \Pr[\Gindiff_{\construct{C}, \simulator}(\advD) \Rightarrow 1 | b = 1] -\Pr[\Gindiff_{\construct{C}, \simulator}(\advD) \Rightarrow 1 | b = 0].\]
\end{definition}


%\subsection{Indifferentiability composition}
Indifferentiability is useful because of the following theorem of Maurer et al.~\cite{TCC:MauRenHol04}. In our presentation, we consider only the authenticated key exchange game, although the theorem applies equally well to any single-stage game~\cite{EC:RisShaShr11}. 

\begin{theorem} 
	\label{thm:indiff-comp}
	Let $\KE$ be a key exchange protocol using function space $\ESp$. Let $\construct{C}$ be an indifferentiable construct of $\ESp$ from $\SSp$ with respect to simulator $\simulator$, and let $t'$ be the runtime of $\simulator$ on a single query. We define $\KE'$ to be the following key exchange protocol with function space $\SSp$: $\KE'$ runs $\KE$, but wherever $\KE$ would call its random oracle, $\KE'$ instead computes $\construct{C}$ using its own random oracle. For any adversary $\advA$ against the $\KESEC$ security of $\KE'$ with runtime $t_\advA$ and making $q$ random oracle queries, there exists an adversary $\advB$ and a distinguisher $\advD$ with runtime approximately $t_\advA + q \cdot t$ such that
	\[ \genAdv{\KESEC}{\KE'}{\advA} \leq \genAdv{\KESEC}{\KE}{\advB} + \genAdv{\indiff}{\construct{C},\simulator}{\advD}. \]
\end{theorem}

\begin{proof}
	Adversary $\advB$ is a wrapper for $\advA$ whenever $\advA$ makes a query to its random oracle $\RO$, $\advB$ responds by running the simulator with its own random oracle. The distinguisher $\advD$ simulates the $\KE-Sec$ game of $\KE$ for $\advA$, with two differences: instead of an RO, it gives $\advA$ oracle access to $\PubO$, and where $\KE$ would query its own RO, it instead queries $\PrivO$. 
	We claim that when $b=1$ in the indifferentiability game (the real world), $\advD$ perfectly simulates the $\KESEC$ game of $\KE'$ for $\advA$. This works because the $\PrivO$ oracle computes $\construct{C}$ for $\KE'$, and the $\PubO$ oracle is indeed an RO as $\advA$ expects. When $b=0$, $\advD$ perfectly simulates $\KESEC$ of $\KE$ for $\advB$. The $\PubO$ oracle answers all of $\advA$'s queries using the simulator, so it properly executes the wrapper code that makes up $\advB$. The rest of the simulation is honest, down to the random oracle accessed via $\PrivO$. 
\end{proof}
\section{Key-Schedule Indifferentiability}\label{sec:ks-indiff}

In this section we will argue that the key schedule of TLS~1.3 PSK modes, where the underlying cryptographic hash function is modeled as a random oracle (i.e., the left-hand side of Figure~\ref{fig:tls-handshake} with the underlying hash function modeled as a random oracle), is \emph{indifferentiable}~\cite{TCC:MauRenHol04} from a key schedule that uses \emph{independent} random oracles for each step of the key derivation (i.e., the right-hand side of Figure~\ref{fig:tls-handshake} with all $\TLSKDF_x$ functions modeled as independent random oracles).
%Previous analyses of TLS~1.3 \cite{ACNS:DavGun21,JC:DieJag21} simply assumed that one can view the individual steps of the key derivation as 
%
We stress that this step not only makes our main security proof in Section~\ref{sec:ke-proof} significantly simpler and cleaner, but also it puts the entire protocol security analysis on a firmer theoretical ground than previous works.
For some background on the indifferentiability framework, see
	Section~\ref{app:indiff-background}.

In their proof of tight security, Diemert and Jager~\cite{JC:DieJag21} previously modeled the TLS~1.3 key schedule as four independent random oracles.
Davis and Günther~\cite{ACNS:DavGun21} concurrently modeled the functions $\HKDFExtr$ and $\HKDFExpnd$ used by the key schedule as two independent random oracles. 
Neither work provided formal justification for their modeling. 
Most importantly, both neglected potential dependencies between the use of the hash function in multiple contexts in the key schedule and elsewhere in the protocol. 
In particular, no construction of $\HKDFExtr$ and $\HKDFExpnd$ as independent ROs from one hash function could be indifferentiable, because $\HKDFExtr$ and $\HKDFExpnd$ both call $\HMAC$ directly on their inputs, with $\HKDFExpnd$ only adding a counter byte. 
Hence, the two functions are inextricably correlated by definition. 
We do not claim that the analyses of \cite{JC:DieJag21,ACNS:DavGun21} are incorrect or invalid, but merely point out that their modeling of independent random oracles is currently not justified and might not be formally reachable if one only wants to treat the hash function itself as a random oracle.
%
This is undesirable because the gap between an instantiated protocol and its abstraction in the random oracle model can camouflage serious attacks, as Bellare et al.~\cite{EC:BelDavGun20} found for the NIST PQC KEMs. 
Their attacks exploited dependencies between functions that were also modeled as independent random oracles but instantiated with a single hash function.

In contrast, in this section we will show that our modeling of the TLS~1.3 key schedule is indifferentiable from the key schedule when the underlying cryptographic hash function is modeled as a random oracle. 
To this end, we will require that inputs to the hash function do not appear in multiple contexts. 
For instance, a protocol transcript might collide with a Diffie--Hellman group element or an internal key (i.e., both might be represented by exactly the same bit string, but in different contexts). 
For most parameter settings, we can rule out such collisions by exploiting serendipitous formatting, but for one choice of parameters (the PSK-only handshake using \SHA{384} as hash function), an adversary could conceivably force this type of collision to occur; see 
	Appendix~\ref{app:domsep}
for a detailed discussion.
While this does not lead to any known attack on the handshake, it precludes our indifferentiability approach for that case.

\paragraph{Insights for the design of cryptographic protocols.}
One interesting insight for protocol designers that results from our attempt of closing this gap with a careful indifferentiability-based analysis is that proper domain separation might enable a cleaner and simpler analysis, whereas a lack of domain separation leads to uncertainty in the security analysis. 
No domain separation means stronger assumptions in the best case, and an insecure protocol in the worst case, due to the potential for overlooked attack vectors in the hash functions. 
A simple prefix can avoid this with hardly any performance loss.

\paragraph{Indifferentiability of the TLS~1.3 key schedule.}
Via the indifferentiability framework, we replace the complex key schedule of TLS~1.3 with $12$ independent random oracles: one for each first-class key and $\MAC$ tag, and one more for computing transcript hashes. 
In short, we relate the security of TLS~1.3 as described in the left-hand side of Figure~\ref{fig:tls-handshake} to that of the simplified protocol on the right side of Figure~\ref{fig:tls-handshake} with the key derivation and $\MAC$ functions $\TLSKDF_x$ and modeled as independent random oracles.
We prove the following theorem, which formally justifies our abstraction of the key exchange protocol by reducing its security to that of the original key exchange game.

\begin{theorem}
	\label{thm:full-ks-indiff}
	Let $\ROhash \colon \bits^{*} \to \bits^{\hashlen}$ be a random oracle.
	%
	Let $\KE$ be the TLS~1.3 PSK-only or PSK-(EC)DHE handshake protocol described on the left hand side of Figure~\ref{fig:tls-handshake} with $\abstractHash := \ROhash$
	and $\abstractMAC$, $\abstractExtract$, and $\abstractExpand$ defined from~$\abstractHash$ as in Section~\ref{sec:tls13-psk-protocol}. 
%\fg{That definition is currently missing in Sec.2, \TODO{check}.}
	%
	Let $\KE'$ be the corresponding (PSK-only or PSK-(EC)DHE) handshake protocol on the right hand side of Figure~\ref{fig:tls-handshake}, with $\abstractHash := \ROthash$ and  $\TLSKDF_x := \RO_{x}$, where $\ROthash$, $\RObinder$, \dots, $\ROrms$ are random oracles with the appropriate signatures
		(cf.~\iffull Section~\else Appendix~\fi\ref{sec:many-ros}
	for the signature details).
%
%For any adversary $\advA$ attacking the $\KESEC$ security of $\KE$, 
%Llet $t_\advA$ denote the runtime of $\advA$, and let $\qRO$ and $\qSend$ capture the number of queries $\advA$ makes to the $\RO$ and $\Send$ oracles respectively. 
	Then, %for $t \approx t'$,
%there exists an adversary $\advB$ against the security of $\KE'$ such that 
	\begin{align*}
		\Adv^{\KESEC}_{\KE}(t, \qNewSecret, \qSend, \qRevSessionKey, \qRevLongTermKey, \qTest, \qRO)
		\leq
			\Adv^{\KESEC}_{\KE'}(t, \qNewSecret, \qSend, \qRevSessionKey, \qRevLongTermKey, \qTest, \qRO) \\
			+ \frac{2(12\qSend+\qRO)^2}{2^{\hashlen}}
			+ \frac{2\qRO^2}{2^{\hashlen}}
			+ \frac{8(\qRO+36\qSend)^2}{2^{\hashlen}}.
 	\end{align*}
%	Adversary $\advB$ has runtime approximately equal to $t_{\advA}$ and makes the same number of queries to each of its oracles in the $\KESEC$ game.
\end{theorem}

We establish this result via three modular steps in the indifferentiability framework introduced by Maurer, Renner, and Holenstein~\cite{TCC:MauRenHol04}.
More specifically we will leverage a recent generalization proposed by Bellare, Davis, and Günther (BDG)~\cite{EC:BelDavGun20}, which in particular formalizes indifferentiability for constructions of \emph{multiple} random oracles.

%We can also make a similar claim for read-only indifferentiability. The proof needs only two small adjustments: first, we require that the $\KE-SEC$ game restricts all of its queries to be within the working domain $\workDom$, and second, we have $\advB$ initialize the simulator's state before running $\advA$. 
% Since the simulated game of $\advD$ restricts all its queries to be within $\workDom$, the restriction of the $\PrivO$ oracle does not affect the honesty of the simulation, and the same bound can be shown. 

\subsection{Indifferentiability for the TLS~1.3 Key Schedule in Three Steps}
We move from the left of Figure~\ref{fig:tls-handshake} to the right via three steps.
Each step introduces a new variant of the TLS~1.3 protocol with a different set of random oracles by changing how we implement $\abstractHash$, $\abstractMAC$, $\abstractExpand$, $\abstractExtract$, and eventually the whole key schedule.
Then we view the prior implementations of these functions as constructions of new, independent random oracles.
We prove security for each intermediate protocol in two parts: first, we bound the indifferentiability advantage against that step's construction; then we apply the indifferentiability composition theorem based on~\cite{TCC:MauRenHol04}
	(cf.\ \iffull Section~\else Appendix~\fi\ref{app:indiff-background}, Theorem~\ref{thm:indiff-comp})
to bound the multi-stage key exchange ($\KESEC$) security of the new protocol.

We give a brief description of each step; all details and formal theorem statements and proofs can be found in 
	\iffull Sections~\else Appendices~\fi \ref{sec:domsep}, \ref{sec:hmac}, and \ref{sec:many-ros}, respectively.

\begin{description}
	\setlength{\itemsep}{0.5em}
	
	\item[From one random oracle to two.]
	TLS~1.3 calls its hash function~$\abstractHash$, which we initially model as random oracle~$\ROhash$, for two purposes:
	to hash protocol transcripts, and as a component of $\abstractMAC$, $\abstractExtract$, and $\abstractExpand$ which are implemented using $\HMAC[\abstractHash]$.
	Our eventual key exchange proof needs to make full use of the random oracle model for the latter category of hashes, but we require only collision resistance for transcript hashes. 
	
	Our first intermediate handshake variant, $\KE_1$, replaces $\abstractHash$ with two new functions: $\Thash$ for hashing transcripts, and $\Chash$ for use within $\abstractMAC$, $\abstractExtract$, or $\abstractExpand$.
	While $\KE$ uses the same random oracle $\ROhash$ to implement $\Thash$ and $\Chash$, the $\KE_1$ protocol instead uses two independent random oracles $\ROthash$ and $\ROhmac$.
	To accomplish this without loss in $\KESEC$ security, we exploit some possibly unintentional domain separation in how inputs to these functions are formatted in TLS~1.3 to define a so-called \emph{cloning functor}, following BDG~\cite{EC:BelDavGun20}.
	%
	Effectively, we partition the domain~$\bits^*$ of~$\ROhash$ into two sets $\Dom_{\Thash}$ and $\Dom_{\Chash}$ such that $\Dom_{\Thash}$ contains all valid transcripts and $\Dom_{\Chash}$ contains all possible inputs to $\abstractHash$ from $\HMAC$. 
	We then leverage Theorem~1 of~\cite{EC:BelDavGun20} that guarantees composition for any scheme that only queries $\ROchash$ within the set $\Dom_{\Chash}$ and $\ROthash$ within the set $\Dom_{\Thash}$.
	
	We defer details on the exact domain separation to 
		Appendix~\ref{app:domsep},
	but highlight that the PSK-only handshake with hash function \SHA{384} \emph{fails} to achieve this domain separation
	and consequently this proof step cannot be applied and leaves a gap for that configuration of TLS~1.3. 
	
	
	\item[From SHA to HMAC.]

	Our second variant protocol, $\KE_2$, rewrites the $\abstractMAC$ function. Instead of computing $\HMAC[\ROchash]$, $\abstractMAC$ now directly queries a new random oracle $\ROhmac \colon\allowbreak \bits^{\hashlen} \times \bits^* \to \bits^{\hashlen}$.
	Since $\ROchash$ was only called by $\abstractMAC$, we drop it from the protocol, but we do continue to use $\ROthash$,
	i.e., $\KE_2$ uses two random oracles: $\ROthash$ and $\ROhmac$.
	The security of this replacement follows directly from Theorem~4.3 of Dodis et al.~\cite{C:DRST12}, which proves the indifferentiability of $\HMAC$ with fixed-length keys.%
	\footnote{This requires PSKs to be elements of $\bits^{\hashlen}$, which is true of resumption keys but possibly not for out-of-band PSKs.}

	\item[From two random oracles to 12.]
	Finally, we apply a ``big'' indifferentiability step which yields $12$ independent random oracles and moves us to the right-hand side of Figure~\ref{fig:tls-handshake}.
	The $12$ ROs include
	the transcript-hash oracle~$\ROthash$ and 11 oracles that handle each key(-like) output in TLS~1.3's key derivation, named
		$\RObinder$,
		$\ROets$,
		$\ROeems$,
		$\ROchtk$,
		$\ROcfin$,
		$\ROshtk$,
		$\ROsfin$,
		$\ROcats$,
		$\ROsats$,
		$\ROems$, and
		$\ROrms$.
	(The signatures for these oracles are given in 
		Appendix~\ref{sec:many-ros}.)
	For this step, we view $\TLSKDF$ as a construction of $11$ random oracles from a single underlying oracle ($\ROhmac$).
	We then give our a simulator in pseudocode and prove the indifferentiability of $\TLSKDF$ with respect to this simulator.
	Our simulator uses look-up tables to efficiently identify intermediate values in the key schedule and consistently program the final keys and $\MAC$ tags.
\end{description}

Combining these three steps yields the result in Theorem~\ref{thm:full-ks-indiff}.
In the remainder of the paper, we can therefore now work with the right-hand side of Figure~\ref{fig:tls-handshake}, modeling $\abstractHash$ and the $\TLSKDF$ functions as $12$ independent random oracles.

\iffull
	\def\StepOneTitle{Step 1: Domain-separating the Transcript Hash}
\iffull
	\subsubsection{\StepOneTitle}
\else
	\subsection{\StepOneTitle}
\fi
\label{sec:domsep}
In the original TLS~1.3 PSK/PSK-(EC)DHE handshake, the hash function~$\abstractHash$ is used in two different ways.
It is used directly to compute digests of a \emph{transcript} and it is used as a \emph{component} of $\abstractMAC$, $\abstractExtract$, and $\abstractExpand$.
We will argue now that these two uses are entirely distinct, and we can accordingly write two functions~$\Thash$ and $\Chash$ in place of the two uses of~$\abstractHash$,
and, following BDG~\cite{EC:BelDavGun20}, go from modeling~$\abstractHash$ as one random oracle to modeling $\Thash$ and $\Chash$ as two independent random oracles.

We will refer to our two new random oracles as $\ROthash$ (modeling the \emph{transcript hash} function~$\Thash$) and $\ROchash$ (modeling the \emph{component hash} $\Chash$).
Because TLS~1.3 fully specifies the inputs to each hash function call,
we can show that in PSK-(EC)DHE mode and in PSK-only mode when $\hashlen=256$, TLS~1.3 will never call the same string as an input to both $\Thash$ and $\Chash$. 
This is due to some fortunate coincidences of formatting in the standard, which we describe in full in Appendix~\ref{app:domsep}. 
We can therefore define two disjoint sets $\Dom_{\Thash}$ and $\Dom_{\Chash}$ such that $\Dom_{\Thash} \cup \Dom_{\Thash} = \bits^*$ split up $\hash$'s domain. 

If we define the domain of $\ROthash$ to be $\Dom_{\Thash}$ and the domain of $\ROchash$ to be $\Dom_{\Chash}$, we could prove indifferentiability using a construction called the \emph{identity (cloning) functor}~$\construct{I}$ from~\cite{EC:BelDavGun20}.
The identity functor constructs two or more random oracles $\RO_1, \RO_2,\ldots$ from $\ROhash$ by forwarding all $\RO_i$ queries to $\ROhash$ unchanged.
However, the definitions of sets $\Dom_{\Thash}$ and $\Dom_{\Chash}$ are somewhat complex, especially in PSK-only mode.
We would instead prefer to define both $\ROthash$ and $\ROchash$ with domains $\bits^*$.
This would greatly simplify our later use of $\ROchash$ as a component of $\HMAC$.
Unfortunately, when the domains of $\ROthash$ and $\ROchash$ overlap, the identity functor is \emph{not} indifferentiable.
We can however still provide the desired result by turning to the read-only indifferentiability framework of Bellare, Davis, and Günther~\cite{EC:BelDavGun20}.

Read-only indifferentiability (a.k.a. $\rdindiff$) is similar to standard indifferentiability~\cite{TCC:MauRenHol04}.
One notable change (and the one we will leverage here) is that it is parameterized by a set~$\workDom$ called the ``working domain.''
The security game places a restriction on the $\PrivO$ oracle so that it only responds to queries within~$\workDom$.
Read-only indifferentiability supports a broader composition thoerem than Theorem~\ref{thm:indiff-comp}, which covers security games which call their random oracles only within the working domain.
BDG prove~\cite[Theorem~1]{EC:BelDavGun20}, which states that when $\workDom$ consists of disjoint sets like $\Dom_{\Thash}$ and $\Dom_{\Chash}$, the identity functor is read-only indifferentiable even when the full domains of $\ROthash$ and $\ROchash$ are not disjoint.
Furthermore, the read-only indifferentiability advantage is upper-bounded by $0$, and BDG give a simulator that runs in linear time on the length of its inputs and makes at most one query per execution. 
When we apply the read-only indifferentiability composition theorem, the adversary's runtime and query bounds will not increase.

We formalize this with a lemma:
\begin{lemma}%		Let $\KE$ be the TLS~1.3 key exchange protocol described in Figure~\ref{fig:tls13-psk}. Let $\hh:\bits^{*} \to \KEkeyspace$ be the random oracle used by the protocol. Also, let $\KE'$ be the key exchange protocol of Figure \TODO{?} which uses 2 random oracles $\Thash, \Chash: \bits^* \to bits^{\hashlen}$. Let $\Dom_{\Thash}$ and $\Dom_{\Chash}$ be two disjoint sets such that $\Dom_{\Thash}\cup \Dom_{\Chash} = \bits^*$. For any adversary $\advA$ attacking the $\KESEC$ security of $\KE$, let $t_\advA$ denote the runtime of $\advA$, and let $q_{\RO}$ and $q_{\Send}$ capture the number of queries $\advA$ makes to the $\RO$ and $\Send$ oracles respectively. Then there exists an adversary $\advB$ against the security of $\KE'$ such that
	%	\[ \genAdv{\KESEC}{\KE}{\advA} \leq \genAdv{\KESEC}{\KE'}{\advB}. \]
	%	Adversary $\advB$ has runtime approximately equal to $t_{\advA} + q_{\RO}$, and it makes the same number of queries to each of its oracles in the $\KESEC$ game. 
	
	\label{thm:ks-indiff-hop-1-comp}
	Let $\KE$ be the TLS~1.3 key exchange protocol of Theorem~\ref{thm:full-ks-indiff}.
	Let $\ROthash, \ROchash\colon\allowbreak \bits^* \to \bits^{\hashlen}$ be two random oracles, and
	let  $\KE_1$ be the protocol on the left-hand side of Figure~\ref{fig:tls-handshake}, where
	\begin{itemize}
		\item $\abstractHash := \ROthash$
		\item $\abstractMAC := \HMAC[\ROchash]$
	\end{itemize}
	and $\abstractExpand$ and $\abstractExtract$ are as in $\KE$ (using the new definition of $\abstractMAC$).
	Let $\Dom_{\Thash}$ and $\Dom_{\Chash}$ be two disjoint sets such that $\KE_1.\KERun$ only queries $\ROthash$, resp. $\ROchash$ in $\Dom_{\Thash}$, resp. $\Dom_{\Chash}$, and $\Dom_{\Thash}\cup \Dom_{\Chash} = \bits^*$.
	Furthermore, let $\Dom_{\Thash}$ have an efficient membership function.	
	
	Let $\advA$ be an adversary against the $\KESEC$ security of~$\KE$, running in time~$t_\advA$ and making $\qRO$ and $\qSend$ queries to its random oracle resp.\ $\Send$ oracle.
	Then there exists an adversary $\advB$ against the security of $\KE'$, such that 
	\[
	\genAdv{\KESEC}{\KE}{\advA}
	\leq \genAdv{\KESEC}{\KE_1}{\advB}.
	\]
	
	Adversary~$\advB$'s runtime is $\bigO(t_{\advA} + \qRO)$, and it makes the same number of queries to each of its oracles as~$\advA$ in the $\KESEC$ game. 
\end{lemma}
\begin{proof}
	The function space of $\KE$ is $\SSp = \AllFuncs(\bits^*, \bits^{\hashlen})$, and the function space of $\KE_1$ is $\ESp = \AllFuncs( \{\Thash, \Chash\} \times \bits^*, \bits^{\hashlen})$. 
	We can construct $\ESp$ from $\SSp$ via a construction called the ``identity functor'' defined by BDG~\cite{EC:BelDavGun20}.
	This construction is parameterized by a set $\workDom:= (\{\Thash \}\times \Dom_{\Thash}) \cup (\{\Chash \}\times \Dom_{\Chash})$.
	To answer any query $(i, s)$, the identity functor simply forwards $s$ to its own oracle, regardless of whether $i$ is $\Thash$ or $\Chash$.
	Because $\workDom$ is the union of two disjoint sets with efficient membership functions, the simulator $\Sim$ defined by BDG's Theorem $1$ has the property that for any distinguisher $\advD$,
	\[\genAdv{\rdindiff}{\construct{I}_\workDom, \workDom, \Sim}{\advD} = 0.\]
	$\Sim$ works by using the membership function of $\Dom_{\Thash}$ to check which of the two oracles is being simulated; then it forwards the query to the appropriate oracle.
	
	For this (or any) simulator, the composition theorem for read-only indifferentiability grants the existence of adversary $\advB$ and a distingisher $\advD$ such that
	
	\[
	\genAdv{\KESEC}{\KE}{\advA} \leq \genAdv{\KESEC}{\KE_1}{\advB} + \genAdv{\rdindiff}{\construct{I}_\workDom, \workDom, \Sim}{\advD}
	\leq \genAdv{\KESEC}{\KE_1}{\advB}.
	\]
	This composition theorem crucially rests on the fact that $\KE_1.\KERun$ queries $\ROthash$ and $\ROchash$ only within $\workDom$. The lemma follows.
	
	We require that $\Dom_{\Thash}$ and $\Dom_{\Chash}$ are disjoint sets. We define specific choices of $\Dom_{\Thash}$ and $\Dom_{\Chash}$ based on the low-level formatting of TLS~1.3 in Appendix~\ref{app:domsep}, and there we give detailed arguments that the sets are disjoint for 3 of 4 standardized settings of the PSK/PSK-(EC)DHE handshake. 
	
	In the fourth setting, PSK-only mode with hash function~$\SHA{384}$, there are no disjoint choices for $\Dom_{\Thash}$ and $\Dom_{\Chash}$ with efficient membership functions.
	This is due to a lack of careful domain separation of the hash function calls in TLS~1.3. 
	We therefore cannot apply this indifferentiability step for the PSK-only/$\SHA{384}$ handshake protocol.
	Any security proof of this handshake must either rely on stronger, possibly falsifiable abstractions in the random oracle model, or use a model $\SHA{384}$ as a single random oracle, with no guarantees of independence.
	We avoid the latter approach in order to maintain a modular and readable proof. 
	
	The second inequality follows from our choice of simulator and Theorem~1 of~\cite{EC:BelDavGun20}, which makes at most one query to its random oracle per execution.
	Their simulator, as mentioned above, must efficiently determine for every query $s$ whether to query $\ROthash$ or $\ROchash$. 
	This induces the requirement that $\Dom_{\Thash} \cup \Dom_{\Chash} = \bits^*$, so every possible query can be routed appropriately, and the requirement that $\Dom_{\Thash}$ has an efficient membership function so that the simulator is itself efficient.
	$\Dom_{\Thash}$ and $\Dom_{\Chash}$ satisfy these requirements thanks to the rules given in Appendix~\ref{app:domsep}.
\end{proof}

%\begin{lemma}\label{th-ks-indiff}
%	Let $\KEkeyspace$ be a set of strings, and let $\SSp$ be the space of all functions with domain $[1] \times \bits^*$ and range $\KEkeyspace$. Additionally, let $\Dom$ be a subset of $\bits^*$. Let $\ESp$ be the space of all functions with domain $(\{1\}\times \Dom) \cup (\{2\} \times (\bits \setminus \Dom))$ and range $\KEkeyspace$.
%	Given an oracle $\hh \in \SSp$, we define the following construction $\construct{F}: \SSp \to \ESp$
%	\[ \construct{F}[\hh](i,\Y) := \hh(\Y).\]
%	 For any adversary $\advD$ making $\qPriv$ queries to the $\PrivO$ oracle and $\qPub$ queries to the $\PubO$ oracle, there exists a simulator $\Sim$ with runtime \TODO{runtime}  \TODO{bound}such that
%	\[\genAdv{\indiff}{\construct{F}, \SSp, \ESp, \Sim}{\advD}\leq \frac{6q_{\PubO}^2}{|\KEkeyspace|} + \frac{8(q_{\text{max length}} \cdot q_{\PrivO})^2}{\KEkeyspace}.\]
%\end{lemma}
%\begin{proof}
%	The construct $\construct{F}$ is the identity functor defined by ~\cite{EC:BDG20}. Let the ``working domain''  $W$ be the set $\ESp.\Domain$. In that paper, we defined a working domain as one that separates domains when contains no two points $(i_1, \Y)$ and $(i_2, \Y)$ such that $i_1 \neq i_2$. For our choice of $W$, if $i_1 \neq i_2$, then for any two points $(i_1, \Y)$ and $(i_2, \Y')$, exactly one of $\Y$ and $\Y'$ may be an element of $\Dom$. Therefore $W$ separates domains, and the construct $\construct{F}$ is read-only indifferentiable over the working domain $W$ by Theorem 4.2 of~\cite{EC:BDG20}. We also note from that paper that read-only indifferentiability implies standard indifferentiability when the working domain equals $\ESp.\Domain$, as it does here. 
%\end{proof}

\def\StepTwoTitle{Step 2: Applying the Indifferentiability of HMAC}
\iffull
	\subsubsection{\StepTwoTitle}
\else
	 \subsection{\StepTwoTitle}
\fi

\label{sec:hmac}
Our next key exchange protocol, $\KE_2$, replaces the construction~$\HMAC[\Chash]$ with a single random oracle~$\ROhmac$ in the implementation of $\abstractMAC$ and by extension $\abstractExtract$ and $\abstractExpand$.
We rely on the proof of $\HMAC$'s indifferentiability by Dodis et al.~\cite[Theorem~3]{C:DRST12}.
As a prerequisite for this theorem, we need to restrict HMAC to keys of a fixed length less than the block length of the hash function ($512$~bits for $\SHA{256}$ and $1024$~bits for $\SHA{384}$).
This is consistent with $\HMAC$'s usage in TLS~1.3, where the keys are almost always of length~$\hashlen \in \{256, 384\}$.
The only exception is when pre-shared keys of another length are negotiated out-of-band; we exclude this case.
\begin{lemma}
	\label{thm:ks-indiff-hop-2-comp}
	Let $\ROthash, \ROchash \colon \bits^* \to \bits^{\hashlen}$ and $\ROhmac \colon \bits^{\hashlen} \times \bits^* \to \bits^{\hashlen}$ be random oracles.
	Let $\KE_1$ be the TLS~1.3 key exchange protocol described in Theorem~\ref{thm:ks-indiff-hop-1-comp} using random oracles~$\ROthash$ and~$\ROchash$.
	Let $\KE_2$ be the key exchange protocol given on the left-hand side of  Figure~\ref{fig:tls-handshake}, where
	\begin{itemize}
		\item $\abstractHash := \ROthash$
		\item $\abstractMAC := \ROhmac$
	\end{itemize}
	and $\abstractExtract$ and $\abstractExpand$ are defined as Section~\ref{sec:tls13-psk-protocol}.
	Let $\advA$ be an adversary against the $\KESEC$ security of~$\KE_1$, running in time~$t_\advA$ and making $\qRO$ and $\qSend$ queries to its random oracle resp.\ $\Send$ oracle.
	Then there exists an adversary $\advB$ against the security of $\KE_2$ such that 
	\[
	\genAdv{\KESEC}{\KE_1}{\advA}
	\leq \genAdv{\KESEC}{\KE_2}{\advB} + \frac{2(12\qSend+\qRO)^2}{2^{\hashlen}}.
	\]
	Adversary~$\advB$ has runtime $\bigO(t_{\advA} + \qRO)$ and makes the same number of queries to each of its oracles as~$\advA$ in the $\KESEC$ game.
\end{lemma}
\begin{proof}
	$\KE_1$ uses function space $\ESp$, defined in the proof of Lemma~\ref{thm:ks-indiff-hop-1-comp}, and $\KE_2$ uses function space $\ESp_2 = \AllFuncs((\{\Thash\} \times \bits^*) \cup (\{\HMAC\} \times \bits^{\hashlen} \times \bits^*), \bits^{\hashlen})$.
	The construction $\construct{C}$ of $\ESp_2$ from $\ESp$ simply forwards all queries to $\ROthash$. It answers $\ROhmac$ queries with $\HMAC [\ROchash ]$.
	
	For any simulator~$\Sim$, Theorem $5$ grants the existence of a distinguisher $\advD$ and an adversary $\advB$ such that 
	\[ \genAdv{\KESEC}{\KE_1}{\advA} \leq \genAdv{\KESEC}{\KE_2}{\advB} + \genAdv{\indiff}{\construct{C},\Sim}{\advD}. \]
	The distinguisher $\advD$ makes up to $12$ queries to $\PrivO$ for each $\Send$ query made by $\advA$, and makes one $\PubO$ query for each $\RO$ query of $\advA$. 
	
	We consider the simulator~$\Sim_2$ defined by Dodis et al. for~\cite[Theorem~4.3]{EPRINT:DRST13} (the full version of~\cite[Theorem~3]{C:DRST12}). 
	This simulator relies on the requirement that $\HMAC$ keys are a fixed length, and shorter than the block length of the underlying hash function.
	$\HMAC$ pads its keys with zero bits up to the block length, so each hash function call made by $\HMAC$ contains a segment containing the byte \texttt{0x36} for the first of the two calls and \texttt{0x5c} for the second.
	$\Sim_2$ uses this segment to identify whether a particular query is intended to simulate the first or second hash function call.
	It answers the ``first'' calls with random strings and logs these responses.
	Then it programs the ``second'' calls by using its stored intermediate values to find which $\ROhmac$ query should be simulated.
	We augment the simulator to forward all queries to $\ROthash$; this does not change its runtime or effectiveness.
	This simulator works perfectly unless there is a collision among the $2\qPriv+\qPub$ intermediate values, which Dodis et al. bound with a birthday bound.
	That theorem states that for a distinguisher $\advD$ making $12\qSend$ queries to $\PrivO$ and $\qRO$ queries to $\PubO$, 
	\[ \genAdv{\indiff}{\construct{C}, \Sim}{\advD} \leq \frac{2(12\qSend+\qRO)^2}{2^{\hashlen}}. \]
	The lemma follows.
\end{proof}

\def\StepThreeTitle{Step 3: Applying Indifferentiability to the TLS Key Schedule}
\iffull
	\subsubsection{\StepThreeTitle}
\else
	 \subsection{\StepThreeTitle}
\fi

\label{sec:many-ros}
In the last step, we move to the right-hand side of Figure~\ref{fig:tls-handshake} and introduce $11$ new independent random oracles to model the key schedule.
We start by rephrasing the TLS key schedule and message authentication codes as eleven functions $\TLSKDF_{\binder}$, \dots, $\TLSKDF_{\rms}$ as in Section~\ref{sec:tls13-psk-protocol}.
This abstraction does not change any of the operations performed by the key schedule; the $\TLSKDF$ functions simply rename the key derivation steps already performed by $\KE_2$. 
In our last key exchange protocol $\KE'$, we model each $\TLSKDF$ function as a independent random oracle. 
We name these oracles after the keys or values they derive:
{\allowdisplaybreaks
\begin{align*}
1.~ &\RO_{\binder}[\ROhmac]	&&\colon \bits^{\hashlen}  \times \bits^{\hashlen}\to \bits^{\hashlen} \\
2.~ &\RO_{\ets}[\ROhmac]	&&\colon \bits^{\hashlen} \times \bits^{\hashlen} \to \bits^{\hashlen} \\
3.~ &\RO_{\eems}[\ROhmac]	&&\colon \bits^{\hashlen} \times \bits^{\hashlen}  \to \bits^{\hashlen}\\
4.~ &\RO_{\chtk}[\ROhmac]	&&\colon \bits^{\hashlen}\times \G \times \bits^{\hashlen} \to \bits^{\hashlen + \ivlen} \\
5.~ &\RO_{\cfin}[\ROhmac]		&&\colon \bits^{\hashlen} \times \G \times \bits^{\hashlen} \times \bits^{\hashlen} \to \bits^{\hashlen} \\
6.~ &\RO_{\shtk}[\ROhmac]	&&\colon \bits^{\hashlen} \times \G \times \bits^{\hashlen} \to \bits^{\hashlen + \ivlen} \\
7.~ &\RO_{\sfin}[\ROhmac]		&&\colon \bits^{\hashlen} \times \G \times \bits^{\hashlen} \times \bits^{\hashlen} \to \bits^{\hashlen} \\
8.~ &\RO_{\cats}[\ROhmac]	&&\colon \bits^{\hashlen}\times \G \times \bits^{\hashlen} \to \bits^{\hashlen} \\
9.~ &\RO_{\sats}[\ROhmac] &&\colon \bits^{\hashlen} \times \G \times \bits^{\hashlen} \to \bits^{\hashlen} \\
10.~ &\RO_{\ems}[\ROhmac]	&&\colon \bits^{\hashlen} \times \G \times \bits^{\hashlen} \to \bits^{\hashlen} \\
11.~ &\RO_{\rms}[\ROhmac]	&&\colon \bits^{\hashlen} \times \G \times \bits^{\hashlen} \to \bits^{\hashlen}
\end{align*}
}
The $12^\text{th}$ random oracle is $\ROthash$, used to hash transcripts as in $\KE_1$ and $\KE_2$.


Now we can state Lemma~\ref{th-ks-indiff}.
\begin{lemma}\label{th-ks-indiff}
	Let $\KE_2$ be the key exchange protocol of Lemma~\ref{thm:ks-indiff-hop-2-comp}, and let $\KE'$ be the key exchange protocol of Theorem~\ref{thm:full-ks-indiff}. 
	
	For any adversary $\advA$ against the $\KESEC$ security of $\KE_2$, with runtime $t$ and making $\qRO$ random oracle queries and $\qSend$ queries to $\Send$, there exists adversary $\advB$ against the $\KESEC$ security of $\KE'$ such that
	\[
	\genAdv{\KESEC}{\KE_1}{\advA}
	\leq \genAdv{\KESEC}{\KE_2}{\advB} + \frac{2\qPub^2}{2^{\hashlen}} + \frac{8(\qPub+6\qPriv)^2}{2^{\hashlen}}.
	\]
	Adversary $\advB$ runs in time at most $t + \qRO t_{\G}$, where $t_{\G}$ is the time to perform one group operation in the Diffie--Hellman group $\G$. 
	It makes no more queries to each of the oracles in the $\KESEC$ game than does $\advA$.
\end{lemma}

\begin{proof}
	We view $\TLSKDF$ as defined in Section~\ref{sec:tls13-psk-protocol} as a construction of the function space $\ESp'$ of $\KE'$:
	the arity-$12$ function space whose first subspace is $\AllFuncs(\bits^*\cab \bits^\hashlen)$ and whose remaining 11 subspaces are the spaces of all functions with the domains and ranges specified in the above list. 
	This $\TLSKDF$ construction takes an oracle from $\ESp_2$, the function space of $\KS_2$. 
	
	As in the prior two steps, we consider a particular simulator~$\Sim$ (cf.\ Figure~\ref{fig-sim-ks-indiff}) and rely on Theorem $5$ for the existence of a distinguisher $\advD$ and an adversary $\advB$ such that 
	\[ \genAdv{\KESEC}{\KE_2}{\advA} \leq \genAdv{\KESEC}{\KE'}{\advB} + \genAdv{\indiff}{\construct{\TLSKDF},\Sim}{\advD}. \]
	The distinguisher $\advD$ will make no more than $12$ queries to $\PrivO$ for each $\Send$ query made by $\advA$ and one query to $\PubO$ per $\RO$ query.
	
	Via a sequence of code-based games, we will show that the indifferentiability advantage of any distinguisher $\advD$ making $\qPriv$ queries to the $\PrivO$ oracle and $\qPub$ queries to the $\PubO$ oracle is
	\[\genAdv{\indiff}{\construct{\TLSKDF}, \SSp, \ESp, \Sim}{\advD}\leq \frac{2\qPub^2}{2^{\hashlen}} + \frac{8(\qPub+6\qPriv)^2}{2^{\hashlen}}.\]
	We give fully specified pseudocode for each of our games. 
	
	First, we explain the high-level strategy of our simulator.
	Our simulator takes two inputs: an index $\simIndex \in \{\Thash, \HMAC \}$ and a string $\simInString \in \bits^*$. 
	When $\simIndex = \Thash$, the simulator simulates $\ROthash(\simInString)$ easily; it simply forwards the query to its own random oracle $\ROthash$.
	When $\simIndex = \HMAC$, the simulator will parse $\simInString$ into a key $\hmackey \in \bits^{\hashlen}$ and a context string $\hmactext \in \bits^*$ and simulate $\ROhmac(\hmackey,\hmactext)$. 
	This simulation should be compatible with a view of the random oracles $\RO_x$ as computing $\TLSKDF_x[\ROhmac]$. 
	
	Initially, $\Sim$ randomly samples the response $\simOut$ to any simulated $\ROhmac$ query from $\bits^{\hashlen}$.
	Repeated queries are cached in a table $\cachetable$.
	Next, $\Sim$ checks whether the query could be part of an attempt to compute $\TLSKDF_x[\Sim]$ for some $x$.
	If so, it may have to program its response for consistency with $\RO_x$, or it may store its response in a lookup table $\rlookuptable$ to enable future programming.
	
	The only values that need programming are the first-class keys and MAC values.
	These are all outputs of $\abstractExpand[\ROhmac]$. 
	$\Sim$ can tell if a particular $\ROhmac$ query is made by $\abstractExpand$ by checking its formatting.
	The inputs $\hmactext$ of all $\abstractExpand$'s queries in the key schedule start with $3$ bytes of fixed values and a label $\lbl$ between $8$ and $18$ bytes long that starts with the string \texttt{``tls13''}. They end with a $1$~byte counter that TLS~1.3 fixes to \texttt{0x01}.
	$\Sim$ pattern-matches this label to determine which key is being derived. 
	It has a subroutine $\lblT$ to translate the few labels which are used in the last derivation step for multiple keys.	
	
	Whenever $\simulator$ detects the label of an intermediate key derivation query like the $\abstractExpand$ calls used to compute $\es$, $\hs$, or $\ms$, it stores the response to this query in table $\rlookuptable$ under the name of the key in question.
	If $\advD$ computes $\TLSKDF$ honestly, these tables will allow the simulator to backtrack through the execution to identify all of the inputs to $\TLSKDF$. 
	Inputs to $\ROhmac$ queries made by $\HKDFExtr$ do not contain labels, so some tables contain multiple intermediate values.
	Even without labels, each intermediate value should only appear in one key derivation except in the unlikely event of a collision in $\ROhmac$.  
	
	\begin{figure}[tp]
		\begin{minipage}[t]{0.45\textwidth}
			\NewExperiment[$\Sim(\simIndex,\simInString)$]
			
			\begin{oracle}{$\Sim[\RO](\simIndex,\simInString)$}
				\item if $\cachetable[\simInString] \neq \bot$
				\item \quad then return $\cachetable[\simInString]$
				\item if $\simIndex=\Thash$ then return $\ROthash(K \| \Y)$
				\item[] \comment{If not, this query should simulate $\ROhmac$}
				\item $\hmackey$, $\hmactext \gets \simInString$
				\item[] \comment{Randomly sample a response}
				\item $\simOut \getsr \bits^{\hashlen}$
				
				%computing ES from PSK
				\item if $\hmactext = 0$ 
				\item \quad $\rlookuptable_{\psk}[\simOut] \gets \hmackey$
				
				%computing MS from DHS
				\item else if $\hmackey = 0$ 
				\item \quad $\rlookuptable_{\dhs}[\simOut] \gets \hmactext$ 
				
				% computing MAC tag from BFK, CFK, or SFK
				\item else if $\rlookuptable_{\bfk/\cfk/\sfk}[\hmackey]\neq \bot$
				
				% computing binder from BFK
				\item \quad $\es \gets \rlookuptable_{\es}[\rlookuptable_{\bk/\chts/\shts}[\hmackey]]$
				\item \quad $\psk \gets \rlookuptable_{\psk}[\es]$
				\item \quad if $\psk \neq \bot$ 
				\item \quad \quad $\simOut \gets \RObinder(\psk, \hmactext)$
				
				% computing finished msg from CFK or SFK
				\item \quad \quad $\hts \gets \rlookuptable_{\bk/\chts/\shts}[\hmackey]$
				\item \quad $(\kdflbl', \hs, \hash_2) \gets \rlookuptable_{\hs/\hashcontext}[\hts]$
				\item \quad $(\des, \dhe) \gets \rlookuptable_{\des/\dhe}[\hs]$
				\item \quad $\psk \gets \rlookuptable_{\psk}[\rlookuptable_{\es/\hs}[ \des]]$
				\item \quad if $\psk \neq \bot$ 
				\item \quad \quad $y \gets \RO_{\kdflbl'[1]}(\psk, \dhe,\hash_2, \hmactext)[	\lblT(\kdflbl)]$
				
				%computing HS from DES + DHE
				\item else $\rlookuptable_{\des/\dhe}[\simOut] \gets(\hmackey, \hmactext)$
				
				\item if $(\hmactext[0\ldots2] \neq \hashlen)$
				\item[]\quad$\vee\hmactext[2] < 8) \vee(\hmactext[2] > 18)$
				\item[] \quad $\vee(\hmactext[3\ldots9] \neq \mathsf{"tls13 "})$
				\item[]\quad$\vee (\hmactext[|\hmactext|-1] \neq  1)$ 
				\item[] \quad \comment{This query does not match $\HKDFExpnd$ formatting.}
				
				\item \qquad $\cachetable[\simInString] \gets \simOut$
				\item \qquad return $\simOut$
				
				\item[] \comment{Parse the $\abstractExpand$ formatting to find the label.} 
				\item $\kdflbllen \gets \hmactext[2]$
				\item $\kdflbl \gets \hmactext[3\ldots (3+\kdflbllen)]$
				\item $\hashcontext \gets \hmactext[(3+\kdflbllen)\ldots|\hmactext|]$
				\item[] \ldots \comment{continued in next column}
			\end{oracle}
			
			
		\end{minipage}
		\begin{minipage}[t]{0.54\textwidth}
			\ExptSepSpace
			\begin{oracle}{$\Sim[\RO](\simIndex,\simInString)$\comment{continued}}
				%query computes BK from ES
				\item if $\kdflbl = \kdflbl_{\binder}$ and $\hashcontext = \hash(\emptymessage)$
				\item \quad  $\rlookuptable_{\es}[\simOut] \gets \hmackey$
				
				% query computes dES or dHS from ES or HS
				\item else if $\kdflbl = \kdflbl_{\des/\dhs}$ and $\hashcontext = \hash(\emptymessage)$
				\item \quad $\rlookuptable_{\es/\hs}[\simOut] \gets \hmackey$
				
				% Computing CHTS or SHTS from HS
				\item  else if $\kdflbl \in \{\kdflbl_{\chts}, \kdflbl_{\shts}\}$
				\item \quad $\rlookuptable_{\hs/\hashcontext}[\simOut] \gets (\lblT(\kdflbl), \hmackey, \hashcontext)$
				
				% Computing ETS or EEMS from ES
				\item else if $\exists k \in \{\ets, \eems\}$ with  $\kdflbl  = \kdflbl_k$ and $\rlookuptable_{\psk}[\hmackey] \neq \bot$
				\item \quad $\simOut \gets \RO_{k}(\rlookuptable_{\psk}[\hmackey],\hashcontext)$
				
				% computing CATS, SATS, EMS or RMS from MS
				\item else if $\exists k \in \{\cats, \sats, \ems, \rms\}$ with  $\kdflbl  = \kdflbl_k$
				\item \quad $(\des, \dhe) \gets \rlookuptable_{\des/\dhe}[\rlookuptable_{\es/\hs} [ \rlookuptable_{\dhs}[\hmackey]]]$
				\item \quad $\psk \gets \rlookuptable_{\psk}[\rlookuptable_{\es/\hs}[ \des]]$
				\item \quad if $\psk\neq \bot$
				\item \quad \quad  $\simOut \gets \RO_{k}(\psk, \dhe, \hashcontext)$ 
				
				% Computing BFK, CFK, or SFK from BK, CHTS, or SHTS
				\item else if $\kdflbl = \kdflbl_{\fk}$ and $\hashcontext = \emptymessage$ 
				%\item[] \comment{This query must simulate either the binder key (case 1), or the client or server finished key (case 2)}
				\item \quad $\rlookuptable_{\bk/\chts/\shts}[\simOut] \gets \hmackey$
				
				% computing tkchs or tkshs (key or iv) from CHTS or SHTS
				\item else if $\kdflbl \in \{\texttt{"tls13 key"}, \texttt{ "tls13 iv"} \}$
				\item \quad and $\hashcontext = \hash(\emptymessage)$
				\item \quad $(\kdflbl', \hs, \hash_2) \gets \rlookuptable_{\hs/\hashcontext}[\hmackey]$
				\item \quad $(\des, \dhe) \gets \rlookuptable_{\des/\dhe}[\hs]$
				\item \quad $\psk \gets \rlookuptable_{\psk}[\rlookuptable_{\es/\hs}[ \des]]$
				\item \quad if $\psk \neq \bot$ 
				\item \quad \quad $y \gets \RO_{\kdflbl'[0]}(\psk, \dhe, \hash_2)[	\lblT(\kdflbl)]$
				\item []
				
				\item $\cachetable[\simInString] \gets \simOut$
				\item return $\simOut$
			\end{oracle}
			
			\ExptSepSpace	
			\begin{oracle}{Label translator $\lblT(\kdflbl)$}
				\item if $\kdflbl  = \kdflbl_{\chts}$
				\item \quad return $\chtk, \ClientFinished$
				\item if $\kdflbl  = \kdflbl_{\shts}$
				\item \quad return $\shtk, \ServerFinished$
				\item if $\kdflbl =\texttt{"tls13 key"}$
				\item \quad return $0$
				\item if $\kdflbl =\texttt{"tls13 iv"}$ 
				\item \quad return $1$
				\item return $\bot$
			\end{oracle}
		\end{minipage}
		\label{fig-sim-ks-indiff}
		\caption{Simulator $\Sim$ used in the proof of Lemma~\ref{th-ks-indiff}.}
	\end{figure}
	
	The first game in our sequence is $\Gm_0$~\, which is the ``ideal world'' setting of the indifferentiability game.
	Here, $\PrivO$ queries are answered using a random function $\RO$ drawn from $\ESp$, and $\PubO$ queries are answered with $\Sim[\RO]$.
	
	In $\Gm_1$ (cf.\ Figure~\ref{fig:gm1-ks-indiff}), we set a bad flag $\bad_C$ and abort whenever $\Sim$ samples a random answer $\simOut$ that collides with the input or output of any previous simulator query. 
	We track these inputs and outputs in a list $L$. 
	For each new query, there are at most $2\qPub$ points to collide with.
	Since $\simOut$ is sampled uniformly from $\bits^{\hashlen}$, the probability of such a collision over all queries is at most $\frac{2\qPub^2}{2^{\hashlen}}$ by a birthday and union bound). Then 
	
	\[ |\Pr[\Gm_1] - \Pr[\Gm_0]| \leq \frac{2\qPub^2}{2^{\hashlen}}. \] 
	
	In $\Gm_2$ (Figure\ref{fig:gm23-ks-indiff}), the $\Finalize$ oracle computes $\TLSKDF[\ROhmac]$ on the input to every query to the $\PrivO$ oracle, using $\PubO$ as its hash function. It discards the results of this computation, so this change can affect the outcome of the game only if one of the additional $\PubO$ queries sets the $\bad_C$ flag. The $\TLSKDF$ function queries its oracle at most~$6$ times per execution, so there are no more than $6\qPriv$ new queries. There are now a total of $\qPub + 6\qPriv$ queries to $\PubO$, so the probability that $\bad_C$ is set increases by another birthday bound.
	\[ |\Pr[\Gm_2] - \Pr[\Gm_1]| \leq \frac{2(\qPub+6\qPriv)^2}{2^{\hashlen}}. \] 
	
	
	The next step is the most subtle. 
	In $\Gm_3$ (Figure~\ref{fig:gm23-ks-indiff}), we move the new computations of $\TLSKDF$ from the $\Finalize$ oracle into $\PrivO$.
	When $\PrivO$ is called with index $\simIndex$ and input $X$, it still returns $\RO_{\simIndex}(X)$. 
	First, however, it computes $\TLSKDF_{\simIndex}[\PubO](X)$. 
	It discards the result of this computation, so the behavior of the $\PrivO$ oracle does not change in the adversary's view.
	
	However, queries to $\PrivO$ now run the simulator $\Sim$.
	They can update its state and set the global $\bad_C$ flag.
	This has two consequences.
	First, the changed order of $\PubO$ queries may cause $\bad_C$ to be set in $\Gm_3$ when it was not set in $\Gm_2$, or vice versa. 
	Second, queries to $\PrivO$ in $\Gm_3$ can add entries to the reverse lookup table $\rlookuptable$.
	These new entries can be used to satisfy the conditions the simulator uses to check if a full execution of $\TLSKDF$ has been completed.
	Then the simulator in $\Gm_3$ may program responses that were not programmed in $\Gm_2$. 
	
	We claim that despite the changed order of the queries, $\Gm_3$ and $\Gm_2$ behave identically in the adversary's view except when one of them would set the $\bad_C$ flag, assuming that the same random coins are used in both games.
	Let $E$ denote the event that $\bad_C$ is set either when $\advA$ plays $\Gm_2$ or when $\advA$ plays $\Gm_3$.
	Differences between the two games about when this flag is set are obviously irrelevant unless event $E$ occurs. 
	
	The argument that $\PubO$ responses are identical in both games except when event $E$ occurs is more subtle.
	Assume event $E$ does not occur.
	There must be a first adversarial query to $\PubO$ that gives different responses in $\Gm_3$ and $\Gm_2$, all oracles behave identically in both games.
	We name this query $Q$.
	Both games sample the same random responses, so query $Q$ has its response programmed by the simulator in at least one of the two games.
	
	The simulator decides whether to program based on the entries of reverse lookup table $\rlookuptable$, so we consider the differences in this table between our two games.
	Let $\rlookuptable_2$ be the table in $\Gm_2$ at the time when Query $Q$ is made, and let $\rlookuptable_3$ be the table at the same point in $\Gm_4$.
	Entries in the reverse lookup table are indexed by randomly sampled values $\simOut$, so they cannot be overwritten by later queries unless event $E$ occurs.
	Furthermore, until query $Q$ is made, every $\PubO$ query in $\Gm_2$ that updates $\rlookuptable$ gives the identical response in $\Gm_3$, so every entry in $\rlookuptable_2$ is also an entry in $\rlookuptable_3$.
	Therefore any query which is programmed in $\Gm_2$, up to and including query $Q$, will be programmed to the same response in $\Gm_3$.
	The contrapositive statement says that any response which is randomly sampled in $\Gm_3$ will be also be randomly sampled in $\Gm_2$.
	
	It follows that query $Q$ must have a randomly sampled response in $\Gm_2$ but be programmed in $\Gm_3$.
	There must exist a sequence of entries in $\rlookuptable_3$ that correspond to a full execution of $\TLSKDF[\PubO]$ on some input. 
	We name the queries that created these entries $Q_1, \ldots, Q_i$. 
	In each execution, our simulator either stores an entry in $\rlookuptable$, or it programs the response $\simOut$, never both.
	Therefore queries $Q_1, \ldots Q_i$ have randomly sampled responses.
	By the definition of $\TLSKDF$, the output of each query $Q_j$ is contained in the input of the next query $Q_{j+1}$.
	The output of $Q_i$ is contained in the input of $Q$, so we identify query $Q$ with $Q_{i+1}$.
	
	In $\Gm_2$, one of the entries in the sequence is not present in $\rlookuptable_2$. 
	Therefore one of the queries $Q_1, \ldots, Q_i$ is not made before query $Q$ in $\Gm_2$.
	This query, $Q_j$ must have been one of the $\Finalize$ queries of $\Gm_2$ that were moved earlier in $\Gm_3$.
	It will therefore be made in $\Finalize$, after all of the other queries, including $Q_{j+1}$.
	The randomly sampled output of $Q_j$ will collide with the input of earlier query $Q_{j+1}$, setting $\bad_C$ and causing event $E$ to occur.
	
	The difference in advantage in $\Gm_3$ and $\Gm_2$ is therefore bounded by the probability of event $E$.
	Both games make $\qPub+6\qPriv$ queries to $\PubO$, each of which sets $\bad_C$ is set with probability at most $\frac{2(\qPub+6\qPriv)}{2^{\hashlen}}$. 
	By a union bound,
	\[|\Pr[\Gm_3] - \Pr[\Gm_2]| \leq \frac{4(\qPub+6\qPriv)^2}{2^{\hashlen}}.\]
	
	%%% OLD DRAFT
	%In case $1$, one of the if-conditions in lines $13$, $31$, $36$, or $45$ of our original simulator (cf.\ Figure~\ref{fig-sim-ks-indiff}) must have been satisfied in $\Gm_4$. 
	%In $\Gm_2$, none of these four if-conditions are satisfied when Query $1$ is made.
	%For any of the four conditions to be satisfied in $\Gm_4$, there must exist a sequence of table entries in $\rlookuptable$ such that the contents of each entry are the index of the next entry. 
	%The index of the first entry is $\hmackey_1$, and the contents of the last entry are $\psk_1 \neq \bot$. 
	%In $\Gm_2$, at least one of the entries in this sequence is not present in $\rlookuptable$.
	%That means that  
	%The last query in the sequence must have output either $\hmackey_1$ or $\hmactext_1$. 
	%We break the logic down by each of the four possible conditions which could be satisfied in $\Gm_4$, starting with line $31$.
	%If $\rlookuptable_{\psk}[\hmackey^*] = \bot$ in $\Gm_2$ and $\rlookuptable_{\psk}[\hmackey^*]\neq \bot$ in $\Gm_4$, then some query 
	%$\hmackey^*$ must have been a randomly sampled response to an earlier $\PubO$ query.
	%However, this earlier query took place only in $\Gm_4$, meaning it must have been made by the $\PrivO$ oracle and not the adversary.
	%In $\Gm_2$, the same query would be made in the $\Finalize$ oracle at the end of the game.
	%This means that a query would randomly sample the response $\hmackey*$ \emph{after} the earlier query $\PubO(\hmackey^*, \hmactext^*)$ used the same value as an input. 
	%The later query will set $\bad_C$, and our claim holds in this case.
	%
	%Each of the other three conditions also requires that an entry in $\rlookuptable_{\psk}$ is set to $\bot$ in $\Gm_2$ but not $\Gm_4$. 
	%In these three cases, however, the entry in $\rlookuptable$ may have index $\bot$ in $\Gm_2$ because an entry in another reverse lookup table was set to $\bot$ in $\Gm_2$ but not $\Gm_4$.
	%Regardless of which table contains this entry, the value must have been sampled randomly by an earlier $\PubO$ query, one which takes place in $\PrivO$ in $\Gm_4$ and in $\Finalize$ in $\Gm_2$.
	%Let this be Query $1$.
	%Let $Z$ be the output of this Query $1$ (the index of the entry).
	%
	%There are two situations:
	%either $Z$ contains one or both values $\hmackey*$ and $\hmactext*$, or $Z$ itself is the contents of a $\rlookuptable$ entry which is identical in $\Gm_2$ and $\Gm_4$.
	%In the former case, $Z$ will collide with $\hmackey^*$ or $\hmactext^*$ in the $\Finalize$ oracle of $\Gm_2$, and the $\bad_C$ flag will be set.
	%In the latter case, we give the index of the entry containing $Z$ the name $Z'$
	%Unlike $Z$, $Z'$ is defined in both $\Gm_2$ and $\Gm_4$.
	%$Z'$ is added to $\rlookuptable$ in both $\Gm_2$ and $\Gm_4$ before $\Finalize$, so it is an output of some adversarial query Query $2$.
	%Note that Query $2$ must have occurred before Query $1$; otherwise the entry with index $Z$ would not exist.
	%The input of Query $2$ is $Z$.
	%Then when Query $1$ occurs in the $\Finalize$ oracle of $\Gm_2$ and samples response $Z$, it will collide with the logged input of Query $2$ and set the $\bad_C$ flag.
	%
	%Therefore any time $\PrivO$ would give different responses in $\Gm_4$ than $\Gm_2$, $\Gm_2$ would have set $\bad_C$. It follows that 
	%
	%\[|\Pr[Gm_4] - \Pr[Gm_2]| \leq Pr[\Gm_2\text{ sets }\bad_C] \leq \frac{2(\qPub+6\qPriv)^2}{2^{\hashlen}}.\]
	
	%%%% OLDEST DRAFT
	
	%If this condition is true in $\Gm_2$ and false in $\Gm_4$, then $\rlookuptable_{\psk}[\hmackey^*]$
	%
	%Therefore there is some entry $\rlookuptable_{\psk}[\simOut^*]$ which contains $\bot$ in $\Gm_2$ but does not contain $\bot$ in $\Gm_4$. 
	%This entry must have been created by one of the $\PubO$ queries we moved from $\Finalize$ to $\PrivO$; it was not created by an adversarial $\PubO$ query or it would have created the entry in both $\Gm_4$ and $\Gm_2$.
	%$\simOut^*$ was a random response sampled by this query; since the $\PrivO$ oracle discards the output of $\TLSKDF$, it was not returned to the adversary. it also appears as the contents of an entry in either table $\rlookuptable_{\es}$ or table $\rlookuptable_{\es/\hs}$. 
	%This means $\simOut^*$ was also an input to some $\PubO$ query
	%
	%
	%
	%We claim that if this events. Let $\PubO(K, \Y)$ be the first query which gives a different response in $\Gm_4$ than in $\Gm_3$. This means that in $\Gm_3$, the response to $\PubO(K, \Y)$ was uniformly distributed, but in $\Gm_4$, the query triggered an if-condition at one of lines $29$, $35$, $41$, $46$, or $53$. 
	%
	%Although the exact if-conditions differ, they all require that for some index $j$ and some tuple $t$, $T_j[t] \neq \bot$ and $K \in t$. For $K$ to be stored in a table $T$, some earlier query $\PubO(K', \Y')$ must have randomly sampled $K$ as its response. In $\Gm_3$, where $\PubO(K, \Y)$ is not programmed, the query $\PubO(K', \Y')$ did not occur. The adversary's view is the same in both games until $\PubO(K, \Y)$ occurs, so the order of its queries do not change. Then the query $\PubO(K', \Y')$ was not made by the adversary; instead, it must be one of the queries we moved from $\Finalize$ to $\PrivO$ between games $\Gm_3$ and $\Gm_4$. What happens in $\Gm_3$, when $\PubO(K', \Y')$ is queried in $\Finalize$? There are two possibilities: either it is answered randomly as in $\Gm_4$, or it is programmed. In the former case, the query returns $K$, which is in $L$ thanks to $\PubO(K, \Y)$. This causes a collision and $\bad_C$ is set. In the latter case, $\PubO(K', \Y')$ was itself programmed in $\Gm_3$; its response is therefore different from $K$. We claim that a collision must also have occured in this case.
	%
	%Since $\PubO(K', \Y')$ occurs in $\Finalize$ in $\Gm_3$, it arises during an honest computation of $\TLSKDF$. Since it is not programmed in $\Gm_4$, it is not the last query in this computation; its output represents an intermediate value. However, that it is programmed means that this query is also the final query in a different, possibly adversarial computation of $\TLSKDF$. The two computations have distinct inputs (otherwise the queries would be cached), so in order for them to contain identical queries, there must be a collision in the outputs of earlier queries. \hd{this logic is very unsound.} This earlier collision would set $\bad_C$. 
	%
	%The changed state of the simulator also affects whether the $\bad_C$ flag will be set in $\Gm_4$. Collisions among outputs are no more or less likely depending on the order of the queries. However, in $\Gm_3$ we also set the $\bad_C$ flag when an output from $\PubO$ collides with a prior input by the adversary. This captures the possibility that the adversary guesses an intermediate state of the KDF. Because the $\PrivO$ oracle also queries $\PubO$, the $\bad_C$ flag now may not be set in the event of such a guess. This decreases $\Pr[\Gm_4]$ by an amount no greater than $\Pr[\Gm_3 \text{sets} \bad_C]$.
	%
	%We know that $\Pr[\Gm_4 \text{ sets } \bad_C] \leq \Pr[\Gm_3 \text{ sets } \bad_C] \leq\frac{2\qPub^2}{2^{\hashlen}} + \frac{2(6 \qPriv)^2}{2^{\hashlen}}$, so we have
	%\[|\Pr[Gm_4] - \Pr[Gm_3]| \leq Pr[\Gm_3\text{ sets }\bad_C] \leq \frac{2(\qPub+6\qPriv)^2}{2^{\hashlen}}.\]
	Pseudocode for the last three games is given in Figure~\ref{fig:gm456-ks-indiff}.
	Now we adjust $\PrivO$ in $\Gm_4$ to return the result of $\construct{C}[\PubO]$ instead of querying $\RO$. 
	Unless $\bad_C$ is set, $\TLSKDF[\PubO](\roIndex, \X) = \RO_{\roIndex}( \X)$. 
	The function $\TLSKDF$ makes sequential queries to $\PubO$ that are properly formatted, so our $\Sim$ will program the last query in the sequence for consistency with the appropriate $\RO$. 
	This programming occurs every time $\TLSKDF[\PubO]$ is called, unless the last query is a repeated query.
	In that case, it will be answered using table $\cachetable$ instead of $\RO$.
	However, if the queries in the sequence occur out of order, they will always cause $\bad_C$ to be set because the output of a later query will match the input to an earlier query.
	Then the adversary wins in $\Gm_4$ with the same likelihood as $\Gm_3$, unless $\bad_C$ is set. 
	If $\bad_C$ is set, both games have a win probability of $0$ thanks to the check in the $\Finalize$ oracle, so
	\[\Pr[\Gm_4] = \Pr[\Gm_3].\]
	
	Starting with $\Gm_5$, we stop returning $0$ in $\Finalize$ when $\bad_C$ is set. This increases the win probability by at most $\Pr[\Gm_4 \text{ sets }\bad_C] \leq \frac{2(\qPub+6\qPriv)^2}{2^{\hashlen}}$, by the same birthday and union bounds over the $\qPub + 6\qPriv$ queries to $\PubO$. 
	\[|\Pr[\Gm_5] - \Pr[\Gm_4]| \leq \frac{2(\qPub+6\qPriv)^2}{2^{\hashlen}}.\]
	
	From $\Gm_4$ onward, all queries to $\ROhmac$ are made by $\simulator$. In $\Gm_6$, therefore, we can inline the lazily sampled $\ROhmac$ oracle as part of the simulator. 
	Repeated queries to $\simulator$ are cached, so the random oracle does not need to maintain its own lookup table. 
	Now all responses from $\PubO$ are randomly sampled from $\bits^{\hashlen}$, regardless of the contents of table $\rlookuptable$. 
	The table and the conditional statements used to maintain it are now redundant bookkeeping, as is the unused $\bad_C$ flag after $\Gm_5$. 
	We eliminate all of this code from $\Gm_6$ without detection by the adversary.
	Then 
	\[ \Pr[\Gm_6] = \Pr[\Gm_5]. \]
	
	The remaining code of $\simulator$ just implements random oracles $\ROhmac$ and $\ROthash$.
	Consequently $\Gm_6$ is identical to the ideal indifferentiability game for the $\TLSKDF$ construction. 
	Collecting bounds proves the theorem.
	
	
	%%%%%%%%%%%%%%%%%%%%%%%%%%%%%%%%%%%%%%%%%%%%%%%%%%%%%
	
	%In $\Gm_1$, we introduce a private random oracle $\hh'$ with domain $\bits^*$ and range $\KEkeyspace$. The simulator computes the output $y$ by calling $\hh'((\HKDFExtr, \HKDFExpnd)||\Y)$ instead of randomly sampling $y$. Using a random oracle means that repeated queries will receive identical responses; however this was already guaranteed since $\PubO$ caches its responses in table $M$. Then $\Pr[\Gm_0] = \Pr[\Gm_1].$
	%
	%Next we exclude collisions among the simulator's responses. Starting with game $\Gm_2$, the simulator stores all its queries and responses in a list $L$. If it generates a response $y$ that collides with a prior response or with part of a prior query, we set a flag called $\bad_C$. In $\Gm_3$, the Finalize oracle returns $0$ if the $\bad_C$ flag has been set. By caching queries as well as responses, we are also excluding the possibility that the adversary can compute $\TLSKDF$ by guessing an intermediate value. Accurate guesses will cause the $\bad_C$ flag to be set. 
	%We have that $\Pr[\Gm_2]=\Pr[\Gm_1]$ because the list $L$ is fully internal to the game,
	%and we have $\Pr[\Gm_3] \replace{\leq}{\geq} \Pr[\Gm_2]$ because the Finalize check strictly decreases the probability that the game returns $1$ and no other oracle's behavior is changed.
	%
	%In game $\Gm_4$, we execute the construct $\construct{C}$ in the $\PrivO$ oracle using the simulator. The calls to $\simulator$ from $\PrivO$ use a separate, independent state from those in $\PubO$. Therefore repeated queries from $\PrivO$ and $\PubO$ do not cache their responses, and prior queries from $\PrivO$ do not influence programming in $\PubO$. Simulator calls made by $\PubO$, update both states, so programming in $\PrivO$ is dependent on prior queries from $\PubO$. The responses of the $\PrivO$ oracle do not change. For this reason, the adversary's view of $\PrivO$ and $\PubO$ does not change between $\Gm_3$ and $\Gm_4$. However, the simulator calls in $\PrivO$ use the same list $L$ to track collisions as those in $\PubO$, so the additional queries increase the chance that the $\bad_C$ flag will be set. Therefore $\Pr[\Gm_4]\leq \Pr[\Gm_3]$. 
	%
	%In game $\Gm_5$, $\Sim$ sets a new flag if it encounters an $\HKDFExpnd$ that should be programmed based on tables $T'$ but not based on tables $T$. In $\Gm_6$, we return $0$ in $\Finalize$ whenever this flag has been set. (This occurs only when the adversary has guessed an intermediate state of some $\TLSKDF$ without obtaining it from $\Sim$.) Also in $\Gm_6$, we do program when the flag is set, meaning that $\Sim$ effectively programs using tables $T'$. 
	%
	%In $\Gm_7$, $\Sim$ only uses tables $T'$ and list $M'$. This changes nothing: $\Sim$ was already programming based on tables $T'$ and any value cached in $M'$ would have been identically derived by $\Sim$ because $\Sim$ and $\Sim_2$ use the same private RO to sample random coins. (This could only be untrue if $\Sim_2$ sampled a random output for $y$ because it did not need to be programmed at the time, but later $\Sim$ wished to program $y$. Since $\Sim_2$ is only called in evaluating $\TLSKDF$, however, the last query in the chain is always programmed. Therefore $\Sim$ would need to program a query that was intermediate in a different $\TLSKDF$ execution. This would imply a collision occurred and the $\bad_C$ flag would have been set.) $\Sim$ does still maintain table $T$ in order to set the bad flag from $\Gm_5$.
	%
	%In $\Gm_8$, we stop using a private random oracle and return to random sampling. Since all values are cached between $\Sim$ and $\Sim_2$ by means of $M'$, it does not matter that repeated queries become inconsistent. 
	%
	%In $\Gm_9$, the Priv oracle returns the output of $\Sim_2$, not of the RO. We claim that the output is always identical unless one of our bad flags is set. (This is true since the last query in any $\TLSKDF$ execution is always programmed except in the event of a collision.) From this point forward, the random oracle $\RO$ is not accessible to the adversary. Additionally, no query to $\RO$ is repeated by $\Sim$ or $\Sim_2$ except when a collision flag would be set. Therefore in $\Gm_{10}$ we replace $\RO$ with random sampling. 
	%
	%In $\Gm_{11}$, we stop returning $0$ when the $\bad$ flags have been set. We bound the probability of this event using a birthday bound for collisions and a union bound for guesses. At this point, the checks for programming the queries and setting the bad flags are redundant, so we remove them in $\Gm_{12}$ along with the unused $T$ and $T'$ tables. The remaining code in both $\Sim$ and $\Sim_2$ is just that of a random oracle. Therefore we are now in the real world.
	
\end{proof}

We have now established that in order to give a (tight) security proof for TLS~1.3 PSK-only and PSK-(EC)DHE, it suffices to prove (tight) security of the protocol on the right-hand side of Figure~\ref{fig:tls-handshake}. 
%\tj{With or without encryption? Swap Sections \ref{sec:ks-indiff} and \ref{sec:modularizing}?}

%%% Local Variables:
%%% mode: latex
%%% TeX-master: "main"
%%% End:


\begin{figure}[tp]
	\begin{minipage}[t]{0.48\textwidth}
		\NewExperiment[Game $\Gm_0$]
		
		\begin{algorithm}{$\Initialize()$}
			\item $b \gets 0$
			\item $\RO \getsr \ESp$
			\item $\state \getsr \emptystring$
		\end{algorithm}
			\ExptSepSpace
			\begin{oracle}{$\Sim(\simIndex,\simInString,\state)$}
				\item if $\simIndex=\Thash$ then return $\ROthash, (\simInString)$
				\item $\rlookuptable, \cachetable \gets \state$
				\item if $\cachetable[\simInString] \neq \bot$
				\item \quad then return $\cachetable[\simInString]$
				\item $\hmackey,\hmactext \gets \simInString$
				\item $\simOut \gets \simulator[\RO](\hmackey, \hmactext,\rlookuptable)$
				\item $\cachetable[\simInString] \gets \simOut$
				\item return $y$
			\end{oracle}
		\ExptSepSpace
		\begin{oracle}{$\Sim[\RO](\hmackey, \hmactext, \rlookuptable)$}
		\item[] \comment{Randomly sample a response}
		\item $\simOut \getsr \bits^{\hashlen}$
		
		%computing ES from PSK
		\item if $\hmactext = 0$ 
		\item \quad $\rlookuptable_{\psk}[\simOut] \gets \hmackey$
		
		%computing MS from DHS
		\item else if $\hmackey = 0$ 
		\item \quad $\rlookuptable_{\dhs}[\simOut] \gets \hmactext$ 
		
		% computing MAC tag from BFK, CFK, or SFK
		\item else if $\rlookuptable_{\bfk/\cfk/\sfk}[\hmackey]\neq \bot$
		
		% computing binder from BFK
			\item \quad $\es \gets \rlookuptable_{\es}[\rlookuptable_{\bk/\chts/\shts}[\hmackey]]$
		\item \quad $\psk \gets \rlookuptable_{\psk}[\es]$
		\item \quad if $\psk \neq \bot$ 
		\item \quad \quad $\simOut \gets \RObinder(\psk, \hmactext)$
		
		% computing finished msg from CFK or SFK
		\item \quad \quad $\hts \gets \rlookuptable_{\bk/\chts/\shts}[\hmackey]$
		\item \quad $(\kdflbl', \hs, \hash_2) \gets \rlookuptable_{\hs/\hashcontext}[\hts]$
		\item \quad $(\des, \dhe) \gets \rlookuptable_{\des/\dhe}[\hs]$
		\item \quad $\psk \gets \rlookuptable_{\psk}[\rlookuptable_{\es/\hs}[ \des]]$
		\item \quad if $\psk \neq \bot$ 
		\item \quad \quad $y \gets \RO_{\kdflbl'[1]}(\psk, \dhe,  \hash_2, \hmactext)[	\lblT(\kdflbl)]$
		
		%computing HS from DES + DHE
		\item else $\rlookuptable_{\des/\dhe}[\simOut] \gets(\hmackey, \hmactext)$
		
		\item if $(\hmactext[0\ldots2] \neq \hashlen)$
		\item[]\quad$\vee\hmactext[2] < 8) \vee(\hmactext[2] > 18)$
		\item[] \quad $\vee(\hmactext[3\ldots9] \neq \mathsf{"tls13 "})$
		\item[]\quad$\vee (\hmactext[|\hmactext|-1] \neq  1)$ 
		\item[] \quad \comment{This query does not match $\HKDFExpnd$ formatting.}
		
		\item \qquad return $\simOut$
		
		\item[] \comment{Parse the $\abstractExpand$ formatting to find the label.} 
		\item $\kdflbllen \gets \hmactext[2]$
		\item $\kdflbl \gets \hmactext[3\ldots (3+\kdflbllen)]$
		\item $\hashcontext \gets \hmactext[(3+\kdflbllen)\ldots|\hmactext|]$
		\item[] \ldots \comment{continued in next column}
		\end{oracle}
		
			
		\end{minipage}
		\begin{minipage}[t]{0.49\textwidth}
			\ExptSepSpace
			\begin{oracle}{$\simulator[\RO](\hmackey, \hmactext, \rlookuptable)$\comment{...continued}}
			%query computes BK from ES
			\item if $\kdflbl = \kdflbl_{\binder}$ and $\hashcontext = \hash(\emptymessage)$
			\item \quad  $\rlookuptable_{\es}[\simOut] \gets \hmackey$
			
			% query computes dES or dHS from ES or HS
			\item else if $\kdflbl = \kdflbl_{\des/\dhs}$ and $\hashcontext = \hash(\emptymessage)$
			\item \quad $\rlookuptable_{\es/\hs}[\simOut] \gets \hmackey$
			
			% Computing CHTS or SHTS from HS
			\item  else if $\kdflbl \in \{\kdflbl_{\chts}, \kdflbl_{\shts}\}$
			\item \quad $\rlookuptable_{\hs/\hashcontext}[\simOut] \gets (\lblT(\kdflbl), \hmackey, \hashcontext)$
			
			% Computing ETS or EEMS from ES
			\item else if $\exists k \in \{\ets, \eems\}$ with  $\kdflbl  = \kdflbl_k$ and $\rlookuptable_{\psk}[\hmackey] \neq \bot$
			\item \quad $\simOut \gets \RO_{k}(\rlookuptable_{\psk}[\hmackey],\hashcontext)$
			
			% computing CATS, SATS, EMS or RMS from MS
			\item else if $\exists k \in \{\cats, \sats, \ems, \rms\}$ with  $\kdflbl  = \kdflbl_k$
			\item \quad $(\des, \dhe) \gets \rlookuptable_{\des/\dhe}[\rlookuptable_{\es/\hs}[ \rlookuptable_{\dhs}[\hmackey])]]$
			\item \quad $\psk \gets \rlookuptable_{\psk}[\rlookuptable_{\es/\hs}[ \des]]$
			\item \quad if $\psk\neq \bot$
			\item \quad \quad  $\simOut \gets \RO_{k}(\psk, \dhe, \hashcontext)$ 
			
			% Computing BFK, CFK, or SFK from BK, CHTS, or SHTS
			\item else if $\kdflbl = \kdflbl_{\fk}$ and $\hashcontext = \emptymessage$ 
			%\item[] \comment{This query must simulate either the binder key (case 1), or the client or server finished key (case 2)}
			\item \quad $\rlookuptable_{\bk/\chts/\shts}[\simOut] \gets \hmackey$
			
			% computing tkchs or tkshs (key or iv) from CHTS or SHTS
			\item else if $\kdflbl \in \{\texttt{"tls13 key"}, \texttt{ "tls13 iv"} \}$
			\item \quad and $\hashcontext = \hash(\emptymessage)$
			\item \quad $(\kdflbl', \hs, \hash_2) \gets \rlookuptable_{\hs/\hashcontext}[\hmackey]$
			\item \quad $(\des, \dhe) \gets \rlookuptable_{\des/\dhe}[\hs]$
			\item \quad $\psk \gets \rlookuptable_{\psk}[\rlookuptable_{\es/\hs}[ \des]]$
			\item \quad if $\psk \neq \bot$ 
			\item \quad \quad $y \gets \RO_{\kdflbl'[0]}(\psk, \dhe, \hash_2)[	\lblT(\kdflbl)]$
			\item []
			
			\item return $\simOut$
		\end{oracle}
			\ExptSepSpace
	\begin{algorithm}{$\PubO(\simIndex,\simInString)$}
		\item $(z,\state) \gets \Sim(\simIndex,\simInString,\state)$
		\item return $z$
	\end{algorithm}
	\ExptSepSpace
	\begin{algorithm}{$\PrivO(\roIndex,\X)$}
		\item return $\RO_{\roIndex}(\X)$
	\end{algorithm}	
			\ExptSepSpace
	\begin{algorithm}{$\Finalize(b')$}
		\item return $b'$
	\end{algorithm}
		\end{minipage}
		\label{fig:gm0-ks-indiff}
		\caption{Indiff game instantiated with simulator $\simulator$, also Game $\Gm_0$ in the proof of Lemma~\ref{th-ks-indiff}.}
	\end{figure}

\begin{figure}[tp]
	\begin{minipage}[t]{0.46\textwidth}
		\NewExperiment[Games $\Gm_1$]
	
			\begin{oracle}{$\Sim(\simIndex,\simInString,\state)$}
			\item if $\simIndex=\Thash$ then return $\ROthash(\simInString)$
			\item $\rlookuptable, \cachetable, \gamechange{L} \gets \state$
			\item if $\cachetable[\simInString] \neq \bot$
			\item \quad then return $\cachetable[\simInString]$
			\item $\hmackey,\hmactext \gets \simInString$
			\item $\simOut \gets \simulator[\RO](\hmackey, \hmactext,\rlookuptable, \gamechange{L})$
			\item $\cachetable[\simInString] \gets \simOut$
			\item \gamechange{$L \gets L \cup \{\simOut, \simInString\}$}
			\item return $y$
		\end{oracle}
	\end{minipage}
	\begin{minipage}[t]{0.49\textwidth}
			\ExptSepSpace
		\begin{oracle}{$\Sim[\RO](\hmackey, \hmactext, \rlookuptable,\gamechange{L})$}
			\item $\simOut \getsr \bits^{\hashlen}$
			\item \gamechange{if $\simOut \in L$ or $\exists t \in L$ such that $\simOut \in t$}
			\item \quad \gamechange{$\bad_C \gets \true$}
			\item[] \ldots
		\end{oracle}
	\ExptSepSpace
	\begin{algorithm}{$\Finalize(b')$}
		\item \gamechange{if $\bad_C$ then return $0$}
		\item return $b'$
	\end{algorithm}
	\end{minipage}
	\label{fig:gm1-ks-indiff}
	\caption{ Game $\Gm_1$ in the proof of Lemma~\ref{th-ks-indiff}. }
\end{figure}

\begin{figure}[tp]
	\begin{minipage}[t]{0.48\textwidth}
		\NewExperiment[Game $\Gm_2$]
		
			\begin{algorithm}{$\PrivO(\roIndex,\X)$}
				\item $Q \gets Q \bigcup \{(\roIndex,\X)\}$
				\item return $\RO_{\roIndex}(\X)$
			\end{algorithm}	
			\ExptSepSpace
			\begin{algorithm}{$\Finalize(b')$}
				\item \gamechange{for $(\roIndex,\X) \in Q$ do }
				\item \quad \gamechange{$z \gets\TLSKDF_{\roIndex}[\PubO](\X)$}
				\item if $\bad_C$ then return $0$
				\item return $b'$
			\end{algorithm}
	
	\end{minipage}
\vline
\hspace{.02\textwidth}
	\begin{minipage}[t]{0.49\textwidth}
		\NewExperiment[Game $\Gm_3$]
	
		\ExptSepSpace
	\begin{algorithm}{$\PrivO(\roIndex,\X)$}
		\item \gamechange{$z \gets\TLSKDF_{\roIndex}[\PubO](\X)$}
		\item return $\RO_{\roIndex}(\X)$
	\end{algorithm}	
	\ExptSepSpace
\begin{algorithm}{$\Finalize(b')$}
	\item if $\bad_C$ then return $0$
	\item return $b'$
\end{algorithm}
\end{minipage}
\label{fig:gm23-ks-indiff}
\caption{ Games $\Gm_2$ and $\Gm_3$ in the proof of Lemma~\ref{th-ks-indiff}.}
\end{figure}

\begin{figure}[tp]
	\begin{minipage}[t]{0.48\textwidth}
		\NewExperiment[Games \fbox{$\Gm_4$}, $\Gm_5$]
		
		\ExptSepSpace
		\begin{algorithm}{$\PrivO(\roIndex,\X)$}
			\item $z \gets\TLSKDF_{\roIndex}[\PubO](\X)$
			\item return \gamechange{$z$}
		\end{algorithm}	
		\ExptSepSpace
		\begin{algorithm}{$\Finalize(b')$}
			\item \fbox{if $\bad_C$ then return $0$}
			\item return $b'$
		\end{algorithm}
	\end{minipage}
	\vline
	\hspace{.02\textwidth}
	\begin{minipage}[t]{0.8\textwidth}
		\NewExperiment[Game $\Gm_6$]

		\begin{oracle}{$\simulator[\RO](\simIndex, \simInString,\rlookuptable)$}
			\item $\simOut \getsr \bits^{\hashlen}$
			\item return $\simOut$
		\end{oracle}
		
	\end{minipage}
	
	\label{fig:gm456-ks-indiff}
	\caption{ Games $\Gm_4$, $\Gm_5$, and $\Gm_6$ in the proof of Lemma~\ref{th-ks-indiff}.}
\end{figure}

\fi

%%% Local Variables:
%%% mode: latex
%%% TeX-master: "main"
%%% End:

\section{Modularizing Handshake Encryption}
\label{sec:modularizing}
%\TODO{possibly rename $KE_1$ and $KE_2$ in section 5 to match naming in corollary section}

Next will argue that using ``internal'' keys to encrypt handshake messages on the TLS~1.3 record-layer does not impact the security of other keys established by the handshake. 
%This will then enable us to use an as-simple-as-possible key exchange security model.
%In particular, we avoid considering such record-layer encryption within the model, as in ACCE-style security models~\cite{C:JKSS12}, for instance, which reduces the complexity of the model and proofs significantly. 
%

	Theorem~\ref{thm:maul}
		below %
formulates our argument in a general way, applicable to any multi-stage key exchange protocol, so that future analyses of similar protocols might take advantage of this modularity as well.

Intuitively, we argue as follows. Let $\KE_{2}$ be a protocol that provides multiple different stages with different external keys (i.e., none of the keys is used in the protocol, e.g., to encrypt messages), and let $\KE_{1}$ be the same protocol, except that some keys are ``internal'' and used, e.g., to encrypt certain protocol messages.
We argue that either using ``internal'' keys in $\KE_{1}$ does not harm the security of \emph{other} keys of $\KE_{1}$, or $\KE_{2}$ cannot be secure in the first place.
This will establish that we can prove security of a variant TLS~1.3 \emph{without} handshake encryption (in an accordingly simpler model),
and then lift this result to the actual TLS~1.3 protocol \emph{with} handshake encryption and the handshake traffic keys treated as ``internal'' keys.


\begin{theorem}
	\label{thm:TLS-transform}
	Let $\KE_1$ be the TLS~1.3 PSK-only resp.\ PSK-(EC)DHE mode \emph{with} handshake encryption (i.e., with internal stages $\KE_1.\INT = \{3,4\}$) as specified on the right-hand side in Figure~\ref{fig:tls-handshake}.
	Let $\KE_2$ be the same mode \emph{without} handshake encryption (i.e., $\KE_1.\INT = \emptyset$ and AEAD-encryption/decryption of messages is omitted).
	Let $\MaulSend$ and $\MaulRecv$ be the AEAD encryption resp.\ decryption algorithms deployed in TLS~1.3 and $\MaulKeys = \KE_1.\INT = \{3,4\}$.
Then we have	
%	For any adversary~$\advA$ against the multi-stage key exchange security of~$\KE_2$, running in time $t$ and making $\qSend$ queries to its $\Send$ oracle,
%	there exists an adversary~$\advB$ that runs in time $\approx t + \qSend \cdot t_{\mathrm{AEAD}}$ (where $t_{\mathrm{AEAD}}$ is the maximum time required to execute AEAD encryption or decryption of TLS~1.3 messages) and makes at most $\qSend$ queries to $\RevSessionKey$ in addition to queries made by $\advA$, and the same number of queries as~$\advA$ to all other oracles in the $\KESEC$ game,
%	such that
	\shortlongeqn[,]{
		\Adv^{\KESEC}_{\KE_1}(t, \qNewSecret, \qSend, \qRevSessionKey, \qRevLongTermKey, \qTest, \qRO)
		\leq
		\Adv^{\KESEC}_{\KE_2}(t+ t_{\mathrm{AEAD}} \cdot \qSend, \qNewSecret, \qSend, \qRevSessionKey + \qSend, \qRevLongTermKey, \qTest, \qRO)
% 		\genAdv{\KESEC}{\KE_2}{\advA} \leq \genAdv{\KESEC}{\KE_1}{\advB}.
	}
	where $t_{\mathrm{AEAD}}$ is the maximum time required to execute AEAD encryption or decryption of TLS~1.3 messages.
\end{theorem}




% we establish the following formal result saying that to analyze security of the TLS~1.3 PSK-only or PSK-(EC)DHE mode \emph{with} handshake encryption (captured as~$\KE_2$), it suffices to analyze the security of the same protocol \emph{without} handshake encryption (denoted $\KE_1$)---in an accordingly simpler model only treating ``external'' keys---, on which we can apply the AEAD encryption/decryption transformation.


For TLS~1.3 this means that we will not consider any security guarantees provided by the additional encryption of handshake messages.
We consider this as reasonable for PSK-mode ciphersuites, because the main purposes of handshake message encryption in TLS~1.3 is to hide the identities of communicating parties, e.g., in digital certificates, cf.\ \cite{PoPETS:ABFNO19}. 
In PSK mode there are no such identities. The $\pskid$ might be viewed as a string that could identify communicating parties, but it is sent unencrypted in the $\ClientHello$ message, anyway; the encryption of subsequent handshake messages would not contribute to its protection. 
%Therefore we consider it reasonable to essentially ignore the additional privacy-protecting properties of handshake message encryption in our model and proofs.


% For the case of the TLS~1.3 PSK modes,
% we will treat as $\KE_2$ the TLS~1.3 PSK-only resp.\ PSK-(EC)DHE mode as specified on the left-hand side of Figure~\ref{fig:tls-handshake} (i.e., \emph{with} handshake encryption using the internal keys from stages~3 and~4 ($\chtk$ and $\shtk$),
% and as $\KE_1$ the corresponding PSK handshake \emph{without} handshake encryption.


%\TODO{update Theorem 3 in Section 5 to take place after Section 4}

\iffull
	\iffull
	\subsection{Handshake Encryption as a Modular Transformation}
\else
	\section{Handshake Encryption as a Modular Transformation}
	\label{app:modular-transform}
\fi

Formally, let $\KE_2 = (\KEKGen, \KEActivate, \KERun)$ be a key exchange protocol with no internal keys. 
We define another key exchange protocol~$\KE_1$ which is parameterized by two functions $\MaulSend$ and $\MaulRecv$ and a list $\MaulKeys \subseteq \{1, \ldots, \STAGES\}$, where $\STAGES$ is the number of stages of $\KE_{2}$. $\KE_1$ inherits its key generation and activation algorithms from $\KE_2$. 
In its $\KE_1.\KERun$ algorithm, described in Figure~\ref{fig:encrypted-ke}, it essentially applies $\MaulRecv$ to a message before calling $\KE_2.\KERun$, and then $\MaulSend$ to the returned message, to transform the protocol messages as they pass over a wire.
This transformation may be, for instance, the encryption and decryption of messages of $\KE_{2}$ using an internal key.

\begin{figure}[p]
	\centering
	\begin{minipage}[t]{9cm}
		\begin{algorithm}{$\KE_1.\KERun(u,\pi^i_u, \pskey, m)$}
			\item $\keys \gets (\pi^i_u.\skey[\stage]$ for $\stage \in \MaulKeys)$
			\item $\acc \gets (\pi^i_u.\taccepted[\stage] \neq \infty$ for $\stage$ in $[1\ldots\STAGES])$
			\item $\tilde{m} \gets \MaulRecv(\keys, \pi^i_u.\role, \acc, m)$
			\item $(\pi^i_u, \tilde{m}') \gets \KE_2.\KERun(u,\pi^i_u, \pskey, \tilde{m})$
			\item $\keys \gets (\pi^i_u.\skey[\stage]$ for $\stage \in \MaulKeys)$
			\item $\acc \gets (\pi^i_u.\taccepted[\stage] \neq \infty$ for $\stage$ in $[1\ldots\STAGES])$
			\item $m' \gets \MaulSend(\keys, \pi^i_u.\role, \acc, \tilde{m}')$
			\item return $(\pi^i_u, m')$
		\end{algorithm}
	\end{minipage}
	\caption{Key exchange $\KE_1$ built by transforming protocol messages of $\KE_2$.}
	\label{fig:encrypted-ke}
\end{figure}

In addition to the messages, both algorithms take as input the list of stages that have been accepted by the current session, its role (initiator or responder) in the protocol, and a list of the keys from all stages in $\MaulKeys$. 
In the security game for $\KE_1$, the stages in $\MaulKeys$ will produce internal keys; all other keys remain external.

Although $\MaulSend$ and $\MaulRecv$ change the messages as they pass over the wire, the way that the messages are processed after receipt by $\KE_2.\KERun$ must not change.
In particular, $\KE_2.\KERun$, internally run within~$\KE_1.\KERun$, still expects messages of the same format and content;
also, $\KE_1$ defines its session and contributive identifiers, as well as all other session-specific information in the same way as $\KE_2$.


\paragraph{Correctness.}

Not all choices of $\MaulSend$ and $\MaulRecv$ are ``good choices''. 
For example, if mauling overwrites critical pieces of the protocol messages, then no honest session would ever accept a key. 
The resulting key exchange $\KE_2$ would be vacuously ``secure'' because it would be unusable. 


For our perspective to be meaningful, we therefore need a correctness property that guarantees that two honest parties executing $\KE_1$ with no adversarial interference will accept at all stages. 
%
Informally, we wish that if two sessions honestly executing $\KE_2$ will accept keys for stage $s$ with probability $p$, then two sessions honestly executing $\KE_1$ will accept keys for stage $s$ with probability close to $p$. 
This property only needs to hold when the protocol messages are relayed honestly, with no changes or delivery failures beyond those caused by the application of $\MaulSend$ and $\MaulRecv$.

We do not give a formal definition or proof of correctness for TLS~1.3, but we note that in TLS~1.3, the transformation algorithms are AEAD encryption and decryption.
Since decryption failures cannot occur in the standardized AEAD algorithms if messages are honestly relayed (due to their perfect correctness),
received messages will always match their corresponding sent message, and correctness of $\MaulSend$ and $\MaulRecv$ follows.


\paragraph{Security.}

We wish $\KE_1$ to be secure if $\KE_2$ is secure.
This should be independent of $\MaulSend$ and $\MaulRecv$, i.e., should hold even if $\MaulSend$ leaks its keys and fully overwrites all protocol messages.
The following theorem established this result, using that the keys used for the transformation are internal and $\MaulSend$ and $\MaulRecv$ have no access to other privileged information.
Therefore, their behavior can be mimicked by a reduction to the security of~$\KE_2$ as long as $\KE_2$ has ``public session matching'' for the stages in~$\MaulKeys$ of $\KE_1$, i.e., session partnering (or matching) for those stages is decidable from the publicly exchanged messages.%
\footnote{The property of ``public session matching'' has already already come up when considering the composition of (regular or multi-stage) key exchange protocols with subsequent symmetric-key protocols~\cite{CCS:BFWW11,CCS:DFGS15,EPRINT:DFGS15,thesis:Guenther18}.}

\begin{theorem}
	\label{thm:maul}
	Let $\KE_2$ be a key exchange protocol with $\STAGES$ stages, $\KE_2.\INT$ being empty, and public session matching.
	Let $\MaulSend$ and $\MaulRecv$ be algorithms as above and $\MaulKeys \subseteq \{1, \ldots, \STAGES\}$.
	Define key exchange $\KE_1$ such that $\KE_1.\KERun$ is described in Figure~\ref{fig:encrypted-ke}, $\KE_1.\INT = \MaulKeys$, and all other attributes of $\KE_1$ are identical to those of $\KE_2$.
	
	Let $\advA$ be an adversary with running time $t$ against the multi-stage key exchange security of $\KE_1$, making $\qSend$ queries to the $\Send$ oracle.
	Then there exists an adversary $\advB$ with running time $\approx t + \qSend m$, where $m$ is the maximum running time of $\MaulSend$ and $\MaulRecv$, such that
	\[
	\genAdv{\KESEC}{\KE_1}{\advA} \leq \genAdv{\KESEC}{\KE_2}{\advB}.
	\]
	$\advB$ makes at most $\qSend$ queries to $\RevSessionKey$ in addition to queries made by $\advA$
	and the same number of queries as~$\advA$ to all other oracles in the $\KESEC$ game.
\end{theorem}

\begin{proof}
	Adversary $\advB$ runs adversary $\advA$ and relays all of its queries to the appropriate oracles in its own $\KESEC$ game, except for $\Send$ queries.
	It maintains the time~$\time$ of the $\KESEC$ game itself, incrementing it once per query.
	For each session $\pi_u^i$, it maintains a list $\keys_u^i$ that is initially empty and a list $\acc_u^i$ in which $\acc_u^i[\stage]$ is initially $\false$ for each $\stage \in \MaulKeys$. 
	
	When $\advA$ makes a query $\Send(u, i, m)$, $\advB$ first checks for each $\stage \in \MaulKeys$ with $\acc_u^i[\stage] = \false$ whether $\pi_u^i.\accepted[\stage] \neq \infty$. 
	For each $\stage$ which satisfies this condition, $\advB$ checks whether $\pi_u^i.\tested[\stage]$ or $\pi_u^i.\revealed[\stage]$ is true and if $\pi_u^i$ has a partnered session (matching $\sid[\stage]$) which has been tested or revealed.
	(The latter check for partnering is possible because $\KE_1$ has public session matching.)
	% 	\hd{In order to do this, session IDs must be public. This is true of TLS~1.3, but we may need to make that assumption explicit.}
	If any of these conditions is true, then $\advB$ knows $\pi_u^i.\skey[\stage]$. 
	Otherwise, it makes an extra query $\RevSessionKey(u, i, \stage)$ and adds the response to $\keys_u^i$. 
	Then it marks $\acc_u^i[\stage] \gets \true$ and computes $\tilde{m} \gets \MaulRecv(\keys_u^i, \pi^i_u.\role, \acc_u^i, m)$. 
	It queries its own $\Send$ oracle on the tuple $(u,i, \tilde{m})$ and captures the response $\tilde{m'}$. 
	Then it returns $m' \gets \MaulSend(\keys_u^i, \pi^i_u.\role, \acc_u^i, \tilde{m}')$ to $\advA$.
	
	$\advB$ perfectly simulates $\KE_1$ for $\advA$, so we wish that if $\advA$ wins its simulated game, $\advB$ should also win its game.
	%
	$\advA$ can win the $\KESEC$ game in one of three ways: it can violate the $\Sound$ predicate, it can violate the $\ExplicitAuth$ predicate, or it can satisfy the $\Fresh$ predicate and guess the secret bit $b$. 
	All of the variables tracked by the $\ExplicitAuth$ and $\Sound$ predicates are maintained by the $\KESEC$ game for $\KE_1$, not by $\advB$.
	Therefore $\advA$ wins the simulated game by violating $\Sound$ or $\ExplicitAuth$ only if $\Sound$ or $\ExplicitAuth$ is violated in the $\KESEC$ game for $\KE_2$. 
	In this case, $\advB$ also wins. 
	
	If $\advA$ wins by guessing the secret bit $b$, the story is more complicated. 
	The bit $b$ is chosen by the $\KESEC$ game, so if $\advA$ guesses correctly, then so will $\advB$. 
	However, a correct guess only matters if the queries do not violate the $\Fresh$ predicate.
	Even if $\advA$ did not violate the $\Fresh$ predicate, $\advB$ makes up to $\qSend$ additional $\RevSessionKey$ queries.
	Each of these could cause $\Fresh$ to be set to false. 
	We claim that none of these queries violate the $\Fresh$ predicate.
	
	The $\Fresh$ predicate requires that no session be both tested and revealed.
	$\advB$ only reveals keys that have not already been tested, so the only  worry is that $\advA$ will test this key later. 
	However, all keys that $\advB$ reveals are in $\MaulKeys$, which is a subset of $\KE_1.\INT$, meaning they are internal keys.
	These keys cannot be tested if any session which has accepted it has moved on with the protocol.
	Since $\advB$ only reveals a key when a session has both accepted that key and received the next protocol message, it will have moved on and $\advA$ can not make any later $\Test$ queries on a key that $\advB$ has revealed.
	
	The next condition of $\Fresh$ is that a tested session's partner cannot be tested or revealed. 
	$\advB$ ensures that such a $\Test$ query does not occur before the $\RevSessionKey$ query. 
	Again, the $\Test$ query cannot happen after the $\RevSessionKey$ query because the session whose key was revealed has moved on with the protocol.
	Since all the revealed keys are internal in the simulated game, $\advA$ cannot test them after this point. 
	
	The remaining three conditions of the $\Fresh$ predicate establish different levels of forward secrecy. 
	They check for the existence of a contributive partner.
	We want to exclude the situation that a contributive partner exists in $\advA$'s simulated game, but not in $\advB$'s game.
	However, contributive identifiers are defined identically in $\KE_1$ and $\KE_2$. 
	Therefore if two sessions $\pi_u^i$ and $\pi_v^j$ have matching contributive identifiers in the simulated game for $\KE_2$, they will also have matching identfiers in the game for $\KE_1$.
	
	It is therefore not possible for $\advA$ to win its simulated $\KESEC$ game unless $\advB$ also wins its $\KESEC$ game, and the theorem follows.
\end{proof}
\else
\fi

 
%%% Local Variables:
%%% mode: latex
%%% TeX-master: "main"
%%% End:

\section{Tight Security of the TLS~1.3 PSK Modes}
\label{sec:ke-proof}

In this section, we apply the insights gained in Sections~\ref{sec:ks-indiff} and~\ref{sec:modularizing} to obtain tight security bounds for both the PSK-only and the PSK-(EC)DHE mode of TLS~1.3.
To that end, we first present the protocol-specific properties of the TLS~1.3 PSK-only and PSK-(EC)DHE modes such that they can be viewed as multi-stage key exchange (MSKE) protocols as defined in Section~\ref{sec:ake-model}.
Then, we prove tight security bounds in the MSKE model in Theorem~\ref{thm:full-psk-ecdhe-ke} for the TLS~1.3 PSK-(EC)DHE mode and 
	in Theorem~\ref{thm:psk-ke} 
for the TLS~1.3 PSK-only mode%
	.

\subsection{TLS~1.3 PSK-only/PSK-(EC)DHE as a MSKE Protocol}
\label{sec:tls-formal-def}

We begin by capturing the TLS~1.3 PSK-only and PSK-(EC)DHE modes, specified in \autoref{fig:tls-handshake}, formally as MSKE protocols.
To this end, we must explicitly define the variables discussed in Section~\ref{sec:ake-model}.
In particular, we have to define the stages themselves, which stages are internal and which replayable, the session and contributive identifiers, when stages receive explicit authentication, and when stages become forward secret.

\paragraph{Stages.}
The TLS~1.3 PSK-only/PSK-(EC)DHE handshake protocol has eight stages (i.e., $\STAGES = 8$), corresponding to the keys $\ets$, $\eems$, $\shtk$, $\chtk$, $\cats$, $\sats$, $\ems$, and $\rms$ in that order.
The set $\INT$ of internal keys contains $\chtk$ and $\shtk$, the handshake traffic encryption keys. 
Stages $\ets$ and $\eems$ are replayable: $\REPLAY[s]$ is true for $s \in \{1,2\}$ and false for all others.

\paragraph{Session and contributive identifiers.}
The session and contributive identifiers for stage$s$ are tuples $(\mathit{label}_s, \mathit{ctxt})$, where $\mathit{label}_s$ is a unique label identifying stage~$s$, and $\mathit{ctxt}$ is the transcript that enters key's derivation.
The session identifiers $(\sid[s])_{s \in \{1, \dotsc, 8\}}$ are defined as follows:%
\footnote{Components marked with ${}^\dagger$ are only part of the TLS~1.3 PSK-(EC)DHE handshake.}
%
{\allowdisplaybreaks
\begin{align*}
	\iffull
	\sid[1] &= \left(``\ets", (\mathlist{\CH, \CKS^\dagger, \CPSKtls}) \right), \\
	\sid[2] &= \left( ``\eems", (\mathlist{\CH, \CKS^\dagger, \CPSKtls}) \right), \\
	\else
	\sid[1] / \sid[2] &= \left(``\ets" / ``\eems", (\mathlist{\CH, \CKS^\dagger, \CPSKtls}) \right), \\
	\fi
	\iffull
	\sid[3] &= \left( ``\chtk", (\mathlist{\CH, \CKS^\dagger, \CPSKtls, \SH, \SKS^\dagger, \SPSKtls}) \right), \\
	\sid[4] &= \left( ``\shtk", (\mathlist{\CH, \CKS^\dagger, \CPSKtls, \SH, \SKS^\dagger, \SPSKtls}) \right), \\
	\else
	\sid[3] / \sid[4] &= \left( ``\chtk" / ``\shtk", (\mathlist{\CH, \CKS^\dagger, \CPSKtls, \SH, \SKS^\dagger, \SPSKtls}) \right), \\
	\fi
	\iffull
	\sid[5] &= \left( ``\cats", (\mathlist{\CH, \CKS^\dagger, \CPSKtls, \SH, \SKS^\dagger, \SPSKtls, \EE, \SF}) \right), \\
	\sid[6] &= \left( ``\sats", (\mathlist{\CH, \CKS^\dagger, \CPSKtls, \SH, \SKS^\dagger, \SPSKtls, \EE, \SF}) \right), \\
	\sid[7] &= \left( ``\ems", (\mathlist{\CH, \CKS^\dagger, \CPSKtls, \SH, \SKS^\dagger, \SPSKtls, \EE, \SF}) \right), \text{ and} \\
	\else
	\sid[5] / \sid[6] / \sid[7] &= \left( ``\cats" / ``\sats" / ``\ems", (\mathlist{\CH, \dots, \SPSKtls, \EE, \SF}) \right), \text{ and} \\
	\fi
	\sid[8] &= \left( ``\rms", (\mathlist{\CH, \CKS^\dagger, \CPSKtls, \SH, \SKS^\dagger, \SPSKtls, \EE, \SF, \CF}) \right).
\end{align*}}%
%
To make sure that a server that received $\ClientHello$, $\ClientKeyShare^\dagger$, and $\ClientPreSharedKey$ untampered can be tested in stages~$3$ and~$4$,
even if the sending client did not receive the server's answer,
we set the contributive identifiers of stages~$3$ and~$4$ such that~$\rolecid$ reflects the messages that a session in role~$\role$ must have honestly received for testing to be allowed.
Namely, we let clients (resp. servers) upon sending (resp. receving) the messages $(\CH, \CKS^\dagger, \CPSKtls)$ set
\iffull
\begin{align*}
	\respcid[3] &= \left( ``\chtk", (\mathlist{\CH, \CKS^\dagger, \CPSKtls}) \right) \text{ and}\\
	\respcid[4] &= \left( ``\shtk", (\mathlist{\CH, \CKS^\dagger, \CPSKtls}) \right).
\end{align*}
\else
$\respcid[3] = (``\chtk", (\mathlist{\CH, \CKS^\dagger, \CPSKtls}))$ and $\respcid[4] = (``\shtk", (\mathlist{\CH, \CKS^\dagger, \CPSKtls}))$.
\fi
Further, when the client receives (resp. the server sends) the message $(\mathlist{\SH, \SKS^\dagger, \SPSKtls})$, they set
\iffull
\[
	\initcid[3] = \sid[3] \quad\text{and}\quad \initcid[4] = \sid[4].
\]
\else
$\initcid[3] = \sid[3]$ and $\initcid[4] = \sid[4]$.
\fi
For all other stages $s \in \{\mathlist{1,2,5,6,7,8}\}$, $\initcid[s] = \respcid[s] = \sid[s]$ is set upon acceptance of the respective stage (i.e., when $\sid[s]$ is set as well).


\paragraph{Explicit authentication.}
For initiator sessions, all stages achieve explicit authentication when the $\ServerFinished$ message is verified successfully.
This happens right before stage~$5$ (i.e., $\cats$) is accepted.
That is, upon accepting stage~$5$ all previous stages receive explicit authentication retroactively and all following stages are explicitly authenticated upon acceptance.
Formally, we set $\EAUTH[\initiator, s] = 5$ for all stages $s\in \{1,\dotsc,8\}$.

\iffull
For responder session, all stages receive explicit authentication upon (successful) verification of the $\ClientFinished$ message. 
This occurs right before the acceptance of stage~$8$ (i.e., $\rms$).
Similar to initiators, responders receive explicit authentication for all stages upon acceptance of stage~$8$ since this is the last stage of the protocol.
Accordingly, we set $\EAUTH[\responder, s] = 8$ for all stages $s \in \{1,\dotsc,8\}$.
\else
Analogously, responder sessions receive explicit authentication right before accepting stage~$8$ via the $\ClientFinished$ message;
i.e., $\EAUTH[\responder, s] = 8$ for all stages $s \in \{1,\dotsc,8\}$.
\fi

\paragraph{Forward secrecy.}
Only keys dependent on a Diffie--Hellman secret achieve forward secrecy, so all stages~$s$ of the PSK-only handshake have $\FS[r, s, \fs] = \FS[r, s, \wfstwo] = \infty$ for both roles~$r \in \{\initiator, \responder\}$.
In the PSK-(EC)DHE handshake, full forward secrecy is achieved at the same stage as explicit authentication for all keys except $\ets$ and $\eems$, which are never forward secret.
That is, for both roles~$r$ and stages $s \in \{3,\dotsc,8\}$ we have $\FS[r, s, \fs] = \EAUTH[r, s]$.
All keys except $\ets$ and $\eems$ possess weak forward secrecy $2$ upon acceptance, so we set $\FS[r, s, \wfstwo] = s$ for stages~$s \in \{3,\dotsc,8\}$.
Finally, as stages~$1$ and~$2$ (i.e., $\ets$ and $\eems$) never achieve forward secrecy we set $\FS[r, s, \fs] = \FS[r, s, \wfstwo] = \infty$ for both roles~$r$ and stages $s \in \{1,2\}$.

% paragraph is in the intro now
%\paragraph{\replace{View on the TLS~1.3 handshake protocol}{Scope of our analysis}.}
%Like many previous cryptographic analyses~\cite{C:JKSS12,C:KraPatWee13,CCS:DFGS15,EuroSP:FisGue17,JC:DFGS21,JC:DieJag21,ACNS:DavGun21} of the TLS handshake protocol, we focus only on the ``cryptographic core'' of the PSK modes of the TLS~1.3 handshake protocol.
%In particular, our analysis given in Section~\ref{sec:psk-ecdhe-ke-proof} (resp.\ Section~\ref{sec:psk-only-ke-proof}) focuses solely on the TLS~1.3 PSK-(EC)DHE (resp.\ TLS~1.3 PSK-only) handshake protocol (with optional $0$-RTT) for a fixed Diffie--Hellman group and for a fixed cipher suite (i.e., AEAD algorithm and hash algorithm).
%This means we neither consider the negotiation of other versions of TLS than version 1.3 nor consider the negotiation of the DH group for the PSK-(EC)DHE mode and the cipher suite for both PSK-only and PSK-(EC)DHE mode,
%as done, e.g., in~\cite{ACISP:DowSte15,SP:BBFGKZ16}.
%Since we only consider TLS~1.3 ``in isolation'' we also do not consider backward compatibility.
%\dd{This was a big question by one reviewer on our JoC submission.}
%Moreover, our view does not include the TLS record layer
%and we we do not consider the composition of the handshake protocol with the record layer encryption as, e.g., done in \cite{CCS:DFGS15,thesis:Guenther18,JC:DFGS21,JC:DieJag21}.
%
%\TODO{find a better caption.}
%\dd{Pls check if anything important is missing (or if the record layer part seems misplaced.)}

\subsection{Tight Security Analysis of TLS~1.3 PSK-(EC)DHE}
\label{sec:psk-ecdhe-ke-proof-short}

We now come to the tight MSKE security result for the TLS~1.3 PSK-(EC)DHE handshake.

\begin{theorem}
	\label{thm:full-psk-ecdhe-ke}
	Let $\TLSPSKECDHE$ be the TLS~1.3 PSK-(EC)DHE handshake protocol (with optional 0-RTT) as specified on the right-hand side in \autoref{fig:tls-handshake} without handshake encryption.
	Let $\G$ be the Diffie--Hellman group of order~$p$.
	Let $\nl$ be the length in bits of the nonce, let $\hashlen$ be the output length in bits of $\abstractHash$, and let the pre-shared key space be $\KEpskeyspace = \bits^\hashlen$.
	We model the functions $\abstractHash$ and $\TLSKDF_x$ for each $x \in \{\binder, \dotsc, \rms\}$ as $12$ independent random oracles $\mathlist{\RO_\Thash,\RO_{\binder},\dotsc, \RO_{\rms}}$.
	%
% 	Let $\advA$ be an adversary against the $\KESEC$ security of $\TLSPSKECDHE$ running in time $t$, and let $\advA$ make $\qNewSecret$, resp.\ $\qSend$ queries to its $\NewSecret$, resp.\ $\Send$ oracles.
% 	Further, let $\qRO$ be the number of random oracle queries $\advA$ makes in total to its random oracles $\mathlist{\RO_\Thash,\RO_{\binder},\dotsc, \RO_{\rms}}$.
	Then,
	\begin{align*}
		&\genAdv{\KESEC}{\TLSPSKECDHE}{t, \qNewSecret, \qSend, \qRevSessionKey, \qRevLongTermKey, \qTest, \qRO}
		\leq \frac{2\qSend^2}{2^{\nl} \cdot p} \\
		&+ \frac{(\qRO+\qSend)^2 + \qNewSecret^2 + (\qRO+6\qSend)^2 + \qRO \cdot \qNewSecret + \qSend}{2^{\hashlen}}
		+  \frac{4(t + 4 \log(p) \cdot \qRO)^2}{p}.
	\end{align*}
\end{theorem}

\begin{remark}
	Our MSKE model from Section~\ref{sec:ake-model} assumes pre-shared keys to be uniformly random sampled from~$\KEpskeyspace$, where here $\KEpskeyspace = \bits^\hashlen$.
	This matches how pre-shared keys are derived for session resumption, as well as our analysis of domain separation, which assumes pre-shared keys to be of length~$\hashlen$.
\end{remark}

\begin{remark}\label{rem:psk-distribution}
	Our bound is easily adapted to any distribution on $\bits^{\hashlen}$ in order to accommodate out-of-band pre-shared keys that satisfy the length requirement but do not have full entropy.
	%Since we need to avoid collisions for the pre-shared keys, we obtain lower bound on the distributions entropy for a desired security level.
	Expectedly, lower-entropy PSK distributions result in weaker bounds, due to the increased chance for collisions between PSKs as well as the adversary guessing a PSK.
\end{remark}

\iffull
	\begin{proof}
	\iffull
\else
	\section{Proof of Theorem~\ref{thm:full-psk-ecdhe-ke}}
	\label{app:full-psk-ecdhe-ke-proof}
\fi
	To prove our bound, we make an incremental series of changes to the key exchange security game $G_{{\TLSPSKECDHE},\advA}$.
	We divide the proof into three phases reflecting the three ways of the adversary to win the security game.
\begin{enumerate}
\item We establish that the adversary cannot violate the predicate $\Sound$.
\item We establish the same for the predicate $\ExplicitAuth$.
\item Finally, we ensure that all $\Test$ queries return uniformly random keys independent of the challenge bit $b$ if predicate $\Fresh$ is not violated.
\end{enumerate}
We can then conclude that the adversary cannot do better than random guessing to win the game, i.e., its advantage is $0$.

	% Game 0 -- Initial game 
	\gamehop[Initial game]{game:init}%
	The initial game $\refGameMath{game:init}$ is the key exchange security game $G_{{\TLSPSKECDHE},\advA}$ played for the TLS~1.3 PSK-(EC)DHE handshake (with optional 0-RTT) as specified in Figure~\ref{fig:tls-handshake} (right), but without handshake encryption.
	Note that the functions $\abstractHash$ and $\TLSKDF_x$ for $x \in \{\binder, \dotsc, \rms\}$ are modeled as $12$ independent random oracles $\mathlist{\RO_\Thash,\RO_{\binder},\dotsc, \RO_{\rms}}$.
	We implement random oracle~$\RO_x$ by a look-up table $\roTable_x$ assigning inputs to outputs. We assume that every look-up table implementing a random oracle is stored in a data structure that enables constant time access when indexed either by random oracle inputs or by random oracle outputs, using two hash tables, for instance.
	%
	By definition, we have
	\[
		\Pr [ \refGameMath{game:init} \outputs 1 ] = \genAdv{\KESEC}{\TLSPSKECDHE}{\advA}.
	\]
	
	\subsection*{Phase~1: Ensuring Predicate~$\Sound$ cannot be violated}
	
	% Game 4 -- Abort if there is a collision in the Hello message 
	\gamehop[Exclude collisions of nonces and group elements]{game:hello-coll}%
	In \thisGameMath, we eliminate collisions among nonces and group elements computed by honest sessions via two new flags:
	\begin{itemize}
		\item $\bad_C$ is set when two honest sessions choose the same nonce and group element, and
		\item $\bad_O$ is set when an honest responder samples some nonce and group element that have already been received by another session. 
		We view this nonce and group element as having been chosen by an adversarial session.
	\end{itemize}
	If either $\bad_C$ or $\bad_O$ is set, the game returns $0$ from $\Finalize$.
	
	By the well-known identical-until-bad-lemma \cite[Lemma 2]{EPRINT:BelRog04}, we get
	\begin{align}\label{eq:game:hello-coll-raw}\nonumber
		\Pr[ \prevGameMath \outputs 1 ] &\leq \Pr[ \thisGameMath \outputs 1 ] + \Pr[ \thisGameMath \text{ sets } \bad_C ] \\ 
		&\qquad+ \Pr[ \thisGameMath \text{ sets } \bad_O ] .
	\end{align}
	Let us separately analyze the probabilities that $\thisGameMath$ sets the flags~$\bad_C$ and~$\bad_O$.
	Each $\Send$ query causes at most one session to uniformly and independently sample a nonce $r \sample \bits^{\nl}$ and a group element $g \sample \G$. 
	If the $\bad_C$ flag is set, we have that there exists some $\Send$ query that creates a session $\pi_u^i$ using $\KEActivate$.
	This new session samples nonce and group element$(r, g)$ which were previously sampled by another session $\pi_{u'}^{i'}$.
	That is, the probability for $\bad_C$ to be set is the probability of a collision among the (up to) $\qSend$ pairs of uniformly and independently sampled nonces and group elements; we can use the birthday bound to bound the probability of setting $\bad_C$ by
	\begin{equation}\label{eq:badC}
		\Pr[ \thisGameMath \text{ sets } \bad_C ]  \leq \frac{\qSend^2}{2^{\nl} \cdot p}.
	\end{equation}
	where $\qSend$ is the number of $\Send$ queries.

	Next, if the game sets $\bad_O$, we have that there is a $\Send$ query which creates a new session $\pi_v^j$. 
	This session samples a nonce $r_S \sample \bits^{\nl}$ and a group element $Y \sample \G$, which were already received by another session $\pi_u^i$.
	There are at most $\qSend$ sessions, so there are no more than $\qSend$ received pairs which which $(r_S, Y)$ can collide.
	Since $\pi_v^j$ samples its nonce and group element uniformly and independently at random from $\bits^{\nl} \times \G$, we get by the union bound that the probability that $\pi_v^j$ samples one of the already received pairs is bounded from above by $\qSend / (2^{nl} \cdot p)$.
	Overall, we again get by the union bound that there is such a collision for any $\pi_v^j$ with probability
	\begin{equation}\label{eq:badO}
		\Pr[ \thisGameMath \text{ sets } \bad_O ]  \leq \qSend \cdot \frac{\qSend}{2^{\nl} \cdot p} = \frac{\qSend^2}{2^{\nl} \cdot p}.
	\end{equation}
	
	\noindent
	Combining Equations (\ref{eq:game:hello-coll-raw})--(\ref{eq:badO}), we get
	\begin{equation}\label{eq:game:hello-coll}
		\Pr[ \prevGameMath[\advA^*] \outputs 1 ] \leq \Pr[ \thisGameMath[\advA^*] \outputs 1 ] + \frac{2\qSend^2}{2^{\nl} \cdot p} .
	\end{equation}

	% Exclude binder collisions
	\gamehop[Exclude binder collisions]{game:binder-coll}
	In game $\thisGameMath$, we let the adversary lose if there is a collision among the binder values computed by any honest session.
	Whenever two distinct queries to $\RObinder$ return the same value, we set a flag $\bad_\binder$ and return $0$ from $\Finalize$.
	
	To implement this, we add a table~$\collTracker_\binder$ to the random oracle~$\RObinder$ (this table is currently redundant to the table implementing $\RObinder$, but will be useful in later game hops, where we will introduce changes such that it is not guaranteed anymore that all $\binder$ values will be contained in the $\RObinder$ table).
	Whenever $\RObinder$ computes a binder value $b = \RO_\binder(\mathlist{\psk, \mathit{ctxt}})$, we log $\collTracker_\binder[b] \assign (\mathlist{\psk, \mathit{ctxt}})$.
	Now, whenever $\RObinder$ computes some binder~$b$ for some tuple~$s$ and~$\collTracker_\binder[b]$ is not empty, there has to be a tuple $s' = (\psk, \mathit{ctxt})$ with $\RObinder(\mathlist{psk, \mathit{ctxt}}) = b$ queried before and we have found a collision if $s \neq s'$.
	In this case we set $\bad_\binder$.

	Again by the identical-until-bad-lemma,
	\[
	\Pr[ \prevGameMath \outputs 1 ] \leq \Pr[ \thisGameMath \outputs 1 ] + \Pr[\thisGameMath \text{ sets }\bad_\binder] .	
	\]
	To bound the probability that the game sets flag $\bad_\binder$, we construct a reduction~$\advB_1$ to the collision-resistance of $\RObinder$.
	The reduction~$\advB_1$ simulates \thisGame for adversary $\advA$.
	It implements all oracles itself except for $\RObinder$.
	$\advB_1$ will need to query its own oracle $\RObinder$ at most once per $\RO$ query and once per $\Send$ query, so it makes $\qRO + \qSend$ queries in total.
	If the flag~$\bad_\binder$ would be set in \thisGame, which can be checked efficiently using $\collTracker_\binder$ as described before, then the reduction has found a collision $(s,s')$ with $s \neq s'$ such that $\RObinder(s) = \RObinder(s')$.
	Reduction~$\advB_1$ then outputs $(s,s')$ and wins the collision-resistance game.
	
	Therefore, we have that
	\begin{equation}\label{eq:game:binder-coll}
		\Pr[ \prevGameMath \outputs 1 ] \leq \Pr[ \thisGameMath \outputs 1 ] + \genAdv{\CollRes}{\RObinder}{\qRO+\qSend} .
	\end{equation}
	
	% Game -- exclude psk colls
	\gamehop[Exclude collisions of pre-shared keys]{game:psk-coll}
	In game $\thisGameMath$, we set a flag $\bad_{PC}$ and return $0$ from $\Finalize$ whenever the $\NewSecret$ oracle samples a previously sampled pre-shared key (again).
	Formally, we set $\bad_{PC}$ if there exist two distinct tuples $(u, v,\pskid)$ and $(u', v',\pskid')$ with $\pskeys[(u, v, \pskid)] = \pskeys[(u',v',\pskid')]$. 
	By the identical-until-bad-lemma,
	\[
	\Pr[ \prevGameMath \outputs 1 ] \leq \Pr[ \thisGameMath \outputs 1 ] + \Pr[\thisGameMath \text{ sets }\bad_{PC}] .	
	\]
	Since the pre-shared keys are uniformly distributed%
	\footnote{As mentioned in Remark~\ref{rem:psk-distribution}, this term has to be adapted for a different distribution on $\bits^\hashlen$, i.e., for any distribution $\mathcal D$ on $\bits^\hashlen$, the denominator would change to $2^{\alpha}$, where $\alpha$ is the min-entropy of $\mathcal D$.}
	on $\bits^\psklen$, by the birthday bound
	\[
		\Pr[\thisGameMath \text{ sets } \bad_{PC} ] \leq \frac{\qNewSecret^2}{2^\psklen}. 
	\]
	
	\paragraph{Conclusion of Phase~1.}
	At this point, we argue that in \thisGame and any subsequent games, adversary~$\advA$ cannot violate the $\Sound$ predicate without also causing $\Finalize$ to return~$0$.
	If any $\Sound$ check fails, one of the checks we have added to the $\Finalize$ oracle will also fail.
	According to the definition of the MSKE game, there are six events that cause the predicate $\Sound$ to be violated (see Figure~\ref{fig:MSKE-preds}).
	In the following, we argue why each of these events cannot occur in \thisGame and thus $\Sound = \true$ needs to hold from \thisGame on.
	%
	\begin{enumerate}
		\item \textit{There are three honest sessions that have the same session identifier at any non-replayable stage.}
		
		Since the only replayable stages are stages~$1$ ($\ets$) and~$2$ ($\eems$), consider any later stage $s \geq 3$.
		Recall that session identifiers $\sid$ for all stages $s\geq 3$ contain a $\ClientHello$ message containing the initiator session's nonce and group element and a $\ServerHello$ message containing the responder session's nonce and group element (see Section~\ref{sec:tls-formal-def}).
		Every session's $\sid$ therefore contains its own randomly sampled nonce-group element pair.
		For three sessions to accept the same $\sid[s]$ for $s \geq 3$, there must be two honest sessions who have sampled the same nonce and group element.
		Due to \refGameText{game:hello-coll}, this would trigger the $\bad_C$ flag, leading $\Finalize$ to return $0$.
		
		\item \textit{There are two sessions with the same session identifier in some non-replayable stage that have the same role.}
		
		Session identifiers $\sid[s]$ for $s \geq 3$ as defined by TLS~1.3 (see Section~\ref{sec:tls-formal-def}) contain only one pair of nonce and group element per initiator and responder.
		If two honest sessions share a $\sid$ and a role, they must also share a nonce and group element.
		This case would also trigger the $\bad_C$ flag. 
		
		\item \textit{There are two sessions with the same session identifier in some stage that do not share the same contributive identifier in that stage.}
		
		Once a session holds both a contributive identifier and a session identifier for the same stage, both are equal by our definition (see Section~\ref{sec:tls-formal-def}) of the session and contributive identifiers for TLS~1.3.
		This case will therefore never occur.
		
		\item \textit{There are two sessions that hold the same session identifier for different stages.}
		
		This is impossible as the session identifier of stage~$s$ begins with the unique label $\mathit{label}_s$ for stage~$s$.
		
		\item \textit{There are two honest sessions with the same session identifier in some stage that disagree on the identity of their peer or their $\pskid$.}
		
		Two sessions which hold the same session identifier must necessarily agree on the value of the $\binder$, which is part of the $\ClientHello$ message.
		In \refGameText{game:binder-coll}, we required that $\Finalize$ returns $0$ if two queries to the oracle $\RObinder$ collide. 
		The two sessions must therefore also agree on the pre-shared key, which they obtained from the list $\pskeys$.
		From \refGameText{game:psk-coll}, we have that $\Finalize$ returns $0$ if any two distinct entries in $\pskeys$ contain the same value.
		Therefore two sessions can obtain the same pre-shared key from $\pskeys$ only if they hold the same tuple $(u, v, \pskid)$, meaning they agree on both the peer identities and the pre-shared key identity.
		\item \textit{Sessions with the same session identifier in some stage do not hold the same key in that stage.}
		
		We have just established that two sessions with the same session identifier must agree on the peer identities and $\pskid$ (contained in $\CPSKtls$ and $\SPSKtls$), meaning they also share the same $\psk$. 
		Session identifiers for stages whose keys are derived from a Diffie--Hellman secret $\dhe$ must include both Diffie--Hellman shares~$\X$ and~$\Y$ (contained in $\CKS$ and $\SKS$). 
		These shares uniquely determine $\dhe$. 
		Besides that the session identifier also contains the context required to derive the respective stage keys, which then uniquely determines the stage key.
		Therefore, agreement on a session identifier implies agreement on a stage key.
	\end{enumerate}

	\subsection*{Phase~2: Ensuring Predicate~$\ExplicitAuth$ cannot be violated}
	
	\newcommand{\pskTable}{\mathsf{PSKList}}
	
		% Game -- Exclude hash colls
	\gamehop[Exclude transcript hash collisions]{game:hash-coll}
	In \thisGameMath, we let the adversary lose if two distinct queries to $\ROthash$ lead to colliding outputs.
	This ensures that each transcript has a unique hash.
	When such a collision occurs, we set a new flag $\bad_\hash$ and let the game return $0$ from $\Finalize$.
	
	As in \refGameText{game:binder-coll}, we introduce a table~$\collTracker_\Thash$ to random oracle~$\ROthash$.
	Whenever it computes a hash $d = \ROthash(s)$ for some string s, we log $\collTracker_\Thash[d] \assign s$.
	This table then is used to set $\bad_\Thash$ as in \refGameText{game:binder-coll}.
	
	Analogously to \refGameText{game:binder-coll}, we can construct a reduction $\advB_2$ to the collision-resistance of $\ROthash$.
	As it simulates \thisGame, the adversary $\advB_2$ will need to make one query to its $\ROthash$ oracle for each $\ROthash$ query of $\advA$ and up to~$6$ $\ROthash$ queries for the up to 6 \emph{distinct} transcript hash values computed in a protocol step per $\Send$ query of~$\advA$; in total $\qRO+6\qSend$ queries.
	
	Therefore, we have that
	\[
	\Pr[\thisGameMath \text{sets }\bad_\hash] \leq \genAdv{\CollRes}{\ROthash}{\qRO+6\qSend}
	\]
	and it follows that
	\[ 
	\Pr[ \prevGameMath \outputs 1 ] \leq \Pr[ \thisGameMath \outputs 1 ] + \genAdv{\CollRes}{\ROthash}{\qRO+6\qSend}.
	\]
	
	
	% Game  -- Abort if the adversary guesses a PSK
	\gamehop[Abort if adversary guess a uncorrupted $\psk$]{game:abort-psk}%
	In \thisGameMath, we make the adversary lose when it queries any random oracle on a pre-shared key $\psk$ \emph{before} that key has been corrupted via $\RevLongTermKey$.
	
	We introduce some bookkeeping in order to implement this change.
	First, we add a reverse look-up table~$\pskTable$ that is maintained by the $\NewSecret$ oracle.
	When $\NewSecret(u,v,\pskid)$ samples a fresh pre-shared key $\psk$, we log the tuple under index $\psk$ as $\pskTable[\psk] \assign (u,v,\pskid)$.
	Note that the pre-shared keys might repeat, so we may have multiple entries in $\pskTable$ indexed by a single $\psk$.
	Second, we add a time log $\ROtimelog$ to the $12$ random oracles $\RO_x$.
	Each random oracle query containing a pre-shared key~$\psk$ now creates an entry $\ROtimelog[\psk] \gets \time$, where $\time$ is the counter maintained by the key exchange experiment, unless $\ROtimelog[\psk]$ already exists.
	
	The actual check whether the adversary queries any random oracle with a $\psk$ before it was corrupted is performed by the $\Finalize$ oracle.
	We set a flag $\bad_\psk$ if $\ROtimelog(\psk) \leq \revpsk_{(u,v,\pskid)}$ for any $\psk \in \ROtimelog$ and $(u,v,\pskid) \in P[\psk]$.
	If the $\bad_\psk$ flag was set during this process, the $\Finalize$ oracle returns $0$.
	
	Next, let us analyze the probability that the game is lost due to flag $\bad_\psk$ being set.
	Each random oracle query could hit one out of $\qNewSecret$ many pre-shared keys.
	Before a given pre-shared key is corrupted or queried to a random oracle, the adversary knows nothing about its value. 
	Since we assume that pre-shared keys are sampled uniformly at random from $\bits^\psklen$, the probability to hit a specific one is at most $2^{-\psklen}$.%
	\footnote{Note that at this point, we use that the pre-shared key distribution is uniform. As already mentioned before, for any distribution $\mathcal D$ on $\bits^\hashlen$, the probability would be $2^{-\alpha}$, where $\alpha$ is the min-entropy of $\mathcal D$.}
	By the union bound, we obtain that the probability that the adversary hits any of the pre-shared keys in a single random oracle query is upper-bounded by $\qNewSecret \cdot 2^{-\psklen}$. 	
	Thus, the probability that $\bad_\psk$ is set in response to any of the $\qRO$ many random oracle queries overall is limited by $\qRO \cdot \qNewSecret \cdot 2^{-\psklen}$. 
	This follows again by applying the union bound.
	
	Hence, we get by the identical-until-bad lemma, 
	\begin{align*}
		\Pr[ \prevGameMath \outputs 1 ] &\leq \Pr[ \thisGameMath \outputs 1 ] + \Pr[ \thisGameMath \text{ sets }\bad_\psk ] \\
		&\leq \Pr[ \thisGameMath \outputs 1 ] + \frac{\qRO \cdot \qNewSecret}{2^\psklen} .
	\end{align*}
	
	
	In the next two games, we change the way that partnered sessions compute their session keys, $\binder$ values, and $\Finished$ MAC tags. 
	Since we have established in Phase~1 that partnered sessions will always share the same key, we can compute these keys only once and let partnered sessions copy the results. 
	This will make it easier to maintain consistency between partners as we change the way we compute keys and tags.
	This approach follows the tight key exchange security proof techniques of Cohn-Gordon et al.~\cite{C:CCGJJ19}.

	% Game -- Log session keys
	\gamehop[Log session keys and MAC tags]{game:log-keys}
	First, we will store all session keys in a look-up table $\skeyTable$ under their session identifiers.
	Sessions will be able to use this table to easily check if they share a session identifier with another honest session and thus share a key with a partner.
	
	Honest sessions $\pi_u^i$ in the initiator role will derive the keys $\ets$, $\eems$, and $\rms$ before their partners. 
	In \thisGame, when an initiator session accepts in stage~$1$ ($\ets$), $2$ ($\eems$), or~$8$ ($\rms$) it creates a new entry in $\skeyTable$, i.e.,
	\[
	\skeyTable[\pi_u^i.\sid[s]] \assign \pi_u^i.\skey[s]
	\]
	for $s \in \{1,2,8\}$.
	Honest responder sessions $\pi_v^j$ will derive the keys $\shtk$, $\chtk$, 
	$\cats$, $\sats,$ and $\ems$ before their partners. 
	These sessions also log their keys in $S$ under the appropriate session identifier:  
	\[
	\skeyTable[\pi_v^j.\sid[s]] \assign \pi_v^j.\skey[s]
	\]
	for $s \in \{3,\dotsc,7\}$.
	
	Note that no two sessions will ever log keys in table $\skeyTable$ under the same $\sid$.
	From $\Sound$, we know that only one initiator and one responder session may have the same session identifier $\sid[s]$ in any stage~$s$.
	Note that for the replayable stages~1 and~2 ($\ets$ and $\eems$) we only log once because the messages will only be logged by the initiator that output the replayed messages and not by the receivers that are receiving them.
	
	We also store $\binder$, $\cfin$ and~$\sfin$ MAC tags.
	When any honest session queries $\RO_x$ with $x \in \{\binder, \cfin, \sfin\}$, it logs the response in a second look-up table, $\macTable$, indexed by $x$ and the inputs to $\RO_x$.
	That is, for a query $(\psk, \dhe, d_1, d_2)$ to $\RO_{\sfin}$, we log
	\[
	\macTable[ \sfin, \psk, \dhe, d_1, d_2 ] \assign \RO_{\sfin}(\psk, \dhe, d_1, d_2).
	\]
	
	\noindent
	Since \thisGame only introduces book-keeping steps, we have that
	\[
	\Pr[ \prevGameMath \outputs 1 ] = \Pr[ \thisGameMath \outputs 1 ].
	\]
	
	\gamehop[Copy session keys and MAC tags from partnered session]{game:copy-keys}
	In this game, we change the way the sessions compute their keys and MAC tags.
	Namely, if a session has an honest partner in stage~$s$, instead of computing a key itself, it copies the stage-$s$ key already computed by the partner via the table $\skeyTable$ introduced in \prevGameText.
	Concretely, the sessions compute their keys depending on their role as follows.
	
	\paragraph{Honest server sessions.}
	An honest server session~$\pi_v^j$, upon receiving $(\mathlist{\CH, \CKS,\CPSKtls})$, sets its session identifier for stages~$1$ ($\ets$) and~$2$ ($\eems$).
	It then checks whether keys have been logged in $\skeyTable$ under $\pi_v^j.\sid[1]$ and $\pi_v^j.\sid[2]$.
	If such log entries exist, then $\pi_v^j$ has an honest partner in stages~$1$ and~$2$, and copies the keys $\ets$ and $\eems$ from~$\skeyTable$ when they would instead be computed directly.
	
	Analogously, upon receiving $\CF$, $\pi_v^j$ uses $\skeyTable$ to check whether there is an honest client session that shares the same stage-$8$ ($\rms$) session identifier $\pi_v^j.\sid[8]$, and it copies the $\rms$ key if this is the case.
	If there are no entries in $\skeyTable$ under the appropriate session identifiers, $\pi_v^j$ proceeds as in \prevGameText and computes its keys using the random oracles.
	
	\paragraph{Honest client sessions.}
	An honest client session $\pi_u^i$, upon receiving $(\mathlist{\SH, \SKS, \SPSKtls})$, sets its session identifiers for stages $3$--$7$, which identify the keys $\shtk$, $\chtk$, $\cats$, $\sats$ and $\ems$. 
	It then searches for entries in $\skeyTable$ indexed by $\pi_u^i.\sid[s]$ for $s \in \{3,\dotsc,7\}$. 
	If these entries are present for stage~$s$, then $\pi_v^i$ copies the stage-$s$ keys from $\skeyTable$ instead of computing them itself.
	Otherwise, $\pi_u^i$ proceeds as in \prevGameText and computes the keys using the random oracle in each case.
	%
	
	\paragraph{Computation of MAC tags.}
	Finally, all honest sessions (both client and server) which would query $\RO_x$ to compute $x \in \{\binder, \cfin, \sfin\}$ in \prevGameText first check the look-up table $\macTable$ to see if their query has already been logged.
	If so, they copy the response from $\macTable$ instead of making the query to $\RO_x$.
	
	It remains to argue that the procedure of copying the keys in partnered sessions described in this game is consistent with computing the keys in \prevGameText.
	Recall that sessions which are partnered in stage~$s$ must agree on the stage-$s$ key, since the $\Sound$ predicate (Property~6) cannot be violated.	
	Consider a session~$\pi_u^i$ which accepts the stage-$s$ key $\pi_u^i.\skey[s]$. 
	By $\Sound$, any other session~$\pi_v^j$ in \prevGameText which accepts in stage~$s$ with $\pi_v^j.\sid[s] = \pi_u^i.\sid[s]$ must set its stage-$s$ key equal to $\pi_u^i.\skey[s]$.
	Although in \thisGame the session $\pi_v^j$ may copy $\pi_u^i.\skey[s]$ from table~$\skeyTable$ instead of deriving it directly, the value of $\pi_v^j.\skey[s]$ does not change between the two games.
	
	Sessions may also copy queries from look-up table~$\macTable$ instead of making the appropriate random oracle query themselves.
	However, table~$\macTable$ simply caches the response to random oracle queries and does not change them.  
	Hence, the view of the adversary is identical.
	This implies that
	\[ 
		\Pr[ \prevGameMath \outputs 1 ] = \Pr[ \thisGameMath \outputs 1 ]. 
	\]
	
	With the next two games, we finalize Phase~2.
	First, we postpone the sampling of the pre-shared key to the $\RevLongTermKey$ oracle such that only corrupted sessions hold pre-shared keys.
	As a consequence of this change, we can no longer compute session keys and MAC tags using the random oracles.
	We will instead sample these uniformly at random from their respective range and only program the random oracles upon corruption of the corresponding pre-shared key.
	After this change, we can show that in order to break explicit authentication, the adversary must predict a uniformly random $\Finished$ MAC tag, which is unlikely.
	
	% postpone PSK sampling until after corruption
	\gamehop[Postpone PSK sampling until after corruption]{game:mac-forgery-prep}
	In this game, we postpone the sampling of pre-shared keys from the $\NewSecret$ oracle to the $\RevLongTermKey$ oracle (if the pre-shared key gets corrupted) or the $\Finalize$ oracle (if the key remains uncorrupted).
	
	Since we now do not have a $\psk$ anymore for uncorrupted sessions, we cannot use the random oracle to compute keys or MAC tags in those sessions, but instead sample them uniformly at random.
	If the corresponding pre-shared key is corrupted later and a $\psk$ is chosen (in $\RevLongTermKey$), we will retroactively program the affected random oracles to ensure consistency.
	
	Concretely, we change the implementation of the game as follows.
	When $\NewSecret$ receives a query $(u,v,\pskid)$, we set $\pskeys[ (u,v,\pskid) ]$ to a special symbol $\star$ instead of a randomly chosen pre-shared key.
	The $\star$ serves as a placeholder and signalizes that the $\NewSecret$ oracle already received a query $(u,v,\pskid)$, but no $\psk$ has been chosen yet.
	We add $(u,v,\pskid)$ to the set~$\pskTable[\star]$ to keep track of all tuples with an undefined $\psk$.
	
	We let honest sessions whose pre-shared key has not been sampled (yet) but equals $\star$ sample their session keys as well as $\binder$ and $\Finished$ MAC tags uniformly at random.
	Due to the changes introduced in \refGameText{game:copy-keys} we do not need to ensure consistency when sampling, as we sample each value once and partnered sessions copy the suitable value from the tables~$\skeyTable$ and~$\macTable$.
	(When sessions would log $\MAC$ tags in $\macTable$ under their pre-shared keys in \prevGameText, those with no pre-shared key instead use the tuple $(u, v, \pskid)$ in this game.)
	We further log the respective random oracle query that sessions would normally have used for the computation in a look-up table $\programmingTable_x$ for later programming of the respective random oracle $\RO_x$.
	Sessions which would log their $\RO$-derived values in tables~$\skeyTable$ and~$\macTable$ now log their randomly chosen values instead. 
	That is, if a session in \prevGameText would issue a query $(\star, \dhe, \mathit{ctxt})$ (where $\dhe$ might be $\bot$) to random oracle $\RO_x$ to compute a value $k$, in \thisGame it chooses $k$ uniformly at random from $\RO_x$'s range and logs 
	\[
		\programmingTable_x[ (u,v, \pskid) , \dhe , \mathit{ctxt} ] \assign k
	\]
	in the look-up table $\programmingTable_x$, where $(u,v, \pskid)$ uniquely identifies the used $\psk$.
	Note that the table $\programmingTable_x$ is closely related to the random oracle table $\roTable_x$ for $\RO_x$.
	Table $\programmingTable_x$ is always used when there is no $\psk$ defined for a session, i.e., it has not (yet) been corrupted.
	Therefore, we need to make sure that if the $\psk$ (identified by $(u,v,\pskid)$) gets corrupted we are able to reprogram $\RO_x$.
	Using $\programmingTable_x$ we can upon corruption of the pre-shared key associated with $(u,v,\pskid)$ efficiently look-up the entries we need to program from $\programmingTable_x$ and transfer them to the random oracle table~$\roTable_x$ after $\psk$ has been set.
	We will discuss the precise process below when we describe how to adapt the $\RevLongTermKey$ oracle.
	
	We must be particularly careful when $x = \binder$, because we still wish to set the $\bad_{\binder}$ flag when two randomly chosen binder values collide.
	Therefore, honest sessions still record the sampled binder values in list~$\collTracker_\binder$, so that the $\bad_{\binder}$ flag is set as before.
	This ensures that the probability of setting the flag does not change.
	
	We also need to adapt the corruption oracle $\RevLongTermKey$.
	Upon a query $(u,v,\pskid)$ for which $\pskeys[ (u,v,\pskid) ] = \star$, we perform the following additional steps:
	First, we sample a fresh pre-shared key~$\psk \sample \KEpskeyspace$ and update $\pskeys$, i.e., set $\pskeys[ (u,v,\pskid) ] \assign \psk$.
	Next, we need to reprogram the random oracles using the lists $R_x$ to ensure consistency.
	Thus, for all $x$ we update the random oracle tables $\roTable_x$ for $\RO_x$ using $\programmingTable_x$.
	For every entry $\programmingTable_x[((u,v, \pskid) , \dhe , \mathit{ctxt})] = k$, we set 
	\[
	\roTable_x[ \psk, \dhe , \mathit{ctxt}  ] \assign k
	\]
	where $\roTable_x$ is the random oracle table of $\RO_x$.
	Lastly, we remove $(u,v,\pskid)$ from the set $\pskTable[\star]$ and add it to $\pskTable[\psk]$.
	
	To be able to still set $\bad_\psk$, we also make sure that in the $\Finalize$ procedure every pre-shared key is defined before the check against the random oracle time log $\ROtimelog$ introduced in \refGameText{game:abort-psk}.
	We sample a pre-shared key for every tuple $(u,v,\pskid) \in P[\star]$, setting $\pskeys[ (u,v,\pskid) ] \sample \KEpskeyspace$, and update the reverse look-up table $\pskTable$ accordingly.
	As a result, also uncorrupted sessions now have a pre-shared key defined and we can check the condition for $\bad_\psk$ being set as introduced in \refGameText{game:abort-psk}.
	
	The changes introduced in \thisGame are unobservable for the adversary as it never queries the random oracle for an uncorrupted pre-shared key, as otherwise the game would be aborted due to $\bad_\psk$ introduced in \refGameText{game:abort-psk}.
	It hence does not matter whether the pre-shared key is already set before or upon corruption, because from the view of the adversary the keys (and the pre-shared key) are uniformly random bitstrings anyway up to this point.
	Upon corruption of a pre-shared key, we make sure by reprogramming the random oracle that all session keys and MAC tag computations are consistent with sessions that would have otherwise used this pre-shared key but derived all session keys and MAC tags without it.
	The change to the $\Finalize$ procedure does not affect the view of the adversary as it only retroactively defines keys on which the adversary cannot get any information about anymore.
	Consequently,
	\[
	\Pr[ \prevGameMath \outputs 1 ] = \Pr[ \thisGameMath \outputs 1 ].
	\]
	
	% Game -- exclude mac forgeries
	\gamehop[Exclude that honest sessions accept without a partner]{game:honest-mac-forgery}
	In this game, we set a flag~$\bad_{\mathrm{MAC}}$ and return $0$ from $\Finalize$ if any session with an uncorrupted pre-shared key accepts stage~$5$ ($\chtk$) as initiator, or stage~$8$ ($\rms$) as responder, without having a partnered session.
	Formally, we set $\bad_{\mathrm{MAC}}$ if there is a session $\pi_u^i$ such that $\pi_u^i.\taccepted[s] < \revpsk_{(u,v,\pi_u^i.\pskid)}$ with $v = \pi_u^i.\peerid$ and 
	\[
	s = \begin{cases}
		5 & \mbox{if } \pi_u^i.\role = \initiator \\
		8 & \mbox{if } \pi_u^i.\role = \responder
	\end{cases}
	\]
	and there is no session $\pi_v^j$ with $\pi_u^i.\sid[s] = \pi_v^j.\sid[s]$ when $\pi_u^i$ accepts stage~$s$.
	
	Let us analyze the probability $\Pr[ \thisGameMath \text{ sets } \bad_\mathrm{MAC}]$.
	Consider a session~$\pi_u^i$ which triggers the $\bad_{\mathrm{MAC}}$ flag. 
	In the following analysis, let $\pi_u^i$ be an initiator.
	For responder sessions the arguments are analogous.
	The pre-shared key of session~$\pi_u^i$ is uncorrupted, which means that by the changes of \refGameText{game:mac-forgery-prep} it has not been sampled. 
	Therefore $\pi_u^i$ either samples the $\ServerFinished$ MAC tag uniformly at random or copies it from table~$\macTable$ (in which case the MAC tag was uniformly sampled and logged by another honest session).
	
	First observe that session~$\pi_u^i$ will not copy the $\ServerFinished$ MAC tag from table~$\macTable$ as this would imply that $\pi_u^i$ is partnered when it accepts in stage~$5$. 
	This in turn contradicts that $\pi_u^i$ has triggered flag~$\bad_{\mathrm{MAC}}$. 
	Namely, if $\pi_u^i$ would be able to copy the $\ServerFinished$ MAC tag from table~$\macTable$ there must have been another honest session that computed the same $\ServerFinished$ MAC, i.e., using the same tuple $(u,v,\pskid)$, DHE secret, and transcript hash. 
	Recall that the session identifier of stage~$5$ contains both the $\ServerFinished$ message and the transcript hashed to computed the $\ServerFinished$ MAC tag. 
	Further, we have that transcript hashes are unique due to \refGameText{game:hash-coll}. 
	This implies that the session that logged the $\ServerFinished$ MAC tag in $\macTable$ needs to have the same stage-$5$ session identifier than $\pi_u^i$ meaning $\pi_u^i$ would be partnered in stage~$5$.

	Thus, if $\pi_u^i$ triggers $\bad_{\mathrm{MAC}}$, it must have sampled its $\ServerFinished$ MAC tag at random and the received $\ServerFinished$ message will match this tag with probability no more than~$2^{-\hashlen}$.

	Thus the probability that $\pi_u^i$ triggers the flag $\bad_{\mathrm{MAC}}$ is bounded by $2^{-\hashlen}$.
	A union bound over all sessions gives 
	\[
	\Pr[ \thisGameMath \text{ sets } \bad_\mathrm{MAC}] \leq \frac{\qSend}{2^{\hashlen}} .
	\]
%	\fg{Why is there a factor~$2$? My guess is because of initiator + responder, but $q_\Send$ already accounts for all of them.}
%	\dd{My line of thought was either a intiator or a responder triggers $\bad_{\mathrm{MAC}}$ each Pr can be analysed as above, so a factor of $2$ then. But I guess dropping it is fine, if we say wlog. we consider intiators.}
	
	Overall, we get by the identical-until-bad-lemma
	\begin{align*}
		\Pr[ \prevGameMath \outputs 1 ] &\leq \Pr[ \thisGameMath \outputs 1 ] + \Pr[ \thisGameMath \text{ sets } \bad_\mathrm{MAC}] \\ &\leq \Pr[ \thisGameMath \outputs 1 ] + \frac{\qSend}{2^{\hashlen}}.
	\end{align*}

	\paragraph{Conclusion of Phase~2.}
	At this point, we argue that in \thisGame and any subsequent games, adversary $\advA$ cannot violate the $\ExplicitAuth$ predicate without also causing $\Finalize$ to return $0$. 
	To this end, we argue that $\ExplicitAuth = \true$ holds with certainty from \thisGame on.
	
	The predicate $\ExplicitAuth$ is set to $\false$ if there is a session~$\pi_u^i$ accepting an explicitly authenticated stage~$s$, whose pre-shared key was not corrupted before accepting the stage $s' \geq s$ in which it received (perhaps retroactively) explicit authentication, and (1)~there is no honest session~$\pi_v^j$ partnered to $\pi_u^i$ in stage~$s'$, or (2)~there is an honest partner session~$\pi_v^j$ for $\pi_u^i$ in stage~$s'$ but it accepts with a peer identity $w \neq u$, with a different pre-shared key identity than $\pi_u^i$, i.e. $\pi_v^j.\pskid \neq \pi_u^i.pskid$, or with a different stage-$s$ session identifier, i.e. $\pi_v^j.\sid[s] \neq \pi_u^i.\sid[s]$.
	
	Recall that initiator (resp.\ responder) sessions receive explicit authentication with acceptance of stage~$5$ (resp.\ stage~$8$) meaning that all previous stages~$1$--$4$ (resp.\ stages~$1$--$7$) receive explicit authentication retroactively and all future stages~$6$--$8$ upon their acceptance.
	From \thisGame, we have that any initiator session $\pi_u^i$ accepting stage~$5$ (resp.\ any responder session accepting stage $8$)  with uncorrupted $\psk$ must have a partnered session in that stage.
	Consequently, case (1) is impossible to achieve.
	

	We next address the possibility of case (2).
	To achieve explicit authentication for stage $s \leq 8$, a responder session must have accepted stage~$8$.
	From \thisGame on, we know that $\pi_u^i$  must have a partner with the same stage~$8$ session identifier.
	Observe that the transcripts contained in $\pi_u^i$'s session identifiers for all stages are ``sub-transcripts'' of the transcript contained in the session identifier of stage~$8$.
	Therefore the partner must also have the same stage~$s$ session identifier.
	Property~5 of the $ \Sound $ predicate then ensures that all partnered sessions agree on the peer identity and the pre-shared key identity, so $\ExplicitAuth$ is not violated by session $\pi_u^i$.
	The same property holds for initiator sessions accepting stages $s \leq 5$.
	So $\ExplicitAuth$ can only be violated if an initiator session's stage-$5$ partner accepts in stage $s > 5$ with a different peer identity, pre-shared key identifier, or session ID. 
	Since peer and pre-shared key identifiers do not change after they are set, only the session identifiers may not match in stage~$s$.
	The ``sub-transcripts'' of stage~$6$ ($\cats$) and~$7$ ($\sats$) session identifiers are identical to those of stage~$5$, so a partner in stage $5$ will also be a partner in stages~$6$ and~$7$.
	Then the only way to violate predicate $\ExplicitAuth$ is to convince the stage-$5$ partner, a responder session, to accept a forged $\ClientFinished$ message and accept stage~$8$.
	This is impossible because the partner will verify the received $\ClientFinished$ message against the message sent by $\pi_u^i$, which it copies from table~$\macTable$.
	It follows that no session, responder or initiator, can violate the $\ExplicitAuth$ predicate.
	
	\subsection*{Phase~3: Ensuring that the Challenge Bit is Idependently Random}
	
	\gamehop{game:abort-dhe}
	In this game, we rule out that the adversary manages to guess the DHE secret of two honestly partnered session to learn about the keys they are computing. Here, we only look at those session that have a corrupted pre-shared key, because we already ruled out in \refGameText{game:abort-psk} that the adversary learns something about the keys computed by these sessions. To that end, we add another flag~$\bad_\dhe$ to the game and return $0$ from $\Finalize$ when it is set.
	Flag $\bad_\dhe$ is set if the adversary ever queries a random oracle
	\[
	\RO_x(\psk, \dhe, \ROthash(\sid[s]))
	\]
	for $(x,s) \in \{\mathlist{(\chtk,3), (\shtk,4), (\sfin, 5) (\cats,5), (\sats,6), (\ems,7), (\cfin, 8), (\rms,8)}\}$ such that
	\begin{itemize}
		\item $\psk$ is corrupted, i.e., the adversary made a prior query~$\RevLongTermKey(\mathlist{u,v,\pskid})$ with $\pskeys[(\mathlist{u,v,\pskid})] = \psk$,
		\item there are honest sessions $\pi_u^i$ and $\pi_v^j$ that are contributively partnered in stage~$s$ with $\pi_v^j.\rolecid[\pi_u^i.\role][s] = \pi_u^i.\rolecid[\pi_u^i.\role][s] = (\mathlist{\CH, \CKS, \CPSKtls, \SH, \SKS, \SPSKtls,\dotsc})$, and
		\item $\dhe = g^{xy}$ such that $\CKS = g^x$ and $\SKS = g^y$.\footnote{Note that the game knows the exponents $x$ and $y$ used by the sessions, but the reduction constructed in the remainder will not.}
	\end{itemize}
	
	We bound the probability of flag $\bad_\dhe$ being set via a reduction $\advB_\dhe$ to the strong Diffie--Hellman assumption in group~$\G$.
	Reduction~$\advB_\dhe$ simulates \thisGame for $\advA$, and it wins the strong Diffie--Hellman whenever the simulated game would set the $\bad_{\dhe}$ flag.


\begin{definition}
  \label{def:stDH}
Let $\G$ be a group of order $p$ generated by $g$. We define
\[
\genAdv{\stDH}{\G}{t_{\advB_\dhe}, 2 \qRO} := \Pr\left[  g^{ab} \getsr \advB_\dhe^{\stDH_a(\cdot, \cdot)}(g^{a}, g^{b}) : a,b \getsr \Z_{p} \right]
\]
where $\stDH_a$ is a special ``fixed-exponent DDH oracle'' that on input $(B,C)$ returns $1$ if and only if $C = B^a$.
\end{definition}

 %        Recall that in the strong Diffie--Hellman problem in a group we consider an adversary $\advB_\dhe$ that receives as input $A=g^a$ and~$B = g^b$, where $a,b \rand \Z_{p}$. We defined
 % the challenge is to compute $C = g^{ab}$, however it additionally is provided with a  .


	
	\paragraph{Construction of reduction $\advB_\dhe$.}
	The reduction~$\advB_\dhe$ gets as input a strong DH challenge $(A = g^a, B = g^b)$ as well as access to the oracle $\stDH_a$ for the Decisional Diffie--Hellman problem with the first argument fixed. 
	%	Given inputs $C:=g^c$ and $W$ for any $c \in \Z_p$, $\stDH_a(C, W)$ returns $\true$ if and only if $W = g^{ac} = C^a$.
	Adversary $\advB_\dhe$ then honestly executes the $\Initialize$, $\RevSessionKey$, $\Test$, and $\NewSecret$ oracles as \thisGame would, managing all game variables itself.
	We explain in more detail how $\advB_\dhe$ answers $\Send$, $\RevLongTermKey$, and random oracle queries. 
	
	When $\advA$ makes a query to the $\Send$ oracle, $\advB_\dhe$ delivers the message to a protocol session in the same way as \thisGame.
	However, the sessions themselves handle messages quite differently.
	At a high level, $\advB_\dhe$ embeds its strong DH challenges into the key shares of every initiator session and every partnered responder session.
	When $\bad_{\dhe}$ is triggered, $\advB_\dhe$ learns the Diffie--Hellman secret $\dhe$ associated with two of these embedded key shares, and it can extract a solution to the strong DH challenge using some basic algebra.
	However, $\advB_\dhe$ must take care to appropriately program random oracles queries after corruptions, since it cannot compute Diffie--Hellman secrets for embedded key shares as it does not know the corresponding exponents.
Next, we describe how client and server sessions are implemented in \thisGame.
	
		But first we explain the (constant-time accessible) look-up tables that are used (or defined) by reduction~$\advB_\dhe$ to ensure an efficient implementation:
		\begin{itemize}
			\item The look-up table~$\ksRandomizer$ is maintained for all sessions. It holds the random exponent $\tau$ used by the honest sessions to randomize their key share $G$, indexed by the session's nonce and key share $(r,G)$ (see the implementation of the session for further details). To identify a session uniquely we use its nonce $r$ and key share $G$ as the index.
			\item Each random oracle $\RO_x$ maintains a look-up table~$\dheTable_x$. For each query $RO_x(\psk, Z, d)$, the table stores the group element $Z$ indexed by $\psk$ and $d$.
			\item Each random oracle $\RO_x$ maintains a look-up table~$\rndTable_x$. It holds a tuple $(\tau,\tau',\mathit{ctxt},\mathit{key})$ 
%\old{of a randomizer for each of the initiator and the responder, a transcript $\mathit{ctxt}$ and a key $\mathit{key}$}
 indexed by the pair $(\psk, d)$.
% \old{of pre-shared key and hash}
			The table holds all necessary information that is required to reprogram of the random oracle $\RO_x$.
			The fields $\psk$ and $\mathit{key}$ can hold special values.
			If a $\psk$ is uncorrupted, we cannot log the information under it because it is not defined.
			Therefore, we can use the tuple $(u,v,\pskid)$ uniquely identifying $\psk$ instead.
			Moreover, $\mathit{key}$ can sometimes be an empty field, because reprogramming of that value will never occur. When this field is empty, it will not be accessed as we instead use the remaining information of $\rndTable_x$ to solve the stDH challenge. 
			See the remainder of the proof for details.
		\end{itemize}

	
	\paragraph{Implementation of honest server sessions.}
	Consider any server session $\pi_v^j$.
	
	\begin{enumerate}
		\item Upon receiving $(\CH, \CKS, \\CPSKtls)$, the reduction~$\advB_\dhe$ first checks whether $\pi_v^j$ has an honest partner in stages~$1$ ($\ets$) and~$2$ ($\eems$) by checking for entries indexed by $\pi_v^j.\sid[1]$ and $\pi_v^j.\sid[2]$ in the look-up table $\skeyTable$ introduced in \refGameText{game:log-keys}.
		If no such entries exist, then $\advB_\dhe$ answers this and all future $\Send$ queries just as specified in \thisGame.
		For the rest of the discussion, we assume the entries do exist.
		
		Session~$\pi_v^j$ generates its key share $\SKS$ by randomizing the challenge key share~$B$. 
		Namely, it chooses  a randomizer~$\tau_v^j \sample \Z_p$ uniformly at random and sets $Y \assign B \cdot g^{\tau_v^j}$.
		Then, it logs $\tau_v^j$ under index $(r_S, Y)$ in the look-up table~$\ksRandomizer$.
		%		
		\item Before $\pi_v^j$ outputs $(\SH,\SKS, \SPSKtls)$, it computes the keys $\chtk$ and $\shtk$.
		By \refGameText{game:mac-forgery-prep}, these keys are sampled randomly when $\psk$ is uncorrupted and computed using $\RO_{\chtk}$, resp.\ $\RO_{\shtk}$ otherwise.
		In both cases, $\advB_\dhe$ needs to know the Diffie--Hellman secret $\dhe$ to log in table~$\programmingTable_x$ or to query~$\RO_x$ for $x \in \{\chtk, \shtk\}$.
		However, $\advB_\dhe$ cannot compute $\dhe$ because it does not know the discrete logarithms of either $\CKS$ or $\SKS$.
		
		Therefore, $\advB_\dhe$ needs to compute the keys without knowing the $\dhe$ key using the control over the random oracles.
		%		We use $\chtk$ as an example of how the session can do this; it will follow the same process for all of the values $\chtk$, $\shtk$, $\sfin$, $\cats$, $\sats$, $\ems$, $\cfin$, and $\rms$ computed by $\pi_v^j$.
		
		%		First, $\advB_\dhe$ checks whether the $\bad_{\dhe}$ flag will be set by session $\pi_v^j$. 
		If the pre-shared key has been corrupted, the adversary could potentially have already queried the random oracle $\RO_{\chtk}$ with the query $\pi_v^j$ should make. 
		To that end, $\advB_\dhe$ first checks whether the corresponding query for~$\chtk$ was already made to~$\RO_{\chtk}$.
		Concretely, $\advB_\dhe$ computes the context hash~$d = \RO_\Thash(\CH \concat \dotsb \concat \SPSKtls)$ and checks for a suitable $\RO_{\chtk}$ query using the look-up table $\dheTable_{\chtk}[\psk, d]$ maintained in $\RO_{\chtk}$ (see above for the definition).
		Reduction~$\advB_\dhe$ queries $\stDH_a(Y, Z \cdot Y^{-\tau_u^i})$ for all $Z \in \dheTable_{\chtk}[\psk, d]$, where $\tau_u^i$ is the randomizer used by the honest partner of $\pi_v^j$, which can be looked up from $\ksRandomizer[r_C,X]$ using $\pi_u^i$'s nonce and key share.
		(Although this may cause several $\stDH_a$ queries in response to a single $\Send$ query, $\advB_\dhe$ is still efficient because it only checks random oracle queries whose context is $d$, and due to the lack of both nonce/group element and hash collisions $d$ is unique to session~$\pi_u^i$ and its partner. Therefore each entry in $\dheTable_{\chtk}[\psk, d]$ will be checked at most twice over the course of the entire reduction.)
		
		If any one of these queries is answered positively, we have by the definition of $\stDH_a$ that $Z \cdot Y^{-\tau_u^i} = Y^a$, which implies that $Z = Y^{a + \tau_u^i} = X^{b + \tau_v^j}$ by definition of $Y$ and $X$, which was computed by the honest partner $\pi_u^i$ that has output the $\CH$ message received by $\pi_v^j$.
		This exactly is the $\dhe$ value that $\pi_v^j$ would have computed if we would have known the discrete logarithm of $B$.
		Hence, we have found the right $Z$ value and only need to derandomize it to win the challenge.
		Therefore, we let $\advB_\dhe$ submit the value
		\[
		Z \cdot Y^{-\tau_u^i} \cdot A^{-\tau_v^j} = Y^a \cdot A^{-\tau_v^j} = (g^a)^{b + \tau_v^j} \cdot (g^a)^{-\tau_v^j} = g^{ab}
		\]
	 to the $\Finalize$ oracle as a solution to the strong Diffie--Hellman problem.
		
		Observe that if $\bad_\dhe$ is set due to a query to $\ROchtk$ in \thisGame, there is a random oracle query such that one of the above $\stDH_a$ queries will be answered positively. Thus, $\advB_\dhe$ will win if $\bad_{\dhe}$ is set.
		We do the same for $\shtk$ with $\RO_{\shtk}$.
		
		If in the above process no query is answered positively, i.e., $\bad_{\dhe}$ will also not be set, then $\pi_v^j$ samples the key $\chtk \sample \KEkeyspace[3]$ itself and logs the following information so that future $\RO$ queries can be answered appropriately:
		\[
		\rndTable_{\chtk}(\psk, d = \hash(\CH \concat \dotsb \concat \SPSKtls) ) \assign \left(  \tau_u^i, \tau_v^j, (\CH \concat \dotsb \concat \SPSKtls), \bot \right).
		\]
		Again, we do the same for $\shtk$.
		
		If $\psk$ is not corrupted, then $\bad_{\dhe}$ cannot possibly have been set and we do not need to worry about consistency with earlier random oracle queries.
		Therefore, we do not need to do the process described above and immediately sample $\chtk$ and $\shtk$ randomly as in \thisGame.
		It logs the keys in table~$\skeyTable$ under their respective session identifiers, which do not contain $\dhe$ or any unknown values.
		In \thisGame, we added entries to $\programmingTable_{\chtk}$ and $\programmingTable_{\shtk}$ in order to program future random oracle queries upon corruption.
		The reduction cannot do this here as it does not know $\dhe$; instead, it logs
		\[
		\rndTable_x[((u, v, \pskid), d = \hash(\CH \concat \dotsb \concat \SPSKtls) )] \assign \left(  \tau_u^i, \tau_v^j, (\CH \concat \dotsb \concat \SPSKtls), \bot \right).
		\]
		for $x \in \{\chtk, \shtk\}$.
		This will allow $\advB_\dhe$ to win if a later $\RevLongTermKey$ or random oracle query triggers $\bad_{\dhe}$.
		%		
		\item To compute the $\ServerFinished$ message $\advB_\dhe$ proceeds exactly as in Step~2 except that it uses the random oracle $\RO_{\sfin}$ and context $\CH \concat \dotsb \concat \EE$ through the $\EncryptedExtensions$. 
		Also, the $\ServerFinished$ message is computed first by the server, so $\advB_\dhe$ does not check table $\skeyTable$ or $\macTable$ for any entries.
		Reduction~$\advB_\dhe$ also cannot log the inputs to random oracle query $\RO_{\sfin}$ in table $\macTable$ (as done since game \refGameText{game:log-keys}) because it does not know $\dhe$.
		Instead, it logs the derived value of $\sfin$ in table~$\macTable$ and replaces $\dhe$ in the index of $\macTable$ by $\left(  \tau_u^i, \tau_v^j, (\CH \concat \dotsb \concat \EE) \right)$.
		That is, if it computes $\sfin$ for inputs $\psk$, $d_1$, and $d_2$, it logs
			\[
			\macTable[ \sfin, \psk, (\tau_u^i, \tau_v^j, (\CH \concat \dotsb \concat \EE)), d_1, d_2 ] \assign \sfin.
			\]
			That way, it is possible to identify $\dhe$ without knowing it.
		For $\sfin$, we keep the same notation for the sets $\dheTable_x$, $\rndTable_x$ and $\roTable_x$ numbered as the corresponding random oracle $\RO_x$.
		%		
		\item Reduction $\advB_\dhe$ proceeds exactly as for $\sfin$ above,
		except that we again use different random oracles and the context $\cid_\cats = \CH \concat \dotsb \concat \SF = \cid_\sats = \cid_\ems$, where $\cid_x$ denotes transcript contained in the contributive identifier which is prefixed by ``$x$'', and thus the hash $d = \RO_\Thash(\CH \concat \dotsb \concat \SF)$.
		With respect to random oracles, we have $\RO_{\cats}$ for $\cats$, $\RO_{\sats}$ for $\sats$ and $\RO_{\ems}$ for $\ems$, respectively.
		Reduction~$\advB_\dhe$ logs the keys in table~$\skeyTable$ under their respective session identifiers, which do not contain $\dhe$ or any unknown values.
		
		After this is done, $\pi_v^j$ outputs $(\EE, \SF)$.
		%		
		\item Upon receiving $\CF$, $\advB_\dhe$ looks for a suitable entry for $\cfin$ in $\macTable$.
		If there is a value $\cfin$ consistent with $\pi_v^j$'s view, $\advB_\dhe$ terminates the session as specified if $\CF$ does not match the looked-up value of $\cfin$.
		Otherwise, $\advB_\dhe$ continues to compute $\rms$.
		To this end, $\advB_\dhe$ checks whether there is an entry in $\skeyTable$ that matches the stage-$8$ session identifier of $\pi_v^j$, if yes $\pi_v^j$ simply copies that entry.
		If not, first observe that if there is no entry in $\skeyTable$ there is no honest stage-$8$ partner, which implies that $\psk$ needs to be corrupted as otherwise the game would have been aborted due to $\bad_{\mathrm{MAC}}$ introduced in \refGameText{game:honest-mac-forgery}.
		Therefore, the adversary also would be allowed to query $\RO_\rms$ to compute $\rms$.
		Thus, $\advB_\dhe$ needs to check whether the value for $\rms$ is already set.
		Here, we need to distinguish two cases.
		Namely, whether there is an honest contributive stage-$3$ partner or not.
		
		First note that as described in Step~$1$, $\advB_\dhe$ does not embed its challenge in $\SKS$ if there is no honest session output the $\ClientHello$ received, i.e., there is no honest contributive stage-$3$ partner.
		Therefore, here $\advB_\dhe$ can simply implement $\pi_v^j$ as specified in \thisGame.
		
		In case there is an honest contributive stage-$3$ partner, then $\advB_\dhe$ proceeds as described in Step~2 for oracle $\RO_\rms$ and context hash $d = \RO_\Thash(\cid_\rms) = \RO_\Thash(\CH \concat \dotsb \concat \CF)$ to check whether the adversary already solved the stDH challenge for $\advB_\dhe$.
		Note that the stage-$3$ session identifier uniquely defines the $\dhe$ key, thus if there is an honest partner and there is a respective $\RO_\rms$ query, the adversary has to break stDH to submit the query.
	\end{enumerate}
	
	\paragraph{Implementation of honest client sessions.}
	Consider any client session $\pi_u^i$.
	
	\begin{enumerate}
		\item The reduction~$\advB_4$ proceeds exactly as in \thisGame until the session chooses its key share.
		Instead of choosing a fresh exponent as specified in \autoref{fig:tls-handshake}, it chooses a value $\tau_u^i \sample \Z_p$ uniformly at random and sets $X \assign A \cdot g^{\tau_u^i}$ as its key share in the $\ClientKeyShare$ message.
		Further, it logs $\tau_u^i$ in $\ksRandomizer$ indexed with $(r_C, X)$.
		The rest is exactly as specified in \thisGame. 
		That is, it computes $\ets$ and $\eems$ and outputs $(\CH, \CKS, \\CPSKtls)$.
		%		
		\item 	Upon receiving $(\SH,\SKS, \SPSKtls)$, $\pi_u^i$ checks whether there is an entry 
		\[
		\skeyTable[("\chtk",\CH, \dotsc, \SPSKtls)] \neq \bot.
		\]
		If this is the case, $\pi_u^i$ knows that there is an honest stage-$3$ partner, and it copies all the keys stored under $\pi_u^i$'s session identifier as defined in \thisGame.
		If there is no suitable entry, $\advB_\dhe$ faces the problem that it already ``committed'' to not knowing the discrete logarithm of $\pi_u^i$'s key share $X$ by embedding $A$ into it and thus we are not able to compute the DHE value.
		Since there is no entry in $\skeyTable$ for $\chtk$, we know that there is no honest stage-$3$ partner session by definition of $\skeyTable$.
		That is, no honest server session computed $\SKS$ and thus it must have been chosen by the adversary.
		If the pre-shared key is corrupted, $\advB_\dhe$ needs to use the $\stDH_a$ oracle to check whether there already was a query issued to $\RO_x$ for $x \in \{\chtk, \shtk\}$.
		If this is not the case, $\pi_u^i$ freshly samples random keys and remembers them for possible retroactive reprogramming of the random oracle.
		Concretely, we do the following for each random oracle $\RO_{x}$ for $x \in \{\chtk, \shtk\}$:
		
		First compute $d = \RO_\Thash(\CH \concat \dotsc \concat \SPSKtls)$ and then query the $\stDH_a$ oracle for all $Z \in \dheTable_x[\psk, d]$, where $\psk$ is the pre-shared key used by $\pi_u^i$, as
		\[
		\stDH_a(Y, Z \cdot Y^{-\tau_u^i}) = 1 \iff Z = Y^a,
		\]
		where $Y$ is the DH key share contained in $\SPSKtls$.
		See the server session implementation above for further explanation.
		If there is any of these queries is answered positively, let the respective key be $\RO_{x}(\psk, Z, d)$.
		If there is no $Z$ that results in a positive query, let $\mathit{key} \sample \KEkeyspace[x]$ be sampled at random, and $\advB_\dhe$ logs the value for possible later reprogramming of the random oracle $\RO_{x}$, i.e.,
		\[
		\rndTable_x[(\psk, d = \RO_\Thash(\CH \concat \dotsb \concat \SPSKtls) )] \assign \left(  \tau_u^i, \bot, (\CH \concat \dotsb \concat \SPSKtls), \mathit{key} \right).
		\]
		
		After that $\pi_v^i$ either has copied the keys or chose them itself and will accept all of the stage keys among these keys.
		
		If the $\psk$ of $\pi_u^i$ has not been corrupted, then no ``right'' query can have been made and the keys be sampled randomly. 
		However, we still need to program future ``right'' $\RO$ queries after a corruption.
		Therefore set
		\[
		\rndTable_x[(\psk, d = \RO_\Thash(\CH \concat \dotsb \concat \SPSKtls) )] \assign \left(  \tau_u^i, \bot, (\CH \concat \dotsb \concat \SPSKtls), \mathit{key} \right).
		\]
		$\programmingTable_{x}$ is not updated as in \thisGame, because $\dhe$ is unknown.
		%		
		\item 	Upon receiving $(\EE, \SF)$, similar to the previous step, $\pi_v^j$ checks whether there is an entry in $\skeyTable$ and $\macTable$ (to verify $\SF$) corresponding to its stage-$5$ session identifier.
		If this is the case, it copies the keys from that list.
		In case there is none, we have that there is no honest stage-$5$ partner.
		Here, we need to distinguish the case whether there was an honest stage-$3$ partner before or not.
		
		Namely, the adversary could corrupt the $\psk$, then change the $\EE$ output by an honest session and then compute a new $\SF$ message for the changed transcript.
		Hence, there is an honest stage-$3$ partner, but no stage-$5$ partner.
		In this case, $\advB_\dhe$ again applies the approach from above (see implementation of server session, Step 2) for the random oracles $\RO_{x}$ for $x \in \{\cats, \sats, \ems\}$ and the context $d = \RO_\Thash(\CH \concat \dotsb \concat \SF)$ checking whether the random oracles received already a correct query which set the keys $\cats$, $\sats$ and $\ems$.
		If this is the case and since there was a stage-$3$ partner, $\advB_\dhe$ has embedded the DH challenge in both the client and the server, this solves the strong Diffie--Hellman problem.
		When there is no such query the keys are chosen at random and all necessary information for possible retroactive programming of the random oracles $\RO_{x}$ is logged in the table $\rndTable_{x}$.
		Please see above for details.
		
		However, if there is no honest stage-$3$ partner, $\SKS$ was chosen by the adversary.
		Hence, $\advB_\dhe$ needs to apply the procedure described in the previous step (Step 2) and use the oracle $\stDH_a$ to check the random oracles $\RO_{x}$ for $x \in \{\cats, \sats, \ems\}$ whether they already set the keys.
		The important difference here is that a positive answer of the $\stDH_a$ oracle does not solve stDH,
		as $\SKS$ was chosen by the adversary.
		Note that $\advB_\dhe$ again needs to make sure that it gathers all the information needed to make retroactive programming of the random oracles possible by logging information in $\rndTable_{x}$ as before.
		%		
		\item $\pi_u^i$ computes $\cfin$ using the same process as above: if $\psk$ is corrupted, it checks for $\RO$ queries in $\dheTable_{\cfin}[\psk, d]$ that could set $\bad_{\dhe}$ when $\pi_u^i$ has an honest partner in stage $8$ or fix the value of $\cfin$ when no honest partner exists.
		It then calls $\Finalize$ or sets $\cfin$ accordingly.
		If no earlier $\RO$ query matches $\cfin$, then we sample $\cfin$ randomly and log $\tau_u^i$, $\cfin$, and the transcript in table $\rndTable_{\cfin}$ under $\psk$ and the transcript hash $d$.
		If $\psk$ is uncorrupted, $\pi_u^i$ immediately samples $\cfin$ randomly and logs $\tau_u^i$, $\cfin$, and the transcript in $\rndTable_{\cfin}$ under index $((u,v,\pskid), d)$.
		
		Next we compute $\rms$.
		As $\pi_u^i$ is not able to compute $\dhe$ independent of there being a honest stage-$3$ partner or not, $\advB_\dhe$ need to apply the same procedure that was described before in Step 3, when there was no stage-$5$ partner for random oracle $\RO_\rms$ and context $d = \RO_\Thash(\CH \concat \dotsb \concat \CF)$.
		The only difference is that in case there was a stage-$3$ partner, $\Finalize$ is queried when the $\stDH$ oracle returns $\true$, and if there is no stage-$3$ partner, $\rms$ is only programmed.
		%
		Then, $\pi_u^i$ outputs $\CF$.
	\end{enumerate}
	
	Besides changing the implementation of the session oracles, we also need to adapt the random oracles $\RO_x$ for $x \in \{\chtk, \dotsc,\rms\}$ to make sure
	(1)~$\advB_\dhe§$ programs the random oracle retroactively if the random oracle receives the right query and
	(2)~to check whether the adversary computed $\dhe$ for $\advB_\dhe$ for honestly partnered sessions.
	
	\paragraph{Implementation of random oracle $\RO_x$.}
	If $\RO_x$ receives a query that was already answered, it answers consistently.
	However, if there is a new query of the form $(\psk, Z, d)$, it appends $Z$ to the set $\dheTable_k[\psk, d]$.
	If $\rndTable_k[\psk, d] \neq \bot$, then there already was a session using $\psk$ and context hash $d$ trying to compute a key without knowing the correct $\dhe$ secret.
	Therefore, $\advB_\dhe$ uses the $\stDH_a$ oracle to check whether $Z$ is that secret.
	Let $(\tau_u^i, \tau_v^j, \mathit{ctxt}, \mathit{key})$ be the entry of $\rndTable_k[\psk, d]$, where $\tau_u^i$ and $\tau_v^j$ denote the randomness used by the client and the server to randomize the stDH challenge, respectively, $\mathit{ctxt} = \CH \concat \CKS \concat \\CPSKtls \concat \SH \concat \SKS \concat \SPSKtls \concat \dotsb$ denotes the context such that $d = \RO_\Thash(\mathit{ctxt})$ and $\mathit{key}$ denotes the key chosen by the session since there was no random oracle fixing it.
	Using this information, it fetches $\SKS = Y$ and queries $\stDH_a(Y, Z \cdot Y^{-\tau_u^i})$.
	If this query is answered positively, $\advB_\dhe$ knows that the right DH value $Z$ was queried.
	If $\tau_u^j = \bot$, i.e., the log in $\rndTable_k$ was set by a client without an honestly partnered server, $\advB_\dhe$ needs to program the random oracle to be consistent.
	That is, $\roTable_k[\psk, Z, d] \assign \mathit{key}$.
	Otherwise, $\advB_\dhe$ knows that the $\programmingTable_x$ entry was set by an honestly partnered session, and thus $Z$ is a randomized solution to the stDH challenge.
	Thus, $\advB_\dhe$ submits the solution $Z \cdot Y^{-\tau_u^i} \cdot A^{-\tau_v^j}$ to its stDH $\Finalize$ oracle.
	
	Unless $\advB_\dhe$ solved the stDH challenge, the oracle outputs $\roTable_x[\psk, Z, d]$.
	
	\paragraph{Implementation of corruption oracle $\RevLongTermKey$.} Finally, $\advB_\dhe$ needs to handle corruptions via the $\RevLongTermKey$ oracle.
	Since \refGameText{game:mac-forgery-prep}, the $\RevLongTermKey$ oracle upon input $(u, v, \pskid)$ samples a fresh $\psk$.
	It then uses lists $\programmingTable_x$ to program all the random oracles $\RO_{x}$ for consistency with any sessions whose pre-shared key is now $\psk$. 
	Reduction~$\advB_\dhe$ still does this, but in our reduction, the lists $\programmingTable_x$ are no longer comprehensive. 
	Some sessions fix the outputs of $\RO_{x}$ on some query without knowing the $\dhe$ input to that query. 
	These sessions create log entries in $\rndTable_x$, not $\programmingTable_x$, and the entries have indices of the form $((u,v,\pskid), d)$. 
	$\advB_\dhe$ cannot use these entries to program past $\RO_x$ queries, but this is not necessary since any past $\RO_x$ query containing $\psk$ would set the $\bad_{\psk}$ flag and cause the game to abort.
	$\advB_\dhe$ also cannot program future queries because we still do not know $\dhe$. 
	Instead, $\advB_\dhe$ just updates each matching entry in $\programmingTable_x$ so that its index is $(\psk, d)$ instead of $((u, v, \pskid), d)$. 
	Future $\RO_x$ queries containing $\psk$ will then handle strong DH checking and programming for $\advB_\dhe$.
	
	By the considerations above, we have that if $\bad_\dhe$ is set the $\advB_\dhe$ wins the strong DH challenge.
	The identical-until-bad-lemma gives us that
	\begin{align} \nonumber
		\Pr[ \prevGameMath \outputs 1 ] &\leq \Pr[ \thisGameMath \outputs 1 ] + \Pr[ \bad_\dhe ] \\
		&\leq \Pr[ \thisGameMath \outputs 1 ] + \genAdv{\stDH}{\G}{t_{\advB_\dhe}, 2 \qRO},
	\end{align}
	where the number of $\stDH_a$ oracle queries is no greater than $2 \qRO$, since $\advB_\dhe$ will query the oracle at most twice (once for each partner) for every random oracle query issued by the adversary, and $t_{\advB_{\dhe}}$ with $t_{\advB_{\dhe}} \approx t + 4 \log(p) \cdot \qRO$ is the running time of $\advB_{\dhe}$.
	Note that for every $\stDH_a$ query, $\advB_{\dhe}$ needs to perform one group operation and one exponentiation in $\G$, the latter can be done in $2 \log(p)$ many group operations using, e.g., the square-and-multiply algorithm.
	Thus, the time to answer a single $\stDH_a$ query take approximately time $2 \log(p)$ and taking this together with the bound on the number of $\stDH_a$ yields the approximate runtime $t_{\advB_{\dhe}}$.
	
	\paragraph{Conclusion of Phase~3.}
	We finally argue that the adversary's probability in determining the challenge bit~$b$ in \thisGame is at most $\tfrac{1}{2}$ if the $\Fresh$ predicate is true.
	First, recall that $\Fresh = \true$ implies no session can be tested and revealed in the same stage, and a tested session's partner may also be neither tested nor revealed in that stage.
	In the following, we refer to a session being ``fresh'' in a stage if this session does not violate the conditions defined in the predicate~$\Fresh$ in that stage.
	The $\Fresh$ predicate depends on the level of forward secrecy reached at the time of each $\Test$ query.
	First, if a session is tested in a non-forward secret stage, it remains only fresh if the $\psk$ was never corrupted.
	Second, if a session is tested in a weak forward secret 2 stage $s$, it remains fresh if the $\psk$ was never corrupted or if there is a contributive partner in stage $s$.
	Lastly, if a session is tested on a forward secret stage $s$, it remains fresh the $\psk$ was corrupted after forward secrecy was established for that stage (perhaps retroactively) or if there is a contributive partner.
	
	Next, we argue for each level of forward secrecy that all tested keys in \thisGame which do not violate $\Fresh$ are uniformly and independently distributed from the view of the adversary.
	For the non-forward secret stages~$1$ ($\ets$) and~$2$ ($\eems$), the adversary cannot corrupt the $\psk$ of all sessions that it queried $\Test$ on stage~$1$ or~$2$. 
	Since \refGameText{game:mac-forgery-prep}, we sample all session keys derived from uncorrupted pre-shared keys uniformly at random, or copy uniformly random keys from $\skeyTable$. 
	That is, the key returned by the $\Test$ query is a uniformly random key independent of the challenge bit $b$.
	Therefore, it cannot learn anything about either $\ets$ nor $\eems$ of any session with an uncorrupted key, and thus the response of a $\Test$ query will be a uniformly random string independent of the challenge bit $b$ from the view of the adversary.
	
	All other stages, i.e., stages~$3$--$8$, are weak forward secret 2 upon acceptance and become forward secret as soon as the session achieves explicit authentication.
	If the pre-shared key is never corrupted, we have by the same arguments given for the non-forward secret stages that the adversary receives a uniformly random key in response to the $\Test$ query independent of the challenge bit.
	
	It remains to argue that the same is true if there is a contributive partner and the $\psk$ is corrupted. 
	In this case, the adversary would need to make a random oracle query that triggers $\bad_\dhe$ introduced in \refGameText{game:abort-dhe} and would cause $\Finalize$ to return $0$.
	Without such a query the respective key is just a uniformly and independently distributed bitstring from the adversary's view.
	Hence, without losing the game, the adversary cannot learn anything about a weak forward secret 2 key, and thus it does not learn anything from the response of the $\Test$ query.
	
	Since forward secret stages are weak forward secret 2 until explicit authentication is established, we only consider the case that a session that is tested on a weak forward secret 2 stage was corrupted after forward secrecy has been (retroactively) established.
	As we only establish forward secrecy after explicit authentication has been achieved, we can be sure due to $\ExplicitAuth$ never beeing violated that there is a partnered session for that stage.
	Hence, there also is a contributive partner and by the same arguments as given before the adversary would trigger $\bad_\dhe$ and lose the game before it can learn something about the session.
	
	Overall, we have that the adversary in \thisGame cannot gain any information on the challenge bit $b$ without violating any of the predicates $\Sound$, $\ExplicitAuth$, or $\Fresh$. Thus, the probability that $\Finalize$ and thus \thisGame returns $1$ is no greater than $1/2$. Formally,
	\[
		\Pr[ \thisGameMath \outputs 1 ] \leq \frac 12.
	\]
	
	Collecting all the terms, we get the final bound
	\begin{align*}
		&\genAdv{\KESEC}{\TLSPSKECDHE}{t, \qNewSecret, \qSend, \qRevSessionKey, \qRevLongTermKey, \qTest, \qRO} \\
		&\qquad\leq \frac{2\qSend^2}{2^{\nl} \cdot p} + \genAdv{\CollRes}{\RObinder}{\qRO + \qSend} + \frac{\qNewSecret^2}{2^{\hashlen}} + \genAdv{\CollRes}{\ROthash}{\qRO + 6\qSend} \\
		&\qquad\qquad + \frac{\qRO \cdot \qNewSecret}{2^{\hashlen}} + \frac{\qSend}{2^{\hashlen}} + \genAdv{\stDH}{\G}{t_{\advB_\dhe}, 2 \qRO}
	\end{align*}
	Applying the result of Appendix~\ref{app:coll-res-ro}, we can make the collision resistance terms explicit
	\begin{align*}
		&\genAdv{\KESEC}{\TLSPSKECDHE}{t, \qNewSecret, \qSend, \qRevSessionKey, \qRevLongTermKey, \qTest, \qRO} \\
		&\qquad\leq \frac{2\qSend^2}{2^{\nl} \cdot p} + \frac{(\qRO+\qSend)^2}{2^{\hashlen}} + \frac{\qNewSecret^2}{2^{\hashlen}} + \frac{(\qRO+6\qSend)^2}{2^{\hashlen}} + \frac{\qRO \cdot \qNewSecret}{2^{\hashlen}} + \frac{\qSend}{2^{\hashlen}} \\
		&\qquad\qquad +  \genAdv{\stDH}{\G}{t_{\advB_\dhe}, 2 \qRO}
	\end{align*}
	Further, applying the GGM bound for the strong Diffie--Hellman problem proven by Davis and Günther in \cite{ACNS:DavGun21}, we get the final result
	\begin{align*}
		&\genAdv{\KESEC}{\TLSPSKECDHE}{t, \qNewSecret, \qSend, \qRevSessionKey, \qRevLongTermKey, \qTest, \qRO} \\
		&\qquad\leq \frac{2\qSend^2}{2^{\nl} \cdot p} + \frac{(\qRO+\qSend)^2}{2^{\hashlen}} + \frac{\qNewSecret^2}{2^{\hashlen}} + \frac{(\qRO+6\qSend)^2}{2^{\hashlen}} + \frac{\qRO \cdot \qNewSecret}{2^{\hashlen}} + \frac{\qSend}{2^{\hashlen}} \\
		&\qquad\qquad +  \frac{4(t + 4 \log(p) \cdot \qRO)^2}{p} \\
		&\qquad= \frac{2\qSend^2}{2^{\nl} \cdot p} + \frac{(\qRO+\qSend)^2 + \qNewSecret^2 + (\qRO+6\qSend)^2 + \qRO \cdot \qNewSecret + \qSend}{2^{\hashlen}} \\
		&\qquad\qquad +  \frac{4(t + 4 \log(p) \cdot \qRO)^2}{p}
	\end{align*}


%%% Local Variables:
%%% mode: latex
%%% TeX-master: "main"
%%% End:

	\end{proof}
\else
\subsection{Proof overview}
\label{sec:psk-ecdhe-ke-proof-overview}

The proof proceeds via a sequence of games in three phases, corresponding to the three ways for an adversary to win the $\KESEC$ security game.
We begin with $\Gm_0$, the original $\KESEC$ game for protocol $\TLSPSKECDHE$ described above. 
In the first phase, we establish that the adversary cannot violate the $\Sound$ predicate.
In the second phase, we establish the same for the $\ExplicitAuth$ predicate.
In the third phase, we ensure that all $\Test$ queries return random keys regardless of the value of the challenge bit~$b$, so long as the $\Fresh$ predicate is not violated.
After that, the adversary cannot win the game with probability better than guessing, rendering its advantage to be~$0$.
We bound the advantage difference introduced by each game hop; collecting these intermediate bounds yields the overall bound.
For space reasons, we only provide a summary of the proof in the following and refer to 
	Appendix~\ref{app:full-psk-ecdhe-ke-proof} 
for the full details.


\subsection*{Phase~1: Ensuring $\Sound$}
The $\Sound$ predicate 
	(cf.\ Figure~\ref{fig:MSKE-preds}) 
checks that no more than two sessions can be partnered in a non-replayable stage, and that any two partnered sessions must agree on the stage, pre-shared key identifier, the stage-$s$ key, and each others' identities and roles. 
We defined our session identifiers so that the stage-$s$ session identifier contains (1) a label unique to that stage,  (2) a unique $\ClientHello$ and $\ServerHello$ message, (3) the $\binder$ message: a $\MAC$ tag authenticating the $\ClientHello$ and pre-shared key, and (4) sufficient information to fix the stage-$s$ key. 
(This does not mean the key is computable from the $\sid$; it is not.)

We then perform three incremental game hops that cause the $\Finalize$ oracle to return $0$ in the event of a collision between two $\Hello$ messages, $\binder$ tags, or pre-shared keys.
We bound the difference in advantage in the first two game hops via a birthday bound over the number of potentially colliding values (i.e., pairs of nonces and $\KeyShares$ in $\G$ for $\Hello$ message collisions, and sampled $\psk$ keys for pre-shared key collisions),
and the third hop by a reduction to the collision resistance of the $\RObinder$ random oracle whose advantage in turn is upper bounded by a birthday bound~$\genAdv{\CollRes}{\RObinder}{\qRO + \qSend} \leq \frac{(\qRO + \qSend)^2}{2^\hashlen}$%
	; cf.\ Appendix~\ref{app:coll-res-ro}.
The resulting bounds are, in this order:
\shortlongeqn[.]{
	\Pr[\Gm_0] - \Pr[\Gm_3] \leq \frac{2\qSend^2}{2^{\nl} \cdot p} + \frac{\qNewSecret^2}{2^{\hashlen}} + \frac{(\qRO + \qSend)^2}{2^\hashlen}
}
As long as no such collisions occur, each stage-$s$ session identifier uniquely determines one client session, one server session (for non-replayable stages), one pre-shared key (and therefore one peer and identifier owning that key), and one stage-$s$ session key.
At this point, the $\Sound$ predicate will always be $\true$ unless $\Finalize$ would return $0$, so the adversary cannot win by violating $\Sound$.
%\fg{This is a summary for what would be a combined proof for PSK-only and PSK-(EC)DHE; separate.}
%\replace{In the PSK-(EC)DHE handshake, h}{H}

\subsection*{Phase~2: Ensuring $\ExplicitAuth$}
In the second phase of the proof, we change the key-derivation process to avoid sampling pre-shared keys wherever possible,
instead replacing keys and $\MAC$ tags derived from those pre-shared key by uniformly random strings.
We then make the adversary lose if it makes queries that would allow him to detect these changes and bound that probability;
in particular we ensure that the adversary does not correctly guess a now-random $\ClientFinished$ or $\ServerFinished$ MAC tag.
Sessions achieve explicit authentication just after verifying their received $\Finished$ message;
eliminating possible forgeries hence ensures that the $\ExplicitAuth$ predicate cannot be $\false$ without $\Finalize$ returning $0$.
All changes in this phase apply only to sessions whose pre-shared key has not been corrupted.

\lightparagraph{Game~4}
Our first of six game hops eliminates collisions in the ``transcript hash'' function $\ROthash$.
We reduce to the collision resistance of $\ROthash$ and bound this advantage with a birthday bound:
\shortlongeqn[.]{
	\Pr[\Gm_3] - \Pr[\Gm_4] \leq \frac{\qRO+6\qSend}{2^{\hashlen}}
}
(The factor~$6$ comes from the up to~$6$ transcript hashes computed in any~$\Send$ query.)

\lightparagraph{Game~5}
Our next game forces $\Finalize$ to return $0$ if the adversary guesses any uncorrupted pre-shared key in any random oracle query.
Since we assume pre-shared keys are uniformly random,
\shortlongeqn[.]{
	\Pr[\Gm_4] - \Pr[\Gm_5] \leq \frac{\qRO \cdot \qNewSecret}{2^{\hashlen}}
}

\lightparagraph{Games~6 and~7}
In our third game hop, we ask log the stage~$s$ key computed in any session in a look-up table~$\skeyTable$ under its session identifier.
Sessions whose partners have logged a key can then, in a fourth game hop, copy the key from $\skeyTable$ instead of deriving it.
Partnered sessions will always derive the same key as guaranteed by the $\Sound$ predicate, so the adversary cannot detect the copying and its advantage does not change.
In addition to logging and copying keys, we also log and copy the three $\MAC$ tags: $\binder$, $\sfin$, and $\cfin$ using another look-up table~$\macTable$. 
Since $\MAC$ tags do not have associated session identifiers, they are logged under the inputs to~$\RObinder$, $\ROsfin$, resp.\ $\ROcfin$.
This technique is inspired by the work of Cohn-Gordon et al.~\cite{C:CCGJJ19}. 

\lightparagraph{Game~8}
In preparation for the final step in this phase, our fifth game hop eliminates uncorrupted pre-shared keys altogether.
We postpone the sampling of the pre-shared key to the $\RevLongTermKey$ oracle so that only corrupted sessions hold pre-shared keys.
As a consequence of this change, we can no longer compute session keys and MAC tags using the random oracles.
Sessions will instead sample these uniformly at random from their respective range. 
In another look-up table, they log the $\RO$ queries they would have made so that these queries can be programmed later if the pre-shared key gets corrupted.
Queries to~$\RO$ before corruption cannot contain the pre-shared key thanks to the previous game, so we do not have to worry about consistency with past queries. 
%
We also cannot  implement the previous games' check for guessed pre-shared keys in $\RO$ queries until these keys are sampled,
so we sample new pre-shared keys for all uncorrupted identifiers at the end of the game in the $\Finalize$ oracle, then perform the check.
The programming of the random oracles is perfectly consistent with their responses in earlier games, so the adversary cannot detect when pre-shared keys are chosen in the game and its advantage does not change.
%We don't need to 

\lightparagraph{Game~9}
The final game in this phase ensures that either $\ExplicitAuth = \true$ or $\Finalize$ returns~$0$.
In this game, we return $0$ from $\Finalize$ if any honest session would accept the first explicitly-authenticated stage (stage~$5$ ($\cats$) for initiators and stage~$8$ ($\rms$) for responders) with an uncorrupted pre-shared key and no honest partner.
By the previous game, we established that sessions with uncorrupted pre-shared keys randomly sample their $\MAC$ tags,
unless they copy a cached result in which case the same computation was made by another session.
Thanks to the way we defined our session identifiers, no unpartnered session will copy their $\MAC$ tags:
the computation of the $\ServerFinished$ $\MAC$ tag contains the hash of the stage-$5$ $\sid$ (excluding $\sfin$); likewise the $\ClientFinished$ tag contains the hash of the stage-$8$ $\sid$.
Since we ruled out hash collisions in the first game of the phase, any two sessions computing the same $\ServerFinished$ message are stage-$5$ partners and any two sessions computing the same $\ClientFinished$ message are stage-$8$ partners.
So any unpartnered session with an uncorrupted pre-shared key has a random $\MAC$ tag, and the odds of the adversary guessing such a tag is bounded by $\frac{\qSend}{2^{\hashlen}}$.
With the prior two games not changing the adversary's advantange, we have
\shortlongeqn[.]{
	\Pr[\Gm_5] - \Pr[\Gm_9] \leq \frac{\qSend}{2^{\hashlen}}
}

We are now guaranteed that any session accepting the stage that achieves explicit authentication without a corrupted pre-shared key has a partner in that stage.
The $\Sound$ predicate guarantees that the partner agrees on the peer and pre-shared key identities, which is sufficient to guarantee explicit authentication for all responder sessions.
For initiator sessions, we must also note that a partner in stage $5$ will become, upon their acceptance, a partner in stages~$6$ ($\sats$) and~$7$ ($\ems$), whose $\sid$s are identical to that of stage $5$ apart from their labels.
An initiator's stage-$5$ partner will only accept a $\ClientFinished$ message identical to the one sent by the initiator, at which point they will become a partner also in stage~$8$.
This ensures that the $\ExplicitAuth$ predicate can never be false unless one of the flags introduced in this phase causes $\Finalize$ to return~$0$.


\subsection*{Phase~3: Ensuring the Challenge Bit is Random and Independent}

Our goal in the third and last phase is to ensure that all session keys targeted by a $\Test$ query are uniformly random and independent of the challenge bit $b$ whenever the $\Fresh$ predicate is true.
Freshness ensures that no session key can be tested twice or tested and revealed in the same stage either by targeting the same session twice or two partnered sessions.
It also handles our three levels of forward secrecy.

We can already establish this for $\Test$ queries to sessions in non-forward secret stages~$1$ ($\ets$) and~$2$ ($\eems$).
These queries violate $\Fresh$ unless the sessions' pre-shared keys are never corrupted. 
Since $\Gm_8$, all sessions with uncorrupted pre-shared keys either randomly sample their session keys, or copy random keys from a partner session.
If one of these session keys is tested, it cannot have been output by another $\Test$ or $\RevSessionKey$ query without violating $\Fresh$.
Therefore the response to the $\Test$ query is a uniformly random string, independent of all other oracle responses and the challenge bit $b$.

The remaining stages ($3$--$8$) have weak forward secrecy $2$ until explicit authentication is achieved, then they have full forward secrecy.
These stages' keys may be tested even if the session's pre-shared key has been corrupted, so long as there is a contributive partner (or, in the case of full forward secrecy, that the corruption occurred after forward secrecy was achieved).
We use one last game hop to ensure these keys are uniformly random when they are tested.

\lightparagraph{Game~10}
In $\Gm_{10}$, we cause the $\Finalize$ oracle to return $0$ if the adversary should ever make a random oracle query containing the Diffie--Hellman secret $\dhe$ of an honest partnered session whose pre-shared key was corrupted.
Without such a query, all keys derived from a Diffie--Hellman secret sampled uniformly at random by the random oracles.

We bound the probability of this event via a reduction $\advB_\dhe$ to the strong Diffie--Hellman problem in group $\G$.
(Recall that $\G$ has order $p$ and generator~$g$.)
In this problem, the adversary $\advB_\dhe$ gets as input a strong DH challenge $(A = g^a, B = g^b)$ as well as access to an oracle $\stDH_a$ for the decisional Diffie--Hellman (DDH) problem with the first argument fixed. 
Given inputs $C \assign g^c$ and $W$ for any $c \in \Z_p$, $\stDH_a(C, W)$ returns $\true$ if and only if $W = g^{ac} = C^a$.
The goal of~$\advB_\dhe$ is to submit $Z$ to its $\Finalize$ oracle such that $Z = g^{ab}$.

The reduction $\advB_\dhe$ simulates $\Gm_{10}$ for the $\KESEC$ adversary $\advA$.
At a high level, it uses rerandomization to embed its strong DH challenge $A$, resp. $B$, into the key shares of every initiator session, resp. every partnered responder session.
To embed a challenge $A$ in its key share, a session samples a ``randomizer'' $\tau \sample \Z_p$, and sets its key share to $X \assign A \cdot g^{\tau}$. 
If $\advA$ should make an $\RO$ query containing the Diffie--Hellman secret associated with two embedded key shares, the reduction can detect this query with its DDH oracle.
It then extracts the solution to its strong DH challenge from the query's DH secret, calls the $\Finalize$ oracle, and wins its own game.

There are a few subtleties to the reduction, which requires us to extend the technique of CCGJJ~\cite{C:CCGJJ19}.
Unlike honest executions of the protocol, the reduction's simulated sessions with embedded key shares do not know their own secret Diffie--Hellman exponents.
If their pre-shared keys are never corrupted, this does not matter because session keys and $\MAC$ tags are randomly sampled.
Corrupted sessions, however, cannot use the random oracles to compute these values as they would in $\Gm_{10}$.
%
Instead, $\advB_\dhe$ samples session keys and $\MAC$ tags uniformly at random and uses several look-up tables to program random oracle queries and maintain consistency between sessions.

\iffull
Recall that since $\Gm_6$, partnered sessions store their random session keys in a table $\skeyTable$ under their session identifiers.
Since session identifiers do not contain Diffie--Hellman secrets, simulated sessions with embedded key shares can also log and copy their keys using $\skeyTable$, even if their pre-shared key has been corrupted.
Partnered sessions with embedded key shares need not maintain consistency with later random oracle queries, as such query will enable our reduction to call $\Finalize$ and immediately end the game before a response is required.

Logging $\MAC$ tags in the table~$\macTable$ however does require the Diffie--Hellman secret, so partnered sessions with embedded key shares must use a separate table $\rndTable$ to achieve consistency.
This table also logs the information used to compute a $\MAC$ tag, but instead of a Diffie--Hellman secret it stores the initiator and responder's randomizers.
It also stores the full transcript instead of its hash, so that the reduction can efficiently recall the initiator and responders' key shares.
\fi

With this infrastructure in place, the reduction proceeds in the following way.
Whenever a partnered session with embedded key share would need its Diffie--Hellman secret, it searches all past $\RO$ queries for this secret.
It looks up the initiator's stored randomizer $\tau$ and the responder's randomizer $\tau'$.
Then for each guess $\Z$ in a past $\RO$ query, the reduction queries the strong Diffie--Hellman oracle on the responder's key share $\SKS$ and $C \assign \Z \cdot g^{-\tau} $.
This query will return $\true$ if the adversary correctly guessed the Diffie--Hellman secret; in this case the reduction calls $\Finalize(Z \cdot g^{-\tau} \cdot g^{-\tau'})$ and solves its strong DH challenge.
Unpartnered sessions do the same thing, except that the responder has no randomizer; in response to the strong DH oracle answering $\true$ they hence merely program their session keys instead of calling $\Finalize$.
We emphasize that for tightness, it is crucial to maintain efficiency during this process.
We do so by only checking $\RO$ queries whose context matches the hashed protocol transcript; this ensures $\advB_\dhe$ makes at most $2$ $\stDH_a$ queries for each $\RO$ query.

After a session chooses its session key or $\MAC$ tag, it stores the chosen value, its transcript, and all known randomizers in a table $\rndTable$.
When the reduction answers future $\RO$ queries, it will use this table to check if a query contains the Diffie--Hellman secret of an accepted session using the strong DH oracle as above; if so, they program or call $\Finalize$ in the same way.

This reduction solves the strong Diffie--Hellman problem whenever the adversary makes an $\RO$ query containing a partnered session's Diffie--Hellman secret,
so for reduction $\advB_\dhe$ with runtime $t_{\advB_{\dhe}}$, we have
\shortlongeqn[.]{
	\Pr[\Gm_9] - \Pr[\Gm_{10}] \leq \genAdv{\stDH}{\G}{t_{\advB_\dhe}, 2\qRO}
}
Davis and G{\"u}nther gave a bound in the generic group model for the strong DH problem; applying their Theorem~3.3~\cite{ACNS:DavGun21} results in
\shortlongeqn[.]{
	\Pr[\Gm_9] - \Pr[\Gm_{10}] \leq \tfrac{t_{\advB_\dhe}^2}{p}
}
\smallskip


At this point in the proof, the adversary $\advA$ cannot possibly make a $\RO$ query that outputs any tested session key of a forward secret (full or wfs2) stage~$s$.
If the tested session's pre-shared key is uncorrupted, $\advA$ cannot make the query because of $\Gm_5$.
If the session has a contributive partner in stage~$s$, then from $\Gm_{10}$, $\advA$ cannot make the query because it contains the Diffie--Hellman secret of a partnered session.
If it has accepted with no contributive partner and a corrupted pre-shared key, then by the guarantees we established in Phase 2, the corruption must have occurred before forward secrecy and explicit authentication were achieved.

As a result, the output of any $\Test$ query (that does not violate $\Fresh$) is a random string, sampled by either a session or the $\RO$ oracle independently of all other game variables including the challenge bit $b$.
The adversary therefore has a probability no greater than $\frac{1}{2}$ of winning $\Gm_{10}$.
Collecting this probability with the other bounds between games in our sequence gives the proof. %\qed

\fi % end short proof overview

\subsection{Full Security Bound for TLS~1.3 PSK-(EC)DHE and PSK-only}
\label{sec:psk-ecdhe-ke-full-bound}

We can finally combine the results of Sections~\ref{sec:ks-indiff}, \ref{sec:modularizing}, and our key exchange bound above to produce fully concrete bounds for the TLS~1.3 PSK-(EC)DHE and PSK-only handshake protocols as specified on the left-hand side of Figure~$1$.
This bound applies to the protocol \emph{with handshake traffic encryption} and \emph{internal keys} when \emph{only modeling as random oracle}~$\ROhash$ the hash function~$\abstractHash$.

First, we define three variants of the TLS~1.3 PSK handshake:
\begin{itemize}
	\item $\KE_{0}$, as defined in Theorem~\ref{thm:full-ks-indiff} with handshake traffic encryption and one random oracle $\ROhash$.
		(This is the variant we want to obtain our overall result for.)	
		
	\item $\KE_{1}$, as defined in Theorem~\ref{thm:full-ks-indiff} with handshake traffic encryption and $12$ random oracles $\ROthash$, $\RObinder$, \dots, $\ROrms$.
	\item $\KE_{2}$: as defined in Theorem~\ref{thm:TLS-transform}, with no handshake traffic encryption and $12$ random oracles $\ROthash$, $\RObinder$, \dots, $\ROrms$.
\end{itemize}

Theorem~\ref{thm:full-ks-indiff} grants that
\shortlongalign[.]{
	\Adv^{\KESEC}_{\KE_{0}}(t, \qNewSecret, \qSend, \qRevSessionKey, \qRevLongTermKey, \qTest, \qRO)
	\leq
	\fullonly{&~}
	\Adv^{\KESEC}_{\KE_{1}}(t\cab \qNewSecret\cab \qSend\cab \qRevSessionKey\cab \qRevLongTermKey\cab \qTest\cab \qRO)
	\fullonly{\\&}
	+ \frac{2(12\qSend+\qRO)^2}{2^{\hashlen}}
	+ \frac{2\qRO^2}{2^{\hashlen}}
	+ \frac{8(\qRO+36\qSend)^2}{2^{\hashlen}}
}

Next, we apply Theorem~\ref{thm:TLS-transform}, yielding the bound 
	\shortlongeqn[,]{
	\Adv^{\KESEC}_{\KE_1}(t\cab \qNewSecret\cab \qSend\cab \qRevSessionKey\cab \qRevLongTermKey\cab \qTest\cab \qRO)
	\leq
	\Adv^{\KESEC}_{\KE_2}(t+ t_{\mathrm{AEAD}}\cdot \qSend\cab \qNewSecret\cab \qSend\cab \qRevSessionKey + \qSend\cab \qRevLongTermKey\cab \qTest\cab \qRO)
}
where $t_{\mathrm{AEAD}}$ is the maximum time required to execute AEAD encryption or decryption of TLS~1.3 messages. 

Theorem~\ref{thm:full-psk-ecdhe-ke} then finally and entirely bounds the advantage against the $\KESEC$ security of $\KE_{2}$.
\iffull
Collecting these bounds gives 
\begin{align*}
	\Adv^{\KESEC}_{\KE_{0}}(t, \qNewSecret, \qSend, \qRevSessionKey, \qRevLongTermKey, \qTest, \qRO)
	&\leq
	\Adv^{\KESEC}_{\KE_{1}}(t, \qNewSecret, \qSend, \qRevSessionKey, \qRevLongTermKey, \qTest, \qRO)
	\\&\quad
	+ \frac{2(12\qSend+\qRO)^2}{2^{\hashlen}}
	+ \frac{2\qRO^2}{2^{\hashlen}}
	+ \frac{8(\qRO+36\qSend)^2}{2^{\hashlen}}\\
	&\leq
	\Adv^{\KESEC}_{\KE_2}(t+ t_{\mathrm{AEAD}}\cdot \qSend, \qNewSecret, \qSend, \qRevSessionKey + \qSend, \qRevLongTermKey, \qTest, \qRO)
	\\&\quad
	+\frac{2(12\qSend+\qRO)^2 + 2\qRO^2 + 8(\qRO+36\qSend)^2}{2^{\hashlen}} \\
	&\leq \frac{2\qSend^2}{2^{\nl} \cdot p} + \frac{(\qRO+\qSend)^2 + \qNewSecret^2 + (\qRO+6\qSend)^2 + \qRO \cdot \qNewSecret + \qSend}{2^{\hashlen}} \\
	&\quad + \frac{4(t + t_{\mathrm{AEAD}}\cdot \qSend + 4 \log(p) \cdot \qRO)^2}{p} \\
	&\quad + \frac{2(12\qSend+\qRO)^2 + 2\qRO^2 + 8(\qRO+36\qSend)^2}{2^{\hashlen}}.
\end{align*}

This yields the following overall result for the $\KESEC$ security of the TLS~1.3 PSK-(EC)DHE handshake protocol.
\else
Collecting these bounds yields the following overall result for the $\KESEC$ security of the TLS~1.3 PSK-(EC)DHE handshake protocol.
\fi

\begin{corollary}\label{cor:full-psk-ecdhe-ke}
	Let $\TLSPSKECDHE$ be the TLS~1.3 PSK-(EC)DHE handshake protocol as specified on the left-hand side in \autoref{fig:tls-handshake}.
	Let $\G$ be the Diffie--Hellman group of order~$p$.
	Let $\nl$ be the length in bits of the nonce, let $\hashlen$ be the output length in bits of $\abstractHash$, and let the pre-shared key space be $\KEpskeyspace = \bits^\hashlen$.
	Let $\abstractHash$ be modeled as a random oracle $\RO_\hash$.
	%
% 	Let $\advA$ be an adversary against the $\KESEC$ security of $\TLSPSKECDHE$ running in time $t$, and let $\advA$ make $\qRO$, $\qNewSecret$, resp.\ $\qSend$ queries to its $\RO_\hash$, $\NewSecret$, resp.\ $\Send$ oracles.
	Then,
{\allowdisplaybreaks
	\begin{align*}
		&\genAdv{\KESEC}{\TLSPSKECDHE}{t, \qNewSecret, \qSend, \qRevSessionKey, \qRevLongTermKey, \qTest, \qRO} \\
		&\qquad\leq \frac{2\qSend^2}{2^{\nl} \cdot p} + \frac{(\qRO+\qSend)^2 + \qNewSecret^2 + (\qRO+6\qSend)^2 + \qRO \cdot \qNewSecret + \qSend}{2^{\hashlen}} \\
		&\qquad\qquad +  \frac{4(t + t_{\mathrm{AEAD}} \cdot \qSend + 4 \log(p) \cdot \qRO)^2}{p}\\
		&\qquad\qquad + \frac{2(12\qSend+\qRO)^2 + 2\qRO^2 + 8(\qRO+36\qSend)^2}{2^{\hashlen}}.
	\end{align*}
}
\end{corollary}

	Our tight security proof for the TLS~1.3 PSK-(EC)DHE handshake given in Section~\ref{sec:psk-ecdhe-ke-proof-short} can be adapted to the PSK-only handshake.
The structure and resulting bounds are largely the same between the two modes, with a couple of significant changes.
Naturally, we have no Diffie--Hellman group, no key shares in the $\ClientHello$ or $\ServerHello$ messages, and no reduction to the strong Diffie--Hellman problem.
Without the reduction to $\stDH$, we cannot achieve forward secrecy for any key: an adversary in possession of the pre-shared key can compute all session keys. 
% We therefore do not consider any key established by TLS~1.3 in PSK-only mode to have achieved forward secrecy, or even weak forward secrecy 2.

The security proof for the TLS~1.3 PSK-only handshake uses the same sequence of games $\Gm_0$ to $\Gm_9$ (excluding the reduction to the strong Diffie--Hellman problem in $\Gm_{10}$).
There only is a difference in $\Gm_1$, in which we exclude collisions of nonces and group elements sampled by honest session to compute there $\Hello$ messages.
Since we do not have any key shares in the PSK-only mode, the session will consequently also not sample a group elements.
Thus, the bound for $\Gm_0$ changes to
\shortlongeqn[.]{
	\Pr[ \Gm_0 \outputs 1 ] \leq \Pr[ \Gm_1 \outputs 1 ] + \frac{2\qSend^2}{2^{\nl}}
}
The rest of the arguments follow similarly as given in Section~\ref{sec:psk-ecdhe-ke-proof-short}.
We obtain the following result.

\begin{theorem}\label{thm:psk-ke}
	Let $\TLSPSK$ be the TLS~1.3 PSK-only handshake protocol as specified on the right-hand side in \autoref{fig:tls-handshake} without handshake encryption. 
	Let functions $\abstractHash$ and $\TLSKDF_x$ for each $x \in \{\binder, \dotsc, \rms\}$ be modeled as $12$ independent random oracles $\mathlist{\RO_\Thash,\RO_{\binder},\dotsc, \RO_{\rms}}$.
	Let $\nl$ be the length in bits of the nonce, let $\hashlen$ be the output length in bits of $\abstractHash$, and let the pre-shared key space $\KEpskeyspace$ be the set $\bits^\hashlen$.
% 	Let $\advA$ be an adversary against the MSKE security of $\TLSPSK$ running in time $t$, and let $\advA$ make $\qNewSecret$, resp.\ $\qSend$ queries to its $\NewSecret$, resp.\ $\Send$ oracles.
% 	Further, let $\qRO$ be the number of random oracle queries $\advA$ makes in total to its random oracles $\mathlist{\RO_\Thash,\RO_{\binder},\dotsc, \RO_{\rms}}$.
	Then,
	\begin{align*}
		&\genAdv{\KESEC}{\TLSPSK}{t, \qNewSecret, \qSend, \qRevSessionKey, \qRevLongTermKey, \qTest, \qRO} \\
		&\qquad\leq \frac{2\qSend^2}{2^{\nl}} + \frac{(\qRO+\qSend)^2 + \qNewSecret^2 + (\qRO+6\qSend)^2 + \qRO \cdot \qNewSecret + \qSend}{2^{\hashlen}}
	\end{align*}
\end{theorem}

From this we obtain the following overall result for the TLS~1.3 PSK-only mode via the same series of arguments as in Section~\ref{sec:psk-ecdhe-ke-full-bound}.

\begin{corollary}\label{cor:psk-ke}
	Let $\TLSPSK$ be the TLS~1.3 PSK-only handshake protocol as specified on the left-hand side in \autoref{fig:tls-handshake}.
	Let $\nl$ be the length in bits of the nonce, let $\hashlen$ be the output length in bits of $\abstractHash$, and let the pre-shared key space be $\KEpskeyspace = \bits^\hashlen$.
	Let $\abstractHash$ be modeled as a random oracle $\RO_\hash$.
	%
% 	Let $\advA$ be an adversary against the $\KESEC$ security of $\TLSPSK$ running in time $t$, and let $\advA$ make $\qRO$, $\qNewSecret$, resp.\ $\qSend$ queries to its $\RO_\hash$, $\NewSecret$, resp.\ $\Send$ oracles.
	Then,
	\begin{align*}
		&\genAdv{\KESEC}{\TLSPSK}{t, \qNewSecret, \qSend, \qRevSessionKey, \qRevLongTermKey, \qTest, \qRO} \\
		&\qquad\leq \frac{2\qSend^2}{2^{\nl}} + \frac{(\qRO+\qSend)^2 + \qNewSecret^2 + (\qRO+6\qSend)^2 + \qRO \cdot \qNewSecret + \qSend}{2^{\hashlen}} \\
		&\qquad\qquad + \frac{2(12\qSend+\qRO)^2 + 2\qRO^2 + 8(\qRO+36\qSend)^2}{2^{\hashlen}}.
	\end{align*}
\end{corollary}



%%% Local Variables:
%%% mode: latex
%%% TeX-master: "main"
%%% End:

\iffull
	% no short table
\else

	\begin{table}[t]
		\centering
		\small
		
		\renewcommand{\arraystretch}{0.001}
		\renewcommand{\tabcolsep}{0.15cm}
		\begin{tabular}{@{}llllllllllll@{}}
		\toprule
		 & \multicolumn{4}{c}{Adversary resources}	& & & & \multicolumn{2}{c}{Security bound}	\\
		 \cmidrule{2-5} \cmidrule{9-10} \\
		$b$ & $t$~~~~~	& $\#N$	& $\#S$ & $\#RO$ & Target  &&  Mode~~~~~~~ & DFGS\,{\scriptsize\cite{JC:DFGS21}}	& Us (Cor.~\ref{cor:full-psk-ecdhe-ke},~\ref{cor:psk-ke})	\\
		\midrule
	128 & $2^{60}$ & $2^{25}$ & $2^{35}$ & $2^{50}$ & $2^{-68}$ && PSK-only & \cellcolor{green!25}$\approx 2^{-119}$	& \cellcolor{green!25}$\approx 2^{-152}$	\\ 
	128 & $2^{80}$ & $2^{35}$ & $2^{55}$ & $2^{70}$ & $2^{-48}$ && PSK-only & \cellcolor{green!25}$\approx 2^{-59~}$	& \cellcolor{green!25}$\approx 2^{-112}$	\\ 
	\midrule
	128 & $2^{60}$ & $2^{25}$ & $2^{35}$ & $2^{50}$ & $2^{-68}$ && \texttt{secp256r1} & $\approx 2^{-61}$	& \cellcolor{green!25}$\approx 2^{-132}$	\\ 
	128 & $2^{80}$ & $2^{35}$ & $2^{55}$ & $2^{70}$ & $2^{-48}$ && \texttt{secp256r1} & $1$	& \cellcolor{green!25}$\approx 2^{-92~}$	\\ 
	\midrule
	128 & $2^{60}$ & $2^{25}$ & $2^{35}$ & $2^{50}$ & $2^{-68}$ && \texttt{x25519} & $\approx 2^{-57}$	& \cellcolor{green!25}$\approx 2^{-128}$	\\ 
	128 & $2^{80}$ & $2^{35}$ & $2^{55}$ & $2^{70}$ & $2^{-48}$ && \texttt{x25519} & $1$	&\cellcolor{green!25}$\approx 2^{-88~}$	\\ 
	\midrule
	192 & $2^{60}$ & $2^{25}$ & $2^{35}$ & $2^{50}$ & $2^{-132}$ && \texttt{secp384r1} & \cellcolor{green!25}$\approx 2^{-189}$ 	&\cellcolor{green!25}$\approx 2^{-259}$ 	\\ 
	192 & $2^{80}$ & $2^{35}$ & $2^{55}$ & $2^{70}$ & $2^{-112}$ && \texttt{secp384r1} & $\approx 2^{-108}$	&\cellcolor{green!25}$\approx 2^{-219}$ 	\\ 
	\midrule
	224 & $2^{60}$ & $2^{25}$ & $2^{35}$ & $2^{50}$ & $2^{-164}$ && \texttt{x448} & $\cellcolor{green!25}\approx 2^{-200}$ 	&\cellcolor{green!25}$\approx 2^{-280}$ 	\\ 
	224 & $2^{80}$ & $2^{35}$ & $2^{55}$ & $2^{70}$ & $2^{-144}$ && \texttt{x448} & $\approx 2^{-110}$	&\cellcolor{green!25}$\approx 2^{-240}$ 	\\ 
	\midrule
	256 & $2^{60}$ & $2^{25}$ & $2^{35}$ & $2^{50}$ & $2^{-196}$ && \texttt{secp521r1} & $\cellcolor{green!25}\approx 2^{-200}$ 	&\cellcolor{green!25}$\approx 2^{-280}$ 	\\ 
	256 & $2^{80}$ & $2^{35}$ & $2^{55}$ & $2^{70}$ & $2^{-176}$ && \texttt{secp521r1} & $\approx 2^{-110}$	&\cellcolor{green!25}$\approx 2^{-240}$ 	\\ 
	%
	% $2^{60}$	&$2^{20}$	&$2^{35}$	& \texttt{x25519}	&$2^{-68}$  & $\approx 2^{-119}$	& $\approx 2^{-152}$	&& $\approx 2^{-57}$	&$\approx 2^{-129}$ \\	 \midrule 
	% $2^{80}$	&$2^{30}$	&$2^{55}$	& \texttt{secp256r1}	&$2^{-48}$  & $\approx 2^{-59}$	& $\approx 2^{-112}$	&& 1			&$\approx 2^{-93}$ \\	 
	% $2^{80}$	&$2^{30}$	&$2^{55}$	& \texttt{x25519}	&$2^{-48}$  & $\approx 2^{-59}$	& $\approx 2^{-112}$	&& 1			&$\approx 2^{-89}$ \\
	% $2^{80}$	&$2^{30}$	&$2^{55}$	& \texttt{secp384r1}	&$2^{-112}$ & $\approx 2^{-146}$	& ---	&& $\approx 2^{-109}$	&$\approx 2^{-220}$	\\
		\bottomrule
		\end{tabular}

		\medskip
		
		\caption{%
			Exemplary concrete advantages of a key exchange adversary with given resources $t$ (running time), $\#N$ (number of pre-shared keys), $\#S$ (number of sessions), and $\#RO$ (number of random oracle queries) in breaking the security of the TLS~1.3 PSK handshake protocols.
			%
			Numbers based on the prior bounds by Dowling et al.~\cite{JC:DFGS21}
			and our bounds for PSK-(EC)DHE and PSK-only (in Corollaries~\ref{cor:full-psk-ecdhe-ke} resp.~\ref{cor:psk-ke}).
			``Target'' indicates the maximal advantage~$t/2^b$ tolerable for a given bound on $t$ when aiming for the respective curve's (or hash function's, in case of PSK-only mode) bit security level~$b$;
			entries in \colorbox{green!25}{green}-shaded cells meet that target.
			Mode indicates PSK-only mode (with \SHA{384}) or otherwise PSK-(EC)DHE mode with the given curve \texttt{secp256r1}, \texttt{x25519} (with \SHA{256}), or \texttt{secp384r1}, \texttt{x448}, \texttt{secp521r1} (with \SHA{384}).
		}
		\label{tbl:bounds-overview}
	\end{table}
\fi


\section{Evaluation}
\label{sec:evaluation}
Asymptotically, our tighter security bounds improve on prior analysis of TLS~1.3 by a quadratic factor.
We evaluate ours and prior bounds over a wide range of fully concrete resource parameters, following the approach of Davis and Günther~\cite{ACNS:DavGun21}.
\iffull
	The
full range of evaluated parameters is given in 
	Tables~\ref{tbl:bounds-full-psk-only} and~\ref{tbl:bounds-full-psk-dhe} 
		below,
along with reasoning for how we chose the various ranges of resource parameters.
The tables show that while the prior PSK-(EC)DHE bound by Dowling et al.~\cite{JC:DFGS21} meets the target security goals in a number of configurations,
there are at least some settings for all elliptic-curve groups in which the targeted security is not met.
Our bounds do significantly better than the target in all configurations we considered.
The gap for the PSK-only handshake is less significant as the loosest prior reduction for TLS~1.3 was to the Diffie--Hellman problem.

Overall, our bounds improve on previous analyses of the PSK-only handshake by~$15$ to~$53$ bits of security, and those of the PSK-(EC)DHE handshake by~$60$ to~$131$ bits of security, across all our parameters evaluated.

	\def\EvalTitle{Evaluation Details}
\iffull
	\subsection{\EvalTitle}
\else
	\newpage
	\section{\EvalTitle}
\fi
\label{app:evaluation}

In the following, we will briefly explain the reasoning behind each of our specific resource parameter estimates. 
An adversary in the MSKE game (cf.\ Definition~\ref{def:MSKE-security}) is limited in its runtime~$t$, the number of pre-shared keys~$\#N$, and distinct protocol sessions~$\#S$ it can observe or interact with, and the number of random oracle queries~$\#RO$ it can make.
This last quantity captures offline work the adversary spends on computing the hash function~$\Hash$, which in our analysis we model as random oracle.
The choice of ciphersuite enters the bound through the length of the symmetric session keys and pre-shared keys.
For the PSK-(EC)DHE handshake, the bound also depends on the underlying Diffie--Hellman group.


\paragraph{Runtime $t \in \{2^{40}, 2^{60}, 2^{80}\}$.}
We consider a range of adversarial runtimes from easily achievable ($2^{40}$ operations) to state-scaled computational power ($2^{80}$ operations). 

\paragraph{Random oracle queries $\#RO \in \{2^{40}, 2^{60}, 2^{80}\}$.}
The number of random oracle queries models the number of hash function computations an adversary is capable of computing. Accordingly, we scale the number of RO queries with the runtime, always setting $\#RO = t/2^{10}$.

\paragraph{Number of pre-shared keys $\#N \in \{2^{25}, 2^{35}\}$.}
The world's largest certificate authority Let's~Encrypt reports $\approx 2^{27.5}$ active certificates for fully-qualified domains.%
\footnote{\url{https://letsencrypt.org/stats/}} %% last checked 2021-09-29: 193M active certs, 252M fully-qualified domains certified
While not every \emph{user} of TLS~1.3 will perform resumption, our model counts the number of \emph{pre-shared keys},
where typically users may hold many pre-shared keys, with servers regularly issuing several PSKs per full-handshake connection for later resumption.
We hence estimate that the number of pre-shared keys accessible to a globally-scaled adversary may well exceed the reported number of (server) certificates.

\paragraph{Number of sessions $\#S \in \{2^{35}, 2^{45}, 2^{55}\}$.}
We use the same estimates as Davis and G{\"u}nther~\cite{ACNS:DavGun21}, based on Google's and Firefox's usage reports.%
\footnote{\url{https://transparencyreport.google.com/}, \url{https://telemetry.mozilla.org/}}
With a daily browser user base of $2$ billion ($\approx 2^{31}$) and an HTTPS traffic encryption rate in the range of $76$--$98\%$,
we estimate an adversary could encounter up to~$2^{55}$ distinct sessions over an extended time period.
Note that although the PSK handshakes are less commonly used by browsers than the full TLS~1.3 handshake, they are frequently used by embedded and low-powered devices which do not appear in these reports.
Naturally, we do not allow the number of sessions to exceed the adversary's runtime $t$.

\paragraph{Diffie--Hellman groups.}
There are ten Diffie--Hellman groups standardized for use with the PSK-(EC)DHE handshake: five elliptic-curve groups and five finite-field groups. 
We reduce to the security of the strong Diffie--Hellman assumption in each of these groups.
Davis and Günther gave a proof of hardness in the generic group model (GGM) for the strong DH problem.
This result is a good heuristic for elliptic-curve groups, but not for finite-field ones because they are vulnerable to index-calculus based attacks not covered by the GGM.
The elliptic-curve groups are more efficient and more widely used than finite-field groups, so we restrict our analysis to these groups:
\texttt{secp256r1}, \texttt{x25519}, \texttt{secp384r1}, \texttt{x448}, \texttt{secp521r1}.
For each group, we give in Table~\ref{tbl:bounds-full-psk-dhe} the order~$p$ and the expected security level~$b$ in bits.
We use the security level $b$ to determine the choice of hash function and the target security level for the entire PSK-(EC)DHE handshake.

\paragraph{Ciphersuite and symmetric lengths.}
Our bounds reduce to the collision resistance of the random oracle $\ROthash$, which models the handshake's hash function.
The choice of hash function also determines the length of the session and resumption keys.
TLS~1.3 has five ciphersuites, all of which set the hash function to be either $\SHA{256}$ or $\SHA{384}$.
For PSK-(EC)DHE mode, we select $\SHA{256}$ as the hash function whenever a curve with $128$-bit security is used and we select $\SHA{384}$ for higher-security curves.
As our results of Section~\ref{sec:ks-indiff} only apply to PSK-only mode when $\SHA{256}$ is the hash function, we always use $\SHA{256}$ and a target-security level of $128$ bits.

\begin{table}[t]
	\centering
% 	\fontsize{4.5}{5}\selectfont % smaller than \tiny
	\footnotesize
	\renewcommand{\arraystretch}{0.01}
	\renewcommand{\tabcolsep}{0.15cm}
	\vspace{-0.3cm} %% a little higher
	\begin{tabular}{@{}lllllll@{}}
		\toprule
		\multicolumn{4}{c}{Adversary resources}		&		& \multicolumn{2}{c}{PSK-only}	\\
		\cmidrule{1-4} \cmidrule{6-7}
		$t$	& $\#N$	& $\#S$ & $\#RO$ & Target $t/2^b$	& DFGS\,{\cite{JC:DFGS21}}~	& Us~{(Cor.~\ref{cor:psk-ke})} \\
		\midrule	
$2^{40}$	&$2^{25}$	&$2^{35}$	&$2^{30}$	&$2^{-88}$	&\cellcolor{green!25}$\approx 2^{-158}$	&\cellcolor{green!25}$\approx 2^{-173}$	\\
$2^{40}$	&$2^{35}$	&$2^{35}$	&$2^{30}$	&$2^{-88}$	&\cellcolor{green!25}$\approx 2^{-150}$	&\cellcolor{green!25}$\approx 2^{-173}$	\\
\midrule
$2^{60}$	&$2^{25}$	&$2^{35}$	&$2^{50}$	&$2^{-68}$	&\cellcolor{green!25}$\approx 2^{-119}$	&\cellcolor{green!25}$\approx 2^{-152}$	\\
$2^{60}$	&$2^{25}$	&$2^{45}$	&$2^{50}$	&$2^{-68}$	&\cellcolor{green!25}$\approx 2^{-109}$	&\cellcolor{green!25}$\approx 2^{-151}$	\\
$2^{60}$	&$2^{25}$	&$2^{55}$	&$2^{50}$	&$2^{-68}$	&\cellcolor{green!25}$\approx 2^{-99}$	&\cellcolor{green!25}$\approx 2^{-133}$	\\
$2^{60}$	&$2^{35}$	&$2^{35}$	&$2^{50}$	&$2^{-68}$	&\cellcolor{green!25}$\approx 2^{-119}$	&\cellcolor{green!25}$\approx 2^{-152}$	\\
$2^{60}$	&$2^{35}$	&$2^{45}$	&$2^{50}$	&$2^{-68}$	&\cellcolor{green!25}$\approx 2^{-109}$	&\cellcolor{green!25}$\approx 2^{-151}$	\\
$2^{60}$	&$2^{35}$	&$2^{55}$	&$2^{50}$	&$2^{-68}$	&\cellcolor{green!25}$\approx 2^{-99}$	&\cellcolor{green!25}$\approx 2^{-133}$	\\
\midrule
$2^{80}$	&$2^{25}$	&$2^{35}$	&$2^{70}$	&$2^{-48}$	&\cellcolor{green!25}$\approx 2^{-79}$	&\cellcolor{green!25}$\approx 2^{-112}$	\\
$2^{80}$	&$2^{25}$	&$2^{45}$	&$2^{70}$	&$2^{-48}$	&\cellcolor{green!25}$\approx 2^{-69}$	&\cellcolor{green!25}$\approx 2^{-112}$	\\
$2^{80}$	&$2^{25}$	&$2^{55}$	&$2^{70}$	&$2^{-48}$	&\cellcolor{green!25}$\approx 2^{-59}$	&\cellcolor{green!25}$\approx 2^{-112}$	\\
$2^{80}$	&$2^{35}$	&$2^{35}$	&$2^{70}$	&$2^{-48}$	&\cellcolor{green!25}$\approx 2^{-79}$	&\cellcolor{green!25}$\approx 2^{-112}$	\\
$2^{80}$	&$2^{35}$	&$2^{45}$	&$2^{70}$	&$2^{-48}$	&\cellcolor{green!25}$\approx 2^{-69}$	&\cellcolor{green!25}$\approx 2^{-112}$	\\
$2^{80}$	&$2^{35}$	&$2^{55}$	&$2^{70}$	&$2^{-48}$	&\cellcolor{green!25}$\approx 2^{-59}$	&\cellcolor{green!25}$\approx 2^{-112}$	\\
	\bottomrule
	\end{tabular}
	
	\medskip

	\caption{%
		Concrete advantages of a key exchange adversary with given resources $t$ (running time), $\#N$ (number of pre-shared keys), $\#S$ (number of sessions), and $\#RO$ (number of random oracle queries) in breaking the security of the TLS~1.3 PSK-only handshake protocol with a ciphersuite targeting $128$-bit security.
		%
		Numbers based on the prior bounds by Dowling et al.~\cite{JC:DFGS21}
		and our bound for PSK-only in Corollary~\ref{cor:psk-ke}.
		``Target'' indicates the maximal advantage~$t/2^b$ tolerable for a given bound on $t$ when aiming for the bit security level~$b = 128$;
		entries in \colorbox{green!25}{green}-shaded cells meet that target.
		We assume that the ciphersuite uses $\SHA{256}$ as its hash function (see Appendix~\ref{app:domsep} for further explanation).
	}
\label{tbl:bounds-full-psk-only}
\end{table}

\begin{table}[p]
	\centering
% 	\fontsize{4.5}{5}\selectfont % smaller than \tiny
	% \tiny
	\renewcommand{\arraystretch}{0.01}
	\renewcommand{\tabcolsep}{0.15cm}
	%\vspace{-0.5cm} %% a little higher
	\resizebox{!}{.42\textheight}{%
	\begin{tabular}{@{}llllllll@{}}
		\toprule
		\multicolumn{4}{c}{Adversary resources}		&&		& \multicolumn{2}{c}{PSK-(EC)DHE}	\\[-0.5mm]
		\cmidrule{1-4} \cmidrule{7-8}
		$t$	& $\#N$	& $\#S$ & $\#RO$ & Curve (\fullelse{bit security~$b$, group order~$p$}{bit sec.\!~$b$,\! order~$p$})	& Target $t/2^b$	& DFGS\,{\cite{JC:DFGS21}}~	& Us~{(Cor.~\ref{cor:full-psk-ecdhe-ke})} \\[-0.5mm]
		\midrule
$2^{40}$	&$2^{25}$	&$2^{35}$	&$2^{30}$	&\texttt{secp256r1} ($b \!=\! 128$, \! $p \!\approx\! 2^{256}$)	&$2^{-88}$	&\cellcolor{green!25}$\approx 2^{-92}$	&\cellcolor{green!25}$\approx 2^{-167}$	\\
$2^{40}$	&$2^{35}$	&$2^{35}$	&$2^{30}$	&\texttt{secp256r1} ($b \!=\! 128$, \! $p \!\approx\! 2^{256}$)	&$2^{-88}$	&$\approx 2^{-82}$	& \cellcolor{green!25}$\approx 2^{-167}$	\\
\midrule
$2^{40}$	&$2^{25}$	&$2^{35}$	&$2^{30}$	&\texttt{x25519} ($b \!=\! 128$, \! $p \!\approx\! 2^{252}$)	&$2^{-88}$	&\cellcolor{green!25}$\approx 2^{-92}$	&\cellcolor{green!25}$\approx 2^{-163}$	\\
$2^{40}$	&$2^{35}$	&$2^{35}$	&$2^{30}$	&\texttt{x25519} ($b \!=\! 128$, \! $p \!\approx\! 2^{252}$)	&$2^{-88}$	&$\approx 2^{-82}$	& \cellcolor{green!25}$\approx 2^{-163}$	\\
\midrule
$2^{40}$	&$2^{25}$	&$2^{35}$	&$2^{30}$	&\texttt{secp384r1} ($b \!=\! 192$, \! $p \!\approx\! 2^{384}$)	&$2^{-152}$	&\cellcolor{green!25}$\approx 2^{-220}$	&\cellcolor{green!25}$\approx 2^{-294}$	\\
$2^{40}$	&$2^{35}$	&$2^{35}$	&$2^{30}$	&\texttt{secp384r1} ($b \!=\! 192$, \! $p \!\approx\! 2^{384}$)	&$2^{-152}$	&\cellcolor{green!25}$\approx 2^{-210}$	&\cellcolor{green!25}$\approx 2^{-294}$	\\
\midrule
$2^{40}$	&$2^{25}$	&$2^{35}$	&$2^{30}$	&\texttt{x448} ($b \!=\! 224$, \! $p \!\approx\! 2^{446}$)	&$2^{-184}$	&\cellcolor{green!25}$\approx 2^{-220}$	&\cellcolor{green!25}$\approx 2^{-301}$	\\
$2^{40}$	&$2^{35}$	&$2^{35}$	&$2^{30}$	&\texttt{x448} ($b \!=\! 224$, \! $p \!\approx\! 2^{446}$)	&$2^{-184}$	&\cellcolor{green!25}$\approx 2^{-210}$	&\cellcolor{green!25}$\approx 2^{-301}$	\\
\midrule
$2^{40}$	&$2^{25}$	&$2^{35}$	&$2^{30}$	&\texttt{secp521r1} ($b \!=\! 256$, \! $p \!\approx\! 2^{521}$)	&$2^{-216}$	&\cellcolor{green!25}$\approx 2^{-220}$	&\cellcolor{green!25}$\approx 2^{-301}$	\\
$2^{40}$	&$2^{35}$	&$2^{35}$	&$2^{30}$	&\texttt{secp521r1} ($b \!=\! 256$, \! $p \!\approx\! 2^{521}$)	&$2^{-216}$	&$\approx 2^{-210}$	& \cellcolor{green!25}$\approx 2^{-301}$	\\
\midrule[1pt]
$2^{60}$	&$2^{25}$	&$2^{35}$	&$2^{50}$	&\texttt{secp256r1} ($b \!=\! 128$, \! $p \!\approx\! 2^{256}$)	&$2^{-68}$	&$\approx 2^{-61}$	& \cellcolor{green!25}$\approx 2^{-132}$	\\
$2^{60}$	&$2^{25}$	&$2^{45}$	&$2^{50}$	&\texttt{secp256r1} ($b \!=\! 128$, \! $p \!\approx\! 2^{256}$)	&$2^{-68}$	&$\approx 2^{-40}$	& \cellcolor{green!25}$\approx 2^{-132}$	\\
$2^{60}$	&$2^{25}$	&$2^{55}$	&$2^{50}$	&\texttt{secp256r1} ($b \!=\! 128$, \! $p \!\approx\! 2^{256}$)	&$2^{-68}$	&$\approx 2^{-12}$	& \cellcolor{green!25}$\approx 2^{-127}$	\\
$2^{60}$	&$2^{35}$	&$2^{35}$	&$2^{50}$	&\texttt{secp256r1} ($b \!=\! 128$, \! $p \!\approx\! 2^{256}$)	&$2^{-68}$	&$\approx 2^{-60}$	& \cellcolor{green!25}$\approx 2^{-132}$	\\
$2^{60}$	&$2^{35}$	&$2^{45}$	&$2^{50}$	&\texttt{secp256r1} ($b \!=\! 128$, \! $p \!\approx\! 2^{256}$)	&$2^{-68}$	&$\approx 2^{-32}$	& \cellcolor{green!25}$\approx 2^{-132}$	\\
$2^{60}$	&$2^{35}$	&$2^{55}$	&$2^{50}$	&\texttt{secp256r1} ($b \!=\! 128$, \! $p \!\approx\! 2^{256}$)	&$2^{-68}$	&$\approx 2^{-2}$	& \cellcolor{green!25}$\approx 2^{-127}$	\\
\midrule
$2^{60}$	&$2^{25}$	&$2^{35}$	&$2^{50}$	&\texttt{x25519} ($b \!=\! 128$, \! $p \!\approx\! 2^{252}$)	&$2^{-68}$	&$\approx 2^{-57}$	& \cellcolor{green!25}$\approx 2^{-128}$	\\
$2^{60}$	&$2^{25}$	&$2^{45}$	&$2^{50}$	&\texttt{x25519} ($b \!=\! 128$, \! $p \!\approx\! 2^{252}$)	&$2^{-68}$	&$\approx 2^{-37}$	& \cellcolor{green!25}$\approx 2^{-128}$	\\
$2^{60}$	&$2^{25}$	&$2^{55}$	&$2^{50}$	&\texttt{x25519} ($b \!=\! 128$, \! $p \!\approx\! 2^{252}$)	&$2^{-68}$	&$\approx 2^{-12}$	& \cellcolor{green!25}$\approx 2^{-123}$	\\
$2^{60}$	&$2^{35}$	&$2^{35}$	&$2^{50}$	&\texttt{x25519} ($b \!=\! 128$, \! $p \!\approx\! 2^{252}$)	&$2^{-68}$	&$\approx 2^{-57}$	& \cellcolor{green!25}$\approx 2^{-128}$	\\
$2^{60}$	&$2^{35}$	&$2^{45}$	&$2^{50}$	&\texttt{x25519} ($b \!=\! 128$, \! $p \!\approx\! 2^{252}$)	&$2^{-68}$	&$\approx 2^{-32}$	& \cellcolor{green!25}$\approx 2^{-128}$	\\
$2^{60}$	&$2^{35}$	&$2^{55}$	&$2^{50}$	&\texttt{x25519} ($b \!=\! 128$, \! $p \!\approx\! 2^{252}$)	&$2^{-68}$	&$\approx 2^{-2}$	& \cellcolor{green!25}$\approx 2^{-123}$	\\
\midrule
$2^{60}$	&$2^{25}$	&$2^{35}$	&$2^{50}$	&\texttt{secp384r1} ($b \!=\! 192$, \! $p \!\approx\! 2^{384}$)	&$2^{-132}$	&\cellcolor{green!25}$\approx 2^{-189}$	&\cellcolor{green!25}$\approx 2^{-259}$	\\
$2^{60}$	&$2^{25}$	&$2^{45}$	&$2^{50}$	&\texttt{secp384r1} ($b \!=\! 192$, \! $p \!\approx\! 2^{384}$)	&$2^{-132}$	&\cellcolor{green!25}$\approx 2^{-168}$	&\cellcolor{green!25}$\approx 2^{-259}$	\\
$2^{60}$	&$2^{25}$	&$2^{55}$	&$2^{50}$	&\texttt{secp384r1} ($b \!=\! 192$, \! $p \!\approx\! 2^{384}$)	&$2^{-132}$	&\cellcolor{green!25}$\approx 2^{-140}$	&\cellcolor{green!25}$\approx 2^{-254}$	\\
$2^{60}$	&$2^{35}$	&$2^{35}$	&$2^{50}$	&\texttt{secp384r1} ($b \!=\! 192$, \! $p \!\approx\! 2^{384}$)	&$2^{-132}$	&\cellcolor{green!25}$\approx 2^{-188}$	&\cellcolor{green!25}$\approx 2^{-259}$	\\
$2^{60}$	&$2^{35}$	&$2^{45}$	&$2^{50}$	&\texttt{secp384r1} ($b \!=\! 192$, \! $p \!\approx\! 2^{384}$)	&$2^{-132}$	&\cellcolor{green!25}$\approx 2^{-160}$	&\cellcolor{green!25}$\approx 2^{-259}$	\\
$2^{60}$	&$2^{35}$	&$2^{55}$	&$2^{50}$	&\texttt{secp384r1} ($b \!=\! 192$, \! $p \!\approx\! 2^{384}$)	&$2^{-132}$	&$\approx 2^{-130}$	& \cellcolor{green!25}$\approx 2^{-254}$	\\
\midrule
$2^{60}$	&$2^{25}$	&$2^{35}$	&$2^{50}$	&\texttt{x448} ($b \!=\! 224$, \! $p \!\approx\! 2^{446}$)	&$2^{-164}$	&\cellcolor{green!25}$\approx 2^{-200}$	&\cellcolor{green!25}$\approx 2^{-280}$	\\
$2^{60}$	&$2^{25}$	&$2^{45}$	&$2^{50}$	&\texttt{x448} ($b \!=\! 224$, \! $p \!\approx\! 2^{446}$)	&$2^{-164}$	&\cellcolor{green!25}$\approx 2^{-170}$	&\cellcolor{green!25}$\approx 2^{-279}$	\\
$2^{60}$	&$2^{25}$	&$2^{55}$	&$2^{50}$	&\texttt{x448} ($b \!=\! 224$, \! $p \!\approx\! 2^{446}$)	&$2^{-164}$	&$\approx 2^{-140}$	& \cellcolor{green!25}$\approx 2^{-261}$	\\
$2^{60}$	&$2^{35}$	&$2^{35}$	&$2^{50}$	&\texttt{x448} ($b \!=\! 224$, \! $p \!\approx\! 2^{446}$)	&$2^{-164}$	&\cellcolor{green!25}$\approx 2^{-190}$	&\cellcolor{green!25}$\approx 2^{-280}$	\\
$2^{60}$	&$2^{35}$	&$2^{45}$	&$2^{50}$	&\texttt{x448} ($b \!=\! 224$, \! $p \!\approx\! 2^{446}$)	&$2^{-164}$	&$\approx 2^{-160}$	& \cellcolor{green!25}$\approx 2^{-279}$	\\
$2^{60}$	&$2^{35}$	&$2^{55}$	&$2^{50}$	&\texttt{x448} ($b \!=\! 224$, \! $p \!\approx\! 2^{446}$)	&$2^{-164}$	&$\approx 2^{-130}$	& \cellcolor{green!25}$\approx 2^{-261}$	\\
\midrule
$2^{60}$	&$2^{25}$	&$2^{35}$	&$2^{50}$	&\texttt{secp521r1} ($b \!=\! 256$, \! $p \!\approx\! 2^{521}$)	&$2^{-196}$	&\cellcolor{green!25}$\approx 2^{-200}$	&\cellcolor{green!25}$\approx 2^{-280}$	\\
$2^{60}$	&$2^{25}$	&$2^{45}$	&$2^{50}$	&\texttt{secp521r1} ($b \!=\! 256$, \! $p \!\approx\! 2^{521}$)	&$2^{-196}$	&$\approx 2^{-170}$	& \cellcolor{green!25}$\approx 2^{-279}$	\\
$2^{60}$	&$2^{25}$	&$2^{55}$	&$2^{50}$	&\texttt{secp521r1} ($b \!=\! 256$, \! $p \!\approx\! 2^{521}$)	&$2^{-196}$	&$\approx 2^{-140}$	& \cellcolor{green!25}$\approx 2^{-261}$	\\
$2^{60}$	&$2^{35}$	&$2^{35}$	&$2^{50}$	&\texttt{secp521r1} ($b \!=\! 256$, \! $p \!\approx\! 2^{521}$)	&$2^{-196}$	&$\approx 2^{-190}$	& \cellcolor{green!25}$\approx 2^{-280}$	\\
$2^{60}$	&$2^{35}$	&$2^{45}$	&$2^{50}$	&\texttt{secp521r1} ($b \!=\! 256$, \! $p \!\approx\! 2^{521}$)	&$2^{-196}$	&$\approx 2^{-160}$	& \cellcolor{green!25}$\approx 2^{-279}$	\\
$2^{60}$	&$2^{35}$	&$2^{55}$	&$2^{50}$	&\texttt{secp521r1} ($b \!=\! 256$, \! $p \!\approx\! 2^{521}$)	&$2^{-196}$	&$\approx 2^{-130}$	& \cellcolor{green!25}$\approx 2^{-261}$	\\
\midrule[1pt]
$2^{80}$	&$2^{25}$	&$2^{35}$	&$2^{70}$	&\texttt{secp256r1} ($b \!=\! 128$, \! $p \!\approx\! 2^{256}$)	&$2^{-48}$	&$\approx 2^{-21}$	& \cellcolor{green!25}$\approx 2^{-92}$	\\
$2^{80}$	&$2^{25}$	&$2^{45}$	&$2^{70}$	&\texttt{secp256r1} ($b \!=\! 128$, \! $p \!\approx\! 2^{256}$)	&$2^{-48}$	&$\approx 2^{-1}$	& \cellcolor{green!25}$\approx 2^{-92}$	\\
$2^{80}$	&$2^{25}$	&$2^{55}$	&$2^{70}$	&\texttt{secp256r1} ($b \!=\! 128$, \! $p \!\approx\! 2^{256}$)	&$2^{-48}$	&$\approx 2^{19}$	& \cellcolor{green!25}$\approx 2^{-92}$	\\
$2^{80}$	&$2^{35}$	&$2^{35}$	&$2^{70}$	&\texttt{secp256r1} ($b \!=\! 128$, \! $p \!\approx\! 2^{256}$)	&$2^{-48}$	&$\approx 2^{-21}$	& \cellcolor{green!25}$\approx 2^{-92}$	\\
$2^{80}$	&$2^{35}$	&$2^{45}$	&$2^{70}$	&\texttt{secp256r1} ($b \!=\! 128$, \! $p \!\approx\! 2^{256}$)	&$2^{-48}$	&$\approx 2^{-1}$	& \cellcolor{green!25}$\approx 2^{-92}$	\\
$2^{80}$	&$2^{35}$	&$2^{55}$	&$2^{70}$	&\texttt{secp256r1} ($b \!=\! 128$, \! $p \!\approx\! 2^{256}$)	&$2^{-48}$	&$\approx 2^{20}$	& \cellcolor{green!25}$\approx 2^{-92}$	\\
\midrule
$2^{80}$	&$2^{25}$	&$2^{35}$	&$2^{70}$	&\texttt{x25519} ($b \!=\! 128$, \! $p \!\approx\! 2^{252}$)	&$2^{-48}$	&$\approx 2^{-17}$	& \cellcolor{green!25}$\approx 2^{-88}$	\\
$2^{80}$	&$2^{25}$	&$2^{45}$	&$2^{70}$	&\texttt{x25519} ($b \!=\! 128$, \! $p \!\approx\! 2^{252}$)	&$2^{-48}$	&$\approx 2^{3}$	& \cellcolor{green!25}$\approx 2^{-88}$	\\
$2^{80}$	&$2^{25}$	&$2^{55}$	&$2^{70}$	&\texttt{x25519} ($b \!=\! 128$, \! $p \!\approx\! 2^{252}$)	&$2^{-48}$	&$\approx 2^{23}$	& \cellcolor{green!25}$\approx 2^{-88}$	\\
$2^{80}$	&$2^{35}$	&$2^{35}$	&$2^{70}$	&\texttt{x25519} ($b \!=\! 128$, \! $p \!\approx\! 2^{252}$)	&$2^{-48}$	&$\approx 2^{-17}$	& \cellcolor{green!25}$\approx 2^{-88}$	\\
$2^{80}$	&$2^{35}$	&$2^{45}$	&$2^{70}$	&\texttt{x25519} ($b \!=\! 128$, \! $p \!\approx\! 2^{252}$)	&$2^{-48}$	&$\approx 2^{3}$	& \cellcolor{green!25}$\approx 2^{-88}$	\\
$2^{80}$	&$2^{35}$	&$2^{55}$	&$2^{70}$	&\texttt{x25519} ($b \!=\! 128$, \! $p \!\approx\! 2^{252}$)	&$2^{-48}$	&$\approx 2^{23}$	& \cellcolor{green!25}$\approx 2^{-88}$	\\
\midrule
$2^{80}$	&$2^{25}$	&$2^{35}$	&$2^{70}$	&\texttt{secp384r1} ($b \!=\! 192$, \! $p \!\approx\! 2^{384}$)	&$2^{-112}$	&\cellcolor{green!25}$\approx 2^{-149}$	&\cellcolor{green!25}$\approx 2^{-219}$	\\
$2^{80}$	&$2^{25}$	&$2^{45}$	&$2^{70}$	&\texttt{secp384r1} ($b \!=\! 192$, \! $p \!\approx\! 2^{384}$)	&$2^{-112}$	&\cellcolor{green!25}$\approx 2^{-129}$	&\cellcolor{green!25}$\approx 2^{-219}$	\\
$2^{80}$	&$2^{25}$	&$2^{55}$	&$2^{70}$	&\texttt{secp384r1} ($b \!=\! 192$, \! $p \!\approx\! 2^{384}$)	&$2^{-112}$	&$\approx 2^{-109}$	& \cellcolor{green!25}$\approx 2^{-219}$	\\
$2^{80}$	&$2^{35}$	&$2^{35}$	&$2^{70}$	&\texttt{secp384r1} ($b \!=\! 192$, \! $p \!\approx\! 2^{384}$)	&$2^{-112}$	&\cellcolor{green!25}$\approx 2^{-149}$	&\cellcolor{green!25}$\approx 2^{-219}$	\\
$2^{80}$	&$2^{35}$	&$2^{45}$	&$2^{70}$	&\texttt{secp384r1} ($b \!=\! 192$, \! $p \!\approx\! 2^{384}$)	&$2^{-112}$	&\cellcolor{green!25}$\approx 2^{-129}$	&\cellcolor{green!25}$\approx 2^{-219}$	\\
$2^{80}$	&$2^{35}$	&$2^{55}$	&$2^{70}$	&\texttt{secp384r1} ($b \!=\! 192$, \! $p \!\approx\! 2^{384}$)	&$2^{-112}$	&$\approx 2^{-108}$	& \cellcolor{green!25}$\approx 2^{-219}$	\\
\midrule
$2^{80}$	&$2^{25}$	&$2^{35}$	&$2^{70}$	&\texttt{x448} ($b \!=\! 224$, \! $p \!\approx\! 2^{446}$)	&$2^{-144}$	&\cellcolor{green!25}$\approx 2^{-180}$	&\cellcolor{green!25}$\approx 2^{-240}$	\\
$2^{80}$	&$2^{25}$	&$2^{45}$	&$2^{70}$	&\texttt{x448} ($b \!=\! 224$, \! $p \!\approx\! 2^{446}$)	&$2^{-144}$	&\cellcolor{green!25}$\approx 2^{-150}$	&\cellcolor{green!25}$\approx 2^{-240}$	\\
$2^{80}$	&$2^{25}$	&$2^{55}$	&$2^{70}$	&\texttt{x448} ($b \!=\! 224$, \! $p \!\approx\! 2^{446}$)	&$2^{-144}$	&$\approx 2^{-120}$	& \cellcolor{green!25}$\approx 2^{-240}$	\\
$2^{80}$	&$2^{35}$	&$2^{35}$	&$2^{70}$	&\texttt{x448} ($b \!=\! 224$, \! $p \!\approx\! 2^{446}$)	&$2^{-144}$	&\cellcolor{green!25}$\approx 2^{-170}$	&\cellcolor{green!25}$\approx 2^{-240}$	\\
$2^{80}$	&$2^{35}$	&$2^{45}$	&$2^{70}$	&\texttt{x448} ($b \!=\! 224$, \! $p \!\approx\! 2^{446}$)	&$2^{-144}$	&$\approx 2^{-140}$	& \cellcolor{green!25}$\approx 2^{-240}$	\\
$2^{80}$	&$2^{35}$	&$2^{55}$	&$2^{70}$	&\texttt{x448} ($b \!=\! 224$, \! $p \!\approx\! 2^{446}$)	&$2^{-144}$	&$\approx 2^{-110}$	& \cellcolor{green!25}$\approx 2^{-240}$	\\
\midrule
$2^{80}$	&$2^{25}$	&$2^{35}$	&$2^{70}$	&\texttt{secp521r1} ($b \!=\! 256$, \! $p \!\approx\! 2^{521}$)	&$2^{-176}$	&\cellcolor{green!25}$\approx 2^{-180}$	&\cellcolor{green!25}$\approx 2^{-240}$	\\
$2^{80}$	&$2^{25}$	&$2^{45}$	&$2^{70}$	&\texttt{secp521r1} ($b \!=\! 256$, \! $p \!\approx\! 2^{521}$)	&$2^{-176}$	&$\approx 2^{-150}$	& \cellcolor{green!25}$\approx 2^{-240}$	\\
$2^{80}$	&$2^{25}$	&$2^{55}$	&$2^{70}$	&\texttt{secp521r1} ($b \!=\! 256$, \! $p \!\approx\! 2^{521}$)	&$2^{-176}$	&$\approx 2^{-120}$	& \cellcolor{green!25}$\approx 2^{-240}$	\\
$2^{80}$	&$2^{35}$	&$2^{35}$	&$2^{70}$	&\texttt{secp521r1} ($b \!=\! 256$, \! $p \!\approx\! 2^{521}$)	&$2^{-176}$	&$\approx 2^{-170}$	& \cellcolor{green!25}$\approx 2^{-240}$	\\
$2^{80}$	&$2^{35}$	&$2^{45}$	&$2^{70}$	&\texttt{secp521r1} ($b \!=\! 256$, \! $p \!\approx\! 2^{521}$)	&$2^{-176}$	&$\approx 2^{-140}$	& \cellcolor{green!25}$\approx 2^{-240}$	\\
$2^{80}$	&$2^{35}$	&$2^{55}$	&$2^{70}$	&\texttt{secp521r1} ($b \!=\! 256$, \! $p \!\approx\! 2^{521}$)	&$2^{-176}$	&$\approx 2^{-110}$	& \cellcolor{green!25}$\approx 2^{-240}$	\\
\bottomrule
	\end{tabular}}
	
	\medskip

	\caption{%
		Concrete advantages of a key exchange adversary with given resources $t$ (running time), $\#N$ (number of pre-shared keys), $\#S$ (number of sessions), and $\#RO$ (number of random oracle queries) in breaking the security of the TLS~1.3 PSK-(EC)DH handshake protocol.
		%
		Numbers based on the prior bounds by Dowling et al.~\cite{JC:DFGS21}
		and our bound for PSK-(EC)DHE in Corollary~\ref{cor:full-psk-ecdhe-ke}.
		``Target'' indicates the maximal advantage~$t/2^b$ tolerable for a given bound on $t$ when aiming for the respective curve's bit security level~$b$;
		entries in \colorbox{green!25}{green}-shaded cells meet that target.
		See Section~\ref{sec:evaluation} and Appendix~\ref{app:evaluation} for further details.
	}
\label{tbl:bounds-full-psk-dhe}
\end{table}
\iffull
\else
	\clearpage
\fi




	

%%% Local Variables:
%%% mode: latex
%%% TeX-master: "main"
%%% End:

\section{A Careful Discussion of Domain Separation}
\label{app:domsep}

In our indifferentiability treatment of the TLS~1.3 key schedule (cf.\ Section~\ref{sec:ks-indiff}), we change what we capture as random oracles in the key exchange model.
We start with one random oracle, $\ROhash$, used wherever the hash function $\abstractHash$ would be called in the protocol.
We change this to classify queries to $\ROhash$ into two types:
\begin{description}
	\item[Type 1 queries:]
	\emph{component hashes} (via function~$\Chash$) used within $\abstractExtract$, $\abstractExpand$, and $\abstractMAC$ to compute $\HKDFExtr$, $\HKDFExpnd$, resp.\ $\HMAC$.
	
	\medskip
	
	\item[Type 2 queries:]
	\emph{transcript hashes} (via function~$\Thash$) computing hash values of protocol transcripts (or empty strings).
\end{description}
We wish to model $\Chash$ and $\Thash$ now as \emph{two} independent random oracles: $\ROchash$ resp.\ $\ROthash$.

To change the model, we can just change the pseudocode of the protocol to replace $\ROhash$ with whichever of $\ROchash$ and $\ROthash$ seems more appropriate. 
However, we must define an explicit construction that performs this substitution in a systematic way in order to give a formal proof of security.
This construction needs a Boolean condition to determine which of $\ROchash$ and $\ROthash$ should be queried, and this condition cannot be dependent on the higher-level context of the protocol's usage. 
Instead, we must define two disjoint sets $\Dom_{\Chash}$ and $\Dom_{\Thash}$ such that honest executions of TLS~1.3 only query $\ROhash$ on inputs in $\Dom_{\Chash}$ when computing $\HKDFExtr$, $\HKDFExpnd$, or $\HMAC$, and it otherwise only queries $\ROhash$ on inputs in $\Dom_{\Thash}$.


This separation must hold even when an honest session is responding to adversarially-chosen messages.
We do make some assumptions about the way that honest sessions process incoming messages. 
We assume that a server receiving a first $\ClientHello$ message from a client will not respond or execute the protocol unless the message contains correct encodings of all of the mandatory parameters for TLS~1.3. 
If the client fails to specify a valid group and key share in PSK-(EC)DHE mode, or version number, mode, and pre-shared key in any mode, the server should abort.
Of course, the $\ClientHello$ message may also contain invalid encodings of these values or even arbitrary data; we do not exclude this possibility.
Note that our conditions apply only to random-oracle queries made by honest executions of the protocol. 
An adversary may of course call $\ROhash$ on any input it chooses in either $\Dom_{\Chash}$ or $\Dom_{\Thash}$.

The TLS~1.3 handshake protocol does not provide any intentional domain separation between Type~$1$ and Type~$2$ queries.
We therefore turn to the formatting of queries to $\ROhash$ in the hopes of finding some unintentional separation.
We identify seven subtypes of query: five subtypes of Type $1$ and two subtypes of Type $2$.
Queries of each subtype have some unique formatting: a fixed length, a byte with a particular value, an encoded label. 
These attributes are heavily dependent on the specific configuration of the TLS~1.3 protocol; we therefore analyze four separate cases: two modes of operation (PSK-(EC)DHE and PSK-only mode) and two ciphersuites defining $\ROhash$ as $\SHA{256}$ and $\SHA{384}$ respectively.
Throughout, we will assume that any pre-shared-keys are the same length as the output length of $\ROhash$, i.e., $\hashlen$~bits.
This is true of resumption keys, but may not be true in general for pre-shared keys negotiated out-of-band. 
As TLS~1.3 fields length are given in (full) \emph{bytes}, we will be talking about \emph{byte lengths} if not otherwise stated in the following and use the shorthand $\hashlenBytes := \hashlen/8$ for the output length of $\ROhash$ in \emph{bytes}.
We also assume that if a Diffie--Hellman group is used, it is one of the standardized elliptic curve or finite field groups. 

All Type $1$ queries to $\ROhash$ are intermediate steps in the computation of $\HMAC$, $\HKDFExtr$, and $\HKDFExpnd$.
They consequently share some formatting which we discuss here before addressing each subtype individually.
$\HKDFExtr$ and $\HMAC$ are two names for the same function.
Given a key $K$ and input $s$, $\HKDFExpnd(K,s)$ pads $s$ with a single trailing counter byte with value \texttt{0x01}, then returns $\HMAC(K, s\|\|\texttt{0x01})$.
Therefore all Type $1$ queries to $\ROhash$ arise in the computation of $\HMAC$. 
$\HMAC[\ROhash](K,s)$ takes a key $K$ of length $\hashlenBytes$ bytes.
It then pads this key with zeroes up to the block length $\blocklenBytes$ of its hash function.
The block lengths of $\SHA{256}$ and $\SHA{384}$ are $64$ and $128$ bytes respectively. 
We call the padded key $K'$. 
Then $\HMAC[\ROhash]$ makes two queries to $\ROhash$:
\begin{enumerate}
	\item $d \gets \ROhash(K' \oplus \ipad || s)$
	\item $\ROhash(K' \oplus \opad || d)$
\end{enumerate}
$\ipad$ and $\opad$ are strings of $\blocklenBytes$ bytes. Each byte in $\ipad$ is fixed to \texttt{0x36}, and each byte in $\opad$ is fixed to \texttt{0x5c}.
The padded key $K'$ is $\blocklenBytes$ long, longer than $K$, so every Type $1$ query has a segment of length $\blocklenBytes-\hashlenBytes$ bytes in which each byte equals one of \texttt{0x36} and \texttt{0x5c}. 

Now we can present the seven subtypes of queries made by TLS~1.3. The first five types are Type~$1$ queries, and the last two (Empty and Transcript) are Type~$2$ queries.

The seven subtypes of queries are: 
\begin{enumerate}
	\item \textbf{Outer $\HMAC$ queries.}
	These queries are the second query made in the computation of $\HMAC$.
	Its key has length $\hashlenBytes$, and the digest $d$ also has length $\hashlenBytes$. 
	In between these is a segment containing $\blocklenBytes-\hashlenBytes$ bytes \texttt{0x5c}. 
	We will often refer to this segment as the ``fixed region''.  
	When the hash function is $\SHA{256}$, resp. $\SHA{384}$, the fixed region is $32$, resp $80$ bytes long. 
	The total query is $96$, resp. $176$ bytes long.
	
	\item \textbf{Inner $\HMAC$ queries.} We divide the first $\ROhash$ query made by $\HMAC$ into several subtypes; this type includes only those where the input to $\HMAC$ is an arbitrary string of length $\hashlenBytes$.
	This subtype is formatted identically to an outer $\HMAC$ query, except that the bytes of the fixed region are fixed to the value \texttt{0x36} instead of \texttt{0x5c}.  
	TLS~1.3 makes inner $\HMAC$ queries while computing $\Finished$ and $\binder$ messages (where the input is a hashed transcript), the early and master secrets, and in PSK-only mode, also the handshake secret.
	
	\item \textbf{Diffie--Hellman $\HMAC$ query.} In PSK-(EC)DHE mode, TLS~1.3 computes the handshake secret by calling $\HMAC$ on an encoded Diffie--Hellman key share. $\HMAC$'s first query is a Diffie--Hellman $\HMAC$ query. The formatting is the same as an inner $\HMAC$ hash except that the segment following the fixed region has a different length. The byte lengths ($|\G|/8$) of the encodings for each standardized Diffie--Hellman group can be found in Table~\ref{tbl:tls-groups}.

	\item \textbf{\tlsfunction{Derive-Secret} hashes.} The \tlsfunction{Derive-Secret} function is a component of the TLS key schedule~\cite[Section 7.1]{rfc8446}. Its inputs are a key of length $\hashlenBytes$, a label string of $2$ to $12$-bytes in length, and an input \tlsfield{Messages} string. 
	
	\tlsfunction{\tlsfunction{Derive-Secret}} queries $\ROhash$ three times: once to hash the \tlsfield{Messages} string, and twice as part of $\HKDFExpnd$. 
	The first of these three queries is a transcript query, and the third is an Outer $\HMAC$ query.
	The second query we call a \tlsfunction{Derive-Secret} query. 
	The \tlsfunction{Derive-Secret} query has the same formatting as Inner $\HMAC$ queries and Diffie--Hellman queries, but the segment following the fixed region contains a strictly formatted \tlsfield{HkdfLabel} struct~\cite[Section 7.1]{rfc8446}.
	
	This struct begins with a two-byte field encoding the integer value $\hashlenBytes$.
	This is followed by a variable-length vector with a $1$-byte length field containing the string \texttt{''tls13 ''} followed by a label string with length between $2$ and $12$ bytes. 
	Lastly comes a vector of length $\hashlenBytes$, prefixed with a $1$-byte field encoding its length. 
	The last byte in the input contains the \texttt{0x01}. 
	This byte is the counter mandated by the definition of $\HKDFExpnd$; however since $\HKDFExpnd$ is never called on inputs longer than $\hashlenBytes$, the counter never reaches a value higher than $1$.
	
	The total length of a the label struct, including the counter byte, is at least $\hashlenBytes+13$ bytes and at most $\hashlenBytes+23$ bytes. 
	
	\item \textbf{$\Finished$ hash.} The \tlsfunction{HKDF-Expand-Label} function is a subroutine of the \tlsfunction{Derive-Secret} function, but also called during the computation of $\Finished$ messages and the $\binder$ value~\cite[Section 4.4.4]{rfc8446}. 
	\tlsfunction{HKDF-Expand-Label} makes two calls to $\ROhash$. The second is an Outer $\HMAC$ hash; we call the first a $\Finished$ hash. 
	A $\Finished$ hash is identical to a \tlsfunction{Derive-Secret} hash, except that the label string is fixed to \texttt{finished} and the final vector has length $0$. 
	The counter byte is still present. 
	In total, the label struct occupies $19$ bytes.

	\item \textbf{Empty hashes.} Occasionally in the key schedule, TLS~1.3 calls $\ROhash$ on the empty string. 
	
	\item \textbf{Transcript hashes.} The last use of $\ROhash$ is to condense partial transcripts. 
	Each transcript includes at least a partial $\ClientHello$ message.
	We assume calling $\ROhash$. on a transcript which includes at least a partial $\ClientHello$.
	The minimum length of a partial $\ClientHello$ message in PSK-only mode is $69$~bytes.
	This includes the following fields~\cite[Section~4.1.2]{rfc8446}:
	\begin{itemize}
		\item 2 bytes \tlsfield{legacy\_version} fixed to \texttt{0x0303}
		\item 32 bytes \tlsfield{random}
		\item 1 byte \tlsfield{legacy\_session\_id} (for an empty vector with $1$-byte length field)
		\item 4 bytes \tlsfield{ciphersuites} (must include a $2$-byte length field and the value, e.g., \texttt{0x1301})
		\item 2 bytes \tlsfield{legacy\_compression\_methods} (must include a $1$-byte length field and the value \texttt{0x00})
		\item 2 bytes encoded length of \tlsfield{extensions} field
		\item 7 bytes \tlsfield{supported\_versions extension} extension~\cite[Section~4.2.1]{rfc8446} (must start with \texttt{0x002b} and include \texttt{0x0304})
		\item 6 bytes \tlsfield{psk\_key\_exchange\_modes} extension~\cite[Section~4.2.9]{rfc8446} (must start with \texttt{0x002d} and include \texttt{0x00})
		\item 9 bytes \tlsfield{pre\_shared\_key} extension~\cite[Section~4.2.11]{rfc8446} (partial: excluding the binder list; must come last, must start with \texttt{0x0029})
	\end{itemize}
	The first $43$ bytes (through the \tlsfield{extensions}' length encoding), must appear in the order displayed, although the \tlsfield{legacy\_session\_id}, \tlsfield{ciphersuites}, and \tlsfield{legacy\_compression\_methods} fields can be longer than the lengths given above. 
	We will occasionally refer to this segment as the ``fixed preface'' of a $\ClientHello$ because it must appear at the beginning of every well-formed $\ClientHello$ message. 
	The extensions can be reordered arbitrarily (except for the \tlsfield{pre\_shared\_key} extension) and additional extensions and ciphersuites can be added or repeated, up to a maximum length of $2^{16}-1$~bytes of ciphersuites and $2^{16}-2$ bytes for extensions. The overall maximum length of a $\ClientHello$ is then $2^{32} + 289$~bytes.
	A full $\ClientHello$ in PSK-only mode, including the binder list, adds at least another $3+\hashlenBytes$~bytes for a \tlsfield{binder} vector with a $3$~bytes of encoded length. The $\ClientHello$ message thus contains a minimum of $72+\hashlenBytes$~bytes and a maximum of $2^{32}+292+\hashlenBytes$ bytes.
		
	In PSK-(EC)DHE mode, two additional extensions are also mandatory: the \tlsfield{key\_share} and \tlsfield{supported\_groups} extensions~\cite[Section 9.2]{rfc8446}, so the minimum $\ClientHello$ length increases by at least $17$+$|\G|/8$ bytes, cf.\ Table~\ref{tbl:tls-groups}.
	This increase occurs for both truncated and full $\ClientHello$ messages.
	In this mode, a truncated $\ClientHello$ message is at least $86+|\G|/8$ bytes long, and a full $\ClientHello$ is at least $89+|\G|/8$ bytes long.
	
\end{enumerate}

\begin{figure}[tp]
	\centering
	\begin{tabular}{p{2.5cm}p{4cm}p{3.5cm}}
		\toprule
		Group name & \tlsfield{NamedGroup} enum value & Encoding length $|\G|/8$ \\ \midrule
		\groupname{secp256r1}~\cite{FIPS:186-4} & \texttt{0x0017} & $32$ \\ 
		\groupname{secp384r1}~\cite{FIPS:186-4} & \texttt{0x0018} & $48$ \\ 
		\groupname{secp521r1}~\cite{FIPS:186-4} & \texttt{0x0019} & $66$ \\ 
		\groupname{x25519}~\cite{rfc7748} & \texttt{0x001d} & $32$ \\ 
		\groupname{x448}~\cite{rfc7748} & \texttt{0x001E} & $56$ \\ 
		\groupname{ffdhe2048}~\cite{rfc7919} & \texttt{0x0100} & $128$ \\ 
		\groupname{ffdhe3072}~\cite{rfc7919} & \texttt{0x0101} & $192$ \\ 
		\groupname{ffdhe4096}~\cite{rfc7919} & \texttt{0x0102} & $256$ \\ 
		\groupname{ffdhe6144}~\cite{rfc7919} & \texttt{0x0103} & $384$ \\ 
		\groupname{ffdhe8192}~\cite{rfc7919} & \texttt{0x0104} & $512$ \\ \bottomrule
	\end{tabular}
	\caption{Table displaying the standardized groups for use with TLS~1.3, their encodings in the \tlsfield{NamedGroup} enum, and the length of an encoded group element in bytes.}
	\label{tbl:tls-groups}
\end{figure}
		
 
\subsection{PSK-only mode with \SHA{256}}

The block length of this hash function is $64$~bytes, and the output length is $32$~bytes.
In Table~\ref{tbl:domsep-psk-256}, we give the minimum and maximum input lengths for each of the six call types. (Diffie--Hellman $\HMAC$ calls do not occur in this mode.)

\begin{table}[tp]
	\centering
	\begin{tabular}{ccc}
		\toprule
		Type & Minimum length (bytes)~~ & Maximum length (bytes) \\
		\midrule
		Outer $\HMAC$ & $96$ & $96$ \\
		Inner $\HMAC$ & $96$ & $96$ \\
		\tlsfunction{Derive-Secret} & $109$ & $119$ \\
		$\Finished$ & $83$ & $83$ \\
		Empty & $0$ & $0$ \\
		Transcript & $69$ & $2^{32}+324$ \\
		\bottomrule
	\end{tabular}
	\medskip
	
	\caption{Table showing input lengths for hash function calls made by TLS~1.3 in PSK-only mode with \SHA{256}.}
	\label{tbl:domsep-psk-256}
\end{table}

In Table~\ref{tbl:domsep-psk-256} we note the minimum and maximum input lengths of each type of message.
For those types with overlapping length ranges, we must show they have separate domains by other means.
Outer and Inner $\HMAC$ hashes have identical lengths; however each of them has a $32$-byte fixed region. 
In outer $\HMAC$ hashes, the fixed region contains $\opad$; in inner $\HMAC$ hashes, it contains $\ipad$. 
These are distinct values, so no string can be both an outer and an inner $\HMAC$ hash. 

Transcript hashes are not domain-separated by length from any hash except the empty hashes.
We therefore turn to formatting to separate these from other types.
In the following, we visually lay out each byte of potentially overlapping inputs.

For a string to be both a transcript and an $\HMAC$ hash (outer or inner), it must be $96$~bytes (cf.~Table~\ref{tbl:domsep-psk-256}) long.
We diagram and compare a transcript hash containing a partial $\ClientHello$%
\footnote{A full $\ClientHello$ contains at least $72+\hashlenBytes\geq 104$ bytes, which is too long to be an $\HMAC$ hash.}
and an $\HMAC$ hash (outer or inner) in Figure~\ref{fig:domsep-PSKonly-256:partial-vs-HMAC}.
	
% 	\begin{tabular}{|p{4.3cm}|p{4.4cm}|p{0.9cm}|}
% 		\hline
% 		Fixed preface: $43$~bytes & Extension data: $44$~bytes & End of \tlsfield{pre\_shared\_key} extension: $9$~bytes \\
% 		\hline
% 	\end{tabular}
% 
% Next, we diagram an $\HMAC$ hash (outer or inner)
% 	
% 	\begin{tabular}{|p{3.2cm}|p{3.2cm}|p{3.2cm}|}
% 		\hline
% 		Key: $32$~bytes & Fixed region ($\ipad$ or $\opad$):  $32$~bytes & Arbitrary string: $32$~bytes \\
% 		\hline
% 	\end{tabular}

\begin{figure}[h]
	\centering
	\begin{tabular}{|p{4.3cm}|p{4.4cm}|p{0.9cm}|}
		\hline
		Fixed preface: $43$\;B & Extension data: $44$\;B & End \tlsfield{PSK}: $9$\;B \\
		\hline
	\end{tabular}
	
	\noindent
	\begin{tabular}{|p{3.2cm}|p{3.2cm}|p{3.2cm}|}
		\hline
		Key: $32$\;B & Fixed $\ipad$/$\opad$: $32$\;B & Arbitrary string: $32$~B \\
		\hline
	\end{tabular}
	
	\caption{%
		Domain separation in PSK-only mode with \SHA{256}:
		Transcript hash containing a partial $\ClientHello$ (top)
		vs.\
		(outer or inner) $\HMAC$ hash (bottom).
		``End \tlsfield{PSK}'' is the end of the \tlsfield{pre\_shared\_key} extension.
	}
	\label{fig:domsep-PSKonly-256:partial-vs-HMAC}
\end{figure}

We can see that the fixed preface of the transcript hash overlaps the fixed region of the $\HMAC$ hash that is fixed to either $\ipad$ or $\opad$. 
Consequently, the \tlsfield{legacy\_session\_id} vector must begin within the fixed region (at byte~$35$). 
This is a variable-length vector preceded by a $1$-byte length field, and its maximum length is $32$ bytes~\cite[Section 4.1.2]{rfc8446}. 
Therefore the maximum value of the length field is \texttt{0x20} and it cannot contain either byte \texttt{0x36} or \texttt{0x5c}. 
Any string containing a valid partial $\ClientHello$ therefore cannot also be a correctly formatted $\HMAC$ hash.
	
The same argument applies to $\Finished$ and \tlsfunction{Derive-Secret} hashes, both of which contain the same fixed region in the same location as inner $\HMAC$ hashes.
%	\item \textbf{Transcript vs.\ $\Finished$.}
%	Collisions may occur only on $83$-byte transcript inputs. 
%	A transcript hash of $83$~bytes contains:\\
%	\begin{tabular}{|p{4.3cm}|p{3.1cm}|p{0.9cm}|}
%		\hline
%		Fixed preface ($43$~bytes) & Arbitrary extensions ($31$~bytes) & End of PSK extension ($9$~bytes) \\
%		\hline
%	\end{tabular}
%
%	A $\Finished$ hash \new{contains}:\\
%	\begin{tabular}{|p{3.2cm}|p{3.2cm}|p{1.9cm}|}
%		\hline
%		Key ($32$~bytes) & Fixed to $\ipad$ ($32$~bytes) & \texttt{0x00200e} ||\texttt{``tls13 finished''} ||\texttt{0x0001} ($19$~bytes) \\
%		\hline
%	\end{tabular}
%	As for $\HMAC$ hashes, the fixed preface will differ from $\ipad$ in at least one byte between bytes $33$ and $43$ because this section contains the ciphersuite encodings.
%	
%	\item \textbf{Transcript vs.\ \tlsfunction{Derive-Secret}.}
%	Collisions may occur on transcripts of any length between $109$ and $119$~bytes.
%
%	We diagram a $109$~byte transcript, but the same principle applies to any input.
%	 
%	\begin{tabular}{|p{4.3cm}|p{5.7cm}|p{0.9cm}|}
%		\hline
%		Fixed preface ($43$~bytes) & Arbitrary extensions ($57$~bytes) & End of PSK extension ($9$~bytes) \\
%		\hline
%	\end{tabular}
%
%	A $109$-byte \tlsfunction{Derive-Secret} hash:
%	 
%	\begin{tabular}{|p{3.2cm}|p{3.2cm}|p{1.1cm}|p{3.4cm}|}
%		\hline
%		Key ($32$~bytes) & Fixed to $\ipad$ or $\opad$ ($32$~bytes) & \small{ \texttt{0x0020} || \texttt{0x08} || \texttt{``tls13 iv''} || \texttt{0x20} } & Transcript hash vector ($33$~bytes) || \texttt{0x01} \\
%		\hline
%	\end{tabular}
%	
%	As for $\HMAC$ hashes, the fixed preface will differ from $\ipad$ in at least one byte between bytes $33$ and $43$ because this section contains the ciphersuite encodings.
%\end{itemize}

For this mode, we define the set $\Dom_{\Thash}$ to include of the empty string and all strings of length greater than or equal to $69$~bytes for which the $35^{\text{th}}$ byte is not equal to $\ipad$ or $\opad$.
We let $\Dom_{\Chash}$ contain all other elements of $\bits^*$.

\subsection{Pre-shared key with Diffie--Hellmann mode with \SHA{256}}

Again, we present the minimum and maximum lengths of each hash type; see Table~\ref{tbl:domsep-psk-dhe-256}.
We now include Diffie--Hellman $\HMAC$ hashes, and transcript hashes include additional mandatory extensions for PSK-(EC)DHE mode.
\begin{table}[tp]
	\centering
	\begin{tabular}{ccc}
		\toprule
		Type & Minimum length (bytes)~~ & Maximum length (bytes) \\ \midrule
		Outer $\HMAC$ & $96$ & $96$ \\ 
		Inner $\HMAC$ & $96$ & $96$ \\ 
		Diffie--Hellman $\HMAC$ & $64+|\G|/8$ & $64+|\G|/8$ \\
		\tlsfunction{Derive-Secret} & $109$ & $119$ \\ 
		$\Finished$ & $83$ & $83$\\ 
		Empty & $0$ & $0$ \\ 
		Transcript & $86+|\G|/8$ & $2^{32}+324$ \\ 
		\bottomrule
	\end{tabular}
	\medskip
	
	\caption{Table showing input lengths for hash function calls made by TLS~1.3 in PSK-(EC)DHE mode with \SHA{256}. For transcript hashes, the encoding lengths $|\G|/8$ can be found in Table~\ref{tbl:tls-groups}.}
	\label{tbl:domsep-psk-dhe-256}
\end{table}

In this mode, Diffie--Hellman $\HMAC$ hashes may collide with Inner $\HMAC$ or \tlsfunction{Derive-Secret} hashes for certain choices of $\G$.
This is not a failure of domain separation because these inputs to these three types will all belong to $\Dom_{\Chash}$.
Transcript hashes now only have length overlaps with Diffie--Hellman $\HMAC$ and \tlsfunction{Derive-Secret} hashes.
In both cases, however, the same argument about the $35^{\text{th}}$ byte containing the length of \tlsfield{legacy\_session\_id} applies, and no string can be two different types. 

For this mode, the set $\Dom_{\Thash}$ consists of the empty string and all strings of length greater than or equal to $86+|\G|$~bytes for which the $35^{\text{th}}$ byte is not equal to $\ipad$ or $\opad$. $\Dom_{\Chash}$ contains all other elements of $\bits^*$. 

\subsection{Pre-shared key with Diffie--Hellmann mode with \SHA{384}}

Table~\ref{tbl:domsep-psk-dhe-384} shows the minimum and maximum lengths of each hash type for this configuration.
The hash function \SHA{384} has $48$-byte output and $128$-byte block length, so the fixed region in $\HMAC$, $\Finished$, and \tlsfunction{Derive-Secret} hashes will be $80$ bytes long.

Because $48$~byte $\HMAC$ keys are longer than the $43$~byte fixed preface of a $\ClientHello$, we cannot rely on the distinction between \tlsfield{legacy\_session\_id} and the fixed region for domain separation. Instead, we consider whether a minimum-length $\ClientHello$ can accommodate the mandatory extensions for this mode.

\begin{table}[tp]
	\centering
	\begin{tabular}{ccc}
		\toprule
		Type & Minimum length (bytes)~~ & Maximum length (bytes) \\ \midrule 

		Outer $\HMAC$ & $176$ & $176$ \\
		Inner $\HMAC$ & $176$ & $176$ \\ 
		Diffie--Hellman $\HMAC$ & $128+|\G|/8$ & $128+|\G|/8$ \\  
		\tlsfunction{Derive-Secret} & $189$ & $199$ \\
		$\Finished$ & $147$ & $147$\\ 
		Empty & $0$ & $0$ \\
		Transcript & $86+|\G|/8$ & $2^{32}+324$ \\ \hline 
	\end{tabular}
	\medskip
	
	\caption{Table showing input lengths for hash function calls made by TLS~1.3 in PSK-(EC)DHE mode with \SHA{384}.}
	\label{tbl:domsep-psk-dhe-384}
\end{table} 

We worry only about possible collisions between transcript hashes and the other types: $\Finished$, $\HMAC$, and \tlsfunction{Derive-Secret}.
We diagram a transcript hash of $176$~bytes together with an outer $\HMAC$ hash as a demonstration of the domain-separation argument in Figure~\ref{fig:domsep-PSKDHE-384:transcript-vs-HMAC}, but the same argument applies to all. 

% Transcript hash of $176$ bytes: 
% 
% \begin{tabular}{|p{3.21cm}|p{9.3cm}|p{0.7cm}|}
% 	\hline
% 	Fixed preface: $43$~bytes & Extension data: $124$~bytes & End of \tlsfield{pre\_shared\_key} extension: $9$~bytes \\
% 	\hline
% \end{tabular}
% 
% Outer $\HMAC$ hash:
% 
% \begin{tabular}{|p{3.6cm}|p{6cm}|p{3.6cm}|}
% 	\hline
% 	Key: $48$~bytes) & Fixed region ($\opad$): $80$~bytes & Arbitrary string: $48$~bytes \\
% 	\hline
% \end{tabular}

\begin{figure}[h]
	\centering
	
	\scalebox{0.9}{%
	\begin{tabular}{|p{3.21cm}|p{9.3cm}|p{0.7cm}|}
		\hline
		Fixed preface: $43$\;B & Extension data: $124$\;B & End \tlsfield{PSK}: $9$\;B \\
		\hline
	\end{tabular}
	}

	\noindent
	\scalebox{0.9}{%
	\begin{tabular}{|p{3.6cm}|p{6cm}|p{3.6cm}|}
		\hline
		Key: $48$\;B & Fixed region ($\opad$): $80$\;B & Arbitrary string: $48$\;B \\
		\hline
	\end{tabular}
	}
	
	\caption{%
		Domain separation in PSK-(EC)DHE mode with \SHA{384}:
		Transcript hash of $176$~bytes (top)
		vs.\
		outer $\HMAC$ hash (bottom).
		``End \tlsfield{PSK}'' is the end of the \tlsfield{pre\_shared\_key} extension.
	}
	\label{fig:domsep-PSKDHE-384:transcript-vs-HMAC}
\end{figure}

There are no obvious conflicts here: the fixed preface of a $\ClientHello$ message is covered by the key section of the $\HMAC$ hash, and the \tlsfield{pre\_shared\_key} extension is covered by the arbitrary string at the end.
However, notice that of the $124$ bytes available for extension data in the $\ClientHello$, $80$~of them must be fixed to $\opad$ to allow a collision. 
Even including the $5$~bytes immediately after the fixed preface and $9$~bytes reservedf or the \tlsfield{pre\_shared\_key} extension, this leaves only $58$~bytes. 
In PSK-(EC)DHE mode, five extensions are mandatory even for truncated $\ClientHello$ messages. 
They are \tlsfield{supported\_versions}~\cite[Section 4.2.1]{rfc8446} (minimum $7$~bytes), \tlsfield{supported\_groups}~\cite[Section 4.2.7]{rfc8446} (minimum $7$~bytes), \tlsfield{key\_share}~\cite[Section 4.2.8]{rfc8446} (minimum $16+|\G|/8$~bytes), \tlsfield{psk\_key\_exchange\_modes}~\cite[Section 4.2.9]{rfc8446} (minimum $6$~bytes), and \tlsfield{pre\_shared\_key}~\cite[Section 4.2.11]{rfc8446} (minimum $13$~bytes). 
Even for the smallest choice of $\G$, at least $71$ bytes are required to contain these extensions. 
At least one of the extensions must overlap with the fixed field, and will differ from $\opad$ in at least one byte.

Any valid transcript hash will need at least $92+|\G|/8$ bytes outside the fixed region: $43$~bytes for the preface and $49+|\G|/8$ for the mandatory extensions. An outer $\HMAC$ hash has only $124$ unfixed bytes and cannot meet this threshold. 
This is true also for inner $\HMAC$ hashes ($96$ unfixed bytes), and Diffie--Hellman $\HMAC$ hashes, which have $48+|\G|/8$ unfixed bytes. 
It is true for $\Finished$ hashes, which have $48$ unfixed bytes. 
And it is true for \tlsfunction{Derive-Secret} hashes, which have at most $119$ unfixed bytes.

Let us be even more clear about why this overlap means no collision is possible. 
We cannot fit all of the extensions in the $48+|\G|$ bytes after the fixed region. 
Therefore one of the extensions must start either in the fixed region, or before the fixed region. 
None of these extensions can start in the fixed region because they all begin with an extension type different from $\ipad$ or $\opad$. 
Therefore one of them must start before the fixed region and continue into the fixed region. 
We call this the ``first extension''.
The \tlsfield{pre\_shared\_key} extension must be the last extension, so it cannot be the first extension.
Therefore  the first extension is one of \tlsfield{key\_share}, \tlsfield{supported\_groups}, and \tlsfield{psk\_key\_exchange\_modes}, and \tlsfield{supported\_versions}.

All extensions start with a $4$~byte encoding of their type and length. 
This means that the first extension may contain only one arbitrary byte of data before $80$ bytes of $\ipad$ or $\opad$. All four possible extensions consist of variable-length vectors. TLS encodes all variable-length vectors with a $1$ or $2$~byte prefix encoding their length. Consequently, the entries of the vector fall in or after the fixed region. 

Each of the vector entries in the four possible first extensions begins with an element from an enum: either the \tlsfield{NamedGroup}, \tlsfield{ProtocolVersion}, or \tlsfield{PskKeyExchangeMode} enums. 
Luckily, none of these enums contain the bytes \texttt{0x36} or \texttt{0x5c}. To demonstrate this, we present the \tlsfield{NamedGroup} values in Table~\ref{tbl:tls-groups}~\cite[Section 4.2.7]{rfc8446}. The \tlsfield{ProtocolVersion} encoding for TLS~1.3 is \texttt{0x0304} \cite[Section 4.2.1]{rfc8446}, and the elements of the \tlsfield{PskKeyExchangeMode} enums are \texttt{0x00}, \texttt{0x01}, and \texttt{0xff}~\cite[Section 4.2.9]{rfc8446}. 
Of course, a $\ClientHello$ message can contain badly formed extensions. 
We assume, however, that each of the mandatory extensions must contain one correctly formatted vector entry. 
Without these entries, communication partners will not be able to select the correct version, group, or mode to execute the protocol; we assume that in this case they would abort. 
Because the fixed region contains no valid enum elements, this correctly formatted vector entry must begin after the fixed region. 
Therefore the first extension uses at most $1$ byte of the fixed region to encode meaningful data (a possible second byte of the vector length encoding). 
The mandatory extensions must occupy no more than $5$~bytes before the fixed region, $1$~byte in the fixed region, and either $71$~bytes after the fixed region (for the longest possible \tlsfunction{Derive-Secret} hash) or $|\G|/8$~bytes after (for an inner $\HMAC$ hash). But summing their minimum lengths gives $49+|\G|/8$ bytes. Even for the smallest possible $|\G|/8 = 32$, the extensions just do not fit in the given space.
It is therefore impossible to construct a valid $\ClientHello$ message, truncated or otherwise, that collides with a possible $\HMAC$, \tlsfunction{Derive-Secret}, or \tlsfunction{Finished} hash. 

Consequently we can set $\Dom_{\Thash}$ to contain the empty string and all strings of at least $86$ bytes for which at least one of bytes $48$ through $128$ does not equal either $\ipad$ or $\opad$. As usual, we set $\Dom_{\Chash}$ to be all other elements of $\bits^*$. 


\subsection{PSK-only mode with \SHA{384}}

In this mode/hash function combination, the transcript hash can collide with outer $\HMAC$ hashes.
There are other collisions as well, but one is sufficient to demonstrate the lack of domain separation.
We illustrate this via a $176$-byte transcript hash (containing a truncated $\ClientHello$) and an outer $\HMAC$ hash, shown in Figure~\ref{fig:domsep-PSKonly-384:transcript-vs-HMAC}.

% A $176$-byte transcript hash (containing a truncated $\ClientHello$):
% 
% \begin{tabular}{|p{3.2cm}|p{6.42cm}|p{1.81cm}|p{1.69cm}|}
% 	\hline
% 	Fixed preface ($43$~bytes) & \tlsfield{supported\_version} ($87$~bytes) & \tlsfield{cookie} ($24$~bytes)&Mandatory extensions ($22$~bytes) \\
% 	\hline
% \end{tabular}
% 
% An outer $\HMAC$ hash: 
% 
% \begin{tabular}{|p{3.6cm}|p{6.0cm}|p{3.6cm}|}
% 	\hline
% 	Key: $48$~bytes & Fixed region ($\opad$): $80$~bytes & Arbitrary string: $48$~bytes \\
% 	\hline
% \end{tabular}


\begin{figure}[h]
	\centering
	
	\scalebox{0.68}{%
% 	\begin{tabular}{|p{3.21cm}|p{9.3cm}|p{0.7cm}|}
% 		\hline
% 		Fixed preface: $43$\;B & Extension data: $124$\;B & End \tlsfield{PSK}: $9$\;B \\
% 		\hline
% 	\end{tabular}

	\begin{tabular}{|p{4.3cm}|p{8.7cm}|p{2.4cm}|p{1.755cm}|} %% last column should be 2.2cm, but this tabular has one column more, so adjust for spacing here
	\hline
		Fixed preface: $43$\;B & \tlsfield{supported\_version}: $87$\;B & \tlsfield{cookie}: $24$\;B & Mandatory extensions: $22$\;B \\
		\hline
	\end{tabular}
	}

	\noindent
	\scalebox{0.68}{%
% 	\begin{tabular}{|p{3.6cm}|p{6cm}|p{3.6cm}|}
% 		\hline
% 		Key: $48$\;B & Fixed region ($\opad$): $80$\;B & Arbitrary string: $48$\;B \\
% 		\hline
% 	\end{tabular}
	\begin{tabular}{|p{4.8cm}|p{8.0cm}|p{4.8cm}|}
		\hline
		Key: $48$~bytes & Fixed region ($\opad$): $80$~bytes & Arbitrary string: $48$~bytes \\
		\hline
	\end{tabular}
	}
	
	\caption{%
		Failing domain separation in PSK-only mode with \SHA{384}:
		Transcript hash of $176$~bytes, containing a truncated $\ClientHello$, (top)
		vs.\
		outer $\HMAC$ hash (bottom).
		``End \tlsfield{PSK}'' is the end of the \tlsfield{pre\_shared\_key} extension.
	}
	\label{fig:domsep-PSKonly-384:transcript-vs-HMAC}
\end{figure}



We construct the following message, which is both a truncated $\ClientHello$ (and therefore a transcript hash) and an outer $\HMAC$ hash. 
We let the first extension be the \tlsfield{supported\_versions} extension. 
This extension will contain $41$ protocol versions. 
The first $40$ versions will be two bytes of $\opad$: \texttt{0x5c5c}; the last will be the real version number \texttt{0x0304}. 
These extra version numbers match the $\HMAC$ key padding, and the real version number lies in the last $48$~bytes, which are unrestricted by the formatting of an $\HMAC$ hash. 

We place the remaining mandatory extensions at the end of the content section. 
In PSK-only mode, these are only  \tlsfield{psk\_key\_exchange\_modes}, and (the truncated) \tlsfield{pre\_shared\_key}, and they take up $22$~ bytes. 
This leaves $24$ bytes unaccounted for between the end of the \tlsfield{supported\_versions} extension and the start of \tlsfield{psk\_key\_exchange\_modes}. 
We can fill these with a \tlsfield{cookie} ~\cite[Section 4.2.2]{rfc8446} extension with arbitrary content. 
(We can also fill these bytes without including additional extensions.)

This type of collision is unavoidable, so there are no disjoint sets $\Dom_{\Thash}$ and $\Dom_{\Chash}$ that capture the way TLS~1.3 calls \SHA{384} in pre-shared key only mode. Consequently the indifferentiability step of Section~\ref{sec:domsep} does not apply to this mode. 

\iffull
\subsection{Repairing domain separation for TLS~1.3-like protocols}\label{apx-domsep-fix}
The above analysis demonstrates that complete domain separation is nontrivial to achieve for a protocol like TLS~1.3 which uses a hash function for multiple purposes and at multiple levels of abstraction.
We would like to present our suggestions for how this could be achieved most simply and efficiently in future iterations of TLS and other schemes. 
As discussed by Bellare et al.~\cite{EC:BelDavGun20}, the most well-known method of domain separation is the inclusion of distinct labels into each hash function call; this is precisely the method adopted by TLS~1.3 to distinguish calls to its \tlsfunction{Derive-Secret} function.
Ideally, a future scheme could specify a unique label string for each purpose: not only the various derived secrets, but also each time the transcript is hashed and each internal call made by $\HMAC$, $\HKDFExtr$, and $\HKDFExpnd$.

Unfortunately, this ideal method is not compatible with the existing specifications of $\HMAC$ and $\HKDF$. 
Both of these functions make ``Outer $\HMAC$ queries'' as discussed above; these calls have a fixed input length of $\blocklenBytes+ \hashlenBytes$ bytes and this input does not include a label.
A protocol could avoid this roadblock by using a custom implementation of $\HMAC$ or $\HKDF$ whose underlying hash function prepends an $\HMAC$-specific label to its input.
This approach would be both standard-compliant and efficient, but we don't recommend it because existing cryptographic libraries already have trustworthy $\HMAC$ and $\HKDF$ functionality and encouraging custom implementations for every new protocol increases the probability of accidental errors in these new implementations.
Instead, we suggest making no adjustments to the internal execution of $\HMAC$ or $\HKDF$ and instead altering direct hash function calls (the other six subtypes we discuss) to avoid collisions.

In practice, this means that under our recommendation, all hash function calls which are not outer $\HMAC$ queries should obey two simple rules: first, they should end with a unique label and second, that their input must not be $\blocklenBytes+\hashlenBytes$ long.
To conform with the first rule, TLS 1.3 would need to make the following changes.
\begin{enumerate}
	\item Add distinct labels to the end of each transcript before hashing; for clarity we suggest using the names of the last message in the transcript; i.e. ``$\PartialClientHello$'', ``$\ClientHello$'', ``$\ServerHello$'', etc. If $\HKDF$ is used, we would also recommend that these labels should not end with the byte \texttt{0x01}.
	\item Add distinct labels to the end of the input each time $\HMAC$ is called; this would include for Inner $\HMAC$ queries, Diffie--Hellman $\HMAC$ queries, $\Finished$ queries, and $\tlsfunction{Derive-Secret}$ queries. Note that the labels should be postpended to the $\HMAC$ payload and not the key. The labels used by $\tlsfunction{Derive-Secret}$ could then be omitted, although this is not necessary.
	\item Ensure that none of the labels used is a suffix of another; this can introduce collisions even if the labels are distinct.
\end{enumerate}
We encourage using suffixes for domain separation, although prefixes are more commonly-used, because they are easier to use in conjunction with $\HMAC$ and $\HKDF$.
Although we are not applying labels to outer $\HMAC$ queries, we would still like to use them to domain separate inner $\HMAC$ queries (and the other subtypes).
The inputs to these queries begin with the $\HMAC$ key, which undergoes an XOR operation with $\ipad$ before it is hashed.
So prefixed labels would need to remain unique and prefix-free after this XOR operation; this introduces some confusion that we prefer to avoid.
Additionally, the second step of our indifferentiability proof relies crucially on the fact that $\HMAC$ uses fixed-length keys shorter than $\blocklenBytes$; prefixed labels would therefore need to share a fixed length shorter than $\blocklenBytes-\hashlenBytes$ bytes. 
With suffixes, we still need to contend with the counter byte that $\HKDFExpnd$ appends to its input, but in TLS~1.3 where this byte is always \texttt{0x01}, this presents less of a restriction.

To conform with the second rule, TLS~1.3 would need to enforce that it never hashes a string of $\blocklenBytes+\hashlenBytes$ except as an Outer $\HMAC$ query. 
The easiest and least error-prone way to do this would be to pad every non-empty hash function call and input to $\HMAC$ and $\HKDF$ with exactly $\blocklenBytes + \hashlenBytes$ bytes (before the suffixed labels); all calls would strictly longer than $\blocklenBytes+\hashlenBytes$.
This method adds two additional compression function calls to each hash function execution.
There are some ways to lessen this requirement without impacting the effectiveness of the length-based domain separation.
Calls which already have input longer than $\blocklenBytes+\hashlenBytes$ bytes can omit the padding; so can calls which have strictly shorter input.
It would also be possible to use only as much padding is needed to make input at least $\blocklenBytes+\hashlenBytes+1$ bytes long.
However, non-uniform padding should be done carefully so that, for example, two previously distinct $\ClientHello$ messages do not collide after being padded.
\fi
\mathversion{normal}

\chapter{Derive-then-Derandomize: Stronger Security Proofs for EdDSA Signatures}\label{chap:eddsa}
\mathversion{normal4}
\section{Introduction}\label{sec:intro}

In designing schemes, and proving them secure, theoreticians implicitly assume certain things, such as on-demand fresh randomness and correct implementation. In practice, these assumptions can fail. Weaknesses in system random-number generators are common and have catastrophic consequences. (An example relevant to this paper is the well-known key-recovery attack on $\Schnorr$ signatures when signing reuses randomness. Another striking example are Ps and Qs attacks~\cite{USENIX:HDWH12,EPRINT:LHABKW12}.) Meanwhile, implementation errors can be exploited, as shown by Bleichenbacher's attack on RSA signatures~\cite{Bl-rump-C06}. 

In light of this, practitioners may try to ``harden'' theoretical schemes before standardization and usage. A prominent and highly successful instance is $\EdDSA$, a hardening of the $\Schnorr$ signature scheme proposed by Bernstein, Duif, Lange, Schwabe, and Yang (BDLSY)~\cite{bernstein2012high}. It incorporates explicit, simple key-derivation, makes signing deterministic, adds protection against sidechannel attacks via ``clamping,'' and for simplicity confines itself to a single hash function, namely $\SHAfive$. The scheme is widely standardized~\cite{NIST:EdDSA,RFC:EDDSA} and used~\cite{Ed-uses}.

There is however a subtle danger here, namely that the hardening attempt introduces new vulnerabilities. In other words, hardening needs to be done right; if not, it may even ``soften'' the scheme! Thus it is crucial that the hardened scheme be vetted via a proof of security. This is of particular importance for $\EdDSA$ given its widespread deployment. In that regard, Brendel, Cremers, Jackson and Zhao (BCJZ)~\cite{SP:BCJZ21} showed that $\EdDSA$ is secure if the Discrete-Log (DL) problem is hard and the hash function is modeled as a random oracle. This is significant as a first step but has at least two important limitations: (1) Due to the extension attack, a random oracle is not an appropriate model for the $\SHAfive$ hash function $\EdDSA$ actually uses, and (2) the reduction is so loose that there is no security guarantee for group sizes in use today. 

Extrapolating $\EdDSA$, the first part of this paper defines a general hardening transform on signature schemes called Derive-then-Derandomize ($\DRTransform$), and proves its soundness. Next we prove the indifferentiability of a general class of constructions, that we call shrink-MD; %  they apply a ``shrinking'' output transform to the result of an MD-style hash function. 
it includes the well-studied chop-MD construction~\cite{C:CDMP05} and also the modulo-a-prime construction arising in $\EdDSA$. 
% Furthermore our proof closes a gap in earlier analyses of chop-MD~\cite{hfrobook}. 
Armed with these results, the second part of the paper returns to give new proofs for $\EdDSA$ that in particular fill the above gaps.
We begin with some background.

% As part of a broader treatment that delivers more general results, our work will fill these gaps. 


 
\heading{Respecting hash structure in proofs.} Recall that the MD-transform~\cite{C:Merkle89a,C:Damgaard89b} defines a hash function $\HH = \construct{MD}[\compF] \Colon\bits^*\to\bits^{2k}$ by iterating an underlying compression function $\compF\Colon\bits^{b+2k}\to\bits^{2k}$. (See Section~\ref{sec-prelims} for details.) $\SHAtwo$ and $\SHAfive$ are obtained in this way, with $(b, k)$ being $(512, 128)$ and $(1024, 256)$, respectively. This structure gives rise to attacks, of which the most well known is the extension attack. The latter allows an attacker given $t \gets \MD[\compF](e_2\|M)$, where $e_2$ is a secret unknown to the attacker and $M\in\bits^*$ is public, to compute compute $t'=\MD[\compF](e_2\|M')$, for some $M'\in\bits^{*}$ of its choice. This has been exploited to violate the UF-security of the so-called prefix message authentication code $\pfMAC_{e_2}(M) = \HH(e_2\|M)$ when $\HH$ is an MD-hash function; $\HMAC$~\cite{C:BelCanKra96} was designed to overcome this.

A proof of security of a scheme (such as $\EdDSA$) that uses a hash function $\HH$ will often model $\HH$ as a random oracle~\cite{CCS:BelRog93}, in what we'll call the $(\HH,\HH)$-model: scheme algorithms, and the adversary, both have oracle access to the same random $\HH$.  However the presence of the above-discussed structure in ``real'' hash functions led Dodis, Ristenpart and Shrimpton (DRS) \cite{EC:DodRisShr09} to argue that the ``right'' model in which to prove security of a scheme that uses $\HH = \construct{MD}[\compF]$ is to model the compression function $\compF$ ---rather than the hash function $\HH=\construct{MD}[\compF]$--- as a random oracle. We'll call this the $(\construct{MD}[\compF],\compF)$-model: the adversary has oracle access to a random $\compF$, with scheme algorithms having access to $\construct{MD}[\compF]$. There is now widespread agreement with the DRS thesis that proofs of security of MD-hash-using schemes should use the $(\construct{MD}[\compF],\compF)$ model.

Giving from-scratch proofs in the $(\construct{MD}[\compF],\compF)$ model is, however, difficult. Maurer, Renner and Holenstein (MRH)~\cite{TCC:MauRenHol04} show that if a construction $\construct{F}$ is indifferentiable (abbreviated indiff) and a scheme is secure in the $(\HH,\HH)$ model, then it remains secure in the $(\construct{F}[\compF],\compF)$ model. (This requires the game defining security of the scheme to be single-stage~\cite{EC:RisShaShr11}, which is true for the relevant ones here.)
Unfortunately, $\construct{F}=\MD$ is provably \textit{not} indiff~\cite{C:CDMP05}, due exactly to the extension attack. So the MRH result does not help with $\MD$. This led to a search for indiff variants. DRS~\cite{EC:DodRisShr09} and YMO~\cite{yoneyama2009leaky} (independently) offer public-indiff and show that it suffices to prove security, in the $(\construct{MD}[\compF],\compF)$ model, of schemes that use $\construct{MD}$ in some restricted way. However, $\EdDSA$ does not obey these restrictions. Thus, other means are needed.




%\begin{figure}[t]
%\twoCols{0.35}{0.55}
%{
%  \begin{algorithm-initial}{$\DS_{\Schnorr}.\Kg$}
%  \item $\s \getsr \Z_{\Prime}$ ; $\curvepoint{A} \gets \s\cdot \generator$
%  \item Return $(\curvepoint{A},\s)$
%  \end{algorithm-initial}
%  \ExptSepSpace
%\begin{algorithm-subsequent}{$\DS_{\Schnorr}.\Sign[\HH](\s, \curvepoint{A}, \msg)$}
%\item $r \getsr \Z_{\Prime}$ ; $\curvepoint{R} \gets \littler\cdot \generator$
%\item $c \gets \HH(\curvepoint{R}\|\curvepoint{A}\| \msg)$
%\item $\z \gets (\s c + \littler) \bmod \Prime$
%\item Return $(\curvepoint{R}, \z)$
%  \end{algorithm-subsequent}
%  \ExptSepSpace
%  \begin{algorithm-subsequent}{$\DS_{\Schnorr}.\Vf[\HH](\curvepoint{A}, \msg, \sigma)$}
%  \item $(\curvepoint{R}, \z) \gets \sigma$
%  ;  $c \gets \HH(\curvepoint{R}\|\curvepoint{A}\| \msg)$
%  \item Return ($\z \cdot\generator = \curvepoint{R}+ c\cdot\curvepoint{A} $)
%  \end{algorithm-subsequent}  \vspace{2pt}
% }
% {
%	\begin{algorithm-subsequent}{$\DS_{\EdDSA}.\Kg[\HH]$}
%		\item $\sk \getsr \bits^{k}$ 
%		;  $\e \gets \HH(\sk)$ 
%		% \item $\sk\gets  \HH(\emptystring)$ 
%		; $\e_1 \gets \e[1..k]$ 
%		\item $\s \gets \Clamp(\e_1)$ ; $\curvepoint{A} \gets \s\cdot \generator$
%		\item Return $(\curvepoint{A}, \sk)$
%	\end{algorithm-subsequent}\ExptSepSpace
%	\begin{algorithm-subsequent}{$\DS_{\EdDSA}.\Sign[\HH](\sk,\curvepoint{A}, \msg)$}
%		\item $\e \gets \HH(\sk)$ 
%		; $\e_1 \gets \e[1..k]$ ; $\e_2 \gets \e[k\! +\! 1..2k]$ 
%		\item $\s \gets \Clamp(\e_1)$ %; $\curvepoint{A} \gets \s\cdot \generator$
%		\item $\littler \gets \HH(\e_2\|\msg) \bmod \Prime$ 
%		; $\curvepoint{R} \gets \littler \cdot \generator$
%		\item $c \gets \HH(\curvepoint{R}\|\curvepoint{A}\|\msg) \bmod \Prime$
%		; $\z \gets (\s c + \littler) \bmod \Prime$
%		\item Return $(\curvepoint{R},\z)$
%	\end{algorithm-subsequent}\ExptSepSpace
%	\begin{algorithm-subsequent}{$\DS_{\EdDSA}.\Vf[\HH](\curvepoint{A}, \msg, \sigma)$}
%		\item $(\curvepoint{R}, \z) \gets \sigma$
%		; $c \gets \HH(\curvepoint{R}\|\curvepoint{A}\|\msg) \bmod \Prime$
%		\item Return ($\z\cdot \generator = \curvepoint{R}+ c\cdot\curvepoint{A} $)
%	\end{algorithm-subsequent}  \vspace{2pt}
%	}
%\vspace{-8pt}
%\caption{On the left is the $\Schnorr$ signature scheme. On the right is $\EdDSA$. Here $\generator$ is a generator of an additively-written group $\group$ that has prime order $\Prime$. $\HH$ is a hash function. $\Clamp$ is the $\EdDSA$ clamping function.}
%\label{fig-eddsa-intro}
%\hrulefill
%\vspace{-10pt}
%\end{figure}

 \heading{The $\EdDSA$ scheme.} The Edwards curve Digital Signature Algorithm ($\EdDSA$) is a Schnorr-based signature scheme introduced by Bernstein, Duif, Lange, Schwabe and Yang~\cite{bernstein2012high}. $\Edtwo$, which uses the Curve25519 Edwards curve and $\SHAfive$ as the hash function, is its most popular instance. The scheme is standardized by NIST~\cite{NIST:EdDSA} and the IETF~\cite{RFC:EDDSA}. It is used in TLS 1.3, OpenSSH, OpenSSL, Tor, GnuPGP, Signal and WhatsApp. It is also the preferred signature scheme of the Corda, Tezos, Stellar and Libra blockchain systems. Overall, IANIX~\cite{Ed-uses} reports over 200 uses of $\Edtwo$. Proving security of this scheme is accordingly of high importance.
 
  Figure~\ref{fig-eddsa} shows $\EdDSA$ on the right, and, on the left, the classic $\Schnorr$ scheme~\cite{JC:Schnorr91} on which $\EdDSA$ is based. % (Note there are several $\Schnorr$ variants, that differ in details. See Section~\ref{sec-schemes} for a discussion of how the one used here relates to others.)
  The schemes are over a cyclic, additively-written group $\group$ of prime order $\Prime$ with generator $\generatorEDSA$. The public verification key is $\curvepoint{A}$. The $\Schnorr$ hash function has range $\Z_{\Prime} = \{0,\ldots,\Prime-1\}$, while, for $\EdDSA$, function $\HH_1$ has range $\bits^{2k}$ where $k$, the bit-length of $\Prime$, is $256$ for $\Edtwo$. Functions $\HH_2,\HH_3$ have range $\Z_{\Prime}$.

$\EdDSA$ differs from $\Schnorr$ in significant ways. While the $\Schnorr$ secret key $\s$ is in $\Z_{\Prime}$, the $\EdDSA$ secret key $\sk$ is a $k$-bit string. This is hashed and the $2k$-bit result is split into $k$-bit halves $e_1\|e_2$. A Schnorr secret-key $\s$ is derived by applying to $e_1$ a clamping function $\Clamp$ that zeroes out the three least significant bits of $e_1$. (Note: This means $\s$ is \textit{not} uniformly distributed over $\Z_{\Prime}$.) Clamping increases resistance to side-channel attacks~\cite{bernstein2012high}. Signing is made deterministic by a standard de-randomization technique~\cite{C:Goldreich86a,SAC:MNPV98,AC:BelPoeSte16,EC:BelTac16}, namely obtaining the Schnorr randomness $\littler$ by hashing the message $\msg$ with a secret-key dependent string $e_2$. We note that all of $\HH_1,\HH_2,\HH_3$ are instantiated via the same hash function, namely $\SHAfive$.

 % function $\HH$ is used in three ways for three purposes: to derive keys (lines~1,12), to de-randomize signing (line~14) and, as in $\Schnorr$, for the Fiat-Shamir hashing (line~15)~\cite{C:FiaSha86}. 


% We write the scheme as using three hash functions $\HH_1,\HH_2,\HH_3$, but ---this is important for our proposed work--- in $\Edtwo$ they are all $\SHAfive$. 

\heading{Prior work and our questions.} Recall that the security goal for a signature scheme is UF (UnForgeability under Chosen-Message Attack)~\cite{GolMicRiv88}. $\Schnorr$ is well studied, and proven UF under $\DLP$ (Discrete Log in $\group$) when $\HH$ is a random oracle~\cite{JC:PoiSte00,EC:AABN02}. The provable security of $\EdDSA$, however, received surprisingly little attention until the work of Brendel, Cremers, Jackson and Zhao (BCJZ)~\cite{SP:BCJZ21}. They take the path also used for $\Schnorr$ and other identification-based signature schemes~\cite{JC:PoiSte00,EC:AABN02}, seeing $\EdDSA$ as the result of the Fiat-Shamir transform on an underlying identification scheme $\EdID$ that they define, proving security of the latter under $\DLP$, and concluding UF of $\EdDSA$ under $\DLP$ when $\HH$ is a random oracle. This is an important step forward, but the BCJZ proof~\cite{SP:BCJZ21} remains in the $(\HH,\HH)$ model. We ask and address the following two questions.


\medskip
\textbf{1. Can we prove security in the $(\MD[\compF],\compF)$ model?} The NIST standard~\cite{NIST:EdDSA} mandates that $\Edtwo$ uses $\SHAfive$, which is an MD-hash function. Accordingly, as explained above, the BCJZ proof~\cite{SP:BCJZ21}, being in the $(\HH,\HH)$ model, does not guarantee security; to do the latter, we need a proof in the $(\MD[\compF],\compF)$ model.

The gap is more than cosmetic. As we saw above with the example of the prefix MAC, a scheme could be secure in the $(\HH,\HH)$ model, yet totally insecure in the more realistic $(\MD[\compF],\compF)$ model, and thus also in practice. And $\EdDSA$ skirts close to the edge: line~14 is using the prefix-MAC that the extension attack breaks, and overlaps in inputs across the three uses of $\HH$ could lead to failures. Intuitively what prevents attacks is that the MAC outputs are taken modulo $\Prime$, and inputs to $\HH$ in two of the three uses involve secrets. Thus, we'd expect that the scheme is indeed secure in the $(\MD[\compF],\compF)$ model. 

Proving this, however, is another matter. We already know that $\construct{MD}$ is not indiff. It is public indiff~\cite{EC:DodRisShr09,yoneyama2009leaky}, but this will not suffice for $\EdDSA$ because $\HH_1,\HH_2$ are being called on secrets. We ask, first, can $\EdDSA$ be proved secure in the $(\MD[\compF],\compF)$ model, and second, can this be done in some modular way, rather than from scratch?

\medskip
\textbf{2. Can we improve reduction tightness?} The reduction of BCJZ~\cite{SP:BCJZ21} is so loose that, in the 256-bit curve over which $\Edtwo$ is implemented, it guarantees little security. Let's elaborate. Given an adversary $\advAUF$ violating the UF-security of $\EdDSA$ with probability $\epsUF$, the reduction builds an adversary $\advADL$ breaking $\DLP$ with probability \smash{$\epsDL = \epsUF^2/q_h$} where $q_h$ is the number of $\HH$-queries of $\advAUF$ and the two adversaries have about the same running time $t$. (The square arises from the use of rewinding, analyzed via the Reset Lemma of~\cite{C:BelPal02}.) In an order $\Prime$ elliptic curve group, \smash{$\epsDL \approx t^2/p$} so we get \smash{$\epsUF = t\cdot \sqrt{q_h/p}$}. $\Edtwo$ has \smash{$p\approx 2^{256}$}. Say $t=q_h=2^{70}$, which (as shown by BitCoin mining capability) is not far from attacker reach. Then $\epsDL = 2^{-116}$ is small but \smash{$\epsUF = 2^{70}\cdot 2^{-(256-70)/2} = 2^{-23}$} is in comparison quite high. 

Now, one might say that one would not expect better because the same reduction loss is present for $\Schnorr$. The classical reductions for $\Schnorr$ \cite{JC:PoiSte00,EC:AABN02} did indeed display the above loss, but that has changed: recent advances for $\Schnorr$ include a tighter reduction from $\DLP$~\cite{C:RotSeg21}, an almost-tight reduction from the MBDL problem~\cite{INDOCRYPT:BelDai20} and a tight reduction from $\DLP$ in the Algebraic Group Model~\cite{EC:FucPloSeu20}. We'd like to put $\EdDSA$ on par with the state of the art for $\Schnorr$. We ask, first, is this possible, and second, is there a modular way to do it that leverages, rather than repeats, the (many, complex) just-cited proofs for $\Schnorr$? 

% \authnote{Reviwer}{Goal 1 is very clear.  Goal 2 not as much.  Setting $q_h = 2^{80}$ seems extreme.  }{red}

%\medskip
%\textbf{3. Weak multi-user security.} UF-security, the target in all the above, pertains to the classical setting where there is just one user (key) under attack. There is now broad acceptance that the more realistic setting is the multi-user one~\cite{EC:BelBolMic00,galbraith2002public,menezes2004security}, where there are $u\geq 1$ users, with independent keys. (This in particular is the setting in TLS.) Denoting security here by MUF, if $\epsMUF,\epsUF$ are the MUF and UF advantages (success probabilities), respectively, a standard hybrid argument shows that $\epsMUF \leq u\cdot\epsUF$~\cite{galbraith2002public}. But with just $u = 300$ users (TLS has way more), and $\epsUF=2^{-8}$ as above, this fails to even guarantee $\epsMUF<1$. 

%Now, for $\Schnorr$, one can do better than the hybrid argument; it is known that $\epsMUF\approx \epsUF$~\cite{EPRINT:Bernstein15,C:KilMasPan16}. Can we show the same for $\EdDSA$?


\heading{Contributions for $\EdDSA$.} We simultaneously simplify and strengthen the security proofs for $\EdDSA$ as follows.  
\smallskip

\textbf{1. Reduction from $\Schnorr$.} Rather than, as in prior work, give a reduction from $\DLP$ or some other algebraic problem, we give a simple, direct reduction from $\Schnorr$ itself. That is, we show that if the $\Schnorr$ signature scheme is UF-secure, then so is $\EdDSA$. Furthermore, the reduction is \textit{tight} up to a constant factor. This allows us to leverage prior work~\cite{C:RotSeg21,INDOCRYPT:BelDai20,EC:FucPloSeu20} to obtain tight proofs for $\EdDSA$ under various algebraic assumptions and justify security for group sizes in actual use. But there are two further dividends. First, $\Schnorr$~\cite{JC:Schnorr91} is over 30 years old and has withstood the tests of time and cryptanalysis, so our proof that $\EdDSA$ is just as secure as $\Schnorr$ allows the former to inherit, and benefit from, this confidence. Second, our result formalizes and proves what was the intuition and belief in the first place~\cite{bernstein2012high}, namely that, despite the algorithmic differences, $\EdDSA$ is a sound hardening of $\Schnorr$.
\smallskip

\textbf{2. Accurate modeling of the hash function.} As noted above, BCJZ~\cite{SP:BCJZ21} assume the hash function $\HH$ is a random oracle, but
% , as Coron, Dodis, Malinaud and Puniya (CDMP)~\cite{C:CDMP05} explained, 
this, due to the extension attack, is not an accurate model for the MD-hash function $\SHAfive$ used by $\EdDSA$. We fill this gap by instead proving security in the $(\MD[\compF],\compF)$ model, where $\HH=\MD[\hh]$ is derived via the MD-transform~\cite{C:Merkle89a,C:Damgaard89b} and the compression function $\hh$ is a random oracle.
% We explain why this is \textit{not} obtained directly by combining the BCJZ result with indifferentiability or public indifferentiability. 
  

\begin{sloppypar}
\heading{Approach and broader contributions.} The above-mentioned results on $\EdDSA$ are obtained as a consequence of more general ones.
\end{sloppypar}
\smallskip
\textbf{3. The $\DRTransform$ transform and its soundness.}
We extend the hardening technique used in $\EdDSA$ to define a general transform that we call Derive-then-Derandomize ($\DRTransform$). It takes an \textit{arbitrary} signature scheme $\DS$, and with the aid of a PRG $\HH_1$ and a PRF $\HH_2$, constructs a hardened signature scheme $\fDS$. We provide (Theorem~\ref{th-dd}) a strong and general validation of $\DRTransform$, showing that $\fDS$ is UF-secure assuming $\DS$ is UF-secure. Moreover \textit{the reduction is tight} and the proof is simple. This shows that the $\EdDSA$ hardening method is generically sound.

\smallskip
\textbf{4. Indifferentiability of Shrink-MD.} It is well-known that $\MD$ is not indifferentiable~\cite{TCC:MauRenHol04} from a random oracle, but that
the $\ChopMD$~\cite{C:CDMP05}, which truncates the output of an an $\MD$ hash by some number of bits, is indifferentiable.
Unfortunately, we identified gaps in two prominent proofs of indifferentiability of $\ChopMD$~\cite{C:CDMP05,hfrobook}.
$\EdDSA$ uses a similar construction that reduces the $\MD$ hash output modulo a prime $\Prime$ sufficiently smaller than the size of the range of $\MD$, due to which we refer to this construction as $\ModMD$.
The $\ModMD$ construction has not been proven indifferentiable.
We simultaneously give new proofs of indifferentiability for $\ChopMD$ and $\ModMD$ as part of a more general class of constructions that we call $\ShrinkMD$ functors.
These are constructions of the form $\Out(\MD)$ where $\Out$ is some output-processing function, and we prove indifferentiability under certain ``shrinking'' conditions on $\Out$.
  

\smallskip
\textbf{5. Application to $\EdDSA$.} $\EdDSA$ is obtained as the result $\fDS$ of the $\DRTransform$ transform applied to the $\DS=\Schnorr$ signature scheme, and with the PRG and PRF defined via $\MD$, specifically $\HH_1(\sk) = \MD[\hh](\sk)$ and $\HH_2(e_2,M) = \MD[\hh](e_2\|M)\bmod\Prime$ where $\Prime$ is the prime order of the underlying group. Additionally, the hash function used in $\Schnorr$ is also $\HH_3(X) = \MD[\hh](X)\bmod\Prime$. Due to Theorem~\ref{th-dd} validating $\DRTransform$, we are left to show the PRG security of $\HH_1$, the PRF security of $\HH_2$ and the UF-security of $\Schnorr$, all with $\hh$ modeled as a random oracle. We do the first directly. We obtain the second as a consequence of the indifferentiability of $\ModMD$. (In principle it follows from the PRF security of AMAC~\cite{EC:BelBerTes16}, but we found it difficult to extract precise bounds via this route.) For the third, we again exploit indifferentiability of $\ModMD$, together with a technique from BCJZ~\cite{SP:BCJZ21} to handle clamping, to reduce to the UF security of regular $\Schnorr$, where the hash function is modeled as a random oracle. Putting all this carefully together yields our above-mentioned results for $\EdDSA$. We note that one delicate and important point is that the idealized compression function $\hh$ is \textit{the same} across $\HH_1,\HH_2$ and $\HH_3$, meaning these are not independent. This is handled through the building blocks in Theorem~\ref{th-dd} being functors~\cite{EC:BelDavGun20} rather than functions.




%\heading{Our answers, in brief.} We give affirmative answers to both questions above. The first element of our approach is that our reduction for $\EdDSA$, rather than being from $\DLP$ or some other algebraic problem, is directly from $\Schnorr$ itself, and is \textit{tight} up to a constant factor. This in one step answers the second question (tightness) discussed above because we immediately inherit, for $\EdDSA$, the guarantees of the known tight(er) proofs of $\Schnorr$~\cite{C:RotSeg21,INDOCRYPT:BelDai20,EC:FucPloSeu20}. Our proof is, moreover, in the $(\MD[\compF],\compF)$-model. To obtain it in a modular way, we introduce filtered indifferentiability (f-indiff), show that f-indiff of $\MD$ suffices to prove security of $\EdDSA$, and separately establish f-indiff of $\MD$ (despite its lack of indiff) using a combination of new techniques and results from the indifferentiability literature~\cite{EC:DodRisShr09,C:CDMP05}. This answers the first question above. We now look at all this in more detail.
%
%\heading{Reduction from $\Schnorr$.} Let's write $\probP\reducesTo\DS$ to mean that we prove UF security of signature scheme $\DS$ with a reduction from (i.e., assuming hardness of) problem $\probP$ in the ROM. (The notation says nothing about tightness, which will be discussed separately.) Thus, BCJZ~\cite{SP:BCJZ21} show that $\DLP\reducesTo\EdDSA$. We show instead that $\Schnorr\reducesTo\EdDSA$. That is, if $\Schnorr$ is secure, so is $\EdDSA$. Furthermore, while the BCJZ reduction is loose, ours is \textit{tight} up to a small constant factor. 
%
%The immediate dividend is that any (known, or even future) proof $\probP\reducesTo\Schnorr$ automatically yields, via our result, a proof $\probP\reducesTo\EdDSA$, with only a constant factor loss in tightness compared to the original proof. In particular we get a tighter $\DLP\reducesTo\EdDSA$ proof via~\cite{C:RotSeg21}, an almost tight $\MBDLP\reducesTo\EdDSA$ proof via~\cite{INDOCRYPT:BelDai20} and an up-to-constant tight $\DLP\reducesTo\EdDSA$ Algebraic Group Model proof via~\cite{EC:FucPloSeu20}. This answers the second question (tightness) above.
%
%%\authnote{Reviwer}{Not the biggest fan of this $X \reducesTo Y$ notation, which conceals both the model (e.g., ROM) and the tightness of the reduction, in a paper that focuses on both the model and the tightness of the reduction.   I'd really like this groundbreaking paper to be more precise/specific documenting the implications in that "Reduction from Schnorr" section.  }{red}
%
%But there are two further dividends. First, $\Schnorr$~\cite{JC:Schnorr91} is over 30 years old and has withstood the tests of time and cryptanalysis. Our proof that $\EdDSA$ is just as secure as $\Schnorr$ allows the former to inherit, and benefit from, this confidence. Second, our result formalizes and proves what was the intuition and belief for $\EdDSA$ in the first place, namely that, despite the algorithmic differences, it is ``the same'' as $\Schnorr$ in security.
%
%Our proof that $\Schnorr\reducesTo\EdDSA$ would be novel and interesting already even in the basic $(\HH,\HH)$ model. However, we actually prove this in the $(\MD[\compF],\compF)$ model, so that we also answer the first question above. We now turn to this.
%
%


%\heading{Outline of proof.} Our main result (Theorem~\ref{th-eddsa-md}) is a tight (up to a constant factor) $\Schnorr\reducesTo\EdDSA$ reduction in the $(\MD[\compF],\compF)$ model. (The constant is 16 when $k=256$.)  That is, we show UF-security of $\EdDSA$, assuming only UF-security of $\Schnorr$, even when the hash function used is an MD-style one like $\SHAfive$. The statement of this result does not involve f-indiff or filters; these arise only in the proof. The latter has a few steps that we now outline. A fuller explanation is in Section~\ref{sec-schemes} and a picture is in Figure~\ref{pic-indiff}.

% \authnote{Reviewer}{The repeated claim of being tight "up to a constant factor" for Theorem 2 has me curious what that constant is.  Why not be explicit?}{red}

%Let $\DS$ denote the target $\EdDSA$ scheme, whose security we consider in the UF game. The first step is to cast $\DS$ as an alternative scheme $\fDS$ (shown on the right in Figure~\ref{fig-eddsa}) whose security we consider in a filtered unforgeability (fUF) game that we define via Figure~\ref{fig:fUF}. The idea is that the role of the secret signing key is now played by the filter seed. The filter here is our $\FltEDDSA$ one discussed above, and Lemma~\ref{lm-fUF-eq-UF} says that the schemes have equivalent security. We now need to show fUF security of $\fDS$. 
%
%The next step, Theorem~\ref{th-use-findiff}, is an f-indiff composition theorem, in the vein of the indiff composition theorem of~\cite{TCC:MauRenHol04}. This reduces the task to two sub-tasks. The first is to show fUF security of $\fDS$ when the oracle called by the $\FltEDDSA$ filter is a random oracle rather than $\MD[\compF]$. That is, the scheme should be secure in an ``ideal,'' even if still filtered, setting. The second sub-task is to show that $\MD$ is f-indiff relative to the $\FltEDDSA$ filter.
%
%The first sub-task is handled by Theorem~\ref{th-ideal-eddsa}, and this is where $\Schnorr$ enters, the reduction being from the latter. As an abstraction boundary, we use a version of $\Schnorr$ in which the signing key is drawn, not uniformly, but via a key-generation algorithm that performs the $\EdDSA$ clamping. A technique of BCJZ~\cite{SP:BCJZ21} separately allows $\Schnorr$ itself to reduce to clamped $\Schnorr$ with the above constant factor loss in advantage. 
% (The constant is 16 when $k=256$.)

%The final and most technical step is Theorem~\ref{th-md-indiff}, showing that $\MD$ is f-indiff relative to the $\FltEDDSA$ filter. This first exploits the presence of the seed to reduce the task to showing a weak form of public indiff for $\MD$. It concludes by exploiting the public indifferentiability of $\MD$ as shown in DRS~\cite{EC:DodRisShr09}. An alternative proof would first generalize and extend the indiff of truncated $\MD$ shown in~\cite{C:CDMP05} to $\MD$ taken modulo a prime whose bit-length is half that of the $\MD$ output, and then reduce to this.

%\authnote{Reviewer}{Figure~\ref{fig-ge1} appears on page 17, but is really needed to understand the discussion here.  Also this page makes lots of other references (e.g., Lemma~\ref{lm-fUF-eq-UF}) that assume the reader is already familiar with the later main body material.  Can you make this more self-contained with pointers forward?}{red}

 
\heading{Discussion and related work.} Both BCJZ~\cite{SP:BCJZ21} and CGN~\cite{10.1007/978-3-030-64357-7_4} note that there are a few versions of $\EdDSA$ out there, the differences being in their verification algorithms. What Figure~\ref{fig-eddsa} shows is the most basic version of the scheme, but we will be able to cover the variants too, in a modular way, by reducing from $\Schnorr$ with the same verification algorithm.

BBT~\cite{EC:BelBerTes16} define the function $\AMAC[\compF]$ to take a key $e_2$ and message $M$, and return $\MD[\compF](e_2\|M)\bmod\Prime$. This is the $\HH_2$ in $\EdDSA$. We could exploit their results to conclude PRF security of $\HH_2$, but it requires putting together many different pieces from their work, and it is easier and more direct to establish PRF security of $\HH_2$ by using our lemma on the indifferentiability of $\ModMD$.

In the Generic Group Model (GGM)~\cite{EC:Shoup97}, it is possible to prove UF-security of $\Schnorr$ under standard (rather than random oracle) model assumptions on the hash functions~\cite{neven2009hash,C:CLMQ21}. But use of the GGM means the result applies to a limited class of adversaries. Our results, following the classical proofs for identification-based signatures~\cite{JC:PoiSte00,C:OhtOka98,EC:AABN02,C:KilMasPan16}, instead use the standard model for the group, while modeling the hash function (in our case, the compression function) as a random oracle. 

In an earlier version of this paper, our proofs had relied on a variant of indifferentiability that we had introduced. At the suggestion of a Crypto 2022 reviewer, this has been dropped in favor of a direct proof based on PRG and PRF assumptions on $\HH_1,\HH_2$. We thank the (anonymous) reviewer for this suggestion.

Theorem~\ref{th-dd} is in the standard model if the PRG, PRF and starting signature scheme $\DS$ are standard-model, hence can be viewed as a standard-model justification of the hardening template underlying $\EdDSA$. However, when we want to justify $\EdDSA$ itself, we need to consider the specific, $\MD$-based instantiations of the PRG, PRF and $\Schnorr$ hash function, and for these we use the model where the compression function is ideal.

Several works study de-randomization of signing by deriving the coins via a PRF applied to the message, considering different ways to key the PRF~\cite{C:Goldreich86a,SAC:MNPV98,AC:BelPoeSte16,EC:BelTac16}. We use their techniques in the proof of Theorem~\ref{th-dd}.

One might ask how to view the UF-security of $\Schnorr$ signatures as an assumption. What is relevant is not its form (it is interactive) but that (1) it can be seen as a hub from where one can bridge to other assumptions that imply it, such as DL (non-tightly)~\cite{JC:PoiSte00,EC:AABN02} or MBDL (tightly)~\cite{INDOCRYPT:BelDai20}, and (2) it is validated by decades of cryptanalysis.

Our results have been stated for UF but extend to SUF (Strong unforgeability), meaning our proofs also show SUF-security of $\EdDSA$ in the $(\MD[\compF],\compF)$ model assuming SUF security of $\Schnorr$, with a tight (up to the usual constant factor) reduction.

$\EdDSA$ could be used with other hash functions such as $\SHAKE{256}$. The extension attack does not apply to the latter, so the proof of BCJZ~\cite{SP:BCJZ21} applies, but gives a loose reduction from DL; our results still add something, namely a tight reduction from $\Schnorr$ and thus improved tightness in several ways as discussed above.

% Given the way our work uses code-based games, it could benefit from being cast in the state-separation~\cite{AC:BDFKK18} or constructive cryptography~\cite{ICS:MauRen11,TCC:MauRen16} frameworks. We leave these as directions for future work.










\section{Preliminaries} \label{sec-prelims}

\noskipheadingu{Notation.} If $n$ is a positive integer, then $\Z_n$ denotes the set $\{0, \ldots, n-1\}$ and $[n]$ or $[1..n]$ denote the set $\{1,\ldots,n\}$. 
If $\vecxx$ is a vector then $|\vecxx|$ is its length (the number of its coordinates), $\vecxx[i]$ is its $i$-th coordinate 
and $[\vecxx] = \set{\vecxx[i]}{1\leq i\leq |\vecxx|}$ is the set of all its coordinates. 
A string is identified with a vector over $\bits$, so that if $x$ is a string then $x[i]$ is its $i$-th bit and $|x|$ is its length. We denote $x[i..j]$ the $i$-th bit to the $j$-th bit of string $x$.
By $\emptystring$ we denote the empty vector or string. The size of a set $S$ is denoted $|S|$. 
For sets $D,R$ let $\AllFuncs(D,R)$ denote the set of all functions $f\Colon D\to R$. If $f\Colon D\to R$ is a function then $\Img(f) = \set{f(x)}{x\in D}\subseteq R$ is its image. We say that $f$ is \textit{regular} if every $y\in\Img(f)$ has the same number of pre-images under $f$.
By $\bits^{\leq L}$ we denote the set of all strings of length at most $L$.
For any variables $a$ and $b$, the expression $[[a = b]]$ denotes the Boolean value $\true$ when $a$ and $b$ contain the same value and $\false$ otherwise.

Let $S$ be a finite set. We let $x \getsr S$ denote sampling an element uniformly at random from $S$ and assigning it to $x$. 
% We let $x \getsrr S$ be a short-hand for the operation sequence $x \getsr S$ ; $S
% \gets S \setminus \{x\}$. (Sampling without replacement.)  
We let $y \gets A[\Oracle_1, \ldots](x_1,\ldots ;
r)$ denote executing algorithm $A$ on inputs $x_1,\ldots$ and coins $r$ with
access to oracles $\Oracle_1, \ldots$ and letting $y$ be the result. We let $y
\getsr A[\Oracle_1, \ldots ](x_1,\ldots)$ be the resulting of picking $r$ at
random and letting $y \gets A[\Oracle_1, \ldots](x_1,\ldots; r)$ be the equivalent. We let
$\Outputs(A[\Oracle_1, \ldots ](x_1,\ldots)])$ denote the set of all possible outputs
of $A$ when invoked with inputs $x_1,\ldots$ and oracles $\Oracle_1, \ldots$.
% We use $q^{\Oracle_i}_{\advA}$ to denote the number of queries that $\advA$
% makes to oracle $\Oracle_i$ in the worst case.  
Algorithms are randomized unless
otherwise indicated. Running time is worst case.
%%% COMPLETED

%\mihirnote{For the whole paper, starting with the above, how about we change $A^{\Oracle_1, \ldots }$ to $A[\Oracle_1,\ldots]$? Given the complexity of some of our oracles, not making them superscripts may help. Also let's introduce better notation for what is $[A^{\Oracle_1, \ldots }(x_1,\ldots)]$ above, maybe $\mathrm{Outputs}(A[\Oracle_1, \ldots ](x_1,\ldots))$?}
%\hd{This is in progress.}

\headingu{Games.} We use the code-based game playing framework of
\cite{EC:BelRog06}. (See Fig. 1 for an example.) Games have procedures, also called oracles. 
% Each oracle is designated either ``public'' or ``private.''
 Among the oracles are $\Initialize$ and a $\Finalize$. In executing an adversary $\advA$ with a game $\Gm$, the adversary may query the oracles at will. We require that the adversary's first oracle query be to $\Initialize$ and its last to $\Finalize$ and it query these oracles at most once. The value return by the $\Finalize$ procedure is taken as the game output. By $\Gm(\advA) \Rightarrow y$ we denote the event that the execution of game $\Gm$ with adversary $\advA$ results in output $y$. We write $\Pr[\Gm(\advA)]$ as shorthand for $\Pr[\Gm(\advA)\Rightarrow \true]$, the probability that the game returns $\true$.

In writing game or adversary pseudocode, it is assumed that Boolean variables are initialized to $\false$, integer variables are initialized to $0$ and set-valued variables are initialized to the empty set $\emptyset$.

%\mihirnote{Above, also define the notation like $[[b=b']]$ that we use in games to indicate the boolean result of some test. Not sure how generally we use it or how best to define it.}
%\hd{Done.}
%\mihirnote{Where is it done? I don't see $[[b=b']]$ defined above.}
%\hd{Last sentence of the first paragraph under Notations. I used a and b instead of b and b', though.}
%\mihirnote{Change the conventions for games above as follows, which may reflect some prior papers. The adversary simply calls all oracles, beginning with $\Initialize$ and ending with $\Finalize$, so that the adversary has no inputs or outputs. Also say that some oracles can be designated private, meaning the adversary is not allowed to query them.}
%\hd{Done except for adding citations to prior work.}

We adopt the convention that the running time of an adversary is the time for the execution of the game with the adversary, so that the time for oracles to respond to queries is included. In counting the number of queries to an oracle $\Oracle$, we have two metrics. We let $\Queries{\Oracle}{\advA}$ denote the number of queries made to $\Oracle$ in the execution of the game with $\advA$. (This includes not just queries made directly by $\advA$ but also those made by game oracles, the latter usually arising from game executions of scheme algorithms that use $\Oracle$.) In particular, under this metric, the number of queries to a random oracle $\HASH$ includes those made by scheme algorithms executed by game procedures. 
% When adversary $\advA$ is executed with game $\Gm$, we consider two running times. The time of the execution, denoted $\Time{\Gm(\advA)}$, includes the time taken by game procedures, while the time of the adversary, denoted $\Time{\advA}$, assumes game procedures take unit time to respond. 
 With $\QueriesD{\Oracle}{\advA}$ we count only queries made directly by $\advA$ to $\Oracle$, not by other game oracles or scheme algorithms.
These counts are all worst case.

% We use $t_{\advA}$ to denote the running time of an adversary $\advA$.

\headingu{Groups.} Throughout the paper, we fix integers $k$ and $b$, an odd prime $\Prime$, and a positive integer $\cofactor$ such that $2^\cofactor < \Prime$. 
We then fix two groups: $\G$, a group of order $\Prime \cdot 2^\cofactor$ whose elements are $k$-bit strings, and its cyclic subgroup $\G_{\Prime}$ of order $\Prime$. 
We prove in \fullorAppendix{apx:group} that this subgroup is unique, and that it has an efficient membership test. 
We also assume an efficient membership test for $\G$. 
We will use additive notation for the group operation, and we let $0_{\G}$ denote the identity element of $\G$. 
We let $\G_{\Prime}^*=\G\setminus\{0_{\G}\}$ denote the set of non-identity elements of $\G_{\Prime}$, which is its set of generators. 
We fix a distinguished generator $\generatorEDSA \in \G_{\Prime}^*$. 
Then for any $X \in \G^{*}$, the discrete logarithm base $\generatorEDSA$ of $X$ is denoted $\DL_{\G, \generatorEDSA}(X)$, and it is in the set $\Z_{|\G|}$.
The instantiation of $\G$ used in Ed25519 is described in Section \ref{sec-gp-instantiation}.

\section{Functor framework}\label{sec-our-def-framework}

Our treatment relies on the notion of functors~\cite{EC:BelDavGun20}, which are functions that access an idealized primitive. We give relevant definitions, starting with signature schemes whose security is measured relative to a functor. Then we extend the notions of PRGs and PRFs to functors.

\headingu{Function spaces.} In using the random oracle model~\cite{CCS:BelRog93}, works in the literature sometimes omit to say what exactly are the domain and range of the underlying functions, and, when multiple functions are present, whether or not they are independent. (Yet, implicitly their proofs rely on certain choices.) For greater precision, we use the language of function spaces of~\cite{EC:BelDavGun20}, which we now recall. 

A \textit{function space} $\roSp$ is a set of tuples $\HH=(\HH_1,\ldots,\HH_n)$ of functions. The integer $n$ is called the arity of the function space, and can be recovered as $\roSp.\arity$. We view $\HH$ as taking an input $X$ that it parses as $(i,x)$ to return $\HH_i(x)$. 



\heading{Functors.} Following~\cite{EC:BelDavGun20}, we use the term functor for a transform that constructs one function from another. A functor $\construct{F}\Colon\FuncSp{SS}\to\FuncSp{ES}$ takes as oracle a function $\hh$ from a starting function space $\FuncSp{SS}$ and returns a function $\construct{F}[\hh]$ in the ending function space $\FuncSp{ES}$. (The term is inspired by category theory, where a functor maps from one category into another. In our case, the categories are function spaces.) If $\FuncSp{ES}$ has arity $n$, then we also refer to $n$ as the arity of $\construct{F}$, and write $\construct{F}_i$ for the functor which returns the $i$-th component of $\construct{F}$. That is, $\construct{F}_i[\hh]$ lets $\HH\gets\construct{F}[\hh]$ and returns $\HH_i$.


\heading{MD functor.} We are interested in the Merkle-Damg{\aa}rd~\cite{C:Merkle89a,C:Damgaard89b} transform. This transform constructs a hash function with domain $\bits^*$ from a compression function $\hh\Colon\bits^{b+2k}\to\bits^{2k}$ for some integers $b$ and $k$. The compression function takes a $2k$-bit chaining variable $y$ and a $b$-bit block $B$ to return a $2k$ bit output $\hh(y\|B)$. In the case of $\SHA512$, the hash function used in $\EdDSA$, the compression function $\sha{512}$ has $b=1024$ and  $k=256$ (so the chaining variable is 512 bits and a block is 1024 bits), while $b= 512$ and $k=128$ for $\SHA256$. In our language, the Merkle-Damg{\aa}rd transform is a functor $\construct{MD}\Colon \allowbreak \AllFuncs(\bits^{b+2k},\allowbreak \bits^{2k}) \allowbreak  \to \allowbreak  \AllFuncs(\bits^{*}, \allowbreak  \bits^{2k})$. It is parameterized by a padding function $\padF$ that takes the length $\ell$ of an input to the hash function and returns a padding string such that $\ell + |\padF(\ell)|$ is a multiple of $b$.
Specifically, $\padF(\ell)$ returns $10^*\birep{\ell}$ where $\birep{\ell}$ is a $64$-bit, resp. $128$-bit encoding of $\ell$ for $\SHAtwo$ resp. $\SHAfive$, and $0^*$ indicates the minimum number $p$ of $0$s needed to make $\ell+ 1 + p + 64$, resp. $\ell + 1 + p + 128$ a multiple of $b$. We also fix an ``initialization vector'' $\IV \in \bits^{2k}$. Given oracle $\hh$, the functor defines hash function $\HH = \construct{MD}[\hh]\Colon\bits^{*}\to\bits^{2k}$ as follows:
\begin{tabbing}
\underline{Functor $\construct{MD}[\hh](X)$} \\[2pt]
$y[0] \gets \IV$ \\
$P\gets \padF(|X|)$ ; $X'[1]\ldots X'[m] \gets X\|P$ \comment{Split $X\|P$ into $b$-bit blocks} \\
For $i=1,\ldots,m$ do $y[i] \gets \hh(y[i-1]\|X'[i])$ \\
Return $y[m]$
\label{MD}
\end{tabbing}
Strictly speaking, the domain is only strings of length less than $2^{64}$ resp. $2^{128}$, but since this is huge in practice, we view the domain as $\bits^*$. 

\headingu{Signature scheme syntax.} We give an enhanced, flexible syntax for a signature scheme $\DS$. We want to cover ROM schemes, which means scheme algorithms have oracle access to a function $\HH$, but of what range and domain? Since these can vary from scheme to scheme, we have the scheme begin by naming the function space $\DS.\HASHSET$ from which $\HH$ is drawn. We see the key-generation algorithm $\DS.\Kg$ as first picking a signing key $\sk\getsr\DS.\MakeSK$ via a signing-key generation algorithm $\DS.\MakeSK$, then obtaining the public verification key $\pk \gets \DS.\MakePK[\HH](\sk)$ by applying a deterministic verification-key generation algorithm $\DS.\MakePK$, and finally returning $(\pk,\sk)$. (For simplicity, $\DS.\MakeSK$, unlike other scheme algorithms, does not have access to $\HH$.) We break it up like this because we may need to explicitly refer to the sub-algorithms in constructions. Continuing, via $\sigma\gets \DS.\Sign[\HH](\sk,\pk,\msg;r)$ the signing algorithm takes $\sk,\pk$, a message $\msg \in \bits^*$, and randomness $r$ from the randomness space $\DS.\SigCoins$ of the algorithm, to return a signature $\sigma$. As usual, $\sigma\getsr \DS.\Sign[\HH](\sk,\pk,\msg)$ is shorthand for picking $r\getsr\DS.\SigCoins$ and returning $\sigma\gets \DS.\Sign[\HH](\sk,\pk,\msg;r)$. Via $b\gets \DS.\Vf[\HH](\pk,\msg,\sigma)$, the verification algorithm obtains a boolean decision $b \in \{\true, \false\}$ about the validity of the signature. The correctness requirement is that for all $\HH \in \DS.\HASHSET$, all $(\pk, \sk) \in \Outputs(\DS.\Kg[\HH])$, all $\msg \in \bits^*$ and all $\sigma\in \Outputs(\DS.\Sign[\HH](\sk,\pk,\msg))$ we have $\DS.\Vf[\HH](\pk, \msg, \sigma) = \true$.





\begin{figure}[t]
	\oneCol{0.8}{
		\ExperimentHeader{Game $\UFCMA_{\DS,\fF}$}

		\begin{oracle}{$\Initialize$}
			\item
			$\hh \getsr \startSpace$ ; $\HH \gets \fF[\HASH]$ 
			; $(\pk, \sk) \getsr \DS.\Kg[\HH]$
			; Return $\pk$
		\end{oracle}
		\ExptSepSpace

		\begin{oracle}{$\SignOO(\msg)$}
			\item $\sigma \getsr \DS.\Sign[\HH](\sk,\pk,\msg)$
			; $S \gets S \cup \{\msg\}$
			; Return $\sigma$
		\end{oracle}
		\ExptSepSpace

		\begin{oracle}{$\HASH(X)$}
			\item Return $\hh(X)$
		\end{oracle}
		\ExptSepSpace

		\begin{oracle}{$\Finalize(\chmsg, \chsig)$}
			\item If ($\chmsg\in S$) then return $\false$
			\item Return $\DS.\Vf[\HH](\pk, \chmsg, \chsig)$ \vspace{2pt}
		\end{oracle}
	} \vspace{-3pt}
		\twoCols{0.38}{0.53}{
		\ExperimentHeader{Game $\gamePRG_{\construct{P}}$}

		\begin{oracle}{$\Initialize$}
			\item
			$\hh \getsr \startSpace$  
			; $c \getsr \bits$  
			\item $s\getsr\bits^k$ ; $y_1\gets \construct{P}[\HASH](s)$ 
			\item  $y_0\getsr\bits^{\ell}$
			\item Return $y_c$
		\end{oracle}
		\ExptSepSpace

		\begin{oracle}{$\HASH(X)$}
			\item Return $\hh(X)$
		\end{oracle}
		\ExptSepSpace

		\begin{oracle}{$\Finalize(c')$}
			\item Return ($c=c'$) \vspace{2pt}
		\end{oracle}
	}
	{
\ExperimentHeader{Game $\gamePRF_{\construct{F}}$}

		\begin{oracle}{$\Initialize$}
			\item
			$\hh \getsr \startSpace$  
			; $c \getsr \bits$ 
		; $K\getsr\bits^k$  
		\end{oracle}
		\ExptSepSpace

		\begin{oracle}{$\FUNCO(X)$}
			\item If $\YTable[X]\neq\bot$ then
			\item \hindent If ($c=1$) then $\YTable[X]\gets \construct{F}[\HASH](K,X)$ 
			\item \hindent Else $\YTable[X] \getsr R$
			\item Return  $\YTable[X]$
		\end{oracle}
		\ExptSepSpace
		
		\begin{oracle}{$\HASH(X)$}
			\item Return $\hh(X)$
		\end{oracle}
		\ExptSepSpace


		\begin{oracle}{$\Finalize(c')$}
			\item Return ($c=c'$) \vspace{2pt}
		\end{oracle}
}	

		\vspace{-5pt}
	\caption{Top: Game defining UF security of signature scheme $\DS$ relative to functor $\fF\Colon\startSpace\to\DS.\HASHSET$. Bottom Left: Game defining PRG security of functor $\construct{P}\Colon\startSpace\to\AllFuncs(\bits^k,\bits^{\ell})$. Bottom Right: Game defining PRF security of functor $\construct{F}\Colon\startSpace\to\AllFuncs(\bits^k\cross\bits^*,R)$.}
	\label{fig:UF}\label{fig:fUF}\label{fig-prf}\label{fig-prg}
	\hrulefill
	\vspace{-10pt}
\end{figure}

\heading{UF security.} We want to discuss security of a signature scheme $\DS$ under different ways in which the functions in $\DS.\HASHSET$ are chosen or built. Game $\UFCMA_{\DS,\fF}$ in Fig.~\ref{fig:UF} is thus parameterized by a functor $\fF\Colon\startSpace\to\DS.\HASHSET$. At line~1, a starting function $\hh$ is chosen from the starting space of the functor, and then the function $\HH \in \DS.\HASHSET$ that the scheme algorithms (key-generation, signing and verification) get as oracle is determined as $\HH \gets \fF[\hh]$. The adversary, however, via oracle $\HASH$, gets access to $\hh$, which here is the random oracle. The rest is as per the usual unforgeability definition. (Given in the standard model in~\cite{GolMicRiv88} and extended to the ROM in~\cite{CCS:BelRog93}.) We define the UF advantage of adversary $\advA$ as $\ufAdv{\DS,\fF}{\advA} = \Pr[\UFCMA_{\DS,\fF}(\advA)]$.


\heading{PRGs and PRFs.} The usual definition  of a PRGs is for a function; we define it instead for a functor $\construct{P}$. The game $\gamePRG_{\construct{P}}$ is in Figure~\ref{fig-prg}. It picks a function $\hh$ from the starting space $\startSpace$ of the functor. The functor now determines a function $\construct{P}[\hh]\Colon\bits^k\to\bits^{\ell}$. The game then follows the usual PRG one for this function, additionally giving the adversary oracle access to $\hh$ via oracle $\HASH$. We let $\prgAdv{\construct{P}}{\advA} = 2\Pr[\gamePRG_{\construct{P}}(\advA)]-1$.

Similarly we extend the usual definition of PRG security to a functor $\construct{F}$, via game $\gamePRF_{\construct{F}}$ of Figure~\ref{fig-prf}. Here, for $\hh$ in the starting space $\startSpace$ of the functor, the defined function maps as $\construct{F}[\hh]\Colon\bits^k\cross\bits^*\to R$ for some $k$ and range set $R$. We let $\prfAdv{\construct{F}}{\advA} = 2\Pr[\gamePRF_{\construct{F}}(\advA)]-1$.






\section{The soundness of Derive-then-Derandomize}\label{sec-direct} 


%What has happened with this paper, in the review process, is unusual. We have one reviewer, at C22, who insisted that some kind of direct proof is possible and findiff is not needed. At AC22, a reviewer who had been privy to that view echoed it. 
%
%If we continue to ignore this and submit the same paper, we are likely to keep meeting the same response. Instead, let's try to turn the situation to our advantage rather than fight against it. This means we make a serious attempt to follow the path the reviewers appear to indicate. We will either see that it works, or find good reasons it does not.
%
%In this section I am going to outline such a path. It makes a number of separate assumptions on certain functors, all with access to a function $\hh$ representing the compression function. 
%
%A possible target here is a PKC 2023 submission with deadline November 1, 2022. Alternatively we can submit to a security conference like Usenix, with deadline Octover 11, 2022, or IEEE S\&P with deadline December 2, 2022. Regardless of deadlines, the earlier we are done with the paper, the better of course. 
%
%Once you have formed an impression of the updated version of the paper and what needs to be done, we can meet and discuss.
%
%Note that definitions have changed. The new definitions, which is the ones to which the following refers, are in Section~\ref{sec-our-def-framework}. You may want to start by looking over these. Then go on to what is below.
%
%I had seen the difficulty as being that $\hh$ is the same across all the functors. I am no longer sure this is really a difficulty. To get around it, definitions of PRG and PRF security are now for functors, not functions. The UF definition for Theorem~\ref{th-dd} has a single choice of $\hh$, yet the Theorem still seems to be true by exploiting the PRF and PRF security of the functors in game hops. I'd like to understand this better and check that you agree that it works.
%
%The first part is about the $\DRTransform$ transform. Then one has to bridge the gap to $\EdDSA$ by analyzing the functor $\ourF$. I have inserted skeleton Lemma and Theorem statements below to cover this. Start working through all this, filling in gaps, elaborating and extending as you see fit, including adding missing proofs or claims. 
%
%How can we extend these ideas, expand and add value? Think about this. One thought is that the $\DRTransform$ is more general than $\EdDSA$. It is a way to take any signature scheme and set its key space to $\bits^k$ while de-randomizing it. Is this interesting? Can we make this the focus of the paper, with $\EdDSA$ as one application?
%
%So far this is for single-user security. Can we say something interesting about multi-user security? The difficulty is the factor 16 security loss for clamping, which amplifies exponentially in the number of users under the naive reduction. Is there a better way? What about using the AGM?
%
%Can we get any similar or related results for other signature schemes like PSS?
%
%For simplicity I have removed the content of the prior paper that no longer felt directly relevant. Of course, it may be relevant or useful in some form, so pull it back as necessary from our prior files in this or other directories. It will likely need updating, though, to fit the new approach.


\begin{figure}[t]
\twoCols{0.45}{0.45}
{
  \begin{algorithm-initial}{$\fDS.\MakeSK$}
  \item $\osk\getsr\bits^k$ 
  ; Return $\osk$
  \end{algorithm-initial}

\begin{algorithm-subsequent}{$\fDS.\MakePK[\HH](\osk)$}
  \item $\eEDSA_1\|\eEDSA_2\gets\HH_1(\osk)$ 
  ; $\sk \gets \CFEDSA(\eEDSA_1)$ 
  \item $\pk \gets \DS.\MakePK[\HH_3](\sk)$ 
  \item Return $\pk$
  \end{algorithm-subsequent}
  
  \begin{algorithm-subsequent}{$\fDS.\Sign[\HH](\osk, \pk, \msg)$}
\item $\eEDSA_1\|\eEDSA_2\gets\HH_1(\osk)$ 
  ; $\sk \gets \CFEDSA(\eEDSA_1)$ 
  \item $r \gets \HH_2(\eEDSA_2,\msg)$  
\item $\sigma \gets \DS.\Sign[\HH_3](\sk,\pk,\msg;r)$ 
\item  Return $\sigma$
  \end{algorithm-subsequent}

  \begin{algorithm-subsequent}{$\fDS.\Vf[\HH](\pk, \msg, \sigma)$}
    \item Return $\DS.\Vf[\HH_3](\pk,\msg,\sigma)$
  \end{algorithm-subsequent}  \vspace{2pt}
 }
 {
  \begin{algorithm-initial}{$\jDS.\MakeSK$}
  \item $\osk\getsr\bits^k$ 
  ; Return $\osk$
  \end{algorithm-initial}

\begin{algorithm-subsequent}{$\jDS.\MakePK[\GH](\osk)$}
  \item $\sk \gets \CFEDSA(\osk)$ 
  \item $\pk \gets \DS.\MakePK[\GH](\sk)$ 
  \item Return $\pk$
  \end{algorithm-subsequent}
  
  \begin{algorithm-subsequent}{$\jDS.\Sign[\GH](\osk, \pk, \msg)$}
\item $\sk \gets \CFEDSA(\osk)$  
\item $\sigma \getsr \DS.\Sign[\GH](\sk,\pk,\msg)$ 
\item  Return $\sigma$
  \end{algorithm-subsequent}

  \begin{algorithm-subsequent}{$\jDS.\Vf[\GH](\pk, \msg, \sigma)$}
    \item Return $\DS.\Vf[\GH](\pk,\msg,\sigma)$
  \end{algorithm-subsequent}  \vspace{2pt}
 }
 
 
\vspace{-8pt}
\caption{\textbf{Left:} The signature scheme $\fDS = \DRTransform[\DS,\CFEDSA]$ constructed by the $\DRTransform$ transform applied to signature scheme $\DS$ and clamping function $\CFEDSA\Colon\bits^k\to\Outputs(\DS.\MakeSK)$.  \textbf{Right:} The signature scheme $\fDS = \JCTransform[\DS,\CFEDSA]$ constructed by the $\JCTransform$ transform.}
\label{fig-dd}
\hrulefill
\vspace{-10pt}
\end{figure}


We specify a general signature-hardening transform that we call Derive-then-Derandomize ($\DRTransform$) and prove that it preserves the security of the starting signature scheme.


\heading{The $\DRTransform$ transform.} Let $\DS$ be a given signature scheme that we call the base signature scheme. It will be the (general) Schnorr scheme in our application. Assume for simplicity that its function space $\DS.\HASHSET$ has arity~1.

The $\DRTransform$ (derive then de-randomize) transform constructs a signature scheme $\fDS = \DRTransform[\DS,\CFEDSA]$ based on $\DS$ and a function $\CFEDSA\Colon\bits^k\to\Outputs(\DS.\MakeSK)$, called the clamping function, that turns a $k$-bit string into a signing key for $\DS$. The algorithms of $\fDS$ are shown in Figure~\ref{fig-dd}. They have access to oracle $\HH$ that specifies sub-functions $\HH_1,\HH_2,\HH_3$. Function $\HH_1\Colon\bits^k\to\bits^{2k}$ expands the signing key $\osk$ of $\fDS$ into sub-keys $\e_1$ and $\e_2$. The clamping function is applied to $\e_1$ to get a signing key for the base scheme, and its associated verification key is returned as the one for the new scheme at line~4. At line~6, function $\HH_2\Colon\bits^k\cross\bits^*\to\DS.\SigCoins$ is applied to the second sub-key $\e_2$ and the message $\msg$ to determine signing randomness $r$ for the line~5 invocation of the base signing algorithm. Finally, $\HH_3 \in \DS.\HASHSET$ is an oracle for the algorithms of $\DS$. Formally the oracle space $\fDS.\HASHSET$ of $\fDS$ is the arity~3 space consisting of all $\HH = (\HH_1,\HH_2,\HH_3)$ that map as above. 

Viewing the PRG $\HH_1$, PRF $\HH_2$ and oracle $\HH_3$ for the base scheme as specified in the function space is convenient for our application to $\EdDSA$, where they are all based on $\MD$ with the \textit{same} underlying idealized compression function.

\heading{Just clamp.} Given a signature scheme $\DS$ and a clamping function $\CFEDSA\Colon\bits^k\to\Outputs(\DS.\MakeSK)$, it is useful to also consider the signature scheme $\jDS = \JCTransform[\DS,\CFEDSA]$ that does just the clamping. The scheme is shown in Figure~\ref{fig-dd}. Its oracle space is the same as that of $\DS$ and is assumed to have arity~1. On the right of Figure~\ref{fig-dd} the function drawn from it is denoted $\GH$; it will be the same as $\HH_3$ on the left.

\heading{Security of $\DRTransform$.} We study the security of the scheme $\fDS = \DRTransform[\DS,\CFEDSA]$ obtained via the $\DRTransform$ transform.

When we prove security of $\fDS$, it will be with respect to a functor $\fF$ that constructs all of $\HH_1,\HH_2,\HH_3$. This means that these three functions could all depend on the same starting function that $\fF$ uses, and in particular not be independent of each other. An important element of the following theorem is that it holds even in this case, managing to reduce security to conditions on the individual functors despite their using related (in fact, the same) underlying starting function. 


\begin{theorem}\label{th-dd} Let $\DS$ be a signature scheme. Let $\CFEDSA\Colon\bits^k\to\Outputs(\DS.\MakeSK)$ be a clamping function. Let $\fDS = \DRTransform[\DS,\CFEDSA]$ and $\jDS = \JCTransform[\DS,\CFEDSA]$ be the signature schemes obtained by the above transforms. Let $\fF\Colon \startSpace\to\fDS.\HASHSET$ be a functor that constructs the function $\HH$ that algorithms of $\fDS$ use as an oracle. Let $\advA$ be an adversary attacking the $\UFCMA$ security of $\fDS$. Then there are adversaries $\advA_1,\advA_2,\advA_3$ such that
\begin{align*}
	\ufAdv{\fDS,\fF}{\advA} &\leq   \prgAdv{\fF_1}{\advA_1} + \prfAdv{\fF_2}{\advA_2} + \ufAdv{\jDS,\fF_3}{\advA_3} \;.
\end{align*}
The constructed adversaries have $\Queries{\HASH}{\advA_i}=\Queries{\HASH}{\advA}$ ($i=1,2,3$) and approximately the same running time as $\advA$.  
% Let $\ell$ be an integer such that all messages queried to $\SignOO$ are no more than $4k \cdot \ell-k$ bits long. Adversary $\advA_1$ makes at most $1 + 2\ell \cdot \Queries{\SignOO}{\advA}+ \Queries{\HASH}{\advA}$ queries to $\HASH$.
Adversary $\advA_2$ makes $\Queries{\SignOO}{\advA}$ queries to $\FUNCO$.
%  and $1+\ell \cdot \Queries{\SignOO}{\advA} + \Queries{\HASH}{\advA}$ queries to $\HASH$. The runtime of $\advA_1$ and $\advA_2$ are both approximately $\Time{\advA} + \Time{ \DS.\MakePK} +  \Queries{\SignOO}{\advA}  \Time{ \DS.\Sign}+ \Time{ \DS.\VF}$.
Adversary $\advA_3$ makes $\Queries{\SignOO}{\advA}$ queries to $\SignOO$.
\end{theorem}
Recall that $\Queries{\Oracle}{\advB}$ means the number of queries made to oracle $\Oracle$ in the execution of the game with adversary $\advB$, so queries made by scheme algorithms, run in the game in response to $\advB$'s queries, are included. The theorem says the number of queries to $\HASH$ is preserved under this metric. The number of direct queries to $\HASH$ is not necessarily preserved. Thus $\QueriesD{\HASH}{\advA_i}$ could be more than $\QueriesD{\HASH}{\advA}$. For example $\QueriesD{\HASH}{\advA_1}$ is $\QueriesD{\HASH}{\advA}$ plus the number of queries to $\HASH$ made by the calls to $\fF_3[\HASH]$, the latter calls in turn made by the execution of $\DS.\Sign[\fF_3[\HASH]]$ across the different queries to $\SignOO$. Accounting precisely for this is involved, whence a preference where possible for the game-inclusive query metric $\Queries{\cdot}{\cdot}$.



\begin{figure}[t]
	\oneCol{0.9}{
		\ExperimentHeader{Games $\Gm_0,\Gm_1,\Gm_2$}

		\begin{oracle}{$\Initialize$}
			\item $\hh\getsr\FuncSp{SS}$
			\item $\osk\getsr\bits^k$ 
			; $\eEDSA_1\|\eEDSA_2\gets\fF_1[\HASH](\osk)$ \comment{Game $\Gm_0$} 
			\item $\eEDSA_1\|\eEDSA_2\getsr\bits^{2k}$ \comment{Games $\Gm_1,\Gm_2$} 
  \item $\sk \gets \CFEDSA(\eEDSA_1)$ 
  ; $\pk \gets \DS.\MakePK[\fF_3[\HASH]](\sk)$ 
			; Return $\pk$
		\end{oracle}
		\ExptSepSpace

		\begin{oracle}{$\SignOO(\msg)$}
		\item If $\STable[\msg]\neq\bot$ then return $\STable[\msg]$ 
			\item $r \gets \fF_2[\HASH](\e_2,\msg)$ \comment{Games $\Gm_0,\Gm_1$}
			\item $r\getsr\DS.\SigCoins$ \comment{Game $\Gm_2$}
\item $\STable[\msg] \gets \DS.\Sign[\fF_3[\HASH]](\sk,\pk,\msg;r)$ ; Return $\STable[\msg]$ 
		\end{oracle}
		\ExptSepSpace

		\begin{oracle}{$\HASH(X)$}
			\item Return $\hh(X)$
		\end{oracle}
			\ExptSepSpace

		\begin{oracle}{$\Finalize(\chmsg, \chsig)$}
			\item If ($\STable[\chmsg]\neq\bot$) then return $\false$
			\item Return $\DS.\Vf[\fF_3[\HASH]](\pk, \chmsg, \chsig)$ \vspace{2pt}
		\end{oracle}
	}
	
	\vspace{-5pt}
	\caption{Games for proof of Theorem~\ref{th-dd}. A line annotated with names of games is included only in those games.}
	\label{fig-dd-proof-1}
	\hrulefill
	\vspace{-10pt}
\end{figure}

\begin{proof}[Theorem~\ref{th-dd}] The proof uses code-based game playing~\cite{EC:BelRog06}. Consider the games of Figure~\ref{fig-dd-proof-1}. Let $\epsilon_i = \Pr[\Gm_i(\advA)]$ for $i=0,1,2$. 

Game $\Gm_0$ is the $\UFCMA$ game for $\fDS$ except that the signature of $M$ is stored in table $\STable$ at line~8, and, at line~5, if a signature for $M$ already exists, it is returned directly. Since signing in $\fDS$ is deterministic, meaning the signature is always the same for a given message and signing key, this does not change what $\SignO$ returns, and thus
\begin{align*}
	\ufAdv{\fDS,\fF}{\advA} &=  \epsilon_0 \\
	&= (\epsilon_0-\epsilon_1)+(\epsilon_1-\epsilon_2)+\epsilon_2 \;.
\end{align*}
We bound each of the three terms above in turn.

The change in moving to game $\Gm_1$ is at line~3, where we sample $\eEDSA_1\|\eEDSA_2$ uniformly from the set $\bits^{2k}$ rather than obtaining it via $\fF_1[\HASH]$ as in game $\Gm_0$. We build PRG adversary $\advA_1$ such that 
\begin{align}
	\epsilon_0-\epsilon_1 & \leq 
	\prgAdv{\fF_1}{\advA_1}\;. \label{eq-th-prg-advA}
\end{align}
Adversary $\advA_1$ is playing game $\gamePRG_{\fF_1}$. It gets its challenge via $\eEDSA_1\|\eEDSA_2 \gets \gamePRG_{\fF_1}.\Initialize$. It lets $\sk\gets\CFEDSA(\eEDSA_1)$ and $\vk\gets\DS.\MakePK[\fF_3[\gamePRG_{\fF_1}.\HASH]](\sk)$ where $\gamePRG_{\fF_1}.\HASH$ is the oracle provided in its own game. It runs $\advA$, returning $\vk$ in response to $\advA$'s $\Initialize$ query. It answers $\SignO$ queries as do $\Gm_0,\Gm_1$ except that it uses $\gamePRG_{\fF_1}.\HASH$ in place of $\HASH$ at lines~6,8. As part of this simulation, it maintains table $\STable$. It answers $\HASH$ queries via $\gamePRG_{\fF_1}.\HASH$. When $\advA$ calls $\Finalize(\chmsg, \chsig)$, adversary $\advA_1$ lets $c'\gets 1$ if $\DS.\Vf[\fF_3[\gamePRG_{\fF_1}.\HASH]](\pk, \chmsg, \chsig)$ is true and $\STable[\chmsg]=\bot$, and otherwise lets $c'\gets 0$. It then calls $\gamePRG_{\fF_1}.\Finalize(c')$. When the challenge bit $c$ in game $\gamePRG_{\fF_1}$ is $c=1$, the view of $\advA$ is as in $\Gm_0$, and when $c=0$ it is as in $\Gm_1$, which explains Eq.~\eqref{eq-th-prg-advA}.

Moving to $\Gm_2$, the change is that line~6 is replaced by line~7, meaning signing coins are now chosen at random from the randomness space $\DS.\SigCoins$ of $\DS$. We build  PRF adversary $\advA_2$ such that 
\begin{align}
	\epsilon_1-\epsilon_2 &\leq  \prfAdv{\fF_2}{\advA_2}\;.\label{eq-th-prf-advA}
\end{align}
Adversary $\advA_2$ is playing game $\gamePRF_{\fF_2}$. It picks $\eEDSA_1\|\eEDSA_2\getsr\bits^{2k}$. It lets $\sk\gets\CFEDSA(\eEDSA_1)$ and $\vk\gets\DS.\MakePK[\fF_3[\gamePRF_{\fF_2}.\HASH]](\sk)$ where $\gamePRG_{\fF_2}.\HASH$ is the oracle provided in its own game. It runs $\advA$, returning $\vk$ in response to $\advA$'s $\Initialize$ query. It answers $\SignO$ queries as does $\Gm_1$ except that it uses $\gamePRF_{\fF_2}.\FUNCO$ in place of $\fF_2[\HASH]$ at line~6 and $\gamePRF_{\fF_2}.\HASH$ in place of $\HASH$ in line~$8$. As part of this simulation, it maintains table $\STable$. It answers $\HASH$ queries via $\gamePRF_{\fF_2}.\HASH$. When $\advA$ calls $\Finalize(\chmsg, \chsig)$, adversary $\advA_2$ lets $c'\gets 1$ if $\DS.\Vf[\fF_3[\gamePRF_{\fF_2}.\HASH]](\pk, \chmsg, \chsig)$ is true and $\STable[\chmsg]=\bot$, and otherwise lets $c'\gets 0$. It then calls $\gamePRF_{\fF_2}.\Finalize(c')$. When the challenge bit $c$ in game $\gamePRF_{\fF_2}$ is $c=1$, the view of $\advA$ is as in $\Gm_1$, and when $c=0$ it is as in $\Gm_2$, which explains Eq.~\eqref{eq-th-prf-advA}.

Finally we build adversary $\advA_3$ such that 
\begin{align}
	\epsilon_2
	&\leq   \ufAdv{\jDS,\fF_3}{\advA_3} \;. \label{eq-th-uf-advA}
\end{align}
Adversary $\advA_3$ is playing game $\UFCMA_{\jDS,\fF_3}$. It lets $\vk \gets \UFCMA_{\jDS,\fF_3}.\Initialize$. It runs $\advA$, returning $\vk$ in response to $\advA$'s $\Initialize$ query. When $\advA$ makes query $M$ to $\SignO$, it answers as per the following:

\begin{tabbing}
If $\STable[\msg]\neq\bot$ then return $\STable[\msg]$ \\
$\STable[\msg] \getsr \UFCMA_{\jDS,\fF_3}.\SignO(M)$ ; Return $\STable[\msg]$ 
\end{tabbing} 

\noindent Note that memoizing signatures in $\STable$ is important here to ensure that the $\SignO$ queries of $\advA$ are correctly simulated. It answers $\HASH$ queries via $\UFCMA_{\jDS,\fF_3}.\HASH$. When $\advA$ calls $\Finalize(\chmsg, \chsig)$, adversary $\advA_2$ calls $\UFCMA_{\jDS,\fF_3}.\Finalize(\chmsg, \chsig)$. The distribution of signatures that $\advA$ is given, and of the keys underlying them, is as in $\Gm_2$, which explains Eq.~\eqref{eq-th-uf-advA}.

Note that the constructed adversaries having access to oracle $\HASH$ in their games is important to their ability to simulate $\advA$ faithfully. 

With regard to the costs (number of queries, running time) of the constructed adversaries, recall that we have defined these as the costs in the execution of the adversary with the game that the adversary is playing, so for example the number of queries to $\HASH$ includes the ones made by algorithms executed in the game. When this is taken into account, queries to $\HASH$ are preserved, and the other claims are direct.
\qed
\end{proof}


%Finally, we get the stated bound
%\begin{align}
%	\ufAdv{\fDS,\fF}{\advA} & =  \epsilon_0 \nonumber \\
%	& =  (\epsilon_0-\epsilon_1)+(\epsilon_1-\epsilon_2)+\epsilon_2 \nonumber \\
%	&\leq   \prgAdv{\fF_1}{\advA_1} + \prfAdv{\fF_2}{\advA_2} + \ufAdv{\DS,\fF_3}{\advA_3}\;. \nonumber
%\end{align}

%We construct $\advA_1$ in Figure~\ref{fig-ddA1}. We use $\Gm_1.\HASH$ for calling the $\HASH$ oracle in $\Gm_1$. The same principle applys when calling other oracles. By calling $\gamePRG_{\construct{\fF_1}}.\Initialize()$, our adversary $\advA_1$ gets the input $y_c$ for $c \in \bits$, where $y_1 \gets \fF_1[\hh](\osk)$ and $y_0 \getsr \bits^{2k}$. $\advA_1$ then uses $y_c$ as $\eEDSA_1\|\eEDSA_2$ to simulate game $\UFCMA_{\fDS,\fF}$ for $\advA$. 
%
%$\advA_1$ outputs 1 only if $\advA$ forges a valid signature message pair that passes the verification check. Notice when $c=0$, we have
%
% $\eEDSA_1\|\eEDSA_2 \getsr\bits^{k}$ and $\advA_1$ simulates $\Gm_1$; when $c=1$, $\eEDSA_1\|\eEDSA_2\gets\fF_1[\hh](\osk)$ and  $\advA_1$ simulates $\Gm_0$ perfectly. Hence, we have  
%\begin{align}
%	\prgAdv{\fF_1}{\advA_1}  &= \Pr[\advA_1 \Rightarrow 1 | c = 1] - \Pr[\advA_1 \Rightarrow 1 | c = 0] \nonumber \\
%	& \geq \epsilon_0 - \epsilon_1\nonumber
%\end{align}
%
%\begin{figure}[t]
%	\oneCol{0.75}
%	{
%		\begin{algorithm-initial}{adversary $\advA_1(y_c)$}
%			\item $(\chmsg, \chsig) \gets \advA[\Initialize, \SignO, \HASH]()$
%			\item If ($\STable[\chmsg]\neq\bot$) then return $0$
%			\item If $\DS.\Vf[\Gm_1.\fF_3[\HASH_1]](\pk, \chmsg, \chsig)$ then return $1$; Else return $0$
%		\end{algorithm-initial}
%	
%		\begin{algorithm-subsequent}{$\Initialize$}
%			\item $ y_c \gets \gamePRG_{\construct{\fF_1}}.\Initialize()$
%			\item $\e_1\|\e_2\gets y_c$ 
%			\item $\sk \gets \CF(\e_1)$ 
%			; $\pk \gets \DS.\MakePK[\fF_3[\Gm_1.\HASH]](\sk)$ 
%			\item Return $\pk$
%		\end{algorithm-subsequent}
%		\ExptSepSpace
%		
%		\begin{algorithm-subsequent}{$\SignO(\msg)$}
%			\item Return $\Gm_1.\SignO(\msg)$
%		\end{algorithm-subsequent}
%		
%		\begin{algorithm-subsequent}{$\HASH(X)$}
%			\item Return $\Gm_1.\HASH(X)$
%		\end{algorithm-subsequent} \vspace{2pt}
%	}
%	\vspace{-8pt}
%	\caption{Adversary $\advA_1$ for $\gamePRG_{\construct{\fF_1}}$ given adversary $\advA$ for $\UFCMA_{\fDS,\fF}$ in the proof of Theorem~\ref{th-dd}.}
%	\label{fig-ddA1}
%	\hrulefill
%	\vspace{-10pt}
%\end{figure}
%
%We construct $\advA_2$ in Figure~\ref{fig-ddA2}. In $\gamePRF_{\construct{\fF_2}}$, $\Initialize$ has no output and samples a key $K\getsr\bits^k$ privately. When $\advA_2$ simulates game $\UFCMA_{\fDS,\fF}$ for $\advA$, we let $K$ be $e_2$ such that $r$ is either the output of a PRF with key $e_2$ or a string randomly sampled from the randomness space $\DS.\SigCoins$. $\advA_2$ only needs to sample $e_1$ to derive $\pk$. It redirects $\advA$'s queries to $\HASH$ to $\Gm_2.\HASH$ oracle, and manually runs the signing algorithm except that it gets $r$ from $\FUNCO_2$ oracle. When $c = 0$, $\advA_2$ perfectly simulates $\Gm_2$ and $r \getsr \DS.\SigCoins$; when $c = 1$, $\advA_2$ perfectly simulates $\Gm_1$ and $r \gets \fF_2[\hh](\eEDSA_2, \cdot)$. Similarly to the probability bound for $\advA_1$, we have
%\begin{align}
%	\prfAdv{\fF_2}{\advA_2}  &= \Pr[\advA_2 \Rightarrow 1 | c = 1] - \Pr[\advA_2 \Rightarrow 1 | c = 0] \nonumber \\
%	& \geq \epsilon_1 - \epsilon_2 \nonumber 
%\end{align}

%\begin{figure}[t]
%	\oneCol{0.75}
%	{
%		\begin{algorithm-initial}{adversary $\advA_2$}
%			\item $(\chmsg, \chsig) \gets \advA[\Initialize,\SignO, \HASH]()$
%			\item If ($\STable[\chmsg]\neq\bot$) then return $0$
%			\item If $\DS.\Vf[\fF_3[\hh]](\pk, \chmsg, \chsig)$ then return $1$; Else return $0$
%		\end{algorithm-initial}
%	
%		\begin{algorithm-subsequent}{$\Initialize$}
%			\item $ \gamePRF_{\construct{\fF_2}}.\Initialize()$
%			\item $\e_1\getsr\bits^{k}$ 
%			\item $\sk \gets \CF(\e_1)$ 
%			; $\pk \gets \DS.\MakePK[\fF_3[\Gm_2.\HASH]](\sk)$ 
%			\item Return $\pk$
%		\end{algorithm-subsequent}
%		\ExptSepSpace
%		
%		\begin{algorithm-subsequent}{$\SignO(\msg)$}
%			\item If $\STable[\msg]\neq\bot$ then return $\STable[\msg]$ 
%			\item $r \gets\FUNCO_2(\msg)$
%			\item $\STable[\msg] \gets \DS.\Sign[\fF_3[\hh]](\sk,\pk,\msg;r)$ ; Return $\STable[\msg]$
%		\end{algorithm-subsequent}
%		
%		\begin{algorithm-subsequent}{$\HASH(X)$}
%			\item Return $\Gm_2.\HASH(X)$
%		\end{algorithm-subsequent} \vspace{2pt}
%	}
%	\vspace{-8pt}
%	\caption{Adversary $\advA_2$ for $\gamePRF_{\construct{\fF_2}}$ given adversary $\advA$ for $\UFCMA_{\fDS,\fF}$ in the proof of Theorem~\ref{th-dd}.}
%	\label{fig-ddA2}
%	\hrulefill
%	\vspace{-10pt}
%\end{figure}
%
%Finally, we define $\advA_3$ for $\UFCMA_{\DS,\fF_3}$ in Figure~\ref{fig-ddA3}.
%Our $\advA_3$ plays game $\Gm_2=\UFCMA_{\DS,\fF_3}$ and simulates $\UFCMA_{\fDS,\fF}$ for $\advA$ by redirecting queries to all oracles to $\Gm_2$ oracles. Obviously, $\advA_3$ simulates $\Gm_2$ perfectly and $\ufAdv{\DS,\fF_3}{\advA_3}  \geq \Pr[\Gm_2(\advA)] \geq \epsilon_2$.
%

%\begin{figure}[t]
%	\oneCol{0.75}
%	{
%		\begin{algorithm-initial}{adversary $\advA_3$}
%			\item $(\chmsg, \chsig) \gets \advA[\Initialize,\SignO, \HASH]()$
%			\item Return $\Gm_2.\Finalize(\chmsg, \chsig)$
%		\end{algorithm-initial}
%		
%		\begin{algorithm-subsequent}{$\Initialize$}
%			\item Return $\Gm_2.\Initialize()$
%		\end{algorithm-subsequent}
%		\ExptSepSpace
%		
%		\begin{algorithm-subsequent}{$\SignO(\msg)$}
%			\item Return $\Gm_2.\SignO(\msg)$
%		\end{algorithm-subsequent}
%		
%		\begin{algorithm-subsequent}{$\HASH(X)$}
%			\item Return $\Gm_2.\HASH(X)$
%		\end{algorithm-subsequent} \vspace{2pt}
%	}
%	\vspace{-8pt}
%	\caption{Adversary $\advA_3$ for $\UFCMA_{\DS,\fF_3}$ given adversary $\advA$ for $\UFCMA_{\fDS,\fF}$ in the proof of Theorem~\ref{th-dd}.}
%	\label{fig-ddA3}
%	\hrulefill
%	\vspace{-10pt}
%\end{figure}

\heading{Security of $\JCTransform$.} We have now reduced the security of $\fDS$ to that of $\jDS$. To further reduce the security of $\jDS$ to that of $\DS$, we give a general result on clamping. Let $\keySet = \Outputs(\DS.\MakeSK)$ and let $\CFEDSA\Colon\bits^k\to\keySet$ be a clamping function. As per terminology in Section~\ref{sec-prelims}, recall that $\Img(\CFEDSA) = \set{\CFEDSA(\osk)}{|\osk|=k}\subseteq \keySet$  is the image of the clamping function, and $\CFEDSA$ is {regular} if every $y\in\Img(\CFEDSA)$ has the same number of pre-images under $\CFEDSA$.


\begin{theorem}\label{th-jc} Let $\DS$ be a signature scheme such that $\DS.\MakeSK$ draws its signing key $\sk\getsr\keySet$ at random from a set $\keySet$.
 Let $\CFEDSA\Colon\bits^k\to\keySet$ be a regular clamping function. Let $\delta = |\Img(\CFEDSA)|/|\keySet| > 0$. Let $\jDS = \JCTransform[\DS,\CFEDSA]$ be the signature scheme obtained by the just-clamp transform. Let $\fF\Colon\FuncSp{SS}\to \DS.\HASHSET$ be any functor.  Let $\advB$ be an adversary attacking the $\UFCMA$ security of $\jDS$. Then % there is an adversary $\advB$ such that
\begin{align*}
	\ufAdv{\jDS,\fF}{\advB} &\leq  (1/\delta)\cdot \ufAdv{\DS,\fF}{\advB} \;.
\end{align*}
% The constructed adversary preserves the number of oracle queries of the original, and approximately preserves its running time.
\end{theorem}

\begin{proof}[Theorem~\ref{th-jc}] We consider running $\advB$ in game $\UFCMA_{\DS,\fF}$, where the signing key is $\sk\getsr\keySet$. With probability $\delta$ we have $\sk\in \Img(\CFEDSA)$. Due to the regularity of $\CFEDSA$, key $\sk$ now has the same distribution as a key $\CFEDSA(\osk)$ for $\osk\getsr\bits^k$ drawn in game $\UFCMA_{\jDS,\fF}$. Thus $\ufAdv{\DS,\fF}{\advB} \geq \delta\cdot \ufAdv{\jDS,\fF}{\advB}$. \qed  
\end{proof}


\section{Security of EdDSA} \label{sec-schemes}

%\heading{Obtaining $\EdDSA$ via $\DRTransform$.} Now one can see that if $\DS$ is the (general) $\Schnorr$ scheme then (for appropriate choices of the other arguments of the transform) $\fDS$ is $\EdDSA$. Thus the transform generalizes the way $\EdDSA$ is built from $\Schnorr$. Detail this. 


\begin{figure}[t]
\twoCols{0.3}{0.4}
{
  \begin{algorithm-initial}{$\DS.\MakeSK$}
  \item $\s \getsr \Z_{\Prime}$ % \comment{$\DS=\SchSig$}
 %  \item $\eEDSA_1\getsr\bits^k$ ; $\s\gets\CFEDSA(\eEDSA_1)$ \comment{$\DS = \SchSigCl{\CFEDSA}$}
  \item Return $\s$
  \end{algorithm-initial}
  
\begin{algorithm-subsequent}{$\DS.\MakePK(\s)$}
  \item  $\curvepoint{A} \gets \s\cdot \generatorEDSA$
  ; Return $\curvepoint{A}$
\end{algorithm-subsequent}

\begin{algorithm-subsequent}{$\DS.\Sign[\HH](\s, \curvepoint{A}, \msg)$}
\item $r \getsr \Z_{\Prime}$ ; $\curvepoint{R} \gets \littler\cdot \generatorEDSA$
\item $c \gets \HH(\curvepoint{R}\|\curvepoint{A}\| \msg)$
\item $\z \gets (\s c + \littler) \mod \Prime$
\item Return $(\curvepoint{R}, \z)$
  \end{algorithm-subsequent}
  \begin{algorithm-subsequent}{$\DS.\Vf[\HH](\curvepoint{A}, \msg, \sigma)$}
  \item $(\curvepoint{R}, \z) \gets \sigma$
  \item  $c \gets \HH(\curvepoint{R}\|\curvepoint{A}\| \msg)$
  \item Return $\VF(\curvepoint{A},\curvepoint{R},c,\z)$
  \end{algorithm-subsequent}  \vspace{2pt}
 }
{
\begin{algorithm-initial}{$\fDS.\MakeSK$}
		\item $\sk \getsr \bits^k$
		; Return $\sk$
\end{algorithm-initial}	
		
\begin{algorithm-subsequent}{$\fDS.\MakePK(\sk)$}
  \item  $\eEDSA_1\|\eEDSA_2 \gets \HH_1(\sk)$
		; $\s \gets \Clamp(\eEDSA_1)$
		\item $\curvepoint{A} \gets \s\cdot \generatorEDSA$
  ; Return $\curvepoint{A}$
\end{algorithm-subsequent}		


	\ExptSepSpace
	\begin{algorithm-subsequent}{$\fDS.\Sign[\HH](\sk,\curvepoint{A},\msg)$}
		\item $\eEDSA_1\|\eEDSA_2 \gets \HH_1(\sk)$
		; $\s \gets \Clamp(\eEDSA_1)$ 
		\item $\littler \gets \HH_2(\eEDSA_2,\msg)$
		; $\curvepoint{R} \gets \littler \cdot \generatorEDSA$
		\item $c \gets \HH_3(\curvepoint{R}\|\curvepoint{A}\|\msg)$
		\item $\z \gets (\s c + \littler) \bmod \Prime$
		\item Return $(\curvepoint{R},\z)$
	\end{algorithm-subsequent}
	\ExptSepSpace
	\begin{algorithm-subsequent}{$\fDS.\Vf[\HH](\curvepoint{A}, \msg, \sigma)$}
		\item $(\curvepoint{R}, \z) \gets \sigma$
		\item $c \gets \HH_3(\curvepoint{R}\|\curvepoint{A}\|\msg) \bmod \Prime$
		\item Return $\VF(\curvepoint{A},\curvepoint{R},c,\z)$
	\end{algorithm-subsequent}  \vspace{2pt}
 }
	\twoCols{0.3}{0.4}{
	\begin{oracle}{\underline{$\CF(e)$}\comment{$e \in \bits^{k}$}}
		
		\item $t \gets 2^{k-2}$
		\item for $i \in [4..k-2]$
		\item \quad $t \gets t+2^{i-1}\cdot e[i]$
		\item $s\gets t\bmod \Prime$
		\item return $s$
	\end{oracle}
	}
	{
		\begin{algorithm-initial}{$\sVF(\curvepoint{A}, \curvepoint{R}, c, \z)$}
			\item Return ($\z \cdot \generatorEDSA = c \cdot \curvepoint{A} + \curvepoint{R}$)
		\end{algorithm-initial} \vspace{10pt}
	
		\begin{algorithm-initial}{$\pVF(\curvepoint{A}, \curvepoint{R}, c, \z)$}
			\item Return $2^\cofactor(\z \cdot  \generatorEDSA)= 2^\cofactor(c \cdot \curvepoint{A} + \curvepoint{R})$
		\end{algorithm-initial} 
	}

\vspace{-8pt}
\caption{\textbf{Top Left:} the Schnorr scheme. \textbf{Top Right:} The EdDSA scheme. \textbf{Bottom Left:} EDDSA clamping function (generalized for any $k$; in the original definition, $k=256$). \textbf{Bottom Right:} Strict and Permissive verification algorithms as choices for $\VF$.
}
\label{fig-eddsa}\label{fig-schnorr}\label{fig:VFs}
\hrulefill
\vspace{-10pt}
\end{figure}

\headingu{The Schnorr scheme.} Let the prime-order group $\G_{\Prime}$ of $k$-bit strings with generator $\generatorEDSA$ be as described in Section~\ref{sec-prelims}. The algorithms of the  Schnorr signature scheme $\DS=\SchSig$
%  and also give an extension $\SchSigCl{\CFEDSA}$ where $\CFEDSA\Colon\bits^k\to\Z_{\Prime}$ is a clamping function. The algorithms of the two schemes 
are shown on the left in Figure~\ref{fig-schnorr}. 
% The schemes differ only in their $\MakeSK$ algorithms, as indicated at lines 1,2. (The reason to introduce $\SchSigCl{\CFEDSA}$ is that we reduce $\EdDSA$ to it via Theorem~\ref{th-dd} and then reduce it to $\SchSig$ for the particular $\EdDSA$ clamping function exploiting self-reducibility of the DL problem.) 
The function space $\DS.\HASHSET$ is $\AllFuncs(\bits^*,\Z_{\Prime})$. (Implementations may use a hash function that outputs a string and embed the result in $\Z_{\Prime}$ but following prior proofs~\cite{EC:AABN02} we view the hash function as directly mapping into $\Z_{\Prime}$.) Verification is parameterized by an algorithm $\VF$ to allow us to consider strict and permissive verification in a modular way. The corresponding choices of verification algorithms are at the bottom of Figure~\ref{fig:VFs}. The signing randomness space is $\DS.\SigCoins = \Z_{\Prime}$.

% \authnote{Reviewer}{At some point you seem to assume that the hash function H for Schnorr signatures outputs values from $\Z_{\Prime}$. I've never seen this. I think it's always a t-bit string embedded into $\Z_{\Prime}$. }{red}
% \textcolor{blue}{In pratice, it is indeed the case that H maps to $\{0,1\}^k$ instead of $\Z_{\Prime}$, but the proof by Seurin Eurocrypt'12 for the security of Schnorr under RO still works by replacing $\Prime$ with $2^k$}

Schnorr signatures have a few variants that differ in details. In Schnorr's paper~\cite{JC:Schnorr91}, the challenge is $c = \HH(\curvepoint{R}\|\msg) \bmod \Prime$. Our inclusion of the public key in the input to $\HH$ follows Bernstein~\cite{EPRINT:Bernstein15} and helps here because it is what $\EdDSA$ does. It doesn't affect security. (The security of the scheme that includes the public key in the hash input is implied by the security of the one that doesn't via a reduction that includes the public key in the message.) Also in~\cite{JC:Schnorr91}, the signature is $(c, \z)$. The version we use, where it is $(\curvepoint{R},\z)$, is from~\cite{EC:AABN02}. However, BBSS~\cite{NORDSEC:BBSS18} shows that these versions have equivalent security.





\heading{The EdDSA scheme.} Let the prime-order group $\G_{\Prime}$ of $k$-bit strings with generator $\generatorEDSA$ be as before and assume $2^{k-5} < \Prime < 2^k$. Let $\CFEDSA\Colon\bits^k\to \Z_{\Prime}$ be the clamping function shown at the bottom of Figure~\ref{fig-eddsa}. The algorithms of the scheme $\fDS$ are shown on the right side of Figure~\ref{fig-eddsa}. The key length is $k$. As before, the verification algorithm $\VF$ is a parameter. The $\HH$ available to the algorithms defines three sub-functions. The first, $\HH_1\Colon\bits^k\to\bits^{2k}$, is used at lines 2,4, where its output is parsed into $k$-bit halves. The second, $\HH_2\Colon\bits^k\cross\bits^*\to\Z_{\Prime}$, is used at line~5 for de-randomization. The third, $\HH_3\Colon\bits^*\to\Z_{\Prime}$, plays the role of the function $\HH$ for the Schnorr schemes. Formally, $\fDS.\HASHSET$ is the arity-3 function space consisting of all $\HH$ mapping as just indicated.

In~\cite{bernstein2012high,SP:BCJZ21}, the output of the clamping is an integer that (in our notation) is in the range $2^{k-2},\ldots,2^{k-1}-8$. When used in the scheme, however, it is (implicitly) modulo $\Prime$. It is convenient for our analysis, accordingly, to define $\CFEDSA$ to be the result modulo $\Prime$ of the actual clamping. Note that in $\EdDSA$ the prime $\Prime$ has magnitude a little more than $2^{k-4}$ and less than $2^{k-3}$.

There are several versions of EdDSA depending on the choice for verification algorithms: strict,  permissive or batch $\VF$. We specify the first two choices in Figure~\ref{fig:VFs}. Our results hold for all choices of $\VF$, meaning $\EdDSA$ is secure with respect to $\VF$ assuming $\Schnorr$ is secure with respect to $\VF$. It is in order to make this general claim that we abstract out $\VF$. 

%We discuss their security in \fullorAppendix{sec-vf}. 

\heading{Security of $\EdDSA$ with independent ROs.} As a warm-up, we show security of $\EdDSA$ when the three functions it uses are independent random oracles, the setting assumed by BCJZ~\cite{SP:BCJZ21}. However, while they assume hardness of DL, our result is more general, assuming only security of $\Schnorr$ with a monolithic random oracle. We can then use known results on $\Schnorr$~\cite{JC:PoiSte00,EC:AABN02} to recover the result of BCJZ~\cite{SP:BCJZ21}, but the proof is simpler and more modular. Also, other known results on $\Schnorr$~\cite{C:RotSeg21,INDOCRYPT:BelDai20,EC:FucPloSeu20} can be applied to get better bounds. Following this, we will turn to the ``real'' case, where the three functions are all $\MD$ with a random compression function.

The Theorem below is for a general prime $\Prime > 2^{k-5}$ but in $\EdDSA$ the prime is $2^{k-4} < \Prime < 2^{k-3}$ so the value of $\delta$ below is $\delta = 2^{k-5}/\Prime > 2^{k-5}/2^{k-3} = 1/4$, so the factor $1/\delta$ is $\leq 4$. We capture the three functions of $\EdDSA$ being independent random oracles by setting functor $\idFunctortwo$ below to the identity functor, and similarly capture $\Schnorr$ being with a monolithic random oracle by setting $\idFunctor$ to be the identity functor. 

% It may be worth recalling our convention that query counts of an adversary include those made by oracles in its game, so for example $\Queries{\HASH}{\advA} \geq \Queries{\SignO}{\advA}$ will always be true, in case one asks why the added term in the bound below shows no explicit dependency on the latter.


% Recall that $\Queries{\HASH}{\advA}$ counts the number of $\HASH$ queries made in the execution of the game with $\advA$, while $\QueriesD{\HASH}{\advA}$, referred to below, is the number of direct queries, meaning those made by $\advA$ to $\HASH$ directly.



\begin{theorem}\label{th-eddsa-r} Let $\DS=\SchSig$ be the $\Schnorr$ signature scheme of Figure~\ref{fig-schnorr}. Let $\CFEDSA\Colon\allowbreak\bits^k\allowbreak\to\allowbreak\Z_{\Prime}$ be the clamping function of Figure~\ref{fig-eddsa}. Assume % $2^{k} > 
$\Prime > 2^{k-5}$ and let $\delta = 2^{k-5}/\Prime$. Let $\fDS = \DRTransform[\DS,\CFEDSA]$ be the $\EdDSA$ signature scheme. 
Let $\idFunctor\Colon\AllFuncs(\bits^*,\Z_{\Prime})\to \AllFuncs(\bits^*,\Z_{\Prime})$ be the identity functor.  Let $\idFunctortwo\Colon\fDS.\HASHSET\to \fDS.\HASHSET$ be the identity functor. Let $\advA$ be an adversary attacking the $\UFCMA$ security of $\fDS$. 
% Let $ b \cdot (\ell-1) - 2k$ be the maximum length in bits of a message input to $\SignO$. 
Then there is an adversary $\advB$ such that
\begin{align*}
	\ufAdv{\fDS,\idFunctortwo}{\advA} \leq & 
	 (1/\delta)\cdot\ufAdv{\DS,\idFunctor}{\advB} 
	 + \frac{2\cdot \Queries{\HASH}{\advA}}{2^k} 	 \;.
\end{align*}
Adversary $\advB$ preserves the queries and running time of $\advA$. 
\end{theorem}
\begin{proof}[Theorem~\ref{th-eddsa-r}] Let $\jDS = \SchSigCl{\CFEDSA}$. 
By Theorem~\ref{th-dd}, we have 
	$$\ufAdv{\fDS,\idFunctortwo}{\advA} \leq   \prgAdv{\idFunctortwo_1}{\advA_1} + \prfAdv{\idFunctortwo_2}{\advA_2} + \ufAdv{\jDS,\idFunctortwo_3}{\advA_3} \;.$$
It is easy to see that
\begin{align*}
	\prgAdv{\idFunctortwo_1}{\advA_1} &\leq \frac{\QueriesD{\HASH}{\advA_1}}{2^k} \leq \frac{\Queries{\HASH}{\advA}}{2^k} \\
	% \frac{1 + 2\ell \cdot \Queries{\SignOO}{\advA} + \Queries{\HASH}{\advA}}{2^k} \\
	\prfAdv{\idFunctortwo_2}{\advA_2}  \leq& \frac{\QueriesD{\HASH}{\advA_2}}{2^k}\leq \frac{\Queries{\HASH}{\advA}}{2^k} \;.
\end{align*}  
Under the assumption $\Prime > 2^{k-5}$ made in the theorem, 
	BCJZ~\cite{SP:BCJZ21} established that $|\Img(\CFEDSA)|=2^{k-5}$. So $|\Img(\CFEDSA)|/|\Z_{\Prime}| = 2^{k-5}/\Prime = \delta$. Let $\advB = \advA_3$ and note that $\idFunctortwo_3 = \idFunctor$. So by Theorem~\ref{th-jc} we have 
\begin{align}
	\ufAdv{\jDS,\idFunctortwo_3}{\advA_3} &\leq (1/\delta)\cdot\ufAdv{\DS,\idFunctor}{\advB} \;.
\end{align}
Collecting terms, we obtain the claimed bound stated in Theorem~\ref{th-eddsa-r}. \qed
\end{proof}


\begin{figure}[t]
\oneCol{0.8}{
\begin{algorithm-subsequentC}{Functor $\ourF_1[\hh](\sk)$}{$|\sk|=k$}
\item $\e \gets \MD[\hh](\sk)$ ; Return $\eEDSA$ \comment{$|\eEDSA|=2k$} 
\end{algorithm-subsequentC}
\begin{algorithm-subsequentC}{Functor $\ourF_2[\hh](\eEDSA_2,\msg)$}{$|\eEDSA_2|=k$}
\item Return $\MD[\hh](\eEDSA_2\|\msg)\bmod\Prime$
\end{algorithm-subsequentC}
\begin{algorithm-subsequentC}{Functor $\ourF_3[\hh](X)$}{also called $\ModMD$}
\item Return $\MD[\hh](X)\bmod\Prime$
\end{algorithm-subsequentC}
}
\vspace{-8pt}
\caption{The arity-3 functor $\ourF$ for $\EdDSA$. Here $\hh\Colon\bits^{b+2k}\to\bits^{2k}$ is a compression function.}
\label{fig-our-functor}
\hrulefill
\vspace{-10pt}
\end{figure}




\heading{Analysis of the $\ourF$ functor.} Let $\fDS$ be the result of the $\DRTransform$ transform applied to $\SchSig$ and a clamping function $\CFEDSA\Colon\bits^k\to\Z_{\Prime}$. Security of $\EdDSA$ is captured as security in game $\UFCMA_{\fDS,\ourF}$ when $\ourF$ is the functor that builds the component hash functions in the way that $\EdDSA$ does, namely from a MD-hash function. To evaluate this security, we start by defining the functor $\ourF$ in Figure~\ref{fig-our-functor}. It is an arity-3 functor, and we separately specify $\ourF_1,\ourF_2,\ourF_3$. (Functor $\ourF_3$ will be called $\ModMD$ in later analyses.) The starting space, from which $\hh$ is drawn, is $\AllFuncs(\bits^{b+2k},\bits^{2k})$, the set of compression functions. The prime $\Prime$ is as before, and is public.


We want to establish the three assumptions of Theorem~\ref{th-dd}. Namely: (1) $\ourF_1$ is PRG-secure (2) $\ourF_2$ is PRF secure and (3) security holds in game $\UFCMA_{\SchSig^*,\ourF_3}$ where $\SchSig^* = \JCTransform[\SchSig,\CFEDSA]$. Bridging from $\SchSig^*$ to $\SchSig$ itself will use Theorem~\ref{th-jc}.

\begin{figure}[t]
	\oneCol{0.6}{
		\ExperimentHeader{Games {$\Gm_0$}, \fbox{$\Gm_1$}}
		
		\begin{oracle}{$\Initialize$}
			\item $\sk\getsr\bits^k$ 
			; $\eEDSA\getsr \bits^{2k}$  
			\item Return $\eEDSA$
		\end{oracle}
		\ExptSepSpace
		
		\begin{oracle}{$\HASH(X)$}
			\item If $\FTable[X]\neq\bot$ then return $\FTable[X$] 
			\item $Y\getsr\bits^{2k}$ 
			\item If $X = \IV\|\sk\|P$ then $\bad \gets \true$ ; \fbox{$Y \gets e$}
			\item $\FTable[X]\gets Y$ ; Return $\FTable[X]$
		\end{oracle}
		\ExptSepSpace
		
		\begin{oracle}{$\Finalize(c')$}
			\item Return ($c'=1$) \vspace{2pt}
		\end{oracle}
	}
	\vspace{-5pt}
	\caption{Games $\Gm_0$ and $\Gm_1$ in the proof of Lemma~\ref{lm-ourF1}. Boxed code is only in $\Gm_1$.}
	\label{fig-lm2}
	\hrulefill
	\vspace{-10pt}
\end{figure}

\begin{lemma}\label{lm-ourF1} Let functor $\ourF_1\Colon \AllFuncs(\bits^{b+2k},\bits^{2k}) \to \AllFuncs(\bits^k,\bits^{2k})$ be defined as in Figure~\ref{fig-our-functor}. Let $\advA_1$ be an adversary. Then
\begin{align}
	\prgAdv{\ourF_1}{\advA_1} & \leq \frac{\QueriesD{\HASH}{\advA_1}}{2^k}  \leq \frac{\Queries{\HASH}{\advA_1}}{2^k} \;. \label{eq-lm-ourF1}
\end{align}
\end{lemma}

\begin{proof}[Lemma~\ref{lm-ourF1}] Since the input $\sk$ to $\ourF_1[\hh]$ is $k$-bits long, the $\MD$ transform defined in Section~\ref{sec-our-def-framework} only iterates once and the output is $e = \hh(\IV\|\sk\|P)$, for padding $P \in \bits^{3k}$ and initialization vector $\IV \in \bits^{2k}$ that are fixed and known. Now consider the games in Figure~\ref{fig-lm2}, where the boxed code is only in $\Gm_1$. Then we have
\begin{align*}
	\prgAdv{\ourF_1}{\advA_1} &= \Pr[\Gm_1(\advA_1)] - \Pr[\Gm_0(\advA_1)] \\
	&\leq \Pr[\Gm_0(\advA_1)\mbox{ sets }\bad] \\
    &\leq 	\frac{\Queries{\HASH}{\advA_1}}{2^k} \;.
\end{align*}
The second line above is by the Fundamental Lemma of Game Playing, which applies since $\Gm_0,\Gm_1$ are identical-until-$\bad$. \qed
\end{proof}

% The proof of this lemma can be found in Appendix~\ref{sec-lem3}.

We turn to PRF security of the $\ourF_2$ functor. Note that the construction is what BRT called $\AMAC$~\cite{EC:BelBerTes16}. They proved its PRF security by a combination of standard-model and ROM results. First they showed $\AMAC$ is PRF-secure if the compression function $\hh$ is PRF-secure under leakage of a certain function of the key. Then they show that ideal compression functions have this PRF-under-leakage security. Putting this together implies PRF security of $\ourF_2$. However, we found it hard to put the steps and Lemmas in BRT together to get a good, concrete bound for the PRF security of $\ourF_2$. Instead we give a direct proof, with an explicit bound, using our result on the indifferentiability of $\ModMD$ from Theorem~\ref{th-md-indiff} together with the indifferentiability composition theorem~\cite{TCC:MauRenHol04}.


\begin{lemma}\label{lm-ourF2} Let functor $\ourF_2\Colon \AllFuncs(\bits^{b+2k},\bits^{2k}) \to \AllFuncs(\bits^k\cross\bits^*,\Z_{\Prime})$ be defined as in Figure~\ref{fig-our-functor}. Let $\ell$ be an integer such that all messages queried to $\HASH$ are no more than $b \cdot (\ell-1) - k$ bits long. Let $\advA_2$ be an adversary. Then
\begin{align*}
	\prfAdv{\ourF_2}{\advA_2}  \leq& \frac{\Queries{\HASH}{\advA_2}}{2^k}
	+ \frac{2\Prime (\QueriesD{\HASH}{\advA_2}+ \ell \Queries{\FUNCO}{\advA_2})}{2^{2k}}
	+ \frac{(\QueriesD{\HASH}{\advA_2} + \ell \Queries{\FUNCO}{\advA_2})^2}{2^{2k}}
	+ \frac{\Prime\QueriesD{\HASH}{\advA_2} \cdot \ell\Queries{\FUNCO}{\advA_2}}{2^{2k}} 
	.
	\label{eq-lm-ourF2}
\end{align*}
\end{lemma}
\begin{proof}[Lemma~\ref{lm-ourF2}]
	In Section~\ref{sec-chop}, we prove the indifferentiability of functor $\ourF_3$ (c.f. Figure~\ref{fig-our-functor}), which we also call $\ModMD$.
	Define $\construct{R}\Colon\AllFuncs(\bits^*,\Z_{\Prime})\to \AllFuncs(\bits^k \times \bits^*,\Z_{\Prime})$ to be the identity functor such that $\construct{R}[\HH](x, y)=\HH(x \concat y)$ for all $x, y, \HH$ in the appropriate domains.
	Notice that when $\construct{R}$ is given access to the $\ModMD$ functor as its oracle, the resulting functor is exactly $\ourF_2$. 
	Using this property, we will reduce the PRF security of functor $\ourF_2$ to the indifferentiability of $\ModMD$.

	For any simulator algorithm $\simulator$, the indifferentiability composition theorem~\cite{TCC:MauRenHol04} grants the existence of distinguisher $\advD$ and adversary $\advA_5$ such that
	\[
	\prfAdv{\ourF_2}{\advA_2} \leq \prfAdv{\construct{R}}{\advA_5}+ \genAdv{\indiff}{\ModMD, \simulator}{\advD}.
	\]
	We let $\simulator$ be the simulator guaranteed by Theorem~\ref{th-md-indiff} and separately bound each of these terms.
	Adversary $\advA_5$ simulates the PRF game for its challenger $\advA_2$ by forwarding all $\FUNCO$ queries to its own $\FUNCO$ oracle and
	answering $\HASH$ queries using the simulator, which has access to the $\HASH$ oracle of $\advA_5$.
	Since the simulator is efficient and makes at most one query to its oracle each time it is run, we can say the runtime of $\advA_5$ is approximately the same as that of $\advA_2$.
	$\advA_5$ makes the same number of $\FUNCO$ and $\HASH$ queries as $\advA_2$. 

	Next, we want to compute $\prfAdv{\construct{R}}{\advA_5}$. When $\construct{R}$ is evaluated with access to a random function $\hh$, its outputs are random unless the adversary makes a relevant query involving the secret key.
	The adversary can only distinguish if the output of $\FUNCO$ is randomly sampled or from $\construct{R}[\hh]$ if it queries $\HASH$ on the $k$-bit secret key ($\eEDSA_2$), which has probability $\frac{1}{2^{k}}$ for a single query. 
	Taking a union bound over all $\HASH$ queries, we have
	$$\prfAdv{\construct{R}}{\advA_5} \leq \frac{\Queries{\HASH}{\advA_{2}}}{2^{k}}.$$

	Distinguisher $\advD$ simulates the PRF game for $\advA_2$, by replacing functor $\ModMD$ with its own $\PrivO$ oracle within the $\FUNCO$ oracle and forwarding $\advA_2$'s direct $\HASH$ queries to $\PubO$.
	$\advD$ hence makes $\Queries{\advA_2}{\FUNCO}$ queries to $\PrivO$ of maximum length $b \cdot (\ell-1)$ and $\QueriesD{\advA_2}{\HASH}$ to $\PubO$. 
	To bound the second term, we apply Theorem~\ref{th-md-indiff} on the indifferentiability of shrink-MD transforms.
	This theorem is parameterized by two numbers $\gamma$ and $\epsilon$; 
	in Section~\ref{sec-chop}, we show that $\ModMD$ belongs to the shrink-MD class for $\gamma = \lfloor \frac{2^{2k}}{\Prime} \rfloor$ and $\epsilon = \frac{\Prime}{2^{2k}}$.
	Then the theorem gives
	\[\genAdv{\indiff}{\ModMD, \simulator}{\advD} \leq 2(\Queries{\PubO}{\advD}+ \ell \Queries{\PrivO}{\advD}) \epsilon
	+ \frac{(\Queries{\PubO}{\advD} + \ell \Queries{\PrivO}{\advD})^2}{2^{2k}} 
	+ \frac{ \Queries{\PubO}{\advD} \cdot \ell \Queries{\PrivO}{\advD}}{\gamma}.\]
	
	By substituting $\Queries{\PubO}{\advD} = \QueriesD{\HASH}{\advA_2}$ and $\Queries{\PrivO}{\advD} = \Queries{\FUNCO}{\advA_2}$, we obtain the bound stated in the theorem.\qed
\end{proof}

Finally we turn to $\ourF_3$. The following considers the UF security of $\jDS = \SchSigCl{\CFEDSA}$ with the hash function being an MD one,  meaning with $\ourF_3$, and reduces this to the UF security of the same scheme with the hash function being a monolithic random oracle. Formally, the latter is captured by game $\UFCMA_{\jDS,\construct{R}}$ where $\construct{R}$ is the identity functor. One route to this result is to exploit the public-indifferentiability of $\MD$ established by DRS~\cite{EC:DodRisShr09}. However we found it simpler to give a direct proof and bound based on our Theorem~\ref{th-md-indiff}.

\begin{lemma}\label{lm-ourF3} Let functor $\ourF_3\Colon \AllFuncs(\bits^{b+2k},\bits^{2k}) \to \AllFuncs(\bits^*,\Z_{\Prime})$ be defined as in Figure~\ref{fig-our-functor}. Assume $2^k > \Prime$. Let $\jDS = \SchSigCl{\CFEDSA}$ where $\CFEDSA\Colon\bits^k\to\Z_p$ is a clamping function. Let $\construct{R}\Colon\AllFuncs(\bits^*,\Z_{\Prime})\to \AllFuncs(\bits^*,\Z_{\Prime})$ be the identity functor, meaning $\construct{R}[\HH]=\HH$. Let $\advA_3$ be a $\UFCMA$ adversary 
% making $\Queries{\HASH}{\advA_3}, \Queries{\SignO}{\advA_3}$ queries to its respective oracles, 
and let $\ell$ be an integer such that the maximum message length
	$\advA_3$ queries to $\SignO$ is at most $ b \cdot (\ell-1) - 2k$ bits. Then we can construct adversary $\advA_4$ such that
\begin{align}
	\ufAdv{\jDS,\ourF_3}{\advA_3} & \leq \ufAdv{\jDS,\construct{R}}{\advA_4}+ \frac{2\Prime(\QueriesD{\HASH}{\advA_3}+ \ell \Queries{\SignO}{\advA_3})}{2^{2k}} \\
	& + \frac{(\QueriesD{\HASH}{\advA_3}+ \ell \Queries{\SignO}{\advA_3})^2}{2^{2k}}+ \frac{\Prime \QueriesD{\HASH}{\advA_3}\cdot \ell \Queries{\SignO}{\advA_3}}{2^{2k}} \;.
\end{align}
	Adversary $\advA_4$ has approximately equal runtime and query complexity to $\advA_3$.
\end{lemma}
\begin{proof}[Lemma~\ref{lm-ourF3}]
	Again, we rely on the indifferentiability of functor $\ourF_3 = \ModMD$, as shown in Section~\ref{sec-chop}.
	The general indifferentiability composition theorem~\cite{TCC:MauRenHol04} states that for any simulator $\simulator$ and adversary $\advA_3$, there exist distinguisher $\advD$ and adversary $\advA_4$ such that
	\[\ufAdv{\jDS,\ourF_3}{\advA_3}  \leq \ufAdv{\jDS,\construct{R}}{\advA_4} + \genAdv{\indiff}{\ourF_3, \simulator}{\advD}.\]

	Let $\simulator$ be the simulator whose existence is implied by Theorem~\ref{th-md-indiff}.
	The distinguisher runs the unforgeability game for its adversary, replacing $\ourF_3[\HASH]$ in scheme algorithms and adversarial $\HASH$ queries with its $\PrivO$ and $\PubO$ oracles respectively.
	It makes $\QueriesD{\HASH}{\advA_3}$ queries to $\PubO$ and $\Queries{\SignO}{\advA_3}$ queries to $\PrivO$, and the maximum length of any query to $\PrivO$ is $b \cdot (\ell-1)$ bits
	because each element of group $\G_{\Prime}$ is a $k$-bit string (c.f. Section~\ref{sec-prelims}).
 	We apply Theorem~\ref{th-md-indiff} to obtain the bound 
	\[\genAdv{\indiff}{\ourF_3, \simulator}{\advD} \leq 2(
	\QueriesD{\HASH}{\advA_3}+ \ell \Queries{\SignO}{\advA_3}) \epsilon
	+ \frac{(\QueriesD{\HASH}{\advA_3}+ \ell \Queries{\SignO}{\advA_3})^2}{2^{2k}}+\frac{ \QueriesD{\HASH}{\advA_3}\cdot \ell \Queries{\SignO}{\advA_3}}{\gamma}.\]
	
	Adversary $\advA_4$ is a wrapper for $\advA_3$, which answers all of its queries to $\HASH$ by running $\simulator$ with access to its own $\HASH$ oracle; since the simulator runs in
	constant time and makes only one query to its oracle, the runtime and query complexity approximately equal those of $\advA_3$.
	
	Substituting $\frac{1}{\gamma} \geq \frac{\Prime}{2^{2k}}$ and $\epsilon = \frac{\Prime}{2^{2k}}$ gives the bound.\qed
\end{proof}
	
\heading{Security of $\EdDSA$ with MD.} We now want to conclude security of $\EdDSA$, with an MD-hash function, assuming security of $\Schnorr$ with a monolithic random oracle.  The Theorem is for a general prime $\Prime$ in the range $2^{k} > \Prime > 2^{k-5}$ but in $\EdDSA$ the prime is $2^{k-4} < \Prime < 2^{k-3}$ so the value of $\delta$ below is $\delta = 2^{k-5}/\Prime > 2^{k-5}/2^{k-3} = 1/4$, so the factor $1/\delta$ is $\leq 4$. Again recall our convention that query counts of an adversary include those made by oracles in its game, implying for example that $\Queries{\HASH}{\advA} \geq \Queries{\SignO}{\advA}$.

\begin{theorem}\label{th-eddsa-1} Let $\DS=\SchSig$ be the $\Schnorr$ signature scheme of Figure~\ref{fig-schnorr}. Let $\CFEDSA\Colon\allowbreak\bits^k\allowbreak\to\allowbreak\Z_{\Prime}$ be the clamping function of Figure~\ref{fig-eddsa}. Assume $2^{k} > \Prime > 2^{k-5}$ and let $\delta = 2^{k-5}/\Prime$. Let $\fDS = \DRTransform[\DS,\CFEDSA]$ be the $\EdDSA$ signature scheme. 
Let $\construct{R}\Colon\AllFuncs(\bits^*,\Z_{\Prime})\to \AllFuncs(\bits^*,\Z_{\Prime})$ be the identity functor.  
Let $\ourF$ be the functor of Figure~\ref{fig-our-functor}. Let $\advA$ be an adversary attacking the $\UFCMA$ security of $\fDS$. Again let $ b \cdot (\ell-1) - 2k$ be the maximum length in bits of a message input to $\SignO$. Then there is an adversary $\advB$ such that
\begin{align*}
	\ufAdv{\fDS,\ourF}{\advA} \leq & 
	 (1/\delta)\cdot\ufAdv{\DS,\construct{R}}{\advB} 
	+ \frac{ % 1 + 2\ell \cdot \Queries{\SignOO}{\advA} + \Queries{\HASH}{\advA}}{2^k}
	\Queries{\HASH}{\advA}}{2^{k-1}}
	+ \frac{\Prime (\QueriesD{\HASH}{\advA}+ \ell \Queries{\SignO}{\advA})}{2^{2k-2}}\\
	& + \frac{(\QueriesD{\HASH}{\advA} + \ell \Queries{\SignO}{\advA_2})^2}{2^{2k-1}}
	+ \frac{\Prime\QueriesD{\HASH}{\advA} \cdot \ell\Queries{\SignO}{\advA}}{2^{2k-1}} 
	\;.
\end{align*}
Adversary $\advB$ preserves the queries and running time of $\advA$. 
\end{theorem}
\begin{proof}[Theorem~\ref{th-eddsa-1}] Let $\jDS = \SchSigCl{\CFEDSA}$. 
By Theorem~\ref{th-dd}, we have 
	$$\ufAdv{\fDS,\ourF}{\advA} \leq   \prgAdv{\ourF_1}{\advA_1} + \prfAdv{\ourF_2}{\advA_2} + \ufAdv{\jDS,\ourF_3}{\advA_3}.$$
Now applying Lemma~\ref{lm-ourF1}, we have  
	$$\prgAdv{\ourF_1}{\advA_1} \leq \frac{
	% 1 + 2\ell \cdot \Queries{\SignOO}{\advA} + 
	\Queries{\HASH}{\advA}}{2^k} \;.$$ 
Applying Lemma~\ref{lm-ourF2}, we have
	\begin{align*}
		\prfAdv{\ourF_2}{\advA_2}  \leq& \frac{\Queries{\HASH}{\advA_2}}{2^k}
		+ \frac{2\Prime (\QueriesD{\HASH}{\advA_2}+ \ell \Queries{\FUNCO}{\advA_2})}{2^{2k}}
		+ \frac{(\QueriesD{\HASH}{\advA_2} + \ell \Queries{\FUNCO}{\advA_2})^2}{2^{2k}}
		+ \frac{\Prime\QueriesD{\HASH}{\advA_2} \cdot \ell\Queries{\FUNCO}{\advA_2}}{2^{2k}} 
		.
	\end{align*}
	We substitute $\Queries{\HASH}{\advA_2} = \Queries{\HASH}{\advA}$, $\QueriesD{\HASH}{\advA_2} = \QueriesD{\HASH}{\advA}$  and
	$\Queries{\FUNCO}{\advA_2} = \Queries{\SignO}{\advA}$.
	By Lemma~\ref{lm-ourF3} we obtain
	\begin{align*}
	\ufAdv{\jDS,\ourF_3}{\advA_3} \leq& \ufAdv{\jDS,\construct{R}}{\advB} + \frac{2\Prime(\Queries{\HASH}{\advA_3}+ \ell \Queries{\SignO}{\advA_3})}{2^{2k}} \\
	 	& + \frac{(\Queries{\HASH}{\advA_3}+ \ell \Queries{\SignO}{\advA_3})^2}{2^{2k}}+ \frac{\Prime \Queries{\HASH}{\advA_3}\cdot\ell \Queries{\SignO}{\advA_3}}{2^{2k}} \;.
	\end{align*}
	Recall that adversary $\advA_3$ has the same query complexity as $\advA$.

Under the assumption $\Prime > 2^{k-5}$ made in the theorem, 
	BCJZ~\cite{SP:BCJZ21} established that $|\Img(\CFEDSA)|=2^{k-5}$. So $|\Img(\CFEDSA)|/|\Z_{\Prime}| = 2^{k-5}/\Prime = \delta$. So by Theorem~\ref{th-jc} we have 
\begin{align}
	\ufAdv{\jDS,\construct{R}}{\advB} &\leq (1/\delta)\cdot\ufAdv{\DS,\construct{R}}{\advB} \;.
\end{align}
By substituting with the number of queries made by $\advA$ as in Theorem~\ref{th-dd} and collecting terms, we obtain the claimed bound stated in Theorem~\ref{th-eddsa-1}. \qed
\end{proof}

%\heading{Handling secret-key clamping.}
%As a final step, therefore, we give a simple reduction from the ``clamped'' scheme $\SchSigCl{\CFEDSA}$ to the original Schnorr scheme. 
%
%\begin{lemma}\label{lm-clamp}
%	Let $\DS = \SchSigCl{\CF}$ where $\CF\Colon\bits^k\to\Z_p$ is the clamping function defined at the bottom of Figure~\ref{fig-eddsa}. Let $\construct{R}\Colon\AllFuncs(\bits^*,\Z_{\Prime})\to \AllFuncs(\bits^*,\Z_{\Prime})$ be the identity functor.  Let $\advA_4$ be any adversary attacking the $\UFCMA$ security of $\DS$.
%	Then
%\begin{align}
%	\ufAdv{\DS, \construct{R}}{\advA_4} & \leq 16 \cdot \ufAdv{\SchSig, \construct{R}}{\advA_4} \;.
%\end{align}
%\end{lemma}
%\begin{proof}[Lemma~\ref{lm-clamp}] 
%	The proof is a direct result of applying Theorem~\ref{th-jc}. 
%	BCJZ established in~\cite{SP:BCJZ21} that there are $2^{k-5}$ valid clamped secret keys for $\EdDSA$, so $\delta = |\Img(\CF)|/|\keySet| = \frac{1}{16}$ and we have the claimed bound.\qed
%\end{proof}
%

We can now obtain security of $\EdDSA$ under number-theoretic assumptions via known results on the security of $\Schnorr$. Namely, we use the known results to bound $\ufAdv{\DS,\construct{R}}{\advB}$ above. From~\cite{JC:PoiSte00,EC:AABN02} we can get a bound and proof based on the DL problems, and from~\cite{C:RotSeg21} with a better bound. We can also get an almost tight bound under the MBDL assumption via~\cite{INDOCRYPT:BelDai20} and a tight bound in the AGM via~\cite{EC:FucPloSeu20}. 





\section{Indifferentiability of the shrink-MD class of functors}
\label{sec-chop}
\headingu{Indifferentiability}
We want the tuple of functions returned by a functor $\construct{F}: \FuncSp{SS} \to \FuncSp{ES}$ to be able to ``replace" a tuple drawn directly from $\FuncSp{ES}$. 
Indifferentiability is a way of defining what this means.  
We adapt the original MRH definition of indifferentiability~\cite{TCC:MauRenHol04} to our game-based model in Figure~\ref{fig-gm-indiff}.
In this game, $\simulator$ is a simulator algorithm. 
The advantage of an adversary $\advA$ against the indifferentiability of functor $\construct{F}$ with respect to simulator $\simulator$ is defined to be 
\[\genAdv{\indiff}{\construct{F},\simulator}{\advA} := 2\Pr[\Gindiff_{\construct{F},\simulator}(\advA) \Rightarrow 1] - 1.\]

\begin{figure}
	
	\twoCols{0.45}{0.45}{
		\ExperimentHeader{Game $\Gindiff_{\construct{F}, \simulator}$}
		
		\begin{algorithm}{$\Initialize()$}
			\item $c \getsr \bits$
			\item $\hh \getsr \FuncSp{SS}$
			\item $\HH \getsr \FuncSp{ES}$
		\end{algorithm}
		\ExptSepSpace
			\begin{algorithm}{$\PubO(i,\YEDSA)$}
			\item if $c = 0$ then
			\item \quad return $\simulator[\HH](i, \YEDSA)$
			\item else return $\hh(i,\YEDSA)$
		\end{algorithm}
	
	}
	{	\ExptSepSpace
		\begin{algorithm}{$\PrivO(i,\XEDSA)$}
			\item if $c = 0$ then return $\HH(i, \XEDSA)$
			\item else return $\construct{F}[\hh](i, \XEDSA)$
		\end{algorithm}	
		\ExptSepSpace
		\begin{algorithm}{$\Finalize(c')$}
			\item return $[[c = c']]$
		\end{algorithm}
	}
	\vspace{5pt}
	\caption{The game  $\Gindiff_{\construct{F}, \simulator}$ measuring indifferentiability of a functor $\construct{F}$ with respect to simulator $\simulator$.}
	\label{fig-gm-indiff}
\end{figure}



\headingu{Modifying the Merkle-Damg{\aa}rd Transform}
Coron et al. showed that the Merkle-Damg{\aa}rd transform is not indifferentiable with respect to
any efficient simulator due to its susceptibility to length-extension attacks~\cite{C:CDMP05}.
In the same work, they analysed the indifferentiability of several closely related indifferentiable constructions, including the ``chop-MD'' construction.
Chop-MD is a functor with the same domain as the MD transform; it simply truncates a specified number of bits from the output of MD.
The $\ourF_3$ functor of Figure~\ref{fig-our-functor} operates similarly to the chop-MD functor, except that $\ourF_3$ reduces the output modulo a prime $\Prime$ instead of truncating.
This small change introduces some bias into the resulting construction that affects its indifferentiability due to the fact that the outputs of the MD transform, which are $2k$-bit strings, are not distributed uniformly over $\Z_{\Prime}$.

In this section, we establish indifferentiability for a general class of functors that includes both chop-MD and $\ourF_3$.
We rely on the indifferentiability of $\ourF_3$ in Section~\ref{sec-schemes} as a stepping-stone to the unforgeability of $\EdDSA$;
however, we think our proof for chop-MD is of independent interest and improves upon prior work.

The original analysis of the chop-MD construction~\cite{C:CDMP05} was set in the ideal cipher model and
accounted for some of the structure of the underlying compression function. A later proof by Fischlin and 
Mittelbach~\cite{hfrobook} adapts the proof strategy to the simpler construction we address here and works in the random
oracle model as we do. Both proofs, however, contain a subtle gap in the way they use their simulators.

At a high level, both proofs define stateful simulators $\simulator$ which simulate a random compression function by
sampling uniform answers to some queries and programming others with the help of their random oracles.
These simulators are not perfect, and fail with some probability that the proofs bound.
In the ideal indifferentiability game, the $\PubO$ oracle answers queries using the simulator and the $\PrivO$ oracle
answers queries using a random oracle.
Both proofs at some point replace the random oracle $\HH$ in $\PrivO$ with $\ChopMD[\simulator]$ and claim that
because $\ChopMD[\simulator[\HH]](X)$ will always return $\HH(X)$ if the simulator does not fail, the adversary cannot detect the change.
This argument is not quite true, because the additional queries to $\simulator$ made by the $\PrivO$ oracle can affect its
internal state and prevent the simulator from failing when it would have in the previous game.
In our proof, we avoid this issue with a novel simulator with \textit{two internal states} to enforce separation between
$\PrivO$ and $\PubO$ queries that both run the simulator.

Our result establishes indifferentiability for all members of the $\ShrinkMD$ class of functors, which
includes any functor built by composing of the MD transform with a function $\Out: \bits^{2k} \to \S$ that satisfies three conditions, namely that for some $\gamma, \epsilon \geq 0$,
\begin{enumerate}
	\item For all $y \in \S$, we can efficiently sample from the uniform distribution on the preimage set $\{\Out^{-1}(y)\}$. We permit the sampling algorithm to fail with probability at most $\epsilon$, but require that upon failure the algorithm outputs a (not necessarily random) element of $\{\Out^{-1}(y)\}$.
	\item For all $y \in \S$, it holds that $\gamma \leq |\{\Out^{-1}(y)\}|$.
	\item The statistical distance $\delta(D)$ between the distribution
	\[D:= z \getsr \Out^{-1}(y) \colon y \getsr \S\] 
	and the uniform distribution on $\bits^{2k}$ is bounded above by $\epsilon$.
\end{enumerate}
In principle, we wish $\gamma$ to be large and $\epsilon$ to be small; if this is so,
then the set $\S$ will be substantially smaller than $\bits^{2k}$ and the function $\Out$
``shrinks'' its domain by mapping it onto a smaller set.

Both chop-MD and mod-MD are members of the $\ShrinkMD$ class of functors;
we briefly show the functions that perform bit truncation and modular reduction by a prime satisfy
our three conditions.
Truncation by any number of bits trivially satisfies condition (1) with $\epsilon = 0$.

Reduction modulo $\Prime$ also satisfies condition (1) because the following algorithm samples from the equivalence class of $x$ modulo $\Prime$ with failure probability at most $\frac{\Prime}{2^{2k}}$.
Let $\ell$ be the smallest integer
such that $\ell > \frac{2^{2k}}{\Prime}$. Sample $w \getsr [0 \ldots \ell-1]$ and output $w \cdot \Prime + x$, or $x$ if $w  \cdot \Prime + x> 2^{2k}$.
We say this algorithm ``fails'' in the latter case, which occurs with probability at most $\frac{1}{\ell} < \frac{\Prime}{2^{2k}}$.
In the event the algorithm does not fail, it outputs a uniform element of the equivalence class of $x$.

Bellare~et al. showed that the truncation of $n$ trailing bits satisfies condition (2) for $\gamma = 2^{2k-n}$ and reduction modulo prime $\Prime$ satisfies (2) for 
$\gamma = \lfloor {2^{2k}}/\Prime \rfloor$ .
It is clear that sampling from the preimages of a random $2k-n$-bit string under $n$-bit truncation produces a uniform $2k$-bit string, so truncation satisfies condition (3) with $\epsilon = 0$.
Also from Bellare~et al.~\cite{EC:BelBerTes16}, we have that the statistical distance between a uniform element of $\Z_{\Prime}$ and the modular reduction of a uniform $2k$-bit string is $\epsilon = \frac{\Prime}{2^{2k}}$.
The statistical distance of our distribution $z \getsr \Out^{-1}(Y)$ for uniform $Y$ over $\S$ from the uniform distribution over $\bits^{2k}$ is bounded above by the same $\epsilon$; hence condition (3) holds.

Given a set $\S$ and a function $\Out: \bits^{2k} \to \S$, we define the functor $\construct{F}_{\S, \Out}$ as
the composition of $\Out$ with $\construct{MD}$. In other words, for any $x \in \bits^*$ and
$\hh \in \AllFuncs(\bits^{b+2k}, \bits^{2k})$, let $\construct{F}_{\S, \Out}[\hh](x) := \Out(\construct{MD}[\hh](x))$.

\begin{theorem}
	\label{th-md-indiff}
	Let $k$ be an integer and $\S$ a set of bitstrings. Let $\Out:\bits^{2k} \to S$ be a function satisfying conditions (1), (2), and (3) above with respect to $\gamma, \epsilon > 0$. Let $\construct{MD}$ be the Merkle-Damg{\aa}rd functor(c.f. Section~\ref{sec-prelims}) $\construct{F}_{\S, \Out}:= \Out \circ \construct{MD}$ be the functor described in the prior paragraph.
	Let $\padF$ be the padding function used by $\construct{MD}$, and let $\UnPadF$ be the function that removes padding from its input (i.e., for all $\XEDSA \in \bits^*$, it holds that $\UnPadF(\XEDSA \concat \padF(|\XEDSA|)) = \XEDSA$). Assume that $\UnPadF$ returns $\bot$
	if its input is incorrectly padded and that $\UnPadF$ is injective on its support.
	Then there exists a simulator $\simulator$ such that for any adversary $\advA$ making $\PrivO$ queries of maximum length $b \cdot (\ell-1)$ bits then
	\[\genAdv{\indiff}{\construct{F},\simulator}{\advA} \leq 
	 2(\Queries{\PubO}{\advA}+ \ell \Queries{\PrivO}{\advA}) \epsilon
	 + \frac{(\Queries{\PubO}{\advA} + \ell \Queries{\PrivO}{\advA})^2}{2^{2k}} 
	 + \frac{ \Queries{\PubO}{\advA} \cdot \ell \Queries{\PrivO}{\advA}}{\gamma}.\]
\end{theorem}
\begin{proof}[Theorem~\ref{th-md-indiff}]
We first give a brief overview of our proof strategy
and its differences from previous indifferentiability proofs for the chop-MD construction~\cite{C:CDMP05,hfrobook}.

Our simulator, $\simulator$, is defined in Figure~\ref{fig-chop-sim}. 
It is inspired by, but distinct from, that of Mittelbach and Fischlin's simulator for the chop-MD construction (~\cite{hfrobook} Figure 17.4.),
	which in turn adapts the simulator of Coron et al~\cite{C:CDMP05} from the ideal cipher model to the random oracle model.
	These simulators all present the interface of a random compression function $\hh$ and internally maintain a graph
	in which each edge represents an input-output pair under the simulated compression function.
	The intention is that each path through this graph will represent a possible evaluation of $\construct{F}_{\S, \Out}[\hh]$.
	The fundamental difference between our simulator and previous ones is that we maintain two internal graphs instead of one: one graph for all queries,
	and one graph for public interface queries only.
	This novel method of using two graphs avoids the gap in prior proofs described above by tracking precisely which
	parts of the simulator's state are influenced by private and public interface queries respectively.

% In our proof, we transform the ideal indifferentiability game by evaluating our functor $\construct{F}$ in each
% 	query to the $\PrivO$ oracle.
% 	Initially, we discard the output of this evaluation and use a separate graph in our simulator so that
% 	these additional queries do not influence the $\PubO$ oracle.
% 	In later games, we bound the probability that the private queries influence the public graph in a way
% 	that is detectable by the adversary (such as creating collisions, cycles, or duplicate edges in the public
% 	simulator's graph), and begin using the same graph for both types of query.
% 	We also claim that if the graph is free of collisions, cycles, and duplicate edges, then we
% 	can respond to $\PrivO$ queries with the evaluation of $\construct{F}$ without detection.
% 	We then use the statistical closeness of sampling a random preimage of a random element (property (3) of $\Out$)
% 	to argue that our simulator is honestly behaving as a random oracle except with some small probability.
% 	The resulting game is then equivalent to the real indifferentiability game, and the theorem follows by collecting
% 	the bounded differences between each pair of adjacent games.

\begin{figure}
		
	\twoCols{0.44}{0.44}{
	\ExperimentHeader{Simulator $\simulator[\HH](\YEDSA, \graph)$}
	\begin{algorithm}{}
		\item $(y, m) \gets \YEDSA$
		\item if $\exists z$ such that $(y, z, m) \in \graph.\edges$
		\item \quad return $z$
		\item $\msg \gets \graph.\ppathO(\IV, y)$
		\item if $\msg \neq \bot$ and $\UnPadF(\msg \concat m) \neq \bot$ then 
		\item \quad if $\Thh[\YEDSA, \msg] \neq \bot$ then $z \gets \Thh[\YEDSA, \msg]$
		\item \quad else  $z \getsr  \Out^{-1}(\HH(\UnPadF(\msg \concat m)))$
		\item \qquad $\Thh[\YEDSA, \msg] \gets z$
		\item else if $\Thh[\YEDSA] \neq \bot$ then $z \gets \Thh[\YEDSA]$
		\item else $z \getsr \bits^{2k}$; $\Thh[\YEDSA] \gets z$
		\item add $(y, z, m)$ to $\graph.\edges$
		\item add $(y, z, m)$ to $\graph_{\all}.\edges$
		\item return $z$
	\end{algorithm}
}
{
		\ExperimentHeader{Game $\Gm_0:=\Gindiff_{\construct{F}, \simulator} | b = 0$}

		\begin{algorithm}{$\Initialize()$}
			\item $\HH \getsr \AllFuncs{\bits^*, \S}$
			\item $\graph_{\all}, \graph_{\public} \gets (\IV)$
		\end{algorithm}
		\ExptSepSpace
		\begin{algorithm}{$\PrivO(\XEDSA)$}
			\item return $\HH(\XEDSA)$
		\end{algorithm}	
			\ExptSepSpace
		\begin{algorithm}{$\PubO(\YEDSA)$}
			\item $z \gets \simulator[\HH](\YEDSA, \graph_{\public})$
			\item return $z$
		\end{algorithm}
		\ExptSepSpace
		\begin{algorithm}{$\Finalize(c')$}
			\item return $c'$
		\end{algorithm}
	}
	\vspace{5pt}
	\caption{Left: Indifferentiability simulator for the proof of Theorem~\ref{th-md-indiff}. Right: The ideal game  $\Gindiff_{\construct{F}, \simulator}$ measuring indifferentiability of a functor $\construct{F}$ with respect to simulator $\simulator$}
	\label{fig-chop-gm0}
	\label{fig-chop-sim}
\end{figure}

In the ``ideal'' indifferentiability game, $\PrivO$ queries are answered by random oracle $\HH\getsr \AllFuncs{\bits^*,\S}$.
$\PubO$ queries are answered by the simulator $\simulator$, which maintains the two graphs $\graph_{\public}$ and $\graph_{\all}$.
We present pseudocode for this game ($\Gm_0$) in Figure~\ref{fig-chop-gm0}.
In each graph, the nodes and edges are labeled with $2k$-bit strings.
An edge from node $y$ to node $z$ with label $m$ is denoted $(y, z, m)$, and represents a single value of the simulated compression function; namely, on $6k$-bit input $y \concat m$, the simulated compression function should output $z$.
Queries made in the process of evaluating $\construct{MD}[\S]$ will form a path that begins at the node labeled with the initialization vector $\IV$; the path's edges will be labeled with the $4k$-bit blocks of $\padF(\msg)$.
 
Whenever the simulator receives a fresh query $(y, m)$, it uses a pathfinding algorithm $\ppathO$ to check whether the query extends an existing path from $\IV$ and thus continues an
existing evaluation of the MD transform.
If so, it reads the message from the path's edge labels then appends the new block $m$ to the end.
If the result is a properly padded message, the simulator removes the padding and uses its oracle $\HH$ to compute the output of functor $\construct{F}$ on the original message.
This output $w$ is an element of $\S$, and it should be consistent with $\Out$ when applied to the $2k$-bit simulator output.
The simulator therefore samples its response from the preimages of $w$ under $\Out$.
If any of these steps fail, then the query does not need to be programmed, so the simulator samples a uniformly random response $z$ and updates its graph with the new edge from $y$.
Because we are attempting to simulate a random function, the simulator must cache its responses to maintain consistency between repeated queries.
It does this in two ways: via the graphs and via table $\Thh$. We require two forms of caching because the simulator may use two graphs and thus responses may not be cached consistently
between private and public queries in the graphs alone.

Our $\Gm_0$ differs from this ideal indifferentiability game only in the $\Finalize$ oracle, which returns the adversary's challenge guess $c'$. 
Thus the probability that game $\Gm_0$ returns $1$ exactly equals $1 - \Pr[\Gindiff_{\construct{F}, \simulator}(\advA) | c = 0]$.

We move to $\Gm_1$, where the $\PrivO$ oracle uses $\simulator$ to calculate the output
of functor $\construct{F}$, then discards the result.
We wish for the adversary's view of games $\Gm_0$ and $\Gm_1$ to be identical, so
we must ensure that the additional queries to $\simulator$ do not influence its state
or its responses to $\PubO$ queries.
We therefore call the simulator with different graphs in the two oracles.
It responds to public queries based only on the public graph, and queries made by $\PrivO$ are private and do not update the public graph.
We do use shared table $\Thh$ to cache outputs across all queries; in this sense a 
private query can affect a public query; however, we cache responses separately for each branch of
the simulator, so our caching does not alter the simulator's branching behavior and the distribution of public queries' responses does not change.
The adversary cannot detect at what time a response $z$ is first sampled, so its
view does not change, and
\[\Pr[\Gm_0] = \Pr[\Gm_1].\]
\begin{figure}
	
	\twoCols{0.44}{0.47}{
		\ExperimentHeader{Game $\Gm_1$}
		
		\begin{algorithm}{$\Initialize()$}
			\item $\HH \getsr \AllFuncs{\bits^*, \S}$
			\item $\graph_{\all}, \graph_{\public} \gets (\IV)$
		\end{algorithm}
		\ExptSepSpace
		\begin{algorithm}{$\PubO(\YEDSA)$}
			\item $z \gets \simulator[\HH](\YEDSA, \graph_{\public})$
			\item return $z$
		\end{algorithm}
	}{
		\ExptSepSpace
		\begin{algorithm}{$\PrivO(\XEDSA)$}
			\item\gamechange{$w \gets \construct{F}[\simulator[\HH](\cdot, \graph_{\all})](\XEDSA)$}
			\item return $\HH(\XEDSA)$
		\end{algorithm}	
		\begin{algorithm}{$\Finalize(c')$}
			\item return $c'$
		\end{algorithm}	
	}
	\vspace{5pt}
	\caption{Game $\Gm_1$ in the proof of Theorem~\ref{th-md-indiff}.  Highlighted code is changed from the previous game, and algorithms not shown are unchanged from the previous game.}
	\label{fig-chop-gm1}
\end{figure}
In game $\Gm_2$, we set a $\bad$ flag if the simulator if $\graph_{\all}$ contains
any collisions, cycles, or ``duplicate'' edges: edges with the same starting node and label but different ending nodes.

Collisions and cycles are formed only when a new edge is created whose ending node is already present in the graph;
we set $\bad$ in this case.
The caching in line $2$ prevents duplicate edges except when the $\PrivO$ and $\PubO$ oracles query the simulator on the same input $(y, m)$, in that order.
Even in this case, caching in table $\Thh$ prevents duplicate edges unless one query detects a path that the other did not, or the two queries detect different paths.

If the $\PubO$ query detects a path to node $y$ that did not exist during the previous $\PrivO$ query, or there are two distinct paths to $y$ in $\graph_{\all}$, then $\graph_{\all}$ must contain a collision or a cycle, and the $\bad$ flag will be set when that is detected.
Furthermore, $\graph_{\public}$ is a subgraph of $\graph_{\all}$, so it cannot contain a path to $y$ that $\graph_{\all}$ does not.
To catch the formation of duplicate edges, it is therefore sufficient to set $\bad$ if $\graph_{\all}$ contains a path from $\IV$ to $y$ that is not detected by the subsequent $\PubO$ query.

The $\bad$ flag is internal and does not affect the view of the game, so
\[\Pr[\Gm_2] = \Pr[\Gm_1] \]

In $\Gm_3$, we force the adversary to lose when the $\bad$ flag is set.
This strictly decreases their advantage, so
\[ \Pr[\Gm_3] \leq \Pr[\Gm_2]. \]

\begin{figure}
	\twoCols{0.34}{0.46}{
		\ExperimentHeader{Game $\Gm_2$, \fbox{$\Gm_3$}}

		\ExptSepSpace
		\begin{algorithm}{$\Finalize(c')$}
			\item \gamechange{\fbox{if $\bad$ then return $0$}}
			\item return $c'$
		\end{algorithm}	
		
		\hrulefill

		\ExptSepSpace
			\ExperimentHeader{Game $\Gm_4$}

			\ExptSepSpace
			\begin{algorithm}{$\PrivO(\XEDSA)$}
				\item$w \gets \construct{F}[\simulator[\HH](\cdot, \graph_{\all})](\XEDSA)$
				\item \gamechange{ return $w$}
			\end{algorithm}	
		}{
		\begin{algorithm}{$\simulator[\HH](\YEDSA, \graph)$}
			\item $(y, m) \gets \YEDSA$
			\item if $\exists z$ such that $(y, z, m) \in \graph.\edges$
			\item \quad return $z$
			\item $\msg \gets \graph.\ppathO(\IV, y)$
			\item \gamechange{$\msg_{\all} \gets \graph_{\all}.\ppathO(\IV, y)$}
			\item if $\msg \neq \bot$ and $\UnPadF(\msg \concat m) \neq \bot$ then 
			\item \quad if $\Thh[\YEDSA, \msg] \neq \bot$ then $z \gets \Thh[\YEDSA, \msg]$
			\item \quad else  $z \getsr  \Out^{-1}(\HH(\UnPadF(\msg \concat m)))$
			\item \qquad $\Thh[\YEDSA, \msg] \gets z$
			\item else if $\Thh[\YEDSA] \neq \bot$ then $z \gets \Thh[\YEDSA]$
			\item else $z \getsr \bits^{2k}$; $\Thh[\YEDSA] \gets z$
			\item \gamechange{if ($z \in \graph_{\all}.\nodes$ and $(y, z, m) \not\in \graph_{\all}.\edges$)}
			\item \quad \gamechange{ or $\msg \neq \msg_{\all}$}
			\item \qquad \gamechange{$\bad \gets \true$}
			\item add $(y, z, m)$ to $\graph.\edges$
			\item add $(y, z, m)$ to $\graph_{\all}.\edges$
			\item return $z$
		\end{algorithm}
	}
	\vspace{5pt}
	\caption{Games $\Gm_2$, $\Gm_3$, and $\Gm_4$ in the proof of Theorem~\ref{th-md-indiff}. Highlighted code is changed from the previous game, and boxed code is present only in $\Gm_3$ (and subsequent games). Algorithms not shown are unchanged from the previous game.}
	\label{fig-chop-gm2}
	\label{fig-chop-gm3}
	\label{fig-chop-gm4}
\end{figure}
In our next game, we stop querying $\HH$ directly in the $\PrivO$ oracle and instead return
$w$, the result of our functor on the query.
We claim that in $\Gm_3$, either $w = \HH(\XEDSA)$ or $\bad = \true$; thus if the adversary wins $\Gm_3$, then in all $\PrivO$ queries we have $w = \HH(\XEDSA)$. 
From this claim, we can see that the change does not affect the view of the adversary and 
\[\Pr[\Gm_4] = \Pr[\Gm_3].\]

To prove the claim, consider a query $\PrivO(\XEDSA)$. Let $(\XEDSA_1, \ldots, \XEDSA_n)$ be the $b$-bit blocks of $\XEDSA \concat \padF(|\XEDSA|)$.
By the definition of the MD transform, $\PrivO$ makes $n$ queries to $\simulator$ of the form $(y_i, \XEDSA_i), \graph_{\all}$, where $y_1 = \IV$ and $y_i = \simulator((y_{i-1}, \XEDSA_{i-1}), \graph_{\all})$ for all $i > 1$.
These may not be fresh queries, but they must be made in order or $\bad$ will be set:
if query $\simulator((y_i, \XEDSA_i))$ outputs $y_{i+1}$ and this has already been the input of a prior query, then $y_{i+1}$ is a node in $\graph_{\all}$; a collision has occurred and the query will set $\bad$.
Unless $\bad$ is set, there exists exactly one path in $\graph_{\all}$ from $\IV$ to $y_i$, and the labels on this path are $(\XEDSA_1, \ldots, \XEDSA_{i-1})$.
This is trivially true for $i = 1$; the path is the empty path.
The query $\simulator((y_{i-1}, \XEDSA_{i-1}), \graph_{\all})$ creates the edge $(y_{i-1}, y_i, \XEDSA_{i-1})$ in $\graph_{\all}$.
By induction on $i$, there is always a path from $\IV$ to $y_i$ with labels $(\XEDSA_1, \ldots, \XEDSA_{i-1})$.
If there exists more than one path from $\IV$ to $y_i$, then $\graph_{\all}$ must contain either a cycle or two edges with the same ending node; in either case the $\bad$ flag will be set.

Therefore, when $\PrivO$ first makes the query $\simulator((y_{n-1}, \XEDSA_n), \graph_{all})$, it will detect the path, compute $\UnPadF(\msg \concat \XEDSA_n) = \XEDSA$ and output an element $z \in \Out^{-1}(\HH(\XEDSA))$. By the definition of $\Out^{-1}$, we have $w = \Out(z) = \HH(\XEDSA)$, so the claim holds.

At this point, the adversary can no longer directly query random oracle $\HH$, so
we allow the simulator to lazily sample the function.
Also in this game, the simulator queries $\HH$ on the path from $\IV$ to $y$ in $\graph_{\all}$ for all queries, not just private queries.  
If the path in $\graph$ is different from the path in $\graph_{\public}$, then the $\bad$ flag
will be set and the adversary will lose anyway.
Therefore the view in any winning game is unchanged, and
\[ \Pr[\Gm_{4}] = \Pr[\Gm_5]. \]

\begin{figure}
	
	\twoCols{0.46}{0.45}{
\ExperimentHeader{Game $\Gm_5$}

\ExptSepSpace
	\begin{algorithm}{$\simulator(\YEDSA, \graph)$}
			\item $(y, m) \gets \YEDSA$
			\item if $\exists z$ such that $(y, z, m) \in \graph.\edges$
			\item \quad return $z$
			\item $\msg \gets \graph.\ppathO(\IV, y)$
			\item $\msg_{\all} \gets \graph_{\all}.\ppathO(\IV, y)$
			\item \gamechange{if $\msg_{\all} \neq \bot$}
			\item[] \qquad \gamechange{ and $\UnPadF(\msg_{\all} \concat m) \neq \bot$ then} 
			\item \quad \gamechange{if $\Thh[\YEDSA, \msg_{\all}] \neq \bot$ then}
			\item \qquad \gamechange{ $z \gets \Thh[\YEDSA, \msg_{\all}]$}
			\item \quad else
			\item \qquad \gamechange{if $\THH[\UnPadF(\msg_{\all} \concat m)] \neq \bot$}
			\item \quad \qquad \gamechange{$y \gets \THH[\UnPadF(\msg_{\all} \concat m)]$}
			\item \qquad \gamechange{$\THH[\UnPadF(\msg_{\all} \concat m)] \gets y$}
			\item \qquad \gamechange{$z \getsr  \Out^{-1}(y)$; $\Thh[\YEDSA, \msg_{\all}] \gets z$}
			\item else if $\Thh[\YEDSA] \neq \bot$ then $z \gets \Thh[\YEDSA]$
			\item else $z \getsr \bits^{2k}$; $\Thh[\YEDSA] \gets z$
			\item if ($z \in \graph_{\all}.\nodes$ and $(y, z, m) \not\in \graph_{\all}.\edges$)
			\item \quad or $\msg \neq \msg_{\all}$
			\item \qquad $\bad \gets \true$
			\item add $(y, z, m)$ to $\graph.\edges$
			\item add $(y, z, m)$ to $\graph_{\all}.\edges$
			\item return $z$
		\end{algorithm}
	}{
	\ExperimentHeader{Game $\Gm_6$}
	\ExptSepSpace
	
		\begin{algorithm}{$\simulator(\YEDSA, \graph)$}
			\item $(y, m) \gets \YEDSA$
			\item if $\exists z$ such that $(y, z, m) \in \graph.\edges$
			\item \quad return $z$
			\item $\msg \gets \graph.\ppathO(\IV, y)$
			\item $\msg_{\all} \gets \graph_{\all}.\ppathO(\IV, y)$
			\item if $\msg_{\all} \neq \bot$ and $\UnPadF(\msg_{\all} \concat m) \neq \bot$ then
			\item \quad \gamechange{$z \getsr \bits^{2k}$}
			\item \quad \gamechange{if $\THH[\UnPadF(\msg_{\all} \concat m)] \neq \bot$}
			\item \qquad \gamechange{$z \gets \THH[\UnPadF(\msg_{\all} \concat m)]$}
			\item \quad \gamechange{$\THH[\UnPadF(\msg_{\all} \concat m)] \gets z$}
			\item else if $\Thh[\YEDSA] \neq \bot$ then $z \gets \Thh[\YEDSA]$
			\item else $z \getsr \bits^{2k}$; $\Thh[\YEDSA] \gets z$
			\item if ($z \in \graph_{\all}.\nodes$ and $(y, z, m) \not\in \graph_{\all}.\edges$)
			\item \quad or $\msg \neq \msg_{\all}$
			\item \qquad $\bad \gets \true$
			\item add $(y, z, m)$ to $\graph.\edges$
			\item add $(y, z, m)$ to $\graph_{\all}.\edges$
			\item return $z$
\end{algorithm}}
	\vspace{5pt}
	\caption{Left: Game $\Gm_5$ in the proof of Theorem~\ref{th-md-indiff}. Right:  Game $\Gm_6$ in the proof of Theorem~\ref{th-md-indiff}. Highlighted code is changed from the previous game, and algorithms not shown are unchanged from the previous game.}
	\label{fig-chop-gm5}
	\label{fig-chop-gm6}
\end{figure} 

In our next game $\Gm_6$, we replace the sampling of $z$ from the preimages of a random point $y$ with sampling a uniformly random $2k$-bit string.
The sampling will never fail to be uniform, which means the adversary can distinguish the game if it were to fail in $\Gm_5$; from condition (1) we have that the probability of failure was
at most $\epsilon$ per query.
Otherwise, we have from condition (3) on $\Out$ that the statistical distance of the distribution $(z \getsr \Out^{-1}(y) \Colon: y \getsr \S)$ from the uniform distribution on $\bits^{2k}$ is at most $\epsilon$.
By a hybrid argument over the $\Queries{\PubO}{\advA} + \ell \Queries{\PrivO}{\advA}$ queries to the simulator,
the probability that $\advA$ can distinguish $\Gm_5$ from $\Gm_6$ is bounded above by $2(\Queries{\PubO}{\advA} + \ell \Queries{\PrivO}{\advA}) \epsilon$.

Now that we are caching $z$ in table $\THH$ when the check of line $6$ holdss $\true$,
it has become redundant to cache it in table $\Thh$, so we stop doing this caching.
We must be careful since table $\THH$ is indexed by labels of the form $\UnPadF(\msg_{\all} \concat m)$
where $\Thh$ was indexed by tuples $(\YEDSA, \msg_{\all})$.
Since $\msg_{\all}$ is a path from $\IV$ to $y$ in a
graph with no duplicate edges provided $\bad$ is not set, $\msg_{\all}$ uniquely
determines its ending node $y$ and $\UnPadF(\msg_{\all}\concat m)$ uniquely determines a tuple
$((y, m), \msg_{\all})$ because $\UnPadF$ is injective.
Thus the entries of $\THH$ are in one-to-one correlation with the entries of $\Thh$, and we can safely retain only the former, and 
\[ \Pr[\Gm_6] \leq \Pr[\Gm_5] + 2(\Queries{\PubO}{\advA} + \ell \Queries{\PrivO}{\advA}) \epsilon \]

In $\Gm_7$, all queries are sampled randomly from $\bits^{2k}$ and cached in table $\Thh$ under
the input $\YEDSA$, instead of some being cached under the message $\UnPadF(\msg_{\all} \concat m)$.
We claim that in $\Gm_6$ if a query $\simulator(y, m)$ stores $z$ in $\THH[\XEDSA]$, then a later query $\simulator(y', m')$ will return $z$ if and only if $(y, m) = (y', m')$ or $\bad$ is set.
The forward direction is trivial.
If $\simulator(y', m')$ returns $\THH[\XEDSA]$, then either we have 
\[\XEDSA = \UnPadF(\graph_{\all}.\ppathO(\IV, y') \concat m') = \UnPadF(\graph_{\all}.\ppathO(\IV, y) \concat m),\]
or there was a $\bad$-setting collision between $\THH[\XEDSA]$ and the randomly-sampled response $z$.

In the former case, the function $\UnPadF$ is injective, so we know $m = m'$, and the paths from $\IV$ to $y'$ and $y'$ respectively have the same
sequence of edge labels.
Unless $\bad$ is set, there are no duplicate edges, so a starting node and sequence of edge labels uniquely identify the ending node on the path; consequently $y = y'$ and the claim follows.

Queries in $\Gm_7$ therefore hit a cache indexed by $\YEDSA$ if and only if they would hit a cache indexed by $\XEDSA$ in $\Gm_6$.
We do not need to worry that the new entries in $\Thh$ overlap with those created in line
$11$; if the check in line $6$ holds true during some query, then it cannot have been false in an earlier query with the same $\YEDSA$
unless $\bad$ would be set. Thus no queries are answered from table $\Thh$ in $\Gm_7$ that would not have been
cached in earlier games, and
\[ \Pr[\Gm_7] = \Pr[\Gm_6]. \]

Notice that both branches of the simulator now identically sample $z \getsr \bits^{2k}$ uniformly, subject to 
caching in table $\Thh$ under $\YEDSA$; in the next game we will eliminate the redundant check on $\msg_{all}$ in line $6$.

\begin{figure}
\twoCols{0.44}{0.47}{
	\ExperimentHeader{Game $\Gm_7$}
	\ExptSepSpace
	
	\begin{algorithm}{$\simulator(\YEDSA, \graph)$}
		\item $(y, m) \gets \YEDSA$
		\item if $\exists z$ such that $(y, z, m) \in \graph.\edges$
		\item \quad return $z$
		\item $\msg \gets \graph.\ppathO(\IV, y)$
		\item $\msg_{\all} \gets \graph_{\all}.\ppathO(\IV, y)$
		\item if $\msg_{\all} \neq \bot$ and $\UnPadF(\msg_{\all} \concat m) \neq \bot$ then 
		\item \quad \gamechange{$z \getsr \bits^{2k}$}
		\item \quad \gamechange{if $\Thh[\YEDSA] \neq \bot$ then $z \gets \Thh[\YEDSA]$}
		\item \quad \gamechange{$\Thh[\YEDSA] \gets z$}
		\item else if $\Thh[\YEDSA] \neq \bot$ then $z \gets \Thh[\YEDSA]$
		\item else $z \getsr \bits^{2k}$; $\Thh[\YEDSA] \gets z$
		\item if $z \in \graph_{\all}.\nodes$ or $\msg \neq \msg_{\all}$
		\item \quad $\bad \gets \true$
		\item add $(y, z, m)$ to $\graph.\edges$
		\item add $(y, z, m)$ to $\graph_{\all}.\edges$
		\item return $z$
	\end{algorithm}
}{
	\ExperimentHeader{Game $\Gm_8$}
	\ExptSepSpace
	
\begin{algorithm}{$\simulator(\YEDSA)$}
\item if $\Thh[\YEDSA] \neq \bot$ then $z \gets \Thh[\YEDSA]$
\item else $z \getsr \bits^{2k}$; $\Thh[\YEDSA] \gets z$
\item return $z$
\end{algorithm}

\ExptSepSpace
\begin{algorithm}{$\Finalize(c')$}
\item return $c'$
\end{algorithm}
}
\vspace{5pt}
\caption{Left: Game $\Gm_7$ in the proof of Theorem~\ref{th-md-indiff}. Right: Game $\Gm_8$ in the proof of Theorem~\ref{th-md-indiff}. Highlighted code is changed from the previous game, and algorithms not shown are unchanged from the previous game.}
\label{fig-chop-gm7}
\label{fig-chop-gm8}
\end{figure}

In our final game, $\Gm_8$, we remove the $\bad$ flag and the internal variables used to set it. 
This increases the adversary's advantage, since it can now win even if the game would set $\bad$.
The probability of a collision among the $\Queries{\PubO}{\advA} + \ell \Queries{\PrivO}{\advA}$ randomly sampled nodes of
$\graph_{\all}$ is at most $\frac{(\Queries{\PubO}{\advA} + \ell \Queries{\PrivO}{\advA})^2}{2^{2k}}$ by a birthday bound.
The probability that $\graph_{\all}$ contains a path to $y$ that $\graph_{\public}$ does not is the probability that the adversary $\advA$ queries $\PubO$ on one of the $\ell q_{\PrivO}$ intermediate nodes on a path in $\graph_{\all}$, before it learns the label of that node from $\PubO$.
$\advA$ may use $\PrivO$ to learn the output $y$ of $\Out$ an intermediate node, but it does not learn anything about which of the equally likely preimages of $y$ is the label; from condition (2) we have that there are at least $\gamma$ such preimages to guess from.
Then the probability that $\advA$ sets $\bad$ with a single $\PubO$ query is at most $\frac{\ell \Queries{\PrivO}{\advA}}{\gamma}$; a union bound over all $\PubO$ queries gives
that a path exists in $\graph_{\all}$ but not $\graph_{\public}$ with probability no greater than $\frac{\Queries{\PubO}{\advA} \cdot \ell \Queries{\PrivO}{\advA}}{\gamma}$.

We also stop maintaining the graphs $\graph_{\public}$ and $\graph_{\all}$, which are now only used to cache queries whose responses are already cached in table $\Thh$.
This changes nothing about the view of the adversary, so
\[ \Pr[\Gm_8] \leq \Pr[\Gm_7] + \frac{(\Queries{\PubO}{\advA} + \ell \Queries{\PrivO}{\advA})^2}{2^{2k}} + \frac{\Queries{\PubO}{\advA} \cdot \ell \Queries{\PrivO}{\advA}}{\gamma}. \]
If we look closely at $\Gm_8$, we can see that the ``simulator'' is actually just a lazily-sampled random function with domain $\bits^{6k}$ and codomain $\bits^{2k}$.
In fact, $\Gm_8$ is identical to the ``real'' indifferentiability game for functor $\construct{F}$, save for its choice of challenge bit.
Thus 
\[\Pr[\Gm_8] =  \Pr[\Gindiff_{\construct{F}, \simulator}(\advA) | c = 1]. \]
Collecting bounds across all gamehops gives the theorem.\qed
\end{proof}
\section{The unique order-$\Prime$ subgroup of $\group$}\label{apx:group}
Here, we briefly prove that our choice in Section~\ref{sec-prelims} of $\G_{\Prime}$ as the unique subgroup of order $\Prime$ of group $\G$, which has order $\Prime \cdot 2^\cofactor$, is well-defined. (We do not prove that $\G_{\Prime}$ is cyclic as this follows directly from the fact that its order is prime.)
We also give an efficient test for membership in $\G_{\Prime}$.
 
    \begin{proposition}\label{pr-group} Let $\Prime$ be an odd prime, let $2^\cofactor < \Prime$ be a positive integer, and let $\G$ be a group of order $2^\cofactor \cdot \Prime$. Then (1) the group $\G$ has a unique subgroup of order $\Prime$, and (2) For all $X\in\group$ it is the case that $X$ is in this subgroup iff $\Prime\cdot X = 0_{\group}$.
    \end{proposition}
    
    \begin{enumerate}[label=(\arabic*)]
    \item Let $n$ be the number of $\Prime$-order subgroups of $\G$. According to Sylow's theorem $n \equiv 1 \mod \Prime$. We now have two cases: either $n = 1$, or $n > 1$. We prove that $n=1$ by contradiction; therefore we assume $n > 1$. It follows that $n \geq \Prime +1$. Two distinct groups of prime order can intersect only at the identity, so each of the $n$ subgroups of $\G$ contains $\Prime-1$ unique elements. Consequently the order of $\G$ is at least $n (\Prime-1) \geq (\Prime+1)(\Prime -1)  \geq \Prime (\Prime+1)$. Since we have already defined the order of $\G$ to be $2^\cofactor \cdot \Prime$, we have that $2^\cofactor \geq \Prime+ 1$. This contradicts our initial assumption that $2^\cofactor < \Prime$; thus our assumption that $n>1$ must be false and $\G$ must have exactly one subgroup of order $\Prime$. This subgroup is $\G_{\Prime}$.
    
    \item Let $X \in \G$ be a group element and assume that $\Prime \cdot X = 0_{\G}$. This implies that the order of $X$ divides $\Prime$. Since $\Prime$ is prime, either the order of $X$ is $1$ or it is $\Prime$. In the first case, $x = 0_{\G}$. Otherwise, $X$ generates a subgroup with order $\Prime$, which by part (1) is the unique such subgroup $\G_{\Prime}$. Therefore $X$ generates $\G_{\Prime}$ and must belong to it.
    	
    For the reverse direction, assume that $X$ is in $\G^{\Prime}$. The order of $X$ must divide the order of $\G_{\Prime}$; so $X$ must either have order $\Prime$ or order $1$. In either case, $\Prime \cdot X = 0_{\G}$.
    \end{enumerate}
%\section{Group Instantiation for Ed25519\label{sec-gp-instantiation}}
Ed25519 is EdDSA with twisted Edwards
curve which is birationally equivalent to the curve Curve25519. Curve25519 is of Montgomery form with equation $v^2 = u^3 + 486662u^2 + u$ over the field $\mathbb{F}_q$, and it is birationally equivalent to the Edwards curve $x^2 + y^2 = 1 + (121665/121666)x^2y^2$ where $x = \sqrt{486664}u/v$ and $y = (u - 1)/(u + 1)$. 

In general EdDSA, the group is the set of points on an Edwards curve $E$, namely the $E(\mathbb{F}_q) = \{(x, y) \in \mathbb{F}_q \times \mathbb{F}_q : -x^2 + y^2 = 1 + dx^2y^2\}$. The Edwards curve in Ed25519 is isomorphic to $-x^2 + y^2 = 1 - (121665/121666)x^2y^2$, and so it has $d = - (121665/121666) \in \mathbb{F}_q$. Then defines $\group$ to be $E(\mathbb{F}_q) = \{(x, y) \in \mathbb{F}_q \times \mathbb{F}_q : -x^2 + y^2 = 1 - (121665/121666)x^2y^2\}$. The field size $q$ is the prime $2^{255} - 19$. Other parameters of the group descriptor include the following: $\cofactor = 3$ is the $\log_2$ of cofactor $2^{\cofactor}$; $\Prime = 2^{252} + 27742317777372353535851937790883648493$ is the odd prime order of subgroup $\group_{\Prime}$; $\generator = (x, 4/5) \in E$ is the generator of $\group_{\Prime}$, and $\Prime \cdot \generator = 0$. The exact implementation of Ed25519 also fixes $b = 256$ and $\HASH$ function to be SHA-512 such that the output of $\HASH$ is $2b$ bits. 

For an elliptic curve point $(x, y)$, the encoding function $\EncF$ encodes it as the $(b-1)$-bit little-endian encoding of $y$ followed by a sign bit of $x$. The sign bit is 1 if and only if $x$ is negative and $x$ is negative if its $(b-1)$-bit encoding is lexicographically larger than that of $-x$. The output length $\EncL$ is therefore $b = 256$. For decoding, $\DecF$ recovers $y$ immediately from its $(b-1)$-bit encoding, and then recovers $x$ via $x = \pm \sqrt{(y^2 - 1)/(dy^2 + 1)}$ and the sign bit. If the resulting point is not on the curve or if taking the square root fails, $\DecF$ returns $\bot$.
\mathversion{normal}
\chapter{Verifiable Distributed Aggregation Functions}\label{chap:vdaf}
\mathversion{normal5}
\input{vdaf/body}
\input{vdaf/appendix}
%\chapter{Figures and Such}
%This demonstrates how OGS wants figures and tables formatted. For
%figures, the caption goes below the figure and ``Figure'' is in bold.
%See Figure~\ref{fig:zen}. Tables are formatted with the caption above
%the table. See Table~\ref{tab:bad}.
%
%Of course, Table~\ref{tab:bad} looks horrible. It should probably be
%formatted like Table~\ref{tab:good} instead.
%
%For facing caption pages, see Table~\ref{tab:facing}. Of course,
%facing caption pages are vaguely ridiculous and my implementation of
%them in the class file is by far the most brittle part of the
%implementation. It's entirely possible that something has changed and
%these don't work at all. I implemented it merely for the challenge.

%\begin{figure}
%\centering
%\fbox{\parbox{.9\linewidth}{%
%	\noindent
%	{\Huge PHD ZEN}\par
%	\vskip.5in
%	\centerline{comic here}
%	\vskip.5in
%}}
%\caption[``Ph.D. Zen'']{Comic entitled ``Ph.D. Zen'' by Jorge Cham, 2005. Copyright
%has not been obtained and so it isn't displayed.}
%\label{fig:zen}
%\end{figure}
%
%\begin{table}
%\centering
%\caption[Electronic Dissertation Submission Rates]{Electronic
%Dissertation Submission Rates at UCSD, Fall 2005 and Winter 2006.
%(First two quarters that the program was available to all Ph.D.
%candidates not in a Joint Doctoral Program with SDSU.)}
%\label{tab:bad}
%\begin{tabular}{|*{5}{>{\centering\arraybackslash}m{.15\linewidth}|}}
%\hline
%&Ph.D.s awarded (Including Joint degrees) & Electronic submission of
%Dissertation & Paper Submission of Dissertation & Percentage of
%Electronic Submission\\
%\hline
%Fall\par 2005 & 84 & 37 & 47 & 44.05\%\\
%\hline
%Winter\par 2006 & 64 & 42 & 22 & 65.63\%\\
%\hline
%\end{tabular}
%\end{table}
%
%\begin{table}
%\centering
%\caption[Electronic Dissertation Submission Rates]{Electronic
%Dissertation Submission Rates at UCSD, Fall 2005 and Winter 2006.
%(First two quarters that the program was available to all Ph.D.
%candidates not in a Joint Doctoral Program with SDSU.)}
%\label{tab:good}
%\renewcommand\tabularxcolumn[1]{>{\RaggedRight\arraybackslash}p{#1}}
%\begin{tabularx}{.9\linewidth}{lcccc}
%\toprule
%&\multicolumn{1}{X}{Ph.D.s awarded (Including Joint degrees)}
%&\multicolumn{1}{X}{Electronic submission of Dissertation}
%&\multicolumn{1}{X}{Paper Submission of Dissertation}
%&\multicolumn{1}{X}{Percentage of Electronic Submission}\\
%\midrule
%Fall 2005 & 84 & 37 & 47 & 44.05\%\\
%Winter 2006 & 64 & 42 & 22 & 65.63\%\\
%\bottomrule
%\end{tabularx}
%\end{table}

%\begin{facingcaption}{table}
%\caption[UCSD Gender Distribution]{University of
%California, San Diego Gender Distribution for the Campus Population,
%October~2005\\
%(http://assp.ucsd.edu/analytical/Campus\%20Population.shtml)\\
%\emph{(This is an example of a facing caption page, the next page is
%the example of the table/figure/etc.\ that corresponds to this
%caption. It is also an example of table/figure that is rotated 90
%degrees to fit the page.)}}
%\label{tab:facing}
%\renewcommand\tabularxcolumn[1]{>{\RaggedLeft\arraybackslash}p{#1}}
%\parindent=0pt
%\setbox0=\vbox{%
%\vsize\textwidth
%\hsize\textheight
%\linewidth\hsize
%\columnwidth\hsize
%\textwidth\hsize
%\textheight\vsize
%\begin{tabularx}{\linewidth}{lXXXXXX}
%\toprule
%& \multicolumn{2}{c}{\textbf{Women}}
%& \multicolumn{2}{c}{\textbf{Men}}
%& \multicolumn{2}{c}{\textbf{Total}}\\
%\cmidrule(l){2-3}\cmidrule(l){4-5}\cmidrule(l){6-7}
%\textbf{Population Segment}
%& \multicolumn{1}{c}{\textbf{N}} & \multicolumn{1}{c}{\textbf{\%}}
%& \multicolumn{1}{c}{\textbf{N}} & \multicolumn{1}{c}{\textbf{\%}}
%& \multicolumn{1}{c}{\textbf{N}} & \multicolumn{1}{c}{\textbf{\%}}\\
%\midrule
%Students & 12,987 & 51\% & 12,686 & 49\% & 25,673 & 100\%\\
%Employees & 9,943 & 56\% &  7,671 & 44\% & 17,614 & 100\%\\
%\addlinespace
%\hfill\textbf{Total} & \textbf{22,930} & \textbf{53\%} &
%\textbf{20,357} & \textbf{47\%} & \textbf{43,287} & \textbf{100\%}\\
%\bottomrule
%\end{tabularx}
%\singlespacing
%
%\emph{Notes}:
%\begin{enumerate}
%\item The counts shown below will differ from the official quarterly
%Registrar's registration report because 1) data for residents in the
%Schools of Medicine and Pharmacy and Pharmaceutical Science are
%excluded, and 2) registered, non-matriculated, visiting students are
%included.
%\item Student workers are excluded from employees; however emeritus
%faculty and others on recall status are included.
%\end{enumerate}
%
%Campus Planning. Analytical Studies and Space Planning\\
%31 January 2006
%}
%\centerline{\rotatebox{90}{\box0}}
%\end{facingcaption}
%
%% This will give us some more text
%\Blinddocument
%
%% Skipping a bunch of chapters
%\setcounter{chapter}{50}
%\chapter{Another chapter}
%\setcounter{figure}{73}
%\setcounter{table}{88}
%\begin{figure}
%\centering
%\fbox{\hbox to.8\linewidth{\hss Another figure\hss}}
%\caption{Another figure caption}
%\end{figure}
%\begin{table}
%\centering
%\caption{Another table caption}
%\begin{tabular}{ccc}
%\toprule
%X&Y&Z\\
%\midrule
%a&b&c\\
%\bottomrule
%\end{tabular}
%\end{table}
%\begin{figure}
%\caption{ASDF fig}
%\end{figure}
%\begin{table}
%\caption{ASDF tab}
%\end{table}

%\appendix
%\Blinddocument
\nocite{eddsa,tight-ake-ext,vdaf}
% Stuff at the end of the dissertation goes in the back matter
%\nocite{domsep-thesis/local, sigma-thesis/local,tls-thesis/local}
\backmatter
\bibliographystyle{abbrv} % Or whatever style you want like plainnat
\bibliography{bibfiles/abbrev3,bibfiles/mypubs,domsep-thesis/local,sigma-thesis/local,tls-thesis/local,eddsa/main,vdaf/cite,bibfiles/crypto}

\end{document}