\renewcommand*{\backref}[1]{(Cited on page~#1.)}
\DeclareMathAlphabet{\mathsc}{OT1}{cmr}{m}{sc}
\DeclareMathAlphabet{\mathslbf}{OT1}{cmr}{bx}{sl}

\setlength{\fboxsep}{1pt}

% ====================================================================
%
\newcommand{\gamesfontsize}{\small}
\newcommand{\gameName}[1]{\underline{$\main$ #1}\\[2pt]}


\newcommand{\twoColsNoDivide}[4]{
\begin{center}
        \framebox{
        \begin{tabular}{c@{\hspace*{.4em}}c@{\hspace*{.4em}}c}
        \begin{minipage}[t]{#1\textwidth}\setstretch{1.0}\gamesfontsize #3 \end{minipage}
        &
        \begin{minipage}[t]{#2\textwidth}\setstretch{1.0}\gamesfontsize #4 \end{minipage}
        \end{tabular}
        }
\end{center}
}

\newcommand{\oneCol}[2]{
\begin{center}
        \framebox{
        \begin{tabular}{c}
        \begin{minipage}[t]{#1\textwidth}\setstretch{1.0}\gamesfontsize #2 \end{minipage}
        \end{tabular}
        }
\end{center}
}


\newcommand{\twoCols}[4]{
\begin{center}
        \framebox{
        \begin{tabular}{c@{\hspace*{.4em}}|c@{\hspace*{.4em}}}
        \begin{minipage}[t]{#1\textwidth}\setstretch{1.0}\gamesfontsize #3 \end{minipage}
        &
        \begin{minipage}[t]{#2\textwidth}\setstretch{1.0}\gamesfontsize #4 \end{minipage}
        \end{tabular}
        }
\end{center}
}
%
%% Theorem environments
%
\newtheorem{thm}{Theorem}
\newtheorem{lem}{Lemma}


\newenvironment{theorem}{\begin{thm}}{\end{thm}}
\newenvironment{lemma}{\begin{lem}}{\end{lem}}

\newcommand{\secref}[1]{Section~\ref{#1}}
\newcommand{\apref}[1]{Appendix~\ref{#1}}
\newcommand{\thref}[1]{Theorem~\ref{#1}}

\newcommand{\figref}[1]{Figure~\ref{#1}}
\newcommand{\tabref}[1]{Table~\ref{#1}}

%%%fancy proof env %%%
\def\qedsym{\hspace{1pt}\rule[-1pt]{3pt}{9pt}}

\newlength{\saveparindent}
\setlength{\saveparindent}{\parindent}
\newlength{\saveparskip}
\setlength{\saveparskip}{\parskip}


\def\qsym{\vrule width0.6ex height1em depth0ex}
\newcount\proofqeded
\newcount\proofended
\def\qed{{\hspace{1pt}\rule[-1pt]{3pt}{9pt}}
\end{rm}\addtolength{\parskip}{-0pt}
\setlength{\parindent}{\saveparindent}
\global\advance\proofqeded by 1 }
\def\qedenv{
\end{rm}\addtolength{\parskip}{-0pt}
\setlength{\parindent}{\saveparindent}
\global\advance\proofqeded by 1 }
\renewenvironment{proof}%
 {\proofstart}%
 {\ifnum\proofqeded=\proofended~\qed\fi \global\advance\proofended by 1
  \medskip}
\newenvironment{proofenv}%
 {\proofenvstart}%
 {\ifnum\proofqeded=\proofended\qedenv\fi \global\advance\proofended by 1
  \medskip}
\makeatletter
\def\proofstart{\@ifnextchar[{\@oprf}{\@nprf}}
\def\proofenvstart{\@ifnextchar[{\@osprf}{\@nsprf}}
\def\@oprf[#1]{\begin{rm}\protect\vspace{6pt}\noindent{\bf Proof of #1:\ }%
\addtolength{\parskip}{5pt}\setlength{\parindent}{0pt}}
\def\@osprf[#1]{\begin{rm}\protect\vspace{6pt}\noindent
\addtolength{\parskip}{5pt}\setlength{\parindent}{0pt}}
\def\@nprf{\begin{rm}\protect\vspace{6pt}\noindent{\bf Proof:\ }%
\addtolength{\parskip}{5pt}\setlength{\parindent}{0pt}}
\def\@nsprf{\begin{rm}\protect\vspace{6pt}\noindent%
\addtolength{\parskip}{5pt}\setlength{\parindent}{0pt}}


% ========================================================================

% Lists

\newcounter{ctr}
\newenvironment{bignewenum}{%
\begin{list}{{\bf (\arabic{ctr})}\hfill}{\usecounter{ctr} \labelwidth=17pt%
\labelsep=7pt \leftmargin=24pt \topsep=4pt%
\setlength{\listparindent}{\saveparindent}%
\setlength{\parsep}{\saveparskip}%
\setlength{\itemsep}{4pt} }}{\end{list}}


\newenvironment{newmath}{\begin{displaymath}%
\setlength{\abovedisplayskip}{4pt}%
\setlength{\belowdisplayskip}{4pt}%
\setlength{\abovedisplayshortskip}{5pt}%
\setlength{\belowdisplayshortskip}{5pt} }{\end{displaymath}}

% =========================================================================


%\makeatletter
%\def\appearsin#1{\gdef\@appearsin{#1}}
%\def\maketitle{\par
% \begingroup
% \def\thefootnote{\arabic{footnote}}
% \def\@makefnmark{\hbox
% to 0pt{$^{\@thefnmark}$\hss}}
% \if@twocolumn
% \twocolumn[\@maketitle]
% \else \newpage
% \global\@topnum\z@ \@maketitle \fi\thispagestyle{plain}\@thanks
% \endgroup
% \setcounter{footnote}{0}
% \let\maketitle\relax
% \let\@maketitle\relax
% \gdef\@thanks{}\gdef\@author{}\gdef\@title{}\gdef\@appearsin{}
%          \let\thanks\relax}
%\def\@maketitle{\newpage
% \noindent \@appearsin
% \vskip 0.5in \begin{center}
% {\LARGE \@title \par} \vskip 1.5em {\large \lineskip .5em
%\begin{tabular}[t]{c}\@author
% \end{tabular}\par}
% \vskip 1em {\normalsize \@date} \end{center}
% \par
% \vskip 1.5em}
%\def\abstract{\if@twocolumn
%\section*{Abstract}
%\else \small
%\begin{center}
%{\bf Abstract\vspace{-.5em}\vspace{0pt}}
%\end{center}
%\quotation
%\fi}
%\def\endabstract{\if@twocolumn\else\endquotation\fi}
%\mark{{}{}}
%


% =========================================================================

% General
%\newcommand{\hindent}{\quad}

\newcommand{\mysubsubsection}[1]{\subsubsection{#1}}
\newcommand{\headingb}[1]{{\vspace{5pt}\noindent\textbf{#1.}}}
\newcommand{\headingg}[1]{{\sc{#1}}}
\newcommand{\heading}[1]{{\vspace{5pt}\noindent\sc{#1}}}

\newcommand{\tuple}[1]{\langle{#1}\rangle}
\def\from{\mbox{from}\ }
\def\From{\mbox{From}\ }
\def\bits{\{0,1\}}
\def\cross{\times}
\newcommand{\cab}{,\allowbreak}
\newcommand{\xor}{\oplus}
\newcommand{\Colon}{{:\;\;}}
\def\emptystring{\varepsilon}

\newcommand{\OL}[0]{\mathsf{OL}}

\newcommand{\Dom}{\mathsf{Dom}}
\newcommand{\DomR}{\mathsf{DomR}}
\newcommand{\Rng}{\mathsf{Rng}}
\newcommand{\RngR}{\mathsf{RngR}}
\newcommand{\calA}{{\cal A}}
\newcommand{\calB}{{\cal B}}
\newcommand{\calC}{{\cal C}}
\newcommand{\calF}{{\cal F}}
\newcommand{\N}{{{\mathds N}}}
\newcommand{\Z}{{{\mathds Z}}}
\newcommand{\R}{{{\mathds R}}}
\def\union{\cup}
\def\intersection{\cap}
\newcommand{\set}[2]{\{#1 \,:\, #2\}}
\newcommand{\getsr}{{{\,\leftarrow{\hspace*{-3pt}\raisebox{.75pt}{$\scriptscriptstyle\$$}}\,}}}
\newcommand{\OR}{{\textstyle\bigvee}}
\newcommand{\Var}{{\mbox{\bf Var}}}
\newcommand{\E}{\mathbf{E}}
\def\e{{\epsilon}}
\newcommand{\concat}{{\,\|\,}}
% ========================================================================

% ===================================================================
\newcommand{\comment}[1]{\hspace{8pt}{\small /$\!\!$/\ #1}}
\newcommand{\Comment}[1]{\hspace{5pt}{/$\!\!$/\ #1}}
\newcommand{\ccomment}[1]{\hspace{1pt}{/$\!\!$/\ #1}}

\newcommand{\authnote}[2]{ \begin{center}\fbox{\begin{minipage}{5.7in}
\textbf{#1 says:} #2\end{minipage}}\end{center}}

\newcommand{\nGame}[2]{\mathbf{G}^{\mathrm{#1}}_{#2}}
\newcommand{\pk}{pk}
\newcommand{\dk}{dk}
\newcommand{\FSOutSet}[0]{\mathrm{Out}}

\newcommand{\schemefont}[1]{{\mathsf{#1}}}
\newcommand{\algfont}[1]{{\mathrm{#1}}}
\newcommand{\Adv}{\mathbf{Adv}}
\newcommand{\genAdv}[3]{\mathbf{Adv}^{\mathrm{#1}}_{#2}(#3)}
\newcommand{\prfAdv}[1]{\genAdv{prf}{#1}}

\newcommand{\advA}{\mathcal{A}}
\newcommand{\advB}{\mathcal{B}}
\newcommand{\advC}{\mathcal{C}}
\newcommand{\advD}{\mathcal{D}}
\newcommand{\INDCCA}{\mbox{IND-CCA}\xspace}

\newcommand{\atk}{\mathrm{atk}}
\newcommand{\cpa}{\mathrm{cpa}}
\newcommand{\cca}{\mathrm{cca}}
\newcommand{\ind}{\mathrm{ind}}

\addtolength{\belowcaptionskip}{-2mm}
\addtolength{\abovecaptionskip}{-3mm}
\addtolength{\topsep}{-1mm}
\newcommand{\schalg}[2]{\mathsf{#1.#2}}


\newcommand{\procfont}[1]{\mathsc{#1}}
\newcommand{\tablefont}[1]{\mathsf{#1}}
\newcommand{\Initialize}{\Init}
\newcommand{\Finalize}{\Fin}
\newcommand{\Init}{\procfont{init}}
\newcommand{\Fin}{\procfont{fin}}

\newcommand{\RO}{\procfont{RO}}
\newcommand{\ROSim}{\procfont{ROSim}}
\newcommand{\FUNC}{\procfont{FUNC}}
\newcommand{\DecO}{\procfont{Dec}}

\newcommand{\win}{\mathsf{win}}

\newcommand{\Fimage}[0]{\mathrm{Im}}
\newcommand{\simulator}[0]{\mathsf{Sim}}
\newcommand{\Gensimulator}[1]{\mathsf{S}_{#1}}


\newcommand{\group}{\mathds{G}}
\newcommand{\wins}{\textrm{ wins}}
\newcommand{\true}{\mathsf{true}}
\newcommand{\false}{\mathsf{false}}
\newcommand{\AND}{\;\wedge\;}
\newcommand{\bad}{\mathsf{bad}}
\newcommand{\chal}{\mathsf{chal}}
\newcommand{\Good}{\mathsf{Good}}
\newcommand{\Gm}{\textnormal{G}}
\newcommand{\fn}{\footnotesize}

\newcommand{\vecx}{\mathbf{x}}
\newcommand{\pfvec}[0]{\mathbf{p}}
\newcommand{\vecY}{\mathslbf{Y}}
\newcommand{\vecV}{\mathslbf{V}}
\newcommand{\vecU}{\mathslbf{U}}
\newcommand{\vecW}{\mathslbf{W}}
\newcommand{\vecxx}[0]{\mathbf{x}}
\newcommand{\Oracle}{\mathsc{O}}
\newcommand{\pkeScheme}{\schemefont{PKE}}
\newcommand{\pkeEnc}{\schalg{\pkeScheme}{E}}
\newcommand{\pkeDecR}{\schalg{\pkeScheme}{DecR}}
\newcommand{\pkeML}{\schalg{\pkeScheme}{ml}}
\newcommand{\prf}{\mathsf{G}}
\newcommand{\prfKl}{\prf.\mathsf{kl}}
%
\newcommand{\pkeToKem}[0]{\mathbf{T}}
\newcommand{\QpkeToKem}[0]{\mathbf{T}}
\newcommand{\pkeCiph}[0]{C}
\newcommand{\kemCiph}[0]{C^*}
%
\newcommand{\kemScheme}{\schemefont{KE}}
\newcommand{\kemKg}{\schalg{\kemScheme}{K}}
\newcommand{\kemEnc}{\schalg{\kemScheme}{E}}
\newcommand{\kemDec}{\schalg{\kemScheme}{D}}
\newcommand{\kemRl}{\schalg{\kemScheme}{rl}}
\newcommand{\kemKl}{\schalg{\kemScheme}{kl}}
\newcommand{\kemRoSp}{\schalg{\kemScheme}{FS}}

\newcommand{\FkemScheme}{\overline{\mathsf{KE}}}
\newcommand{\FkemRoSp}{\schalg{\FkemScheme}{FS}}
\newcommand{\FkemEnc}{\schalg{\FkemScheme}{E}}
%
\newcommand{\GkemScheme}[1]{\schemefont{KE}_{#1}}
\newcommand{\GkemEnc}[1]{\schalg{\GkemScheme{#1}}{E}}
%
\newcommand{\DFunc}[0]{\mathsf{D}}
\newcommand{\KLenFunc}[0]{\mathsf{k}^*}
%

\newcommand{\advQINDIFF}[2]{\genAdv{QINDIFF}{#1}{#2}}
\newcommand{\advTINDIFF}[2]{\genAdv{TINDIFF}{#1}{#2}}
\newcommand{\advINDIFF}[2]{\genAdv{indiff}{#1}{#2}}
\newcommand{\advRINDIFF}[2]{\genAdv{reset\text{-}indiff}{#1}{#2}}

\newcommand{\Func}[1][]{\mathcal{F}_{#1}}
\newcommand{\FuncHon}[1][]{\mathcal{F}_{#1}.hon}
\newcommand{\FuncAdv}[1][]{\mathcal{F}_{#1}.adv}

\newcommand{\lazysample}{\mathsf{ls}}

\newcommand{\secPropPKE}[0]{\mathrm{S}_{\mathrm{pke}}}


%
% Game hops
%
\newcommand{\gamehopchange}[2][0.85]{\hspace{0pt}\smash{\fbox{\rule[-1.75pt]{0pt}{#1\baselineskip}\smash{#2}}}} % frame around changes from one game to another, use optional parameter to adapt box height (as factor of \baselineskip), leading \hspace{0pt} prevents weird jump if nothing before \gamehopchange{..}

\newcommand{\supportQuT}[1]{\mathbf{sup}(#1)}
\newcommand{\algOutput}[0]{\mathrm{OUT}}
%
% FREE / DOMSEP predicates
%
\newcommand{\aPrefix}{P}
\newcommand{\prefix}{\preceq}
\newcommand{\listRO}{\mathcal{L}}

\newcommand{\gameFREE}[4]{\mGame{#3,FREE}{#4,#1,#2}{\advA}}
\newcommand{\gameFIXED}[4]{\mGame{#3,FIXED}{#4,#1,#2}{\advA}}
\newcommand{\gameDOMSEP}[5]{\mGame{#2,#3,DOMSEP}{#4,#5,#1}{\advA}}

\newcommand{\ngameCORR}[1]{\nGame{CORR}{#1}}
\newcommand{\ngamePRF}[1]{\nGame{prf}{#1}}

\newcommand{\ngameINDCCA}[1]{\nGame{ind-cca}{#1}}
\newcommand{\ngameINDCPA}[1]{\nGame{INDCPA}{#1}}
\newcommand{\ngameOWCPA}[1]{\nGame{OWCPA}{#1}}
\newcommand{\ngameOWPCA}[1]{\nGame{OWPCA}{#1}}
\newcommand{\ngameDOM}[1]{\nGame{wdom}{#1}}
\newcommand{\ngameQINDIFF}[1]{\nGame{QINDIFF}{#1}}
\newcommand{\ngameTINDIFF}[2]{\nGame{TINDIFF-{#1}}{#2}}

\newcommand{\DRTransform}[0]{\mathbf{DR}}
\newcommand{\RTransform}[0]{\mathbf{R}}
\newcommand{\InsTransform}[0]{\mathbf{Inst}}

\newcommand{\Aset}[2]{\{#1\,:\,#2\}}
\newcommand{\domain}[0]{\mathcal{D}}
\newcommand{\rangeSet}[0]{\mathcal{R}}
\newcommand{\domainFunc}[0]{\mathcal{F}}
\newcommand{\usedDomain}[0]{\mathcal{U}}
\newcommand{\SchemeQuery}{\mathsf{sq}}
\newcommand{\restrict}[2]{#1|_{#2}}

\newcommand{\pfFunctor}[1]{\construct{F}_{\mathrm{pf}(#1)}}
\newcommand{\ldFunctor}[0]{\construct{F}_{\mathrm{ld}}}
\newcommand{\idFunctor}[0]{\construct{F}_{\mathrm{id}}}
\newcommand{\splFunctor}[0]{\construct{F}_{\mathrm{spl}}}

\newcommand{\queryRO}[0]{\mathbf{Q}}
\newcommand{\GenqueryRO}[1]{\mathbf{#1}}
\newcommand{\FixedprefixqueryRO}[1]{\QuT_{\mathrm{pf}(#1)}}
\newcommand{\FixedprefixanswerRO}[1]{\AnT_{\mathrm{pf}(#1)}}
\newcommand{\FixedprefixqueryInv}[1]{\QuTInv_{\mathrm{pf}(#1)}}
\newcommand{\FixedprefixanswerInv}[1]{\AnTInv_{\mathrm{pf}(#1)}}

\newcommand{\VarprefixqueryRO}{\GenqueryRO{QPv}}
\newcommand{\LengthqueryRO}{\QuT_{\mathrm{ld}}}
\newcommand{\LengthanswerRO}{\AnT_{\mathrm{ld}}}
\newcommand{\IdqueryRO}{\QuT_{\mathrm{id}}}
\newcommand{\IdanswerRO}{\AnT_{\mathrm{id}}}
\newcommand{\IdqueryInv}{\QuTInv_{\mathrm{id}}}
\newcommand{\IdanswerInv}{\AnTInv_{\mathrm{id}}}
\newcommand{\SplittingqueryRO}{\QuT_{\mathrm{spl}}}
\newcommand{\SplittingqueryInv}{\QuTInv_{\mathrm{spl}}}
\newcommand{\SplittinganswerRO}{\AnT_{\mathrm{spl}}}
\newcommand{\SplittinganswerInv}{\AnTInv_{\mathrm{spl}}}
\newcommand{\answerRO}{\GenqueryRO{AT}}
\newcommand{\answerInv}{\answerRO^{-1}}
\newcommand{\NIntrusivequeryRO}{\GenqueryRO{QNI}}
\newcommand{\IntrusivequeryRO}{\GenqueryRO{QI}}
\newcommand{\AddgqueryRO}[1]{\GenqueryRO{QAGx}[#1]}
\newcommand{\IDqueryRO}{\GenqueryRO{QId}}

\newcommand{\Outputsplitting}{\mathbf{OS}}


\newcommand{\FComb}[1]{[[#1]]}
\newcommand{\aFunc}[1]{\mathit{#1}}
\newcommand{\aFuncDom}[0]{\mathsf{Dom}}
\newcommand{\aFuncRng}[0]{\mathsf{Rng}}

\newcommand{\bFunc}[1]{\mathrm{#1}}
\newcommand{\bFuncDom}[0]{\mathsf{Dom}}
\newcommand{\bFuncRng}[0]{\mathsf{Rng}}

\newcommand{\encode}[1]{\langle #1\rangle}
\newcommand{\Encode}{\texttt{Encode}}


%\newcommand{\AllFuncs}[2]{\llbracket #1\rightarrow#2\rrbracket}
\newcommand{\AllFuncs}[2]{\mathrm{FUNC}(#1,#2)}
\newcommand{\AllSOLFuncs}[2]{\mathrm{SOL}(#1,#2)}
\newcommand{\AllXOLFuncs}[1]{\mathrm{XOL}(#1)}
\newcommand{\FuncSp}[1]{\mathsf{#1}}
\newcommand{\aRO}[0]{\aFunc{F}}
\newcommand{\roSp}[0]{\mathsf{FS}}
\newcommand{\GenroSp}[1]{\mathsf{#1}}
\newcommand{\GenroSpCardinality}[1]{\GenroSp{#1}.\mathsf{n}}
\newcommand{\GenroSpFuncs}[1]{\GenroSp{#1}.\mathsf{F}}
\newcommand{\GenroSpDom}[1]{\mathrm{Dom}(\GenroSp{#1})}
\newcommand{\GenroSpDomP}[1]{\mathrm{Dom}_{*}(\GenroSp{#1})}
\newcommand{\GenroSpRng}[1]{\mathrm{Rng}(\GenroSp{#1})}
\newcommand{\roSpCardinality}[0]{\roSp.\mathsf{n}}
\newcommand{\roSpFuncs}[0]{\roSp.\mathsf{Fncs}}
\newcommand{\roSpDom}[0]{\mathrm{Dom}(\roSp)}
\newcommand{\roSpRng}[0]{\mathrm{Rng}(\roSp)}
\newcommand{\GGenroSpCardinality}[1]{{#1}.\mathsf{n}}
\newcommand{\GGenroSpFuncs}[1]{{#1}.\mathsf{Fncs}}
\newcommand{\GGenroSpDom}[1]{\mathrm{Dom}({#1})}
\newcommand{\GGenroSpDomP}[1]{\mathrm{Dom}_{*}({#1})}
\newcommand{\GGenroSpRng}[1]{\mathrm{Rng}({#1})}

\newcommand{\appendRO}[2]{{#1}\!\!+\!\!{#2}}
\newcommand{\chopRO}[1]{\mathrm{Chop}(#1)}
\newcommand{\addRO}[1]{\mathrm{Add}(#1)}
\newcommand{\expandRO}[1]{\mathrm{Expand}(#1)}
\newcommand{\compressRO}[1]{\mathrm{Compress}(#1)}

%\newcommand{\SHA}[1]{\mathsf{SHA#1}}
\newcommand{\SHAA}[2]{\mathsf{SHA#1}\mbox{-}\mathsf{#2}}
\newcommand{\SHAKE}[1]{\mathsf{SHAKE#1}}
\newcommand{\cSHAKE}{\mathsf{cSHAKE}}

\newcommand{\ulheading}[1]{\medskip\noindent\textsf{\underline{\smash{#1}}}}
\newcommand{\uldheading}[2]{\medskip\noindent\textsf{\underline{\smash{#1:}} #2}}

% \newcommand{\pqcname}[1]{\texttt{#1}}
% \newcommand{\pqcheading}[1]{\medskip\noindent\underline{\smash{\pqcname{#1}}}}
\newcommand{\pqcnameRoundOne}[1]{\texttt{\color{darkgray}#1}}
\newcommand{\pqcnameRoundTwo}[1]{\texttt{\fontseries{b}\selectfont#1}}

\newcommand{\functionOut}{e}
\newcommand{\functionIn}{s}
\newcommand{\functionInSet}{SS}
\newcommand{\functionOutSet}{ES}

\newcommand{\construct}[1]{\mathbf{#1}}
\newcommand{\constructDom}[1]{\construct{#1}.\mathsf{Dom}}
\newcommand{\constructEv}[1]{\construct{#1}.\mathsf{Ev}}
\newcommand{\constructRng}[1]{\construct{#1}.\mathsf{Rng}}

\newcommand{\commoncoins}{st}
\newcommand{\ccell}{st\ell}
\newcommand{\SimgenCC}{\simulator.\algfont{Setup}}
\newcommand{\Simeval}{\simulator.\algfont{Ev}}
\newcommand{\GenSimgenCC}[1]{\Gensimulator{#1}.\algfont{Setup}}
\newcommand{\GenSimeval}[1]{\Gensimulator{#1}.\algfont{Ev}}
\newcommand{\ccsample}{\mathsf{sample}}

\newcommand{\UHF}{H} % universal hash function

\newcommand{\functionality}{\mathfrak{f}}
\newcommand{\functionalityClass}{\mathfrak{F}}
\newcommand{\priv}{\procfont{priv}}
\newcommand{\pub}{\procfont{pub}}
\newcommand{\FnO}{\procfont{FnO}}
\newcommand{\functionalityPriv}{\functionality_{\priv}}
\newcommand{\functionalityPub}{\functionality_{\pub}}


\newcommand{\ngameCCINDIFF}[1]{\nGame{rd\mbox{-}indiff}{#1}}
\newcommand{\AdvCCINDIFF}[2]{\genAdv{rd\mbox{-}indiff}{#1}{#2}}
\newcommand{\indccaAdv}[2]{\genAdv{ind\mbox{-}cca}{#1}{#2}}
\newcommand{\wdomAdv}[2]{\genAdv{wdom}{#1}{#2}}


\newcommand{\ngameTI}[1]{\nGame{ti}{#1}}
\newcommand{\AdvTI}[2]{\genAdv{ti}{#1}{#2}}

\newcommand{\ngameEXEC}[1]{\nGame{exec}{#1}}

\newcommand{\ngameMSReal}[1]{\nGame{ms\mbox{-}1}{#1}}
\newcommand{\ngameMSIdeal}[1]{\nGame{ms\mbox{-}0}{#1}}
\newcommand{\ngameMS}[1]{\nGame{ms}{#1}}

\newcommand{\env}{\algfont{FG}}

\newcommand{\execoracle}{OR}
\newcommand{\execfinalize}{\mathsf{Finalize}}

\newcommand{\workDom}[0]{{\cal W}}
\newcommand{\WDInv}[0]{\mathrm{In}}

\newcommand{\QuT}[0]{\mathsf{QT}}
\newcommand{\AnT}[0]{\mathsf{AT}}
\newcommand{\QuTInv}[0]{\mathsf{QTI}}
\newcommand{\AnTInv}[0]{\mathsf{ATI}}

\newcommand{\Codewords}{\mathcal{C}}

\newcommand{\lncsorfull}[1]{{#1}}
\DeclareMathVersion{normal1}