% !TEX root = main.tex

\section{Introduction}\label{sec-intro}

Theoretical works giving, and proving secure, schemes in the random oracle (RO) model~\cite{CCS:BelRog93}, often, for convenience, assume access to \textit{multiple, independent} ROs. Implementations, however, like to implement them all via a \textit{single} hash function like $\SHA256$ that is assumed to be a RO. 

The transition from one RO to many is, in principle, easy. One can use a method suggested by BR~\cite{CCS:BelRog93} and usually called ``domain separation.'' For example to build three random oracles $H_1,H_2,H_3$ from a single one, $H$, define 
\begin{equation}
H_1(x) = H(\langle 1\rangle\|x), \hspace{6pt} H_2(x) = H(\langle 2\rangle\|x) \hspace{6pt} \mbox{and}\hspace{6pt} H_3(x) = H(\langle 3\rangle\|x)  \;,  \label{eq-intro-domain-sep-eg}
\end{equation}
where $\langle i\rangle$ is the representation of integer $i$ as a bit-string of some fixed length, say one byte. One might ask if there is justifying theory: a proof that the above ``works,'' and a definition of what ``works'' means. A likely response is that it is obvious it works, and theory would be pedantic. 

If it were merely a question of the specific domain-separation method of Equation~(\ref{eq-intro-domain-sep-eg}), we'd be inclined to agree. But we have found some good reasons to revisit the question and look into theoretical foundations. They arise from the NIST Post-Quantum Cryptography (PQC) standardization process~\cite{NIST-PQC}. 

We analyzed the KEM submissions. We found attacks, breaking some of them, that arise from incorrect ways of turning one random oracle into many, indicating that the process is error-prone. We found other KEMs where methods other than Equation~(\ref{eq-intro-domain-sep-eg}) were used and whether or not they work is unclear. In some submissions, instantiations for multiple ROs were left unspecified. In others, they differed between the specification and reference implementation.  

Domain separation as per Equation~(\ref{eq-intro-domain-sep-eg}) is a \textit{method}, not a \textit{goal}. We identify and name the underlying goal, calling it \textit{oracle cloning}--- given one RO, build many, independent ones. (More generally, given $m$ ROs, build $n>m$ ROs.) We give a definition of what is an ``oracle cloning method'' and what it means for such a method to ``work,'' in a framework we call read-only indifferentiability, a simple variant of classical indifferentiability~\cite{TCC:MauRenHol04}.
%  that yields security not only for usage in single-stage games but also in multi-stage ones~\cite{EC:RisShaShr11,EC:DGHM13,EC:Mittelbach14}. 
We specify and study many oracle cloning methods, giving some general results to justify (prove read-only indifferentiability of) certain classes of them. The intent is not only to validate as many NIST PQC KEMs as possible (which we do) but to specify and validate methods that will be useful beyond that. 

Below we begin by discussing the NIST PQC KEMs and our findings on them, and then turn to our theoretical treatment and results.


 




%\heading{Background.} 
%Fujisaki and Okamoto (FO)~\cite{C:FujOka99} showed how to build an IND-CCA-secure KEM in a simple way from a more easily and directly realizable object, namely an OW-CPA-secure PKE scheme.
%give several, more modular such transforms with tight security and proofs in the quantum random oracle model (RO), which were
%further improved by Jiang, Zhang, Chen, Wang and Ma (JZCWM)~\cite{C:JZCWM18}. Other similar quantum-secure KEM transforms were given by Saito, Xagawa, and Yamakawa~\cite{EC:SaiXagYam18}.
%These transforms, which are becoming the canonical way to build IND-CCA KEMs, use \textit{multiple random oracles} and are proven to work assuming these random oracles are independent. It is worth noting that the independence assumption is implicit rather than explicit.

\heading{NIST PQC KEMs.} In late 2016, NIST put out a call for post-quantum cryptographic algorithms~\cite{NIST-PQC}. In the first round they received 28 submissions targeting IND-CCA-secure KEMs, of which 17 remain in the second round~\cite{NIST-PQC-round2}.

Recall that in a KEM (Key Encapsulation Mechanism) $\kemScheme$, the encapsulation algorithm $\kemEnc$ takes the public key $\pk$ (but no message) to return a symmetric key $K$ and a ciphertext $\kemCiph$ encapsulating it, $(\kemCiph,K)\getsr\kemEnc(\pk)$. Given an IND-CCA KEM, one can easily build an IND-CCA PKE scheme by hybrid encryption~\cite{CraSho03}, explaining the focus of standardization on the KEMs. 




Most of the KEM submissions (23 in the first round, 15 in the second round) are constructed from a weak (OW-CPA, IND-CPA, ...) PKE scheme using either a method from Hofheinz, H\"{o}velmanns and Kiltz (HHK)~\cite{TCC:HofHovKil17} or a related method from~\cite{IMA:Dent03,EC:SaiXagYam18,C:JZCWM18}. This results in a KEM $\GkemScheme{4}$, the subscript to indicate that it uses up to four ROs that we'll denote $\aFunc{H}_1, \aFunc{H}_2, \aFunc{H}_3, \aFunc{H}_4$. Results of~\cite{TCC:HofHovKil17,IMA:Dent03,EC:SaiXagYam18,C:JZCWM18} imply that $\GkemScheme{4}$ is provably IND-CCA, \textit{assuming the ROs $\aFunc{H}_1, \aFunc{H}_2, \aFunc{H}_3, \aFunc{H}_4$ are independent.}

% We will reserve $\aFunc{H}_1$ for possible use by the PKE scheme on which $\GkemScheme{n}$ is based. 


Next, the step of interest for us, the oracle cloning: they build the multiple random oracles via a single RO $\aFunc{H}$, replacing $\aFunc{H}_i$ with an oracle $\construct{F}[\aFunc{H}](i,\cdot)$, where we refer to the construction $\construct{F}$ as a ``cloning functor,'' and $\construct{F}[\aFunc{H}]$ means that $\construct{F}$ gets oracle access to $\aFunc{H}$. This turns $\GkemScheme{4}$ into a KEM $\GkemScheme{1}$ that uses only a \textit{single} RO $\aFunc{H}$, allowing an implementation to instantiate the latter with a single NIST-recommended primitive like $\SHAA{3}{512}$ or $\SHAKE{256}$~\cite{FIPS202}. (In some cases, $\GkemScheme{1}$ uses a number of ROs that is more than one but less than the number used by $\GkemScheme{4}$, which is still oracle cloning, but we'll ignore this for now.) 

Often the oracle cloning method (cloning functor) is not specified in the submission document; we obtained it from the reference implementation. Our concern is the security of this method and the security of the final, single-RO-using KEM $\GkemScheme{1}$. (As above we assume the starting $\GkemScheme{4}$ is secure if its four ROs are independent.)


% Makes the story too complicated. Deal with later

\heading{Oracle cloning in submissions.}
We surveyed the relevant (first- and second-round) NIST PQC KEM submissions, looking in particular at the reference code, to determine what choices of cloning functor $\construct{F}$ was made, and how it impacted security of $\GkemScheme{1}$.
%how Step 3 choices i
%The attacks we found exploit not only
% (1) but also (2) what type of function $\aFunc{H}$ was used, in the sense of whether it is a fixed output length hash function or a variable output length one.
 Based on our findings, we classify the submissions into groups as follows. 

First is a group of \textit{successfully attacked} submissions. We discover and specify attacks, enabled through erroneous RO cloning, on three (first-round) submissions:
\pqcnameRoundOne{BIG QUAKE}~\cite{nistpqc:BIGQUAKE},
\pqcnameRoundOne{DAGS}~\cite{nistpqc:DAGS} and
\pqcnameRoundOne{Round2}~\cite{nistpqc:Round2}.
(Throughout the paper, first-round submissions are in \pqcnameRoundOne{gray}, second-round submissions in \pqcnameRoundTwo{bold}.)
Our attacks on \pqcnameRoundOne{BIG QUAKE} and \pqcnameRoundOne{Round2} recover the symmetric key $K$ from the ciphertext $\kemCiph$ and public key. Our attack on \pqcnameRoundOne{DAGS} succeeds in partial key recovery, recovering 192~bits of the symmetric key. These attacks are very fast, taking at most about the same time as taken by the (secret-key equipped, prescribed) decryption algorithm to recover the key. None of our attacks needs access to a decryption oracle, meaning we violate much more than IND-CCA. 
% We have implemented all these attacks \hd{No we have not. I never implemented the attacks on Round2 or Lizard or RLizard.} and verified that they work with the reference code of the submissions.

Next is submissions with \textit{questionable oracle cloning}. We put just one in this group, namely \pqcnameRoundTwo{NewHope}~\cite{nistpqc:NewHope}.
Here we do not have proof of security in the ROM for the final instantiated scheme $\GkemScheme{1}$. We do show that the cloning methods used here do not achieve our formal notion of rd-indiff security, but this does not result in an attack on $\GkemScheme{1}$, so we do not have a practical attack either.
%  even when assuming~$\aFunc{H}$ is a RO. 
We recommend changes in the cloning methods that permit proofs.

Next is a group of ten submissions that use \textit{ad-hoc oracle cloning} methods ---as opposed, say, to conventional domain separation as per Equation~(\ref{eq-intro-domain-sep-eg})---  but for which our results (to be discussed below) are able to prove security of the final single-RO scheme.
In this group are
\pqcnameRoundTwo{BIKE}~\cite{nistpqc:BIKE},
\pqcnameRoundOne{KCL}~\cite{nistpqc:KCL},
\pqcnameRoundOne{LAC}~\cite{nistpqc:LAC},
\pqcnameRoundOne{Lizard}~\cite{nistpqc:Lizard},
\pqcnameRoundOne{LOCKER}~\cite{nistpqc:LOCKER},
\pqcnameRoundOne{Odd Manhattan}~\cite{nistpqc:OddM},
\pqcnameRoundTwo{ROLLO-II}~\cite{nistpqc:ROLLO},
\pqcnameRoundTwo{Round5}~\cite{nistpqc:Round5},
\pqcnameRoundTwo{SABER}~\cite{nistpqc:SABER} and
\pqcnameRoundOne{Titanium}~\cite{nistpqc:Titanium}.
Still, the security of these oracle cloning methods remains brittle and prone to vulnerabilities under slight changes.

A final group of twelve submissions \textit{did well}, employing something like Equation~(\ref{eq-intro-domain-sep-eg}). In particular our results can prove these methods secure.
In this group are
\pqcnameRoundTwo{Classic McEliece}~\cite{nistpqc:ClassicMcEliece},
\pqcnameRoundTwo{CRYSTALS-Kyber}~\cite{nistpqc:CRYSTALSKyber},
\pqcnameRoundOne{EMBLEM}~\cite{nistpqc:EMBLEM},
\pqcnameRoundTwo{FrodoKEM}~\cite{nistpqc:FrodoKEM},
\pqcnameRoundTwo{HQC}~\cite{nistpqc:HQC},
\pqcnameRoundOne{LIMA}~\cite{nistpqc:LIMA},
\pqcnameRoundOne{NTRU-HRSS-\allowbreak{}KEM}~\cite{nistpqc:NTRU-HRSS-KEM},
\pqcnameRoundTwo{NTRU Prime}~\cite{nistpqc:NTRUPrime},
\pqcnameRoundTwo{NTS-KEM}~\cite{nistpqc:NTS-KEM},
\pqcnameRoundTwo{RQC}~\cite{nistpqc:RQC},
\pqcnameRoundTwo{SIKE}~\cite{nistpqc:SIKE} and
\pqcnameRoundTwo{ThreeBears}~\cite{nistpqc:ThreeBears}.




This classification omits 14 KEM schemes that do not fit the above framework.
(For example they do not target IND-CCA KEMs, do not use HHK-style transforms, or do not use multiple random oracles.)




%As per the IND-CCA game, the attacker, with oracle access to $\aFunc{H}$, is trying to determine challenge bit $b$ given the public key $\pk$, a ciphertext $\kemCiph$ and a challenge symmetric key $K_b$, where $(\kemCiph,K_1) \getsr \GkemEnc{1}^{\aFunc{H}}(\pk)$ and $K_0\getsr\bits^{|K_1|}$. Our attacks, however,  violate IND-CPA security ---they do not need access to a decryption oracle--- and, in some cases, allow key recovery, meaning from $\pk,\kemCiph$ the attacker recovers $K_1$, which of course allows it to win the above game. (In practice, however, it is much more damaging since it allows decryption of the message encrypted under the symmetric key.)
%


\heading{Lessons and response.} We see that oracle cloning is error-prone, and that it is sometimes done in ad-hoc ways whose validity is not clear. We suggest that oracle cloning not be left to implementations. Rather, scheme designers should give proof-validated oracle cloning methods for their schemes. To enable this, we initiate a theoretical treatment of oracle cloning. We formalize oracle cloning methods, define what it means for one to be secure, and specify a library of proven-secure methods from which designers can draw. We are able to justify the oracle cloning methods of many of the unbroken NIST PQC KEMs.  The framework of read-only indifferentiability we introduce and use for this purpose may be of independent interest.



The NIST PQC KEMs we break are first-round candidates, not second-round ones, and in some cases other attacks on the same candidates exist, so one may say the breaks  are no longer interesting. We suggest reasons they are. Their value is illustrative, showing not only that errors in oracle cloning occur in practice, but that they can be devastating for security. In particular, the extensive and long review process for the first-round NIST PQC submissions seems to have missed these simple attacks, perhaps due to lack of recognition of the importance of good oracle cloning.







\heading{Indifferentiability background.} 
Let $\GenroSp{\functionInSet},\GenroSp{\functionOutSet}$ be sets of functions. (We will call them the starting and ending function spaces, respectively.) A functor $\construct{F} \allowbreak \Colon \allowbreak \GenroSp{\functionInSet} \allowbreak \to \allowbreak\GenroSp{\functionOutSet}$ is a deterministic algorithm that, given as oracle a function $\aFunc{\functionIn}\in\GenroSp{\functionInSet}$, defines a function $\construct{F}[\aFunc{\functionIn}] \allowbreak \in \allowbreak \GenroSp{\functionOutSet}$. Indifferentiability of $\construct{F}$ is a way of defining what it means for $\construct{F}[\aFunc{\functionIn}]$ to emulate $\aFunc{\functionOut}$ when $\aFunc{\functionIn},\aFunc{\functionOut}$ are randomly chosen from $\GenroSp{\functionInSet},\GenroSp{\functionOutSet}$, respectively. It permits a ``composition theorem'' saying that if $\construct{F}$ is indifferentiable then use of~$\aFunc{\functionOut}$ in a scheme can be securely replaced by use of $\construct{F}[\aFunc{\functionIn}]$.  

Maurer, Renner and Holenstein (MRH)~\cite{TCC:MauRenHol04} gave the first definition of indifferentiability and corresponding composition theorem. However, Ristenpart, Shacham and Shrimpton (RSS)~\cite{EC:RisShaShr11} pointed out a limitation, namely that it only applies to single-stage games. MRH-indiff fails to guarantee security in multi-stage games, a setting that includes many goals of interest including security under related-key attack, deterministic public-key encryption and encryption of key-dependent messages. Variants of MRH-indiff~\cite{C:CDMP05,EC:RisShaShr11,EC:DGHM13,EC:Mittelbach14} tried to address this, with limited success. 

\heading{Rd-indiff.} Indifferentiability is the natural way to treat oracle cloning. A cloning of one function into $n$ functions ($n=4$ above) can be captured as a functor (we call it a cloning functor) $\construct{F}$ that takes the single RO $\aFunc{\functionIn}$ and for each $i \in [1..n]$ defines a function $\construct{F}[\aFunc{\functionIn}](i,\cdot)$ that is meant to emulate a RO. We will specify many oracle cloning methods in this way.

We define in Section~\ref{sec-cc-indiff} a variant of indifferentiability we call read-only indifferentiability (rd-indiff). The simulator ---unlike for reset-indiff~\cite{EC:RisShaShr11}--- has access to a game-maintained state $\commoncoins$, but ---unlike MRH-indiff~\cite{TCC:MauRenHol04}--- that state is read-only, meaning the simulator cannot alter it across invocations. Rd-indiff is a stronger requirement than MRH-indiff (if $\construct{F}$ is rd-indiff then it is MRH-indiff) but a weaker one than reset-indiff (if $\construct{F}$ is reset-indiff then it is rd-indiff). Despite the latter, rd-indiff, like reset-indiff, admits a composition theorem showing that an rd-indiff $\construct{F}$ may securely substitute a RO even in multi-stage games. (The proof of RSS~\cite{EC:RisShaShr11} for reset-indiff extends to show this.) We do not use reset-indiff because some of our cloning functors do not meet it, but they do meet rd-indiff, and the composition benefit is preserved.



% and prove our oracle cloning constructions are rd-indiff. rd-indiff differs from other indifferentiability variants because it mandates a simulator that initializes an immutable state $\commoncoins$ and then answers each query without altering $\commoncoins$. The dividend is that, unlike with classical MRH-indiff, we get security for usage in multi-stage games, not just single-stage ones. 

% We can compare rd-indiff to other variants. On the one hand, any rd-indiff cloning functor is also MRH-indiff. (The value of the strengthening is that rd-indiff implies security for multi-stage games while MRH-indiff does not.) On the other hand, any cloning functor that is reset-indiff in the sense of RSS~\cite{EC:RisShaShr11} is rd-indiff. 


\heading{General results.} In Section~\ref{sec-cc-indiff}, we define \textit{translating} functors. These are simply ones whose oracle queries are non-adaptive. (In more detail, a translating functor determines from  its input $W$ a list of queries, makes them to its oracle and, from the responses and $W$, determines its output.) We then define a condition on a translating functor $\construct{F}$ that we call \textit{invertibility} and show that if $\construct{F}$ is an invertible translating functor then it is rd-indiff. This is done in two parts,  Theorems~\ref{th-cc-indiff-invertible} and~\ref{th-cc-indiff-invertible-gen}, that differ in the degree of invertibility assumed. The first, assuming the greater degree of invertibility, allows a simpler proof with a simulator that does not need the read-only state allowed in rd-indiff. The second, assuming the lesser degree of invertibility, depends on a simulator that makes crucial use of the read-only state. It sets the latter to a key for a PRF that is then used to answer queries that fall outside the set of ones that can be trivially answered under the invertibility condition. This use of a computational primitive (a PRF) in the indifferentiability context may be novel and may seem odd, but it works.

 We apply this framework to analyze particular, practical cloning functors, showing that these are translating and invertible, and then deducing their rd-indiff security. But the above-mentioned results are stronger and more general than we need for the application to oracle cloning. The intent is to enable further, future applications.

  




\heading{Analysis of oracle cloning methods.} We formalize oracle cloning as the task of designing a functor (we call it a cloning functor) $\construct{F}$ that takes as oracle a function $\functionIn \in \GenroSp{\functionInSet}$ in the starting space and returns a two-input function $\functionOut = \construct{F}[\functionIn] \in \GenroSp{\functionOutSet} $, where $\functionOut(i,\cdot)$ represents the $i$-th RO for $i\in [1..n]$. Section~\ref{sec-domain-separation} presents the cloning functors corresponding to some popular and practical oracle cloning methods (in particular ones used in the NIST PQC KEMs), and shows that they are translating and invertible. Our above-mentioned results allow us to then deduce they are rd-indiff, which means they are safe to use in most applications, even ones involving multi-stage games. This gives formal justification for some common oracle cloning methods. We now discuss some specific cloning functors that we treat in this way.

The prefix (cloning) functor $\pfFunctor{\pfvec}$ is parameterized by a fixed, public vector~$\pfvec$ such that no entry of $\pfvec$ is a prefix of any other entry of $\pfvec$. Receiving function $\aFunc{\functionIn}$ as an oracle, it defines function $\functionOut = \pfFunctor{\pfvec}[\aFunc{\functionIn}]$ by $\functionOut(i,X) \allowbreak = \allowbreak \aFunc{\functionIn}(\pfvec[i]\|X)$, where $\pfvec[i]$ is the $i^{\text{th}}$ element of vector $\pfvec$. When $\pfvec[i]$ is a fixed-length bitstring representing the integer $i$, this formalizes Equation~(\ref{eq-intro-domain-sep-eg}). 

Some NIST PQC submissions use a method we call output splitting. The simplest case is that we want $e(i,\cdot),\ldots,\e(n,\cdot)$ to all have the same output length~$L$. We then define $e(i,X)$ as bits $(i-1)L\!+\!1$ through $iL$ of the given function $\functionIn$ applied to $X$. That is, receiving function $\aFunc{\functionIn}$ as an oracle, the splitting (cloning) functor $\splFunctor$ returns function $\functionOut = \splFunctor[\aFunc{\functionIn}]$ defined by $\functionOut(i,X) \allowbreak = \allowbreak \functionIn(X)[(i-1)L\!+\!1 .. iL]$. 

An interesting case, present in some NIST PQC submissions, is trivial cloning: just set $\functionOut(i,X)=\functionIn(X)$ for all $X$. We formalize this as the identity (cloning) functor $\idFunctor$ defined by $\idFunctor[\functionIn](i,X) = \functionIn(X)$. Clearly, this is not always secure. It can be secure, however, for usages that restrict queries in some way. One such restriction, used in several NIST PQC KEMs, is length differentiation: $\functionOut(i,\cdot)$ is queried only on inputs of some length $l_i$, where $l_1,\ldots,l_n$ are chosen to be distinct. We are able to treat this in our framework using the concept of working domains that we discuss next, but we warn that this method is brittle and prone to misuse.


% An attractive feature of this method is that queries made by the algorithms of the original scheme, and thus the implementation, do not need modification. For this reason, it is often used to justify omitting conventional domain separation. Our treatment makes this justification explicit and highlights the restricted domain necessary for its use. 




\heading{Working domains.} One could capture trivial cloning with length differentiation as a restriction on the domains of the ending functions, but this seems artificial and dangerous because the implementations do not enforce any such restriction; the functions there are defined on their full domains and it is, apparently, left up to applications to use the functions in a way that does not get them into trouble. The approach we take is to leave the functions defined on their full domains, but define and ask for security over a subdomain, which we called the working domain. A choice of working domain $\workDom$ accordingly parameterizes our definition of rd-indiff for a functor, and also the definition of invertibility of a translating functor. Our result says that the identity functor is rd-indiff for certain choices of working domains that include the length differentiation one. 

Making the working domain explicit will, hopefully, force the application designer to think about, and specify, what it is, increasing the possibility of staying out of trouble. Working domains also provide flexibility and versatility under which different applications can make different choices of the domain.

Working domains not being present in prior indifferentiability formalizations, the comparisons, above, of rd-indiff with these prior formalizations assume the working domain is the full domain of the ending functions.  Working domains alter the comparison picture; a cloning functor which is rd-indiff on a working domain may not be even MRH-indiff on its full domain.

 
%\heading{NewHope.} Illustrating how our formalism can establish limits of formal security, we also consider the candidate from the group above with questionable oracle cloning, \pqcnameRoundTwo{NewHope}.
%By giving a read-only indifferentiability adversary that clearly differentiates \pqcnameRoundTwo{NewHope}'s effective function space from the function space considered in its security proof,
%we point out a gap in the scheme's proof of security.



% Indifferentiability formalizations give the adversary two oracles, $\priv$ and $\pub$. We generalize prior definitions to restrict queries to $\priv$ to a subset of the domain (of the ending functions) that we call the working domain. This will allow us to capture and prove rd-indiff for certain cloning functors that work only for queries that are restricted in some way. 


% We analyze this via our above-mentioned concept of working domains, defining a working domain to capture the re 
 



\heading{Application to KEMs.} The framework above is broad, staying in the land of ROs and not speaking of the usage of these ROs in any particular cryptographic primitive or scheme. As such, it can be applied to analyze RO instantiation in many primitives and schemes. In Section~\ref{sec-kem}, we exemplify its application in the realm of KEMs as the target of the NIST PQC designs.

This may seem redundant, since an indifferentiability composition theorem says exactly that once indifferentiability of a functor has been shown, ``all'' uses of it are secure. However, prior indifferentiability frameworks do not consider working domains, so the known composition theorems apply only when the working domain is the full one. (Thus the reset-indiff composition theorem of~\cite{EC:RisShaShr11} extends to rd-indiff so that we have security for applications whose security definitions are underlain by either single or multi-stage games, but only for full working domains.) 

To give a composition theorem that is conscious of working domains, we must first ask what they are, or mean, in the application. We give a definition of the \textit{working domain of a KEM $\kemScheme$.} This is the set of all points that the scheme algorithms query to the ending functions in usage, captured by a certain game we give. (Queries of the adversary may fall outside the working domain.) Then we give a working-domain-conscious composition theorem for KEMs (Theorem~\ref{th-kem}) that says the following. Say we are given an IND-CCA KEM $\kemScheme$ whose oracles are drawn from a function space $\kemRoSp$. Let $\construct{F}\Colon \GenroSp{\functionInSet} \to \kemRoSp$ be a functor, and let $\FkemScheme$ be the KEM obtained by implementing the oracles of the $\kemScheme$ via $\construct{F}$. (So the oracles of this second KEM are drawn from the function space $\FkemRoSp =  \GenroSp{\functionInSet}$.) Let $\workDom$ be the working domain of $\kemScheme$, and assume $\construct{F}$ is rd-indiff over $\workDom$. Then $\FkemScheme$ is also IND-CCA. Combining this with our rd-indiff results on particular cloning functors justifies not only conventional domain separation as an instantiation technique for KEMs, but also more broadly the instantiations in some NIST PQC submissions that do not use domain separation, yet whose cloning functors are rd-diff over the working domain of their KEMs. The most important example is the identity cloning functor used with length differentiation.

A key definitional element of our treatment that allows the above is, following~\cite{EC:BelBerTes16}, to embellish the \textit{syntax} of a scheme (here a KEM $\kemScheme$) by having it name a function space $\kemRoSp$ from which it wants its oracles drawn. Thus, the scheme specification must say how many ROs it wants, and of what domains and ranges. In contrast, in the formal version of the ROM in~\cite{CCS:BelRog93}, there is a single, scheme-independent RO that has some fixed domain and range, for example mapping $\bits^*$ to $\bits$. This leaves a gap, between the object a scheme wants and what the model provides, that can lead to error. We suggest that, to reduce such errors, schemes specified in standards include a specification of their function space.

%With this in place, Theorem~\ref{thm:kem-query-translation} shows that cloning functors which are rd-indiff preserve the security of the KEM.
%Combining this with our theorems proving rd-indiff for several classes of cloning functors 
%We demonstrate both by relating the specific instantiations of one scheme from the two strongest groups above (\pqcnameRoundTwo{SABER}, resp.~\pqcnameRoundTwo{Classic McEliece}) as cloning functors in our formalism, providing the formal link for their provable security results in the MROM to carry over to their single-RO instantiations.



%\heading{Related work.} The concern of work on global random oracles~\cite{CCS:CanJaiSca14,EC:CDGLN18} is security when the same RO may be used across an arbitrary and a priori unknown set of applications and schemes. We do not go this far. Our concern is correct and secure usage within the usual context of the game defining security of the scheme, for example IND-CCA for KEMs.
