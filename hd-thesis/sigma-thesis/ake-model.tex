\section{AKE Security Model\fullelse{}{ and Multi-User Building Blocks}}
\label{sec:ake-model}

We provide our results in a game-based key exchange model formalized in Figure~\ref{fig:AKE-security}, at its core following the seminal work by Bellare and Rogaway~\cite{C:BelRog93} considering an active network adversary that controls all communication (initiating sessions and determining their next inputs through $\Send$ queries) and is able to corrupt long-term secrets ($\RevLongTermKey$) as well as session keys ($\RevSessionKey$).
The adversary's goal is then to
(a) distinguish the established shared \emph{session key} in a ``fresh'' (not trivially compromised, captured through a $\Fresh$ predicate) session from a uniformly random key obtained through $\Test$ queries (breaking \emph{key secrecy}),
or (b) make a session accept without matching communication partner (breaking \emph{explicit authentication}).

Following Cohn-Gordon et al.~\cite{C:CCGJJ19}, we formalize our model in a real-or-random version (following Abdalla, Fouque, and Pointcheval~\cite{PKC:AbdFouPoi05} with added forward secrecy~\cite{SP:AbdBenMac15}) with \emph{many} $\Test$ queries which all answer with a real or uniformly random session key based on the \emph{same} random bit~$b$.
We focus on the security of the \emph{main} session key established.
While our proofs (for both \SIGMA and TLS~1.3) establish security of the intermediate encryption and MAC keys, too,
we do not treat them as first-class keys available to the adversary through $\Test$ and $\RevSessionKey$ queries.
We expect that our results extend to a multi-stage key exchange (MSKE~\cite{CCS:FisGue14}) treatment
and refer to the concurrent work by Diemert and Jager~\cite{JC:DieJag20} for tight results for TLS~1.3 in a MSKE model.

In contrast to the work by Cohn-Gordon et al.~\cite{C:CCGJJ19} and Diemert and Jager~\cite{JC:DieJag20}, our model additionally captures explicit authentication through the $\ExplicitAuth$ predicate in Figure~\ref{fig:AKE-security}, ensuring sessions with non-corrupted peer accept with an honest partner session.
We and~\cite{JC:DieJag20} further treat protocols where the communication partner's identity of a session may be unknown at the outset and only learned during the protocol execution; this setting of ``post-specified peers''~\cite{C:CanKra02} particularly applies to the \SIGMA protocol family~\cite{C:Krawczyk03} as well as TLS~1.3~\cite{rfc8446}.


\subsection{Key Exchange Protocols}

\label{sec:ake-syntax}
We begin by formalizing the syntax of key exchange protocols.
A key exchange protocol $\KE$ consists of three algorithms~$(\KEKGen\cab \KEActivate\cab \KERun)$
and an associated key space~$\KEkeyspace$ (where most commonly $\KEkeyspace = \bits^n$ for some $n \in \NN$).
The key generation algorithm~$\KEKGen() \tor (\pk\cab \sk)$ generates new long-term public/secret key pairs.
In the security model, we will associate key pairs to distinct \emph{users} (or \emph{parties}) with some identity~$u \in \NN$ running the protocol,
and log the public long-term keys associated with each user identity in a list $\peerpk$.
(The adversary will be in control of initializing new users, identified by an increasing counter, and we assume it only references existing user identities.)
The activation algorithm~$\KEActivate(\id, \sk, \peerid, \peerpk, \role) \tor (\st', m')$ initiates a new session for a given user identity~$\id$ (and associated long-term secret key~$\sk$) acting in a given role~$\role \in \{\initiator\cab \responder\}$ and aiming to communicate with some peer user identity~$\peerid$.
$\KEActivate$ also takes as input the list $\peerpk$ of all users' public keys; protocols may use this list to look up their own and their peers' public keys. 
We provide the entire list instead of just the user's and peers' public keys to accommodate protocols with post-specified peer. 
These protocols may leave $\peerid$ unspecified at the time of session activation; when the peer identity is set at some later point, the list can be used to find the corresponding long-term key. 
Activation outputs a session state and (if $\role = \initiator$) first protocol message~$m'$, and will be invoked in the security model to create a new session~$\pi_u^i$ at a user~$u$ (where the label~$i$ distinguishes different sessions of the same user).
Finally, $\KERun(\id, \sk, \st,\peerpk, m) \tor (\st', m')$ delivers the next incoming key exchange message $m$ to the session of user~$\id$ with secret key~$\sk$ and state $\st$, resulting in an updated state~$\st'$ and a response message~$m'$. Like $\KEActivate$, it relies on the list $\peerpk$ to look up its own and its peer's long-term keys. 

The state of each session in a key exchange protocol contains at least the following variables, beyond possibly further, protocol-specific information:
\begin{description}
	\setlength{\itemsep}{0.25em} % little more space
	
	\item[$\peerid \in \NN$.]
	Reflects the (intended) partner identity of the session;
	\fullelse{in protocols with post-specified peers this is learned and set (once) by the session during the protocol execution.}
	{if post-specified, this is learned and set (once) during protocol execution.}
	
	\item[$\role \in \{\initiator,\responder\}$.]
	The session's role, determined upon activation.
	
	\item[$\status \in \{\running,\accepted,\rejected\}$.]
	The session's status;
	initially $\status = \running$,
	a session accepts when it switches to $\status = \accepted$ (once).
	
	\item[$\skey \in \KEkeyspace$.]
	The derived session key (in\fullonly{ the protocol-specific key space~}$\KEkeyspace$), set upon acceptance.
	
	\item[$\sid$.]
	The session identifier used to define partnered session in the security model;
	initially unset, $\sid$ is determined (once) during protocol execution.
\end{description}


\subsection{Key Exchange Security}


We formalize our key exchange security game~$G^{\KESEC}_{\KE,\advA}$ in Figure~\ref{fig:AKE-security},
based on the concepts introduced above in Figure~\ref{fig:AKE-security}
and following the framework for code-based game playing by Bellare and Rogaway~\cite{EC:BelRog06}.
After initializing the game, %generating public/secret keys for~$n$ users,
the adversary~$\advA$ is given access to queries
$\NewUser$ (generating a new user's public/secret key pair),
$\Send$ (controlling activation and message processing of sessions),
$\RevSessionKey$ (revealing session keys),
$\RevLongTermKey$ (corrupting user's long-term secrets),
and~$\Test$ (providing challenge real-or-random session keys),
as well as a $\Finalize$ query to which it will submit its guess~$b'$ for the challenge bit~$b$, ending the game.
\begin{figure}[tp]
	\begin{minipage}[t]{0.5\textwidth}
		\NewExperiment[$G^{\KESEC}_{\KE,\advA}$]
		
		\begin{oracle}{$\Initialize$}
			\item $\time \gets 0$; $\users \gets 0$
			\item $b \getsr \bits$
		\end{oracle}
		
		\ExptSepSpace
		
		\begin{oracle}{$\NewUser$}
			\item $\users \gets \users + 1$
			\item $(pk_\users, sk_\users) \getsr \KEKGen()$
			\item $\revltk_\users \gets \infty$
			\item $\peerpk[\users] \gets \pk_{\users}$
			\item return $pk_\users$
		\end{oracle}
		
		\ExptSepSpace
		
		\begin{oracle}{$\Send(u, i, m)$}
			\item if $\pi_u^i = \bot$ then
			\item \hindent $(\peerid,\role) \gets m$
			\item \hindent $(\pi_u^i, m') \getsr \KEActivate(u\cab \sk_u\cab \peerid\cab \peerpk\cab \role)$
			\item \hindent $\pi_u^i.\taccepted \gets 0$
			
			\item else
			\item \hindent $(\pi_u^i, m') \getsr \KERun(u, \sk_u, \pi_u^i, \peerpk, m)$
			
			\item if $\pi_u^i.\status = \accepted$ then
			\item \hindent $\time \gets \time + 1$
			\item \hindent $\pi_u^i.\taccepted \gets \time$
			
			\item return $m'$
		\end{oracle}
		
		\ExptSepSpace
		
		\begin{oracle}{$\RevSessionKey(u, i)$}
			\item if $\pi_u^i = \bot$ or $\pi_u^i.\status \neq \accepted$ then
			\item \hindent return $\bot$
			
			\item $\pi_u^i.\revealed \gets \true$
			\item return $\pi_u^i.\skey$
		\end{oracle}
		
		\ExptSepSpace
		
		\begin{oracle}{$\RevLongTermKey(u)$}
			\item $\time \gets \time + 1$
			\item $\revltk_u \gets \time$
			\item return $sk_u$
		\end{oracle}
		
		\ExptSepSpace
		
		\begin{oracle}{$\Test(u, i)$}
			\item if $\pi_u^i = \bot$ or $\pi_u^i.\status \neq \accepted$ or $\pi_u^i.\tested$ then
			\item \hindent return $\bot$
			
			\item $\pi_u^i.\tested \gets \true$
			
			\item $T \gets T \cup \{\pi_u^i\}$
			\item $k_0 \gets \pi_u^i.\skey$
			\item $k_1 \sample \KEkeyspace$
			\item return $k_b$
		\end{oracle}
	\end{minipage}
	\begin{minipage}[t]{0.49\textwidth}
		\ExptSepSpace
		
		\begin{oracle}{$\Finalize(b')$}
			\item if $\neg \Sound$ then
			\iffull\item \hindent\fi return $1$
			
			\item if $\neg \ExplicitAuth$ then
			\iffull\item \hindent\fi return $1$
			
			\item if $\neg \Fresh$ then
			\iffull\item \hindent\fi $b' \gets 0$
			
			\item return $[[b = b']]$
		\end{oracle}
		
		\ExptSepSpace
		
		\begin{algorithm}{$\Sound$}
			%%% no triple sid match
			\item if $\exists$ distinct $\pi_u^i$, $\pi_v^j$, $\pi_w^k$ with $\pi_u^i.\sid = \pi_v^j.\sid = \pi_w^k.\sid$ then
			\comment{no triple sid match}
			\item \hindent return $\false$
			
			%%% same sid ==> same key
			\item if $\exists \pi_u^i, \pi_v^j$ with \newline
				\null\hindent $\pi_u^i.\status = \pi_v^j.\status = \accepted$ \newline
				\null\hindent and $\pi_u^i.\sid = \pi_v^j.\sid$ \newline
				\null\hindent and $\pi_u^i.\peerid = v$ and $\pi_v^j.\peerid = u$ \newline
				\null\hindent and $\pi_u^i.\role \neq \pi_v^j.\role$, but $\pi_u^i.\skey \neq \pi_v^j.\skey$ then  \comment{partnering implies same key}
			\item \hindent return $\false$
			
			\item return $\true$
		\end{algorithm}
		
		\ExptSepSpace
		
		\begin{algorithm}{$\ExplicitAuth$}
			\item return \newline
			\null \hindent  $\forall \pi_u^i : \pi_u^i.\status = \accepted$ \newline 
			\null \hindent \hindent \hindent \hindent and $\pi_u^i.\taccepted < \revltk_{\pi_u^i.\peerid}$ \newline \comment{all sessions accepting with a non-corrupted peer \dots} \newline
			\null\hindent \hindent  $\implies \exists \pi_v^j : \pi_u^i.\peerid = v$ \newline
			\null \hindent \hindent \hphantom{$\implies$} and $\pi_u^i.\sid = \pi_v^j.\sid$ \newline
			\null \hindent \hindent \hphantom{$\implies$} and $\pi_u^i.\role \neq \pi_v^j.\role$
			\newline 	\comment{\dots\ have a partnered session \dots} \newline
			\null \hindent \hphantom{$\implies$} and $(\pi_v^j.\status = \accepted \!\implies\! \pi_v^j.\peerid = u)$
				\comment{\dots\ agreeing on the peerid (upon acceptance)}
		\end{algorithm}
		
		\ExptSepSpace
		
		\begin{algorithm}{$\Fresh$}
			\item for each $\pi_u^i \in T$
			\item \hindent if $\pi_u^i.\revealed$ then
			\item \hindent \hindent return $\false$
			\comment{tested session may not be revealed}
			
			\item \hindent if $\exists \pi_v^j \neq \pi_u^i : \pi_v^j.sid = \pi_u^i.sid$ 
			\newline \null \hindent \hindent and ($\pi_v^j.\tested$ or $\pi_v^j.\revealed$) then
			\item \hindent \hindent return $\false$
			\comment{tested session's partnered session may not be tested or revealed}
			
			\item \hindent if $\revltk_{\pi_u^i.\peerid} < \pi_u^i.\taccepted$ then
			\item \hindent \hindent return $\false$
			\comment{tested session's peer may not be corrupted prior to acceptance}
			
			\item return $\true$
		\end{algorithm}
	\end{minipage}
	
	\caption{%
		Key exchange security game.
	}
	\label{fig:AKE-security}
\end{figure}

The game~$G^{\KESEC}_{\KE,\advA}$ then (in $\Finalize$) determines whether~$\advA$ was successful through the following three predicates,
formalized in pseudocode in Figure~\ref{fig:AKE-security}:
\iffull
\begin{description}
	\setlength{\itemsep}{0.5em} % little more space
	
	\item[$\Sound$.]
	The soundness predicate~$\Sound$ checks that (a) no three session identifiers collide (hence the session identifier properly serves to identify two partnered sessions).
	Furthermore, it ensures that (b) accepted sessions with the same session identifier, agreeing partner identities, and distinct roles derive the same session key.
	\iffull
	The adversary breaks soundness if it violates either of these properties.
	\fi
	
	\item[$\ExplicitAuth$.]
	The predicate~$\ExplicitAuth$ captures explicit authentication in that it requires that for any session of some user~$\id$ that accepted while its partner~$\peerid$ was not corrupted (captured through logging relative acceptance time~$\taccepted$ and long-term reveal time~$\revltk_{\peerid}$) has
	(a) a partnered session run by the intended peer identity and in an opposite role,
	and (b) if that partnered session accepts, it will do so with peer identity~$\id$.
	\iffull
	The adversary breaks explicit authentication if this predicate evaluates to false.
	\fi
	
	\item[$\Fresh$.]
	Finally, to capture key secrecy, we have to restrict the adversary to testing only so-called \emph{fresh} sessions in order to exclude trivial attacks, which the freshness predicate~$\Fresh$ ensures.
	A tested session is non-fresh, if
	(a) its session key has been revealed (in which case~$\advA$ knows the real key),
	(b) its partnered session (through~$\sid$) has been revealed or tested (in which case~$\advA$ knows the real key or may see two different random keys),
	or (c) its intended peer identity was compromised prior to accepting (in which case~$\advA$ may fully control the communication partner).
	\iffull
	If the adversary violates freshness, we invalidate its guess by overwriting~$b' \gets 0$.
	\fi
\end{description}
\else
$\Sound$ ensures session identifiers are set in a sound manner (non-colliding, ensuring agreement on session keys).
$\ExplicitAuth$ encodes explicit authentication, requiring that accepted sessions agree on the intended peer (if non-corrupted).
Finally, to capture key secrecy, we have to restrict the adversary to testing only \emph{fresh} (i.e., not trivially compromised) sessions in order to exclude trivial attacks; this is ensured through~$\Fresh$.
\fi

We call two distinct sessions~$\pi_u^i$ and~$\pi_v^j$ \emph{partnered} if $\pi_u^i.\sid = \pi_v^j.\sid$.
We refer to sessions generated by $\KEActivate$ (i.e., controlled by the game) as \emph{honest} sessions
to reflect that their behavior is determined honestly by the game and not the adversary.
The long-term key of an honest session may still be corrupted, or its session key may be revealed without affecting this notion of ``honesty''.

%%% old, non-parameterized version
% {\color{gray}
% \begin{definition}[Key exchange security]
% 	Let $\KE$ be a key exchange protocol
% 	and $\advA$ an adversary interacting in the key exchange security game~$G^{\KESEC}_{\KE,\advA}$ defined in Figure~\ref{fig:AKE-security}.
% 	We call $\KE$ secure in our model if the advantage
% 	\[
% 		\Adv^{\KESEC}_{\KE,\advA} := 2 \cdot \Pr \left[ \Gm^{\KESEC}_{\KE,\advA} \Rightarrow 1 \right] - 1
% 	\]
% 	of winning the game is small.
% \end{definition}
% }

\begin{definition}[Key exchange security]
	\label{def:KE-security}
	Let $\KE$ be a key exchange protocol and~$G^{\KESEC}_{\KE,\advA}$ be the key exchange security game defined in Figure~\ref{fig:AKE-security}.
	We define
	\[
		\Adv^{\KESEC}_{\KE}(t, \qNewUser, \qSend, \qRevSessionKey, \qRevLongTermKey, \qTest) := 2 \cdot \max_\advA \Pr \left[ \Gm^{\KESEC}_{\KE,\advA} \Rightarrow 1 \right] - 1,
	\]
	where the maximum is taken over all adversaries, denoted \emph{$(t, \qNewUser, \qSend, \qRevSessionKey, \qRevLongTermKey, \qTest)$-$\KESEC$-adversaries}, running in time at most~$t$ and making at most $\qNewUser$, $\qSend$, $\qRevSessionKey$, $\qRevLongTermKey$, resp.\ $\qTest$ queries to their oracles $\NewUser$, $\Send$, $\RevSessionKey$, $\RevLongTermKey$, resp.\ $\Test$.
\end{definition}


\subsection{Security Properties}

Let us briefly revisit some core security properties captured in our key exchange security model.

First, we capture regular \emph{key secrecy} of the main session key through $\Test$ queries, incorporating \emph{forward secrecy} (sometimes called ``perfect'' forward secrecy) by allowing the adversary to corrupt any user as long as all tested sessions accept prior to corrupting their respective intended peer.
This strengthens our model compared to that of Cohn-Gordon et al.~\cite{C:CCGJJ19} which only captures weak forward secrecy where the adversary has to be passive in sessions where it corrupts long-term secrets.
Diemert and Jager~\cite{JC:DieJag20} additionally treat the security of intermediate keys and further secrets beyond the main session key in a multi-stage approach~\cite{CCS:FisGue14}, but without capturing explicit authentication.

Our model encodes \emph{explicit authentication} (via $\ExplicitAuth$), a strengthening compared to the implicit-authentication model in~\cite{C:CCGJJ19}.

Like~\cite{C:CCGJJ19,JC:DieJag20}, our model captures \emph{key-compromise impersonation} attacks by allowing the session owner's secret key of tested sessions to be corrupted at any point in time.
Similarly, we do \emph{not} capture \emph{session-state or randomness reveals}~\cite{EC:CanKra01,PROVSEC:LaMLauMit07} or \emph{post-compromise security}~\cite{CSF:CohCreGar16}.



% \TODO{%
% To include in intro of model/discussion:
% \begin{itemize}
% 	\item BR93-like model~\cite{C:BelRog93} in terms of active adversary, able to corrupt long-term secrets and session keys; we don't consider session state or randomness reveal like in CK/eCK models~\cite{EC:CanKra01,PROVSEC:LaMLauMit07}
% 	\item like \cite{C:CCGJJ19} we adopt a real-or-random definition from~\cite{PKC:AbdFouPoi05} with forward secrecy~\cite{SP:AbdBenMac15}, allowing multiple $\Test$ queries all answered based on the same challenge bit~$b$
% 	\item extending the work of Cohn-Gordon~\cite{C:CCGJJ19} that was restricted to implicitly authenticated key exchange, our model however consideres \emph{explicitly authenticated} key exchange protocols
% 	\item (similar to \cite{C:CCGJJ19} paper:) describe concepts of sessions and their state (variables)
% 	\item KE definition
% 	\item define partnering / matching
% 	\item describe attacker model and oracles in Figure~\ref{fig:AKE-security}
% 	\item KE security definition
% 	\item discussion of the model: repeat real-or-random, multiple test queries, describe purpose of $\Sound$, $\ExplicitAuth$, and $\Fresh$; we do regular (``sometimes called `perfect' '') forward secrecy in constrast to only weak fs in \cite{C:CCGJJ19}; like them also capture key compromise impersonation 
% \end{itemize}
% }


