% \iffull
% \section{Proof of Theorem~\ref{thm-ggm-bound}}
% \else
\section{Proof of the Strong Diffie--Hellman GGM Bound (Theorem~\ref{thm-ggm-bound})}

% \fi
\label{apx:strongDHproof}
\begin{proof}
We begin by giving a code-based game for the strong Diffie--Hellman problem in the generic group model. First, we establish some preliminaries, using the setting and notation of Bellare and Dai~\cite{INDOCRYPT:BelDai20}.  Let $\group$ be an arbitrary set of strings with prime order $p$, and let $\E: \ZZ_p \to \group$ be a bijection, called the encoding function. For any two strings $A, B \in \group$, we define the operation $A \mathop{\OP_\E} B = \E(\E^{-1}(A) + \E^{-1}(B) \mod p)$. The set $\group$ is a group with respect to this operation, and it is isomorphic to $\ZZ_p$. Therefore, $\group$ has the identity $\E(0)$, and it is generated by $\E(1)$. 

In the generic group model, we wish for the adversary to compute group operations only through an oracle $\OP$. We accomplish this by picking the encoding function $\E$ at random and keeping it secret; then providing oracle access to $\OP_\E$ through $\OP$. In this model, we can give a sequence of games bounding the advantage of any adversary $\advA$ that makes $t$ queries to the $\OP$ oracle and $q$ queries to the $\stDH$ oracle.

\startproof{GGM}
\proofngame[ggm-start]
This first game formalizes the strong Diffie--Hellman problem in the generic group model. Note that for any $a \in \ZZ_p$, $a$ is the discrete logarithm of the group element $\E(a)$. 

It follows that 
\[
	\Adv^{\strongDH}_{\group}(t, \qstDH) = \Pr[\curGm* \Rightarrow 1].
\]
\begin{figure}[tp]
	\begin{minipage}[t]{0.49\textwidth}
		\NewExperiment[$\lblGm{ggm-start}$]
			
			\begin{oracle}{$\Initialize()$}
				\item $p \gets |\group|$; $\E \getsr \Bijections(\ZZ_p, \group)$
				\item $\one \gets \E(0)$; $\generator\gets \E(1)$
				\item $x,y \getsr \ZZ_p^*$; $\X \gets \E(\x)$; $\Y\gets E(\y)$
				\item $\GL \gets \{\one,\generator,\x,\y\}$
				\item return $(\one,\generator,\x,\y)$
			\end{oracle}
			\ExptSepSpace
			\begin{oracle}{$\OP(A,B,\sgn)$}
				\item if $A \not \in \GL$ or $B \not \in \GL$ then return $\bot$
				\item $c \gets \E^{-1}(A) \sgn \E^{-1}(B) \mod p$
				\item $C \gets \E(c)$; $\GL \gets \GL \cup \{C\}$
				\item return $C$
			\end{oracle}
			\end{minipage}
		\begin{minipage}[t]{0.49\textwidth}
		\ExptSepSpace
			\begin{oracle}{$\stDH(A,B)$}
				\item if $A \not \in \GL$ or $B \not \in \GL$ then return $\bot$
				\item $z \gets x \cdot E^{-1}(A) \mod p$
				\item $Z \gets E(z)$
				\item return $[[Z=B]]$
			\end{oracle}
			\ExptSepSpace
			\begin{oracle}{$\Finalize(Z)$}
				\item if $Z \not \in \GL$ then return $\false$
				\item $z \gets \x \cdot \y \mod p$; return $[[Z = \E(z)]]$
			\end{oracle}
		\end{minipage}
		
		\caption[]{%
			Game $\lblGm{ggm-start}$ of the $\stDH$ proof.
		}
		\label{fig:GGM-proof:game:start}
	\end{figure}
\proofngame[ggm-vector]
In Game $\curGm*$, we change the internal notation of the game.
First, for clarity and without loss of generality, we assume the adversary queries its $\OP$ and $\stDH$ oracles only on valid inputs (meaning their inputs are valid group elements in $\GL$).
Instead of representing each element of~$\group$ with an element of~$\ZZ_p$, we use a vector over $\ZZ_p^3$.
We define the basis vectors $\vec{e_1} := (1,0,0)$, $\vec{e_2} := (0,1,0)$, and $\vec{e_3} := (0,0,1)$.
We map  $\ZZ_p^{3}$ to $\ZZ_p$ by taking the inner product with the vector $(1,x,y)$.
(Effectively, we are representing each element of $\ZZ_p$ as a linear combination modulo $p$ of $1$, $x$, and $y$.)
We cache the map from $\ZZ_p^3$ to $\group$ induced by this transformation in a table $\TV$ and its inverse map in a table $\TI$.
% From this point forward, we will assume without loss ofG generality that all queries to the $\OP$ and $\stDH$ oracles are valid (meaning their inputs are valid group elements in $\GL$).

Although one element of $\group$ may now have multiple representations, the bilinearity of the inner product ensures that the view of the adversary is not changed, and $\Pr[\curGm* ] = \Pr[\prevGm* ].$

\begin{figure}[tp]
	\begin{minipage}[t]{0.49\textwidth}
		\NewExperiment[$\lblGm{ggm-vector}$]
		
		\ExptSepSpace
		
		\begin{oracle}{$\Initialize()$}
			\item $p \gets |\group|$; $\E \getsr \Bijections(\ZZ_p, \group)$
			\item $k\gets 0$; $\one \gets \VE(\vec{0})$; $\generator\gets \VE(\vec{e_1})$
			\item $\x,\y \getsr \ZZ_p^*$; $\vec{x} \gets {\one,\x,\y}$
			\item $\X \gets \VE(\vec{e_2})$; $\Y\gets \VE(\vec{e_3})$
			\item return $(\one,\generator,\x,\y)$
		\end{oracle}
		\ExptSepSpace
		\begin{oracle}{$\OP(A,B,\sgn)$}
			\item $\vec{c} \gets \VE^{-1}(A) \sgn \VE^{-1}(B) \mod p$
			\item $C \gets \VE(\vec{c})$; return $C$
		\end{oracle}
	\ExptSepSpace
		\begin{algorithm}{$\VE(\vec{t})$}
			\item if $\TV[\vec{t}] \neq \bot$ then return $\TV[\vec{t}]$
			\item $k \gets k+1$; $\vec{t_k} \gets \vec{t}$
			\item $v \gets \langle \vec{t},\vec{x} \rangle$; $C \gets E(v)$;  $\GL \gets \GL \cup \{C\}$
			\item  $\TV[\vec{t}] \gets C$; $\TI[C] \gets \vec{t}$
			\item return $\TV[\vec{t}]$
		\end{algorithm}
	\end{minipage}
	\begin{minipage}[t]{0.49\textwidth}
		\ExptSepSpace
		\begin{oracle}{$\stDH(A,B)$}
			\item $\vec{a} \gets \VE^{-1}(A)$; $\vec{b} \gets \VE^{-1}(B)$
			\item return $[[\VE(x \vec{a}) = B]]$
		\end{oracle}
		\ExptSepSpace
		\begin{oracle}{$\Finalize(Z)$}
			\item return $[[\VE(x \vec{e_3}) = Z]]$
		\end{oracle}
	\ExptSepSpace
		\begin{algorithm}{$\VE^{-1}(C)$}
			\item return $\TI[C]$
		\end{algorithm}
	\end{minipage}
	
	\caption[]{%
		Game $\lblGm{ggm-vector}$ of the $\stDH$ proof.
	}
	\label{fig:GGM-proof:game:vector}
\end{figure}

\proofngame[ggm-lazysample]
Next, we replace the random encoding function $\E$ with a lazily sampled encoding represented by table $\TV$ for the forward direction and $\TI$ for the backward direction. Because we want our encoding to be one-to-one, we sample from the set $\group\setminus \GL$. This assigns a unique element of $\group$ to each vector $\vec{t}$. However, as we've noted, each integer in $\ZZ_p$ has multiple representations in $\ZZ_p^3$. If two representations of the same integer are submitted to the encoding algorithm $\VE$, we set a $\bad$ flag and program the encoding table to maintain consistency.

We also change the format of the check in the $\stDH$ oracle. Since $\VE(x\vec{a}) = B = \VE(\vec{b})$ if and only if $\langle x \vec{a}, \vec{x} \rangle = \langle \vec{b}, \vec{x} \rangle$, we return $\true$ if the latter condition holds and $\false$ otherwise.  These two conditions are equivalent, so 
$\Pr[\curGm*] = \Pr[\prevGm*]$.

\proofngame[ggm-noprgm]
In this game, we stop programming the encoding table after the bad flag is set. Let $F_1$ denote the event that $\curGm*$ sets the $\bad$ flag at any point. 
By the fundamental lemma of game playing, $\Pr[\prevGm*] \leq  \Pr[\curGm*\text{ and }\overline{F_1} + \Pr[F_1]$. 

\proofngame[ggm-newfin] 
We remove the now-redundant $\bad$ flag, but the $\Finalize$ oracle now returns $\true$ if at any point in game $\prevGm*$ the $\bad$ flag would have been set (i.e. if event $F_1$ occurs). Otherwise, all oracles behave exactly as they did in $\prevGm*$. It follows that 
$\Pr[\prevGm* \text{ and } \overline{F_1}] + \Pr[F_1] \leq \Pr[\curGm*]$.

Additionally, in the $\stDH$ oracle, we separate out checking for trivial queries: if the adversary computed $A = \generator^a$ and $B = X^a$ for an integer $a$ of their choosing. If this is so, then $\vec{a} = a \vec{e_1}$ and $\vec{b} = a \vec{e_2}$, so $\langle x \vec{a}, \vec{x} \rangle = x a = \langle \vec{b},\vec{x} \rangle$, so may return $\true$. If the query is nontrivial but should still return true according to our previous condition, we set a $\bad[2]$ flag. This does not change the oracle's response to any query, so the above bound still holds. 

\proofngame[ggm-nostdh]
In Game $\curGm*$, we no longer return  $\true$ in the $\stDH$ oracle after the $\bad[2]$ flag is set. This makes the second check redundant and has the effect that the $\stDH$ oracle's behavior is no longer dependent on the value of either $x$ or $\vec{x}$. Let event $F_2$ denote the event that $\curGm*$ sets the $\bad[2]$ flag. By the fundamental lemma of game playing, $\Pr[\prevGm*]\leq \Pr[\curGm* \text{ and } \overline{F_2}] + \Pr[F_2]$. 


\proofngame[ggm-quadfin]
In Game $\curGm*$, we remove the redundant check and $\bad$ flag from the $\stDH$ oracle, and in the $\Finalize$ oracle we return $\true$ whenever the $\bad[2]$ flag would have been set in $\prevGm*$. Otherwise all oracles behave precisely as they did in $\prevGm*$. It follows that $\Pr[\prevGm* \text{ and } \overline{F_2}]+\Pr[F_2] \leq \Pr[\curGm*]$.  We also move the initialization of variables $x$, $y$, and $\vec{x}$ from $\Initialize$ to $\Finalize$. Since these variables are not used by any oracle but $\Finalize$, this does not change the view of the adversary. 

\begin{figure}[tp]
	\begin{minipage}[t]{0.49\textwidth}
		\NewExperiment[\frame{$\lblGm{ggm-lazysample}$}, $\lblGm{ggm-noprgm}$]
		
		\ExptSepSpace
		
		\begin{oracle}{$\stDH(A,B)$}
			\item $\vec{a} \gets \VE^{-1}(A)$; $\vec{b} \gets \VE^{-1}(B)$
			\item \gamechange{if $\langle x \vec{a},\vec{x} \rangle = \langle \vec{b}, \vec{x} \rangle$ then return $\true$}{}
			\item return \gamechange{$\false$}{}
		\end{oracle}
		
		\ExptSepSpace
		\begin{algorithm}{$\VE(\vec{t})$}
			\item if $\TV[\vec{t}] \neq \bot$ then return $\TV[\vec{t}]$
			\item \gamechange{$C \gets \group \setminus \GL$}{}
			\item \gamechange{if $(\exists \vec{s} : \TV[\vec{s}] \neq \bot$ and $\langle \vec{t},\vec{x} \rangle = \langle \vec{s}, \vec{x} \rangle )$ }
			\item \quad \gamechange{then $\bad\gets \true$; \frame{$C \gets \TV[\vec{s}]$}}
			\item $k \gets k+1$; $\vec{t_k} \gets \vec{t}$
			\item $\GL \gets \GL \cup \{C\}$
			\item  $\TV[\vec{t}] \gets C$; $\TI[C] \gets \vec{t}$
			\item return $\TV[\vec{t}]$
		\end{algorithm}
	\end{minipage}
	\begin{minipage}[t]{0.49\textwidth}
		\NewExperiment[\frame{$\lblGm{ggm-newfin}$}, $\lblGm{ggm-nostdh}$]
		
		\ExptSepSpace
		
		\begin{oracle}{$\stDH(A,B)$}
			\item $\vec{a} \gets \VE^{-1}(A)$; $\vec{b} \gets \VE^{-1}(B)$; \gamechange{$a \gets \vec{a}[1]$}{}
			\item \gamechange{if $\vec{a} = a\vec{e_1}$ and $\vec{b} = a \vec{e_2}$ then return $\true$}{}
			\item if $\langle x \vec{a},\vec{x} \rangle = \langle \vec{b}, \vec{x} \rangle$ then \gamechange{$\bad[2] \gets \true$}{}; \frame{return $\true$}
			\item return $\false$
		\end{oracle}
		
		\ExptSepSpace
		\begin{algorithm}{$\VE(\vec{t})$}
			\item if $\TV[\vec{t}] \neq \bot$ then return $\TV[\vec{t}]$
			\item $C \gets \group \setminus \GL$
			\item $k \gets k+1$; $\vec{t_k} \gets \vec{t}$
			\item $\GL \gets \GL \cup \{C\}$
			\item  $\TV[\vec{t}] \gets C$; $\TI[C] \gets \vec{t}$
			\item return $\TV[\vec{t}]$
		\end{algorithm}
		
		\ExptSepSpace
		\begin{oracle}{$\Finalize(Z)$}
			\item \gamechange{if $\exists i,j : 1 \leq i < j \leq k$ and $\langle \vec{t_i}-\vec{t_j},\vec{x} \rangle = 0$}
			\item \quad \gamechange{then return $\true$}
			\item return $[[\VE(x \vec{e_3}) = Z]]$ 
		\end{oracle}
	\end{minipage}
	\begin{minipage}[t]{0.49\textwidth}
		\NewExperiment[$\lblGm{ggm-quadfin}$]
		
		\ExptSepSpace
		\begin{oracle}{$\Initialize()$}
			\item $p \gets |\group|$;
			\item $k\gets 0$;$\one \gets \VE(\vec{0})$; $\generator\gets \VE(\vec{e_1})$
			\item $\X \gets \VE(\vec{e_2})$; $\Y\gets \VE(\vec{e_3})$
			\item return $(\one,\generator,\x,\y)$
		\end{oracle}
		\ExptSepSpace
		\begin{oracle}{$\Finalize(Z)$}
			\item $x,y \getsr \ZZ_p^*$; $\vec{x} \gets (\one, x, y)$
			\item if $\exists i,j : 1 \leq i < j \leq k$ and $\langle \vec{t_i}-\vec{t_j},\vec{x} \rangle = 0$)
			\item \quad then return $\true$
			\item \gamechange{if $\exists i,j : 1 \leq i < j \leq k$}
			\item[] \gamechange{and $\langle x\vec{t_i}-\vec{t_j},\vec{x} \rangle = 0$ or $\langle x \vec{t_j}-\vec{t_i},\vec{x} \rangle 0$}
			\item \quad \gamechange{then return $\true$}
			\item return $[[\VE(x \vec{e_3}) = Z]]$ 
		\end{oracle}
	\end{minipage}
	\begin{minipage}[t]{0.49\textwidth}
		\ExptSepSpace
		
		\begin{oracle}{$\stDH(A,B)$}
			\item $\vec{a} \gets \VE^{-1}(A)$; $\vec{b} \gets \VE^{-1}(B)$;$a \gets \vec{a}[1]$
			\item if $( \vec{a} =a\vec{e_1}$ and $\vec{b} = a \vec{e_2})$ then return $\true$
			\item return $0$
		\end{oracle}
		
		
	\end{minipage}
	
	\caption[]{%
		Top left: Games $\lblGm{ggm-lazysample}$ (changes highlighted in \gamechange{gray}) and $\lblGm{ggm-noprgm}$ (changes highlighted in \frame{frames}) of the strong Diffie--Hellman proof. Top right: Games $\lblGm{ggm-newfin}$ and $\lblGm{ggm-nostdh}$. Bottom: Game $\lblGm{ggm-quadfin}$ (changes highlighted in \gamechange{gray}) of the strong Diffie--Hellman proof.
	}
	\label{fig:GGM-proof:game:lazysample}
	\label{fig:GGM-proof:game:quadfin}
\end{figure}
At this point, we can collect the bounds from each gamehop to see that
\shortlongeqn[.]{
	\Adv^{\strongDH}_{\group}(t, \qstDH)\leq \Pr[\curGm*]
}
Therefore we analyze the advantage of an adversary in game $\curGm*$.
  
We can separately analyze each condition of $\Finalize$. We know that $x$ and $y$ are sampled independently of the $t + 4$ entries of $\TV$. For each index $i \in [1\ldots t+4]$, let $F_i$ be the bivariate linear polynomial over $\ZZ_p$ whose coefficients are given by the vector $\vec{t_i}$. Then for any pair  of vectors $(\vec{t_i}, \vec{t_j})$, the condition $\langle \vec{t_i} - \vec{t_j} \rangle = 0$ holds only if $(1,x,y)$ is a root of $F_i-F_j$. Using Lemma 1 of~\cite{EC:Shoup97} and a union bound over all pairs, the probability of this event is at most $(t+4)^2/p$. 

For the second condition; we see that for any $(\vec{t_i}, \vec{t_j})$, it is true that $\langle x \vec{t_i} - \vec{t_j} \rangle = 0$ only if $(1,x,y)$ is a root of $XF_i - F_j$, which is a bivariate quadratic polynomial over $\ZZ_p$. Again Using Lemma 1 and a union bound, this occurs with probability at most $2(t+4)^2/p$.

If neither event occurs, then the adversary wins only if $[\VE(x\vec{e_3}) = Z]$. Because the second condition failed, we know that $(x\vec{e_3})$ is not an entry in table $\TV$. Therefore the response to $\VE(x \vec{e_3})$ will be sampled uniformly at random, and it will equal $Z$ with probability $1/p$.
%I'm lying here, technically it's $1/(p-t)$. I do not care. 
Then by the union bound, $\Pr[\curGm*] \leq (3(t+4)^2+1)/p$. Collecting the bounds gives the theorem statement for all $t >25$.
\end{proof}