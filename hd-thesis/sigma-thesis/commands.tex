%
%% LNCS format modifiers
%\makeatletter
%\newcommand{\lncsonly}{\relax\iffull\expandafter\@gobble\else\expandafter\@firstofone\fi}
%\newcommand{\lncselse}{\relax\iffull\expandafter\@secondoftwo\else\expandafter\@firstoftwo\fi}
\newcommand{\fullonly}{\relax\iffull\expandafter\@firstofone\else\expandafter\@gobble\fi}
\newcommand{\fullelse}{\relax\iffull\expandafter\@firstoftwo\else\expandafter\@secondoftwo\fi}
%\newcommand{\nfullonly}{\relax\iffull\expandafter\@gobble\else\expandafter\@firstofone\fi}
%\makeatother
%\newcommand{\lncsbreak}{\lncsonly{\allowbreak}}
%\newcommand{\lncsforcebreak}{\lncsonly{\linebreak}}
%\newcommand{\lncsnewline}{\lncsonly{\newline}}
%\newcommand{\lncsdot}{\lncsonly{.}}  % lncs-only ending dot for inline headings (e.g., \subsubsection)

%
% Formatting modifiers
%
\newcommand{\shortlongeqn}[2][]{\[ #2 #1 \]}
\newcommand{\lightparagraph}[1]{\paragraph{#1}}


%
% Fonts
%

\DeclareMathAlphabet{\mathsc}{OT1}{cmr}{m}{sc}


%
% Colors
%
\definecolor{darkblue}{rgb}{0,0,0.5}
\definecolor{darkgreen}{rgb}{0,0.5,0}
\definecolor{darkred}{rgb}{0.5,0,0}

\definecolor{tudred}{RGB}{230,0,26}
\definecolor{tudgreen}{RGB}{153,192,0}
\definecolor{tudorange}{RGB}{245,163,0}
\definecolor{tudblue}{RGB}{0,104,157}
\definecolor{tudpurple}{RGB}{149,17,105}
\definecolor{tudbrown}{RGB}{169,73,19}

%
% Basics
%
\newcommand{\minus}{\text{\normalfont-}}

\newcommand{\sample}{\xleftarrow{\smash{\raisebox{-1.75pt}{$\scriptscriptstyle\$$}}}}
\newcommand{\tor}{\xrightarrow{\smash{\raisebox{-1.75pt}{$\scriptscriptstyle\$$}}}}

\newcommand{\conc}{\|}

%\newcommand{\comment}[1]{\textcolor{gray}{\scriptsize/\!\!/\,#1}}

%
% Word macros
%

% no word-break in n-RTT
\newcommand{\ZRTT}{\mbox{0-RTT}\xspace}
\newcommand{\OneRTT}{\mbox{1-RTT}\xspace}

\newcommand{\SIGMA}{\mbox{SIGMA}\xspace}
\newcommand{\SIGMAI}{\mbox{SIGMA-I}\xspace}

% math-mode macros (in theorems)
\newcommand{\mSIGMAI}{\mathrm{SIGMA{\minus}I}}
\newcommand{\mTLS}{\mathrm{TLS\,1.3}}


%
% Redefine standard commands
%

% \paragraph, \subparagraph: automatically append a dot after paragraph titles
\let\originalparagraph\paragraph
\renewcommand{\paragraph}[1]{%
  \ifthenelse{\endswith{#1}{.}}{%
    \originalparagraph{#1}%
  }{%
    \originalparagraph[#1]{#1.}%
  }%
}

\let\originalsubparagraph\subparagraph
\renewcommand{\subparagraph}[1]{%
  \ifthenelse{\endswith{#1}{.}}{%
    \originalsubparagraph{#1}%
  }{%
    \originalsubparagraph[#1]{#1.}%
  }%
}


%
% Definitions, Theorems, etc.
% (already present in LNCS)
%
%
%\iffull
%\theoremstyle{plain}
%\newtheorem{theorem}{Theorem}[section]
%\newtheorem{lemma}[theorem]{Lemma}
%\newtheorem{proposition}[theorem]{Proposition}
%\newtheorem{corollary}[theorem]{Corollary}
%
 \theoremstyle{definition}
\newtheorem{definition}{Definition}
%\newtheorem{construction}[theorem]{Construction}
%
%\theoremstyle{remark}
%\newtheorem{remark}[theorem]{Remark}
%\newtheorem{example}[theorem]{Example}
%\fi



\newcommand{\NN}{\mathds{N}}
\newcommand{\ZZ}{\mathds{Z}}

%\newcommand{\group}{\mathbb{G}}

\newcommand{\gamefont}[1]{\mathrm{#1}}
%\newcommand{\procfont}[1]{\textsc{#1}}
\newcommand{\setfont}[1]{\mathcal{#1}}

%\newcommand{\Adv}{\mathsf{Adv}}
% \newcommand{\genAdv}[3]{\Adv^{\mathsf{#1}}_{#2}(#3)}
%\newcommand{\genAdv}[3]{\Adv^{\mathsf{#1}}_{#2,#3}}
\newcommand{\genAdvA}[2]{\genAdv{#1}{#2}{\advA}}
%\newcommand{\bad}[1][]{\mathsf{bad}_{#1}}
%\newcommand{\Gm}{\gamefont{G}}
%\newcommand{\advA}{\mathcal{A}}
%\newcommand{\advB}{\mathcal{B}}


% ----- Colors -----
\newcommand{\gray}[1]{{\color{gray}#1}}
\newcommand{\grey}{\gray}

% ----- Highlights, etc. -----


%\definecolor{highlight-gray}{gray}{0.90}

%\newcommand{\gamechange}[1][highlight-gray]{\setlength{\fboxsep}{0pt}\colorbox{#1}}

%
% Macro styles
%

\newcommand{\algostyle}[1]{\mathsf{#1}} % algorithm
\newcommand{\predstyle}[1]{\mathsf{#1}} % predicate
\newcommand{\notionstyle}[1]{\mathsf{#1}} % security notion
\newcommand{\oraclestyle}[1]{\mathsc{#1}} % oracle

\newcommand{\protvarstyle}[1]{#1} % protocol variable
\newcommand{\protvalstyle}[1]{\mathsf{#1}} % protocol value
\newcommand{\gamevarstyle}[1]{\mathsf{#1}} % game variable

\newcommand{\querynum}[1]{q_{\oraclestyle{#1}}} % number of queries to oracle #1 (latter may be abbreviated)


%
% Components
%

\newcommand{\New}{\oraclestyle{New}} % multi-user new key query
\newcommand{\qNew}{\querynum{Nw}}
\newcommand{\Corrupt}{\oraclestyle{Corrupt}} % multi-user corrupt query
\newcommand{\qCorrupt}{\querynum{C}}

% PRFs
\newcommand{\PRF}{\algostyle{PRF}}
\newcommand{\muPRFSEC}{\notionstyle{mu{\minus}PRF}}
\newcommand{\PRFfn}{\oraclestyle{Fn}} % PRF function oracle
\newcommand{\qPRFfn}{\querynum{Fn}}
\newcommand{\qPRFfnU}{\querynum{Fn/U}}
%\newcommand{\FUNC}{\mathsf{FUNC}} % function space

% input labels / constants to PRF etc.
\newcommand{\inputlabel}[1][]{\mathtt{L}_{#1}}
\newcommand{\constant}[1][]{\mathtt{C}_{#1}}

% Hash
\newcommand{\Hash}{\algostyle{H}} % the hash function
\newcommand{\COLL}{\notionstyle{CR}} % collision resistance

% Random oracle
%\newcommand{\RO}{\algostyle{RO}}
\newcommand{\qRO}{\querynum{RO}} % number of RO queries

%Encryption
\newcommand{\ENCScheme}{\algostyle{E}}
\newcommand{\ENCEnc}{\algostyle{Enc}}
\newcommand{\ENCDec}{\algostyle{Dec}}

% Signatures
\newcommand{\SIGScheme}{\algostyle{S}}
\newcommand{\SIGKGen}{\algostyle{KGen}}
\newcommand{\SIGSign}{\algostyle{Sign}}
\newcommand{\SIGVerify}{\algostyle{Vrfy}}
\newcommand{\OSign}{\oraclestyle{Sign}}
\newcommand{\qSign}{\querynum{Sg}}
\newcommand{\qSignU}{\querynum{Sg/U}}

% MACs
\newcommand{\MACScheme}{\algostyle{M}}
\newcommand{\MACKGen}{\algostyle{KGen}}
\newcommand{\MACTag}{\algostyle{Tag}}
\newcommand{\MACVerify}{\algostyle{Vrfy}}
\newcommand{\OTag}{\oraclestyle{Tag}}
\newcommand{\qTag}{\querynum{Tg}}
\newcommand{\qTagU}{\querynum{Tg/U}}
\newcommand{\OVerify}{\oraclestyle{Vrfy}}
\newcommand{\qVerify}{\querynum{V\!f}}
\newcommand{\qVerifyU}{\querynum{V\!f/U}}
\newcommand{\EUFCMA}{\notionstyle{EUF{\minus}CMA}}
\newcommand{\muEUFCMA}{\notionstyle{mu{\minus}EUF{\minus}CMA}}

% HKDF
\newcommand{\HKDF}{\algostyle{HKDF}}
\newcommand{\HMAC}{\algostyle{HMAC}}
\newcommand{\Extract}{\algostyle{Extract}} % HKDF Extract function
\newcommand{\Expand}{\algostyle{Expand}} % HKDF Expand function
\newcommand{\sExtract}{\algostyle{Ext}} % short HKDF Extract function
\newcommand{\sExpand}{\algostyle{Exp}} % short HKDF Expand function

% Strong Diffie--Hellman
\newcommand{\strongDH}{\notionstyle{stDH}} % strongDH assumption
\newcommand{\DDH}{\algostyle{DDH}} % DDH oracle
\newcommand{\qstDH}{\querynum{sDH}} % number of DDH/stDH oracle queries

%
% Key Exchange
%

\newcommand{\KE}{\algostyle{KE}}
\newcommand{\KEKGen}{\algostyle{KGen}}
\newcommand{\KEActivate}{\algostyle{Activate}}
\newcommand{\KERun}{\algostyle{Run}}
\newcommand{\KEkeyspace}{\KE.\algostyle{KS}}
\newcommand{\TLS}{\algostyle{TLS}}

% protocol variables
\def\PROTOCOLVARIABLES{pk,sk,id,peerid,peerpk,role,st,sid,status,skey,state,X,Y,W,N,Recv,E,x,y,r,kl,mk}
\foreach \protvar in \PROTOCOLVARIABLES {%
	\expandafter\xdef\csname\protvar\endcsname{\noexpand\protvarstyle{\protvar}}
}
% some more, manually
\newcommand{\nonce}{\protvarstyle{n}}
\newcommand{\ol}{\protvarstyle{ol}}
\newcommand{\nl}{\protvarstyle{nl}}

% protocol values
\def\PROTOCOLVALUES{running,accepted,rejected,initiator,responder}
\foreach \protval in \PROTOCOLVALUES {%
	\expandafter\xdef\csname\protval\endcsname{\noexpand\protvalstyle{\protval}}
}

\newcommand{\labelis}{\inputlabel[is]}
\newcommand{\labelim}{\inputlabel[im]}
\newcommand{\labelrs}{\inputlabel[rs]}
\newcommand{\labelrm}{\inputlabel[rm]}
\newcommand{\ks}{k_s}
\newcommand{\kt}{k_t}
\newcommand{\ke}{k_e}
\newcommand{\Zz}{Z}
\newcommand{\ciph}{c}
\newcommand{\sidhash}{d}

% StrongDH
\newcommand{\stDH}{\mathsf{stDH}} % strongDH oracle

%
% SIGMA
%
\newcommand{\RunInitI}{\algostyle{RunInit1}}
\newcommand{\RunInit}{\algostyle{RunInit2}}
\newcommand{\RunRespI}{\algostyle{RunResp1}}
\newcommand{\RunRespII}{\algostyle{RunResp2}}


%
% Key Exchange Model
%

\newcommand{\KESEC}{\notionstyle{KE{\minus}SEC}}

% Oracles
%\newcommand{\Initialize}{\oraclestyle{Initialize}}
%\newcommand{\Finalize}{\oraclestyle{Finalize}}
\newcommand{\Send}{\oraclestyle{Send}}
\newcommand{\NewUser}{\oraclestyle{NewUser}}
\newcommand{\RevSessionKey}{\oraclestyle{RevSessionKey}}
\newcommand{\RevLongTermKey}{\oraclestyle{RevLongTermKey}}
\newcommand{\Test}{\oraclestyle{Test}}

% number of queries
\newcommand{\qNewUser}{\querynum{N}}
\newcommand{\qSend}{\querynum{S}}
\newcommand{\qRevSessionKey}{\querynum{RS}}
\newcommand{\qRevLongTermKey}{\querynum{RL}}
\newcommand{\qTest}{\querynum{T}}
\newcommand{\qRevState}{\querynum{RSt}}

% Predicates
\newcommand{\Sound}{\predstyle{Sound}}
\newcommand{\ExplicitAuth}{\predstyle{ExplicitAuth}}
\newcommand{\Fresh}{\predstyle{Fresh}}

% game variables
\def\GAMEVARIABLES{revealed,tested,revltk,time,users,S,Q,Sent}
\foreach \gamevar in \GAMEVARIABLES {%
	\expandafter\xdef\csname\gamevar\endcsname{\noexpand\gamevarstyle{\gamevar}}
}
% some more, manually
\newcommand{\taccepted}{\gamevarstyle{t_{acc}}}


%
% Game hop proofs, environments, counters
%

% initiate a proof, giving it a (reference) name
\newcommand{\startproof}[1]{%
	\xdef\currentproof{#1}
	\firstproofngame
}
\newcommand{\currentproof}{}

\newcounter{proofngame} % counter for games in proofs without multiple cases

\renewcommand*{\theHproofngame}{\currentproof.\theproofngame} % make proofngame counter references unique per proof (for hyperref)

\newcommand{\curgame}{}  % label of current game
\newcommand{\prevgame}{} % label of previous game

\newcommand{\firstproofngame}[1][0]{\setcounter{proofngame}{-1+#1}} % reset counter to be 0 next time
\newcommand{\proofngame}[1][]{%  %%% optional argument is used as \label
	\refstepcounter{proofngame}%
	\ifthenelse{\isempty{#1}}{}{%
		\label{game:\currentproof:#1}%
		\edef\prevgame{\curgame}%
		\edef\curgame{game:\currentproof:#1}%
	}%
	\proofparagraph{Game \theproofngame}
}
\newcommand{\proofngames}[2][]{%  %%% (second) argument is number of game steps, optional argument is used as \label for the last gamd
	\stepcounter{proofngame}
	\xdef\oldproofngame{\theproofngame}
	\addtocounter{proofngame}{#2} % add number of game steps
	\addtocounter{proofngame}{-2} % -2 for step before and \ref*step* counter
	\refstepcounter{proofngame}%
	\ifthenelse{\isempty{#1}}{}{%
		\label{game:\currentproof:#1}%
		\edef\prevgame{\curgame}%
		\edef\curgame{game:\currentproof:#1}%
	}%
	\proofparagraph{Games \oldproofngame--\theproofngame}
}

% spacing between proof paragraphs
\newcommand{\proofsep}{\iffull\bigskip\else\medskip\fi}

% proof paragraph style
\makeatletter
\newcommand\gobblepars{% removes \par after command, used to make \proofparagraph a run-in heading
    \@ifnextchar\par%
        {\expandafter\gobblepars\@gobble}%
        {}}
\newcommand{\proofparagraph}[1]{%
% 	\iffullversion\bigskip\else\smallskip\fi
% 	\noindent\textit{#1.}\hspace{1em}\@afterindentfalse\@afterheading\gobblepars} %% this seems to not indent the first paragraph following the paragraph start
	
	\proofsep
	\noindent\textbf{#1.}\hspace{1em}\gobblepars}
\makeatother

% macros for the current, previous, or labeled games
\makeatletter
\newcommand{\curGm}{\@ifstar{\@curGmNL}{\@curGmL}}
	\newcommand{\@curGmNL}{\Gm_{\ref*{\curgame}}} % without hyperlink
	\newcommand{\@curGmL}{\Gm_{\ref{\curgame}}} % with hyperlink
\newcommand{\prevGm}{\@ifstar{\@prevGmNL}{\@prevGmL}}
	\newcommand{\@prevGmNL}{\Gm_{\ref*{\prevgame}}} % without hyperlink
	\newcommand{\@prevGmL}{\Gm_{\ref{\prevgame}}} % with hyperlink
\newcommand{\lblGm}{\@ifstar{\@lblGmNL}{\@lblGmL}}
	\newcommand{\@lblGmNL}[1]{\Gm_{\ref*{game:\currentproof:#1}}} % without hyperlink
	\newcommand{\@lblGmL}[1]{\Gm_{\ref{game:\currentproof:#1}}} % with hyperlink
\makeatother

% macros for current and new \advB reduction
\newcommand{\curadvB}{\advB_{\arabic{advB-\currentproof}}} % current \advB_x (does not increment counter)
\newcommand{\newadvB}{\stepcounter{advB-\currentproof}\curbdv} % new \advB_x (increments counter)



%
% Protocol Figures
%

% For any new figure, you to set X coordinates for client, server, and arrows; and initial Y coordinate, e.g.:
% 	\edef\ClientX{0}
% 	\edef\ArrowLeft{0}
% 	\edef\ArrowRight{9}
% 	\edef\ServerX{9}
% 	\edef\Y{0}

% ClientAction and ServerAction
% Print a command executed by the client/server
% 1st argument: the text
\newcommand{\ClientAction}[2][]{
	\node[right,#1] at (\ClientX, \Y) {#2};
}
\newcommand{\ServerAction}[2][]{
	\node[left,#1] at (\ServerX, \Y) {#2};
}
\newcommand{\SharedAction}[2][]{
	\node[#1] at ($1/2*(\ClientX, \Y)+1/2*(\ServerX, \Y)$) {#2};
}

% ClientToServer and ServerToClient
% Draws a message flow from client-to-server or server-to-client, with text above and below
% 1st argument (optional): line type, default ->
% 2nd argument: text above
% 3rd argument: text below
% Example: \ClientToServer{$Y$}{}
% Example: \ClientToServer[<->,double]{$Y$}{over an encrypted channel}
\newcommand{\ClientToServer}[3][->]{
	\NextLine[0.5]
	\draw[>=latex, very thick, #1] (\ArrowLeft+1,\Y) -- node[above=-0.1] {#2} node[below] {#3} (\ArrowRight-1,\Y) ;
	\NextLine[0.25]
}
\newcommand{\ServerToClient}[3][->]{
	\NextLine[0.5]
	\draw[>=latex, very thick, #1] (\ArrowRight-1,\Y) -- node[above=-0.1] {#2} node[below] {#3} (\ArrowLeft+1,\Y) ;
	\NextLine[0.25]
}

% Spacing factor for NextLines
\iffull
\def\NextLineSpacing{0.55}
\else
\def\NextLineSpacing{0.45}
\fi

% NextLine
% 1st argument (optional): amount of spacing to increment by, default 1.0
% Example: \NextLine
% Example: \NextLine[1.5]
\newcommand{\NextLine}[1][1.0]{
	\pgfmathparse{\Y-\NextLineSpacing*#1}
	\edef\Y{\pgfmathresult}
}

%
% stage separator line
%
\newcommand{\StageSeparator}[1]{
	\draw[very thick,dotted,StageSeparatorColor] (\ArrowLeft,\Y-0.5*\NextLineSpacing) -- (\ArrowRight+0.15,\Y-0.5*\NextLineSpacing) node[right,anchor=west,font=\footnotesize] {stage~#1} ;
}

%
% accept key and stage
%
\newcommand{\AcceptStage}[2]{
	\SharedAction{{\color{StageSeparatorColor}\textbf{accept} #2}}
	\StageSeparator{#1}
}



%%%%%%%%%%%%%%%%%%%%%%%%%%%%%%%%%%%%%%%%%%%%%%%%%%%%%%%%%%%%%%%%%%%%%%%%%%%%%%%%%%%%%%%%%%
% TLS
%%%%%%%%%%%%%%%%%%%%%%%%%%%%%%%%%%%%%%%%%%%%%%%%%%%%%%%%%%%%%%%%%%%%%%%%%%%%%%%%%%%%%%%%%%


%
% TLS messages
%

% long versions
\newcommand{\CHELO}{\mathtt{ClientHello}}
\newcommand{\SHELO}{\mathtt{ServerHello}}
\newcommand{\CKEYS}{\mathtt{ClientKeyShare}}
\newcommand{\SKEYS}{\mathtt{ServerKeyShare}}
%\newcommand{\CPSK}{\mathtt{ClientPreSharedKey}}
%\newcommand{\SPSK}{\mathtt{ServerPreSharedKey}}
\newcommand{\ENCEX}{\mathtt{EncryptedExtensions}}
\newcommand{\CERTR}{\mathtt{CertificateRequest}}
\newcommand{\SCERT}{\mathtt{ServerCertificate}}
\newcommand{\SCERTV}{\mathtt{ServerCertificateVerify}}
\newcommand{\CCERT}{\mathtt{ClientCertificate}}
\newcommand{\CCERTV}{\mathtt{ClientCertificateVerify}}
\newcommand{\SFIN}{\mathtt{ServerFinished}}
\newcommand{\CFIN}{\mathtt{ClientFinished}}

% common names
\newcommand{\HELO}{\mathtt{Hello}}
\newcommand{\KEYS}{\mathtt{KeyShare}}
\newcommand{\PSKS}{\mathtt{PreSharedKey}}
\newcommand{\CERT}{\mathtt{Certificate}}
\newcommand{\CERTV}{\mathtt{CertificateVerify}}
\newcommand{\FIN}{\mathtt{Finished}}

% middle versions
\newcommand{\mCERTR}{\mathtt{CertRequest}}
\newcommand{\mSCERT}{\mathtt{ServerCert}}
\newcommand{\mSCERTV}{\mathtt{ServerCertVfy}}
\newcommand{\mCCERT}{\mathtt{ClientCert}}
\newcommand{\mCCERTV}{\mathtt{ClientCertVfy}}
\newcommand{\mSFIN}{\mathtt{ServerFin}}
\newcommand{\mCFIN}{\mathtt{ClientFin}}

% short versions
\newcommand{\sCHELO}{\mathtt{CH}}
\newcommand{\sSHELO}{\mathtt{SH}}
\newcommand{\sCKEYS}{\mathtt{CKS}}
\newcommand{\sSKEYS}{\mathtt{SKS}}
\newcommand{\sCPSK}{\mathtt{CPSK}}
\newcommand{\sSPSK}{\mathtt{SPSK}}
\newcommand{\sENCEX}{\mathtt{EE}}
\newcommand{\sCERTR}{\mathtt{CR}}
\newcommand{\sSCERT}{\mathtt{SCRT}}
\newcommand{\sSCERTV}{\mathtt{SCV}}
\newcommand{\sCCERT}{\mathtt{CCRT}}
\newcommand{\sCCERTV}{\mathtt{CCV}}
\newcommand{\sSFIN}{\mathtt{SF}}
\newcommand{\sCFIN}{\mathtt{CF}}


%
% TLS keys and computed values
%

\newcommand{\DHE}{\mathrm{DHE}} % Diffie--Hellman shared value
\newcommand{\PSK}{\mathrm{PSK}} % pre-shared key

\newcommand{\ES}{\mathrm{ES}}   % early secret
\newcommand{\dES}{\mathrm{dES}} % "derived" early secret
\newcommand{\HS}{\mathrm{HS}}   % handshake secret
\newcommand{\dHS}{\mathrm{dHS}} % "derived" handshake secret
\newcommand{\MS}{\mathrm{MS}}   % master secret

\newcommand{\BK}{\mathrm{BK}}     % binder key
\newcommand{\ETS}{\mathrm{ETS}}   % early traffic secret
\newcommand{\EEMS}{\mathrm{EEMS}} % early exporter master secret

\newcommand{\HTS}{\mathrm{HTS}}   % handshake traffic secret
\newcommand{\CHTS}{\mathrm{CHTS}} % client handshake traffic secret
\newcommand{\SHTS}{\mathrm{SHTS}} % server handshake traffic secret
\newcommand{\CFK}{\mathrm{fk}_C}  % client finished key
\newcommand{\SFK}{\mathrm{fk}_S}  % server finished key

\newcommand{\CATS}{\mathrm{CATS}} % client application traffic secret
\newcommand{\SATS}{\mathrm{SATS}} % server application traffic secret
\newcommand{\EMS}{\mathrm{EMS}}   % exporter master secret
\newcommand{\RMS}{\mathrm{RMS}}   % resumption master secret
\newcommand{\ATS}{\mathrm{ATS}}   % generalized application traffic secret (combining client+server)

\newcommand{\tkead}{\mathrm{tk}_{\text{eapp}}}    % early application data traffic key
\newcommand{\tkchs}{\mathrm{tk}_{\text{chs}}}     % client handshake traffic key
\newcommand{\tkshs}{\mathrm{tk}_{\text{shs}}}     % server handshake traffic key
\newcommand{\tkcapp}{\mathrm{tk}_{\text{capp}}}   % client application traffic key
\newcommand{\tksapp}{\mathrm{tk}_{\text{sapp}}}   % server application traffic key


%
% TLS figure commands
%
\newcommand{\TLSmsg}[1]{{\color{TLSmsgcolor}#1}}
\newcommand{\PSKECDHEonly}[2][0]{{\color{PSKECDHEonlycolor}[#2]$^{\dagger}$}}
\newcommand{\PSKonly}[2][0]{{\color{PSKonlycolor}[#2]$^{\diamond}$}}

%
% TLS colors
%
\definecolor{TLSmsgcolor}{named}{OliveGreen}
\definecolor{StageSeparatorColor}{named}{MidnightBlue}
\definecolor{PSKonlycolor}{named}{BurntOrange}
\definecolor{PSKECDHEonlycolor}{named}{BrickRed}

\definecolor{inputsecretcolor}{named}{tudorange}
\definecolor{internalcolor}{named}{tudred}
\definecolor{stagekeycolor}{named}{tudblue}
\definecolor{hkdfcolor}{named}{tudgreen}

%
% GGM 
% 

\newcommand{\OP}{\oraclestyle{OP}}
\newcommand{\Bijections}{\mathrm{Bijections}}
\newcommand{\sgn}{\text{ sgn }}
\newcommand{\GL}{GL}
\newcommand{\generator}{g}
\newcommand{\one}{\mathds{1}}
\newcommand{\VE}{\oraclestyle{VE}}
\newcommand{\TV}{TV}
\newcommand{\TI}{TI}

\DeclareMathVersion{normal2}