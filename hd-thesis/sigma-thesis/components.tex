\section{Assumptions, Building Blocks, \fullonly{and }Multi-User Security}
\iffull \label{sec:components} \else \label{apx:components} \fi

\iffull
Before we continue to our main technical results, let us briefly introduce notation and discuss the multi-user security of the involved building blocks: strong Diffie--Hellman (including the GGM bound we prove), PRFs, digital signatures, MAC schemes, and hash functions.
\fi


\iffull
\subsection{Decisional and Strong Diffie--Hellman}
% \iffull \label{sec:components:DH} \else \label{apx:components:DH} \fi
\label{sec:components:DH}

The classical decisional Diffie--Hellman assumption~\cite{Boneh98} states that, when only observing the two Diffie--Hellman shares~$g^x$, $g^y$, the resulting secret~$g^{xy}$ is indistinguishable from a random group element.

\begin{definition}[Decisional Diffie--Hellman ($\DDH$) assumption]
	\label{def:DDH}
	Let $\group = \langle g \rangle$ be a cyclic group of prime order~$p$.
	We define
	\begin{align*}
		\Adv^{\DDH}_{\group}(t) := \max_\advA \Big|
			&\Pr \left[ \advA(\group, g, g^x, g^y, g^{xy}) \Rightarrow 1 \;\middle|\; x, y \sample \ZZ_p \right] - \\
			&\Pr \left[ \advA(\group, g, g^x, g^y, g^z) \Rightarrow 1 \;\middle|\; x, y, z \sample \ZZ_p \right]
			\Big|,
	\end{align*}
	where the maximum is taken over all adversaries, denoted \emph{$(t)$-$\DDH$-adversaries} running in time at most~$t$.
\end{definition}

The strong Diffie--Hellman assumption, a weakening of the gap Diffie--Hellman assumption~\cite{PKC:OkaPoi01}, states that solving the computational Diffie--Hellman problem given a restricted decisional Diffie--Hellman oracle is hard.

\begin{definition}[Strong Diffie--Hellman assumption~\cite{PKC:OkaPoi01}]
	\label{def:strongDH}
	Let $\group = \langle g \rangle$ be a cyclic group of prime order~$p$.
	Let $\DDH(X, Y, Z) := [[ X^{\log_g(Y)} = Z ]]$ be a decisional Diffie--Hellman oracle.
	We define
	\[
		\Adv^{\strongDH}_{\group}(t, \qstDH) := \max_\advA \Pr \left[ \advA^{\DDH(g^x, \cdot, \cdot)}(\group, g, g^x, g^y) = g^{xy} \;\middle|\; x, y \sample \ZZ_p \right],
	\]
	where the maximum is taken over all adversaries, denoted \emph{$(t, \qstDH)$-$\strongDH$-adversaries} running in time at most~$t$ and making at most $\qstDH$ queries to their $\DDH$ oracle.
\end{definition}

The strong (or gap) Diffie--Hellman assumption has been deployed in numerous works to analyze practical key exchange designs, directly or through the PRF-ODH assumption~\cite{C:JKSS12,C:BFGJ17} it supports, including \cite{C:JKSS12,CCS:FisGue14,CCS:DFGS15,EuroSP:KraWee16,EPRINT:DFGS16,EuroSP:FisGue17,JC:DFGS21} as well as in the closely related works on practical tightness by Cohn-Gordon et al.~\cite{C:CCGJJ19} and Diemert and Jager~\cite{JC:DieJag20}.
To argue that it is reasonable to rely on the strong Diffie--Hellman assumption, we turn to the generic group model~\cite{EC:Shoup97,IMA:Maurer05}.
Although some known algorithms for solving discrete logarithms in finite fields like index calculus fall outside the generic group model, the best known algorithms for elliptic curve groups are generic.
Shoup~\cite{EC:Shoup97} proved that, in the generic group model, any adversary computing at most $t$ group operations in a group of prime order~$p$ has advantage at most $\mathcal{O}(t^2/p)$ in solving the discrete logarithm problem or the computational or decisional Diffie--Hellman problem in that group.
We claim, and prove\fullonly{ in Appendix~\ref{apx:strongDHproof}}, that any adversary in the generic group model making at most $t$ group operations and $\DDH$ oracle queries, also has advantage at most $\mathcal{O}(t^2/p)$ in solving the strong Diffie--Hellman problem.

\begin{theorem}
	\label{thm-ggm-bound}
	Let $\group$ be a group with prime order $p$.
	In the generic group model,
	$\Adv_\group^{\strongDH}(t,q) \leq 4t^2 / p$.
\end{theorem}
\fi

\iffull
\subsection{Multi-User PRF Security}
\label{sec:components:muPRF}

Let us recap the multi-user security notion for pseudorandom functions (PRFs)~\cite{FOCS:BelCanKra96}.
\else
\label{apx:components:muPRF}
\fi

\begin{definition}[Multi-user PRF security]
	\label{def:muPRFSEC}
	Let $\PRF \colon \bits^k \times \bits^m \to \bits^n$ be a function (for $k, n \in \NN$ and $m \in \NN \cup \{*\}$)
	and $\Gm^{\muPRFSEC}_{\PRF,\advA}$ be the multi-user PRF security game defined as in Figure~\ref{fig:muPRFSEC}.
	We define
	\shortlongeqn[,]{
		\Adv^{\muPRFSEC}_{\PRF}(t\cab \qNew\cab \qPRFfn\cab\qPRFfnU) := 2 \cdot \max_\advA \Pr \left[ \Gm^{\muPRFSEC}_{\PRF,\advA} \Rightarrow 1 \right] - 1
	}
	where the maximum is taken over all adversaries, denoted \emph{$(t, \qNew, \qPRFfn,\qPRFfnU)$-$\muPRFSEC$-adversaries}, running in time at most~$t$ and making at most $\qNew$ queries to their $\New$ oracle, at most $\qPRFfn$ total queries to their $\PRFfn$ oracle, and at most $\qPRFfnU$ queries $\PRFfn(i, \cdot)$ for any user~$i$.
\end{definition}

Generically, the multi-user security of PRFs reduces to single-user security (formally, $\Gm^{\muPRFSEC}_{\PRF,\advA}$ with $\advA$ restricted to $\qNew = 1$ queries to~$\New$) with a factor in the number of users via a hybrid argument~\cite{FOCS:BelCanKra96}, i.e., 
\shortlongeqn[,]{
	\Adv^{\muPRFSEC}_{\PRF}(t, \qNew, \qPRFfn,\qPRFfnU) \leq \qNew \cdot \Adv^{\muPRFSEC}_{\PRF}(t', 1, \qPRFfnU, \qPRFfnU)
}
where $t \approx t'$. 
(Note that the total number~$\qPRFfn$ of queries to the $\PRFfn$ oracle across all users does not affect the reduction.)
There exist simple and efficient constructions, like AMAC~\cite{EC:BelBerTes16}, that however achieve multi-user security tightly.
\iffull

\fi
If we use a random oracle~$\RO$ as a PRF with key length~$kl$, then
\shortlongeqn[.]{
	\Adv^{\muPRFSEC}_{\RO}(t\cab \qNew\cab \qPRFfn\cab \qPRFfnU\cab \qRO) \leq \frac{\qNew \cdot \qRO}{2^{kl}}
}

\begin{figure}[t]
	\centering
	
	%%% Signatures
	\begin{minipage}[t]{0.2\textwidth}
		\NewExperiment[$\Gm^{\muPRFSEC}_{\PRF,\advA}$]
		
		\begin{oracle}{$\Initialize$}
			\item $b \sample \bits$
			\item $u \gets 0$
		\end{oracle}
	\end{minipage}
	%
	\begin{minipage}[t]{\iffull0.3\else0.5\fi\textwidth}
		\begin{oracle}{$\New$}
			\item $u \gets u + 1$
			\item if $b = 1$ then
			\item \hindent $K_u \getsr \bits^k$
			\fullelse{\item \hindent}{;} $f_u := \PRF(K_u, \cdot)$
			\item else
			\fullonly{\item \hindent} $f_u \sample \FUNC$
		\end{oracle}
	\end{minipage}
	%
	\begin{minipage}[t]{0.25\textwidth}
		\begin{oracle}{$\PRFfn(i, x)$}
			\item return $f_i(x)$
		\end{oracle}
		
		\ExptSepSpace
		
		\begin{oracle}{$\Finalize(b^*)$}
			\item return $[[ b = b^* ]]$
		\end{oracle}
	\end{minipage}
	
	\caption{%
		Multi-user PRF security of a pseudorandom function~$\PRF \colon \bits^k \times \bits^m \to \bits^n$. $\FUNC$ is the space of all functions $\bits^m \to \bits^n$.
	}
	\label{fig:muPRFSEC}
\end{figure}


\iffull
\subsection{Multi-User Unforgeability with Adaptive Corruptions of Signatures and MACs}
\label{sec:components:muSigMac}

We recap the definition of digital signature schemes and message authentication codes (MACs) as well as the natural extension of classical \emph{existential unforgeability under chosen-message attacks}~\cite{GolMicRiv88} to the \emph{multi-user} setting with \emph{adaptive corruptions}.
For signatures, this notion was considered by Bader et al.~\cite{TCC:BHJKL15} and, without corruptions, by Menezes and Smart~\cite{DCC:MenSma04}.

\else
\label{apx:components:muSigMAC}
\fi


\iffull
\begin{definition}[Signature scheme]
	\label{def:signature-scheme}
	A \emph{signature scheme} $\SIGScheme = (\SIGKGen\cab \SIGSign\cab \SIGVerify)$ consists of three efficient algorithms defined as follows.
	\begin{itemize}
		\item $\SIGKGen() \tor (pk, sk)$.
			This probabilistic algorithm generates a public verification key~$pk$ and a secret signing key~$sk$.
		
		\item $\SIGSign(sk, m) \tor \sigma$.
			On input a signing key~$sk$ and a message~$m$, this (possibly) probabilistic algorithm outputs a signature~$\sigma$.
		
		\item $\SIGVerify(vk, m, \sigma) \to d$.
			On input a verification key~$pk$, a message~$m$, and a signature~$\sigma$, this deterministic algorithm outputs a decision bit~$d \in \bits$ (where $d = 1$ indicates validity of the signature).
	\end{itemize}
\end{definition}
\fi


\begin{definition}[Signature $\muEUFCMA$ security]
	\label{def:signature-muEUFCMA}
	Let $\SIGScheme$ be a signature scheme
	and $\Gm^{\muEUFCMA}_{\SIGScheme,\advA}$ be the game for signature multi-user existential unforgeability under chosen-message attacks with adaptive corruptions \fullelse{defined as in Figure~\ref{fig:muEUFCMA}}{(see the full version~\cite{EPRINT:DavGun20} for the formal definition)}.
	We define
	\shortlongeqn[,]{
		\Adv^{\muEUFCMA}_{\SIGScheme}(t\cab \qNew\cab \qSign\cab \qSignU\cab \qCorrupt) := \max_\advA \Pr \left[ \Gm^{\muEUFCMA}_{\SIGScheme,\advA} \Rightarrow 1 \right]
	}
	where the maximum is taken over all adversaries, denoted \emph{$(t, \qNew, \qSign, \qSignU, \qCorrupt)$-}\emph{$\muEUFCMA$-adversaries}, running in time at most~$t$ and making at most $\qNew$, $\qSign$, resp.\ $\qCorrupt$ total queries to their $\New$, $\OSign$, resp.\ $\Corrupt$ oracle, and making at most $\qSignU$ queries $\OSign(i, \cdot)$ for any user~$i$.
\end{definition}


Multi-user EUF-CMA security of signature schemes (with adaptive corruptions) can be reduced to classical, single-user EUF-CMA security (formally, $\Gm^{\muEUFCMA}_{\SIGScheme,\advA}$ with $\advA$ restricted to $\qNew = 1$ queries to~$\New$) by a standard hybrid argument, losing a factor of number of users.
Formally, this yields
\shortlongeqn[,]{
	\Adv^{\muEUFCMA}_{\SIGScheme}(t\cab \qNew\cab \qSign\cab \qSignU\cab \qCorrupt) \leq \qNew \cdot \Adv^{\muEUFCMA}_{\SIGScheme}(t'\cab 1\cab \qSignU\cab \qSignU\cab 0)
}
where $t \approx t'$.
(Note that the reduction is not affected by the total number of signature queries~$\qSign$ across all users.)
In many cases, such loss is indeed unavoidable~\cite{EC:BJLS16}.




% \begin{figure}[t]
% 	\centering
% 	
% 	%%% Signatures
% 	\begin{minipage}[t]{0.48\textwidth}
% 		\NewExperiment[$\Gm^{\muEUFCMA}_{\SIGScheme,\advA}$]
% 		
% 		\begin{oracle}{$\Initialize$}
% 			\item $Q \gets \emptyset$
% 			\item $\setfont{C} \gets \emptyset$
% 			\item $u \gets 0$
% 		\end{oracle}
% 		
% 		\ExptSepSpace
% 		
% 		\begin{oracle}{$\New$}
% 			\item $u \gets u + 1$
% 			\item $(pk_u, sk_u) \getsr \SIGKGen()$
% 			\item return $pk_u$
% 		\end{oracle}
% 		
% 		\ExptSepSpace
% 		
% 		\begin{oracle}{$\OSign(i, m)$}
% 			\item $\sigma \getsr \SIGSign(sk_i, m)$
% 			\item $Q \gets Q \cup \{(i, m)\}$
% 			\item return $\sigma$
% 		\end{oracle}
% 		
% 		\ExptSepSpace
% 		
% 		\vspace*{1.65cm} %% align with MAC game (Verify oracle)
% 		
% 		\begin{oracle}{$\Corrupt(i)$}
% 			\item $\setfont{C} \gets \setfont{C} \cup \{i\}$
% 			\item return $sk_i$
% 		\end{oracle}
% 		
% 		\ExptSepSpace
% 		
% 		\begin{oracle}{$\Finalize(i^*, m^*, \sigma^*)$}
% 			\item $d^* \gets \SIGVerify(pk_{i^*}, m^*, \sigma^*)$
% 			\item return $[[ d^* = 1 \land i^* \notin \setfont{C} \land (i^*\!, m^*) \notin Q ]]$
% 		\end{oracle}
% 	\end{minipage}
% 	%
% 	\hspace*{0.25cm}
% 	%
% 	%%% MACs
% 	\begin{minipage}[t]{0.48\textwidth}
% 		\NewExperiment[$\Gm^{\muEUFCMA}_{\MACScheme,\advA}$]
% 		
% 		\begin{oracle}{$\Initialize$}
% 			\item $Q \gets \emptyset$
% 			\item $\setfont{C} \gets \emptyset$
% 			\item $u \gets 0$
% 		\end{oracle}
% 		
% 		\ExptSepSpace
% 		
% 		\begin{oracle}{$\New$}
% 			\item $u \gets u + 1$
% 			\item $K_u \getsr \MACKGen()$
% 			\item[]
% 		\end{oracle}
% 		
% 		\ExptSepSpace
% 		
% 		\begin{oracle}{$\OTag(i, m)$}
% 			\item $\tau \getsr \MACTag(K_i, m)$
% 			\item $Q \gets Q \cup \{(i, m)\}$
% 			\item return $\tau$
% 		\end{oracle}
% 		
% 		\ExptSepSpace
% 		
% 		\begin{oracle}{$\OVerify(i, m, \tau)$}
% 			\item $d \gets \MACVerify(K_i, m, \tau)$
% 			\item return $d$
% 		\end{oracle}
% 		
% 		\ExptSepSpace
% 		
% 		\begin{oracle}{$\Corrupt(i)$}
% 			\item $\setfont{C} \gets \setfont{C} \cup \{i\}$
% 			\item return $K_i$
% 		\end{oracle}
% 		
% 		\ExptSepSpace
% 		
% 		\begin{oracle}{$\Finalize(i^*, m^*, \tau^*)$}
% 			\item $d^* \gets \MACVerify(K_{i^*}, m^*, \tau^*)$
% 			\item return $[[ d^* = 1 \land i^* \notin \setfont{C} \land (i^*\!, m^*) \notin Q ]]$
% 		\end{oracle}
% 	\end{minipage}
% 	
% 	\caption{%
% 		Multi-user existential unforgeability ($\muEUFCMA$) of signature schemes (left) and MAC schemes (right).
% 	}
% 	\label{fig:muEUFCMA}
% \end{figure}


\iffull
%%% mu-EUF-CMA experiments with tighter layout
\begin{figure}[t]
	\centering
	
	%%% Signature mu-EUF-CMA
	\begin{minipage}[t]{0.2\textwidth}
		\NewExperiment[$\Gm^{\muEUFCMA}_{\SIGScheme,\advA}$]
		
		\begin{oracle}{$\Initialize$}
			\item $Q \gets \emptyset$
			\item $\setfont{C} \gets \emptyset$
			\item $u \gets 0$
		\end{oracle}
		\ExptSepSpace
			\begin{oracle}{$\Corrupt(i)$}
			\item $\setfont{C} \gets \setfont{C} \cup \{i\}$
			\item return $sk_i$
		\end{oracle}
	\end{minipage}
	\begin{minipage}[t]{0.3\textwidth}
		\vspace*{\iffull0.4cm\else0cm\fi}

		\begin{oracle}{$\New$}
			\item $u \gets u + 1$
			\item $(pk_u, sk_u) \getsr \SIGKGen()$
			\item return $pk_u$
		\end{oracle}		
		\ExptSepSpace
	%	\vspace*{1.65cm} %% align with MAC game (Verify oracle)
		\begin{oracle}{$\OSign(i, m)$}
			\item $\sigma \getsr \SIGSign(sk_i, m)$
			\item $Q \gets Q \cup \{(i, m)\}$
			\item return $\sigma$
		\end{oracle}
		
% 		\ExptSepSpace
	\end{minipage}
	\begin{minipage}[t]{0.47\textwidth}
		\vspace*{.4cm}
		
		\begin{oracle}{$\Finalize(i^*, m^*, \sigma^*)$}
			\item $d^* \gets \SIGVerify(pk_{i^*}, m^*, \sigma^*)$
			\item return $[[ d^* = 1 \land i^* \notin \setfont{C} \land (i^*\!, m^*) \notin Q ]]$
		\end{oracle}
	\end{minipage}
	
	\vspace*{0.1cm}
	
	\hrulefill
	
	%%% MACs mu-EUF-CMA
	\begin{minipage}[t]{0.2\textwidth}
		\NewExperiment[$\Gm^{\muEUFCMA}_{\MACScheme,\advA}$]
		
		\begin{oracle}{$\Initialize$}
			\item $Q \gets \emptyset$
			\item $\setfont{C} \gets \emptyset$
			\item $u \gets 0$
		\end{oracle}
		
		\ExptSepSpace
		
			\begin{oracle}{$\Corrupt(i)$}
			\item $\setfont{C} \gets \setfont{C} \cup \{i\}$
			\item return $K_i$
		\end{oracle}
		\ExptSepSpace
	
			\end{minipage}
	\begin{minipage}[t]{0.3\textwidth}
		\vspace*{\iffull0.4cm\else0cm\fi}
		\begin{oracle}{$\New$}
			\item $u \gets u + 1$
			\item $K_u \getsr \MACKGen()$
			\item[]
		\end{oracle}
		\ExptSepSpace
		\begin{oracle}{$\OTag(i, m)$}
			\item $\tau \getsr \MACTag(K_i, m)$
			\item $Q \gets Q \cup \{(i, m)\}$
			\item return $\tau$
		\end{oracle}
		
		
		\ExptSepSpace
			\end{minipage}
	\begin{minipage}[t]{0.47\textwidth}
		\vspace*{.4cm}
		
		\begin{oracle}{$\OVerify(i, m, \tau)$}
			\item $d \gets \MACVerify(K_i, m, \tau)$
			\item return $d$
			\item[]
		\end{oracle}
		\ExptSepSpace
		
		\begin{oracle}{$\Finalize(i^*, m^*, \tau^*)$}
			\item $d^* \gets \MACVerify(K_{i^*}, m^*, \tau^*)$
			\item return $[[ d^* = 1 \land i^* \notin \setfont{C} \land (i^*\!, m^*) \notin Q ]]$
		\end{oracle}
	\end{minipage}
	
	\caption{%
		Multi-user existential unforgeability ($\muEUFCMA$) of signature schemes (top) and MAC schemes (bottom).
	}
	\label{fig:muEUFCMA}
\end{figure}
\fi


\iffull
\begin{definition}[MAC scheme]
	\label{def:MAC-scheme}
	A \emph{MAC scheme} $\MACScheme = (\MACKGen\cab \MACTag\cab \MACVerify)$ consists of three efficient algorithms defined as follows.
	\begin{itemize}
		\item $\MACKGen() \tor K$.
			This probabilistic algorithm generates a key~$K$.
		
		\item $\MACTag(K, m) \tor \tau$.
			On input a key~$K$ and a message~$m$, this (possibly) probabilistic algorithm outputs a message authentication code (MAC)~$\tau$.
		
		\item $\MACVerify(K, m, \tau) \to d$.
			On input a key~$K$, a message~$m$, and a MAC~$\tau$, this deterministic algorithm outputs a decision bit~$d \in \bits$ (where $d = 1$ indicates validity of the MAC).
	\end{itemize}
\end{definition}
\fi


\begin{definition}[MAC $\muEUFCMA$ security]
	\label{def:MAC-muEUFCMA}
	Let $\MACScheme$ be a MAC scheme
	and $\Gm^{\muEUFCMA}_{\MACScheme,\advA}$ be the game for MAC multi-user existential unforgeability under chosen-message attacks with adaptive corruptions \fullelse{defined as in Figure~\ref{fig:muEUFCMA}}{(see the full version~\cite{EPRINT:DavGun20} for the formal definition)}.
	We define
	\shortlongeqn[,]{
		\Adv^{\muEUFCMA}_{\MACScheme}(t\cab \qNew\cab \qTag\cab\qTagU\cab \qVerify\cab \qVerifyU\cab \qCorrupt) := \max_\advA \Pr \left[ \Gm^{\muEUFCMA}_{\MACScheme,\advA} \Rightarrow 1 \right]
	}
	where the maximum is taken over all adversaries, denoted \emph{$(t\cab \qNew\cab \qTag\cab\qTagU\cab \qVerify\cab\qVerifyU\cab \qCorrupt)$-\fullonly{\linebreak}$\muEUFCMA$-adversaries}, running in time at most~$t$ and making at most $\qNew$, $\qTag$, $\qVerify$, resp.\ $\qCorrupt$ queries to their $\New$, $\OSign$, $\OVerify$, resp.\ $\Corrupt$ oracle, and making at most $\qTagU$ queries $\OTag(i, \cdot)$, resp.\ $\qVerifyU$ queries $\OVerify(i, \cdot)$ for any user~$i$.
\end{definition}

As for signature schemes, multi-user EUF-CMA security of MACs reduces to the single-user case ($\qNew = 1$) by a standard hybrid argument:
\iffull
\begin{align*}
	&\Adv^{\muEUFCMA}_{\MACScheme}(t, \qNew, \qTag, \qTagU, \qVerify, \qVerifyU, \qCorrupt) \\
	&\leq \qNew \cdot \Adv^{\muEUFCMA}_{\MACScheme}(t, 1, \qTagU, \qTagU, \qVerifyU, \qVerifyU, 0),
\end{align*}
\else
$\Adv^{\muEUFCMA}_{\MACScheme}(t\cab \qNew\cab \qTag\cab \qTagU\cab \qVerify\cab \qVerifyU\cab \qCorrupt) \leq \qNew \cdot \Adv^{\muEUFCMA}_{\MACScheme}(t\cab 1\cab \qTagU\cab \qTagU\cab \qVerifyU\cab \qVerifyU\cab 0)$,
\fi
where $t \approx t'$.
(Note that the reduction is not affected by the total number of tagging and verification queries~$\qTag$ resp.\ $\qVerify$ across all users.)

Our multi-user definition of MACs provides a verification oracle, which is non-standard (and in general not equivalent to a definition with a single forgery attempts, as Bellare, Goldreich and Mityiagin~\cite{EPRINT:BelGolMit04} showed).
For PRF-based MACs (which in particular includes HMAC used in TLS~1.3), it however is equivalent and the reduction from multi-query to single-query verification is tight~\cite{EPRINT:BelGolMit04}.

In our key exchange reductions, we actually do not need to corrupt MAC keys, i.e., we achieve $\qCorrupt = 0$.
This in particular allows specific constructions like AMAC~\cite{EC:BelBerTes16} achieving tight multi-user security (without corruptions).

If we use a random oracle~$\RO$ as PRF-like MAC with key length~$\kl$ and output length~$\ol$, then
\shortlongeqn[.]{
\Adv^{\muEUFCMA}_{\RO}(t, \qNew, \qTag, \qTagU, \qVerify, \qVerifyU, \qCorrupt,\qRO) \leq \frac{\qVerify}{2^{\ol}}+\frac{(\qNew-\qCorrupt)\cdot \qRO}{2^{\kl}}
}

\iffull 
\subsection{Hash Function Collision Resistance}
\label{sec:components:Hash}

Finally, let us define collision resistance of hash functions.
\else
\label{apx:components:Hash}
\fi


\begin{definition}[Hash function collision resistance]
	\label{def:hash-function}
	Let $\Hash \colon \bits^* \to \bits^{\ol}$ for $\ol \in \NN$ be a function.
	For a given adversary~$\advA$ running in time at most~$t$, we can consider
	\shortlongeqn[.]{
		\Adv^{\COLL}_{\Hash}(t) := \Pr \left[ (m, m') \getsr \advA : m \neq m' \text{ and } \Hash(m) = \Hash(m') \right]
	}
\end{definition}
\noindent
If we use a random oracle~$\RO$ as hash function, then\fullonly{ by the birthday bound}
\shortlongeqn[.]{
\Adv^{\COLL}_{\RO}(t,\qRO) \leq \frac{\qRO^2}{2^{\ol+1}}+\frac{1}{2^{\ol}}
}
