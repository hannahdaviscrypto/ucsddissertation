\section{The SIGMA Protocol}
\label{sec:sigma}

The \SIGMA family of key exchange protocols introduced by Krawczyk~\cite{C:Krawczyk03,SIGMA-fullversion} describes several variants for building authenticated Diffie--Hellman key exchange using the ``SIGn-and-MAc'' approach.
Its design has been adopted in several Internet security protocols, including, e.g., the Internet Key Exchange protocol~\cite{rfc2409,rfc4306} as part of the IPsec Internet security protocol~\cite{rfc2401} and the newest version~1.3 of the Transport Layer Security (TLS) protocol~\cite{rfc8446}.

Beyond the basic \SIGMA design, we are particularly interested in the \SIGMAI variant which forms the basis of the TLS~1.3 key exchange and aims at hiding the protocol participants' identities as additional feature.
We here present an augmented version of the basic \SIGMA/\SIGMAI protocols which includes explicit exchange of session-identifying random numbers (nonces) to be closer to SIGMA(-like) protocols in practice,
somewhat following the ``full-fledged'' \SIGMA variant~\cite[Appendix~B]{SIGMA-fullversion}.
We illustrate these protocol flows in Figure~\ref{fig:sigma-protocol}. %
\iffull
 and Figure~\ref{fig:sigma-formal} formalizes both as key exchange protocols according to the syntax of Section~\ref{sec:ake-syntax}.
\else
% ; we refer to Appendix~\ref{apx:sigma} for their full formalization as key exchange protocols according to Definition~\ref{def:KE-protocol}.
\fi

The \SIGMA and \SIGMAI protocols make use of
a signature scheme~$\SIGScheme = (\SIGKGen\cab \SIGSign\cab \SIGVerify)$,
a MAC scheme~$\MACScheme = (\MACKGen\cab \MACTag\cab \MACVerify)$,
a pseudorandom function~$\PRF$,
and a function~$\RO$ which we model as a random oracle.
The parties' long-term secret keys consist of one signing key, i.e., $\KE.\KEKGen = \SIGScheme.\SIGKGen$.
The protocols consists of three messages exchanged and accordingly two steps performed by both initiator and responder,
which we describe in more detail now.

\begin{figure}[t]
	\centering
	
	% \begin{minipage}[t]{0.9\textwidth}
\iffull\else\resizebox{11cm}{!}{\fi %% resizebox in lncs
\begin{tikzpicture}
	% Set the X coordinates of the client, server, and arrows
	\edef\ClientX{0}
	\edef\ArrowLeft{4.5}
	\edef\ArrowRight{10.5}
	\edef\ServerX{15}
	
	\newcommand{\sigmaY}{\protvarstyle{Y}}
	
	% Set the starting Y coordinate
	\edef\Y{0}

	% Draw header boxes
	\node [rectangle,draw,inner sep=5pt,right] at (\ClientX,\Y) {\textbf{Initiator} $I$};
	\node [rectangle,draw,inner sep=5pt,left] at (\ServerX,\Y) {\textbf{Responder} $R$};

	% shared info
	\node [inner sep=5pt] at ($(\ClientX,\Y) ! 0.5 ! (\ServerX,\Y)$) {cyclic group $\group = \langle g \rangle$ of prime order~$p$};

	\NextLine[2]
	\ClientAction{\underline{$\RunInitI(I, \sk_I, \st)$}}
	\ServerAction{\underline{$\RunRespI(R, \sk_R, \st, \peerpk, m = (\nonce_I, \X))$}}
	\NextLine
	\ClientAction{$\x \getsr \ZZ_p$, $\X \gets g^\x$}
	\ServerAction{$\y \getsr \ZZ_p$, $\sigmaY \gets g^\y$}
	\NextLine
	
	\ClientAction[name=RunInitI-last]{$\nonce_I \getsr \bits^{nl}$}
	\ServerAction{$\nonce_R \getsr \bits^{nl}$}
	
	%%%%%%%%%%%%%%%%%%%%%%%%%%%%%%%%%%%%%%%%%%%%%%%%%%%%%%%%%%%%%%%%%
	\ClientToServer{$\nonce_I, \X$}{}
	\NextLine[0.25]
	%%%%%%%%%%%%%%%%%%%%%%%%%%%%%%%%%%%%%%%%%%%%%%%%%%%%%%%%%%%%%%%%%
	\ServerAction{$\sid \gets (\nonce_I, \nonce_R,\X,\sigmaY)$}
	\NextLine
	\ServerAction{$\mk \gets \RO(\nonce_I,\nonce_R,\X,\sigmaY,\X^\y)$}
	\NextLine
	\ServerAction{$\ks / \kt / \fbox{$\ke$} \gets \PRF(\mk,0 / 1 / 2)$}
	\NextLine
	
	\ServerAction{$\sigma \gets \SIGScheme.\SIGSign(\sk_R, \labelrs\|\nonce_I\|\nonce_R\| \X \| \sigmaY)$}
	\NextLine
	\ServerAction{$\tau \gets \MACScheme.\MACTag(\kt, \labelrm\|\nonce_I\|\nonce_R\|R)$}
	\NextLine
	\ServerAction[name=RunRespI-last]{$c \gets (R, \sigma,\tau)$ \fbox{$c \gets \text{Enc}_{\ke}(R,\sigma,\tau)$}}
	
	\NextLine[-1]
	\ClientAction[name=RunInitII]{\underline{$\RunInit(I, \sk_I, \st, \peerpk, m = (\nonce_R,\sigmaY,c))$}}
	{\NextLine[1]}
	%%%%%%%%%%%%%%%%%%%%%%%%%%%%%%%%%%%%%%%%%%%%%%%%%%%%%%%%%%%%%%%%%
	\ServerToClient{$\nonce_R, \sigmaY, c$}{}
	{\NextLine[-0.5]}
	\ClientAction{$\sid \gets (\nonce_I,\nonce_R,\X,\sigmaY)$}% \fg{Why do we set $\sid$ here already (there's aborts below), not upon acceptance?} \fg{Because we actually need sid be set for explicit auth.}
	\NextLine
	%%%%%%%%%%%%%%%%%%%%%%%%%%%%%%%%%%%%%%%%%%%%%%%%%%%%%%%%%%%%%%%%%

	\ClientAction{$\mk \gets \RO(\nonce_I,\nonce_R,\X,\sigmaY,\sigmaY^{\x})$}
	\NextLine
	\ClientAction{$\ks / \kt / \fbox{$\ke$} \gets \PRF(\mk,0 / 1 / 2)$}
	\NextLine
	\ClientAction{$(R, \sigma,\tau) \gets c$ \fbox{$(R, \sigma, \tau) \gets \text{Dec}_{\ke}(c)$}}
	\NextLine
	
	\ClientAction{\textbf{abort} if $\neg \SIGScheme.\SIGVerify(\peerpk[R], \labelrs\|\nonce_I\|\nonce_R\| \X \| \sigmaY, \sigma)$}
	\NextLine
	\ClientAction{\textbf{abort} if $\neg \MACScheme.\MACVerify(\kt, \labelrm\|\nonce_I\|\nonce_R\|R, \tau)$}
	\NextLine
	
	\ClientAction{$\status \gets \accepted$; $\peerid \gets R$}
	\NextLine
	\ClientAction{$\sigma' \gets \SIGScheme.\SIGSign(\sk_I, \labelis\|\nonce_I\|\nonce_R\|\X\|\sigmaY)$} 
	\NextLine
	\ClientAction{$\tau' \gets \MACScheme.\MACTag(\kt, \labelim\|\nonce_I\|\nonce_R\|I)$}
	\NextLine
	\ClientAction[name=RunInitII-last]{$c' \gets (I,\sigma', \tau')$ \fbox{$c' \gets \text{Enc}_{\ke}(I,\sigma',\tau')$}}
	

	\ServerAction[name=RunRespII]{\underline{$\RunRespII(\id,\sk,\st,\peerpk, m = c')$}}
	{\NextLine[0.25]}
	%%%%%%%%%%%%%%%%%%%%%%%%%%%%%%%%%%%%%%%%%%%%%%%%%%%%%%%%%%%%%%%%%
	{\NextLine[0.5]}
	%%%%%%%%%%%%%%%%%%%%%%%%%%%%%%%%%%%%%%%%%%%%%%%%%%%%%%%%%%%%%%%%%
	\ServerAction{$(I, \sigma',\tau') \gets c'$\fbox{$(I, \sigma', \tau') \gets \text{Dec}_{\ke}(c')$}}
	\NextLine
	\ServerAction{\textbf{abort} if  $\neg \SIGScheme.\SIGVerify(\peerpk[I], \labelis\|\nonce_I\|\nonce_R\|\X\|\sigmaY,\sigma')$}
	\NextLine
	\ServerAction{\textbf{abort} if $\neg \MACScheme.\MACVerify(\kt, \labelim\|\nonce_I\|\nonce_R\|I, \tau')$}
	\NextLine	
	\ServerAction{$\status \gets \accepted$; $\peerid \gets I$}
	
	\NextLine[1.5]
	
	\SharedAction{\textbf{accept} with key~$\skey = \ks$ and session identifier~$\sid = (\nonce_I, \nonce_R, \X, \sigmaY)$}
	
	
	%
	% state passing
	%
	\draw [thick,gray,dashed,-latex] (RunInitI-last.south west) -- (RunInitII.north west) node [midway,right,darkgray] {$\st.\state \gets (\nonce, \X, \x)$};
	\draw [thick,gray,dashed,-latex] (RunRespI-last.south east) -- (RunRespII.north east) node [midway,left,darkgray] {$\st.\state \gets (\nonce, \nonce',\X,\sigmaY,\ks,\kt,\fbox{$\ke$})$};
\end{tikzpicture}
\iffull\else}\fi %% resizebox in lncs
% \end{minipage}


	
	\caption{%
		The \SIGMA/\SIGMAI protocol flow diagram.
		\fbox{Boxed} code is only performed in the \SIGMAI variant.
		Values~$\inputlabel[x]$ indicate label strings (distinct per~$x$).
	}
	\label{fig:sigma-protocol}
\end{figure}
\definecollection{SIGMAformal}
\begin{collect*}{SIGMAformal}{}{}{}{} %%% ===== COLLECT BEGIN =====
\begin{figure}[tp]
  	\begin{minipage}[t]{0.49\textwidth}
		\begin{algorithm}{$\KEActivate(\id, \sk, \peerid,\peerpk,\role)$}
			\item $\st'.\role \gets \role$
			\item $\st'.\status \gets \running$
			\item if $\role = \initiator$ then
			\item \hindent $(\st',m') \gets \RunInitI(\id,\sk,\st')$
			\item else $m' \gets \bot$
			\item return $(\st',m')$
  		\end{algorithm}

  		\ExptSepSpace

		\begin{algorithm}{$\KERun(\id,\sk,\st,\peerpk,m)$}
			\item if $\st.\status \neq \running $ then
			\item return $\bot$
			\item if $\st.\role = \initiator$ then
			\item \hindent $(\st',m') \gets \RunInit(\id,\sk,\st,\peerpk,m)$
			\item else if $\st.\sid = \bot$
			\item \hindent $(\st',m') \gets \RunRespI(\id,\sk,\st,\peerpk,m)$
			\item else
			\item \hindent $(\st',m') \gets \RunRespII(\id,\sk,\st,\peerpk,m)$
			\item return $(\st',m')$
		\end{algorithm}
		
		\ExptSepSpace
		
		\begin{algorithm}{$\RunInitI(\id,\sk,\st)$}
			\item $\nonce_I \sample \bits^{\nl}$
			\item $\x \sample \ZZ_p$
			\item $\X \gets g^{\x}$
			\item $\st'.\state \gets (\nonce_I, \X, \x)$
			\item $m' \gets (\nonce_I, \X)$
			\item return $(\st',m')$
		\end{algorithm}
		
		\ExptSepSpace
		
		\begin{algorithm}{$\RunRespI(\id,\sk,\st,\peerpk,m)$}
			\item $(\nonce_I,\X) \gets m$ 
			\item $\nonce_R \sample \bits^{\nl}$
			\item $\y \sample \ZZ_p$
			\item $\Y \gets g^{\y}$
			\item $\st'.\sid \gets (\nonce_I, \nonce_R,\X,\Y)$
			\item $\sigma \gets \SIGScheme.\SIGSign(\sk,\labelrs\|\nonce_I\|\nonce_R\|\X\|\Y)$
			\item $\mk \gets \RO(\nonce_I\|\nonce_R\|\X\|\Y\|\X^{\y})$
			\item $\ks \gets \PRF(\mk,0)$
			\item $\kt \gets \PRF(\mk,1)$
			\item \frame{$\ke \gets \PRF(\mk,2)$}
			\item $\tau \gets \MACScheme.\MACTag(\kt, \labelrm\|\nonce_I\|\nonce_R\|\id)$
			\item $\st'.\state \gets (\nonce_I,\nonce_R,\X,\Y,\ks,\kt)$ \newline
				\frame{$\st'.\state \gets (\nonce_I, \nonce_R,\X,\Y,\ks,\kt,\ke)$}
			\item $m' \gets (\nonce_R, \Y, \id, \sigma, \tau)$ \newline
				\frame{$m' \gets (\nonce_R, \Y, \ENCEnc(\ke,(\id,\sigma,\tau)))$}
			\item return $(\st', m')$
		\end{algorithm}
	\end{minipage}
	\begin{minipage}[t]{0.49\textwidth}
		\begin{algorithm}{$\RunInit(\id,\sk,\st,\peerpk,m)$}
			\item $(\nonce_R,\Y,\peerid,\sigma, \tau) \gets m$ \newline
				\frame{$(\nonce_R,\Y,\ciph) \gets m$}
			\item $(\nonce_I,\X,\x) \gets \st.\state$
			\item $\st'.\sid \gets (\nonce_I,\nonce_R,\X,\Y)$
			
			\item $\mk \gets \RO(\nonce_I\|\nonce_R\|\X\|\Y\|\Y^{\x})$
			\item $\ks \gets \PRF(\mk,0)$
			\item $\kt \gets \PRF(\mk,1)$
			\item \frame{$\ke \gets \PRF(\mk,2)$}
			
			\item \frame{$(\peerid,\sigma,\tau) \gets \ENCDec(\ke,\ciph)$}
			\item $\st'.\peerid \gets \peerid$
			
			\item if $\SIGScheme.\SIGVerify(\peerpk[\peerid], \labelrs\|\nonce_I\|\nonce_R\|\X\|\Y, \sigma)$\\
				and $\MACScheme.\MACVerify(\kt, \labelrm\|\nonce_I\|\nonce_R\|\peerid, \tau)$ then
			\item \hindent $\st'.\status \gets \accepted$
			\item \hindent $\st'.\skey \gets \ks$
			\item \hindent $\sigma' \gets \SIGScheme.\SIGSign(\sk, \labelis\|\nonce_I\|\nonce_R\|\X\|\Y)$
			\item \hindent $\tau' \gets \MACScheme.\MACTag(\kt, \labelim\|\nonce_I\|\nonce_R\|\id)$
			\item \hindent $m' \gets (\id, \sigma', \tau')$ \newline
				\null\hindent \frame{$m' \gets \ENCEnc(\ke,(\id,\sigma',\tau'))$}
			\item else
			\item \hindent $m' \gets \bot$
			\item \hindent $\st'.\status \gets \rejected$
			\item return $(\st', m')$
		\end{algorithm}

		\ExptSepSpace

		\begin{algorithm}{$\RunRespII(\id,\sk,\st,\peerpk,m)$}
			\item $(\nonce_I,\nonce_R,\X,\Y,\ks,\kt) \gets \st.\state$ \newline
				\frame{$(\nonce_I,\nonce_R,\X,\Y,\ks,\kt,\ke) \gets st.\state$}
			\item $(\peerid,\sigma',\tau')\gets m$ \newline
				\frame{$(\peerid,\sigma',\tau')\gets \ENCDec(\ke,m)$}
			\item $\st'.\peerid\gets \peerid$
			\item if $\SIGScheme.\SIGVerify(\peerpk[\peerid], \labelis\|\nonce_I\|\nonce_R\|\X\|\Y, \sigma')$\\
				and $\MACScheme.\MACVerify(\kt, \labelim\|\nonce_I\|\nonce_R\|\peerid, \tau')$ then
			\item \hindent $\st'.\status \gets \accepted$
			\item \hindent $\st'.\skey \gets \ks$
			\item else $\st'.\status \gets \rejected$
			\item $m' \gets \emptystring$
			\item return $(\st', m')$
		\end{algorithm}
	\end{minipage}
  	\caption{%
		The formalized \SIGMA/\SIGMAI key exchange protocols (cf.\ Section~\ref{sec:ake-syntax}).
		\fbox{Boxed} code is only performed in the \SIGMAI variant.
  	}
  	\label{fig:sigma-formal}
\end{figure}
\end{collect*}

%\SIGMAformal

\begin{description}
	\item[Initiator Step~1.]
	The initiator picks a Diffie--Hellman exponent~$\x \sample \ZZ_p$ and a random nonce~$\nonce_I$ of length~$nl$ and sends $\nonce_I$ and~$g^\x$.
	
	
	\item[Responder Step~1.]
	The responder also picks a random DH exponent~$\y$ and a random nonce~$\nonce_R$.
	It then derives a master key as~$\mk \gets \RO(\nonce_I\cab \nonce_R\cab \X\cab \Y\cab \X^\y)$ from nonces, DH shares, and the joint DH secret~$g^{\x\y} = (g^\x)^\y$.
	From~$\mk$, keys are derived via $\PRF$ with distinct labels:
	the session key~$\ks$,
	the MAC key~$\kt$,
	and (only in \SIGMAI) the encryption key~$\ke$.
	
	The responder computes a signature~$\sigma$ with~$\sk_R$ over nonces and DH shares (and a unique label~$\labelrs$) and a MAC value~$\tau$ under key~$\kt$ over the nonces and its identity~$R$ (and unique label~$\labelrm$).
	It sends $\nonce_I$, $g^\y$, as well as $R$, $\sigma$, and~$\tau$ to the initiator.
	In \SIGMAI the last three elements are encrypted using~$\ke$ to conceal the responder's identity against passive adversaries.
	
	
	\item[Initiator Step~2.]
	The initiator also computes~$\mk$ and keys~$\ks$, $\kt$, and (in \SIGMAI, used to decrypt the second message part) $\ke$.
	It ensures both the received signature~$\sigma$ and MAC~$\tau$ verify, and aborts otherwise.
	
	It computes its own signature~$\sigma'$ under~$\sk_I$ on nonces and DH shares (with a different label~$\labelis$)
	and a MAC~$\tau'$ under~$\kt$ over the nonces and its identity~$I$ (with yet another label~$\labelim$).
	It sends $I$, $\sigma'$, and~$\tau'$ to the responder (in \SIGMAI encrypted under~$\ke$)
	and accepts with session key~$\ks$ using the nonces and DH shares $(\nonce_I, \nonce_R, \X, \Y)$ as session identifier.
	
	\item[Responder Step~2.]
	The responder finally checks the initiator's signature~$\sigma'$ and MAC~$\tau'$ (aborting if either fails)
	and then accepts with session key~$\skey = \ks$ and session identifier~$\sid = (\nonce_I, \nonce_R, \X, \Y)$.
\end{description}
