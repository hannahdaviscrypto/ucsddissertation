\section{Indifferentiability of the shrink-MD class of functors}
\label{sec-chop}
\headingu{Indifferentiability}
We want the tuple of functions returned by a functor $\construct{F}: \FuncSp{SS} \to \FuncSp{ES}$ to be able to ``replace" a tuple drawn directly from $\FuncSp{ES}$. 
Indifferentiability is a way of defining what this means.  
We adapt the original MRH definition of indifferentiability~\cite{TCC:MauRenHol04} to our game-based model in Figure~\ref{fig-gm-indiff}.
In this game, $\simulator$ is a simulator algorithm. 
The advantage of an adversary $\advA$ against the indifferentiability of functor $\construct{F}$ with respect to simulator $\simulator$ is defined to be 
\[\genAdv{\indiff}{\construct{F},\simulator}{\advA} := 2\Pr[\Gindiff_{\construct{F},\simulator}(\advA) \Rightarrow 1] - 1.\]

\begin{figure}
	
	\twoCols{0.45}{0.45}{
		\ExperimentHeader{Game $\Gindiff_{\construct{F}, \simulator}$}
		
		\begin{algorithm}{$\Initialize()$}
			\item $c \getsr \bits$
			\item $\hh \getsr \FuncSp{SS}$
			\item $\HH \getsr \FuncSp{ES}$
		\end{algorithm}
		\ExptSepSpace
			\begin{algorithm}{$\PubO(i,\YEDSA)$}
			\item if $c = 0$ then
			\item \quad return $\simulator[\HH](i, \YEDSA)$
			\item else return $\hh(i,\YEDSA)$
		\end{algorithm}
	
	}
	{	\ExptSepSpace
		\begin{algorithm}{$\PrivO(i,\XEDSA)$}
			\item if $c = 0$ then return $\HH(i, \XEDSA)$
			\item else return $\construct{F}[\hh](i, \XEDSA)$
		\end{algorithm}	
		\ExptSepSpace
		\begin{algorithm}{$\Finalize(c')$}
			\item return $[[c = c']]$
		\end{algorithm}
	}
	\vspace{5pt}
	\caption{The game  $\Gindiff_{\construct{F}, \simulator}$ measuring indifferentiability of a functor $\construct{F}$ with respect to simulator $\simulator$.}
	\label{fig-gm-indiff}
\end{figure}



\headingu{Modifying the Merkle-Damg{\aa}rd Transform}
Coron et al. showed that the Merkle-Damg{\aa}rd transform is not indifferentiable with respect to
any efficient simulator due to its susceptibility to length-extension attacks~\cite{C:CDMP05}.
In the same work, they analysed the indifferentiability of several closely related indifferentiable constructions, including the ``chop-MD'' construction.
Chop-MD is a functor with the same domain as the MD transform; it simply truncates a specified number of bits from the output of MD.
The $\ourF_3$ functor of Figure~\ref{fig-our-functor} operates similarly to the chop-MD functor, except that $\ourF_3$ reduces the output modulo a prime $\Prime$ instead of truncating.
This small change introduces some bias into the resulting construction that affects its indifferentiability due to the fact that the outputs of the MD transform, which are $2k$-bit strings, are not distributed uniformly over $\Z_{\Prime}$.

In this section, we establish indifferentiability for a general class of functors that includes both chop-MD and $\ourF_3$.
We rely on the indifferentiability of $\ourF_3$ in Section~\ref{sec-schemes} as a stepping-stone to the unforgeability of $\EdDSA$;
however, we think our proof for chop-MD is of independent interest and improves upon prior work.

The original analysis of the chop-MD construction~\cite{C:CDMP05} was set in the ideal cipher model and
accounted for some of the structure of the underlying compression function. A later proof by Fischlin and 
Mittelbach~\cite{hfrobook} adapts the proof strategy to the simpler construction we address here and works in the random
oracle model as we do. Both proofs, however, contain a subtle gap in the way they use their simulators.

At a high level, both proofs define stateful simulators $\simulator$ which simulate a random compression function by
sampling uniform answers to some queries and programming others with the help of their random oracles.
These simulators are not perfect, and fail with some probability that the proofs bound.
In the ideal indifferentiability game, the $\PubO$ oracle answers queries using the simulator and the $\PrivO$ oracle
answers queries using a random oracle.
Both proofs at some point replace the random oracle $\HH$ in $\PrivO$ with $\ChopMD[\simulator]$ and claim that
because $\ChopMD[\simulator[\HH]](X)$ will always return $\HH(X)$ if the simulator does not fail, the adversary cannot detect the change.
This argument is not quite true, because the additional queries to $\simulator$ made by the $\PrivO$ oracle can affect its
internal state and prevent the simulator from failing when it would have in the previous game.
In our proof, we avoid this issue with a novel simulator with \textit{two internal states} to enforce separation between
$\PrivO$ and $\PubO$ queries that both run the simulator.

Our result establishes indifferentiability for all members of the $\ShrinkMD$ class of functors, which
includes any functor built by composing of the MD transform with a function $\Out: \bits^{2k} \to \S$ that satisfies three conditions, namely that for some $\gamma, \epsilon \geq 0$,
\begin{enumerate}
	\item For all $y \in \S$, we can efficiently sample from the uniform distribution on the preimage set $\{\Out^{-1}(y)\}$. We permit the sampling algorithm to fail with probability at most $\epsilon$, but require that upon failure the algorithm outputs a (not necessarily random) element of $\{\Out^{-1}(y)\}$.
	\item For all $y \in \S$, it holds that $\gamma \leq |\{\Out^{-1}(y)\}|$.
	\item The statistical distance $\delta(D)$ between the distribution
	\[D:= z \getsr \Out^{-1}(y) \colon y \getsr \S\] 
	and the uniform distribution on $\bits^{2k}$ is bounded above by $\epsilon$.
\end{enumerate}
In principle, we wish $\gamma$ to be large and $\epsilon$ to be small; if this is so,
then the set $\S$ will be substantially smaller than $\bits^{2k}$ and the function $\Out$
``shrinks'' its domain by mapping it onto a smaller set.

Both chop-MD and mod-MD are members of the $\ShrinkMD$ class of functors;
we briefly show the functions that perform bit truncation and modular reduction by a prime satisfy
our three conditions.
Truncation by any number of bits trivially satisfies condition (1) with $\epsilon = 0$.

Reduction modulo $\Prime$ also satisfies condition (1) because the following algorithm samples from the equivalence class of $x$ modulo $\Prime$ with failure probability at most $\frac{\Prime}{2^{2k}}$.
Let $\ell$ be the smallest integer
such that $\ell > \frac{2^{2k}}{\Prime}$. Sample $w \getsr [0 \ldots \ell-1]$ and output $w \cdot \Prime + x$, or $x$ if $w  \cdot \Prime + x> 2^{2k}$.
We say this algorithm ``fails'' in the latter case, which occurs with probability at most $\frac{1}{\ell} < \frac{\Prime}{2^{2k}}$.
In the event the algorithm does not fail, it outputs a uniform element of the equivalence class of $x$.

Bellare~et al. showed that the truncation of $n$ trailing bits satisfies condition (2) for $\gamma = 2^{2k-n}$ and reduction modulo prime $\Prime$ satisfies (2) for 
$\gamma = \lfloor {2^{2k}}/\Prime \rfloor$ .
It is clear that sampling from the preimages of a random $2k-n$-bit string under $n$-bit truncation produces a uniform $2k$-bit string, so truncation satisfies condition (3) with $\epsilon = 0$.
Also from Bellare~et al.~\cite{EC:BelBerTes16}, we have that the statistical distance between a uniform element of $\Z_{\Prime}$ and the modular reduction of a uniform $2k$-bit string is $\epsilon = \frac{\Prime}{2^{2k}}$.
The statistical distance of our distribution $z \getsr \Out^{-1}(Y)$ for uniform $Y$ over $\S$ from the uniform distribution over $\bits^{2k}$ is bounded above by the same $\epsilon$; hence condition (3) holds.

Given a set $\S$ and a function $\Out: \bits^{2k} \to \S$, we define the functor $\construct{F}_{\S, \Out}$ as
the composition of $\Out$ with $\construct{MD}$. In other words, for any $x \in \bits^*$ and
$\hh \in \AllFuncs(\bits^{b+2k}, \bits^{2k})$, let $\construct{F}_{\S, \Out}[\hh](x) := \Out(\construct{MD}[\hh](x))$.

\begin{theorem}
	\label{th-md-indiff}
	Let $k$ be an integer and $\S$ a set of bitstrings. Let $\Out:\bits^{2k} \to S$ be a function satisfying conditions (1), (2), and (3) above with respect to $\gamma, \epsilon > 0$. Let $\construct{MD}$ be the Merkle-Damg{\aa}rd functor(c.f. Section~\ref{sec-prelims}) $\construct{F}_{\S, \Out}:= \Out \circ \construct{MD}$ be the functor described in the prior paragraph.
	Let $\padF$ be the padding function used by $\construct{MD}$, and let $\UnPadF$ be the function that removes padding from its input (i.e., for all $\XEDSA \in \bits^*$, it holds that $\UnPadF(\XEDSA \concat \padF(|\XEDSA|)) = \XEDSA$). Assume that $\UnPadF$ returns $\bot$
	if its input is incorrectly padded and that $\UnPadF$ is injective on its support.
	Then there exists a simulator $\simulator$ such that for any adversary $\advA$ making $\PrivO$ queries of maximum length $b \cdot (\ell-1)$ bits then
	\[\genAdv{\indiff}{\construct{F},\simulator}{\advA} \leq 
	 2(\Queries{\PubO}{\advA}+ \ell \Queries{\PrivO}{\advA}) \epsilon
	 + \frac{(\Queries{\PubO}{\advA} + \ell \Queries{\PrivO}{\advA})^2}{2^{2k}} 
	 + \frac{ \Queries{\PubO}{\advA} \cdot \ell \Queries{\PrivO}{\advA}}{\gamma}.\]
\end{theorem}
\begin{proof}[Theorem~\ref{th-md-indiff}]
We first give a brief overview of our proof strategy
and its differences from previous indifferentiability proofs for the chop-MD construction~\cite{C:CDMP05,hfrobook}.

Our simulator, $\simulator$, is defined in Figure~\ref{fig-chop-sim}. 
It is inspired by, but distinct from, that of Mittelbach and Fischlin's simulator for the chop-MD construction (~\cite{hfrobook} Figure 17.4.),
	which in turn adapts the simulator of Coron et al~\cite{C:CDMP05} from the ideal cipher model to the random oracle model.
	These simulators all present the interface of a random compression function $\hh$ and internally maintain a graph
	in which each edge represents an input-output pair under the simulated compression function.
	The intention is that each path through this graph will represent a possible evaluation of $\construct{F}_{\S, \Out}[\hh]$.
	The fundamental difference between our simulator and previous ones is that we maintain two internal graphs instead of one: one graph for all queries,
	and one graph for public interface queries only.
	This novel method of using two graphs avoids the gap in prior proofs described above by tracking precisely which
	parts of the simulator's state are influenced by private and public interface queries respectively.

% In our proof, we transform the ideal indifferentiability game by evaluating our functor $\construct{F}$ in each
% 	query to the $\PrivO$ oracle.
% 	Initially, we discard the output of this evaluation and use a separate graph in our simulator so that
% 	these additional queries do not influence the $\PubO$ oracle.
% 	In later games, we bound the probability that the private queries influence the public graph in a way
% 	that is detectable by the adversary (such as creating collisions, cycles, or duplicate edges in the public
% 	simulator's graph), and begin using the same graph for both types of query.
% 	We also claim that if the graph is free of collisions, cycles, and duplicate edges, then we
% 	can respond to $\PrivO$ queries with the evaluation of $\construct{F}$ without detection.
% 	We then use the statistical closeness of sampling a random preimage of a random element (property (3) of $\Out$)
% 	to argue that our simulator is honestly behaving as a random oracle except with some small probability.
% 	The resulting game is then equivalent to the real indifferentiability game, and the theorem follows by collecting
% 	the bounded differences between each pair of adjacent games.

\begin{figure}
		
	\twoCols{0.44}{0.44}{
	\ExperimentHeader{Simulator $\simulator[\HH](\YEDSA, \graph)$}
	\begin{algorithm}{}
		\item $(y, m) \gets \YEDSA$
		\item if $\exists z$ such that $(y, z, m) \in \graph.\edges$
		\item \quad return $z$
		\item $\msg \gets \graph.\ppathO(\IV, y)$
		\item if $\msg \neq \bot$ and $\UnPadF(\msg \concat m) \neq \bot$ then 
		\item \quad if $\Thh[\YEDSA, \msg] \neq \bot$ then $z \gets \Thh[\YEDSA, \msg]$
		\item \quad else  $z \getsr  \Out^{-1}(\HH(\UnPadF(\msg \concat m)))$
		\item \qquad $\Thh[\YEDSA, \msg] \gets z$
		\item else if $\Thh[\YEDSA] \neq \bot$ then $z \gets \Thh[\YEDSA]$
		\item else $z \getsr \bits^{2k}$; $\Thh[\YEDSA] \gets z$
		\item add $(y, z, m)$ to $\graph.\edges$
		\item add $(y, z, m)$ to $\graph_{\all}.\edges$
		\item return $z$
	\end{algorithm}
}
{
		\ExperimentHeader{Game $\Gm_0:=\Gindiff_{\construct{F}, \simulator} | b = 0$}

		\begin{algorithm}{$\Initialize()$}
			\item $\HH \getsr \AllFuncs{\bits^*, \S}$
			\item $\graph_{\all}, \graph_{\public} \gets (\IV)$
		\end{algorithm}
		\ExptSepSpace
		\begin{algorithm}{$\PrivO(\XEDSA)$}
			\item return $\HH(\XEDSA)$
		\end{algorithm}	
			\ExptSepSpace
		\begin{algorithm}{$\PubO(\YEDSA)$}
			\item $z \gets \simulator[\HH](\YEDSA, \graph_{\public})$
			\item return $z$
		\end{algorithm}
		\ExptSepSpace
		\begin{algorithm}{$\Finalize(c')$}
			\item return $c'$
		\end{algorithm}
	}
	\vspace{5pt}
	\caption{Left: Indifferentiability simulator for the proof of Theorem~\ref{th-md-indiff}. Right: The ideal game  $\Gindiff_{\construct{F}, \simulator}$ measuring indifferentiability of a functor $\construct{F}$ with respect to simulator $\simulator$}
	\label{fig-chop-gm0}
	\label{fig-chop-sim}
\end{figure}

In the ``ideal'' indifferentiability game, $\PrivO$ queries are answered by random oracle $\HH\getsr \AllFuncs{\bits^*,\S}$.
$\PubO$ queries are answered by the simulator $\simulator$, which maintains the two graphs $\graph_{\public}$ and $\graph_{\all}$.
We present pseudocode for this game ($\Gm_0$) in Figure~\ref{fig-chop-gm0}.
In each graph, the nodes and edges are labeled with $2k$-bit strings.
An edge from node $y$ to node $z$ with label $m$ is denoted $(y, z, m)$, and represents a single value of the simulated compression function; namely, on $6k$-bit input $y \concat m$, the simulated compression function should output $z$.
Queries made in the process of evaluating $\construct{MD}[\S]$ will form a path that begins at the node labeled with the initialization vector $\IV$; the path's edges will be labeled with the $4k$-bit blocks of $\padF(\msg)$.
 
Whenever the simulator receives a fresh query $(y, m)$, it uses a pathfinding algorithm $\ppathO$ to check whether the query extends an existing path from $\IV$ and thus continues an
existing evaluation of the MD transform.
If so, it reads the message from the path's edge labels then appends the new block $m$ to the end.
If the result is a properly padded message, the simulator removes the padding and uses its oracle $\HH$ to compute the output of functor $\construct{F}$ on the original message.
This output $w$ is an element of $\S$, and it should be consistent with $\Out$ when applied to the $2k$-bit simulator output.
The simulator therefore samples its response from the preimages of $w$ under $\Out$.
If any of these steps fail, then the query does not need to be programmed, so the simulator samples a uniformly random response $z$ and updates its graph with the new edge from $y$.
Because we are attempting to simulate a random function, the simulator must cache its responses to maintain consistency between repeated queries.
It does this in two ways: via the graphs and via table $\Thh$. We require two forms of caching because the simulator may use two graphs and thus responses may not be cached consistently
between private and public queries in the graphs alone.

Our $\Gm_0$ differs from this ideal indifferentiability game only in the $\Finalize$ oracle, which returns the adversary's challenge guess $c'$. 
Thus the probability that game $\Gm_0$ returns $1$ exactly equals $1 - \Pr[\Gindiff_{\construct{F}, \simulator}(\advA) | c = 0]$.

We move to $\Gm_1$, where the $\PrivO$ oracle uses $\simulator$ to calculate the output
of functor $\construct{F}$, then discards the result.
We wish for the adversary's view of games $\Gm_0$ and $\Gm_1$ to be identical, so
we must ensure that the additional queries to $\simulator$ do not influence its state
or its responses to $\PubO$ queries.
We therefore call the simulator with different graphs in the two oracles.
It responds to public queries based only on the public graph, and queries made by $\PrivO$ are private and do not update the public graph.
We do use shared table $\Thh$ to cache outputs across all queries; in this sense a 
private query can affect a public query; however, we cache responses separately for each branch of
the simulator, so our caching does not alter the simulator's branching behavior and the distribution of public queries' responses does not change.
The adversary cannot detect at what time a response $z$ is first sampled, so its
view does not change, and
\[\Pr[\Gm_0] = \Pr[\Gm_1].\]
\begin{figure}
	
	\twoCols{0.44}{0.47}{
		\ExperimentHeader{Game $\Gm_1$}
		
		\begin{algorithm}{$\Initialize()$}
			\item $\HH \getsr \AllFuncs{\bits^*, \S}$
			\item $\graph_{\all}, \graph_{\public} \gets (\IV)$
		\end{algorithm}
		\ExptSepSpace
		\begin{algorithm}{$\PubO(\YEDSA)$}
			\item $z \gets \simulator[\HH](\YEDSA, \graph_{\public})$
			\item return $z$
		\end{algorithm}
	}{
		\ExptSepSpace
		\begin{algorithm}{$\PrivO(\XEDSA)$}
			\item\gamechange{$w \gets \construct{F}[\simulator[\HH](\cdot, \graph_{\all})](\XEDSA)$}
			\item return $\HH(\XEDSA)$
		\end{algorithm}	
		\begin{algorithm}{$\Finalize(c')$}
			\item return $c'$
		\end{algorithm}	
	}
	\vspace{5pt}
	\caption{Game $\Gm_1$ in the proof of Theorem~\ref{th-md-indiff}.  Highlighted code is changed from the previous game, and algorithms not shown are unchanged from the previous game.}
	\label{fig-chop-gm1}
\end{figure}
In game $\Gm_2$, we set a $\bad$ flag if the simulator if $\graph_{\all}$ contains
any collisions, cycles, or ``duplicate'' edges: edges with the same starting node and label but different ending nodes.

Collisions and cycles are formed only when a new edge is created whose ending node is already present in the graph;
we set $\bad$ in this case.
The caching in line $2$ prevents duplicate edges except when the $\PrivO$ and $\PubO$ oracles query the simulator on the same input $(y, m)$, in that order.
Even in this case, caching in table $\Thh$ prevents duplicate edges unless one query detects a path that the other did not, or the two queries detect different paths.

If the $\PubO$ query detects a path to node $y$ that did not exist during the previous $\PrivO$ query, or there are two distinct paths to $y$ in $\graph_{\all}$, then $\graph_{\all}$ must contain a collision or a cycle, and the $\bad$ flag will be set when that is detected.
Furthermore, $\graph_{\public}$ is a subgraph of $\graph_{\all}$, so it cannot contain a path to $y$ that $\graph_{\all}$ does not.
To catch the formation of duplicate edges, it is therefore sufficient to set $\bad$ if $\graph_{\all}$ contains a path from $\IV$ to $y$ that is not detected by the subsequent $\PubO$ query.

The $\bad$ flag is internal and does not affect the view of the game, so
\[\Pr[\Gm_2] = \Pr[\Gm_1] \]

In $\Gm_3$, we force the adversary to lose when the $\bad$ flag is set.
This strictly decreases their advantage, so
\[ \Pr[\Gm_3] \leq \Pr[\Gm_2]. \]

\begin{figure}
	\twoCols{0.34}{0.46}{
		\ExperimentHeader{Game $\Gm_2$, \fbox{$\Gm_3$}}

		\ExptSepSpace
		\begin{algorithm}{$\Finalize(c')$}
			\item \gamechange{\fbox{if $\bad$ then return $0$}}
			\item return $c'$
		\end{algorithm}	
		
		\hrulefill

		\ExptSepSpace
			\ExperimentHeader{Game $\Gm_4$}

			\ExptSepSpace
			\begin{algorithm}{$\PrivO(\XEDSA)$}
				\item$w \gets \construct{F}[\simulator[\HH](\cdot, \graph_{\all})](\XEDSA)$
				\item \gamechange{ return $w$}
			\end{algorithm}	
		}{
		\begin{algorithm}{$\simulator[\HH](\YEDSA, \graph)$}
			\item $(y, m) \gets \YEDSA$
			\item if $\exists z$ such that $(y, z, m) \in \graph.\edges$
			\item \quad return $z$
			\item $\msg \gets \graph.\ppathO(\IV, y)$
			\item \gamechange{$\msg_{\all} \gets \graph_{\all}.\ppathO(\IV, y)$}
			\item if $\msg \neq \bot$ and $\UnPadF(\msg \concat m) \neq \bot$ then 
			\item \quad if $\Thh[\YEDSA, \msg] \neq \bot$ then $z \gets \Thh[\YEDSA, \msg]$
			\item \quad else  $z \getsr  \Out^{-1}(\HH(\UnPadF(\msg \concat m)))$
			\item \qquad $\Thh[\YEDSA, \msg] \gets z$
			\item else if $\Thh[\YEDSA] \neq \bot$ then $z \gets \Thh[\YEDSA]$
			\item else $z \getsr \bits^{2k}$; $\Thh[\YEDSA] \gets z$
			\item \gamechange{if ($z \in \graph_{\all}.\nodes$ and $(y, z, m) \not\in \graph_{\all}.\edges$)}
			\item \quad \gamechange{ or $\msg \neq \msg_{\all}$}
			\item \qquad \gamechange{$\bad \gets \true$}
			\item add $(y, z, m)$ to $\graph.\edges$
			\item add $(y, z, m)$ to $\graph_{\all}.\edges$
			\item return $z$
		\end{algorithm}
	}
	\vspace{5pt}
	\caption{Games $\Gm_2$, $\Gm_3$, and $\Gm_4$ in the proof of Theorem~\ref{th-md-indiff}. Highlighted code is changed from the previous game, and boxed code is present only in $\Gm_3$ (and subsequent games). Algorithms not shown are unchanged from the previous game.}
	\label{fig-chop-gm2}
	\label{fig-chop-gm3}
	\label{fig-chop-gm4}
\end{figure}
In our next game, we stop querying $\HH$ directly in the $\PrivO$ oracle and instead return
$w$, the result of our functor on the query.
We claim that in $\Gm_3$, either $w = \HH(\XEDSA)$ or $\bad = \true$; thus if the adversary wins $\Gm_3$, then in all $\PrivO$ queries we have $w = \HH(\XEDSA)$. 
From this claim, we can see that the change does not affect the view of the adversary and 
\[\Pr[\Gm_4] = \Pr[\Gm_3].\]

To prove the claim, consider a query $\PrivO(\XEDSA)$. Let $(\XEDSA_1, \ldots, \XEDSA_n)$ be the $b$-bit blocks of $\XEDSA \concat \padF(|\XEDSA|)$.
By the definition of the MD transform, $\PrivO$ makes $n$ queries to $\simulator$ of the form $(y_i, \XEDSA_i), \graph_{\all}$, where $y_1 = \IV$ and $y_i = \simulator((y_{i-1}, \XEDSA_{i-1}), \graph_{\all})$ for all $i > 1$.
These may not be fresh queries, but they must be made in order or $\bad$ will be set:
if query $\simulator((y_i, \XEDSA_i))$ outputs $y_{i+1}$ and this has already been the input of a prior query, then $y_{i+1}$ is a node in $\graph_{\all}$; a collision has occurred and the query will set $\bad$.
Unless $\bad$ is set, there exists exactly one path in $\graph_{\all}$ from $\IV$ to $y_i$, and the labels on this path are $(\XEDSA_1, \ldots, \XEDSA_{i-1})$.
This is trivially true for $i = 1$; the path is the empty path.
The query $\simulator((y_{i-1}, \XEDSA_{i-1}), \graph_{\all})$ creates the edge $(y_{i-1}, y_i, \XEDSA_{i-1})$ in $\graph_{\all}$.
By induction on $i$, there is always a path from $\IV$ to $y_i$ with labels $(\XEDSA_1, \ldots, \XEDSA_{i-1})$.
If there exists more than one path from $\IV$ to $y_i$, then $\graph_{\all}$ must contain either a cycle or two edges with the same ending node; in either case the $\bad$ flag will be set.

Therefore, when $\PrivO$ first makes the query $\simulator((y_{n-1}, \XEDSA_n), \graph_{all})$, it will detect the path, compute $\UnPadF(\msg \concat \XEDSA_n) = \XEDSA$ and output an element $z \in \Out^{-1}(\HH(\XEDSA))$. By the definition of $\Out^{-1}$, we have $w = \Out(z) = \HH(\XEDSA)$, so the claim holds.

At this point, the adversary can no longer directly query random oracle $\HH$, so
we allow the simulator to lazily sample the function.
Also in this game, the simulator queries $\HH$ on the path from $\IV$ to $y$ in $\graph_{\all}$ for all queries, not just private queries.  
If the path in $\graph$ is different from the path in $\graph_{\public}$, then the $\bad$ flag
will be set and the adversary will lose anyway.
Therefore the view in any winning game is unchanged, and
\[ \Pr[\Gm_{4}] = \Pr[\Gm_5]. \]

\begin{figure}
	
	\twoCols{0.46}{0.45}{
\ExperimentHeader{Game $\Gm_5$}

\ExptSepSpace
	\begin{algorithm}{$\simulator(\YEDSA, \graph)$}
			\item $(y, m) \gets \YEDSA$
			\item if $\exists z$ such that $(y, z, m) \in \graph.\edges$
			\item \quad return $z$
			\item $\msg \gets \graph.\ppathO(\IV, y)$
			\item $\msg_{\all} \gets \graph_{\all}.\ppathO(\IV, y)$
			\item \gamechange{if $\msg_{\all} \neq \bot$}
			\item[] \qquad \gamechange{ and $\UnPadF(\msg_{\all} \concat m) \neq \bot$ then} 
			\item \quad \gamechange{if $\Thh[\YEDSA, \msg_{\all}] \neq \bot$ then}
			\item \qquad \gamechange{ $z \gets \Thh[\YEDSA, \msg_{\all}]$}
			\item \quad else
			\item \qquad \gamechange{if $\THH[\UnPadF(\msg_{\all} \concat m)] \neq \bot$}
			\item \quad \qquad \gamechange{$y \gets \THH[\UnPadF(\msg_{\all} \concat m)]$}
			\item \qquad \gamechange{$\THH[\UnPadF(\msg_{\all} \concat m)] \gets y$}
			\item \qquad \gamechange{$z \getsr  \Out^{-1}(y)$; $\Thh[\YEDSA, \msg_{\all}] \gets z$}
			\item else if $\Thh[\YEDSA] \neq \bot$ then $z \gets \Thh[\YEDSA]$
			\item else $z \getsr \bits^{2k}$; $\Thh[\YEDSA] \gets z$
			\item if ($z \in \graph_{\all}.\nodes$ and $(y, z, m) \not\in \graph_{\all}.\edges$)
			\item \quad or $\msg \neq \msg_{\all}$
			\item \qquad $\bad \gets \true$
			\item add $(y, z, m)$ to $\graph.\edges$
			\item add $(y, z, m)$ to $\graph_{\all}.\edges$
			\item return $z$
		\end{algorithm}
	}{
	\ExperimentHeader{Game $\Gm_6$}
	\ExptSepSpace
	
		\begin{algorithm}{$\simulator(\YEDSA, \graph)$}
			\item $(y, m) \gets \YEDSA$
			\item if $\exists z$ such that $(y, z, m) \in \graph.\edges$
			\item \quad return $z$
			\item $\msg \gets \graph.\ppathO(\IV, y)$
			\item $\msg_{\all} \gets \graph_{\all}.\ppathO(\IV, y)$
			\item if $\msg_{\all} \neq \bot$ and $\UnPadF(\msg_{\all} \concat m) \neq \bot$ then
			\item \quad \gamechange{$z \getsr \bits^{2k}$}
			\item \quad \gamechange{if $\THH[\UnPadF(\msg_{\all} \concat m)] \neq \bot$}
			\item \qquad \gamechange{$z \gets \THH[\UnPadF(\msg_{\all} \concat m)]$}
			\item \quad \gamechange{$\THH[\UnPadF(\msg_{\all} \concat m)] \gets z$}
			\item else if $\Thh[\YEDSA] \neq \bot$ then $z \gets \Thh[\YEDSA]$
			\item else $z \getsr \bits^{2k}$; $\Thh[\YEDSA] \gets z$
			\item if ($z \in \graph_{\all}.\nodes$ and $(y, z, m) \not\in \graph_{\all}.\edges$)
			\item \quad or $\msg \neq \msg_{\all}$
			\item \qquad $\bad \gets \true$
			\item add $(y, z, m)$ to $\graph.\edges$
			\item add $(y, z, m)$ to $\graph_{\all}.\edges$
			\item return $z$
\end{algorithm}}
	\vspace{5pt}
	\caption{Left: Game $\Gm_5$ in the proof of Theorem~\ref{th-md-indiff}. Right:  Game $\Gm_6$ in the proof of Theorem~\ref{th-md-indiff}. Highlighted code is changed from the previous game, and algorithms not shown are unchanged from the previous game.}
	\label{fig-chop-gm5}
	\label{fig-chop-gm6}
\end{figure} 

In our next game $\Gm_6$, we replace the sampling of $z$ from the preimages of a random point $y$ with sampling a uniformly random $2k$-bit string.
The sampling will never fail to be uniform, which means the adversary can distinguish the game if it were to fail in $\Gm_5$; from condition (1) we have that the probability of failure was
at most $\epsilon$ per query.
Otherwise, we have from condition (3) on $\Out$ that the statistical distance of the distribution $(z \getsr \Out^{-1}(y) \Colon: y \getsr \S)$ from the uniform distribution on $\bits^{2k}$ is at most $\epsilon$.
By a hybrid argument over the $\Queries{\PubO}{\advA} + \ell \Queries{\PrivO}{\advA}$ queries to the simulator,
the probability that $\advA$ can distinguish $\Gm_5$ from $\Gm_6$ is bounded above by $2(\Queries{\PubO}{\advA} + \ell \Queries{\PrivO}{\advA}) \epsilon$.

Now that we are caching $z$ in table $\THH$ when the check of line $6$ holdss $\true$,
it has become redundant to cache it in table $\Thh$, so we stop doing this caching.
We must be careful since table $\THH$ is indexed by labels of the form $\UnPadF(\msg_{\all} \concat m)$
where $\Thh$ was indexed by tuples $(\YEDSA, \msg_{\all})$.
Since $\msg_{\all}$ is a path from $\IV$ to $y$ in a
graph with no duplicate edges provided $\bad$ is not set, $\msg_{\all}$ uniquely
determines its ending node $y$ and $\UnPadF(\msg_{\all}\concat m)$ uniquely determines a tuple
$((y, m), \msg_{\all})$ because $\UnPadF$ is injective.
Thus the entries of $\THH$ are in one-to-one correlation with the entries of $\Thh$, and we can safely retain only the former, and 
\[ \Pr[\Gm_6] \leq \Pr[\Gm_5] + 2(\Queries{\PubO}{\advA} + \ell \Queries{\PrivO}{\advA}) \epsilon \]

In $\Gm_7$, all queries are sampled randomly from $\bits^{2k}$ and cached in table $\Thh$ under
the input $\YEDSA$, instead of some being cached under the message $\UnPadF(\msg_{\all} \concat m)$.
We claim that in $\Gm_6$ if a query $\simulator(y, m)$ stores $z$ in $\THH[\XEDSA]$, then a later query $\simulator(y', m')$ will return $z$ if and only if $(y, m) = (y', m')$ or $\bad$ is set.
The forward direction is trivial.
If $\simulator(y', m')$ returns $\THH[\XEDSA]$, then either we have 
\[\XEDSA = \UnPadF(\graph_{\all}.\ppathO(\IV, y') \concat m') = \UnPadF(\graph_{\all}.\ppathO(\IV, y) \concat m),\]
or there was a $\bad$-setting collision between $\THH[\XEDSA]$ and the randomly-sampled response $z$.

In the former case, the function $\UnPadF$ is injective, so we know $m = m'$, and the paths from $\IV$ to $y'$ and $y'$ respectively have the same
sequence of edge labels.
Unless $\bad$ is set, there are no duplicate edges, so a starting node and sequence of edge labels uniquely identify the ending node on the path; consequently $y = y'$ and the claim follows.

Queries in $\Gm_7$ therefore hit a cache indexed by $\YEDSA$ if and only if they would hit a cache indexed by $\XEDSA$ in $\Gm_6$.
We do not need to worry that the new entries in $\Thh$ overlap with those created in line
$11$; if the check in line $6$ holds true during some query, then it cannot have been false in an earlier query with the same $\YEDSA$
unless $\bad$ would be set. Thus no queries are answered from table $\Thh$ in $\Gm_7$ that would not have been
cached in earlier games, and
\[ \Pr[\Gm_7] = \Pr[\Gm_6]. \]

Notice that both branches of the simulator now identically sample $z \getsr \bits^{2k}$ uniformly, subject to 
caching in table $\Thh$ under $\YEDSA$; in the next game we will eliminate the redundant check on $\msg_{all}$ in line $6$.

\begin{figure}
\twoCols{0.44}{0.47}{
	\ExperimentHeader{Game $\Gm_7$}
	\ExptSepSpace
	
	\begin{algorithm}{$\simulator(\YEDSA, \graph)$}
		\item $(y, m) \gets \YEDSA$
		\item if $\exists z$ such that $(y, z, m) \in \graph.\edges$
		\item \quad return $z$
		\item $\msg \gets \graph.\ppathO(\IV, y)$
		\item $\msg_{\all} \gets \graph_{\all}.\ppathO(\IV, y)$
		\item if $\msg_{\all} \neq \bot$ and $\UnPadF(\msg_{\all} \concat m) \neq \bot$ then 
		\item \quad \gamechange{$z \getsr \bits^{2k}$}
		\item \quad \gamechange{if $\Thh[\YEDSA] \neq \bot$ then $z \gets \Thh[\YEDSA]$}
		\item \quad \gamechange{$\Thh[\YEDSA] \gets z$}
		\item else if $\Thh[\YEDSA] \neq \bot$ then $z \gets \Thh[\YEDSA]$
		\item else $z \getsr \bits^{2k}$; $\Thh[\YEDSA] \gets z$
		\item if $z \in \graph_{\all}.\nodes$ or $\msg \neq \msg_{\all}$
		\item \quad $\bad \gets \true$
		\item add $(y, z, m)$ to $\graph.\edges$
		\item add $(y, z, m)$ to $\graph_{\all}.\edges$
		\item return $z$
	\end{algorithm}
}{
	\ExperimentHeader{Game $\Gm_8$}
	\ExptSepSpace
	
\begin{algorithm}{$\simulator(\YEDSA)$}
\item if $\Thh[\YEDSA] \neq \bot$ then $z \gets \Thh[\YEDSA]$
\item else $z \getsr \bits^{2k}$; $\Thh[\YEDSA] \gets z$
\item return $z$
\end{algorithm}

\ExptSepSpace
\begin{algorithm}{$\Finalize(c')$}
\item return $c'$
\end{algorithm}
}
\vspace{5pt}
\caption{Left: Game $\Gm_7$ in the proof of Theorem~\ref{th-md-indiff}. Right: Game $\Gm_8$ in the proof of Theorem~\ref{th-md-indiff}. Highlighted code is changed from the previous game, and algorithms not shown are unchanged from the previous game.}
\label{fig-chop-gm7}
\label{fig-chop-gm8}
\end{figure}

In our final game, $\Gm_8$, we remove the $\bad$ flag and the internal variables used to set it. 
This increases the adversary's advantage, since it can now win even if the game would set $\bad$.
The probability of a collision among the $\Queries{\PubO}{\advA} + \ell \Queries{\PrivO}{\advA}$ randomly sampled nodes of
$\graph_{\all}$ is at most $\frac{(\Queries{\PubO}{\advA} + \ell \Queries{\PrivO}{\advA})^2}{2^{2k}}$ by a birthday bound.
The probability that $\graph_{\all}$ contains a path to $y$ that $\graph_{\public}$ does not is the probability that the adversary $\advA$ queries $\PubO$ on one of the $\ell q_{\PrivO}$ intermediate nodes on a path in $\graph_{\all}$, before it learns the label of that node from $\PubO$.
$\advA$ may use $\PrivO$ to learn the output $y$ of $\Out$ an intermediate node, but it does not learn anything about which of the equally likely preimages of $y$ is the label; from condition (2) we have that there are at least $\gamma$ such preimages to guess from.
Then the probability that $\advA$ sets $\bad$ with a single $\PubO$ query is at most $\frac{\ell \Queries{\PrivO}{\advA}}{\gamma}$; a union bound over all $\PubO$ queries gives
that a path exists in $\graph_{\all}$ but not $\graph_{\public}$ with probability no greater than $\frac{\Queries{\PubO}{\advA} \cdot \ell \Queries{\PrivO}{\advA}}{\gamma}$.

We also stop maintaining the graphs $\graph_{\public}$ and $\graph_{\all}$, which are now only used to cache queries whose responses are already cached in table $\Thh$.
This changes nothing about the view of the adversary, so
\[ \Pr[\Gm_8] \leq \Pr[\Gm_7] + \frac{(\Queries{\PubO}{\advA} + \ell \Queries{\PrivO}{\advA})^2}{2^{2k}} + \frac{\Queries{\PubO}{\advA} \cdot \ell \Queries{\PrivO}{\advA}}{\gamma}. \]
If we look closely at $\Gm_8$, we can see that the ``simulator'' is actually just a lazily-sampled random function with domain $\bits^{6k}$ and codomain $\bits^{2k}$.
In fact, $\Gm_8$ is identical to the ``real'' indifferentiability game for functor $\construct{F}$, save for its choice of challenge bit.
Thus 
\[\Pr[\Gm_8] =  \Pr[\Gindiff_{\construct{F}, \simulator}(\advA) | c = 1]. \]
Collecting bounds across all gamehops gives the theorem.\qed
\end{proof}