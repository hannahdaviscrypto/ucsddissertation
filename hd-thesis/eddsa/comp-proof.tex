\section{Proof of Theorem~\ref{th-use-findiff}}\label{apx:comp}
%\subsection{Proof of Lemma~\ref{th-flt-comp}}
%\begin{proof}
%		The proof of the lemma follows how Merkle-Damg{\aa}rd transform works. Given an adversary $\advA_4$ against the $\mUFCMA$ security of $\DS_3$, we construct adversary $\advA_3$ with same advantage but different number of queries to $\HASH(i, \cdot)$ in Figure~\ref{fig:A4}. 
%		
%		Instead of making queries to $\HASH(i,\cdot)$ for $i$ in $\{1,2\}$, $\advA_3$ unravels MD queries into consecutive queries to compression function $\HASH(3,\cdot)$. It keeps two $\IV$ for $\HASH(1,\cdot)$ and $\HASH(2,\cdot)$, separately, and breaks a padded input query into $2k$-bit inputs to $\HASH(3,\cdot)$. Since $\advA_3$ simulates $\advA_4$'s hash queries in the same way as the underlying Merkle-Damg{\aa}rd transform, two adversaries have the same advantage but different number of queries to $\HASH(i,\cdot)$. That is, given $\advA_4$ making $q_{i}$ queries to $\HASH(i,\cdot)$ for $i$ in $\{1,2,3\}$, $\advA_3$ makes $0$ queries to $\HASH(1, \cdot)$ and $\HASH(2, \cdot)$ and $q_3 + \ell \cdot (q_1 + q_2)$ queries to $\HASH(3, \cdot)$ and 
%		\[\ufAdv{\DS_3}{\advA_4} \leq \ufAdv{\DS_3}{\advA_3}.\]
%		\begin{figure}
%		\oneCol{0.75}
%		{
%			\begin{algorithm-initial}{adversary $\advA_3[ \Filter_{k,\Out}](\vecPk)$}
%				\item $(y_1[0], y_2[0]) \gets \bits^{2k}$
%				\item $(\usr, \msg, \sigma) \gets \advA_4[\HH UFCMASim, \SignO_4 UFCMASim](\vecPk)$
%				\item return $(\usr, \msg, \sigma)$
%			\end{algorithm-initial}  \vspace{2pt}
%			\begin{algorithm-subsequent}{$\HH UFCMASim(i, x)$}
%				\item if $i \in \{1,2\}$ then
%				\item \quad $(x[1], \ldots, x[\ell]) \gets x\|\padF(\ell)$ \comment{$|x_i|=2k$.}
%				\item \quad For $j=1,\ldots,\ell$ do $y_i[j] \gets  \HH(3, x[j]\|y_i[j-1])$ 
%				\item else return $ \HH(3,x)$
%			\end{algorithm-subsequent}  
%			\begin{algorithm-subsequent}{$\SignO_4 UFCMASim(\usr, \msg)$}
%				\item $(\curvepoint{R},\z) \gets \SignO_3(\usr, \msg)$
%				\item Return $(\curvepoint{R}, \z)$
%			\end{algorithm-subsequent} 
%		}
%		\vspace{-5pt}
%		\caption{Adversary $\advA_4$ against the $\mUFCMA$ security of $\DS_3$ which makes $q_{i}$ queries to $\HASH(i,\cdot)$ for $i$ in $\{1,2,3\}$ in the proof of Theorem~\ref{th-comp}.}
%		\label{fig:A4}
%		\hrulefill
%		\vspace{-10pt}
%	\end{figure}
%	
%\end{proof}

\begin{proof}
	To prove this theorem, we must define an $\FUFCMA$ adversary $\advB$ and an $\findiff$ distinguisher $\advD$. 
	Our adversary $\advB$ plays game $\FUFCMA_{\fDS,\roSp_2, \Filter}$ and simulates game $\FUFCMA_{\fDS, \froSp_1,\Filter}$ for $\advA$ by redirecting $\advA$'s queries to $\Initialize$, $\SignO$, $\pHASH$, and $\Finalize$ to its own oracles of the same name. 
	When $\advA$ begins the game by calling $\Initialize$, our adversary will additionally sample the simulator state: $\simstate \getsr \simulator.\Sg$. 
	We do not need to simulate $\iHASH$, because this oracle is not accessible to $\advA$.
	To answer queries of the form $\eHASH(X)$, adversary $\advB$ responds with the value $\simulator[\eHASH](\simstate \concat \pHASH(), X)$, where the oracle $\eHASH$ presented to $\simulator$ is $\advB$'s own $\eHASH$ oracle in the game $\FUFCMA_{\fDS,\froSp_2,\Filter}$).
	Thus $\advB$ makes $\Queries{\advA}{\PubO}$ queries to $\pHASH$. 
	
	If the simulator $\simulator$ makes $\Queries{\simulator}{\HH}$ queries to its oracle in game $\Gfindiff_{\construct{MD},  \simulator, \Filter}$, then adversary $\advB$ likewise makes $\Queries{\simulator}{\HH}$ queries in total to $\PubO$.
	We note that for the simulator of Lemma~\ref{th-md-wp-indiff},$\Queries{\simulator}{\HH} \leq \ell \cdot \Queries{\advD^{\prime}}{\PrivO} + \Queries{\advD^{\prime}}{\PubO}$.  
	
	We present our distinguisher $\advD$ in pseudocode on the right side of Figure~\ref{fig:gm-pf-th-comp-0-1}. 
	On the left side of this figure, we give pseudocode for two games, $\Gm_0$ and $\Gm_1$. 
	The intent is that when the challenge bit $b$ sampled by the game $\Gfindiff_{\construct{F},  \simulator, \Filter}$ equals $0$ then $\advD$ perfectly simulates $\Gm_0$ for $\advA$, and similarly $\advD$ perfectly simulates $\Gm_1(\advA)$ when $b=1$.
	 This should be immediately visible in the side-by-side comparison of Figure~\ref{fig:gm-pf-th-comp-0-1}.
	 It follows that $\findiffAdv{\construct{F},\simulator,\Filter}{\advD} = \Pr[\Gm_1(\advA)] - \Pr[\Gm_0(\advA)]$.
	 
	We make two claims about these games: 
	\begin{enumerate}
		\item[(1)] that $\Pr[\Gm_1(\advA)\Rightarrow 1] = \fufAdv{\fDS,\roSp_1,\Filter}{\advA}$, and 
		\item[(2)] that $\Pr[\Gm_0(\advA)\Rightarrow 1] = \fufAdv{\fDS,\roSp_2,\Filter}{\advB}$.
	\end{enumerate}
	Claim (1) follows from the fact that the pairs of identically named oracles in $\Gm_1$ and the $\FUFCMA_{\fDS, \roSp_1,\Filter}$ game respond identically to all possible queries.
	We can see straightaway that the $\SignO$, $\eHASH$, $\pHASH$, and $\Finalize$ oracles have identical pseudocode. 
	The two	$\Initialize$  oracles differ only because $\Gm_1$ defines $\hh$, $\HH_0$, and $\HH_1$ instead of $\hg$ and $\HG$.
	According to the definition of $\roSp_1$, $\hh = \hg$ and $\HH_1 = \HG = \construct{F}[\hh]$; the function $\HH_0$ is never called in $\Gm_1$. 
	Thus the difference in naming and the existence of $\HH_0$ is undetectable to $\advA$.
	The $\iHASH$ oracle of $\Gm_1$ maintains a list of queries in $\simstate$, but this can only be accessed in line 15, which is unreachable in $\Gm_1$, so this bookkeeping is solely internal.
	The only remaining oracle is $\iHASH$, which $\advA$ cannot access directly.
	The oracles differ only in that $\Gm_1$ queries $\HH_1$ instead of $\HG$; as we have already noted, these are both defined to be $\construct{F}[\hh]$.
	Thus the oracles of both games are identical, and claim (1) must hold.
	
	Claim (2) is slightly more complicated to prove because the adversaries differ in the two games. 
	We argue that when adversary $\advA$ is combined with the simulator run by $\Gm_0$ in the $\eHASH$ oracle and line $12$ of the $\iHASH$ oracle, it can be recast as identical to adversary $\advB$.  
	As before, the two $\Initialize$ oracles behave identically in $\Gm_1$ and the $\FUFCMA$ game, though now it is $\hh$ and $\HH_1$ which are never called and thus undetectable, and $\HH_0 = \hg = \HG$. 
	Again, the $\SignO$, $\pHASH$, and $\Finalize$ oracles are identical in pseudocode in both games.
	The differences remain in the $\iHASH$ and $\eHASH$ oracles.
	In $\Gm_1$, scheme oracles make queries to $\PrivO$, and receives a response $(Y,Q)$. 
	The $\iHASH$ oracle then concatenates $Q$ into the simulator's state $\simstate$.
	In the $\FUFCMA$ game, the $\iHASH$ oracle does not perform this bookkeeping. However, whenever the simulator is called, we still pass in the list $\QL$ of all prior public $\PrivO$ queries as part of $\simstate$.
	So each time it is called in either game, the simulator's state will contain the entire list of prior public queries in both games. 
	The time at which this list is compiled and included is not detectable by the advesary. 
	The remaining change is the $\eHASH$ oracle, which is significantly different. 
	In $\Gm_1$, $\PubO$ runs the simulator with access to $\HH_0$ to respond to queries by $\advA$; in the $\FUFCMA$ game with function space $\roSp_2$, $\PubO$ calls $\hg$ (which is $\HH_0$) directly. 
	However, we note that $\advB$ queries $\PubO$ only when $\advA$ queries its own simulated $\PubO$ oracle, and $\advB$ runs the simulator itself with the same state and oracles.
	Thus the $\PubO$ oracle of $\Gm_1$ behaves identically to the \emph{simulated} $\eHASH$ oracle of $\advB$ in the $\FUFCMA$ game; the underlying adversary $\advA$ in both games receives the same responses and will thus make the same final query to $\Finalize$ and win with the same probability.
	
	Substitution of claims (1) and (2) into the equation 
	\[\findiffAdv{\construct{F},\simulator,\Filter}{\advD} = \Pr[\Gm_1(\advA)] - \Pr[\Gm_0(\advA)]\]
	 gives the theorem statement. \qed
	
	\begin{figure}[t]
		\twoCols{0.48}{0.50}
		{
			\ExperimentHeader{Games $\Gm_b$\comment{$b \in \bits$}}
			
			\ExptSepSpace
				\begin{oracle}{$\Initialize$}
				\item $\h \getsr \FuncSp{SS}$
				\item $\HASH_0 \getsr \FuncSp{ES}$; $\HASH_1 \getsr \construct{F}[\h]$
				\item $\oseed\getsr\Filter.\Sg$; $\simstate \getsr \simulator.\Sg$
				\item $(\pk, \sk) \getsr \fDS.\Kg[\iHASH]$
				\item $\QL \gets \emptyset$
				\item Return $\pk$
			\end{oracle}
			\ExptSepSpace
			
			\begin{oracle}{$\SignOO(\msg)$}
				\item $\sigma \getsr \fDS.\Sign[\iHASH](\sk, \vk, \msg)$
				\item $S \gets S \cup \{\msg\}$
				\item Return $\sigma$
			\end{oracle}
			\ExptSepSpace
			
			\begin{oracle}{$\iHASH(X)$}
				\item $(Y;Q) \gets \Filter.\Eval[\HASH_b](\oseed,X)$
				\item $\QL \gets \QL \concat \{Q\}$
				\item $\simstate\gets\simstate\|Q$  
				\item Return $Y$
			\end{oracle}
			\ExptSepSpace
			
			\begin{oracle}{$\eHASH(X)$}
			\item If $b=1$ then  $Z\gets \hh(\X)$ \comment{$\Gm_1$ only}
			\item Else $(\simstate,Z) \gets \simulator.\Eval[\HH_0, \pHASH] (\simstate,\X)$ \comment{$\Gm_0$ only}
			\item return $Z$
			\end{oracle}
			\ExptSepSpace
			
			\begin{oracle}{$\pHASH$}
				\item Return $\QL$
			\end{oracle}
			\ExptSepSpace
			
			\begin{oracle}{$\Finalize(\chmsg, \chsig)$}
				\item If ($\chmsg\in S$) then return $\false$
				\item Return $\fDS.\Vf[\iHASH](\pk, \chmsg, \chsig)$ \vspace{2pt}
			\end{oracle}
		}{
		\ExperimentHeader{Distinguisher $\advD$}
		\ExptSepSpace
		\begin{oracle}{}
			\item $\advA[\Initialize,\SignO,\eHASH,\pHASH,\Finalize]()$
		\end{oracle}
	\ExptSepSpace
	\begin{oracle}{$\Initialize$}
		\item $\Initialize()$ \comment{Samples $\oseed$, $\h$, $\HASH_0$, $\HASH_1$, and bit $b$}
		\item $(\pk, \sk) \getsr \fDS.\Kg[\iHASH]$
		\item $\QL \gets \emptyset$
		\item Return $\pk$
	\end{oracle}
	\ExptSepSpace
	
	\begin{oracle}{$\SignOO(\msg)$}
		\item $\sigma \getsr \fDS.\Sign[\iHASH](\sk, \vk, \msg)$
		\item $S \gets S \cup \{\msg\}$
		\item Return $\sigma$
	\end{oracle}
	\ExptSepSpace
	
	\begin{oracle}{$\iHASH(X)$}
		\item $(Y, Q) \gets \iHASH(X)$ \comment{$\Filter[\HASH_b](\oseed, X)$}
		\item $\QL \gets \QL \concat \{Q\}$
		\item Return $Y$
	\end{oracle}
	\ExptSepSpace
	
	\begin{oracle}{$\eHASH(X)$}
		\item Return $\eHASH(X)$ \comment{$\h(X)$ when $b=1$; $\simulator[\HASH_0](\simstate, X)$ otherwise}
	\end{oracle}
	\ExptSepSpace
	
	\begin{oracle}{$\pHASH$}
		\item Return $\QL$
	\end{oracle}
	\ExptSepSpace
	
	\begin{oracle}{$\Finalize(\chmsg, \chsig)$}
		\item If ($\chmsg\in S$) then return $\false$
		\item Return $\fDS.\Vf[\iHASH](\pk, \chmsg, \chsig)$ \vspace{2pt}
	\end{oracle}
		}
		\vspace{-5pt}
		\caption{
		Left: Game defining $\Gm_0$ or $\Gm_1$ depending on the random bit $b$ chosen. $\Gm_0$ is equivalent to $\FUFCMA_{\fDS,\froSp_2}(\advB)$ and $\Gm_1$ is equivalent to $\FUFCMA_{\fDS ,\froSp_1}(\advA)$ in the proof of Theorem~\ref{th-use-findiff}. Right: Distinguisher $\advD$ against the $\findiff$ security of functor $\construct{F}$, which simulates either $\Gm_0$ or $\Gm_1$.}
		\label{fig:gm-pf-th-comp-0-1}
		\hrulefill
		\vspace{-10pt}
	\end{figure}
	
		
%	\begin{figure}
%	\oneCol{0.75}
%	{	
%		\begin{algorithm-initial}{adversary $\advA_2[\simulator, \FltEDDSA](\vecPk)$}
%			\item $\simstate \getsr \simulator.\Sg$
%			\item $(\usr, \msg, (\curvepoint{R}, \z)) \gets \advA_3[\HH_3 UFCMASim, \SignO_3 UFCMASim](\vecPk)$
%			\item return $(\usr, \msg, (\curvepoint{R}, \z))$
%		\end{algorithm-initial}  \vspace{2pt}
%		\begin{algorithm-subsequent}{$\HASH_3 UFCMASim(x)$}
%			\item $(\simstate,Z) \gets \simulator.\Eval[\fHASH(2,\cdot)] (x)$
%			\item return $Z$
%		\end{algorithm-subsequent}  
%		\begin{algorithm-subsequent}{$\SignO_3 UFCMASim(\usr, \msg)$}
%			\item $(\curvepoint{R}, \z) \gets \SignO_2(\usr, \msg)$
%			\item Return $(\curvepoint{R}, \z)$
%		\end{algorithm-subsequent} 
%	}
%		\vspace{-5pt}
%		\caption{Adversary $\advA_2$ for $\FUFCMA_{\DS_2,\FltEDDSA}$ given adversary $\advA_3$ for $\FCMA_{\DS_3}$ in the proof of Theorem~\ref{th-use-findiff}.}
%		\label{fig:th-use-findiff-advB}
%		\hrulefill
%		\vspace{-10pt}
%	\end{figure}

%	We construct the distinguisher in the filter indifferentiability game with $\roSp_3$ in the real world and $\roSp_2$ in the ideal world. The oracles of the distinguisher are $(\PrivO, \PubO) = (\Filter_{k,\Out}^{\HH}, \hh)$ for the real world or  $(\PrivO, \PubO) = (\Filter_{k,\Out}^{\HH}, \simulator^{\Filter_{k,\Out}^{\HH}})$ for the ideal world. 
%	
%	If $\advD$ is in the real world, then adversary $\advA_3$ is playing $\Gm_0$. Otherwise, adversary $\advA_3$ is playing $\Gm_1$ in the ideal world. The distinguisher wraps out of the games by generating keys and signatures for $\advA_3$. The hash queries are simulated with $\PubO$ which just returns answers to $\HASH$ oracle. The distinguisher also returns whatever the $\Finalize$ should return. Thus, the distinguisher is set such that
%	\[\Pr[\Gm_0(\advA_3)] - \Pr[\Gm_1(\advA_3)] \leq \findiffAdv{\construct{2MD},\simulator,\Filter_{k,\Out}}{\advD}. \]
%
%	Therefore, we get the bound 
%	\begin{align*}
%		\Pr[\UFCMA_{\DS_3}(\advA_3)] &\leq \Pr[\mFUFCMA_{\DS_2,\Filter_{k,\Out}}(\advA_2)] + (\Pr[\UFCMA_{\DS_3}(\advA_3)] - \Pr[\mFUFCMA_{\DS_2,\Filter_{k,\Out}}(\advA_2)])\\
%		&\leq \Pr[\mFUFCMA_{\DS_2,\Filter_{k,\Out}}(\advA_2)] + \findiffAdv{\construct{2MD},\simulator,\Filter_{k,\Out}}{\advD}
%	\end{align*}
%	in the theorem as expected.

%	\begin{figure}
%		\oneCol{0.75}
%		{
%			\begin{algorithm-initial}{adversary $\advD[\simulator, \FltEDDSA](1^\lambda)$}
%				\item $S \gets \emptyset$
%				\item For $\usr=1$ to $\numUsers$ do:
%				\item \quad $\e_{\usr} \gets \PrivO(1,\usr)$ 
				%\comment{In $\Gm_0$ only}
%				\item \quad $\e_{\usr, 1} \gets \e_{\usr}[0..k-1]$ 
%				\item \quad $\s \gets \CF(\e_1)$ ; $\vecPk[\usr] \gets \s\cdot \generator$
%				\item $(\usr, \msg, \sigma) \gets \advA_3[\HASH UFCMASim, \SignO_3 UFCMASim](\vecPk)$
%				\item return ($(\usr,\msg) \not\in S$) and $\DS_3.\Vf[\HASH](\vecPk[\usr], \msg, \sigma)$
%			\end{algorithm-initial}  \vspace{2pt}
%			\begin{algorithm-subsequent}{$\HASH UFCMASim(x)$}
%				\item return $ \PubO(x)$
%			\end{algorithm-subsequent}  
%			\begin{algorithm-subsequent}{$\SignO_3 UFCMASim(\usr, \msg)$}
%				\item $\e_{\usr, 1} \gets  \e_{\usr}[0..k-1]$ ; $\e_{\usr, 2} \gets  \e_{\usr}[k..2k-1]$ 
%				\item $\s \gets \CF(\e_{\usr, 1})$ ; $\curvepoint{A} \gets \s\cdot \generator$
%				\item $\littler \gets
%				\PrivO(2,\e_{\usr, 2}\|\msg)$; $\curvepoint{R} \gets \littler\cdot \generator$
%				\item $c \gets 
%				\PrivO(2,\curvepoint{R}\|\curvepoint{A}\|\msg)$
%				\item $\z \gets (\s c + \littler) \mod \Prime$
%				\item $\sigma \gets (\curvepoint{R},\z)$
%				\item $S \gets \{(\usr, \msg)\}$
%				\item Return $(\curvepoint{R}, \z)$
%			\end{algorithm-subsequent} 
%		}
%		\vspace{-5pt}
%		\caption{Adversary $\advD$ for $\Gindiff_{\construct{2MD}, \simulator, \FltEDDSA}$ in the proof of Theorem~\ref{th-comp}.}
%		\label{fig:D}
%		\hrulefill
%		\vspace{-10pt}
%	\end{figure}

\end{proof}