\section{Proof of Theorem~\ref{th-pvf-svf}}\label{sec-pvf-svf}

	\begin{proof}
		
		We prove Theorem~\ref{th-pvf-svf} by constructing adversary $\advA_1$ given $\advA_2$.  $\advA_1$ is an adversary attacking the uf-cma security of $\DS_1= \SchSig[\GDesc,\SKS,\sVF]$, while $\advA_2$ is an adversary attacking the uf-cma security of $\DS_2= \SchSig[\GDesc,\SKS,\pVF]$. Both of which are under the multi-user setting. The strict verification algorithm $\sVF$ for $\DS_1$ and the permissive one $\pVF$ for  $\DS_2$ are shown in Figure~\ref{fig:sVF and pVF}. The other algorithms defined for $\DS_1$ and $\DS_2$ are specified in Section~\ref{sec-schemes} by the $\SchSig$ family.
		
		We give pseudocode for our adversary $\advA_1$ in Figure~\ref{fig:A1}. $\advA_1$ is given a vector of public keys $\vecPk$ and access to random oracle $\HASH$ and signing oracle $\SignOO$. Recall that $\HASH$ is drawn from the function space $\AllFuncs(\bits^*, \Z_{\Prime})$. $\advA_1$ simulates the UF-CMA game for $\advA_2$. We refer to the oracles in the simulated game as $\HASH_2$ and $\SignO_2$ in order to distinguish them from $\advA_1$'s own UF-CMA oracles. Our adversary maintains two look up tables: $HT$ to cache random oracle queries and responses, and $T$ to store query-specific internal state.
		
		$\advA_1$'s high-level strategy is to forward all queries to $\SignO_2$ to its own $\SignO$ oracles. It also forwards $\HASH_2$ queries $(\curvepoint{R}, \vk, \msg)$ for which $\curvepoint{R}$ is in the prime-order subgroup $\group_{\Prime}$. When $\curvepoint{R}$ lies outside this subgroup, $\advA_1$ replaces it with a random element $2^c\curvepoint{R}+\curvepoint{D} \in \group_{\Prime}$. If $\advA_2$ generates a valid forgery in which $\curvepoint{R}$ is not in $\group_{\Prime}$, our adversary uses the discrete logarithm of $\curvepoint{D}$ to produce a forgery containing $2^c\curvepoint{R}+\curvepoint{D}$, which passes strict verification. 
		
		We handle random oracle queries of the form $(\pk, \curvepoint{R}, \msg)$ differently depending on whether $\curvepoint{R}$ is an element of the subgroup $\G_\Prime$. If $\curvepoint{R}$ is in $\G_\Prime$, we forward the query to our random oracle $\HASH$. If $\curvepoint{R}$ is not in $\G_\Prime$, we randomize $\curvepoint{R}$ and project it onto $\G_\Prime$ before forwarding the query. We store the randomizer $D$ and its discrete log $x$ in a table $T$ for later use. When $\advA_2$ submits a message $\msg$ for user $\usr$ and a forgery $(\curvepoint{R}, z)$, we check again whether $\curvepoint{R}$ is in $\G_\Prime$. If it is, then $(\curvepoint{R}, z)$ is a forgery on $\msg$ for user $\usr$ under $\DS_1$. Otherwise, we can compute a forgery on $\msg$ for the same user under $\DS_1$ using the stored values $D$ and $x$.
		
		\hannahnote{unify the above two paragraphs}
		
		We will show that except with some small probability, our adversary $\advA_1$ simulates the uf-cma game of $\DS_2$ perfectly. We will then show that if $\advA_2$ wins the simulated game, then $\advA_1$ will win its own uf-cma game. We first show this for a simplified version of $\advA_2$ that makes no $\SignO_2$ queries, then generalize to arbitrary $\advA_2$. 
		
		When $\advA_2$ makes no queries to $\SignO_2$, how well does $\advA_1$ simulate the UF-CMA game for $\DS_2$? The public key vector $\vecPk$ is generated honestly by the UF-CMA game for $\DS_1$, so the simulation can only fail when outputs from $\HASH_2UFCMASim$ are distributed differently than those of a real random oracle. Consider two distinct queries to $\HASH_2UFCMASim$: $(\pk_1, \curvepoint{R}_1, \msg_1)$ and $(\pk_2, \curvepoint{R}_2, \msg_2)$. A random oracle would generate uniformly random independent response to each of these two queries. $\HASH_2UFCMASim$ also selects independently random responses, except when the two distinct queries cause $\HASH$ to be queried on identical inputs. We will call this event $F$. 
		
		The query to $\HASH$ is either $\curvepoint{R}\|\pk\|\msg$ or $(2^\cofactor\curvepoint{R} + \curvepoint{D})\|\pk\|\msg$. When event $F$ occurs, two distinct queries must match on the public keys and messages, so we let $\pk_1 = \pk_2$ and $\msg_1 = \msg_2$. Let the common values be $\msg$ and $\pk$. The possible values of $\curvepoint{R}_1$ and $\curvepoint{R}_2$ fall into three cases. The first case is that $\curvepoint{R}_1$ and $\curvepoint{R}_2$ are both in $\G_\Prime$. In this case, $\HASH$ is queried on $\curvepoint{R}_1\|\pk\|\msg$ and $\curvepoint{R}_2\|\pk\|\msg$ which are different since $\curvepoint{R}_1 \neq \curvepoint{R}_2$. Therefore, event $F$ cannot happen in case 1. 
		
		Case 2 is when neither $\curvepoint{R_1}$ nor $\curvepoint{R_2}$ are  in $\group_{\Prime}$. The inputs to $\HASH$ are $(2^\cofactor\curvepoint{R_1} + \curvepoint{D_1})\|\pk\|\msg$, 
		$(2^\cofactor\curvepoint{R_2} + \curvepoint{D_2})\|\pk\|\msg$, respectively, and the randomizers $\curvepoint{D1}$ and $\curvepoint{D2}$ are computed for each of them. In this situation, the inputs to $\HASH$ collide when $(2^\cofactor\curvepoint{R_1} + \curvepoint{D_1}) = (2^\cofactor\curvepoint{R_2} + \curvepoint{D_2})$. $(2^\cofactor\curvepoint{R_1} + \curvepoint{D_1})$ and $(2^\cofactor\curvepoint{R_2} + \curvepoint{D_2})$ are randomly distributed points in $\group_{\Prime}$ which has size $\GDesc.\Prime$. During the $(q_s + q)$ $\HASH$ queries, the possibility that two of these elements in $\group_{\Prime}$ collide is exactly $\Pr[F | \curvepoint{R_1}, \curvepoint{R_2} \notin \group_{\Prime}] \leq \frac{(q_s + q)^2}{\GDesc.\Prime}$ based on the birthday bound.
		
		Case 3 is when one point is in $\group_{\Prime}$ and the other is not. Without loss of generality, we assume that $(\curvepoint{R_1} \in \group_{\Prime})$ and $(\curvepoint{R_2} \in \G \setminus \group_{\Prime})$. Then the inputs to $\HASH$ are $\curvepoint{R_1}\|\pk\|\msg$, 
		$(2^\cofactor\curvepoint{R_2} + \curvepoint{D})\|\pk\|\msg$, respectively.
		There is only a single randomizer $\curvepoint{D}$ computed for $\curvepoint{R_2}$. In this case, $F$ happens when $\curvepoint{R_1} = 2^\cofactor \curvepoint{R_2} + \curvepoint{D}$ and so $\HASH(\curvepoint{R_1}\|\pk\| \msg) =  \HASH((2^\cofactor\curvepoint{R_2} + \curvepoint{D}) \|\pk\| \msg)$. Since $\curvepoint{R_1}$ and $(2^\cofactor\curvepoint{R_2} + \curvepoint{D})$ are also randomly distributed points in $\group_{\Prime}$, $\Pr[F | \curvepoint{R_1} \in \group_{\Prime} \wedge \curvepoint{R_2} \notin \group_{\Prime}] \leq \frac{(q_s + q)^2}{\GDesc.\Prime}$ is the probability that two out of $(q_s + q)$ elements in $\group_{\Prime}$ collide based on the birthday bound. Combining case 2 and 3, 
		\[\Pr[F] = \Pr[F | \curvepoint{R_1}, \curvepoint{R_2} \notin \group_{\Prime}] + \Pr[F | \curvepoint{R_1} \in \group_{\Prime} \wedge \curvepoint{R_2} \notin \group_{\Prime}] \leq \frac{2(q_s + q)^2}{\GDesc.\Prime}\].
		
		We have shown that $\advA_1$ perfectly simulates the UF-CMA game except when event $F$ occurs. We next demonstrate that when $\advA_2$ wins the simulated game, $\advA_1$ wins its own game, again except when event $F$ occurs. 
		
		Assume that $\advA_2$ wins the simulated game and event $F$ doesn't happen. To win the UF-CMA game, $\advA_2$ submits $(\usr, \msg, (\curvepoint{R}, z))$ to $\Finalize$ such that $\msg$ is never queried for $\SignO_2$ over user $\usr$ before and $\DS.\pVF^{\HASH}(\vecPk[\usr], \msg, (\curvepoint{R}, z)) = 1$. For the latter to be true and based on the permissive verification algorithm defined in Figure~\ref{fig:sVF and pVF}, the forgery $(\usr, \msg, (\curvepoint{R}, z))$ must satisfy 
		\begin{equation}
			2^\cofactor(\z \cdot  \generator) = 2^\cofactor (\h_2 \cdot \vecPk[\usr] + \curvepoint{R}) \label{eq:5}
		\end{equation}
		Then $\advA_1$ falls into two cases:
		
		Case 1: We have $\curvepoint{R} \in \group_{\Prime}$ in this case, and $\advA_1$ passes over the output of $\advA_2$, $(\usr, \msg, (\curvepoint{R}, z))$.
		If we multiply both sides of Equation\eqref{eq:5} by $I$, we will still be in the same group and obtain
		\[\z \cdot  \generator = \h_2 \cdot \vecPk[\usr] + \curvepoint{R}\] which would pass the strict verification check. We are left to prove that multiplying by $I$ is a valid inverse operation. From algebra, we know every point in the group $\G$ can be uniquely written as the sum of a point in the unique prime-order subgroup $\group_{\Prime}$ and a torsion component. Since the torsion factor $2^\cofactor$ is relatively prime to the order $\Prime$ of the subgroup $\group_{\Prime}$, multiplication of group elements by $2^\cofactor$ is an invertible operation and the inverse operation is multiplication by $I = 2^{-\cofactor} \mod \Prime$. 
		Therefore, in case 1, if $F$ doesn't happen, $\advA_1$ wins Game $\mUFCMA_{\DS1}$ exactly when $\advA_2$ wins Game $\mUFCMA_{\DS2}$. We then obtain
		\[\ufAdv{\DS_2}{\advA_2} = \Pr[\mUFCMA_{\DS2}(\advA_2) \cap \neg F] +  \Pr[\mUFCMA_{\DS2}(\advA_2) \cap F] \leq \Pr[\mUFCMA_{\DS1}(\advA_1) \cap \neg F] + \Pr[F] .\] 
		Since 
		\[ \Pr[\mUFCMA_{\DS1}(\advA_1) \cap \neg F] \leq \ufAdv{\DS_1}{\advA_1} ,\]
		\[\ufAdv{\DS_2}{\advA_2} \leq \ufAdv{\DS_1}{\advA_1} + \Pr[F]\]
		\[\ufAdv{\DS_2}{\advA_2} \leq \ufAdv{\DS_1}{\advA_1} +  \frac{2(q_s + q)^2}{\GDesc.\Prime}\]
		
		Case 2: We have $\curvepoint{R} \in \G \setminus \group_{\Prime}$ in this case, and $\advA_1$ passes over $\advA_2$'s output $(\usr, \msg, (2^\cofactor\curvepoint{R} + \curvepoint{D}, 2^\cofactor\z + x))$. Let the signature $(2^\cofactor\curvepoint{R} + \curvepoint{D}, 2^\cofactor\z + x)$ returned by $\advA_1$ be $(\curvepoint{R}', \z')$. $\advA_1$ retrives the randomizer $\curvepoint{D}$ and its discrete log $x$ from $T[\usr, \curvepoint{R},  \msg]$ in line 6. Since $\curvepoint{R} \in \G \setminus \group_{\Prime}$, $\HASH_2UFCMASim$ makes 
		\[\h_2 =  I \cdot h_1 \mod \Prime\]
		where 
		\[I = 2^{-\cofactor} \mod \Prime.\]
		Then we can rewrite Equation\eqref{eq:5}  as follows,
		\begin{align*} 
			2^\cofactor(\z \cdot  \generator) &= 2^\cofactor (\h_2 \cdot \vecPk[\usr] + \curvepoint{R}) \\
			&=  2^\cofactor (I \cdot \h_1 \cdot \vecPk[\usr] + \curvepoint{R})\\
			&= 2^\cofactor (2^{-\cofactor} \cdot \h_1 \cdot \vecPk[\usr] + \curvepoint{R})\\
			&=  \h_1 \cdot \vecPk[\usr] + 2^\cofactor \curvepoint{R}.
		\end{align*}
		We then add $\curvepoint{D} = x \cdot \generator$ on both sides of the equation $2^\cofactor(\z \cdot  \generator) =  \h_1 \cdot \vecPk[\usr] + 2^\cofactor \curvepoint{R}$:
		\begin{align*} 
			2^c(\z \cdot  \generator) +  x \cdot \generator &=  \h_1 \cdot \vecPk[\usr] + 2^c \curvepoint{R} + \curvepoint{D}\\
			(2^c \z + x)\generator &=  \h_1 \cdot \vecPk[\usr] + (2^c \curvepoint{R} + \curvepoint{D})\\
			\z' \cdot  \generator &= \h_1 \cdot \vecPk[\usr] + \curvepoint{R}'.
		\end{align*}
		From Line 12 of $\HASH_2UFCMASim$, we also have
		\[\h_1 = \HASH((2^\cofactor\curvepoint{R} + \curvepoint{D}) \|\vecPk[\usr]\| \msg) = \HASH(\curvepoint{R}'\|\vecPk[\usr]\| \msg).\]
		This implies that if event $F$ doesn't happen and $\advA_2$ wins Game $\mUFCMA_{\DS2}$, then $\advA_1$ wins Game $\mUFCMA_{\DS1}$. Similar to case 1:
		\[\ufAdv{\DS_2}{\advA_2} = \Pr[\mUFCMA_{\DS2}(\advA_2) \cap \neg F] +  \Pr[\mUFCMA_{\DS2}(\advA_2) \cap F] \leq \Pr[\mUFCMA_{\DS1}(\advA_1) \cap \neg F] + \Pr[F]\] 
		and so
		\[\ufAdv{\DS_2}{\advA_2} \leq \ufAdv{\DS_1}{\advA_1} +  \frac{2(q_s + q)^2}{\GDesc.\Prime}.\]
		
		We show that in both cases, if simulators simulate oracles perfectly, we have
		\[\ufAdv{\DS_2}{\advA_2} \leq \ufAdv{\DS_1}{\advA_1} +  \frac{2(q_s + q)^2}{\GDesc.\Prime}.\]
		
		Now we discard the assumption that $\advA_2$ makes no $\SignO_2$ queries and add a simulated $\SignO$ oracle to $\advA_1$. We claim that allowing $\advA_2$ to make $\SignO_2$ queries makes no difference on this inequality since each $\SignO_2$ query will always result in a perfect simulation. The simulator $\SignO_2 UFCMASim(\usr, \msg)$ begins by making $\SignO$ queries and obtains $(\curvepoint{R}, \z)$ pairs. First notice that since $(\curvepoint{R}, \z)$ is returned by $\SignO$ oracle, it has to be valid under $\SignO_2$ oracle, using the same argument as Case 1. We are left to show programming $\HASH_2(\curvepoint{R}\|\vecPk[\usr]\| \msg)$ in line 19-20 will not cause inconsistency for $\HASH_2$ queries. That is, we want to make sure that $\HASH_2UFCMASim(\vecPk[\usr], \curvepoint{R},  \msg) = \h = \HASH(\curvepoint{R}\|\vecPk[\usr]\| \msg)$. Since $(\curvepoint{R}, \z) = \SignO(\usr, \msg)$, by the definition of $\SignO$ in $\DS_1$ we have $\curvepoint{R} \in \group_{\Prime}$. Then whenever $\advA_2$ makes $\HASH_2$ queries for this specific $(\vecPk[\usr], \curvepoint{R}, \msg)$ pair, $\HASH_2UFCMASim$ falls under Case 1 and returns $\HASH(\curvepoint{R}\|\vecPk[\usr]\| \msg)$, i.e. $\HASH_2UFCMASim(\vecPk[\usr], \curvepoint{R},  \msg) = \HASH(\curvepoint{R}\|\vecPk[\usr]\| \msg)$. Hence, as all the $\curvepoint{R}$ generated by $\SignO_2 UFCMASim(\usr, \msg)$ are in $\G_\Prime$ and event $F$ never happens, programming will not cause inconsistency and we still have the same inequality as the theorem claims.
		
		The running time of $\advA_1$ mainly comes from the running time of simulating $\advA_2$ and the running time of $\HASH_2UFCMASim$. All the other codes are linear time assignments or calls to oracles. In $\HASH_2UFCMASim$, it takes $\mathcal{O}(|\GDesc.\Gorder|^3)$ for each elliptic curve point multiplication. Hence, the overall running time of $\advA_1$ is that of $\advA_2$ plus $\mathcal{O}((q_s + q)|\GDesc.\Gorder|^3)$.
	\end{proof}
	
	\begin{figure}
		\twoCols{0.49}{0.49}
		{
			\begin{algorithm-initial}{$\DS_1.\pVF^{\HASH_1}(\pk, \msg, \sigma)$}
				\item $(\curvepoint{R}, \z) \gets \sigma$
				\item if $(\curvepoint{R}, \z) \notin \group_{\Prime} \times \Z_{\Prime}$ then return $\false$
				\item  $\h \gets \HASH_1(\curvepoint{R}\|\pk\| \msg)$
				\item Return $\z \cdot \generator = \h \cdot \pk + \curvepoint{R}$
			\end{algorithm-initial}  \vspace{2pt}
		}
		{
			\begin{algorithm-initial}{$\DS_2.\sVF^{\HASH_2}(\pk, \msg, \sigma)$}
				\item $(\curvepoint{R}, \z) \gets \sigma$
				\item if $(\curvepoint{R}, \z) \notin \G \times \Z_{\Prime}$ then return $\false$
				\item $\h \gets \HASH_2(\curvepoint{R}\|\pk\| \msg)$
				\item Return $2^\cofactor(\z \cdot  \generator)= 2^\cofactor(\h \cdot \pk + \curvepoint{R})$
			\end{algorithm-initial} 
		}
		\vspace{-8pt}
		\caption{Let $\GDesc$ be a group descriptor. Left: the verification algorithm of $\DS_1 = \SchSig[\GDesc,\SKS,\pVF]$. Right: the verification algorithm of $\DS_2 = \SchSig[\GDesc,\SKS,\sVF]$}
		\label{fig:sVF and pVF}
		\hrulefill
		\vspace{-10pt}
	\end{figure}
	
	
	\begin{figure}
		\oneCol{0.75}
		{	
			\begin{algorithm-initial}{adversary $\advA_1[\HASH](\vecPk)$}
				\item $S \gets \emptyset$
				\item $(\usr, \msg, (\curvepoint{R}, \z)) \gets \advA_2[\HASH_2UFCMASim, \SignO_2 UFCMASim](\vecPk)$
				\item $\HASH_2UFCMASim(\vecPk[\usr], \curvepoint{R},  \msg)$
				\item if $(\curvepoint{R} \in \group_{\Prime})$ then return $(\usr, \msg, (\curvepoint{R}, \z))$
				\item if $(\curvepoint{R} \in \G \setminus \group_{\Prime})$:
				\item \quad $(x, D) \gets T[\usr, \curvepoint{R}, \msg]$
				\item Return $(\usr, \msg, (2^\cofactor\curvepoint{R} + \curvepoint{D}, 2^\cofactor\z + x))$
			\end{algorithm-initial}  \vspace{2pt}
			\begin{algorithm-subsequent}{$\HASH_2UFCMASim(\pk, \curvepoint{R},  \msg)$}
				\item if $(HT[\curvepoint{R}, \pk, \msg] \neq \bot)$ then return $HT[\curvepoint{R}, \pk, \msg]$
				\item $I \gets 2^{-\cofactor} \mod \Prime$
				\item if $\exists \usr$ s.t $\vk = \vecPk[\usr]$ and $\curvepoint{R} \notin \group_{\Prime}$:
				\item \quad $x \getsr \Z_{\Prime}$; $D \gets x \cdot \generator$; $T[\usr, \curvepoint{R}, \msg] \gets (x, D)$
				\item \quad $h_1 \gets \HASH((2^\cofactor\curvepoint{R} + \curvepoint{D}) \|\pk\| \msg)$
				\item \quad $h_2 \gets I \cdot h_1 \mod \Prime$
				\item else:
				\item \quad $h_2 \gets \HASH(\curvepoint{R}\|\pk\| \msg)$
				\item $HT[\curvepoint{R}, \pk, \msg] \gets h_2$
				\item return $h_2$
			\end{algorithm-subsequent}  
			\begin{algorithm-subsequent}{$\SignO_2 UFCMASim(\usr, \msg)$}
				\item $(\curvepoint{R}, \z) \gets \SignO(\usr, \msg)$
				\item $h \gets \HASH(\curvepoint{R}\|\vecPk[\usr]\| \msg)$
				\item $HT[\curvepoint{R}, \vecPk[\usr], \msg] \gets h$
				\item $S \gets S \cup \{(\usr, \msg)\}$
				\item Return $(\curvepoint{R}, \z)$
			\end{algorithm-subsequent} 
		}
		\vspace{-5pt}
		\caption{Adversary $\advA_1$ for $\mUFCMA_{\DS_1}$ given adversary $\advA_2$ for $\mUFCMA_{\DS_2}$ in the proof of Theorem~\ref{th-pvf-svf}.}
		%\label{fig:schnorr-sig}
		\label{fig:A1}
		\hrulefill
		\vspace{-10pt}
	\end{figure}
	
	
	
	
	
	
	