%\section{Proof of Theorem~\ref{th-ideal-eddsa}} \label{apx-ts-eddsa}	
	We prove this theorem using a sequence of code-based games in the Bellare-Rogaway framework.
	We highlight changes to games in gray, and omit oracles that do not change from the previous game to conserve space.
	
	The first game, $\Gm_0$, is the filtered unforgeability game for $\fDS$, so 
		\[\Pr[\Gm_0(\advB)] = \Pr[\FUFCMA_{\fDS,\roSp_2,\Filter}(\advB)].\]

	In $\Gm_1$, which we define in Figure~\ref{fig:gm-pf-ts-eddsa-0-3} alongside $\Gm_0$, we sample $\e_1$  uniformly from the set $\bits^{k}$ instead of querying $\iHASH((1,\emptystring))$.
	We also change the filter so that it samples $\e_2$ directly from $\bits^{k}$ instead of querying $\HH(\oseed)$. (This is not shown in the Figure.)
	In the previous game, $\iHASH$ would compute $\e_1$ and $\e_2$ by partitioning $\HH(\oseed)$, a uniform $2k$-bit string.
	Then $\e_1$ and $\e_2$ are uniformly random in both games; thus the adversary cannot possibly detect the change without making the query $\eHASH(\oseed)$; as $\oseed$ is a uniformly random $k$-bit string known only to $\Filter$, this is possible with probability no better than $\frac{\Queries{\advB}{\eHASH}}{2^{k}}$ unless $\Filter$ somehow leaks $\oseed$. 
	$\Filter$ only accesses $\oseed$ while answering queries to \emph{private} ports; thus it does not leak $\oseed$ in the query list $Q$. 
	It also returns only values dependent on $\HH(\oseed)$ and not $\oseed$ itself; thus $\advB$ can do no better than guessing, and we have	
	\[\Pr[\Gm_0(\advB)] \leq  \Pr[\Gm_1(\advB)] + \frac{\Queries{\advB}{\eHASH}}{2^{k}}.\]
	
	Notice that as of $\Gm_1$, the scheme no longer queries $\PrivO$ with port number $i=1$; thus all queries to $\PrivO$ will have port numbers $2$ or $3$, and the filter will return a value that has been reduced modulo $\Prime$. 
	The portion of the output which is discarded by the modular reduction has no impact on the behavior of any oracle, and cannot possibly help the adversary.
	
	We make this explicit in $\Gm_2$ (Figure~\ref{fig:gm-pf-ts-eddsa-0-3}, on the right side) by changing $\eHASH$ so that it reduces its outputs modulo $\Prime$. 
	This goes against adversary $\advB$'s expectations of $2k$-bit responses from $\PubO$, so we use a wrapper adversary $\advB_1$ to simulate a $\PubO$ oracle for $\advB$. 
	$\advB_1$ answers converts elements $Y \in \Z_{\Prime}$ to elements of $\bits^{2k}$ by sampling $Z \gets \bits^{2k}$ and returning $Y + Z -(Z \mod \Prime)$. 
	In the unlikely event (probability $< \frac{2}{\Prime}$) that the result falls outside of $\bits^{2k}$, the wrapper will resample a new $Z$ and start over; this resampling will not significantly increase its runtime.
	We therefore estimate that $\Time{\advB_1} \approx \Time{\advB}$ and note that they make the same number of queries to hall oracles.
	With resampling, the outputs of the simulated $\PubO$ are uniformly random over $\bits^{2k}$, as they should be.
	Since the oracles perceived by $\advB$ are identical between $\Gm_1$ and $\Gm_2$, we have that
	\[\Pr[\Gm_1(\advB)] \leq \Pr[\Gm_2(\advB)].\]
	From this point forward, the queries $\iHASH((3,X))$ and $\eHASH(X)$ will return identical responses for any input $X$.
	
%%%% Multi-user only
% 	However, the set of all users' secret keys are distributed differently between the games: two users have the same secret key in $\Gm_0$ if either a collision occurs between two entries in the $\oseed$ vector or in their outputs under $\HH_1$, so with probability at most $\frac{\numUsers^2}{2^k} + \frac{\numUsers^2}{2^{2k}}$. In $\Gm_1$, collisions occur with probability at most $\frac{\numUsers^2}{2^{2k}}$ by a birthday bound.  Consequently,
% 	\[\Pr[\Gm_1(\advB)] \leq \Pr[\Gm_0(\advB)] + \frac{\numUsers^2}{2^k}.\]

	
	\begin{figure}[t]
		\twoCols{0.45}{0.45}
		{
			\ExperimentHeader{Game $\FUFCMA_{\fDS,\roSp_2,\Filter} = \Gm_0$, $\Gm_1$}
			
			\begin{oracle}{$\Initialize$}
				\item 
				$(\HH,\HH)\getsr \roSp_2$ ;
				$\oseed \getsr \Filter.\Sg$
				\item $\e_1 \gets \iHASH((1, \emptystring))$ \comment{In $\Gm_0$ only}
				\item $\e_1 \getsr \bits^k$\comment{In $\Gm_1$ only}
				\item $\sk \gets \Clamp(\e_1)$ ; $\pk \gets \sk\cdot \generator$
				\item $\QL \gets \emptyset$
				\item Return $\pk$
			\end{oracle}
			\ExptSepSpace
			
			\begin{oracle}{$\SignOO(\msg)$}
				\item if ($\msg \in S$) then return $\bot$
				%\item $\pk \gets \sk \cdot \generator$
				\item $\littler \gets
				\iHASH((2, \msg))$ ; $\curvepoint{R} \gets \littler\cdot \generator$
				\item $c \gets 
			\iHASH((3,\curvepoint{R}\|\pk\|\msg))$
				
				\item $\z \gets (\sk \cdot c + \littler) \mod \Prime$
				\item $\sigma \gets (\curvepoint{R},\z)$
				\item $S \gets S \cup \{\msg\}$
				\item Return $\sigma$
			\end{oracle}
			\ExptSepSpace
			\begin{oracle}{$\iHASH(X)$}
				\item $(Y;Q) \gets \Filter.\Eval[\HH](\oseed,X)$
				\item $\QL \gets \QL \concat \{Q\}$
			\item Return $Y$
			\end{oracle}
			\ExptSepSpace
			
			\begin{oracle}{$\eHASH(X)$}
			    \item Return $\HH(X)$
		        \end{oracle}
		    \ExptSepSpace
		    
		    \begin{oracle}{$\pHASH$}
    			\item Return $\QL$
    		\end{oracle}
    		\ExptSepSpace
			
			\begin{oracle}{$\Finalize(\chmsg, \chsig)$}
			    \item If ($\chmsg\in S$) then return $\false$
				\item Return $\fDS.\VF[\iHASH](\pk,\chmsg, \chsig)$ \vspace{2pt}
			\end{oracle}
		}
		{
			\ExperimentHeader{$\Gm_2$}
			
			\begin{oracle}{$\eHASH(X)$}
				\item \gamechange{$Y \gets \HH(X) \mod \Prime$}
				\item \gamechange{Return $Y$}
			\end{oracle}
			\ExptSepSpace
			
			\ExperimentHeader{\fbox{$\Gm_3$}, $\Gm_4$}
			
			\begin{oracle}{$\SignOO(\msg)$}
					\item if ($\msg \in S$) then return $\bot$
				%\item $\pk \gets \sk \cdot \generator$
				\item \gamechange{$\littler \getsr \Z_{\Prime}$}
				\item \gamechange{$\e_2\getsr \bits^{2k}$}
				\item \gamechange{if $L_1[\e_{2}\concat\msg] \neq \bot$ then $\bad \gets \true$}
				\item \quad \gamechange{\fbox{$\littler \getsr L_1[\e_{2}\concat\msg]$}}
				\item \gamechange{$L_2[\e_{2}\concat\msg] \gets \littler$}
				\item $\curvepoint{R} \gets \littler\cdot \generator$
				\item $c \gets 
			\iHASH((3,\curvepoint{R}\|\pk\|\msg))$
				
				\item $\z \gets (\sk \cdot c + \littler) \mod \Prime$
				\item $\sigma \gets (\curvepoint{R},\z)$
				\item $S \gets S \cup \{\msg\}$
				\item Return $\sigma$
			\end{oracle}
			\ExptSepSpace
			
			\begin{oracle}{$\eHASH(X)$}
			    \item $Y \gets \HH(X) \mod \Prime$
			    \item \gamechange{if $L_2[X] \neq \bot$ then $\bad \gets \true$}
				\item \quad \gamechange{\fbox{$Y \gets L_2[X]$}}
				\item \gamechange{$L_1[X] \gets Y$}
			    \item Return $Y$
		        \end{oracle}
		    \ExptSepSpace
	}
	\vspace{-5pt}
	\caption{
		Left: Game $\Gm_0$ defining Filtered UF security of signature scheme $\fDS$ and $\Gm_1$ in the proof of Theorem~\ref{th-ideal-eddsa}. Line 2 is only in $\Gm_0$, and Line 3 is only in $\Gm_1$. Right: $\Gm_2$, $\Gm_3$, and and $\Gm_4$ for the proof of Theorem~\ref{th-ideal-eddsa}, with changes from $\Gm_1$ highlighted. Boxed code is only in $\Gm_3$. Oracles that are identical to prior games have been omitted.}
		%\label{fig:schnorr-sig}
		\label{fig:gm-pf-ts-eddsa-0-3}
		\hrulefill
		\vspace{-10pt}
	\end{figure}
	
	\begin{figure}[t]
		\twoCols{0.45}{0.45}
		{
		\ExperimentHeader{$\Gm_5$}
		
		\begin{oracle}{$\SignOO(\msg)$}
			\item if ($\msg \in S$) then return $\bot$
				
				%\item $\pk \gets \sk \cdot \generator$
				\item $\littler \getsr \Z_{\Prime}$; $\curvepoint{R} \gets \littler\cdot \generator$
				\item $c \gets 
			\iHASH((3,\curvepoint{R}\|\pk\|\msg))$
				
				\item $\z \gets (\sk \cdot c + \littler) \mod \Prime$
				\item $\sigma \gets (\curvepoint{R},\z)$
				\item $S \gets S \cup \{\msg\}$
				\item Return $\sigma$
		\end{oracle}
		\ExptSepSpace
		\begin{oracle}{$\eHASH(X)$}
			\item Return \gamechange{$\HH(X) \mod \Prime$}
		\end{oracle}
    	\ExptSepSpace
		}
		{
			\ExperimentHeader{$\Gm_6$}
			
			\begin{oracle}{$\SignOO(\msg)$}
				\item if ($\msg \in S$) then return $\bot$
				%\item $\pk \gets \sk \cdot \generator$
				\item $\littler \getsr \Z_{\Prime}$; $\curvepoint{R} \gets \littler\cdot \generator$
				\item \gamechange{if $L_3[\curvepoint{R}\|\pk\|\msg] = \bot$ then}
				\item \quad \gamechange{$L_3[\curvepoint{R}\|\pk\|\msg] \getsr \Z_{\Prime}$}
				\item \gamechange{$c \gets L_3[\curvepoint{R}\|\pk\|\msg]$}
				\item $\z \gets (\sk \cdot c + \littler) \mod \Prime$
				\item $\sigma \gets (\curvepoint{R},\z)$
				\item $S \gets S \cup \{\msg\}$
				\item Return $\sigma$
			\end{oracle}
			\ExptSepSpace
			
			\begin{oracle}{$\eHASH(X)$}
				\item \gamechange{$Y \getsr \Z_{\Prime}$}
				\item \gamechange{if $L_3[X] \neq \bot$ then}
				\item \quad \gamechange{\fbox{$Y \gets L_3[X]$}}
				\item \gamechange{$L_3[X] \gets Y$}
				\item Return $Y$
			\end{oracle}
			\ExptSepSpace
		}
		\vspace{-5pt}
		\caption{Games $\Gm_5$ and $\Gm_6$ for the proof of Theorem~\ref{th-ideal-eddsa}, with changes from prior games highlighted.}
%		%\label{fig:schnorr-sig}
		\label{fig:gm-pf-ts-eddsa-56}
		\hrulefill
		\vspace{-10pt}
	\end{figure}
	
	In the next game, $\Gm_3$, we unroll the operation of $\Filter$ on queries with port number $2$, and we alter the $\SignO$ oracle to sample $\littler$ uniformly at random from $\Z_{\Prime}$ instead of calling $\HH(\e_2 \concat \msg)$. 
	This threatens to make the $\SignO$ oracle inconsistent with the $\eHASH$ oracle, if the adversary should make the two queries $\SignO(\msg)$ and $\eHASH(\e_2||\msg)$ for any message $\msg$.
	We thus set a $\bad$ flag if these two queries occur and program both oracles using tables $L_1$ and $L_2$.
	The reprogramming enforces consistency between the oracles $\Gm_2$, but the distribution of $\littler$ changes from the distribution $\distrib_{\Prime, 2k}$ of Section~\ref{sec-prelims} to the uniform distribution $U_{\Z_{\Prime}}$ over $\Z_{\Prime}$. 
	We can bound the difference in advantage across the two resulting games by the statistical distance between these distributions, which is at most $\frac{\Prime}{2^{2k+2}}$ according to Lemma~\ref{th-distribution-bias} . Therefore
	\[\Pr[\Gm_2(\advB_1)] \leq \Pr[\Gm_3(\advB_1)] + \frac{\Prime}{2^{2k+2}}.\]
	In $\Gm_4$, we stop programming $\iHASH((2, \msg))$ after setting the $\bad$  flag. By the Fundamental Lemma of Game-Playing from~\cite{EC:BelRog06} we have 
		\[\Pr[\Gm_3(\advB)] \leq \Pr[\Gm_4(\advB)] + \Pr[\Gm_4(\advB) \text{ sets } \bad].\]
	We know that $\Gm_4(\advB)$ sets the $\bad$ flag only if $\advB$ makes two queries $\SignO(\msg)$ and $\eHASH(\e_2||\msg)$. Since $\e_{2}$ is a uniformly random $k$-bit string and is never revealed by any oracle, the probability is at most $\frac{1}{2^k}$ that any one query to $\eHASH$  begins with $\e_2$ . By the union bound over all queries to $\eHASH$, we have 
	\[ \Pr[\Gm_4(\advB) \text{ sets } \bad] \leq \frac{\Queries{\advB}{\eHASH}}{2^k}. \]
	
	In game $\Gm_5$,  we make certain bookkeeping changes that do not alter the behavior of the oracles. We discard the redundant $\bad$ flag, the check that sets it, and tables $L_1$, $L_2$ which are no longer used. 
	As a result,
	\[\Pr[\Gm_4(\advB)] \leq \Pr[\Gm_5(\advB)].\]
	Finally, for $\Gm_6$ we change the $\eHASH$ oracle to implement a random function in $\AllFuncs(\bits^{*},\Z_{\Prime})$ via lazy sampling in table $L_3$.
	As queries $\iHASH(3, X)$ and $\eHASH(X)$ must return identical responses, we also replace the last remaining query to $\iHASH$ in the $\SignO$ oracle with an update to $L_3$. 
	This changes the distribution of $c$ from $\distrib_{\Prime, 2k}$ to $U_{\Z_{\Prime}}$; we again rely on Lemma~\ref{th-distribution-bias} to produce the bound
	\[\Pr[\Gm_5(\advB)] \leq \Pr[\Gm_6(\advB)] + \frac{\Prime}{2^{2k+2}}.\]
	Combining the bounds from previous gamehops, we see that
	\[\Pr[\FUFCMA_{\fDS,\roSp_2,\Filter}(\advB)] \leq \Pr[\Gm_6(\advB)] + \frac{2\Queries{\advB}{\eHASH}}{2^k} + \frac{\Prime}{2^{2k+2}}\]
	
	We now define in pseudocode (see Figure~\ref{fig:A1}) a new adversary $\advA_1$ which plays the $\UFCMA$ game against $\DS_1$ and simulates $\Gm_6$ for $\advB_1$.
	This game samples $\HH_1$ from function space $\roSp_3$, so $\HH_1\Colon\bits^*\to\Z_{\Prime}$ is a random function with the same domain and range as the lazily-sampled function of $\Gm_6$.
	$\advA_1$ simulates $\Gm_6$ for $\advB$.
	It receives public key $\vk$ from its own $\Initialize$ oracle and forwards this to $\advB$. 
	It answers queries to $\SignOO_2$ and $\eHASH_2$ by forwarding the response of its own $\SignOO$ and $\eHASH$ oracle, respectively.
	It has no $\pHASH$ oracle, but it can simulate this perfectly by maintaining a list $\QL$ with an entry $(\curvepoint{R} \concat \vk \concat \msg), \eHASH(\curvepoint{R} \concat \vk \concat \msg)$ for each query $\SignO(\msg)$ which returns $(\curvepoint{R}, z)$.
	Queries to $\pHASH$ will then simply return the current list $\QL$.
	This simulation is perfect, so 
	\[\Pr[\Gm_6(\advB_1)] = \Pr[\UFCMA_{\DS_1, \roSp_3}(\advA_1)].\]
	Since $\advA$ only forwards queries and maintains a list, its runtime is approximately $\Time{\advB_1} \approx \Time{\advB}$. It makes one query to $\SignO$ and one to $\eHASH$ for each simulated $\SignO$ query, so $\Queries{\advA_1}{\eHASH} = \Queries{\advB}{\eHASH}+ \Queries{\advB}{\SignO}$.
	
	 Then by collecting bounds, we have the theorem statement.
	\begin{figure}[t]
		\oneCol{0.98}
		{	
			\begin{algorithm-initial}{adversary $\advA_1(\vk)$}
				\item $\advB[\eHASH, \pHASH^{\prime}, \SignO^{\prime}, \Finalize](\vk)$
			\end{algorithm-initial}  \vspace{2pt} 
			\begin{algorithm-subsequent}{$\pHASH^{\prime}$}
				\item return $\QL$
			\end{algorithm-subsequent}  
			\begin{algorithm-subsequent}{$\SignO^{\prime}(\msg)$}
				\item $(\curvepoint{R}, \z) \gets \SignO(\msg)$
				\item $Q \gets (\curvepoint{R}\|\pk\|\msg, \eHASH(\curvepoint{R}\|\pk\|\msg))$
				\item $\QL \gets \QL \concat \{Q\}$
				\item Return $(\curvepoint{R}, \z)$
			\end{algorithm-subsequent}
		}
		\vspace{-5pt}
		\caption{Adversary $\advA_1$ for $\UFCMA_{\DS_1,\roSp_1}$ given adversary $\advB_1$ for $\FUFCMA_{\fDS,\roSp_2,\Filter}$ in the proof of Theorem~\ref{th-ideal-eddsa}.}
		%\label{fig:schnorr-sig}
		\label{fig:A1}
		\hrulefill
		\vspace{-10pt}
	\end{figure}
